\chapter{Regresión lineal múltiple}
La regresión múltiple permite a los investigadores examinar el efecto de más de una variable independiente en la respuesta al mismo tiempo. Para algunas preguntas de investigación, la regresión se puede utilizar para examinar en qué medida un conjunto particular de variables independientes puede explicar suficientemente el resultado.

\section{Espacio vectorial y Proyección}

\subsection{Espacio vectorial}
Un espacio vectorial es un conjunto de vectores  que es cerrado bajo  la adición y la multiplicación por un escalar de vectores finitos.\\

Una base hace posible expresar cada vector del espacio como una tupla única de los elementos del campo, aunque se debe tener precaución cuando un espacio vectorial no tiene una base finita. En álgebra lineal, una base es un conjunto de vectores que, en una combinación lineal, pueden representar cada vector en un espacio vectorial dado, y tales que ningún elemento del conjunto pueda representarse como una combinación lineal de los demás. En otras palabras, una base es un conjunto generador linealmente independiente. Es decir, cualquier vector $x'=(x_1,x_2,\ldots,x_n)$ en $\mathbb{R}^n$ puede ser una combinación lineal de $e_1,e_2,\ldots,e_n$. En efecto,

$$x=x_1e_1+x_2e_2+x_3e_3,+\cdots+x_ne_n.$$

Esta representación es única. Que quiere decir que no existe otra representación tal que

$$x=x_1^*e_1+x_2^*e_2+x_3^*e_3,+\cdots+x_n^*e_n.$$

Entonces, $x_i=x_i^*$ para todo $i=1,2,\cdots , n.$\\ 

Dado un espacio vectorial $V$, un subconjunto no vacío $W$ de $V$ que es cerrado bajo la suma y la multiplicación escalar se llama subespacio de $V$. La intersección de todos los subespacios que contengan un conjunto dado de vectores es llamado SPAN o generador. Si no se puede eliminar ningún vector sin cambiar el SPAN (generador), se dice que los vectores en este conjunto son linealmente independientes. Un conjunto linealmente independiente cuyo intervalo es $V$ se llama base de $V$. Un vector generado (SPAN) por dos vectores $v$ y $w$ (ambas no necesariamente independientes) se pueden definir como: $x:x=av+bw,$ para todo $(a,b)\in \mathbb{R}^2$. Si un espacio vectorial $S$ está generado por un conjunto de vectores independientes $v_1,v_2,\ldots , v_p$. Es decir, $S$ es el conjunto de vectores 
$$\left\{x:x=a_1+v_1+a_2v_2+\ldots+a_pv_p,\mbox{ para todo } (a_1,a_2,\ldots,a_p)\in \mathbb{R}^p\right\},$$

Entonces la dimensión de $S$ es $p$. Los vectores $v_1, v_2, \ldots , v_p$ son la base del espacio vectorial $S$. La dimensión de un espacio vectorial $S$ es el mayor número de un conjunto de vectores independientes en $S$. Si la dimensión de un espacio lineal $S$ es $p$, se escribe $Dim(S) = p$.
 

\subsection{Vectores linealmente independientes}

Si existe un número finito de vectores distintos $v_1,v_2,\ldots,v_n$ en el espacio vectorial $V$ y los escalares $a_1,a_2,\ldots,a_n$ no todos cero, tal que
$$a_1v_1+a_2v_2+\ldots,a_nv_n=0,$$
entonces los vectores $v_1,v_2,\ldots,v_n$ se dice que son linealmente dependientes. Si $v_1,v_2,\ldots,v_n$ son dependientes entonces de estos $n$ vectores hay al menos un vector que puede expresarse como una combinación lineal de otros vectores. Tenga en cuenta que el cero a la derecha es el vector cero, no el número cero. Si no existen tales escalares, entonces se dice que los vectores $v_1, v_2, \ldots , v_n$ son linealmente independientes. Esta condición se puede reformular de la siguiente manera: Siempre que $a_1 , a_2 , \ldots , a_n$ sean escalares tales que

$$a_1v_1+a_2v_2+\ldots,a_nv_n=0,$$

tenemos $a_i=0$ para $i=1,2,\ldots,n$. Entonces, $v_1,v_2,\ldots,v_n$ son linealmente independientes.\\

Una base de un espacio vectorial $V$ se define como un subconjunto de vectores en $V$ que son linealmente independientes y estos vectores abarcan el espacio $V$. En consecuencia, si $(v_1 , v_2 , \ldots , v_n)$ es una lista de vectores en $V$, entonces estos vectores forman una base si y solo si todo vector $x \in V$ puede expresarse de forma única mediante una combinación lineal de $v_1 , v_2 , \ldots , v_p$. Es decir, 
$$x = a_1 v_1 + a_2 v_2 + \ldots + a_n v_n,$$ 
para cualquier $x \in V$. El número de vectores base en $V$ se denomina dimensión del espacio lineal $V$. Tenga en cuenta que un espacio vectorial puede tener más de una base, pero el número de vectores que forman la base del espacio vectorial $V$ siempre es fijo. Es decir, la dimensión del espacio vectorial $V$ es fija pero habrá más de una base. De hecho, si la dimensión del espacio vectorial $V$ es $n$, entonces cualquier n vector linealmente independiente en $V$ forma su base.


\section{Producto punto y Proyección}

Si $\textbf{x}=(x_1,x_2,\ldots,x_n)$ e $\textbf{y}=(y_1,y_2,\ldots, y_n)$ dos vectores en un espacio Euclidiano vectorial $\textbf{R}^n$. El producto punto de los vectores $\textbf{x}$ e $\textbf{y}$ de define como

$$\textbf{x}\cdot \textbf{y} = x_1y_1+x_2y_2+\cdots + x_ny_n.$$

Dos vectores se llaman ortogonales si su  producto vectorial es $0$. Si $\theta$ es el ángulo entre dos vectores, el coseno del angulo se define por

\begin{tcolorbox}
    \begin{equation}
	\cos(\theta)=\dfrac{\textbf{x}\cdot \textbf{y}}{|\textbf{x}||\textbf{y}|}=\dfrac{x_1y_1+x_2y_2+\ldots  +x_ny_n}{\sqrt{x_1^2+x_2^2+\ldots + x_n^2}\sqrt{y_1^2+y_2^2+\ldots + y_n^2}}.
    \end{equation}
\end{tcolorbox}

Dos vectores ortogonales forman $90^\circ$ ;Es decir, son perpendiculares. Una aplicación importante del producto punto es la proyección. La proyección de un vector \textbf{y} sobre otro vector \textbf{x} forma un nuevo vector que tiene la misma dirección que el vector \textbf{x} y la longitud $|y|\cos(\theta)$, donde $|\textbf{y}|$ denota la longitud del vector \textbf{y} y $\theta$ es el ángulo entre los vectores. Escribimos la proyección como $P_{\textbf{x}}\textbf{y}$. La proyección vectorial puede ser expresado como

$$
\begin{array}{rcl}
    P_{\textbf{x}}\textbf{y} &=& |\textbf{y}|\cos(\theta)\dfrac{\textbf{x}}{|\textbf{x}|}\\\\
			     &=& |\textbf{y}|\dfrac{\textbf{x}\cdot \textbf{y}}{|\textbf{x}||\textbf{y}|}\dfrac{\textbf{x}}{|\textbf{x}|} \\\\
			     &=& \dfrac{x_1y_1+x_2y_2+\cdots + x_ny_n}{x_1^2+x_2^2+\cdots + x_n^2}\textbf{x}\\\\
			     &=& \lambda \textbf{x}.\\\\
\end{array}
$$

Donde $\lambda$ es un escalar, 

\begin{tcolorbox}
    \begin{equation}
	\lambda=\dfrac{x_1y_1+x_2y_2+\cdots x_ny_n}{x_1^2+x_2^2 + \cdots + x_n^2}=\dfrac{\textbf{x}\cdot \textbf{y}}{\textbf{x}\cdot \textbf{x}}.
    \end{equation}
\end{tcolorbox}

Así, la proyección de \textbf{y} sobre el vector \textbf{x} es un vector \textbf{x} que multiplica un escalar $\lambda$ donde $\lambda$ es el $\cos(\theta)$ y $\theta$ es el ángulo entre los vectores \textbf{x} e \textbf{y}.\\

Si \textbf{x} e \textbf{y} son dos vectores en $\mathbb{R}^n$. Consideremos la diferencia vector entre el vector \textbf{e}, $\textbf{e}=\lambda \textbf{x}-\textbf{y}$, y $\;\lambda = \textbf{x}\cdot\textbf{y}/\textbf{x}\cdot \textbf{x}.$ El vector \textbf{e} es perpendicular para el vector \textbf{x} donde $\lambda=(\textbf{x}\cdot \textbf{y})/(\textbf{x}\cdot \textbf{x})$. Para ver esto, simplemente calculamos el producto escalar de \textbf{e} y \textbf{x}:

$$\textbf{e}\cdot \textbf{x} = (\lambda \textbf{x}-\textbf{y})\cdot\textbf{x}=\lambda \textbf{x}\cdot \textbf{x}-\textbf{x}\cdot \textbf{y}=\left(\dfrac{\textbf{x}\cdot \textbf{y}}{\textbf{x}\cdot \textbf{x}}\right)\textbf{x}\cdot \textbf{x}-\textbf{x}\cdot\textbf{y}=0.$$

Por lo tanto, el ángulo entre \textbf{e} y \textbf{x} es $90^\circ$. Es decir, estos vectores son perpendiculares. Además, dado que \textbf{e} es perpendicular a \textbf{x}, es el vector con la distancia más corta entre todos los vectores que comienzan desde el final de \textbf{y}, y terminan en cualquier punto de \textbf{x}.\\

Si un espacio vectorial tiene una base y la longitud de los vectores base es una unidad, entonces esta base es una base ortonormal. Cualquier base dividida por su longitud forma una base ortonormal. Si $S$ es un subespacio p-dimensional de un espacio vectorial $V$, entonces es posible proyectar vectores en $V$ sobre $S$. Si el subespacio $S$ tiene una base ortonormal $(w_1, w_2, \ldots , w_p)$, para cualquier vector \textbf{y} en $V$, la proyección de \textbf{y} sobre el subespacio $S$ es

\begin{equation}
    P_{S}\textbf{y}=\sum_{i=1}^p (\textbf{y}\cdot w_i)w_i.
\end{equation}

Sean dos subespacios $S$ y $T$ del espacio vectorial $V$ y la unión $S\cup T = V$. Si para cualquier vector $\textbf{x}\in S$ y cualquier $\textbf{y}\in T$, tal que el producto punto $\textbf{x}\cdot \textbf{y}=0$, entonces se dice que los dos espacios vectoriales $S$ y $T$ son ortogonales. O podemos decir que $T$ es el espacio ortogonal de $S$, denotado por 
$$T=S^{\perp}.$$

Así, para un espacio vectorial $V$, si $S$ es un subespacio vectorial en $V$, entonces $V=S\cup S^{\perp}$. Cualquier vector $\textbf{y}\in V$ puede ser escrito de manera única como $\textbf{y}_{S}+\textbf{y}_S^{\perp}$, donde $\textbf{y}_S\in S$ e $\textbf{y}_S^{\perp}$ es en $S^{\perp}$, el subespacio ortogonal de $S$.\\

Una proyección de un vector sobre un espacio lineal $S$ es en realidad una transformación lineal del vector y se puede representar mediante una matriz de proyección multiplicada por el vector. Una matriz de proyección $P$ es una matriz cuadrada de $n \times n$ que da la proyección de $\mathbb{R}^n$ al subespacio $S$. Las columnas de $P$ son las proyecciones de los vectores base estándar, y $S$ es la imagen de $P$. Para la matriz de proyección tenemos los siguientes teoremas.

%-------------------- Teorema 3.1 --------------------
\begin{teo}
    Un matriz cuadrada $P$ es una matriz de proyección si y solo si es idempotente. Es decir, $P^2=P$.
\end{teo}

%-------------------- Teorema 3.2 --------------------
\begin{teo}
    Sea $U=(u_1,u_2,\cdots , u_k)$ una base ortonormal para un subespacio $W$ del espacio lineal $V$. La matriz $UU'$ es una matriz proyección de $V$ sobre $W$. Es decir, para cualquier valor $v\in V$ la proyección  de $v$ sobre $W$ es $\Proj_W v=UU'v$.
\end{teo}

La matriz $UU'$ es llamada matriz proyección  par el subespacio $W$. No depende de la elección de la base ortonormal. Si no partimos de una base ortonormal de $W$, podemos construir la matriz de proyección. Esto se puede resumir en el siguiente teorema.

%-------------------- Teorema 3.3 --------------------
\begin{teo}
    Sea $A=(a_1,a_2,\ldots,a_k)$ cualquier base para un subespacio $W$ de $V$. La matriz $A(A'A)^{-1}A'$ es una matriz proyección de $V$ sobre $W$. Es decir, para cualquier vector $v\in V$ la proyección de $v$ sobre $W$ es
    \begin{equation}
	\Proj_W v=A(A'A)^{-1}A'v.
    \end{equation}
\end{teo}

Para comprender los tres teoremas anteriores, es importante el siguiente lema.

%-------------------- Lema 3.1 --------------------
\begin{lema}
    Suponga que $A$ es una matriz $a\times k$ donde sus columnas son linealmente independientes. Entonces, $AA'$ es invertible.\\\\
	Demostración.-\; Considere la transformación $A:\mathbb{R}^k\to \mathbb{R}^k$ determinado por $A$. Ya que las columnas de $A$ son linealmente independientes, esta transformación es uno a uno. Además, el espacio nulo de $A'$ es ortogonal al espacio columna de $A$. Por lo tanto, $A'$ es uno a uno en el espacio columna de $A$ y como resultado, $A'A$ es una transformación uno a uno $\mathbb{R}^k\to \mathbb{R}^k$. Por el teorema de matriz invertida, $A'A$ es invertible.
\end{lema}

Derivemos ahora la matriz de proyección para el espacio de columnas de $A$. Obsérvese que cualquier elemento del espacio de columnas de $A$ es una combinación lineal de las columnas de $A$, es decir, $x_1 a_1 + x_2 a_2 + \cdots + x_k a_k$ . Si escribimos

$$
x=
\left(
    \begin{array}{c}
	x_1\\
	x_2\\
	\vdots\\
	x_k
    \end{array}
\right),
$$

entonces, tenemos

$$x_1a_1+x_2a_2+\cdot + x_ka_k=A\textbf{x}.$$

Ahora, para cualquier vector $v\in \mathbb{R}^n$, denotemos la proyección de $v$ sobre $W$ por $x_p$.

$$\Proj_w v = Ax_p.$$

La matriz de proyección se puede encontrar calculando $x_p$. La proyección del vector $v$ sobre $W$ se caracteriza por el hecho de que $v - \Proj_W v$ es ortogonal a cualquier vector $w$ en $W$. Así tenemos

$$w\cdot (v-\Proj_w v)=0$$

para todo $w$ in $W$. Ya que $w=A\textbf{x}$, tenemos

$$A\textbf{x}(v-Ax_p)=0$$

para todo $\textbf{x}\in \mathbb{R}^n$. Escriba este producto escalar en términos de rendimientos de matrices

