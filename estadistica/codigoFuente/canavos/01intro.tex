\chapter{Introducción y estadística descriptiva}

\setcounter{section}{2}
\section{Medidas numéricas descriptivas}

%--------------------definición 1
\begin{def.} La media de las observaciones $x_1,x_2,\ldots , x_n$ es el promedio aritmético de éstas y se denota por 
    \begin{equation}
	\overline{x} = \sum_{i=1}^{n} \dfrac{x_i}{n}
    \end{equation}
\end{def.}

%--------------------definición 2
\begin{def.} La mediana de un conjunto de observaciones es el valor para el cual, todas las observaciones se ordenan de manera creciente, la mitad de éstas es menor que este valor y la otra mitad mayor.
\end{def.}

Si el número de observaciones en el conjunto es impar, la mediana es el valor de la observación que se encuentra a la mitad del conjunto ordenando. Si el número es par se considera la mediana como el promedio aritmético de los valores de las dos observaciones que se encuentran a la mitad del conjunto ordenado. La mediana es el percentil cincuenta. 

%--------------------definición 3
\begin{def.} La moda de un conjunto de observaciones es el valor de la observación que ocurre con mayor frecuencia en el conjunto.
\end{def.}

    \begin{equation}
	\overline{x} = \sum_{i=1}^{k} \dfrac{f_i x_i}{n}
    \end{equation}

    \begin{equation}
	Mediana = L + c\left( \dfrac{j}{f_m}\right)
    \end{equation}

%--------------------definición 4
\begin{def.} La varianza de las observaciones $x_1,x_2,\ldots , x_n$ es, en esencia, el promedio del cuadrado de las distancias entre cada observación y la media del conjunto de observaciones. La varianza se denota por 
    \begin{equation}
	s^2 = \sum_{i=1}^{n} \dfrac{(x_i - \overline{x})^2}{n-1}
    \end{equation}
\end{def.}

%--------------------definición 5
\begin{def.} La raíz cuadrada positiva de la varianza recibe el nombre de la desviación estándar y se denota por 
    \begin{equation}
	s = \sqrt{\sum_{i=1}^{n} \dfrac{(x_i - \overline{x})^2}{n-1}}
    \end{equation}
\end{def.}

El uso de la ecuación (1.4) puede dar origen a errores grandes por redondeo. Con un poco de álgebra se obtiene, a partir de (1.4), una fórmula computacional más exacta para esas condiciones:
$$s^2 = \sum_{i=1}^n \dfrac{(x_i-\overline{x})^2}{n-1} = \dfrac{\sum \left(x_i^2 - 2\overline{x} x_i + \overline{x}^2\right)}{n-1} = \dfrac{\sum x_i^2 - 2\overline{x} \sum x_i + n\overline{x}^2}{n-1} = \dfrac{\sum x_1^2 - \dfrac{2\left(\sum x_i\right)\left(\sum x_1\right)}{n} + \dfrac{n\left(\sum x_i\right)^2}{n^2}}{n-1} = $$

    \begin{equation}
	s^2 = \dfrac{\sum\limits_{i=1}^n x_i^2 - \dfrac{\left(\sum\limits_{i=1}^n x_i\right)^2}{n}}{n-1}
    \end{equation}
    \vspace{.5cm}

    \begin{equation}
	s = \sqrt{\dfrac{\sum\limits_{i=1}^n x_i^2 - \dfrac{\left(\sum\limits_{i=1}^n x_i\right)^2}{n}}{n-1}}
    \end{equation}\\

Para datos agrupados, puede calcularse el valor aproximado de la varianza mediante el uso de la fórmula
    \begin{equation}
	s^2 = \dfrac{\sum\limits_{i=1}^k f_i (x_i-\overline{x})^2}{n-1}
    \end{equation}
ó
    \begin{equation}
	s^2 =  \dfrac{\sum\limits_{i=1}^k f_ix_i^2 - \dfrac{\left(\sum\limits_{i=1}^k f_i x_i\right)^2}{n}}{n-1}
    \end{equation}\\
La fórmula para la desviación estándar es 

    \begin{equation}
	s = \sqrt{\sum\limits_{i=1}^k \dfrac{f_i(x_i-\overline{x})^2}{n-1}}
    \end{equation}

\begin{def.}
    La desviación media es el promedio de los valores absolutos de las diferencias entre cada observación y la media de las observaciones. La desviación media está dada por
    $$D.M. = \dfrac{\sum\limits_{i=1}^n|x_i-\overline{x}|}{n}$$
\end{def.}
Para datos agrupados, el valor de la desviación media se aproxima por 
\begin{equation}
    D.M. = \dfrac{\sum\limits_{i=1}^k f_i|x_i-\overline{x}|}{\sum\limits_{i=1}^k f_i}
\end{equation}

\begin{def.} La desviación mediana es el promedio de los valores absolutos de las diferencias entre cada observación y la mediana de éstas. Esta dada por:
    \begin{equation}
	D.M. = \dfrac{\sum\limits_{i=1}^n|x_i-D.Md|}{n}
    \end{equation}
\end{def.}
