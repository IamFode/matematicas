\chapter{Muestras aleatorias y distribuciones de muestreo}

\begin{def.}
    Si las variables aleatorias $X_1$, $X_2,\ldots, X_n$ tiene la misma función (densidad) de probabilidad que la de la distribución de la población y su función (distribución) conjunta de probabilidad es igual al producto de las marginales, entonces $X_1,X_2,\ldots , X_n$ forman un conjunto de $n$ variables aleatorias independientes e idénticamente distribuidas (IID) que constituyen una muestra aleatoria de la población.
\end{def.}

\setcounter{section}{2}
\section{Distribuciones de muestre de estadísticas}


\begin{def.}
    Un parámetro es una caracterización numérica de las distribuciones de la población de manera que describe, parcial o completamente, la función de densidad de probabilidad de la característica de interés. Por ejemplo, cuando se especifica el valor del parámetro de escala exponencial $\theta$, se describe de manera completa la función de probabilidad
    $$f(x;\theta)=\dfrac{1}{\theta}e^{(-x/\theta)}.$$
    La oración describe de manera completa, sugiere que una vez que se conoce el valor de $\theta$ entonces, puede formularse cualquier proposición probabilistica de interes.
\end{def.}

Antes de dar la definición de una estadística, debe notarse que desde un punto de vista clásico (no bayesiano), un parámetro se considera como una cosntante fija cuyo valor se desconoce. Desde una perspectiva bayesiana un parámetro siempre es una variable aleatoria con algún tipo de distribución de probabilidad.

\begin{def.}
    Una estadística es cualquier función de las variables aleatorias que se observaron en la muestra de manera que esta función no contiene cantidades desconocidas.
\end{def.}

De manera general, denótese una estadística por $T=u(X)$. Dado que $T$ es una función de variables aleatorias, es en sí misma una variable aleatoria, y su valor específico $t=u(x)$ puede determinarse cuando se conozcan las realizaciones $x$ de $X$. Si se emplea una estadística $T$ para estimar un parámetro desconocido $\theta$, entonces $T$ recibe el nombre de \textbf{estimador} de $\theta$, y el valor específico de $t$ como un resultado de los datos muestrales recibe el nombre de \textbf{estimación} de $\theta$. Esto es, un estimador es una estadística que identifica al mecanismo funcional por medio del cual, una vez que las observaciones en la muestra se realizan, se obtiene una estimación.\\

Un parámetro es una constante pero una estadística es una variable aleatoria. Además, un valor del parámetro descrito describe de manera completa un modelo de probabilidad; ningún valor de la estadística puede desempeñar tal papel si cada uno de estos depende del valor de las observaciones de las muestras. 

\begin{def.}
    La distribución de muestre de una estadística $T$ es la distribución de probabilidad de $T$ que puede obtenerse como resultado de un número infinito de muestras aleatorias independientes, cada una de tamaño $n$, provenientes de la población de interés.
\end{def.}
