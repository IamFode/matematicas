\chapter{Variables aleatorias y distribución de probabilidad}

\section{El concepto de variables aleatorias}

%-------------------- Definición 3.1
\begin{tcolorbox}[colframe=white]
    \begin{def.} Sea $S$ un espacio muestral sobre el que se encuentra definida una función de probabilidad. Sea $X$ una función de valor real definida sobre $S$, de manera que transforme los resultados de $S$ en puntos sobre la recta de los reales. Se dice entonces que $X$ es un variable aleatoria.
	\end{def.}
\end{tcolorbox}

%-------------------- Definición 3.2
\begin{tcolorbox}[colframe=white]
	\begin{def.} Se dice que una variable aleatoria $X$ es discreta si el número de valores que puede tomar es contable (ya sea finito o infinito), y si éstos pueden arreglarse en una secuencia que corresponde con los enteros positivos.
	\end{def.}
\end{tcolorbox}
%-------------------- Definición 3.3
\begin{tcolorbox}[colframe=white]
    	\begin{def.} Se dice que una variable aleatoria $X$ es continua si sus valores consisten en uno o más intervalos de la recta de los reales.
	\end{def.}
\end{tcolorbox}

\section{Distribuciones de  probabilidad de variables aleatorias discretas}
