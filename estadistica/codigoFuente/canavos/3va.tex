\chapter{Variables aleatorias y distribución de probabilidad}

\section{El concepto de variables aleatorias}

%-------------------- Definición 3.1
\begin{tcolorbox}[colframe=white]
    \begin{def.} Sea $S$ un espacio muestral sobre el que se encuentra definida una función de probabilidad. Sea $X$ una función de valor real definida sobre $S$, de manera que transforme los resultados de $S$ en puntos sobre la recta de los reales. Se dice entonces que $X$ es un variable aleatoria.
	\end{def.}
\end{tcolorbox}

%-------------------- Definición 3.2
\begin{tcolorbox}[colframe=white]
	\begin{def.} Se dice que una variable aleatoria $X$ es discreta si el número de valores que puede tomar es contable (ya sea finito o infinito), y si éstos pueden arreglarse en una secuencia que corresponde con los enteros positivos.
	\end{def.}
\end{tcolorbox}
%-------------------- Definición 3.3
\begin{tcolorbox}[colframe=white]
    	\begin{def.} Se dice que una variable aleatoria $X$ es continua si sus valores consisten en uno o más intervalos de la recta de los reales.
	\end{def.}
\end{tcolorbox}

\section{Distribuciones de  probabilidad de variables aleatorias discretas}

%-------------------- Definición 3.4
\begin{tcolorbox}[colframe=white]
    \begin{def.} Sea $X$ una variable aleatoria discreta. Se llamará a $p(x) = P(X=x)$ función de probabilidad de la variable aleatoria $X$, si satisface las siguientes propiedades:
	\begin{enumerate}[\bfseries 1.]
	    \item $p(x)\geq 0$  para todos los valores $x$ de $X$;
	    \item $\sum\limits_x p(x)=1.$
	\end{enumerate}
	\end{def.}
\end{tcolorbox}

%-------------------- Definición 3.5
\begin{tcolorbox}[colframe=white]
	\begin{def.}
	    La función de distribución acumulativa de la variable aleatoria $X$ es la probabilidad de que $X$ sea menor o igual a un valor específico de $x$ y está dada por:
	    $$F(x) = P(X\leq x) = \sum\limits_{x_i \leq x} p(x_i)$$
	\end{def.}
\end{tcolorbox}

En general, la función de distribución acumulativa $F(x)$ de una variable aleatoria discreta es una función no decreciente de los valores de $X$, de tal manera que:
\begin{enumerate}[\bfseries 1.]
    \item $0\leq F(x) \leq 1$ para cualquier $x$;
    \item $F(x_i)\geq F(x_j)$ si $x_i\geq x_j;$
    \item $P(X>x) = 1 - F(x).$
    \item $P(X=x) = F(x) - F(x-1);$
    \item $P(x_i \geq X \geq x_j) = F(x_j) - F(x_i - 1)$
\end{enumerate}

\section{Distribuciones de probabilidad de variables aleatorias continuas}

%-------------------- Definición 3.6
\begin{tcolorbox}[colframe=white]
    \begin{def.}
	\begin{enumerate}[\bfseries 1.]
	    \item $f(x)\geq 0$, $-\infty<x<\infty$,
	    \item $\displaystyle\int_{-\infty}^\infty f(x) \;dx$ y 
	    \item $P(a\leq X \leq b) = \displaystyle\int_{a}^{b} f(x) \; dx$
	\end{enumerate}
    \end{def.}
\end{tcolorbox}

Para la función de distribución acumulativa $F(x)$ se tiene:
\begin{tcolorbox}[colframe=white]
    $$P(X\leq x) = F(x) = \int_{-\infty}^x f(t) \; dt$$
\end{tcolorbox}

Dado que para cualquier varible aleatoria continua $X$,
$$P(X=x) = \inf_x^x f(t)\; dt = 0, \qquad \Longrightarrow \qquad P(X\leq x) = P(X<x) = F(x)$$\\

La distribución acumulativa $F(x)$ es una función lisa no decreciente de los valores de la v.a. con las siguiente propiedades:
\begin{enumerate}[\bfseries 1.]
    \item $F(-\infty) = 0;$
    \item $F(\infty) = 1$;
    \item $P(a<X<b) = F(b) - F(a)$
    \item $dF(x)/dx = f(x).$
\end{enumerate}

\section{Valor esperado de una variable aleatoria}

%-------------------- Definición 3.7
\begin{tcolorbox}[colframe=white]
    \begin{def.}
	El valor esperado de una variable aleatoria $X$ es el promedio o valor medio de $X$ y está dado por:
	$$\begin{array}{rcll}
	    E(X)& = & \sum\limits_{x} xp(x) & \mbox{Si $x$ es discreta, o}\\\\
	    E(X) & = & \displaystyle\int_{-\infty}^\infty x f(x) \; dx & \mbox{Si $x$ es continua.}\\\\
	\end{array}$$
	En donde $p(x)$ y $f(x)$ son las funciones de probabilidad y de densidad de probabilidad, respectivamente.\\\\	
	En general, el valor esperado de una función $g(x)$ de la variables aleatoria $X$, está dado por:
	$$\begin{array}{rcll}
	    E(g(X))& = & \sum\limits_{x} g(x) p(x) & \mbox{Si $x$ es discreta, o}\\\\
	    E(g(X)) & = & \displaystyle\int_{-\infty}^\infty g(x)x f(x) \; dx & \mbox{Si $x$ es continua.}\\\\
	\end{array}$$
    \end{def.}
\end{tcolorbox}

\subsection{Propiedades}
\begin{enumerate}[\bfseries 1.]

    %---------- 1.
    \item El valor esperado de una constante $c$ es el valor de la constante.
	$$E(c) = \displaystyle\int_{-\infty}^\infty cf(x)\; dx = c\int_{-\infty}^{\infty} f(x) \; dx = c$$
    
    %---------- 2.
    \item El valor esperado de la cantidad $aX+b,$ en donde $a$ y $b$ son constantes, es el producto de $a$ por el valor esperado de $x$ más $b$.
	$$E(aX+b) = \displaystyle\int_{-\infty}^\infty (ax+b)f(x)\; dx = a\int_{-\infty}^\infty xf(x)\; dx + b \int_{-\infty}^\infty f(x)\; dx = aE(X)+b$$

    %---------- 3.
    \item El valor esperado de la suma de dos funciones $g(X)$ y $h(X)$ de $X$ es la suma de los valores esperados de $g(X)$ y $h(X)$.
	$$E\left[g(X) + h(X)\right] = \displaystyle\int_{-\infty}^\infty \left[g(x) + h(x)\right]\; dx \int_{-\infty}^\infty g(x)f(x)\; dx + \int_{-\infty}^infty h(x)f(x)\; dx = E\left[g(X)\right] + E\left[h(X)\right]$$
\end{enumerate}

\section{Momentos de una variable aleatoria}
