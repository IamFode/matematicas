\chapter{Distribuciones conjuntas de probabilidad}

\section{Distribución de probabilidad bivariadas}

\begin{tcolorbox}
    \begin{def.}
	Sean $X$ e $Y$ dos variables aleatorias discretas. La probabilidad de que $X=x$ e $Y=y$ está determinada por la función de probabilidad bivariada
	$$p(x,y)=P(X=x, Y=x),$$
	en donde $P(x,y)\geq 0$ para toda $x,y,$ de $X$, $Y,$ y $\sum_x \sum_y p(x,y)=1.$
    \end{def.}
\end{tcolorbox}

La \textbf{función de distribución acumulativa bivariada} es la probabilidad conjunta de que $X\leq x$ y $Y\leq y,$ dada por
\begin{tcolorbox}
    $$F(x,y)=P(X\leq x, Y\leq y) = \sum_{x_i\leq x}\sum_{y_i\leq y} p(x_i,y_i).$$
\end{tcolorbox}

La \textbf{función de distribución trinomial} viene dado por:

\begin{tcolorbox}
    $$p(x,y,n,p_1,p_2)=\dfrac{n!}{x!y!(n-x-y)!}p_1^x p_2^y (1-p_1-p_2)^{n-x-y}$$
\end{tcolorbox}
 
y su generalización llamada \textbf{función de distribución multinomial} viene dada por:

\begin{tcolorbox}
    $$p(x_1,x_2,\ldots, x_{k-1};n,p_1,p_2,\ldots,p_{k-1})=\dfrac{n!}{x_1!x_2!\ldots x_k!}p_1^{x_1}p_2^{x_2}\cdots p_k^{x_k}, \quad x_1=0,1,\ldots, n\; \mbox{para}\; i=1,2,\ldots, k$$
    en donde $x_k=n-x_1-x_2-\cdots - x_{k-1}$ y $p_k=1-p_1-p_2-\cdots - p_{k-1}.$
\end{tcolorbox}

\begin{tcolorbox}
    \begin{def.}
	Sean $X$ e $Y$ dos variables aleatorias continuas. Si existe una función $f(x,y)$ tal que la probabilidad conjunta:
	$$P(a<X<b,c<Y<d)=\int_a^b \int_c^d f(x,y)\; dy dx$$
	para cualquier valor de $a,b,c$ y $d$ en donde $f(x,y)\geq 0,$ $-\infty<x, y<\infty$ y $\int_{\infty}^\infty \int_{-\infty}^\infty f(x,y)\; dy dx=1$, entonces $f(x,y)$ es la función de densidad de probabilidad bivariada de $X$ e $Y$. 
    \end{def.}
\end{tcolorbox}

La \textbf{función de distribución bivariada acumulativa} de $X$ e $Y$ es la probabilidad conjunta de que $X\leq x$ e $Y\leq y$, dada por:
\begin{tcolorbox}
    $$P(X\leq x, Y\leq y)=F(x,y)=\int_{-\infty}^x \int_{-\infty}^y f(u,v)\; dvdu.$$
\end{tcolorbox}
Por lo tanto, la función de densidad bivariada se encuentra diferenciando $F(x,y)$ con respecto a $x$ e $y$; es decir,

\begin{tcolorbox}
    $$f(x,y)=\dfrac{\partial^2 F(x,y)}{\partial x \partial y}$$
\end{tcolorbox}


\section{Distribuciones marginales de probabilidad}
Es posible determinar varias distribuciones marginales para cualquier distribución de probabilidad que contenta más de dos variables aleatorias.

\begin{tcolorbox}
    \begin{def.}
	Sean $X$ e $Y$ dos variables aleatorias discretas con una función de probabilidad conjunta $p(x-y)$. Las funciones marginales de probabilidad de $X$ y de $Y$ están dadas por
	$$p_X(x)=\sum_y p(x,y)\qquad \mbox{y}\qquad p_Y(y)=\sum_x p(x,y),$$
	respectivamente.
    \end{def.}
\end{tcolorbox}

\begin{tcolorbox}
    \begin{def.}
	Sean $X$ e $Y$ dos variables aleatorias continuas con una función de densidad de probabilidad conjunta $f(x,y)$. Las funciones de densidad de probabilidad de $X$ e $Y$ están dadas por
	$$f_X(x)=\int_{-\infty}^\infty f(x,y)\; dy \qquad \mbox{y} \qquad f_Y(y)=\int_{-\infty}^\infty f(x,y)\; dx,$$
	respectivamente.
    \end{def.}
\end{tcolorbox}

Para variables aleatorias continuas conjuntas, si se conoce \textbf{\boldmath la función de distribución acumulativa $F(x,y)$}, las distribuciones acumulativas marginales de $X$ e $Y$ se obtienen de la siguiente forma:

\begin{tcolorbox}
    $$P(X\leq x)=F_X(x)=\int_{\infty}^x \int_{-\infty}^\infty f(t,y)\; dydt,\qquad \mbox{y}\qquad F_X(x)=\int_{-\infty}^x f_X(t)\; dt = F(x,\infty)$$
\end{tcolorbox}

De manera similar

\begin{tcolorbox}
    $$P(Y\leq y)=F_Y(y)=\int_{\infty}^y \int_{-\infty}^\infty f(x,t)\; dxdt=\int_{-\infty}^y f_Y(t)\; dt = F(\infty,y).$$
\end{tcolorbox}

Así puede determinarse la distribución acumulativa marginal de $X$ dejando que $Y$ tome un valor igual al límite superior de la función de distribución conjunta de $X$ e $Y$.

\section{Valores esperados y momentos para distribuciones bivariadas}

