\chapter{Teoría elemental de la probabilidad}

%-------------------- axioma 1
\begin{tcolorbox}[colframe=white]
    \begin{axioma} Sea $E$ una colección de elementos $\xi.\eta,\zeta,\ldots,$ que llamaremos sucesos elementales, y $\mathfrak{F}$ un conjunto de subconjuntos de $E$; los elementos del conjunto $\mathfrak{F}$ se llamarán eventos aleatorios.

	\begin{enumerate}[\bfseries I.]

	    %----------I.
	    \item $\mathfrak{F}$ es un campo de conjuntos. (Un sistema de conjuntos se denomina campo si la suma, el producto y la diferencia de dos conjuntos del sistema también pertenecen al mismo sistema).

	    %----------II.
	    \item $\mathfrak{F}$ contiene el conjunto $E$.

	    %----------III.
	    \item A cada conjunto $A$ en $\mathfrak{F}$ se le asigna un número real no negativo $P(A)$. Este número $P(A)$ se llama probabilidad del evento $A$.

	    %----------IV.
	    \item $P(E)$ es igual a $1$.

	    %----------V.
	    \item Si $A$ y $B$ no tienen ningún elemento en común, entonces $$P(A+B)=P(A)+P(B)$$

	\end{enumerate}
    \end{axioma}
\end{tcolorbox}

\setcounter{section}{3}
\section{Corolarios inmediatos de los axiomas; Probabilidades condicionales; teorema de Bayes}
