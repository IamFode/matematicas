\chapter{Teoría elemental de la probabilidad}

%-------------------- axioma 1
\begin{tcolorbox}[colframe=white]
Sea $E$ una colección de elementos $\xi.\eta,\zeta,\ldots,$ que llamaremos sucesos elementales, y $\mathfrak{F}$ un conjunto de subconjuntos de $E$; los elementos del conjunto $\mathfrak{F}$ se llamarán eventos aleatorios.

    %----------I.
    \begin{axioma}
	$\mathfrak{F}$ es un campo de conjuntos. (Un sistema de conjuntos se denomina campo si la suma, el producto y la diferencia de dos conjuntos del sistema también pertenecen al mismo sistema).

    \end{axioma}

    %----------II.
    \begin{axioma}
	$\mathfrak{F}$ contiene el conjunto $E$.
    \end{axioma}

    %----------III.
    \begin{axioma}
	A cada conjunto $A$ en $\mathfrak{F}$ se le asigna un número real no negativo $P(A)$. Este número $P(A)$ se llama probabilidad del evento $A$.
    \end{axioma}

    %----------IV.
    \begin{axioma}
	$P(E)$ es igual a $1$.
    \end{axioma}

    %----------V.
    \begin{axioma}
	Si $A$ y $B$ no tienen ningún elemento en común, entonces $$P(A+B)=P(A)+P(B)$$
    \end{axioma}

\end{tcolorbox}

\setcounter{section}{3}
\section{Corolarios inmediatos de los axiomas; Probabilidades condicionales; teorema de Bayes}
De $A+\overline{A}=E$ y los axiomas IV y V se sigue que, 
\begin{equation}
    P(A)+P(\overline{A})=1
\end{equation}

\begin{equation}
    P(\overline{A} = 1-P(A).
\end{equation}

Ya que $\overline{E}=0$, en particular se tiene, 
\begin{equation}
    P(0)=0.
\end{equation}
\vspace{.2cm}

Si $A,B,...,N$ son incompatibles, entonces por el Axioma V se sigue la fórmula (\textbf{teorema de la suma}), 
\begin{equation}
    P(A+B+\ldots,+N) = P(A) + P(B) + \ldots + P(N)
\end{equation}
\vspace{.2cm}

Si $P(A)>0,$ entonces el cociente 
\begin{equation}
    P_A(B)=\dfrac{P(AB}{P(A)}
\end{equation}
Es definida como la probabilidad condicional del evento $B$ bajo la condición $A$.\\
Luego por (.5) se sigue que, 
\begin{equation}
    P(AB)=P(A)P_A(B).
\end{equation}


