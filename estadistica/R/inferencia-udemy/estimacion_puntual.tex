% Options for packages loaded elsewhere
\PassOptionsToPackage{unicode}{hyperref}
\PassOptionsToPackage{hyphens}{url}
%
\documentclass[
]{article}
\usepackage{amsmath,amssymb}
\usepackage{lmodern}
\usepackage{ifxetex,ifluatex}
\ifnum 0\ifxetex 1\fi\ifluatex 1\fi=0 % if pdftex
  \usepackage[T1]{fontenc}
  \usepackage[utf8]{inputenc}
  \usepackage{textcomp} % provide euro and other symbols
\else % if luatex or xetex
  \usepackage{unicode-math}
  \defaultfontfeatures{Scale=MatchLowercase}
  \defaultfontfeatures[\rmfamily]{Ligatures=TeX,Scale=1}
\fi
% Use upquote if available, for straight quotes in verbatim environments
\IfFileExists{upquote.sty}{\usepackage{upquote}}{}
\IfFileExists{microtype.sty}{% use microtype if available
  \usepackage[]{microtype}
  \UseMicrotypeSet[protrusion]{basicmath} % disable protrusion for tt fonts
}{}
\makeatletter
\@ifundefined{KOMAClassName}{% if non-KOMA class
  \IfFileExists{parskip.sty}{%
    \usepackage{parskip}
  }{% else
    \setlength{\parindent}{0pt}
    \setlength{\parskip}{6pt plus 2pt minus 1pt}}
}{% if KOMA class
  \KOMAoptions{parskip=half}}
\makeatother
\usepackage{xcolor}
\IfFileExists{xurl.sty}{\usepackage{xurl}}{} % add URL line breaks if available
\IfFileExists{bookmark.sty}{\usepackage{bookmark}}{\usepackage{hyperref}}
\hypersetup{
  pdftitle={Estimación puntual},
  pdfauthor={Christian Limbert Paredes Aguilera},
  hidelinks,
  pdfcreator={LaTeX via pandoc}}
\urlstyle{same} % disable monospaced font for URLs
\usepackage[margin=1in]{geometry}
\usepackage{color}
\usepackage{fancyvrb}
\newcommand{\VerbBar}{|}
\newcommand{\VERB}{\Verb[commandchars=\\\{\}]}
\DefineVerbatimEnvironment{Highlighting}{Verbatim}{commandchars=\\\{\}}
% Add ',fontsize=\small' for more characters per line
\usepackage{framed}
\definecolor{shadecolor}{RGB}{248,248,248}
\newenvironment{Shaded}{\begin{snugshade}}{\end{snugshade}}
\newcommand{\AlertTok}[1]{\textcolor[rgb]{0.94,0.16,0.16}{#1}}
\newcommand{\AnnotationTok}[1]{\textcolor[rgb]{0.56,0.35,0.01}{\textbf{\textit{#1}}}}
\newcommand{\AttributeTok}[1]{\textcolor[rgb]{0.77,0.63,0.00}{#1}}
\newcommand{\BaseNTok}[1]{\textcolor[rgb]{0.00,0.00,0.81}{#1}}
\newcommand{\BuiltInTok}[1]{#1}
\newcommand{\CharTok}[1]{\textcolor[rgb]{0.31,0.60,0.02}{#1}}
\newcommand{\CommentTok}[1]{\textcolor[rgb]{0.56,0.35,0.01}{\textit{#1}}}
\newcommand{\CommentVarTok}[1]{\textcolor[rgb]{0.56,0.35,0.01}{\textbf{\textit{#1}}}}
\newcommand{\ConstantTok}[1]{\textcolor[rgb]{0.00,0.00,0.00}{#1}}
\newcommand{\ControlFlowTok}[1]{\textcolor[rgb]{0.13,0.29,0.53}{\textbf{#1}}}
\newcommand{\DataTypeTok}[1]{\textcolor[rgb]{0.13,0.29,0.53}{#1}}
\newcommand{\DecValTok}[1]{\textcolor[rgb]{0.00,0.00,0.81}{#1}}
\newcommand{\DocumentationTok}[1]{\textcolor[rgb]{0.56,0.35,0.01}{\textbf{\textit{#1}}}}
\newcommand{\ErrorTok}[1]{\textcolor[rgb]{0.64,0.00,0.00}{\textbf{#1}}}
\newcommand{\ExtensionTok}[1]{#1}
\newcommand{\FloatTok}[1]{\textcolor[rgb]{0.00,0.00,0.81}{#1}}
\newcommand{\FunctionTok}[1]{\textcolor[rgb]{0.00,0.00,0.00}{#1}}
\newcommand{\ImportTok}[1]{#1}
\newcommand{\InformationTok}[1]{\textcolor[rgb]{0.56,0.35,0.01}{\textbf{\textit{#1}}}}
\newcommand{\KeywordTok}[1]{\textcolor[rgb]{0.13,0.29,0.53}{\textbf{#1}}}
\newcommand{\NormalTok}[1]{#1}
\newcommand{\OperatorTok}[1]{\textcolor[rgb]{0.81,0.36,0.00}{\textbf{#1}}}
\newcommand{\OtherTok}[1]{\textcolor[rgb]{0.56,0.35,0.01}{#1}}
\newcommand{\PreprocessorTok}[1]{\textcolor[rgb]{0.56,0.35,0.01}{\textit{#1}}}
\newcommand{\RegionMarkerTok}[1]{#1}
\newcommand{\SpecialCharTok}[1]{\textcolor[rgb]{0.00,0.00,0.00}{#1}}
\newcommand{\SpecialStringTok}[1]{\textcolor[rgb]{0.31,0.60,0.02}{#1}}
\newcommand{\StringTok}[1]{\textcolor[rgb]{0.31,0.60,0.02}{#1}}
\newcommand{\VariableTok}[1]{\textcolor[rgb]{0.00,0.00,0.00}{#1}}
\newcommand{\VerbatimStringTok}[1]{\textcolor[rgb]{0.31,0.60,0.02}{#1}}
\newcommand{\WarningTok}[1]{\textcolor[rgb]{0.56,0.35,0.01}{\textbf{\textit{#1}}}}
\usepackage{graphicx}
\makeatletter
\def\maxwidth{\ifdim\Gin@nat@width>\linewidth\linewidth\else\Gin@nat@width\fi}
\def\maxheight{\ifdim\Gin@nat@height>\textheight\textheight\else\Gin@nat@height\fi}
\makeatother
% Scale images if necessary, so that they will not overflow the page
% margins by default, and it is still possible to overwrite the defaults
% using explicit options in \includegraphics[width, height, ...]{}
\setkeys{Gin}{width=\maxwidth,height=\maxheight,keepaspectratio}
% Set default figure placement to htbp
\makeatletter
\def\fps@figure{htbp}
\makeatother
\setlength{\emergencystretch}{3em} % prevent overfull lines
\providecommand{\tightlist}{%
  \setlength{\itemsep}{0pt}\setlength{\parskip}{0pt}}
\setcounter{secnumdepth}{-\maxdimen} % remove section numbering
\ifluatex
  \usepackage{selnolig}  % disable illegal ligatures
\fi

\title{Estimación puntual}
\author{Christian Limbert Paredes Aguilera}
\date{2/12/2021}

\begin{document}
\maketitle

\hypertarget{la-media-muestral}{%
\subsection{La media muestral}\label{la-media-muestral}}

Sea \(X_1,\ldots X_n\) una muestra aleatoria simple de tamaño \(n\) de
una v.a. \(X\) de esperanza \(\mu_X\) y desviación típica \(\sigma_X\)
La media muestral es: \[\overline{X} = \dfrac{X_1+\ldots + X_n}{n}\]

En estas condiciones,
\[E(\overline{X})=\mu_X,\qquad \sigma_X = \dfrac{\sigma_X}{\sqrt{n}}\]

donde \(\sigma_{\overline{X}}\) es el error estándar de \(\overline{X}\)

\begin{itemize}
\item Es un estimador puntal de $\mu_X$
\item $E(\overline{X})=\mu_X:$ el valor esperado de de $\overline{X}$ es $\mu_X$.
\item Si tomamos muchas veces una m.a.s. y calculamos la media muestral, el valor medio de estas medias tiende con mucha probabilidad a ser $\mu_X$
\item $\sigma_{\overline{X}} = \sigma_X / \sqrt{n}:$ la variabilidad de los resultados de $\overline{X}$ tiende a $0$ a medida que tomamos muestras más grandes.
\end{itemize}

\hypertarget{ejercicio}{%
\subsubsection{Ejercicio}\label{ejercicio}}

\begin{enumerate}
\item Generar 10000 muestra de tamaño 40 con repocisión de las longitudes del pétalo.
\item A continuación hallaremos los valores medios de cada muestra.
\item Consideraremos la media y la desviación típica de dichos valores medios y los compararemos con los valores exactos dados por las propiedades de la media muestral.
\end{enumerate}

\begin{Shaded}
\begin{Highlighting}[]
\FunctionTok{set.seed}\NormalTok{(}\DecValTok{1001}\NormalTok{)}
\CommentTok{\#1}
\NormalTok{valores.medios.long.pétalo }\OtherTok{\textless{}{-}} \FunctionTok{replicate}\NormalTok{(}\DecValTok{10000}\NormalTok{,}\FunctionTok{mean}\NormalTok{(}\FunctionTok{sample}\NormalTok{(iris}\SpecialCharTok{$}\NormalTok{Petal.Length,}
                            \DecValTok{40}\NormalTok{,}
                            \AttributeTok{replace =} \ConstantTok{TRUE}\NormalTok{)))}
\FunctionTok{head}\NormalTok{(valores.medios.long.pétalo,}\DecValTok{10}\NormalTok{)}
\end{Highlighting}
\end{Shaded}

\begin{verbatim}
##  [1] 3.5975 3.5150 3.9400 3.2650 3.9125 3.9650 4.2825 3.2950 3.8500 3.7850
\end{verbatim}

\begin{Shaded}
\begin{Highlighting}[]
\CommentTok{\#2}
\FunctionTok{mean}\NormalTok{(valores.medios.long.pétalo)}
\end{Highlighting}
\end{Shaded}

\begin{verbatim}
## [1] 3.754478
\end{verbatim}

\begin{Shaded}
\begin{Highlighting}[]
\FunctionTok{sd}\NormalTok{(valores.medios.long.pétalo)}
\end{Highlighting}
\end{Shaded}

\begin{verbatim}
## [1] 0.2796513
\end{verbatim}

\begin{Shaded}
\begin{Highlighting}[]
\CommentTok{\#3}
\FunctionTok{mean}\NormalTok{(iris}\SpecialCharTok{$}\NormalTok{Petal.Length)}
\end{Highlighting}
\end{Shaded}

\begin{verbatim}
## [1] 3.758
\end{verbatim}

\begin{Shaded}
\begin{Highlighting}[]
\FunctionTok{sd}\NormalTok{(iris}\SpecialCharTok{$}\NormalTok{Petal.Length)}\SpecialCharTok{/}\FunctionTok{sqrt}\NormalTok{(}\DecValTok{40}\NormalTok{)}
\end{Highlighting}
\end{Shaded}

\begin{verbatim}
## [1] 0.2791182
\end{verbatim}

\hypertarget{poblaciones-normales}{%
\subsection{poblaciones normales}\label{poblaciones-normales}}

\hypertarget{combinaciuxf3n-lineal-de-distribuciones-normales}{%
\subsubsection{Combinación lineal de distribuciones
normales}\label{combinaciuxf3n-lineal-de-distribuciones-normales}}

La combinación lineal de distribuciones normlaes es normal. es decir, si
\(Y_i,\ldots Y_n\) son v.a. normales independientes, cada
\(Y_i \sim N(\mu_i,\sigma_i)\) y \(a_1,\ldots, a_n, b\in \mathbb{R}\)
entonces \[Y=a_iY_i + \ldots a_nY_n + b\] es una v.a. \(N(\mu,\sigma)\)
con \(\mu\) y \(\sigma\) las que correspondan:

\begin{itemize}
\item $E(Y)a_i\cdot \mu_i + \ldots a_n \cdot \mu_n + b$ 
\item $\sigma\left(Y\right)^2 = a_1^2 \cdot \sigma_1^2 + \ldots + a_n^2 \cdot \sigma_n^2$
\end{itemize}

\hypertarget{distribuciuxf3n-de-la-media-muestral}{%
\subsubsection{Distribución de la media
muestral}\label{distribuciuxf3n-de-la-media-muestral}}

Sea \(X_1, \ldots , X_n\) una m.a.s de una v.a. \(X\) de esperanza
\(\mu_X\) y desviación típica \(\sigma_X\) Si \(X\) es
\(N(\mu_X,\sigma_X)\) entonces
\[\overline{X} \;\mbox{es}\; N\left(\mu_X,\dfrac{\sigma_X}{\sqrt{n}}\right)\]

y por tanto

\[Z = \dfrac{\overline{X}}{\frac{\sigma_X}{\sqrt{n}}}\; \mbox{es}\; N(0,1)\]

\hypertarget{teorema-central-del-luxedmite}{%
\subsubsection{Teorema central del
límite}\label{teorema-central-del-luxedmite}}

Sea \(SX_1,\ldots,X_n\) una m.a.s. de una v.a. \(X\) cualquiera, de
esperanza \(\mu_X\) y desviación típica \(\sigma_X\). Cuando
\({n\to \infty}\)
\[{\overline{X} \to N\left(\mu_X,\dfrac{\sigma_X}{\sqrt{n}}\right)}\]

y por tanto

\[Z = {\dfrac{\overline{X}-\mu_X}{\frac{\sigma_X}{\sqrt{n}}} \to N(0,1)}\]

Esta convergencia se refiere a las distribuciones

\textbf{Caso n grande}: Si \(n\) es grande \(n\geq 30\),
\(\overline{X}\) es aproximadamente normal, con esperanza \(\mu_X\) y
desviación típica \(\dfrac{\sigma_X}{\sqrt{n}}\)

\end{document}
