\begin{enumerate}[\bfseries \mbox{Problema} 1.]

    %-------------------- problema 1
    \item Sean \textbf{\boldmath $x,y \in \mathbb{R}$ con $x,y\neq 0$. Si $\|x\|=\|y\|,$ entonces hallar la media del ángulo entre $\frac{1}{2}(x+y)$ e $y-x$.}\\\\
	Respuesta.-\; Sea,
	\begin{center}
	$<\frac{1}{2}(x+y),y-x>=\|x\|\|y\|\cos \theta$
	\end{center}
	entonces 
	$$\begin{array}{rcl}
	    \frac{1}{2}(x+y)\cdot (y-x) &=& \|x\|\|y\|\cos \theta \\\\
	    \frac{1}{2}xy-\frac{1}{2}x^2+\frac{1}{2}y^2-\frac{1}{2}xy &=& \sqrt{x^2}\sqrt{x^2}\cos \theta\\\\
		\frac{1}{2}y^2-\frac{1}{2}x^2&=&x^2\cos \theta\\\\
		\theta&=&\arccos\left(\dfrac{y^2}{x^2}-\frac{1}{2}\right).\\\\
	\end{array}$$
	\vspace{.7cm}

    %-------------------- problema 2
    \item Demuestre que si $x+y$ y $x-y$ son ortogonales, entonces los vectores $x$ e $y$ deben tener la misma longitud.\\\\
	Demostración.-\; 

\end{enumerate}
