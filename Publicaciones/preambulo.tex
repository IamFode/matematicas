\documentclass[10pt]{book}

% ------------------- Paquetes ----------------------
\usepackage[text=17cm,left=2.8cm,right=2.4cm, headsep=20pt, top=2.5cm, bottom = 2cm,letterpaper,showframe = false]{geometry} %configuración página
\usepackage{latexsym,amsmath,amssymb,amsfonts,amsthm} %(símbolos de la AMS)
\parindent = 0cm  %sangria
\usepackage[T1]{fontenc} %acentos en español
\usepackage{graphicx} %gráficos y figuras.
\usepackage[spanish,english]{babel}
\usepackage{mathpazo} % tipo de letra
\usepackage{titlesec} %formato de títulos
\usepackage[backref=page]{hyperref} %hipervinculos
\usepackage{multicol} %columnas
\usepackage{tcolorbox} %cajas
\usepackage{enumerate} %indice enumerado
\usepackage{marginnote}%notas en el margen
\tcbuselibrary{skins,breakable,listings,theorems}
\usepackage[Bjornstrup]{fncychap}%diseño de portada de capitulos
\usepackage[all]{xy}%flechas
\counterwithout{footnote}{chapter}
\usepackage{xcolor}
\spanishdecimal{.}
\usepackage{pgfplots}
\usepgfplotslibrary{polar}
\usepackage{tkz-fct}
\usepackage{mathrsfs}
\usepackage[htt]{hyphenat}
\usepackage[fit]{truncate}
\usepackage{titling,lipsum}


%--------- encabezado y pie de paginas -----------------
\usepackage{fancyhdr}

%--------- configuración tcolorbox -----------------
\tcbset{colback=black!2,colframe=white}

%---------- configuración de separación de sección ------------ 
\titlespacing*{\section}{0pt}{1.7cm}{.3cm}

%------------------------------------------

\theoremstyle{definition}
\newtheorem{axioma}{Axioma}[part]
\newtheorem{teo}{Teorema}[chapter]%entorno para teoremas
\newtheorem{def.}{Definición}[chapter]%entorno para definiciones
\newtheorem{ejem}{Ejemplo}[chapter]%entorno para ejemplos
\newtheorem{post}{Postulado}[chapter]%entorno de postulados
\newtheorem{cor}{Corolario}[chapter]
\newtheorem{ej}{Ejercicio}[chapter]
\newtheorem{prop}{Propiedad}[part]
\newtheorem{lema}{Lema}[chapter]
\newtheorem{prob}{Problema}[chapter]
\newtheorem{obs}{Obs:}[chapter]

\makeatletter\renewcommand\theenumi{\@roman\c@enumi}\makeatother

\renewcommand\labelenumi{\theenumi)}
\def\sen{\mathop{\mbox{\normalfont sen}}\nolimits}
\def\cotan{\mathop{\mbox{\normalfont cotan}}\nolimits}
\def\cosec{\mathop{\mbox{\normalfont cosec}}\nolimits}

%---------------------------------
\titleformat*{\section}{\bfseries}
\titleformat*{\subsection}{\bfseries}
\titleformat*{\subsubsection}{\bfseries}
\titleformat*{\paragraph}{\bfseries}
\titleformat*{\subparagraph}{\bfseries}

%----------Formato título de capítulos-------------

\titleformat{\chapter}[display]
{\vspace{4ex}\bfseries\huge}
{\filleft\Huge\thechapter}
{2ex}
{\filleft}

