\usepackage{latexsym,amsmath,amssymb,amsfonts,amsthm} %(símbolos de la AMS)

\usepackage[T1]{fontenc} %acentos en español

\usepackage{graphicx} %gráficos y figuras.

\usepackage[spanish,english]{babel} % Lenguajes

\usepackage{mathpazo} % tipo de letra

\usepackage{titlesec} %formato de títulos

\usepackage[backref=page]{hyperref} %hipervinculos

\usepackage{multicol} %columnas

\usepackage{tcolorbox} %cajas

\usepackage{enumerate} %indice enumerado

\usepackage{marginnote}%notas en el margen

\tcbuselibrary{skins,breakable,listings,theorems} %

\usepackage[Bjornstrup]{fncychap} %diseño de portada de capítulos

\usepackage[all]{xy} %flechas


\usepackage{xcolor} % Color para latex


\usepackage{pgfplots} % Plots


\usepackage{tkz-fct} % graficas

\usepackage{mathrsfs} % Paquete para simbolos matemáticos R

\usepackage[htt]{hyphenat} %El paquete hyphenat se puede usar para deshabilitar la separación de palabras en un documento o para habilitar la separación de palabras automática

\usepackage[fit]{truncate} %Truncar texto a un ancho dado

\usepackage{titling,lipsum} %titulos

\usepackage{thmtools} %herramientas para crear teoremas, def, etc.

\usepackage{fancyhdr} % encabezado y pie de paginas



% Comandos automatizados
