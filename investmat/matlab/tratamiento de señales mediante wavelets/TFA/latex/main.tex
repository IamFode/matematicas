\documentclass{beamer}
\usepackage{amsmath}
\usepackage{graphicx}
\usepackage{listings}
\usepackage{biblatex}

\addbibresource{bibliography.bib}

\title{Análisis de Precios del Petróleo en la India usando Descomposición Wavelet y Modelos ARIMA}
\author{Tu Nombre}
\date{\today}

\begin{document}

\frame{\titlepage}

\begin{frame}{Introducción}
\begin{itemize}
    \item Estudio del comportamiento de los precios del petróleo en la India.
    \item Uso de técnicas avanzadas de procesamiento de señales y modelos estadísticos.
    \item Objetivo: Mejorar la precisión de las predicciones de precios.
\end{itemize}
\end{frame}

\begin{frame}{Descomposición Wavelet}
\begin{itemize}
    \item Técnica para descomponer una señal en componentes de diferentes frecuencias.
    \item Usamos la wavelet simétrica 'sym4' para preservar la forma de la señal.
\end{itemize}
\begin{equation}
    f(t) = \sum_{k} c_{k} \phi_{k}(t) + \sum_{j=1}^{J} \sum_{k} d_{j,k} \psi_{j,k}(t)
\end{equation}
\begin{itemize}
    \item \(c_k\): Coeficientes de aproximación.
    \item \(d_{j,k}\): Coeficientes de detalle en el nivel \(j\).
    \item \(\phi_{k}(t)\): Funciones de escala.
    \item \(\psi_{j,k}(t)\): Funciones wavelet.
\end{itemize}
\end{frame}

\begin{frame}{Matemáticas de la Transformada Wavelet}
\begin{itemize}
    \item La Transformada Wavelet Discreta (DWT) descompone una señal en componentes de diferentes frecuencias.
    \item Proceso iterativo que divide la señal en componentes de aproximación y detalle.
\end{itemize}
\begin{equation}
    c_{a,j+1}[n] = \sum_k h[k-2n] c_{a,j}[k]
\end{equation}
\begin{equation}
    c_{d,j+1}[n] = \sum_k g[k-2n] c_{a,j}[k]
\end{equation}
\begin{itemize}
    \item \(c_{a,j}[k]\): Coeficientes de aproximación en el nivel \(j\).
    \item \(c_{d,j}[k]\): Coeficientes de detalle en el nivel \(j\).
    \item \(h[k]\): Coeficientes de filtro de escala.
    \item \(g[k]\): Coeficientes de filtro wavelet.
\end{itemize}
\end{frame}

\begin{frame}{Modelo ARIMA}
\begin{itemize}
    \item Modelo AutoRegresivo Integrado de Media Móvil (ARIMA).
    \item Utilizado para predecir series temporales con componentes estacionarios y no estacionarios.
\end{itemize}
\begin{equation}
    y_t = c + \phi_1 y_{t-1} + \phi_2 y_{t-2} + \cdots + \phi_p y_{t-p} + \epsilon_t + \theta_1 \epsilon_{t-1} + \theta_2 \epsilon_{t-2} + \cdots + \theta_q \epsilon_{t-q}
\end{equation}
\begin{itemize}
    \item \(c\): Constante.
    \item \(\phi_i\): Coeficientes autoregresivos.
    \item \(\theta_i\): Coeficientes de media móvil.
    \item \(\epsilon_t\): Término de error en el tiempo \(t\).
\end{itemize}
\end{frame}

\begin{frame}{Metodología}
\begin{enumerate}
    \item Cargar y filtrar los datos de precios del petróleo.
    \item Aplicar la descomposición wavelet para extraer componentes de baja y alta frecuencia.
    \item Realizar denoising de la señal usando técnicas wavelet.
    \item Ajustar modelos ARIMA a las series originales y denoised.
    \item Comparar la precisión de los modelos usando métricas de error.
\end{enumerate}
\end{frame}

\begin{frame}{Carga y Filtrado de Datos}
\begin{itemize}
    \item Datos obtenidos de la serie de precios del petróleo en la India.
    \item Filtrado para el intervalo de tiempo del 1 de enero de 2019 al 30 de julio de 2021.
\end{itemize}
\begin{lstlisting}[language=Matlab]
data = readtable('MCEI.csv');
prices = str2double(data.Price);
dates = datetime(data.Date, 'InputFormat', 'dd-MM-yyyy');
start_date = datetime('01-Jan-2019', 'InputFormat', 'dd-MMM-yyyy');
end_date = datetime('30-Jul-2021', 'InputFormat', 'dd-MMM-yyyy');
date_filter = (dates >= start_date) & (dates <= end_date);
price = prices(date_filter);
date = dates(date_filter);
\end{lstlisting}
\end{frame}

\begin{frame}{Descomposición Wavelet}
\begin{itemize}
    \item Número de niveles: 5.
    \item Uso de la wavelet 'sym4' para descomponer la serie de precios.
\end{itemize}
\begin{lstlisting}[language=Matlab]
niveles = 5;
[C,L] = wavedec(price, niveles, 'sym4');
\end{lstlisting}
\end{frame}

\begin{frame}{Visualización de la Descomposición}
\begin{itemize}
    \item Señal original y aproximaciones en cada nivel.
    \item Detalles en el último nivel.
\end{itemize}
\begin{lstlisting}[language=Matlab]
figure;
subplot(niveles+2, 1, 1);
plot(date, price);
title('Señal original');
xlabel('Fecha');
for i = 1:niveles
    subplot(niveles+2, 1, i+1);
    approx = wrcoef('a', C, L, 'sym4', i);
    plot(linspace(start_date, end_date, length(approx)), approx);
    title(['Aproximación en el nivel ' num2str(i)]);
    xlabel('Fecha');
end
subplot(niveles+2, 1, niveles+2);
details = wrcoef('d', C, L, 'sym4', niveles);
plot(linspace(start_date, end_date, length(details)), details);
title(['Detalles en el nivel ' num2str(niveles)]);
xlabel('Fecha');
\end{lstlisting}
\end{frame}

\begin{frame}{Denoising}
\begin{itemize}
    \item Aplicación de la técnica de denoising a la serie de precios.
\end{itemize}
\begin{lstlisting}[language=Matlab]
x = wdenoise(price);
figure;
hold on;
plot(date, price);
plot(date, x);
hold off;
legend('Original', 'Denoised');
title('Comparación de la serie de tiempo original y denoised');
xlabel('Fecha');
\end{lstlisting}
\end{frame}

\begin{frame}{Modelos ARIMA}
\begin{itemize}
    \item Ajuste de modelos ARIMA para la serie original y la serie denoised.
    \item Evaluación de los modelos usando RMSE y MAD.
\end{itemize}
\begin{lstlisting}[language=Matlab]
model_original = arima('Constant',0,'D',1,'Seasonality',0);
fit_original = estimate(model_original, price);
res_original = infer(fit_original, price);
fit_denoised = estimate(model_original, x);
res_denoised = infer(fit_denoised, x);
rmse_original = sqrt(mean(res_original.^2));
rmse_denoised = sqrt(mean(res_denoised.^2));
mad_original = mean(abs(res_original));
mad_denoised = mean(abs(res_denoised));
disp(['RMSE Original: ' num2str(rmse_original)]);
disp(['RMSE Denoised: ' num2str(rmse_denoised)]);
disp(['MAD Original: ' num2str(mad_original)]);
disp(['MAD Denoised: ' num2str(mad_denoised)]);
\end{lstlisting}
\end{frame}

\begin{frame}{Resultados}
\begin{itemize}
    \item \textbf{RMSE Original}: \texttt{\num{rmse_original}}
    \item \textbf{RMSE Denoised}: \texttt{\num{rmse_denoised}}
    \item \textbf{MAD Original}: \texttt{\num{mad_original}}
    \item \textbf{MAD Denoised}: \texttt{\num{mad_denoised}}
\end{itemize}
\end{frame}

\begin{frame}{Conclusiones}
\begin{itemize}
    \item La descomposición wavelet y el denoising mejoran la precisión de los modelos ARIMA.
    \item El modelo denoised presenta menores errores en comparación con el modelo original.
    \item Las técnicas de procesamiento de señales pueden ser efectivas para mejorar las predicciones de series temporales financieras.
\end{itemize}
\end{frame}

\begin{frame}{Bibliografía}
\printbibliography
\end{frame}

\end{document}

