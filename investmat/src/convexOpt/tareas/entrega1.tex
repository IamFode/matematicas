\center \textbf{CONVEXIDAD Y OPTIMIZACIÓN}
\center \textbf{\Large ENTREGA 1}
\center \textbf{ \textbf{Christian Limbert Paredes Aguilera}}

\line(1,0){400}


\begin{enumerate}[\bfseries \text{Ejercicio} 1.]

    \item \textbf{\boldmath Demuestra que la envoltura convexa de un conjunto $S\subset \mathbb{R}^n$ es la intersección de todos los conjuntos convexos de $\mathbb{R}^n$ que contienen a $S$.}

	\textbf{Demostración.-}\;

    \item \textbf{\boldmath La descripción de semiespacios de Voronoi: Sean $a$ y $b$ dos puntos distintos de $\mathbb{R}^n$. Demuestra que el conjunto de todos los puntos que están más cerca de $a$ (en la distancia Euclídea) que de $b$, i.e., $\left\{x\in \mathbb{R}^n : \|x-a\|_2 \leq \|x-b\|_2\right\}$, es un semiespacio. Descríbelo explícitamente como una desigualdad de la forma $c^T\cdot x \leq d$ (donde $c$ es un vector columna de $\mathbb{R}^n$). Haz una representación gráfica de la situación.}

	\textbf{Demostración.-}\;

    \item \textbf{\boldmath Demuestra que si $A$ y $B$ son conjuntos convexos de $\mathbb{R}^n$ entonces su intersección $A\cap B$ y su suma de Minkowsky $A+B$ son conjuntos convexos de $\mathbb{R}^n$.}

	\textbf{Demostración.-}\;
	
\end{enumerate}
