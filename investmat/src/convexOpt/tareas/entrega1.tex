\center \textbf{CONVEXIDAD Y OPTIMIZACIÓN}
\center \textbf{\Large ENTREGA 1}
\center \textbf{ \textbf{Christian Limbert Paredes Aguilera}}
\center Latex Source: \url{}

\line(1,0){400}



\begin{enumerate}[\bfseries \text{Ejercicio} 1.]

    % ------------------- EJERCICIOS 1 ------------------
    \item \textbf{\boldmath Demuestra que la envoltura convexa de un conjunto $S\subset \mathbb{R}^n$ es la intersección de todos los conjuntos convexos de $\mathbb{R}^n$ que contienen a $S$.}

	\textbf{Demostración.-}\; Sean $\co(S)$\footnote{
	Se llama \textbf{envoltura convexa} de $S$ al menor conjunto convexo que contienen a $S$, denotado por $\co(S)$.
	También es equivalente a decir que:
	$\co(S)=\left\{\mbox{Combinación convexa de puntos de} S\right\}.$\\
	\label{envoltura}}
	la envoltura convexa del conjunto $S$ e $I$ como la intersección de todos los conjuntos convexos que contienen a $S$. En si lo que vamos a querer demostrar es:
	$$\co(S)=I.$$
	En otras palabras, queremos demostrar que
	$$\co(S)\subseteq I \quad \land \quad I\subseteq \co(S).$$
	Notemos que $\co(S)$\footref{envoltura} es un conjunto convexo que contiene a $S$. Esto significa que cualquier otro conjunto convexo que contenga a $S$ debe ser al menos tan grande como $\co(S)$, y debe contener a $\co(S)$. Por lo que cualquier conjunto convexo que contenga a $S$ debe contener también a $\co(S)$. De esta manera, $I$ debe contener al menos a $\co(S)$. Es decir,
	$$\co(S)\subseteq I.$$
	Para demostrar la otra inclusión, supondremos que 
	$$\co(S)\not\subseteq I.$$
	De lo que notamos que existe al menos un punto $x\in C$ tal que $x\notin \co(S)$. Por definición de envoltura\footref{envoltura}, esta $x$ no puede ser escrito como una combinación convexa\footnote{
	Sean $x_1,x_2,\ldots,x_k\in \mathbb{R}^n$ y $\lambda_1,\lambda_2,\ldots,\lambda_k\in \mathbb{R}$ tales que 
	$\lambda_i\geq 0\; \text{y} \;\sum_{i=1}^{k}\lambda_i=1.$ 
	Al vector
	$\sum_{i=1}^{k}\lambda_ix_i=\lambda_1x_1+\lambda_2x_2+\ldots+\lambda_kx_k$
	se le llama \textbf{combinación convexa} de los puntos $\left\{x_1,\ldots,x_k\right\}$.\\\\
	} de puntos en $S$.

	Luego, $x$ esta en $I$ e $I$ es la intersección de todos los conjuntos convexos que contienen a $S$. Por lo que, $x$ debe estar en cada conjunto convexo que contiene a $S$. Y dado que $I$ es la intersección de todos los conjuntos convexos que contiene a $S$, se sigue que $I$ es un conjunto convexo
	\footnote{
	    \textbf{Demostrar que la intersección de conjuntos convexos es convexo.}\;
		Demostremos por contradicción. Sean $C_1$ y $C_2$ dos conjuntos convexos. Y sea 
		$C=C_1\cap C_2.$
		no convexo. Esto significa que existen $x$ e $y$ tales que 
		$\left\{\lambda x + (1-\lambda)y:\lambda\in \mathbb{R}\right\}\not\subseteq C.$ 
		Supongamos ahora que $x$ e $y$ están en $C$. Cómo ambos $C_1$ y $C_2$ son convexos, el segmento definido debe estar en ambos conjuntos. Es decir,
		$\left\{\lambda x + (1-\lambda)y:\lambda\in \mathbb{R}\right\}\subseteq C.$ 
		Lo que contradice nuestra suposición inicial. Por lo tanto, $C$ es convexo.\\\\
	\label{dos}} que contiene a $S$. 

	Por el hecho de que $x\in I$, $x$ debe ser una combinación convexa de puntos en $S$. Pero esto contradice el hecho de que $x$ no puede ser escrito como una combinación convexa de puntos en $S$. Así concluimos que $I\subseteq \co(S)$. Y por lo tanto,
	$$\co(S)=I.\;\blacksquare$$
	\vspace{.5cm}

    % ----------------------- EJERCICIO 2 -------------------------
    \item \textbf{\boldmath La descripción de semiespacios de Voronoi: Sean $a$ y $b$ dos puntos distintos de $\mathbb{R}^n$. Demuestra que el conjunto de todos los puntos que están más cerca de $a$ (en la distancia Euclídea) que de $b$, i.e., $\left\{x\in \mathbb{R}^n : \|x-a\|_2 \leq \|x-b\|_2\right\}$, es un semiespacio. Descríbelo explícitamente como una desigualdad de la forma $c^T\cdot x \leq d$ (donde $c$ es un vector columna de $\mathbb{R}^n$). Haz una representación gráfica de la situación.}

	\textbf{Demostración.-}\; Primero, representemos gráficamente la situación.
	\begin{center}
	    \begin{tikzpicture}[scale=.75,>=Triangle,rotate=40]
		\pattern[pattern=north east lines, pattern color=orange!50, rotate around={0:(0,0)}] (0,-2) rectangle (2cm,2cm)node[above left,rotate=40,orange]{$c^Tx<b$};
		\draw [fill=black] (0,0) circle (2pt) node[below] {$a$};
		\draw [fill=black] (4,0) circle (2pt) node[below] {$b$};
		\draw [very thick,blue] (2,-2)node[below, rotate=40]{$c^Tx=b$} -- (2,2);
		\draw [->, thick,green] (2,0) -- (3.5,0);
	    \end{tikzpicture}
	\end{center}

	Ahora, con algunas manipulaciones algebraicas llegaremos a la desigualdad
	$$c^T\cdot x \leq d.$$
	Que es la forma estándar de un semiespacio\footnote{
	    Todo \textbf{hiperplano} $H=\left\{x:a^Tx=0, a\neq 0\right\}$, define dos semiespacios 
	    $\left\{x\in\mathbb{R}^n:a^T x \leq 0\right\},\; \left\{x\in \mathbb{R}^n:a^T x\geq 0\right\}.$
	    Más generalmente, 
	    $\left\{x\in\mathbb{R}^n:a^T x \leq b\right\},\; \left\{x\in \mathbb{R}^n:a^T x\geq b\right\}.$
	    son las soluciones de dos sistemas lineales de desigualdades.\\

	}.

	(Por convención diremos que los vectores $x$, $a$ y $b$ son vectores columna en $\mathbb{R}^n$). Dado que tenemos la desigualdad en término de normas (términos no negativos). Podemos elevar al cuadrado sin cambiar el orden de los elementos. Es decir, 
	$$\|x-a\|_2 \leq \|x-b\|_2\quad \Rightarrow \quad \|x-a\|_2^2 \leq \|x-b\|_2^2.$$
	Luego, por la definición de norma Euclídea\footnote{
	    \textbf{\boldmath$\|x\|_2^2=x^Tx.$}
	}
	y la propiedad conmutativa de producto interno, tenemos 
	$$
	\begin{array}{rcl}
	    (x-a)^T(x-a) &\leq&(x-b)^T(x-b)\\\\
	    &\Downarrow&\\\\
	    x^Tx-2a^Tx+a^Ta &\leq& x^Tx-2b^Tx+b^Tb.
	\end{array}
	$$
	Después, por la sustracción de vectores podemos simplificar la desigualdad como:
	$$2b^Tx-2a^Tx\leq b^Tb-a^Ta.$$
	Que es equivalente a,
	$$(b-a)^Tx\leq \dfrac{1}{2}\left(b^Tb-a^Ta\right).$$
	Podemos definir $c=b-a$ y $d=\dfrac{1}{2}\left(b^Tb-a^Ta\right)$, para obtener la desigualdad
	$$c^Tx\leq d.$$
	Por lo tanto, el conjunto de todos los puntos que están más cerca de $a$ que de $b$ en la distancia euclídea es un semiespacio.$\;\blacksquare$
	\vspace{1cm}

    % -------------------------- EJERCICIO 3 -------------------------
    \item \textbf{\boldmath Demuestra que si $A$ y $B$ son conjuntos convexos de $\mathbb{R}^n$ entonces su intersección $A\cap B$ y su suma de Minkowsky $A+B$ son conjuntos convexos de $\mathbb{R}^n$.}

	\textbf{Demostración.-}\; Primero, demostraremos que la intersección $A\cap B$ es convexa. Sabemos por hipótesis que $A$ y $B$ son conjuntos convexos en $\mathbb{R}^n$. Tomemos ahora, dos puntos cualesquiera $x,y\in A\cap B$. Dado que $x,y\in A$. Entonces, para todo $\lambda\in [0,1]$, se tiene:
	$$\lambda x+(1-\lambda)y\in A.\footnote{
	    Un conjunto $C$ en un espacio vectorial es \textbf{convexo}, si para cada par de puntos $x,y\in C$ y para cada número real $\lambda$ en el intervalo $[0,1]$, se cumple que:
	    $\lambda x+(1-\lambda)y\in C.$\\
	\label{convex}}$$
	De manera similar, dado que $x,y\in B$. Entonces, por definición de convexidad para todo $\lambda \in [0,1]$, también tenemos:
	$$\lambda x+(1-\lambda)y\in B \footref{convex}.$$
	Por lo tanto, 
	$$\lambda x+(1-\lambda)y\in A\cap B,$$
	lo que implica que la intersección de conjuntos convexos es convexa. Esto también se puede demostrar con (\footref{dos}).

	Ahora, demostraremos que si $A$ y $B$ son conjuntos convexos. Entonces, $A+B$ es convexo. Sean dos puntos cualesquiera $\alpha,\beta\in A+B$, por la suma de Minkowsky,\footnote{
	    La \textbf{suma de Minkowski} es la operación de conjuntos; es decir, si $A,E\subseteq \mathbb{R}^n$. Entonces,
	    $A=x_0+E=\left\{x_0+e:e\in E \right\} \quad \mbox{o}\quad E=A-x_0=\left\{a-x_0:a\in A\right\}.$
	},   
	existen $a_1,a_2\in A$ y $b_1,b_2\in B$ tales que 
	$$\alpha=a_1+a_1 \quad \mbox{y}\quad \beta=a_2+b_2.$$
	La idea es mostrar que para cualquier $\lambda\in[0,1]$, 
	$$\lambda\alpha+(1-\lambda)\beta\in A+B.$$
	Tenemos que
	$$
	\begin{array}{rcl}
	    \lambda\alpha+(1-\lambda)\beta&=&\lambda(a_1+b_1)+(1-\lambda)(a_2+b_2)\\\\
					  &=&\left[\lambda a_1+(1-\lambda)a_2\right]+\left[\lambda b_1+(1-\lambda)b_2\right].
	\end{array}
	$$
	Como $A$ y $B$ son conjuntos convexos\footref{convex}, sabemos que 
	$$\lambda a_1+(1-\lambda)a_2\in A$$
	$$\text{y}$$
	$$\lambda b_1+(1-\lambda)b_2\in B.$$
	Por lo tanto,
	$$\lambda\alpha+(1-\lambda)\beta\in A+B.$$
	Así, $A+B$ es convexo, como se quería demostrar. Esto completa la demostración.$\;\blacksquare$
	
\end{enumerate}
