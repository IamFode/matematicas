\chapter{¿Cómo escribir artículos científicos?}

Se puede clasificar la investigación matemática en:
\begin{itemize}
    \item Matemática básica.
    \item Matemática aplicada.
\end{itemize}

\begin{itemize}
    \item Se debería investigar en matemática aplicada y matemática aplicable.
    \item La investigación debe estar financiada ya sea del sector público o privado.
    \item El acceso a la bibliografía debe ser también financiada por el estado.
    \item Debe ser publicada en Ingles.
    \item La publicación en congresos es muy útil para compartir opiniones y será enriquecedor.
    \item 
\end{itemize}

Los artículos científicos suelen ser de los siguientes tipos:

\begin{itemize}
    \item Artículos originales de investigación.
    \item Publicaciones cortas.
    \item Artículos de revisión (Overview).
    \item Editoriales.
\end{itemize}

\section{Congresos científicos}

\begin{itemize}
    \item Nacionales e internacionales.
    \item Se suelen presentar resultados parciales de la investigación.
    \item Las ponencias son seleccionadas por el comité científico.
\end{itemize}

\section{Revistas científicas}

Las revistas está indexadas en:
\begin{itemize}
	\item Google Scholar.
	\item Scopus.
	\item Science Citation index. (JCR) 
\end{itemize}

En general las revistas son propiedades de editoriales
\begin{itemize}
    \item Elsevier.
    \item Springer.
    \item IEEE.
    \item MDPI.
    \item Hindawi.
    \item Taylor.
    \item Wiley.
\end{itemize}

\section{Proceso de revisión}
\begin{itemize}
    \item Investigación.
    \item Decido a que revista enviar.
    \item Escribir con los parámetros de la revista.
    \item Revisión de la revista.
    \item Empezar el ciclo hasta que sea aceptado.
\end{itemize}

Debemos tener cuidado en que revista publicar, porque algunos no tienen el Copyright Pruebas DOI.

\section{Elsevier}

Pasos para publicar:

\begin{enumerate}[1.]
    \item Escoger una revista.
    \item Preparar el artículo.
    \item Enviar y revisar.
    \item Seguimiento de su envío.
    \item Realice un seguimiento de su artículo aceptado.
    \item Compartir y promover.
\end{enumerate}

\url{https://www.elsevier.com/authors/submit-your-paper}


\section{Estructura del artículo científico}

\begin{itemize}
    \item Escribir en ingles, con frases cortas y conciso.
    \item No escribir frases compuestas.
\end{itemize}

Los apartados que debe tener son:
\begin{enumerate}[1.]
    \item Introducción -> ¿Cuál es el problema?.
    \item Material y métodos -> ¿Cómo resolvemos el problema?.
    \item Resultados -> ¿Qué hemos encontrado?.
    \item Discusión -> ¿Valor, interpretación, significado?.
    \item Referencias.
\end{enumerate}

De forma más concreta:
\begin{enumerate}[1.]
    \item Título y autores.
    \item Abstract -> Concisos y convincente.
    \item Introducción.
    \item Métodos y herramientas.
    \item Resultados.
    \item Discusión.
    \item Agradecimientos -> cómo un justificante.
    \item Referencias.
\end{enumerate}

\subsection{Título}
\begin{itemize}
    \item Conciso.
    \item Sin requerimientos gramaticales complejos.
    \item Evitar abreviaturas.
    \item ¿De que hablamos?
	\begin{itemize}
	    \item Tema del trabajo.
	    \item Objetivos principales.
	    \item Destacar la controversia.
	    \item Destacar las conclusiones.
	\end{itemize}
\end{itemize}

\subsection{Autoria}
\begin{itemize}
    \item Se nos exige un Orcid e ID.
    \item Intentemos firmar siempre igual. Christian L. Paredes Aguilera.
\end{itemize}

\subsection{Introducción}
\begin{itemize}
    \item ¿Cuál es el problema?
    \item ¿Por qué es importante?
    \item ¿Qué se ha hecho hasta ahora?
\end{itemize}

\subsection{Los métodos y materiales}
\begin{itemize}
    \item 
\end{itemize}
