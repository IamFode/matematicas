\documentclass{beamer}
 
\usetheme{Madrid}	% Temas
\usecolortheme{beaver}	% Color
%\setbeamercolor{block title}{bg=red!20!yellow,fg=red} % bg fondo, fg letra
\usefonttheme{default}	% Fuente

\setbeamertemplate{enumerate item}
{\color{red!80!white}
\fbox{\theenumi${}^{\underline{\circ}}$}}

\usepackage[utf8]{inputenc}
\usepackage[spanish]{babel}


\title[Beamer]{Presentaciones con Beamer}
\author[C. Paredes]{Christian Limbert Paredes Aguilera}
\institute[U.P.V.]
    {Universidad Politécnica de Valencia}
\date{}
    
\begin{document}
    
    % Mostrar título
    \begin{frame}
	\titlepage
    \end{frame}

    % Mostrar cajas o boxes
    \begin{frame}
	\begin{block}{Bloque}
	    Un {\color{red} bloque} es un entorno que permite resaltar información.
	\end{block}
	\rule{\textwidth}{0.4pt}
	\begin{block}{Definición}
	    Un {\color{orange} triángulo} tiene tres ángulos
	\end{block}
	\begin{alertblock}{Alerta}
	    Un {\color{blue} cuadrado} tiene cuatro lados
	\end{alertblock}
    \end{frame}

    \begin{frame}
	\begin{theorem}[Christian]
	    Para cada $a,b>0$, uno tiene $\sqrt{ab}\leq (a+b)/2$.
	\end{theorem}

	\begin{lemma}
	    Sea $a,b>0$. Entonces ...
	\end{lemma}

	\begin{corollary}[Euler]
	    $e^{j\pi}=-1$.
	\end{corollary}

	\pause

	\begin{exampleblock}{Ejemplo}
	    Sea $a,b>0$. Entonces ... $\Box$
	\end{exampleblock}
    \end{frame}

    % Mostrar una columna a la vez
    \begin{frame}
	\begin{columns}
	    \column{.6\textwidth}
		\begin{block}{Pregunta}
		    ¿Cuál es la solución de $\log x = 1$?
		\end{block}
		\pause %Destamar texto poco a poco
		\column{.35\textwidth}
		\begin{block}{Respuesta}
		    $x=e$. \phantom{¿ ?} %Forzar el espacio
		\end{block}
	\end{columns}
    \end{frame}

    % Mostrar poco a poco con opacidad
    \begin{frame}
	\setbeamercovered{transparent}
	    \begin{itemize}
		\item En un lugar de ... \pause
		\item la Mancha, de cuyo nombre ... \pause
		\item no quiero acordarme.
	    \end{itemize}
	\setbeamercovered{invisible}
    \end{frame}

    % Mostrar cambiando partes de texto
    \begin{frame}
	Vamos a resolver la ecuación $x^2=x$ paso a paso
	\only<1>{$$x^2={\color{green} x}$$ Pasamos esta $x$
	al otro lado}
	\only<2>{$$x(x-1)=0$$}
	\only<3>{$$x=0 \quad \text{o} \quad x=1$$}
	\only<4>{$$\begin{array}{rcl}
			x^2&=&x\\
			x^2-x&=&0\\
			x(x-1)&=&0\\
			x=0&\text{o}&x=1
		    \end{array}$$}
    \end{frame}

    % Alertas poco a poco
    \begin{frame}
	\begin{itemize}
	    \item \alert<1>{Pit\'agoras}
	    \item \alert<2>{Euler}
	    \item \alert<3>{Gau\ss}
	\end{itemize}
    \end{frame}

    % Mostrar texto poco a poco
    \begin{frame}
	\begin{itemize}
	    \item <1->{Pit\'agoras}
	    \item <2->{Euler}
	    \item <2->{Gau\ss}
	    \item <3->{Hamilton}
	\end{itemize}
    \end{frame}

    \begin{frame}
	\begin{itemize}[<+->]
	    \item Cada transparencia debe ser autocontenida.
	    \item 
	\end{itemize}
    \end{frame}

\end{document}
