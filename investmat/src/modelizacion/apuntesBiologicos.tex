\chapter{Modelos matemáticos para el tratamiento térmico de tejidos biológicos}

\section{Tratamiento térmico de tejidos biológicos}
Consiste en una terapia ablativa. Es decir, que destruye un tejido, gracias a ampliar una energía que hace que haya un calentamiento y se altere este. Es un fin terapéutico. Por ejemplo, la ligadura de trompas, cuando sellamos y ligamos el tejido con calor, para que las trompas se liguen y sea un método conceptivo. El tratamiento será a más de $80\%$ grados.

\section{Ventajas de los modelos matemáticos}
\begin{itemize}
    \item Agiliza los experimentos.
    \item Reduce los costes.
    \item Se puede hacer pruebas.
\end{itemize}

\section{El artículo científico en ingeniería biomédica}
\begin{enumerate}
    \item Introducción.- Interés, hipótesis, objetivo. Estado del arte (Que se haya hecho hasta ahora).
    \item Materiales y métodos.- Formulación del modelo, mecanismos de validación, programas utilizados (Cómo lo hice). Debemos ser específicos para que cualquier pueda replicarlo.
    \item Resultados. 
    \item Discusión.
    \item Conclusiones.
\end{enumerate}

\section{Planteamiento general general de un modelo}
\begin{enumerate}
    \item Interés, hipótesis y objetivo. Interlocución Interdisciplinar.
	\begin{itemize}
	    \item La hipótesis es lo que uno cree que va a pasar.
	    \item El objetivo es lo que se va a hacer. (Realizar un modelo). La acción. 
	    \item Interés es el por qué se hace.
	\end{itemize}
    \item Geometría y abstracción.
    \item Ecuaciones de gobierno.
    \item Condiciones iniciales y de contorno.
    \item Características de los Materiales.
    \item Resolución computacional.
    \item Postprocesado.
    \item Discusión. Comparación con resultados experimentales.
	\begin{itemize}
	    \item Volver al principio y ver si se cumplieron el objetivo y la hipótesis.
	\end{itemize}
    \item Conclusión.
\end{enumerate}

\subsection{Interés, hipótesis y objetivo}
Revisemos el siguiente artículo: \href{https://doi.org/10.3109/02656736.2011.631076}{Relationship between roll-off occurrence and spatial distributionof dehydrated tissue during RF ablation with cooled electrodes}

\begin{itemize}
	\item Interés: Aunque muchos estudios han analizado el efecto térmico de un electrodo refrigerado y la relación exacta entre la distribución del tejido deshidratado y la ocurrencia del "roll-off", todavía no se caracterizo en detalle. Osea, nuestro interés es caracterizarlo en detalle.
	\item Hipótesis: El episodio de "roll-off" coincide aproximadamente con el momento en que el tejido alrededor del centro del electrodo activo se deseca en gran medida (alrededor de 100 °C), es decir, cuando todo el electrodo está completamente rodeado por tejido deshidratado.
	\item Objetivo: Demostrar esta hipótesis mediante un enfoque doble, teórico y experimental.
\end{itemize}

\subsection{Geometría y abstracción}
Dibujos abstractos del tema a estudiar.

\subsection{Ecuaciones de gobierno}
Tenemos que traducir a ecuaciones matemáticas y ver el dominio de donde se va a trabajar. En este caso será un cilindro. Normalmente se heredan modelos de otros trabajos.

Existen dos problemas:
\begin{itemize}
    \item Uno eléctrico.
    \item Uno térmico.
\end{itemize}

En el caso del microondas, 
\begin{itemize}
	\item Uno electromagnético.
	\item Uno térmico.
\end{itemize}

Y en el caso del laser, 
\begin{itemize}
	\item Uno óptico.
	\item Uno térmico.
\end{itemize}

Veamos como se formularía:

Comencemos con la ecuación de calor:
$$\rho c\dfrac{\partial T}{\partial t}=\triangledown(k\triangledown T).$$
Donde:
\begin{center}
    \begin{tabular}{rcl}
	$\rho$ &:& Densidad.\\ 
	$c$ &:& Calor específico.\\
	$T$ &:& Temperatura.\\
	$t$ &:& Tiempo.\\
	$k$ &:& Conductividad térmica.\\
	$\triangledown$ &:& Operador nabla.
    \end{tabular}
\end{center}

A esta ecuación debemos agregar las siguientes variables

$$\rho c\dfrac{\partial T}{\partial t}=\triangledown(k\triangledown T)+Q_b+Q_m+Q_F.$$

Donde:
\begin{center}
    \begin{tabular}{rcl}
	$Q_b$ &:& Calor proveniente de la sangre, llamado calor por perfusión.\\ 
	$Q_m$ &:& Calor metabólico. (Es pequeña por lo que es despreciable, ya que el tiempo de exposición 12 min).\\
	$Q_F$ &:& Calor de la fuente de energía. (Problema óptico o electromagnético).
    \end{tabular}
\end{center}

Veamos el calor por perfusión ($Q_b$).
$$Q_b=\beta \rho_b c_b \omega_b(T_b-T).$$

Donde:
\begin{center}
	\begin{tabular}{rcl}
	$\rho_b$ &:& Densidad de la sangre.\\
	$c_b$ &:& Calor específico de la sangre.\\
	$\omega_b$ &:& Coeficiente de refrigeración de la sangre.\\
	$T_b$ &:& Temperatura de la sangre. $37^{\circ}C$.\\
	$\beta$ &:& Coeficiente de intercambio de calor.\\ 
	\end{tabular}
\end{center}

$$
\beta=\left\{
\begin{array}{ll}
    1 & \Omega(t)<4.6\\
    0 & \Omega(t)\geq 4.6
\end{array}
\right.
\quad \Rightarrow \quad
\Omega(t)=\int_{0}^{t_{final}} Ae^{-\frac{\delta E}{RT(x,t)}}\;dt
$$


Donde:
\begin{center}
    \begin{tabular}{rcl}
	$\Omega$ &:& Daño térmico, en función de la temperatura y el tiempo.\\
	$A$ &:& Factor de frecuencia.\\
	$\triangle E$ &:& Energía de activación.\\
	$R$ &:& Constante de los gases.\\
	$T$ &:& Temperatura.
    \end{tabular}
\end{center}

\begin{itemize}
    \item Los valores de $A$ y $\triangle E$ se obtienen experimentalmente, no es lo mismo los daños en distintos órganos. Se considere $4.6$ es el $99\%$ del tejido muerto.
    \item Manejar una integral tan grande produce errores.
\end{itemize}

$T_b$ que es la Temperatura de la sangre se considera un refrigerante debido a que $T\geq T_b$. Es decir, la Temperatura del objeto a tratar será mayor al de la sangre en si.

\subsubsection{Problema térmico: Bioheat equation}
Ya dijimos que hay vaporización. Por lo que utilizamos un método llamado Entalpía. Lo que hace es que sustituye $pc\dfrac{\partial T}{\partial t}$ de la ecuación que estamos tratando por:
$$
\dfrac{\partial T}{\partial t} \cdot \left\{
    \begin{array}{ll}
	\partial_l c_l & 0<T\leq 99^{\circ}C\\
	h_{fg}C & 99<T\leq 100^{\circ}C\\
	\rho_g c_g & T>100^{\circ}C.
    \end{array}
\right.
$$

Donde:
\begin{center}
	\begin{tabular}{rcl}
	    $l$ & : & Líquido.\\
	    $C$ & : & Contenido de agua del tejido.\\
	    $\partial_l c_l$ &:& Calor latente liquido. (Se produce cambio de fase).\\
	    $h_{fg}C$ &:& Es el producto del  calor latente de vaporización del agua por la densidad del agua a $99^{\circ}$.\\
	    $\rho_g c_g$ &:& Calor específico del vapor.
	\end{tabular}
\end{center}

Considera al parcial de la temperatura con respecto del tiempo.

\subsection{Condiciones iniciales y de contorno}

Veamos las condiciones:
\begin{itemize}
    \item La temperatura es igual a la temperatura ambiente $T=T_a$.
    \item El flujo es nulo. $\vec{n}\cdot(k\triangledown T)=0$. 
    \item El proceso de calentar y enfriar algo, se llama convección natural. $\vec{n}(k\triangledown T)=h(T_r)-T$.
	Donde:
	\begin{center}
	    \begin{tabular}{rcl}
		$h$ &:& Coeficiente de refrigeración. (Acá se nota si es natural o forzada).\\
		$T_r$ &:& Temperatura de refrigeración.
	    \end{tabular}
	\end{center}
\end{itemize}

GRAN PARTE DEL TRABAJO ES AVERIGUAR CADA VARIABLE. 


\subsection{Características de los Materiales}
Los sacamos de artículos o bien de bases de datos.



