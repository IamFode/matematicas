%------------------------- Definición de nuevas secciones de teoremas definición etc
\declaretheoremstyle[%
    spaceabove=0pt,spacebelow=5pt,%
    headfont=\small\bfseries,% 
    notefont=\bfseries,%
    notebraces={}{. },%
    headpunct={},%
    preheadhook={\hspace{0mm}\newline},%
    postheadhook={},%
    qed=$\blacksquare$,
    postheadspace=0pt,%
    headindent=0pt,%
    bodyfont=\normalfont,%
    headformat={\llap{\smash{\parbox[t]{1.3in}{\raggedleft \NAME \;\;\; \\ \NUMBER\;\;\;}}}\NOTE}%
]{marginheads}
\makeatletter
\renewcommand\thmt@space{}
\makeatother
\declaretheorem[style=marginheads, numberwithin=chapter, title=Teorema]{teo}
\declaretheorem[style=marginheads, numberwithin=chapter, title=Ejemplo]{ejem}
\declaretheorem[style=marginheads, numberwithin=chapter, title=Postulado]{post}
\declaretheorem[style=marginheads, numberwithin=chapter, title=Corolario]{cor}
\declaretheorem[style=marginheads, numberwithin=chapter, title=Ejercicio]{ej}
\declaretheorem[style=marginheads, numberwithin=chapter, title=Lema]{lema}
\declaretheorem[style=marginheads, numberwithin=chapter, title=Problema]{prob}
\declaretheorem[style=marginheads, numberwithin=chapter, title=Ejercicio]{ejer}


\declaretheoremstyle[%
    spaceabove=0pt,spacebelow=5pt,%
    headfont=\small\bfseries,% 
    notefont=\bfseries,%
    notebraces={}{. },%
    headpunct={},%
    preheadhook={\hspace{0mm}\newline},%
    postheadhook={},%
    shaded = {
	%textwidth =450pt,
	bgcolor = black!2,
	rulecolor=black!3,
	rulewidth=1pt,
	margin=3pt
    },
    postheadspace=0pt,%
    headindent=0pt,%
    bodyfont=\normalfont,%
    headformat={\llap{\smash{\parbox[t]{1.3in}{\raggedleft \NAME \;\;\; \\ \NUMBER\;\;\;}}}\NOTE}%
]{margindef}
%hack to kill some extra space
\makeatletter
\renewcommand\thmt@space{}
\makeatother
\declaretheorem[style=margindef, numberwithin=part, title=Axioma]{axioma}
\declaretheorem[style=margindef, numberwithin=chapter, title=Definición]{def.}
\declaretheorem[style=margindef, numberwithin=chapter, title=Observación]{obs}
\declaretheorem[style=margindef, numberwithin=part, title=Propiedad]{prop}
\declaretheorem[style=margindef, numberwithin=chapter, title=Notación]{notacion}
%-------------------------------------------------------------------------------------

\declaretheorem[numberwithin=chapter, title=Nota]{nota}

%-------------------------------------------------------------------------------------

\makeatletter\renewcommand\theenumi{\@roman\c@enumi}\makeatother

%------------------------- Definición de nuevos comandos treigonométricos
\renewcommand\labelenumi{\theenumi)}
\def\sen{\mathop{\mbox{\normalfont sen}}\nolimits}
\def\cotan{\mathop{\mbox{\normalfont cotan}}\nolimits}
\def\cosec{\mathop{\mbox{\normalfont cosec}}\nolimits}
\def\arcsen{\mathop{\mbox{\normalfont arcsen}}\nolimits}
\def\arctan{\mathop{\mbox{\normalfont arctan}}\nolimits}
\def\interior{\mathop{\mbox{\normalfont int}}\nolimits}
\def\relint{\mathop{\mbox{\normalfont relint}}\nolimits}
\def\aff{\mathop{\mbox{\normalfont Aff}}\nolimits}
%------------------------------------------------------------------------

%---------------------------------
\titleformat*{\section}{\bfseries}
\titleformat*{\subsection}{\bfseries}
\titleformat*{\subsubsection}{\bfseries}
\titleformat*{\paragraph}{\bfseries}
\titleformat*{\subparagraph}{\bfseries}

