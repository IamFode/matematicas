
\documentclass[10pt]{article} 						
\usepackage[text=17cm,left=2.5cm,right=2.5cm, headsep=20pt, top=2.5cm, bottom = 2cm,letterpaper,showframe = false]{geometry} %configuración página
\usepackage{latexsym,amsmath,amssymb,amsfonts} %(símbolos de la AMS).7
\parindent = 0cm  %sangria
\usepackage[T1]{fontenc} %acentos en español
\usepackage[spanish]{babel} %español capitulos y secciones
\usepackage{graphicx} %gráficos y figuras.
%-----------------------------------------%
\usepackage{multicol}
\usepackage{titlesec}
\usepackage[rflt]{floatflt}
\usepackage{wrapfig} 
\usepackage{tikz}\usetikzlibrary{shapes.misc}
\usepackage{tikz,tkz-tab}						
\usetikzlibrary{matrix,arrows, positioning,shadows,shadings,backgrounds,
calc, shapes, tikzmark}
\usepackage{tcolorbox, empheq}
\tcbuselibrary{skins,breakable,listings,theorems}
\usepackage{xparse}							
\usepackage{pstricks}							
\usepackage[Bjornstrup]{fncychap}			
\usepackage{rotating}
\usepackage{enumerate}
\usepackage{booktabs}
\usepackage{synttree} 
\usepackage{chngcntr}
\usepackage{venndiagram}
\usepackage[all]{xy}
\usepackage{xcolor}
\usepackage{tikz}
\usetikzlibrary{datavisualization.formats.functions}
\usepackage{marginnote}										
\usepackage{fancyhdr}

%------------------------------------------
\renewcommand{\labelenumi}{\Roman{enumi}.}		%primer piso II) enumerate
\renewcommand{\labelenumii}{\arabic{enumii}$)$ }%segundo piso 2)
\renewcommand{\labelenumiii}{\alph{enumiii}$)$ }%tercer piso a)
\renewcommand{\labelenumiv}{$\bullet$}			%cuarto piso (punto)


\pagestyle{fancy}


% CARATULA CONVEXIDAD Y OPTIMIZACIÓN
%\begin{tabular}{r l }
Universidad: & \textbf{Mayor de San Ándres.}\\
Asignatura: & \textbf{Álgebra Lineal I}\\
Ejercicio: & \textbf{Prueba de Diagnostico.}\\ 
Alumno: & \textbf{PAREDES AGUILERA CHRISTIAN LIMBERT.}
\end{tabular}
\begin{flushleft}
\begin{tikzpicture}
\draw(0,1)--(16.5,1);
\end{tikzpicture}
\end{flushleft}


\begin{document}
%\maketitle
\pagestyle{fancy}
\fancyhead[LE,RO]{\nouppercase{\truncate{0.5\headwidth}{\rightmark}}}
\fancyhead[LO,RE]{\nouppercase{\truncate{0.5\headwidth}{\leftmark}}}

%------------------------------------------------------------------------------------------------
% -------------------- CONVEXIDAD Y OPTIMIZACIÓN ------------------------------------------------
%------------------------------------------------------------------------------------------------
\let\cleardoublepage\clearpage
% APUNTES
%\chapter{Convexidad y Optimización}

\section{Introducción}

$$
(P)
\left\{
\begin{array}{rl}
    \min & f_0(x)\\\\
    s.a. & f_1(x) \leq b_1\\
	 &f_2(x) \leq b_2\\
	 & \vdots\\
	 & f_m(x) \leq b_m.
\end{array}
\right.
\qquad
\begin{array}{rl}
    f_i: & \mathbb{R}^n \rightarrow \mathbb{R}\\
    f_0 : & \mbox{Función objetivo.}\\
    f_j : & \mbox{Función Restricción donde }j=1,\ldots,m.\\
\end{array}
$$
\begin{itemize}
    \item Las funciones objetivos en economía se les puede llamar función de coste.
    \item Las desigualdades tiene un truco, si multiplicamos por $(-1)$ tenemos en la forma que decidamos.
    \item Maximizar es lo mismo que minimizar. Por lo que minimizaremos las funciones. 
\end{itemize}

El objetivo de (P) es encontrar $x^*$ el optimo ($\arg\min$) que cumple 
$$f_0(x^*)\leq f_0(x), \;\forall x\in \mathbb{R}^n / f_j(x)\leq b_j,\; j=1,\ldots,m.$$
Será en cualquier $x$ que cumple las restricciones. Los puntos que no cumplen las condiciones no sirven para nada.

Al valor $f_0(x^*)$ se le llama valor optimo.

$f_i:  \mathbb{R}^n \rightarrow \mathbb{R}$ Existirá algunas funciones que su dominio sera tramposo.

Los Puntos factibles son los $x\in \mathbb{R}^n / f_j(x)\leq b_j,\; j=1,\ldots,m.$

\begin{itemize}
    \item  Si los problemas son lineales se llama programación lineal.
    \item Cuando es convexa se llama optimización convexa.
    \item La habilidad es de identificar las restricciones y pasarlas a convexas.
\end{itemize}

\begin{ejem}
Sean $A\in \mathcal{M}_{k\times n},\; \textbf{x}\in \mathbb{R}^n,\; \textbf{b}\in \mathcal{R}^k$.
$$
\begin{pmatrix}
    x_1\\
    x_2\\
    \vdots\\
    x_n
\end{pmatrix}
\in \mathbb{R}^n,
\qquad 
\textbf{x}^T=(x_1,x_2,\ldots,x_n).
$$
Diremos que el un vector cualquiera sera vector columna.

Ahora, el problema será una minimización global dada por:
$$
\left\{
\begin{array}{rl}
    \min: &\|A\textbf{x}-b\|^2_2\\
    s.a. & \emptyset.
\end{array}
\right.
$$

El subindice $_2$ significa la normal Euclidea. Que es la distancia normal que existe en $\mathbb{R}^2.$

El objetivo será encontrar la $x$ donde la operación dada será la menor posible.

\textbf{Nota}
Imaginemos que tenemos 
$$
\begin{array}{rl}
    \left\{\min\right. & f(x)\\\\
    \left\{ \min \right. & f_0^2(x)
\end{array}
$$

Si las función $f_0$ es positiva las dos formas son equivalentes. El valor optimo no será el mismo porque lo estoy elevando al cuadrado, pero el punto optimo lo será. Porque las funciones son monótomas crecientes. Si el valor al cuadrado me simplifica entonces podemos utilizarla. Esto nos permite que si no tengamos una función convexa podamos convexificarla.

Por diferenciabilidad:
$$f_0(x)=\|Ax-b\|_2^2 = \langle Ax-b,Ax-b\rangle.$$


\textbf{Notación.-} Podemos escribir $Ax$ como
$$
Ax = 
\underset{A^1}{
\begin{pmatrix}
    a_{11}\\
    a_{21}\\
    \vdots\\
    a_{k1}
\end{pmatrix}}
x_1+
\underset{A^2}{
\begin{pmatrix}
	a_{12}\\
	a_{22}\\
	\vdots\\
	a_{k2}
\end{pmatrix}}
x_2+
\cdots +
\underset{A^n}{
\begin{pmatrix}
	a_{1n}\\
	a_{2n}\\
	\vdots\\
	a_{kn}
\end{pmatrix}}
x_n
=
x_1A^1+x_2A^2+\cdots+x_nA^n.
$$
$A^1$ = A super 1 como columna, y $A_1$ = A super 1 como fila.

Ahora, en términos de filas. Si escribimos los vectores A en columna
$$
A = 
\begin{pmatrix}
	A^T_1\\
	A^T_2\\
	\vdots\\
	A^T_k
\end{pmatrix}
$$

Donde,
$$
Ax = 
\begin{pmatrix}
	A^T_1x\\
	A^T_2x\\
	\vdots\\
	A^T_nx
\end{pmatrix}
=
\begin{pmatrix}
	\langle A^T_1,x\rangle\\
	\langle A^T_2,x\rangle\\
	\vdots\\
	\langle A^T_n,x\rangle
\end{pmatrix}
$$

Intentaremos demostrar el punto donde las parciales de $f_0=0$. Para ello, encontraremos 
$$
\begin{array}{rcl}
    D_if_0&=&D_i\left(\langle Ax-b, Ax-b\rangle\right)\\\\
	  &=&\langle D_i\left(Ax-b\right),Ax-b\rangle+\langle Ax-b,D_i\left(Ax-b\right)\rangle\\\\
	  &=& 2\langle Ax-b,D_i\left(Ax-b\right)\rangle.
\end{array}
$$

Veamos la parcial de $D_i\left(Ax-b\right)$.
$$
\begin{array}{rcl}
    D_i\left(Ax-b\right)&=&D_i\left(x_1A^1+x_2A^2+\cdots+x_nA^n-b\right)\\\\
			&=&A^i.
\end{array}
$$
Dado que $b$ que es constante vale cero, y donde todos los suman que no estén las $x_i$ también valen cero.

Por lo tanto,
$$D_if_0 = 2\langle Ax-b,A^i \rangle.$$

Luego,
$$2\langle Ax-b,A^i \rangle = 0 \quad \forall i=1,\ldots,n \quad \Rightarrow \quad \langle Ax-b,A^i\rangle=0,\quad \forall i = 1,\ldots,n.$$

Observemos que,
$$
\overrightarrow{0} = 
\begin{pmatrix}
    \langle Ax-b,A^1\rangle\\
    \langle Ax-b,A^2\rangle\\
    \vdots\\
    \langle Ax-b,A^n\rangle
\end{pmatrix}
=
\begin{pmatrix}
    (A_1)^T\\
    (A_2)^T\\
    \vdots\\
    (A_n)^T
\end{pmatrix}
(Ax-b)
=A^T(Ax-b).
$$

En funciones convexas el extremo local será el mínimo global.
$$A^T(Ax-b)=\overrightarrow{0}\quad \Leftrightarrow\quad A^TAx=A^Tb.$$
Que es una ecuación normal.\\

\textbf{Argumentos geométricos}

Sera el mismo calcular el mínimo de la distancia, calcular:
$$\min \|Ax-b\|^2_2 = d(b_i,Ax)^2$$
Donde $Ax$ tendrá la forma geométrica, de un subespacio vectorial. en el caso de $\mathbb{R}^3$ será un plano.
\begin{center}
    \begin{tikzpicture}[scale=0.5]
      % Define los vértices del romboide
      \coordinate (A) at (0,0);
      \coordinate (B) at (4,0);
      \coordinate (C) at (9,2);
      \coordinate (D) at (5,2);
      % Dibuja el romboide y etiqueta sus vértices
      \draw (A) -- (B) -- (C) -- (D) -- cycle;
      % Que el plano se llame P
      \node [below right] at (B) {$\left\{Ax:x\in \mathbb{R}^n\right\}$};
      % trazar una linea entrecortada perpedicular al plano y los puntos extremos que tengan un punto negro
      \draw[dashed] (4,2.5)node[above left]{$b$} -- (4,.6)node[left]{$b_0$};
      \fill (4,2.5) circle (2pt);
      \fill (4,.6) circle (2pt);
    \end{tikzpicture}
\end{center}

Si $b\in \left\{Ax:x\in \mathbb{R}^n\right\}\quad \Leftrightarrow \quad x^* \in \mathbb{R}^n : Ax^* =b.$ El valor optimo $f_0(x^*)=0.$

Si $b\notin \left\{Ax:x\in \mathbb{R}^n\right\}$, $f_0(x^*)=d(b,b_0)^2$.

Ahora, cual es el optimo?; es decir cual es el $x^*$

Donde la solución es:
$$x^*\in \mathbb{R}^n : Ax^*=b_0.$$ Aquí, $b_0$ está en el plano, si estamos en $\mathbb{R}^3$. ¿Cómo llegamos algebraicamente?:
$$
\begin{array}{rcl}
b-b_0 \perp \left\{Ax:x\in \mathbb{R}^n\right\}&\Leftrightarrow & b-b_o \perp A^i,\; i=1,\ldots,n\\\\  
						 &\Leftrightarrow & \langle b-b_0,A^i\rangle=0,\; i=1,\ldots,n\\\\
							   &\Leftrightarrow& \langle b-Ax^*,A^i\rangle = 0, \; i=1,\ldots,n\\\\
							   &\Leftrightarrow& \langle Ax^*-b,A^i\rangle = 0, \; i=1,\ldots,n\\\\
								   &\leftrightarrow&A^TAx^*=A^Tb.
\end{array}
$$
\end{ejem}
Las ecuaciones normales vienen dadas por la perpendicularidad.

\section{Conjuntos convexos de \boldmath $\mathbb{R}^n$}

El dominio tendrán que ser conjuntos convexos o Dominio efectivo.

% -------------------- DEFINICIÓN 1 LINEAL
\begin{def.}[Lineal]\,\\ 
$$L(x_0,x_1) := \left\{x_0+\lambda(x_1-x_0):\lambda \in \mathbb{R}\right\}$$
\begin{center}
    \begin{tikzpicture}[scale=0.5]
      \coordinate (A) at (0,0);
      \coordinate (B) at (3,1);
      \coordinate (C) at (6,2);
      \coordinate (D) at (-3,-1);
      \draw[->] (A) -- (B)node[below]{$x_1$};
      \draw[dashed] (B) -- (C);
      \draw[dashed] (D) -- (A);
      \fill (A) circle (2pt) node[below]{$x_0$};
    \end{tikzpicture}
\end{center}
\end{def.}

\begin{itemize}
    \item Cuando $\lambda$ vale $1$ me sale $x_1$ cuando valga cero me sale $x_0$ cuando es positivo va hacia la derecha y cuando es negativo va hacia la izquierda.
    \item Toda la recta nos da un concepto que denominamos Afín. 
    \item Para la convexidad no es necesario tener la linea, solo necesitaremos un segmento. 
	$$\left\{x_0+\lambda(x_1-x_0):\lambda \in \left[a,b\right]\right\}$$
    \item El segmento importante será el intervalo que denotaremos como:
	$$\left[x_0,x_1\right]:=\left\{x_0+\lambda(x_1-x_0):\lambda \in \left[0,1\right]\right\}=\left\{(1-\lambda)x_0+\lambda x_1:\lambda \in \left[0,1\right]\right\}$$
	Es cualquier punto que este entre $x_0$ y $x_1$ del gráfico de arriba. 
\end{itemize}


% -------------------- DEFINICIÓN DE AFÍN
\begin{def.}
    Sea $A\subseteq \mathbb{R}^n$. Se dice \textbf{Afín}, si para todo $x, y\in A$ se tiene que la $L(x,y)\subseteq A$. (Subespacios vectoriales desplazados).
\end{def.}

\begin{itemize}
    \item Un circulo no es afín ya que la linea es infinita y un circulo no.
    \item Un plano podría ser Afín
    \item La recta es afín
    \item Todo $\mathbb{R}^n$ es afín.
    \item Un único punto también es afín, dado que $x=y$.
    \item La diferencia entre espacio vectorial y espacio afín es que el espacio afín esta desplazada; es decir, no necesariamente pasa por el cero como en un subespacio vectorial.
\end{itemize}

Manejar el concepto de afín con lineas es un poco incomodo, entonces se utiliza el concepto de combinación afín  

% -------------------- DEFINICIÓN DE combinación afín
\begin{def.}
    Una \textbf{combinación afín} de los vectores $\left\{x_1,x_2,\ldots,x_k\right\}$ es un vector de la forma 
$$\lambda_1 x_1+\lambda_2x_2+\cdots+\lambda_kx_k.$$
tal que 
$$\sum_{i=1}^k \lambda_i = 1.$$
\end{def.}

\begin{itemize}
    \item Lo que decimos que es una combinación lineal de $x_0$ y $x_1$.
    \item Lo demás puntos fuera del segmento son las combinaciones lineales de $x_0$ y $x_1$.
\end{itemize}

% -------------------- TEOREMA 1
\begin{teo}
    $A$ es afín sii $A$ contiene toda combinación afín de sus puntos.

	Demostración.-\; Primero, tomemos puntos arbitrarios $\left\{x_1,x_2,\ldots,x_k\right\}$ en $A$ tal que 
	$$z=\lambda_1x_1+\lambda_2x_2+\cdots+\lambda_kx_k$$
	donde $\sum\limits_{i=1}^k \lambda_i = 1$. 

	Ahora, consideremos dos puntos $x_i,x_j$ de $z$. Dado que $A$ es afín, entonces $L(x_i,x_j)\subseteq A$, para todo $x_i,x_j$. Esto implica que $z$ está en $A$. Intuitivamente, si 
	$$\lambda_1x_1+\lambda_2 x_2,\quad \lambda_3\lambda_3+\lambda_4\lambda_4,\quad \ldots,\quad  \lambda_{k-1}x_{k-1}+\lambda_kx_k$$ 
	están en $A$. Entonces, $z$ tendrá que estar en $A$.

	Para demostrar la otra implicación, tomemos dos puntos cualesquiera $x$ e $y$ en $A$.  Entonces, $L(x,y)\subseteq A$. Por lo tanto, $A$ es afín.
\end{teo}

\begin{itemize}
    \item Quiere decir que este conjunto es estable para combinaciones lineales muy similar al concepto de subespacio vectorial.
\end{itemize}
\vspace{.5cm}

% -------------------- NOTACIÓN 1
\begin{notacion}
La suma de Minkowski es la operación de conjuntos; es decir, si $A,E\subseteq \mathbb{R}^n$. Entonces,
$$A=x_0+E=\left\{x_0+e:e\in E \right\} \quad \mbox{o}\quad E=A-x_0=\left\{a-x_0:a\in A\right\}.$$ 
Es sencillamente trasladar los puntos del plano y desplazarlos o moverlos.
\end{notacion}

\begin{nota}
La definición de subespacio se refiere a tomar dos escalares y dos vectores, realizar la combinación lineal, donde esta combinación lineal no se saldrá del conjunto dado.
\end{nota}


% -------------------- TEOREMA 2
\begin{teo}
    $A\subseteq \mathbb{R}^n$ es afín sii existe un $E\subseteq \mathbb{R}^n$ subespacio vectorial tal que $A=x_0+E$ para todo $x_0\in A$.\\\\
	Demostración.-\; Supongamos que $A$ es afín y fijamos $x_0\in A$. Intentaremos probar que $E=A-x_0$ es un subespacio de $\mathbb{R}^n$, esto es equivalente a decir que:
	$$\lambda,\mu \in \mathbb{R}, e_i,e_2\subseteq E \quad \Rightarrow \quad \lambda e_i+\mu e_2\in E.$$
	Probemos que $\lambda e_1+\mu e_2\in E$; en otras palabras probaremos que $\lambda e_1+\mu e_2$ es $a-x_0$.
	$$
	\begin{array}{rcl}
	    \lambda e_1+\mu e_2&=&\lambda(a_1-x_0)+\mu(a_2-x_0)\\\\
			       &=&\lambda a_1 + \lambda a_2 - \lambda x_0-\mu x_0\\\\
			       &=&\lambda a_1 + \lambda a_2 - \lambda x_0-\mu x_0 +x_0-x_0\\\\
			       &=&\lambda a_1 + \lambda a_2+(1-\lambda-\mu)x_0-x_0.
	\end{array}
	$$
	Observemos que $\lambda a_1 + \lambda a_2+(1-\lambda-\mu)x_0$ está en $A$, dado a que $\lambda+\mu+(1-\lambda-\mu)=1$. Por lo tanto,
	$$A-x_0=E.$$
	Es un subespacio vectorial.

	Ahora, sabemos que $A\subseteq \mathbb{R}^n$ tal que $A=E+x_0$ para todo $x_0\in A$ es afín. Entonces, $E$ es un subespacio vectorial. Para demostrar que $A$ es afín, probaré que cualquier combinación afín de elementos de $A$ sigue estando en $A$. Sean,
	$$\left\{a_1,a_2,\ldots,a_k\right\},\; \lambda_1,\lambda_2,\cdots,\lambda_k:\sum \lambda_i=1.$$
	De donde,
	$$
	\begin{array}{rcl}
	    \lambda_1a_1+\lambda_2a_2+\cdots+\lambda_ka_k&=&\lambda_1(e_1-x_0)+\lambda_2(e_2-x_0)+\cdots+\lambda_k(e_k-x_0)\\\\
							   &=& \lambda_1e_1+\cdots+\lambda_ke_k+\left(\displaystyle\sum_{i=1}^k \lambda_i\right)x_0
	\end{array}
	$$
	Observemos que $\lambda_1e_1+\cdots+\lambda_ke_k$ es una combinación lineal afín el cual existe en $E$ y por definición, $\left(\displaystyle\sum_{i=1}^k \lambda_i\right)=1$. Por lo tanto,
	$$E+x_0=A.$$
\end{teo}

% -------------------- DEFINICIÓN 3 ENVOLTURA O SPAN AFÍN
\begin{def.}[Envoltura Afín]\,\\\\
    La envolura afín de $B$, $Aff(B)$, es el menor conjunto afín que contiene a $B$. Esto implica que es el conjunto de las combinaciones afines de elementos de $B$ o es la intersección de los conjuntos afines que contienen a $B$ 
\end{def.}

% -------------------- DEFINICIÓN
\begin{def.}
    Si $A$ es $Affin$ se llama "dimensión afín de $A$" a la dimensión de su espacio vectorial.
\end{def.}

\begin{itemize}
    \item Dimensión $0$ un punto.
    \item Dimensión $1$ una recta.
    \item Dimensión $2$ una plano.
\end{itemize}



% -------------------- EJEMPLO 2
\begin{ejem}
    Dado $C\in \mathbb{R}^n$ afín. Siempre existirán una matriz $A\in \mathcal{M}_{p\times n}$ y $b\in \mathbb{R}^p$ tal que
    $$C=\left\{x\in \mathbb{R}^n:Ax=b\right\}.$$\\
	Solución.-\; Cuál es el conjunto lineal asociado?, será el núcleo de la aplicación lineal; es decir,
	$$E=\left\{x\in \mathbb{R}^n:Ax=0\right\}.$$
	Cualquier solución de $x_0\in C$, de modo que $Ax_0=b$. Tomando un punto de $C$ y otro de $E$, tenemos 
	$$A(x_0+e)=Ax_0+Ae=b+0=b.$$
	Por lo tanto,
	$$C=\left\{x\in \mathbb{R}^n:Ax=b\right\}=x_0+E.$$
	Así, el conjunto afín no es más que el traslado del espacio vectorial.
\end{ejem}

% -------------------- Definición 1.6
\begin{def.}[Topología de \boldmath$\mathbb{R}^n$]\,\\\\
    Sean $A\subseteq \mathbb{R}^n$ y $a\in \mathbb{R}^n$.
    \begin{enumerate}[1)]
	\item $a\in A$ está en el interior de $A$ $\left(a\in \interior(A) \mbox{ o } \mathring{A} \right)$, cuando existe $\delta>0$ tal que $B(a,\delta)\subseteq A.$
	$$B(a,\delta)=\left\{x\in \mathbb{R}^n:\|x-a\|\leq \delta.\right\}.$$
	Por ejemplo en $\mathbb{R}^2$ será un circulo y $\mathbb{R}^3$ será una esfera.
	\item $A$ se dice abierto si $A=\interior(A)=\mathring{A}.$ (Los puntos frontera del conjunto $A$ serán los abiertos).
	\item Decimos que $c\in\mathbb{R}^n$ está en el cierre (o clausura) de $A$, cuando $\exists \left\{a_n\right\}\in A | a_n\to c$.
	\item Decimos que $A$ es cerrado cuando $A=\overline{A}$ donde $\left\{x\in \mathbb{R}^n: x \mbox{ está en el cierre de }A\right\}.$ (El cierre se divide en los puntos del interior y los puntos frontera).

	\item Se llama frontera de $A$, $\partial{A}$ a la intersección $\overline{A}\cap \left( \overline{\mathbb{R}\backslash A}\right)=\overline{A} \backslash \interior(A)$.
	\item $a\in \relint(A)$ si existe $\delta>0$ tal que $B(a,\delta)\cap Aff(A)\subseteq A$.\\
	    Lo que decimos es que la bola puede ser muy grande y vivir en $\mathbb{R}^3$ e intersecar en el plano, donde se corta transversalmente para proyectar la imagen.
    \end{enumerate}
\end{def.}

\begin{itemize}
    \item El concepto de punto interior será importante, porque me dice que me puedo acercar al punto $a$ desde todas las direcciones, y si es relativo interior me puedo acerca por todos los lados del conjunto.
    \item El punto de adherencia o clausura es un punto que me puedo acercar de alguna forma pero no de todas formas.
\end{itemize}

% -------------------- EJEMPLO 1.3
\begin{ejem}
    Dibujemos un plano
    \begin{center}
      \begin{tikzpicture}[scale=0.5]
	% Define los vértices del romboide
	\coordinate (A) at (-2,0);
	\coordinate (B) at (6,0);
	\coordinate (C) at (11,3);
	\coordinate (D) at (3,3);
	% Dibuja el romboide y etiqueta sus vértices
	\draw (A) -- (B) -- (C) -- (D) -- cycle;
	% Dibuja el contorno del riñón
	\coordinate (Center) at ($(A)!0.5!(C)$);
	\draw[] ($(Center)+(-1,.5)$) .. controls ($(Center)+(0.2,2)$) and ($(Center)+(3,-.1)$) .. ($(Center)+(1,-.4)$) .. controls ($(Center)+(0,-.5)$) and ($(Center)+(0,-1)$) .. ($(Center)+(-1,-.9)$) .. controls ($(Center)+(-1.5,-1)$) and ($(Center)+(-3,-.5)$) .. ($(Center)+(-1,.5)$);
	% Rellena el interior del riñón con líneas
	\pattern[pattern=north east lines, pattern color=gray!50] ($(Center)+(-1,.5)$) .. controls ($(Center)+(0.2,2)$) and ($(Center)+(3,-.1)$) .. ($(Center)+(1,-.4)$) .. controls ($(Center)+(0,-.5)$) and ($(Center)+(0,-1)$) .. ($(Center)+(-1,-.9)$) .. controls ($(Center)+(-1.5,-1)$) and ($(Center)+(-3,-.5)$) .. ($(Center)+(-1,.5)$);
	\node at ($(Center)+(-.2,.2)$) {\small$B$};
	\node at ($(Center)+(1.9,.7)$) {\small$A$};
      \end{tikzpicture}
    \end{center}
    \begin{itemize}
	\item $B$ es el interior con la frontera.
	\item $A$ es la frontera.
	\item El objetivo será encontrar el punto optimo de un esfera que está proyectada en este plano. 
	\item El conjunto tendrá que ser convexo.
    \end{itemize}
    Veamos algunas propiedades de este conjunto.
    \begin{enumerate}[1)]
	\item $A$ es cerrado.- Cualquier punto que ponga en $B$ me puedo acercar por puntos de $B$.
	\item $B$ cerrado.
	\item $\mathring{A}=\emptyset$.- Si yo ponga una bola, se saldrá del conjunto $A$.
	\item $\mathring{B}=\emptyset$.- Ya que no existirá en el plano ninguna esfera. 
	\item $relint(A)=\emptyset$.
	\item $relint(B)=B\backslash A.$
    \end{enumerate}
\end{ejem}

% -------------------- Definición 1.8
\begin{def.}[Combinación convexa]\,\\\\
    Sean $x_1,x_2,\ldots,x_k\in \mathbb{R}^n$ y $\lambda_1,\lambda_2,\ldots,\lambda_k\in \mathbb{R}$ tales que $\lambda_i\geq 0$ y $\displaystyle\sum_{i=1}^{k}\lambda_i=1$. Al vector
    $$\sum_{i=1}^{k}\lambda_ix_i=\lambda_1x_1+\lambda_2x_2+\ldots+\lambda_kx_k$$
    se le llama combinación convexa de los puntos $\left\{x_1,\ldots,x_k\right\}$.\\

	La única diferencia entre combinación convexa y afín es que sean positivos.
\end{def.}

% -------------------- Observación 1.1
\begin{obs}
    La definición para 2 puntos $\left\{x_1,x_2\right\}$ nos de las combinaciones convexas,
    $$\lambda x_1+(1-\lambda)x_2,\qquad \lambda\geq 0, (1-\lambda)\geq 0 \; \Leftrightarrow \; \lambda \in \left[0,1\right].$$
    Esto es el segmento,
    $$\left\{\lambda x_1+(1-\lambda)x_2: \lambda\in \left[0,1\right]\right\}=\left[x_1,x_2\right].$$
    Nos quedamos con el segmento que los une, eso nos permitirá utilizar las propiedades de los números reales. Por lo que podremos realizar análisis.
\end{obs}

% -------------------- Definición 1.9
\begin{def.}[convexo]\,\\\\
    Un conjunto $C\in \mathbb{R}^n$ se dice convexo cuando $C$ contiene las combinaciones convexas de sus puntos, (Decimos que $C$ es cerrado para las combinaciones convexas), si y sólo si
    $$\forall x_1,x_2\in C \Rightarrow \left[x_1,x_2\right]\subseteq C.$$
    Un conjunto es convexo si dados dos puntos el segmento que los une se queda adentro.
\end{def.}

% -------------------- EJEMPLO 1.4
\begin{ejem}
    \begin{multicols}{3}
	\begin{center}
	  \begin{tikzpicture}[scale=0.8]
	    % Dibuja el contorno del riñón
	    \coordinate (Center) at ($(A)!0.5!(C)$);
	    \draw[] ($(Center)+(-1,.5)$) .. controls ($(Center)+(0.2,2)$) and ($(Center)+(3,-.1)$) .. ($(Center)+(1,-.4)$) .. controls ($(Center)+(0,-.5)$) and ($(Center)+(0,-1)$) .. ($(Center)+(-1,-.9)$) .. controls ($(Center)+(-1.5,-1)$) and ($(Center)+(-3,-.5)$) .. ($(Center)+(-1,.5)$);
	    % Rellena el interior del riñón con líneas
	    \pattern[pattern=north east lines, pattern color=gray!50] ($(Center)+(-1,.5)$) .. controls ($(Center)+(0.2,2)$) and ($(Center)+(3,-.1)$) .. ($(Center)+(1,-.4)$) .. controls ($(Center)+(0,-.5)$) and ($(Center)+(0,-1)$) .. ($(Center)+(-1,-.9)$) .. controls ($(Center)+(-1.5,-1)$) and ($(Center)+(-3,-.5)$) .. ($(Center)+(-1,.5)$);
	    % trazar una linea que una dos puntos del contorno pero este fuere del conjunto
	    \draw[dashed](2.6,0.1)--(4,.55);
	    \node at (2.7,-.4) {\small No convexo};
	  \end{tikzpicture}

	  \begin{tikzpicture}[scale=0.7]
	    % Dibuja un óvalo
		\draw (0,0) ellipse (2.5cm and 1.2cm);
		% Agrega un patrón de líneas diagonales al óvalo
		\pattern[pattern=north east lines, pattern color=gray!50] (0,0) ellipse (2.5cm and 1.2cm);
		\node at (0,1.7) {\small Convexo};
	    \end{tikzpicture}

	  \begin{tikzpicture}[scale=0.6]
		\coordinate (A) at (0,0); % Vértice 1
		\coordinate (B) at (2,0); % Vértice 2
		\coordinate (C) at (3,1.73); % Vértice 3
		\coordinate (D) at (2,3.46); % Vértice 4
		\coordinate (E) at (0,3.46); % Vértice 5
		\coordinate (F) at (-1,1.73); % Vértice 6
		\draw (A) -- (B) -- (C) -- (D) -- (E) -- (F) -- cycle; % Dibuja el hexágono
		\draw[pattern=north east lines, pattern color=gray!50] (A) -- (B) -- (C) -- (D) -- (E) -- (F) -- cycle; % Aplica el patrón al hexágono
		\node at (1,4) {\small Convexo};
	    \end{tikzpicture}
	\end{center}
    \end{multicols}
\end{ejem}

Del gráfico 1) ¿Cuál es el menor conjunto convexo que lo contiene?
\begin{center}
  \begin{tikzpicture}[scale=0.8]
    % Dibuja el contorno del riñón
    \coordinate (Center) at ($(A)!0.5!(C)$);
    \draw[red] ($(Center)+(-1,.8)$) .. controls ($(Center)+(0.2,2)$) and ($(Center)+(3,-.1)$) .. ($(Center)+(1,-.5)$) .. controls ($(Center)+(0,-.8)$) and ($(Center)+(0,-1)$) .. ($(Center)+(-1,-.9)$) .. controls ($(Center)+(-1.5,-1)$) and ($(Center)+(-3,-.5)$) .. ($(Center)+(-1,.8)$);
    % Rellena el interior del riñón con líneas
    \pattern[pattern=north east lines, pattern color=gray!50] ($(Center)+(-1,.5)$) .. controls ($(Center)+(0.2,2)$) and ($(Center)+(3,-.1)$) .. ($(Center)+(1,-.4)$) .. controls ($(Center)+(0,-.5)$) and ($(Center)+(0,-1)$) .. ($(Center)+(-1,-.9)$) .. controls ($(Center)+(-1.5,-1)$) and ($(Center)+(-3,-.5)$) .. ($(Center)+(-1,.5)$);
    % trazar una linea que una dos puntos del contorno pero este fuere del conjunto
  \end{tikzpicture}
\end{center}

% ------------------- Definición 1.10
\begin{def.}
    se llama Envoltura convexa de $A$ al menor conjunto convexo que lo contiene $\Leftrightarrow$ la intersección de todos los convexos que contienen a $A$, denotado por $co(A)$.
    Es equivalente a decir que
    $$co(A)=\left\{\mbox{Combinación convexa de puntos de A.}\right\}$$
\end{def.}

% ------------------- Ejercicio 
\begin{ejer}
    Demostrar que la intersección de conjuntos convexos es convexo.\\\\
	Demostración.-\; 
\end{ejer}


% ------------------- Definición 1.11
\begin{def.}[Cono]\,\\\\
    Un conjunto $C\subseteq \mathbb{R}^n$ se llama cono si y sólo si
    $$\lambda x\in C \mbox{ si } x\in C,\; \lambda \geq 0.$$

    Contiene los rayos que pasan por el cero e intersecan a un punto dado.
    \begin{itemize}
	\item Un cono siempre contiene al origen.
	\item La envoltura cónica de un conjunto es $con(A)=\left\{\lambda : \lambda \geq 0, a\in A\right\}$ la intersección de todos los conos que contiene a $A$.
	\item Un cono $C$ es convexo si y sólo si 
	$$\lambda_1,\lambda_2\in C \Rightarrow \lambda_1x_1+\lambda_2x_2\in C, \; \forall \lambda_1,\lambda_2 \geq 0.$$
    \end{itemize}
\end{def.}

El hiperplano es un caso particular del estudio convexo.

% ------------------- Definición 1.12
\begin{def.}[Hiperplano]\,\\\\
    $H\subseteq \mathbb{R}^n$ es un \textbf{hiperplano} si existe $a\in \mathbb{R}^n \backslash\left\{0\right\}$ tal que
    $H=\left\{x\in \mathbb{R}^n : \langle a,x\rangle = a^T x = 0\right\} = a^\perp$
\end{def.}

% ------------------- 


% ENTREGAS
% Entrega 1 
%\center \textbf{CONVEXIDAD Y OPTIMIZACIÓN}
\center \textbf{\Large ENTREGA 1}
\center \textbf{ \textbf{Christian Limbert Paredes Aguilera}}
\center Latex Source: \url{}

\line(1,0){400}



\begin{enumerate}[\bfseries \text{Ejercicio} 1.]

    % ------------------- EJERCICIOS 1 ------------------
    \item \textbf{\boldmath Demuestra que la envoltura convexa de un conjunto $S\subset \mathbb{R}^n$ es la intersección de todos los conjuntos convexos de $\mathbb{R}^n$ que contienen a $S$.}

	\textbf{Demostración.-}\; Sean $\co(S)$\footnote{
	Se llama \textbf{envoltura convexa} de $S$ al menor conjunto convexo que contienen a $S$, denotado por $\co(S)$.
	También es equivalente a decir que:
	$\co(S)=\left\{\mbox{Combinación convexa de puntos de} S\right\}.$\\
	\label{envoltura}}
	la envoltura convexa del conjunto $S$ e $I$ como la intersección de todos los conjuntos convexos que contienen a $S$. En si lo que vamos a querer demostrar es:
	$$\co(S)=I.$$
	En otras palabras, queremos demostrar que
	$$\co(S)\subseteq I \quad \land \quad I\subseteq \co(S).$$
	Notemos que $\co(S)$\footref{envoltura} es un conjunto convexo que contiene a $S$. Esto significa que cualquier otro conjunto convexo que contenga a $S$ debe ser al menos tan grande como $\co(S)$, y debe contener a $\co(S)$. Por lo que cualquier conjunto convexo que contenga a $S$ debe contener también a $\co(S)$. De esta manera, $I$ debe contener al menos a $\co(S)$. Es decir,
	$$\co(S)\subseteq I.$$
	Para demostrar la otra inclusión, supondremos que 
	$$\co(S)\not\subseteq I.$$
	De lo que notamos que existe al menos un punto $x\in C$ tal que $x\notin \co(S)$. Por definición de envoltura\footref{envoltura}, esta $x$ no puede ser escrito como una combinación convexa\footnote{
	Sean $x_1,x_2,\ldots,x_k\in \mathbb{R}^n$ y $\lambda_1,\lambda_2,\ldots,\lambda_k\in \mathbb{R}$ tales que 
	$\lambda_i\geq 0\; \text{y} \;\sum_{i=1}^{k}\lambda_i=1.$ 
	Al vector
	$\sum_{i=1}^{k}\lambda_ix_i=\lambda_1x_1+\lambda_2x_2+\ldots+\lambda_kx_k$
	se le llama \textbf{combinación convexa} de los puntos $\left\{x_1,\ldots,x_k\right\}$.\\\\
	} de puntos en $S$.

	Luego, $x$ esta en $I$ e $I$ es la intersección de todos los conjuntos convexos que contienen a $S$. Por lo que, $x$ debe estar en cada conjunto convexo que contiene a $S$. Y dado que $I$ es la intersección de todos los conjuntos convexos que contiene a $S$, se sigue que $I$ es un conjunto convexo
	\footnote{
	    \textbf{Demostrar que la intersección de conjuntos convexos es convexo.}\;
		Demostremos por contradicción. Sean $C_1$ y $C_2$ dos conjuntos convexos. Y sea 
		$C=C_1\cap C_2.$
		no convexo. Esto significa que existen $x$ e $y$ tales que 
		$\left\{\lambda x + (1-\lambda)y:\lambda\in \mathbb{R}\right\}\not\subseteq C.$ 
		Supongamos ahora que $x$ e $y$ están en $C$. Cómo ambos $C_1$ y $C_2$ son convexos, el segmento definido debe estar en ambos conjuntos. Es decir,
		$\left\{\lambda x + (1-\lambda)y:\lambda\in \mathbb{R}\right\}\subseteq C.$ 
		Lo que contradice nuestra suposición inicial. Por lo tanto, $C$ es convexo.\\\\
	\label{dos}} que contiene a $S$. 

	Por el hecho de que $x\in I$, $x$ debe ser una combinación convexa de puntos en $S$. Pero esto contradice el hecho de que $x$ no puede ser escrito como una combinación convexa de puntos en $S$. Así concluimos que $I\subseteq \co(S)$. Y por lo tanto,
	$$\co(S)=I.\;\blacksquare$$
	\vspace{.5cm}

    % ----------------------- EJERCICIO 2 -------------------------
    \item \textbf{\boldmath La descripción de semiespacios de Voronoi: Sean $a$ y $b$ dos puntos distintos de $\mathbb{R}^n$. Demuestra que el conjunto de todos los puntos que están más cerca de $a$ (en la distancia Euclídea) que de $b$, i.e., $\left\{x\in \mathbb{R}^n : \|x-a\|_2 \leq \|x-b\|_2\right\}$, es un semiespacio. Descríbelo explícitamente como una desigualdad de la forma $c^T\cdot x \leq d$ (donde $c$ es un vector columna de $\mathbb{R}^n$). Haz una representación gráfica de la situación.}

	\textbf{Demostración.-}\; Primero, representemos gráficamente la situación.
	\begin{center}
	    \begin{tikzpicture}[scale=.75,>=Triangle,rotate=40]
		\pattern[pattern=north east lines, pattern color=orange!50, rotate around={0:(0,0)}] (0,-2) rectangle (2cm,2cm)node[above left,rotate=40,orange]{$c^Tx<b$};
		\draw [fill=black] (0,0) circle (2pt) node[below] {$a$};
		\draw [fill=black] (4,0) circle (2pt) node[below] {$b$};
		\draw [very thick,blue] (2,-2)node[below, rotate=40]{$c^Tx=b$} -- (2,2);
		\draw [->, thick,green] (2,0) -- (3.5,0);
	    \end{tikzpicture}
	\end{center}

	Ahora, con algunas manipulaciones algebraicas llegaremos a la desigualdad
	$$c^T\cdot x \leq d.$$
	Que es la forma estándar de un semiespacio\footnote{
	    Todo \textbf{hiperplano} $H=\left\{x:a^Tx=0, a\neq 0\right\}$, define dos semiespacios 
	    $\left\{x\in\mathbb{R}^n:a^T x \leq 0\right\},\; \left\{x\in \mathbb{R}^n:a^T x\geq 0\right\}.$
	    Más generalmente, 
	    $\left\{x\in\mathbb{R}^n:a^T x \leq b\right\},\; \left\{x\in \mathbb{R}^n:a^T x\geq b\right\}.$
	    son las soluciones de dos sistemas lineales de desigualdades.\\

	}.

	(Por convención diremos que los vectores $x$, $a$ y $b$ son vectores columna en $\mathbb{R}^n$). Dado que tenemos la desigualdad en término de normas (términos no negativos). Podemos elevar al cuadrado sin cambiar el orden de los elementos. Es decir, 
	$$\|x-a\|_2 \leq \|x-b\|_2\quad \Rightarrow \quad \|x-a\|_2^2 \leq \|x-b\|_2^2.$$
	Luego, por la definición de norma Euclídea\footnote{
	    \textbf{\boldmath$\|x\|_2^2=x^Tx.$}
	}
	y la propiedad conmutativa de producto interno, tenemos 
	$$
	\begin{array}{rcl}
	    (x-a)^T(x-a) &\leq&(x-b)^T(x-b)\\\\
	    &\Downarrow&\\\\
	    x^Tx-2a^Tx+a^Ta &\leq& x^Tx-2b^Tx+b^Tb.
	\end{array}
	$$
	Después, por la sustracción de vectores podemos simplificar la desigualdad como:
	$$2b^Tx-2a^Tx\leq b^Tb-a^Ta.$$
	Que es equivalente a,
	$$(b-a)^Tx\leq \dfrac{1}{2}\left(b^Tb-a^Ta\right).$$
	Podemos definir $c=b-a$ y $d=\dfrac{1}{2}\left(b^Tb-a^Ta\right)$, para obtener la desigualdad
	$$c^Tx\leq d.$$
	Por lo tanto, el conjunto de todos los puntos que están más cerca de $a$ que de $b$ en la distancia euclídea es un semiespacio.$\;\blacksquare$
	\vspace{1cm}

    % -------------------------- EJERCICIO 3 -------------------------
    \item \textbf{\boldmath Demuestra que si $A$ y $B$ son conjuntos convexos de $\mathbb{R}^n$ entonces su intersección $A\cap B$ y su suma de Minkowsky $A+B$ son conjuntos convexos de $\mathbb{R}^n$.}

	\textbf{Demostración.-}\; Primero, demostraremos que la intersección $A\cap B$ es convexa. Sabemos por hipótesis que $A$ y $B$ son conjuntos convexos en $\mathbb{R}^n$. Tomemos ahora, dos puntos cualesquiera $x,y\in A\cap B$. Dado que $x,y\in A$. Entonces, para todo $\lambda\in [0,1]$, se tiene:
	$$\lambda x+(1-\lambda)y\in A.\footnote{
	    Un conjunto $C$ en un espacio vectorial es \textbf{convexo}, si para cada par de puntos $x,y\in C$ y para cada número real $\lambda$ en el intervalo $[0,1]$, se cumple que:
	    $\lambda x+(1-\lambda)y\in C.$\\
	\label{convex}}$$
	De manera similar, dado que $x,y\in B$. Entonces, por definición de convexidad para todo $\lambda \in [0,1]$, también tenemos:
	$$\lambda x+(1-\lambda)y\in B \footref{convex}.$$
	Por lo tanto, 
	$$\lambda x+(1-\lambda)y\in A\cap B,$$
	lo que implica que la intersección de conjuntos convexos es convexa. Esto también se puede demostrar con (\footref{dos}).

	Ahora, demostraremos que si $A$ y $B$ son conjuntos convexos. Entonces, $A+B$ es convexo. Sean dos puntos cualesquiera $\alpha,\beta\in A+B$, por la suma de Minkowsky,\footnote{
	    La \textbf{suma de Minkowski} es la operación de conjuntos; es decir, si $A,E\subseteq \mathbb{R}^n$. Entonces,
	    $A=x_0+E=\left\{x_0+e:e\in E \right\} \quad \mbox{o}\quad E=A-x_0=\left\{a-x_0:a\in A\right\}.$
	},   
	existen $a_1,a_2\in A$ y $b_1,b_2\in B$ tales que 
	$$\alpha=a_1+a_1 \quad \mbox{y}\quad \beta=a_2+b_2.$$
	La idea es mostrar que para cualquier $\lambda\in[0,1]$, 
	$$\lambda\alpha+(1-\lambda)\beta\in A+B.$$
	Tenemos que
	$$
	\begin{array}{rcl}
	    \lambda\alpha+(1-\lambda)\beta&=&\lambda(a_1+b_1)+(1-\lambda)(a_2+b_2)\\\\
					  &=&\left[\lambda a_1+(1-\lambda)a_2\right]+\left[\lambda b_1+(1-\lambda)b_2\right].
	\end{array}
	$$
	Como $A$ y $B$ son conjuntos convexos\footref{convex}, sabemos que 
	$$\lambda a_1+(1-\lambda)a_2\in A$$
	$$\text{y}$$
	$$\lambda b_1+(1-\lambda)b_2\in B.$$
	Por lo tanto,
	$$\lambda\alpha+(1-\lambda)\beta\in A+B.$$
	Así, $A+B$ es convexo, como se quería demostrar. Esto completa la demostración.$\;\blacksquare$
	
\end{enumerate}

% Entrega A
%\begin{center}
\textbf{CONVEXIDAD Y OPTIMIZACIÓN}

\textbf{\Large ENTREGA A}

\textbf{ \textbf{Christian Limbert Paredes Aguilera}}
\end{center}
\begin{center}
    Latex Source: \url{https://n9.cl/ua8c3}
\end{center}

\line(1,0){400}


\section*{CAPÍTULO 2}

\begin{enumerate}[\bfseries \text{Ejercicio} 1.]

    % ------------------- EJERCICIOS 1 ------------------
    \item \textbf{\boldmath Sean $f, g : \mathbb{R}^n\to \mathbb{R}$ dos funciones convexas cuyo dominio es $\mathbb{R}^n$. Se define su convolución ínfima (infimal convolution) como
    $$ h(x) = (f \diamond g)(x) := \inf\left\{f(y) + g(z) : y + z = x\right\}.$$
    Demuestra que el epígrafo de $h$ es un conjunto convexo.}\\

	\textbf{Demostración.-}\; Para demostrar que el epígrafo de $h$ es un conjunto convexo, necesitamos mostrar que para cualquier par de puntos en el epígrafo de $h$, la línea que los conecta también está en el epígrafo de $h$.

	Dado que $f$ y $g$ son funciones convexas, sus epígrafos son conjuntos convexos 
	\footnote{
	    Teorema: $f$ es convexa sii $\text{Epi}(f)$ es convexo.\\
	\label{teoEpiConvexo}}
	. Esto significa que para cualquier par de puntos en el epígrafo de $f$ o $g$, la línea que los conecta también está en el epígrafo de $f$ o $g$.

	Comenzamos considerando dos puntos arbitrarios $(x_1, t_1)$ y $(x_2, t_2)$ en el epígrafo
	\footnote{
	    Recordemos que el epígrafo de una función es el conjunto de puntos que están por encima del gráfico de la función.\\
	\label{epigrafo}}
	de $h$. Entonces, si $(x_1, t_1)$ y $(x_2, t_2)$ están en el epígrafo de $h$, esto significa que $t_1$ y $t_2$ son mayores o iguales que el valor de $h$ en $x_1$ y $x_2$, respectivamente. Es decir,
	$$t_1 \geq h(x_1) = \inf\{f(y) + g(z) : y + z = x_1\}.$$
	$$t_2 \geq h(x_2) = \inf\{f(y) + g(z) : y + z = x_2\}.$$

	Además, $h$ se define como la convolución ínfima de $f$ y $g$, es decir, 
	$$h(x) = (f \diamond g)(x) := \inf\{f(y) + g(z) : y + z = x\}.$$ 
	Esto significa que para cada punto $x$, buscamos el valor mínimo de 
	$$f(y) + g(z)\; \text{ donde }\; y + z = x.$$

	Entonces, si $(x_1, t_1)$ y $(x_2, t_2)$ están en el epígrafo de $h$, esto significa que existen $y_1, z_1, y_2, z_2$ tal que 
	$$y_1 + z_1 = x_1,\quad  y_2 + z_2 = x_2,\quad f(y_1) + g(z_1) \leq t_1 \quad \text{y}\quad f(y_2) + g(z_2) \leq t_2.$$ 
	En otras palabras, podemos encontrar puntos $y_1, z_1, y_2, z_2$ tal que la suma de sus imágenes bajo $f$ y $g$, respectivamente, es menor o igual que $t_1$ y $t_2$. (Este paso será crucial para la demostración porque nos permite relacionar los puntos en el epígrafo de $h$ con los puntos en los epígrafos de $f$ y $g$, lo cual es necesario para demostrar que el epígrafo de $h$ es un conjunto convexo). 

	Ahora consideremos un punto $(x, t)$ en la línea que conecta $(x_1, t_1)$ y $(x_2, t_2)$. Podemos escribir $(x, t)$ como una combinación convexa de $(x_1, t_1)$ y $(x_2, t_2)$. Es decir, 
	$$x = \lambda x_1 + (1 - \lambda) x_2 \quad \text{y}\quad  t = \lambda t_1 + (1 - \lambda) t_2$$
	para algún $\lambda \in [0, 1]$.

	Dado que los epígrafos de $f$ y $g$ son conjuntos convexos, existen $y$ y $z$ tal que 
	$$
	\begin{array}{rclcrcl}
	    y &=& \lambda y_1 + (1 - \lambda) y_2, & & f(y) &\leq& \lambda f(y_1) + (1 - \lambda) f(y_2).\\\\
	    z &=& \lambda z_1 + (1 - \lambda) z_2, & & g(z) &\leq& \lambda g(z_1) + (1 - \lambda) g(z_2).
	\end{array}
	$$

	Entonces, tenemos que 
	$$y + z = x \quad \text{y}\quad f(y) + g(z) \leq t.$$ 
	Por lo tanto, $(x, t)$ está en el epígrafo de $h$. Así, el epígrafo de $h$ es un conjunto convexo.$\blacksquare$\\\\


    % ------------------- EJERCICIOS 2 ------------------
    \item \textbf{\boldmath Un punto extremal de un conjunto convexo $C$ es aquel punto $x\in C$ que cumple que si existen $y, z$ en $C$ tales que $x = (y + z)/2$ entonces $y = z$. Demuestra que si $C \subseteq \mathbb{R}^n$ es convexo, cerrado y acotado. Entonces, $C$ es la envoltura convexa del conjunto $E(C)$ de sus puntos extremales. [Indicación: Si tienes problemas para hacer la demostración, puedes asumir que $E(C)$ es no vacío].}\\
	   
	    \textbf{Demostración.-}\; Supongamos que existe un punto $x\in C$ que no está en $\co(E(C))$
	    \footnote{
	    Se llama \textbf{envoltura convexa} de $S$ al menor conjunto convexo que contienen a $S$, denotado por $\co(S)$.
	    También es equivalente a decir que:
	    $\co(S)=\left\{\mbox{Combinación convexa de puntos de} S\right\}.$\\
	    \label{envoltura}}
	    . En otras palabras, si $x$ no está en $\co(E(C))$ estos dos conjuntos son disjuntos. Así, por el teorema de separación de Hiperplanos
	    \footnote{
		El \textbf{teorema de separación de hiperplanos} establece que si tenemos dos conjuntos convexos disjuntos, entonces existe un hiperplano que los separa. Es decir, sean $A$ y $B$ conjuntos convexos con $A\cap B=\emptyset$ y $A^i\neq \emptyset$ . Entonces, existe $f\in E^\#$ lineal, tal que $f(a^i)<\inf \left\{f(b),\; b\in B\right\}$.\\
	    \label{separacion}}
	    , podemos encontrar un hiperplano, que separe a $x$ de $\co(E(C))$. 

	    Como $C$ es cerrado (contiene todos sus puntos límite)
	    \footnote{
		Decimos que $A$ es \textbf{cerrado} cuando $A=\overline{A}$ donde $\left\{ x\in \mathbb{R}^n: x \mbox{ está en el cierre de A} \right\}.$\\
	    \label{cerrado}}
	    , y como $C$ es acotado (tiene un tamaño finito). Estas dos propiedades garantizan que $C$ debe tener al menos un punto extremal, es decir, $E(C)$ no es vacío.

	    Dado que $E(C)$ no es vacío y $C$ es cerrado\footref{cerrado} y acotado, debe existir al menos un punto extremal en $C$ que esté en el mismo lado del hiperplano que $x$. Llamamos a este punto $y$.

	    Según la definición de un punto extremal, no puede existir un hiperplano que separe $C$ e $y$. Sin embargo, encontramos un hiperplano que separa $x$ e $y$. Puesto que, $x$ está en $C$, esto significa que hemos encontrado un hiperplano que separa $C$ e $y$, lo cual contradice la definición de un punto extremal. Por lo tanto, nuestra suposición inicial de que existe un punto $x$ en $C$ que no está en $\co(E(C))$ debe ser falsa. 

	    Esto implica que todos los puntos en $C$ deben estar en $\co(E(C))$, y por lo tanto, $C$ es igual a $co(E(C))$. Lo que completa la demostración. $\blacksquare$\\\\


    % ------------------- EJERCICIOS 4 ------------------
    \item[\bfseries Ejercicio 4.] \textbf{\boldmath Prueba que si $A$ es un conjunto cerrado de $\mathbb{R}^n$, con interior no vacío, y tiene un hiperplano soporte en cada punto de su frontera, entonces $A$ es convexo.}\\

	\textbf{Demostración.-}\; Sea  $A$ es un conjunto cerrado con interior no vacío y tiene un hiperplano soporte en cada punto de su frontera
	\footnote{
	    \textbf{Hiperplano soporte} significa que para cada $x_0 \in \partial A$, existe una función $f \in E^\#$ tal que $f(x_0) = \sup \left\{f(a):a\in A\right\}\geq f(a),\; \forall a\in \overline{A}$.\\
	\label{hipersoporte}}
	.Esta función $f$ representa un hiperplano soporte en el punto $x_0$.

	Ahora, sea $A$ no es convexo. De donde, supondremos dos puntos $x, y \in A$ tal que el segmento de línea que los une no está completamente en $A$. Es decir, existe un $t \in [0,1]$ tal que 
	$$tx + (1-t)y \notin A.$$ 
	(Observemos que este punto $tx + (1-t)y$ está en la línea entre $x$ e $y$, pero no está en $A$, lo cual es una contradicción a la convexidad).

	Luego, dado que $tx + (1-t)y \notin A$, debe existir un hiperplano soporte\footref{hipersoporte} en $tx + (1-t)y$. Este hiperplano divide $\mathbb{R}^n$ en dos semiespacios cerrados, uno de los cuales contiene a $A$. Este hiperplano está representado por una función $f \in E^\#$ tal que 
	$$f(tx + (1-t)y) = \sup \left\{f(a):a\in A\right\}\geq f(a),\; \forall a\in \overline{A}.$$
	Sin embargo, tanto $x$ como $y$ están en $A$, por lo que deben estar en el mismo semiespacio que $A$. Esto se debe a que la función $f$ que representa el hiperplano soporte satisface 
	$$f(x) \geq f(a) \quad \text{y}\quad f(y) \geq f(a)$$ 
	para todo $a \in \overline{A}$. Pero, todos los puntos en el segmento que une $x$ e $y$ también deben estar en el mismo semiespacio que $A$, ya que este segmento es un subconjunto del hiperplano.

	Esto contradice nuestra suposición de que 
	$$tx + (1-t)y \notin A.$$ 
	Por lo tanto, la suposición inicial, de que $A$ no es convexo, debe ser incorrecta. Así, si $A$ es un conjunto cerrado con interior no vacío y tiene un hiperplano soporte en cada punto de su frontera, entonces $A$ debe ser convexo. Esta demostración es válida independientemente de si $x$ y $y$ están en el interior o en la frontera de $A$
	\footnote{
	    En la demostración, se supone que $x$ e $y$ son puntos arbitrarios en $A$. Esto significa que pueden ser cualquier punto en $A$, ya sea en el interior de $A$ o en la frontera de $A$. La demostración no depende de la ubicación específica de $x$ e $y$ dentro de $A$.

	    Además, la definición de un hiperplano soporte implica que para cada punto en la frontera de $A$, existe un hiperplano que "soporta" $A$ en ese punto. Esto significa que el hiperplano no intersecta $A$ en ese punto y todos los puntos de $A$ están en un lado del hiperplano. Esta propiedad es válida independientemente de si estamos considerando un punto en el interior de $A$ o en la frontera de $A$.

	    Por lo tanto, la demostración de que $A$ es convexo si tiene un hiperplano soporte en cada punto de su frontera es válida independientemente de si $x$ e $y$ están en el interior o en la frontera de $A$.
	}
	. $\blacksquare$

\end{enumerate}

\pagebreak

\section*{CAPÍTULO 3}

\begin{enumerate}[\bfseries \text{Ejercicio} 1.]

    % ------------------- EJERCICIOS 1 ------------------
    \item \textbf{\boldmath Sea $f$ convexa y $x_0$ en el interior del dominio de $f$. Demuestra que:}
	\begin{enumerate}[\bfseries (a)]
	    \item \textbf{\boldmath El subdiferencial $\partial f(x_0)$ es distinto del vacío, convexo y cerrado.}\\

		\textbf{Demostración.-}\; 

		\begin{itemize}

		    % -------------- INCISO (a) --------------
		    \item \textbf{No vacío:} Por definición, si $f$ es convexa y $x_0$ está en el interior del dominio de $f$, entonces existe al menos un subgradiente
		    \footnote{
			Si $f$ es convexa y $x_0$ en int(dom(f)), encontramos el subestimador afín del resultado anterior. Al vector que define unívocamente este subestimador, se le llama \textbf{subgradiente} de $f$ en el punto.\\
			Esto significa que $g$ satisface la desigualdad $f(x) \geq f(x_0) + g^T (x-x_0).$ para todo $x$ en el dominio de $f$. Esta desigualdad dice que la función $f$ está por encima de la línea que pasa por el punto $(x_0, f(x_0))$ con pendiente $g$, lo que significa que esta línea subestima a $f$ en $x_0$. Por lo tanto, $g$ se llama un subgradiente de $f$ en $x_0$.\\
		    \label{subgradiente}}
			en $x_0$. Esto significa que el conjunto subdiferencial
		    \footnote{	
			Se llama \textbf{subdiferencial} de $f$ en $x_0$ al conjunto de todos los subgrandientes de $f$ en $x_0: \partial f(x_0)$.\\
		    \label{subdiferencial}}
			$\partial f(x_0)$ no puede ser vacío.\\

		    \item \textbf{Convexo:} Supongamos que $g,h \in \partial f(x_0)$ y $\theta \in [0,1]$. Demostraremos que 
			$$\theta g + (1-\theta)h \in \partial f(x_0).$$
			Por la definición de subgradiente\footref{subgradiente}, tenemos que:

			$$f(x) \geq f(x_0) + g^T (x-x_0)$$
			$$\text{y}$$
			$$f(x) \geq f(x_0) + h^T (x-x_0)$$
    
		    Multiplicando la primera desigualdad por $\theta$ y la segunda por $(1-\theta)$ y sumándolas, obtenemos:
		
		    $$f(x) \geq f(x_0) + (\theta g + (1-\theta)h)^T (x-x_0)$$
		    
		    Esto demuestra que $\theta g + (1-\theta)h$ es un subgradiente de $f$ en $x_0$, por lo que $\partial f(x_0)$ es convexo.\\

		    \item \textbf{Cerrado:} Para demostrar que $\partial f(x_0)$ es cerrado, necesitamos demostrar que el límite de cualquier secuencia convergente de subgradientes también pertenece a $\partial f(x_0)$. 

		    Supongamos que tenemos una secuencia $\{g_k\}$ de subgradientes tal que $g_k \to g$ cuando $k \to \infty$\footnote{
			Esto significa que para cada número positivo $\epsilon$, existe un número entero $N$ tal que para todo $k > N$, la distancia entre $g_k$ y $g$ es menor que $\epsilon$. En otras palabras, a medida que $k$ se hace más y más grande, $g_k$ se acerca cada vez más a $g$.
		    \label{convergente}}
		    . Por la definición de subgradiente\footref{subgradiente}, para cada $k$ tenemos que:
		    $$f(x) \geq f(x_0) + g_k^T (x-x_0)$$
		    Tomando el límite cuando $k \to \infty$, obtenemos:
		    $$f(x) \geq f(x_0) + g^T (x-x_0).$$

		    Lo que estamos diciendo es que si $g_k$ se acerca a $g$ cuando $k$ tiende a infinito, entonces la desigualdad también debe mantenerse en el límite cuando $k$ tiende a infinito. \\\\

		    Esto demuestra que $g$ es un subgradiente de $f$ en $x_0$, por lo que $\partial f(x_0)$ es cerrado. $\blacksquare$\\\\
		\end{itemize}
		

	    % ------------------- INCISO (b) ------------------
	    \item \textbf{\boldmath El subdiferencial $\partial f(x_0)$ es un conjunto unipuntual si y sólo si $f$ es diferenciable en $x_0$. En dicho caso $\partial f(x_0) = {\triangledown f(x0)}$}.\\

		\textbf{Demostración.-}\; $\left.\Leftarrow\right]$  Si $f$ es diferenciable en $x_0$, entonces existe un único vector, el gradiente de $f$ en $x_0$, que satisface la definición de subgradiente. Esto se debe a que la función tangente lineal a $f$ en $x_0$ proporciona el mejor subestimador afín de $f$ en $x_0$. Por lo tanto, el subdiferencial $\partial f(x_0)$ es un conjunto unipuntual, y $\partial f(x_0) = {\triangledown f(x0)}$.\\

		Para demostrar esto de manera algebraica, consideremos la definición de subgradiente\footref{subgradiente}. Un vector $g$ es un subgradiente de $f$ en $x_0$ si satisface la desigualdad 
		$$f(x) \geq f(x_0) + g^T (x - x_0).$$ 
		para todo $x$ en el dominio de $f$. Si $f$ es diferenciable en $x_0$, entonces su gradiente 
		$$\triangledown f(x_0)$$
		satisface esta desigualdad, porque la función tangente lineal 
		$$f(x_0) + \triangledown f(x_0)^T (x - x_0)$$ 
		es la mejor
		\footnote{
		    "Mejor" aquí significa que la función tangente lineal es la función lineal que más se acerca a $f$ en el punto $x_0$.\\
		}
		aproximación lineal
		\footnote{
		    Cuando decimos que una función $f$ es diferenciable en un punto $x_0$, esto significa que la función tiene una derivada en ese punto. En términos geométricos, esto significa que existe una línea tangente a la gráfica de la función en el punto $(x_0, f(x_0))$. Esta línea tangente es la mejor aproximación lineal a la función cerca del punto $x_0$, en el sentido de que la distancia entre la función y la línea tangente es mínima en comparación con cualquier otra línea que pase por el punto $(x_0, f(x_0))$.

		    La ecuación de esta línea tangente es $f(x_0) + \triangledown f(x_0)^T (x - x_0)$, donde $\triangledown f(x_0)$ es el gradiente de $f$ en $x_0$. Esta ecuación nos da el valor de la línea tangente para cualquier valor de $x$ cerca de $x_0$.

		    De esta manera, si $f$ es diferenciable en $x_0$, entonces tiene una única mejor aproximación lineal en $x_0$, que es la función tangente lineal. Esto es lo que nos permite definir el subgradiente y el subdiferencial de $f$ en $x_0$.
		\label{diferenciable}}
		a $f$ en $x_0$. Por lo tanto, si $f$ es diferenciable en $x_0$, entonces el gradiente de $f$ en $x_0$ es el único vector que satisface la definición de subgradiente. En otras palabras, el gradiente de $f$ en $x_0$ es el único subgradiente de $f$ en $x_0$. Por lo que, el subdiferencial de $f$ en $x_0$, que es el conjunto de todos los subgradientes de $f$ en $x_0$, es un conjunto unipuntual que contiene solo el gradiente de $f$ en $x_0$. Así, 
		$$\partial f(x_0) = {\triangledown f(x0)}.$$\\

		$\left.\Rightarrow\right]$ Si el subdiferencial $\partial f(x_0)$ es un conjunto unipuntual, entonces existe un único subgradiente en $x_0$. Esto implica que la función tangente lineal a $f$ en $x_0$ es única, lo cual es posible sólo si $f$ es diferenciable en $x_0$.

		Para demostrar esto de manera algebraica, supongamos que $\partial f(x_0)$ es un conjunto unipuntual, es decir, 
		$$\partial f(x_0) = \{g\}$$ para algún vector $g$. Por la definición de subgradiente, sabemos que 
		$$f(x) \geq f(x_0) + g^T (x - x_0)$$ 
		para todo $x$ en el dominio de $f$. Ahora, si $f$ no fuera diferenciable en $x_0$, entonces existiría otro vector $h$ tal que 
		$$f(x) \geq f(x_0) + h^T (x - x_0)$$
		para todo $x$ en el dominio de $f$. Pero esto contradiría nuestra suposición de que $\partial f(x_0)$ es un conjunto unipuntual. Por lo tanto, $f$ debe ser diferenciable en $x_0$, y su gradiente en $x_0$ es el único subgradiente, es decir, $\triangledown f(x_0) = g$.\\

		Así, demostramos que $f$ es diferenciable en $x_0$ si y sólo si el subdiferencial $\partial f(x_0)$ es un conjunto unipuntual. En dicho caso, $\partial f(x_0) = {\triangledown f(x0)}$. $\blacksquare$\\\\


	    % ------------------- INCISO (c) ------------------
	    \item  \textbf{\boldmath Si $f$ y $g$ son dos funciones convexas, demuestra que $\partial (f+g)(x) = \partial (f)(x)+\partial (g)(x)$. ¿Se tiene la igualdad en todos los puntos del dominio? ¿sólo en los puntos del interior del dominio? ¿Es esencial la hipótesis de que las funciones sean convexas?.}\\

		\textbf{Demostración.-}\; Para que se entienda mejor, supongamos que $h = f + g$, y que $x$ está en el interior del dominio de $h$. Entonces, por la definición de subdiferencial\footref{subdiferencial}, para cualquier $v \in \partial h(x)$, tenemos que 
		$$h(y) \geq h(x) + v^T (y - x)$$ 
		para todo $y$ en el dominio de $h$. Ahora, si podemos descomponer $h$ en $f$ y $g$. De lo que nos da
		$$f(y) + g(y) \geq f(x) + g(x) + v^T (y - x)$$ 
		para todo $y$ en el dominio de $h$. Luego, supongamos que 
		$$u \in \partial f(x) \quad \text{y} \quad w \in \partial g(x).$$ 
		Entonces, por la definición de subdiferencial\footref{subdiferencial}, tenemos que 
		$$f(y) \geq f(x) + u^T (y - x) \quad \text{y}\quad g(y) \geq g(x) + w^T (y - x)\qquad (1)$$ 
		Sumando estas dos desigualdades nos da 
		$$f(y) + g(y) \geq f(x) + g(x) + (u + w)^T (y - x)$$ 
		para todo $y$ en el dominio de $h$. Por último, comparando esta desigualdad con $(1)$, vemos que si elegimos $v = u + w$, entonces la desigualdad se mantiene. Por lo tanto, 
		$$u + w \in \partial h(x).$$ 
		Es decir, 
		$$\partial f(x) + \partial g(x) \subseteq \partial (f+g)(x).$$\\

		Demostremos la inclusión. Es decir, 
		$$\partial (f+g)(x) \subseteq \partial f(x) + \partial g(x).$$ 
		Supongamos que 
		$$v \in \partial (f+g)(x).$$
		Por la definición de subgradiente\footref{subgradiente}, esto significa que para todo $y$ en el dominio de $f+g$, tenemos que 
		$$(f+g)(y) \geq (f+g)(x) + v^T (y - x).$$

		Descomponiendo $(f+g)(y)$ y $(f+g)(x)$ en $f$ y $g$, nos da 
		$$f(y) + g(y) \geq f(x) + g(x) + v^T (y - x) \qquad (2)$$ 
		para todo $y$ en el dominio de $f+g$. Ahora, supongamos que existen 
		$$u \in \partial f(x) \quad \text{y} \quad w \in \partial g(x).$$
		Entonces, por la definición de subgradiente\footref{subgradiente} tenemos que 
		$$f(y) \geq f(x) + u^T (y - x) \quad \text{y}\quad g(y) \geq g(x) + w^T (y - x)$$ 
		para todo $y$ en el dominio de $f+g$. Sumamos estas dos desigualdades, nos da 
		$$f(y) + g(y) \geq f(x) + g(x) + (u + w)^T (y - x)$$ 
		para todo $y$ en el dominio de $f+g$. Comparamos esta desigualdad con $(2)$, vemos que si elegimos 
		$$v = u + w,$$ 
		entonces la desigualdad se mantiene. Por lo tanto, 
		$$v \in \partial f(x) + \partial g(x);$$
		es decir, 
		$$\partial (f+g)(x) \subseteq \partial f(x) + \partial g(x).$$\\
		De esta manera, demostramos que 
		$$\partial (f+g)(x) = \partial f(x) + \partial g(x).\; \blacksquare$$

		En cuanto a la primera pregunta, no necesariamente se tiene la igualdad en todos los puntos del dominio. La igualdad se mantiene en los puntos del interior del dominio donde ambas funciones son diferenciables. En los puntos de la frontera del dominio o en los puntos donde las funciones no son diferenciables, la igualdad puede no mantenerse.\\

		En cuanto a la segunda pregunta, sí es esencial la hipótesis de que las funciones sean convexas. La convexidad de las funciones garantiza la existencia de subgradientes en cada punto del interior de su dominio. Si las funciones no son convexas, entonces pueden no existir subgradientes en algunos puntos, y la igualdad  $\partial (f+g)(x) = \partial f(x) + \partial g(x)$ puede no mantenerse.\\\\

	\end{enumerate}


    % ------------------- EJERCICIOS 2 ------------------
    \item \textbf{\boldmath Sea $f : \mathbb{R}^n \to \mathbb{R}$ una función convexa y $K$ un compacto contenido en el interior del dominio de $f$. Demuestra que $f$ es Lipschitz en $K$, es decir, que existe una constante $L>0$ tal que 
    $$|f(x) - f(y)| \leq L\|x- y\|,$$ 
    para todo $x, y \in K$.}\\

	\textbf{Demostración.-}\; Dado que $K$ es compacto, sabemos que es cerrado y acotado. Por lo tanto, existe un $M > 0$ tal que 
	\begin{center}
	    $\|x\| \leq M$ para todo $x \in K$. 
	\end{center}

	Consideremos dos puntos arbitrarios $x, y \in K$. Podemos aplicar la definición de función convexa
	\footnote{
	Una función $f$ es convexa si para todo $x, y$ en su dominio y para todo $t \in [0,1]$, se cumple que $f(tx + (1-t)y) \leq tf(x) + (1-t)f(y).$\\
	\label{funcionConvexa}}
	para obtener
	$$f(y) \geq f(x) + \nabla f(x)^T (y-x)
	\footnote{
	    Se asume que la función $f$ es diferenciable. Esto se evidencia para el uso del gradiente $\nabla f(x)$ en la desigualdad de Jensen.
	}
	.$$
	Esto implica que
	$$f(y) - f(x) \geq \nabla f(x)^T (y-x).$$
	Tomando el valor absoluto de ambos lados, obtenemos
	$$|f(y) - f(x)| \geq |\nabla f(x)^T (y-x)|.$$
	Usando la desigualdad de Cauchy-Schwarz
	\footnote{
	    La \textbf{desigualdad de Cauchy-Schwarz} establece que para cualquier espacio vectorial complejo con un producto escalar, y para cualquier par de vectores $x$ y $y$ en ese espacio, se cumple la siguiente desigualdad: $|x \cdot y| \leq \|x\| \|y\|,$ donde $x \cdot y$ denota el producto escalar de $x$ y $y$, y $\|x\|$ y $\|y\|$ son las normas (o longitudes) de los vectores $x$ y $y$, respectivamente. La igualdad se cumple si y solo si los vectores $x$ y $y$ son linealmente dependientes. Esto significa que uno de los vectores puede ser escrito como un múltiplo escalar del otro.
	}
	,
	$$|\nabla f(x)^T (y-x)| \leq \|\nabla f(x)\| \|y-x\|.$$
	Por lo tanto,
	$$|f(y) - f(x)| \geq \|\nabla f(x)\| \|y-x\|.$$

	Luego, necesitamos demostrar que $\|\nabla f(x)\|$ es acotado en $K$. Dado que $f$ es convexa, su gradiente satisface la condición de monotonía del gradiente. Además, dado que $K$ es compacto, el gradiente de $f$ debe ser acotado en $K$. Por lo tanto, existe una constante $L > 0$ tal que
	\begin{center}
	    $\|\nabla f(x)\| \leq L$ para todo $x \in K$.
	\end{center}

	Finalmente, combinando estas dos desigualdades, obtenemos
	$$|f(y) - f(x)| \leq L \|y-x\|.$$
	Así, la función $f$ es Lipschitz en $K$ con constante de Lipschitz $L$. $\blacksquare$\\\\


    % ------------------- EJERCICIOS 3 ------------------
    \item \textbf{\boldmath Sean $f, g : \mathbb{R}^nn \to \mathbb{R}$ dos funciones convexas cuyo dominio es $\mathbb{R}^n$. Se define su convolución infima (\textit{infimal convolution}) como
    $$h(x) = (f\diamond g)(x) :=  \inf{f(y) + g(z) : y + z = x}.$$
    Demuestra que $h(y)$. ¿Es suficiente esta identidad para demostrar que si $f$ y $g$ son convexas, entonces $h$ es convexa?.}\\
	
	\textbf{Demostración.-}\; Necesitaremos demostrar que para cualquier par de puntos $x, y \in \mathbb{R}^n$ y cualquier $t \in [0,1]$, se cumple que
	$$h(tx + (1-t)y) \leq t h(x) + (1-t) h(y).$$

	Dado que 
	$$h(x) = (f \diamond g)(x) = \inf\{f(u) + g(v) : u + v = x\},$$ 
	podemos elegir $(u_1, v_1)$ y $(u_2, v_2)$ tal que 
	$$
	\begin{array}{rclcrcl}
	    u_1 + v_1 &=& x, & & h(x) &=& f(u_1) + g(v_1)\\\\
	    u_2 + v_2 &=& y, & & h(y) &=& f(u_2) + g(v_2).
	\end{array}
	$$

	Ahora, consideremos 
	$$z = tx + (1-t)y = tu_1 + (1-t)u_2 + tv_1 + (1-t)v_2.$$ 

	Donde, notamos que 
	$$tu_1 + (1-t)u_2 \quad \text{y}\quad tv_1 + (1-t)v_2$$ 
	son combinaciones convexas de $u_1, u_2$ y $v_1, v_2$ respectivamente. Como $f$ y $g$ son convexas, tenemos que
	$$f(tu_1 + (1-t)u_2) \leq t f(u_1) + (1-t) f(u_2),$$
	$$g(tv_1 + (1-t)v_2) \leq t g(v_1) + (1-t) g(v_2).$$

	Sumando estas dos desigualdades, obtenemos
	$$
	\begin{array}{rcl}
	    f(tu_1 + (1-t)u_2) + g(tv_1 + (1-t)v_2) &\leq& t f(u_1) + (1-t) f(u_2) + t g(v_1) + (1-t) g(v_2)\\\\
						    &=& t h(x) + (1-t) h(y).
	\end{array}
	$$

	Dado que el lado izquierdo de esta desigualdad es una cota superior para $h(z)$, tenemos que
	$$h(z) = h(tx + (1-t)y) \leq t h(x) + (1-t) h(y),$$
	lo que demuestra que $h$ es convexa. Por lo tanto, si $f$ y $g$ son convexas, entonces su convolución ínfima $h$ también es convexa. Esta identidad es suficiente para demostrar la convexidad de $h$. Ya que, si $f$ es convexa, entonces para cualquier $x, y$ en el dominio de $f$ y cualquier $t$ en el intervalo $[0,1]$, se cumple que
	$$f(tx + (1-t)y) \leq t f(x) + (1-t) f(y).$$

	En la demostración dada, mostramos que para cualquier par de puntos $x, y$ en el dominio de $h$ y cualquier $t$ en el intervalo $[0,1]$, se cumple que
	$$h(tx + (1-t)y) \leq t h(x) + (1-t) h(y).$$

	Por lo tanto, hemos demostrado que $h$ satisface la definición de una función convexa, lo cual es suficiente para demostrar que $h$ es convexa. En otras palabras, hemos demostrado que si $f$ y $g$ son convexas, entonces su convolución ínfima $h$ también es convexa. $\blacksquare$ \\\\


    \begin{comment}
    % ------------------- EJERCICIOS 4 ------------------
    \item  \textbf{\boldmath Sea ($\Omega, A, \mu$) un espacio de probabilidad, $g : \Omega \to \mathbb{R}$ $\mu$-integrable, $f$ convexa (no necesariamente diferenciable) con $Dom(f) = \mathbb{R}$. Demuestra que
	$$f\left(\int_{\Omega} g \;d\mu\right) \leq \int_{\Omega} (f\circ g) \; d\mu.$$
	En este caso, hemos supuesto que el dominio de $f$ es todo $\mathbb{R}$, ¿Qué debemos pedirle como mínimo al dominio de $f$ para que el resultado siga siendo cierto?.}\\
    \end{comment}

\end{enumerate}

% Entrega B
%\begin{center}
\textbf{CONVEXIDAD Y OPTIMIZACIÓN}

\textbf{\Large ENTREGA B}

\textbf{ \textbf{Christian Limbert Paredes Aguilera}}
\end{center}
\begin{center}
    Latex Source: \url{https://github.com/soyfode/matematicas/blob/master/investmat/src/convexOpt/tareas/entregaB.tex}
\end{center}

\line(1,0){400}

\begin{enumerate}[\bfseries \text{Ejercicio} 1.]

\setcounter{enumi}{2}

\begin{comment}
    % -------------------- Ejercicio 1 -------------------- %
    \item \textbf{\boldmath Se considera el problemas
    $$
    \begin{array}{ll}
	\text{minimiza} & x^2+1\\
	\text{sujeto a} & (x-2)(x-4)\leq 0,
    \end{array}
    $$
    con $x\in \mathbb{R}$.}
    \begin{enumerate}[\bfseries (a)]

	% ---------- (a)
	\item \textbf{\boldmath Describe el conjunto accesible, el valor óptimo y las solución optima.}\\
	

	    \textbf{solución:} El conjunto accesible, esta formado por todos los valores de $x$ que satisfacen
	    $$ (x-2)(x-4)\leq 0.$$
	    Esto significa que $x$ deber estar en el intervalo $[2,4]$. Ya que, si $x<2$ o $x>4$, entonces $(x-2)(x-4)>0$.\\
	    
	    Para calcular el valor óptimo del problema de optimización dado, necesitamos encontrar el valor mínimo de la función objetivo $x^2+1$ sujeto a la restricción $(x-2)(x-4)\leq 0$.\\

	    La función objetivo $x^2+1$ es una función cuadrática que es siempre creciente en el intervalo $[2,4]$. Por lo tanto, el valor mínimo de la función objetivo en este intervalo se alcanza en el extremo inferior del intervalo. Es decir, la solución optima es 
	    $$x=2.$$

	    Sustituyendo $x=2$ en la función objetivo obtenemos el valor óptimo:
	    $$x^2+1 = 2^2+1 = 5.$$

	    Por lo tanto, el valor óptimo del problema de optimización dado es $5$.\\\\


	%  ---------- (b)
	\item \textbf{\boldmath Escribe la función lagrangiana $\mathcal{L}(x,\lambda)$ y dibuja las funciones $x\mapsto \mathcal{L}(x,\lambda)$ para algunos valores positivos de $\lambda$. Verifica que $p^*\geq \inf_x\mathcal{L}(x,\lambda)$ para $\lambda\geq 0$. Obtén y dibuja la gráfica de la función dual de lagrange $g$.}\\

	    \textbf{Solución:} Sean, la función objetivo  $f(x) = x^2 + 1$ y la restricción $g(x) = (x-2)(x-4) \leq 0$. Ahora, la función Lagrangiana se define como:
	    $$\mathcal{L}(x,\lambda) = f(x)-\lambda g(x).$$

	    Sustituyendo $f(x)$ y $g(x)$ en la ecuación anterior, obtenemos la función Lagrangiana para este problema:
	    $$\mathcal{L}(x,\lambda) = \left(x^2+1\right) - \lambda[(x-2)(x-4)].$$

	    El dibujo de la función $x \mapsto \mathcal{L}(x,\lambda)$ para algunos valores positivos de $\lambda$ es:
	    \begin{center}
		\begin{tikzpicture}
		    \begin{axis}[scale=.7,draw opacity =.5,samples=100,smooth, 
		      axis x line=center, 
		      axis y line=center,
		      ylabel = {$\mathcal{L}(x,\lambda)$},
		      xlabel = {$x$},
		      xlabel style={below right},
		      ylabel style={above left},
		      label style={font=\tiny},
		      tick label style={font=\tiny},
		      enlargelimits=upper,
		      legend pos=outer north east] 
		      \addplot[yellow,opacity=1,domain=2:4]{x^2 + 1 - .1*(x-2)*(x-4)};
		      \addlegendentry{$\lambda=0.1$}
		      \addplot[red,opacity=1,domain=2:4]{x^2 + 1 - .5*(x-2)*(x-4)};
		      \addlegendentry{$\lambda=0.5$}
		      \addplot[green,opacity=1,domain=2:4]{x^2 + 1 - 1*(x-2)*(x-4)};
		      \addlegendentry{$\lambda=1$}
		      \addplot[gray,opacity=1,domain=2:4]{x^2 + 1 - 1.5*(x-2)*(x-4)};
		      \addlegendentry{$\lambda=1.5$}
		      \addplot[blue,opacity=1,domain=2:4]{x^2 + 1 - 2*(x-2)*(x-4)};
		      \addlegendentry{$\lambda=2$}
		    \end{axis}
		\end{tikzpicture}
	    \end{center}
	    \vspace{.5cm}

	    Ahora, para encontrar el mínimo de $\mathcal{L}(x,\lambda)$ con respecto a $x$, tomamos la derivada de $\mathcal{L}$ con respecto a $x$ e igualamos a cero, obteniendo 
	    $$x = \frac{3\lambda}{\lambda - 2}.$$ 

	    Sin embargo, debido a la restricción $(x-2)(x-4) \leq 0$, $x$ debe estar en el intervalo $[2,4]$. Por lo tanto, el mínimo de $\mathcal{L}(x,\lambda)$ en el intervalo $[2,4]$ se alcanza en uno de los extremos del intervalo, es decir, en 
	    $$x=2\quad \text{o}\quad x=4.$$

	    Sustituyendo $x=2$ y $x=4$ en $\mathcal{L}(x,\lambda)$ obtenemos 
	    $$\mathcal{L}(2,\lambda) = 5\quad \text{y}\quad \mathcal{L}(4,\lambda) = 17.$$ 
	    Por lo tanto, el valor mínimo de $\mathcal{L}(x,\lambda)$ para $\lambda \leq 0$ y $x \in [2,4]$ es 5, que se alcanza en $x=2$.\\

	    Finalmente, el valor óptimo del problema de optimización original, $p^*$, es el valor mínimo de la función objetivo $x^2 + 1$ sujeto a la restricción $(x-2)(x-4) \leq 0$, que también es 5. Por lo tanto, podemos verificar que 
	    $$p^* = 5 \geq \inf_x \mathcal{L}(x,\lambda) = 5$$ 

	    para $\lambda \leq 0$, lo que verifica la desigualdad. Esto es consistente con la teoría de la optimización, que establece que el valor óptimo de un problema de optimización siempre es mayor o igual que el valor mínimo de su función Lagrangiana correspondiente para cualquier $\lambda \leq 0$.\\




	    La función dual de Lagrange, también conocida como función dual, se obtiene al minimizar la función Lagrangiana con respecto a las variables primales (en este caso, $x$) y luego maximizar con respecto a las variables duales (en este caso, $\lambda$). En otras palabras, la función dual $g(\lambda)$ se define como:
	    $$g(\lambda) = \inf_x \mathcal{L}(x,\lambda)$$

	    donde $\mathcal{L}(x,\lambda)$ es la función Lagrangiana. Para este problema, la función Lagrangiana es:
	    $$\mathcal{L}(x,\lambda) = x^2 + 1 - \lambda[(x-2)(x-4)]$$

	    Para encontrar $g(\lambda)$, primero minimizamos $\mathcal{L}(x,\lambda)$ con respecto a $x$. Como hemos resuelto anteriormente, el mínimo de $\mathcal{L}(x,\lambda)$ en el intervalo $[2,4]$ se alcanza en $x=2$ y su valor es 5. Por lo tanto, la función dual de Lagrange para este problema es una constante:
	    $$g(\lambda) = 5$$

	    para $\lambda \leq 0$. Esto significa que el valor óptimo del problema dual es $5$, que es igual al valor óptimo del problema primal. Esto es consistente con la teoría de la optimización, que establece que bajo ciertas condiciones (llamadas condiciones de Slater), el valor óptimo del problema primal es igual al valor óptimo del problema dual. Esto se conoce como dualidad fuerte.



	% ---------- (c)
	\item \textbf{\boldmath Describe el problema dual y comprueba que se trata de un problema de maximización cóncavo. Encuentra el valor dual óptimo $d^+$ y la solución dual óptima $\lambda^*$ ¿Se verifica la dualidad fuerte? Representa el conjunto $G\subset \mathbb{R}^2$}.\\

	    \textbf{Solución:}

    \end{enumerate}

    % -------------------- Ejercicio 2 -------------------- %
    \item \textbf{\boldmath ¿Cuáles son las condiciones de Slater asociadas al problema lineal
    $$
    \begin{array}{ll}
	\text{minimiza} & x^Tx\\
	\text{sujeto a} & Ax=b?
    \end{array}
    $$
    Se sabe que las condiciones de Slater implican que se verifica la dualidad fuerte. Describe esta situación para el problema lineal planteado. ¿Es posible aplicar las condiciones de Slater al problema 1?}\\

	\textbf{Solución:}

\end{comment}


    % -------------------- Ejercicio 3 -------------------- %
    \item \textbf{\boldmath Considera el problema de optimización
    $$
    \begin{array}{ll}
	\text{minimiza} & e^{-x}\\
	\text{sujeto a} & x^2/y\leq 0,\\
    \end{array}
    $$
    con variables $x$ e $y$, y dominio $D=\left\{(x,y):y>0\right\}$.}
    \begin{enumerate}[\bfseries (a)]

	% ---------- (a)
	\item \textbf{\boldmath Comprueba que es un problema de optimización convexa. Encuentra el valor óptimo}\\

	    \textbf{Solución:} Para demostrar que el problema de optimización dado es convexo y encontrar el valor óptimo, necesitamos verificar dos cosas: la convexidad de la función objetivo y la convexidad de las restricciones.\\

	    \textit{Convexidad de la función objetivo.-} La función objetivo es 
	    $$f(x) = e^{-x}.$$ 
	    Para demostrar que es convexa, necesitamos mostrar que su segunda derivada es siempre no negativa 
	    \footnote{
		Una función $f: \mathbb{R} \rightarrow \mathbb{R}$ es convexa en un intervalo $I$ si para todo $x \in I$, la segunda derivada $f''(x)$ existe y $f''(x) \geq 0$. En otras palabras, si la segunda derivada de una función es siempre no negativa en su dominio, entonces la función es convexa en ese dominio. Esto se debe a que la segunda derivada de una función mide la curvatura de la función, y una curvatura no negativa implica que la función es convexa.
	    }. 

	    Empecemos calculando la primera derivada de $f(x)$ que se obtiene utilizando la regla de la cadena para la derivación. La derivada de $e^u$ con respecto a $u$ es $e^u$, y la derivada de $-x$ con respecto a $x$ es $-1$. Por lo tanto, la primera derivada de $f(x)$ es:
	    $$f'(x) = \frac{d}{dx} e^{-x} = -e^{-x}.$$

	    La segunda derivada de $f(x)$ se obtiene derivando $f'(x)$ con respecto a $x$. Nuevamente, utilizamos la regla de la cadena. La derivada de $e^u$ con respecto a $u$ es $e^u$, y la derivada de $-x$ con respecto a $x$ es $-1$. Por lo tanto, la segunda derivada de $f(x)$ es:
	    $$f''(x) = \frac{d}{dx} (-e^{-x}) = e^{-x}.$$

	    Sabiendo que una función es convexa si su segunda derivada es siempre no negativa. La segunda derivada de $f(x)$ es $f''(x) = e^{-x}$, que es siempre positiva para todo $x$ en el dominio de los números reales, ya que la función exponencial $e^{-x}$ es siempre positiva. Por lo tanto, la función objetivo $f(x) = e^{-x}$ es convexa.\\

	    \textit{Convexidad de las restricciones.-} Sea la restricción,
	    $$g(x,y) = x^2/y \leq 0.$$ 
	     Ya que $y > 0$ (según $D=\left\{(x,y):y>0\right\}$), esta restricción implica que 
	     $$x = 0.$$ 

	    Un conjunto que consiste en un solo punto (en este caso, el punto $x = 0$) es un conjunto convexo. Esto se debe a que para cualquier par de puntos en el conjunto (que en este caso son el mismo punto $x = 0$), el segmento de línea que los conecta también está en el conjunto. Por lo tanto, la restricción $g(x,y) = x^2/y \leq 0$ define un conjunto convexo.\\

	    Así, el problema de optimización dado es convexo.\\

	    Para encontrar el valor óptimo del problema de optimización dado, debemos evaluar la función objetivo en el punto que satisface la restricción. Es decir, debemos evaluar la función objetivo en $x = 0$. Haciendo esto, obtenemos:
	    $$f(0) = e^{-0} = e^0 = 1.$$

	    Por lo tanto, el valor óptimo del problema de optimización dado es $1$.

	    Este resultado se obtiene al considerar que la función objetivo $f(x)$ es decreciente
	    \footnote{
		La función objetivo en este problema de optimización es $f(x) = e^{-x}$. Esta es una función exponencial decreciente para $x \geq 0$, lo que significa que a medida que $x$ aumenta, $f(x)$ disminuye.\\
	    }
	    para $x \geq 0$ y la restricción implica que $x = 0$. Por lo tanto, el valor mínimo de $f(x)$, llamémoslo $p^*$ en el dominio dado es 
	    $$p^*=f(0) = e^0 = 1.$$\\


	% ---------- (b)
	\item \textbf{\boldmath Establece el problema dual y encuentra la solución óptima dual $\lambda^*$ y el valor óptimo $d^*$. ¿Cuál es el salto de dualidad $p^*-d^*$? Representa el conjunto $G\subseteq \mathbb{R}^2$.}\\

	    \textbf{Solución:} Veamos que el problema de optimización primal es:
	    $$
	    \left\{
		\begin{array}{ll}
			\text{minimiza} & e^{-x}\\
			\text{sujeto a} & x^2/y\leq 0,\\
		\end{array}
	    \right.
	    $$

	    con variables $x$ e $y$, y dominio $D=\left\{(x,y):y>0\right\}$.\\

	    Para formular el problema dual, primero expresamos el problema primal en su forma estándar. Sean la función objetivo $f_0(x) = e^{-x}$ y la restricción $f_1(x,y) = x^2/y \leq 0$. Entonces, el problema primal podemos escribir como:
	    $$
	    \left\{
		\begin{array}{ll}
			\text{minimiza} & f_0(x)\\
			\text{sujeto a} & f_1(x,y) \leq 0,\\
		\end{array}
	    \right.
	    $$

	    Luego, procedamos a formular el problema dual. Para ello, necesitamos definir la función Lagrangiana, que incorpora las restricciones del problema primal en la función objetivo a través de las variables duales. En este caso, la función Lagrangiana se define como:
	    $$
	    L(x,y,\lambda) = f_0(x) + \lambda f_1(x,y) = e^{-x} + \lambda \left(\frac{x^2}{y}\right)
	    $$

	    donde $\lambda$ es la variable dual asociada a la restricción $f_1(x,y) \leq 0$.\\

	    La función dual de Lagrange, que es la función objetivo del problema dual, se obtiene al minimizar la función Lagrangiana con respecto a las variables primales. Es decir, buscamos el valor mínimo de la función Lagrangiana sobre todas las posibles elecciones de las variables primales $x$ e $y$. En decir, por el método de los multiplicadores de Lagrange, tenemos:
	    $$
	     \inf_{x,y} L(x,y,\lambda) = \inf_{x,y} \left[e^{-x} + \lambda \left(\frac{x^2}{y}\right)\right]=g(\lambda).
	    $$

	    Para resolverlo, tomamos las derivadas parciales de $L$ con respecto a $x$ e $y$, 
	    $$
	    \left\{
		\begin{array}{rcl}
		    \dfrac{\partial L}{\partial x} &=& -e^{-x} + 2\lambda \dfrac{x}{y}\\\\
		    \dfrac{\partial L}{\partial y} &=& -\lambda \dfrac{x^2}{y^2}.
		\end{array}
	    \right.
	    $$

	    Igualando estas derivadas a cero,
	    $$
	    \left\{
		\begin{array}{rcl}
		    -e^{-x} + 2\lambda \frac{x}{y} &=& 0\\\\
		    -\lambda \frac{x^2}{y^2} &=& 0,
		\end{array}
	    \right.
	    $$
	    Y sabiendo que $y > 0$, obtenemos, 
	    $$x = 0.$$
	    Por lo tanto, la función dual de Lagrange es:
	    $$
	    g(\lambda) = \inf_{x,y} L(x,y,\lambda) = \inf_{x,y} \left[e^{-x} + \lambda \left(\frac{x^2}{y}\right)\right] = e^{-0} + \lambda \left(\frac{0^2}{y}\right) = 1.
	    $$
	    para $\lambda \geq 0$.\\

	    Así, dado que $g(\lambda)$ es constante e igual a $1$ para todo $\lambda \geq 0$, la solución óptima dual $\lambda^*$ puede ser cualquier número real no negativo y el valor óptimo dual 
	    $$d^* = g(\lambda^*) = 1.$$

	    Finalmente, puedes calcular la brecha de dualidad $p^*-d^*$, donde $p^*$ es el valor óptimo del problema primal. Dado que $p^* = d^* = 1$, la brecha de dualidad es $0$.\\\\

	% ---------- (c)
	\item \textbf{\boldmath ¿Se cumple la condición de Slater para este problema?.}\\

	    \textbf{Solución:} La condición de Slater es una regla que se utiliza en optimización convexa. Esta regla dice que si hay un punto que cumple todas las restricciones del problema de optimización de manera estricta, entonces podemos estar seguros de que la solución óptima del problema original (llamado problema primal) y la solución óptima del problema asociado (llamado problema dual) serán iguales. Esto se conoce como dualidad fuerte.\\

	    En este caso, el problema de optimización tiene una restricción que dice que $x^2/y\leq 0$ y que $y>0$. Esto significa que $x=0$ para cualquier par $(x,y)$ que cumpla $y>0$. Por lo tanto, el punto $(0,1)$ cumple la restricción de manera estricta, ya que $0^2/1=0$ es estrictamente menor que 0.\\

	    Por lo tanto, la condición de Slater se cumple para este problema de optimización. Esto significa que la solución óptima del problema original y la solución óptima del problema dual son iguales, lo cual es consistente con los cálculos anteriores donde encontramos que ambas soluciones óptimas son 1. Es decir, no hay diferencia o brecha entre las soluciones óptimas del problema original y del problema dual.\\\\


    \end{enumerate}

    % -------------------- Ejercicio 4 -------------------- %
    \item \textbf{\boldmath Considera el problema:
    $$
    \begin{array}{ll}
	\text{minimiza} & f_0(x)\\
	\text{sujeto a} & f_i(x)\leq 0, \; i=1,\dots,m\\
    \end{array}
    $$
    donde las funciones $f_i:\mathbb{R}^n\to \mathbb{R}$ son diferenciables y convexas. Muestra que la función
    $$\phi(x) = f_0(x)+\alpha \sum_{i=1}^m \max\left\{0,f_i(x)\right\}^2,$$
    es convexa (cuando $\alpha>0)$. Si $\tilde{x}$ minimiza $\phi$, muestra cómo obtener, a partir de $\tilde{x}$, un punto accesible para el problema dual. Encuentra la cota inferior correspondiente al problema primal.}\\

	\textbf{Solución:} Para demostrar que la función $$\phi(x) = f_0(x)+\alpha \sum_{i=1}^m \max\left\{0,f_i(x)\right\}^2$$ es convexa cuando $\alpha>0$, necesitamos mostrar que para cualquier $x, y \in \mathbb{R}^n$ y cualquier $t \in [0,1]$, se cumple que:
$$\phi(tx + (1-t)y) \leq t\phi(x) + (1-t)\phi(y).$$

Sea $f_0(x)$  convexa. Entonces, sabemos que:
$$f_0(tx + (1-t)y) \leq tf_0(x) + (1-t)f_0(y).$$

Además, la función $\max\{0, f_i(x)\}^2$ es convexa porque es la composición de una función convexa no decreciente $\left(x^2\right)$ y una función convexa $\left(\max\{0, f_i(x)\}\right)$. Por lo tanto, para cada $i$:
$$\max\{0, f_i(tx + (1-t)y)\}^2 \leq t\max\{0, f_i(x)\}^2 + (1-t)\max\{0, f_i(y)\}^2.$$

Sumando estas desigualdades para todos los $i$ y multiplicando por $\alpha$, obtenemos:
$$\alpha \sum_{i=1}^m \max\{0, f_i(tx + (1-t)y)\}^2 \leq \alpha \sum_{i=1}^m \left[ t\max\{0, f_i(x)\}^2 + (1-t)\max\{0, f_i(y)\}^2 \right].$$

Finalmente, sumando las dos desigualdades anteriores, obtenemos la desigualdad deseada:
$$
\begin{array}{rcl}
    \phi(tx + (1-t)y) &=& \displaystyle f_0(tx + (1-t)y) + \alpha \sum_{i=1}^m \max\{0, f_i(tx + (1-t)y)\}^2\\\\
		      &\leq& \displaystyle f_0(x) + (1-t)f_0(y) + \alpha \sum_{i=1}^m \left[ t\max\{0, f_i(x)\}^2 + (1-t)\max\{0, f_i(y)\}^2 \right]\\\\
		      &=& \displaystyle t\left[ f_0(x) + \alpha \sum_{i=1}^m \max\{0, f_i(x)\}^2 \right] + (1-t)\left[ f_0(y) + \alpha \sum_{i=1}^m \max\{0, f_i(y)\}^2 \right]\\\\
		      &=& t\phi(x) + (1-t)\phi(y).
\end{array}
$$

Por lo tanto, $\phi(x)$ es convexa cuando $\alpha > 0$.\\\\


    % -------------------- Ejercicio 5 -------------------- %
    \item \textbf{\boldmath Minimizando una función cuadrática. Considera el problema de minimizar una función cuadrática:
    $$\text{minimiza} \; f(x)=(1/2)x^TPx+q^Tx+r,$$
    donde $P$ es una matriz simétrica (no necesariamente en $S^n_{+}$).}\\

    \begin{enumerate}[(a)]

	% ---------- (a)
	\item \textbf{\boldmath Muestra que si $P \nsucceq 0$ (es decir, la función no es convexa) entonces el problema no está acotado inferiormente.}\\

	    \textbf{Solución:} Vamos a demostrar que si la matriz $P$ de una función cuadrática no es semidefinida positiva (es decir, $P\nsucceq0$), entonces la función no está acotada inferiormente. Para ello, vamos a utilizar un argumento de reducción al absurdo.\\

	    Supongamos que la función cuadrática 
	    $$f(x) = \frac{1}{2}x^TPx + q^Tx + r$$
	    está acotada inferiormente con $P \nsucceq 0$. Esto implicaría que existe un número real $M$ tal que 
	    $$f(x) \geq M$$ 
	    para todo $x$. Sin embargo, debido a que $P \nsucceq 0$, lo cual significa que $P$ no es una matriz semidefinida positiva, sabemos por definición que existe un vector $x$ tal que 
	    $$x^TPx < 0.$$

	    Si escalamos este vector por un escalar grande $\alpha$, la expresión $\alpha^2 x^TPx$ se vuelve arbitrariamente negativa. Esto se debe a que si $x^TPx < 0$, entonces $\alpha^2 x^TPx$ será aún más negativo a medida que $\alpha$ se haga más grande.\\

	    Elijamos $\alpha$ de manera que la función $f(\alpha x)$ sea menor que cualquier cota inferior dada. Sea, $f(\alpha x)$ 
	    $$f(\alpha x) = \frac{1}{2}\alpha^2 x^TPx + \alpha q^Tx + r.$$ 

	    Luego, hacemos que esta función sea menor que $-M$. Es decir, 
	    $$\frac{1}{2}\alpha^2 x^TPx + \alpha q^Tx + r < -M.$$ 

	    Multiplicamos ambos lados de la desigualdad por 2,
	    $$\alpha^2 x^TPx + 2\alpha q^Tx + 2r < -2M.$$ 

	    Reorganizando los términos,  
	    $$\alpha^2 x^TPx < -2M - 2\alpha q^Tx - 2r.$$ 

	    Esto garantiza que el término cuadrático en $f(\alpha x)$ sea suficientemente negativo para compensar los otros términos en $f(\alpha x)$ y hacer que toda la función sea menor que $-M$.

	    Por lo tanto, encontramos que 
	    $$f(\alpha x) = \frac{1}{2}\alpha^2 x^TPx + \alpha q^Tx + r < -M,$$

	    lo cual contradice nuestra suposición inicial de que $f(x)$ está acotada inferiormente. Por lo tanto, si $P \nsucceq 0$, la función cuadrática $f(x)$ no puede estar acotada inferiormente. \\\\

	% ---------- (b)
	\item \textbf{\boldmath Si suponemos que $P \succeq 0$, pero que la condición de optimalidad $Px^* = -q$ no tiene solución. Muestra que el problema vuelve a no estar acotado inferiormente.}\\

	    \textbf{Solución:} Si suponemos que $P \succeq 0$ (es decir, $P$ es semidefinida positiva), pero la condición de optimalidad $Px^* = -q$ no tiene solución, entonces podemos demostrar que el problema de minimización no está acotado inferiormente.\\

	    La condición de optimalidad $Px^* = -q$ proviene de tomar el gradiente de la función cuadrática $f(x)$ y establecerlo igual a cero. Si esta ecuación no tiene solución, significa que no existe un punto $x^*$ en el cual el gradiente de $f(x)$ sea cero. En otras palabras, no existe un mínimo local para la función.\\

	    Dado que $P \succeq 0$, la función cuadrática $f(x)$ es convexa. Para una función convexa, cualquier mínimo local es también un mínimo global. Entonces, si no existe un mínimo local para la función (es decir, no existe un punto en el que el gradiente de la función sea cero), la función no tiene un mínimo global. Esto se debe a que un mínimo global para una función es un punto para el cual la función no toma valores menores. En otras palabras, no existe ningún otro punto en el dominio de la función donde el valor de la función sea menor. Si no existe tal punto, entonces la función no tiene un mínimo global. Esto significa que la función no está acotada inferiormente, ya que una función está acotada inferiormente si existe un número real tal que para todo valor en el dominio, el valor de la función es mayor o igual a ese número. Si no existe tal número, entonces la función no está acotada inferiormente.\\

	    Por lo tanto, si suponemos que $P \succeq 0$, pero que la condición de optimalidad $Px^* = -q$ no tiene solución, entonces el problema de minimización no está acotado inferiormente. Esto concluye la demostración.\\\\

    \end{enumerate}

    % -------------------- Ejercicio 6 -------------------- %
    \item \textbf{\boldmath Considera el problema siguiente:
    $$\text{minimiza} \; f(x)=(1/2)\left(x_1^2+\gamma x^2_2\right),\; \gamma >0.$$} 
    \begin{enumerate}[\bfseries (a)]

	% ---------- (a)
	\item \textbf{\boldmath Fija $(a,b)\in \mathbb{R}^2$. Calcula el valor de $t$ que minimiza la función $f\left((a,b)-t\nabla f(a,b)\right)$ en términos de $a$ y $b$, claro.}\\ 

	    \textbf{Solución:} Teniendo en cuenta que estamos trabajando en $\mathbb{R}^2$ y que $\gamma > 0$:

	    Consideremos la función $$f: \mathbb{R}^2 \rightarrow \mathbb{R}$$ definida por $$f(x) = \frac{1}{2}(x_1^2 + \gamma x_2^2),$$ donde $x = (x_1, x_2)$ es un vector en $\mathbb{R}^2$ y $\gamma > 0$ es una constante.\\

	    Sabemos que el gradiente de una función es un vector que apunta en la dirección en la que la función aumenta más rápidamente. En otras palabras, si estamos en un punto en el espacio y queremos saber en qué dirección debemos movernos para aumentar el valor de la función lo más rápido posible, debemos movernos en la dirección del gradiente.\\

	    Para calcular el gradiente de esta función, tomamos las derivadas parciales de la función con respecto a cada una de las variables, $x_1$ y $x_2$. 

	    La derivada parcial de $f$ con respecto a $x_1$ es la derivada de $\frac{1}{2}x_1^2$ con respecto a $x_1$, manteniendo $x_2$ constante. Como la derivada de $x_1^2$ con respecto a $x_1$ es $2x_1$, la derivada parcial de $f$ con respecto a $x_1$ es 
	    $$x_1.$$ 
	    De manera similar, la derivada parcial de $f$ con respecto a $x_2$ es la derivada de $\frac{1}{2}\gamma x_2^2$ con respecto a $x_2$, manteniendo $x_1$ constante. Como la derivada de $\gamma x_2^2$ con respecto a $x_2$ es $2\gamma x_2$, la derivada parcial de $f$ con respecto a $x_2$ es 
	    $$\gamma x_2.$$

	    Por lo tanto, el gradiente de $f$ es 
	    $$(x_1, \gamma x_2).$$

	    Así, el gradiente de la función en el punto $(a,b)$ en $\mathbb{R}^2$ es 
	    $$\nabla f(a,b) = (a, \gamma b).$$

	    Esto significa que si estás en el punto $(a,b)$ en $\mathbb{R}^2$, la dirección en la que la función $f(x)$ aumenta más rápidamente es en la dirección del vector $(a, \gamma b)$.\\

	    Consideremos ahora un paso de gradiente desde el punto $(a,b)$ en la dirección opuesta al gradiente 
	    \footnote{
		Recordemos que una forma de minimizar una función es a través de un proceso llamado "descenso de gradiente". La idea es comenzar en un punto inicial y luego moverse repetidamente en la dirección que disminuye la función más rápidamente, que es la dirección opuesta al gradiente. Entonces, cuando decimos que el punto $(a,b)$ es "desplazado" por $-t\nabla f(a,b)$, lo que realmente queremos decir es que estamos moviendo el punto $(a,b)$ una pequeña cantidad en la dirección que disminuye la función más rápidamente. El tamaño de este movimiento está determinado por $t$, y la dirección de este movimiento está determinada por el gradiente $\nabla f(a,b)$.
		Por lo tanto, $(a,b)-t\nabla f(a,b)$ representa el nuevo punto después de dar un paso de gradiente de tamaño $t$ desde el punto $(a,b)$ en la dirección opuesta al gradiente.\\}. 
	    Este paso se puede representar como 
	    $$(a,b)-t\nabla f(a,b)=(a,b)-t(a,\gamma b),$$ 

	    donde $t$ es el valor que minimiza la función.\\

	    Vamos a calcular el valor de $t$ que minimiza la función $f\left((a,b)-t\nabla f(a,b)\right)$ 
	    \footnote{ El valor de $t$ se calcula en la función $f$ porque estamos buscando el valor que minimiza la función $f$ después de un paso de gradiente.}. 
	    Para hacer esto, primero necesitamos expresar la función $f$ en términos de $t$. Sustituyendo $(a,b)-t\nabla f(a,b)$ en $f$, obtenemos
	    $$
	    \begin{array}{rcl}
	    f\left((a,b)-t\nabla f(a,b)\right) &=& f\left((a,b)-t(a,\gamma b)\right)\\\\
					       &=& f\left((a-ta, b-t\gamma b)\right)\\\\
					       &=& \dfrac{1}{2}\left((a-ta)^2 + \gamma (b-t\gamma b)^2\right).
	    \end{array}
	    $$

	    Ahora, encontremos el valor de $t$ que minimiza la función.  Primero, tenemos la derivada de la función $f\left((a,b)-t\nabla f(a,b)\right)$ con respecto a $t$:
	    $$\frac{df}{dt} = -(a^2 - 2a^2t + \gamma b^2 - 2\gamma^2 b^2t).$$

	    Para encontrar el valor de $t$ que minimiza la función, necesitamos igualar esta derivada a cero. Esto nos da la ecuación:
	    $$-(a^2 - 2a^2t + \gamma b^2 - 2\gamma^2 b^2t) = 0.$$

	    Podemos simplificar esta ecuación distribuyendo el signo negativo a través de los términos en el paréntesis:
	    $$-a^2 + 2a^2t - \gamma b^2 + 2\gamma^2 b^2t = 0.$$

	    Luego, reorganizamos los términos para agrupar los términos que contienen $t$:
	    $$2a^2t + 2\gamma^2 b^2t = a^2 + \gamma b^2.$$

	    Factorizamos $t$ del lado izquierdo de la ecuación:
	    $$t(2a^2 + 2\gamma^2 b^2) = a^2 + \gamma b^2.$$

	    por lo tanto, para cualquier par de puntos $(a,b)\in \mathbb{R}^2$ y $\gamma>0$ el valor de $t$ que minimiza la función $f\left((a,b)-t\nabla f(a,b)\right)$ es
	    $$t = \frac{a^2 + \gamma b^2}{2a^2 + 2\gamma^2 b^2}.$$\\\\


	% ---------- (b)
	\item \textbf{\boldmath Crea un programa en MATLAB o Python tal que:}

	    \begin{itemize}

		\item \textbf{\boldmath Tenga como entrada: el parámetro $\gamma$, una tolerancia $\epsilon$, un número máximo de iteraciones $N$ y un punto inicial $\left(x_1^{(0)},x_2^{(0)}\right)$.}\\

		\item \textbf{Aplique el método del gradiente al problema, utilizando paso exacto (obtenido en el apartado (a)).}\\

		\item \textbf{\boldmath Que pare cuando llegue al número máximo de iteraciones o cuando la precisión sea menor que $\epsilon$.}\\

		\item \textbf{El problema debe devolver la última aproximación y el número de iteraciones utilizado, así como un mensaje que informe de los posibles fallos producidos y de la condición de parada utilizada.}\\

	    \end{itemize}


	% ---------- (c)
	\item \textbf{Si te sientes con ganas, añade la funcionalidad de que al terminar represente el conjunto inicial $S$, las curvas de nivel por las que pasa cada aproximación, destacando el punto aproximado con un dot.}\\

	    \textbf{Solución de (b) y (c):}
\begin{lstlisting}[language=Python]
import numpy as np
import matplotlib.pyplot as plt
from matplotlib.colors import LogNorm

def gradient_descent(gamma, epsilon, N, initial_point):
    # Item 1: La funcion toma como entrada el parametro gamma,
    # una tolerancia epsilon,  un numero maximo de iteraciones N 
    # y un punto inicial (x1(0), x2(0)).
    
    # Definimos la funcion y su gradiente
    f = lambda x: 0.5 * (x[0]**2 + gamma*x[1]**2)
    grad_f = lambda x: np.array([x[0], gamma*x[1]])

    # Inicializamos el punto y el numero de iteraciones
    x = np.array(initial_point)
    num_iterations = 0

    # Creamos una lista para almacenar todas las aproximaciones
    approximations = [x]

    # Bucle principal del metodo del gradiente
    while num_iterations < N:
        # Item 2: Aplicamos el metodo del gradiente al problema, 
	# utilizando paso exacto.
        # Calculamos el gradiente
        gradient = grad_f(x)

        # Calculamos el tamano del paso
        t = (x[0]**2+gamma*x[1]**2)/(2*x[0]**2+2*gamma**2*x[1]**2)

        # Actualizamos el punto
        x = x - t * gradient

        # Anadimos la nueva aproximacion a la lista
        approximations.append(x)

        # Item 3: El bucle continua hasta que se alcanza el numero
	# maximo de iteraciones o hasta que la norma del gradiente 
	# es menor que epsilon.
        if np.linalg.norm(gradient) < epsilon:
            print("El algoritmo  convergio con exito.")
            break

        # Actualizamos el numero de iteraciones
        num_iterations += 1

    # Item 4: La funcion devuelve la ultima aproximacion al minimo 
    # de la funcion y el numero de iteraciones que se utilizaron. 
    # Tambien se imprime un mensaje que informa de los posibles
    # fallos producidos y de la condicion de parada que se utilizo.
    if num_iterations == N:
        print("El algoritmo convergio en el numero maximo de iter.")
    else:
        print("El algoritmo convergio con exito.")

    # Creamos una cuadricula de puntos en el espacio 2D
    y = np.linspace(-10, 10, 400)
    x = np.linspace(-10, 10, 400)
    X, Y = np.meshgrid(x, y)

    # Calculamos los valores de la funcion en cada punto de la 
    # cuadricula
    Z = 0.5 * (X**2 + gamma*Y**2)

    # Creamos una figura y un conjunto de ejes en matplotlib
    fig, ax = plt.subplots()
    
    # Dibujamos las curvas de nivel de la funcion
    ax.contour(X, Y, Z, 
	       levels=np.logspace(0, 5, 35), 
	       norm=LogNorm(), 
	       cmap=plt.cm.jet)

    # Convertimos la lista de aproximaciones en un array de 
    # numpy para facilitar su manejo
    approximations = np.array(approximations)

    # Dibujamos los puntos aproximados
    ax.plot(approximations[:, 0], approximations[:, 1], 'ko')

    # Mostramos la figura
    plt.show()

    return x, num_iterations
\end{lstlisting}
    \end{enumerate}


\end{enumerate}



%------------------------------------------------------------------------------------------------
% -------------------- MODELIZACIÓN MATEMÁTICA EN LA INDUSTRIA ----------------------------------
%------------------------------------------------------------------------------------------------
% APUNTES TRATAMIENTO DE SEÑALES
%\chapter{La transformada de Haar}
\section{Ejemplo de la transformada de Haar}
Las señales son $f=\left(f_1,f_2,\ldots,f_N\right)\in \mathbb{R}^N$, para $N=2^k$, $k\in \mathbb{N}$ ($k$ máximo nivel).

El \textbf{soporte} de la señal es un derive. Donde las señales son no nula.

Supongamos que tenemos 
$$f=(1,7,4,3).$$
Con $2$ cómo el máximo nivel que puedo alcanzar (ya que $2^2$). Entonces, sacamos promedios entre $1$ y el $7$ y luego entre $4$ y $3$. Así,
$$A^1=(4,4,3.5,3.5).$$
Esto, nos da una aproximación de $f$ a primer nivel. Para recuperar la señal original tendremos que restar $A^1$ a $f$.
$$f-A^1=D^1=(-3,3,0.5,-0.5).$$
Por lo tanto,
$$f=A^1+D^1.$$

En vez de el promedio podríamos realizado una un promedio ponderado de la siguiente forma:

Dado $f$,
$$
\begin{array}{rcl}
    \dfrac{1-7}{2} &=& -3\\\\
    \dfrac{7-1}{2} &=& 3\\\\
    \dfrac{4-3}{2} &=& 0.5\\\\
    \dfrac{3-4}{2} &=& -0.5.
\end{array}
$$
Con lo que nos da el mismo resultado que antes.

Ahora, trataremos de encontrar una nueva aproximación con la primera aproximación $A^1$. Es decir,
$$A^2=(3.75,3.75,3.75,3.75).$$
Donde se suma el tercer y la primer componente de $A^1$ dividiendo este por 2. Lo mismo con el cuarto y el segundo componente.

Ahora restamos $A^1$ y $A^2$,
$$A_1-A_ 2= D^2=(0.25,0.25,-0.25,-0.25).$$
Por lo que,
$$A^1=A^2+D^2.$$

Concluimos se puede aproximar $f$ cómo:
$$f=A^2+D^2+D^1.$$

Vamos a reformular este proceso utilizando herramientas de álgebra lineal. Más concretamente con matrices ortogonales.


\section{Aspectos básicos de álgebra lineal}
\begin{itemize}
    \item Si $\left\{v_1,\ldots,v_m\right\}\subset \mathbb{R}^N$, la envoltura (span) lineal de dicho conjunto de vectores es $lin(v_1,\ldots,v_m)$.
    \item Si $v,W\in \mathbb{R}^N$, su producto escalar es $v\cdots w=v_1w_1+\cdots+v_N W_N$.
    \item Un sistema de vectores $\left\{v_1,\ldots,v_m\right\}$ es ortonormal si cada uno tiene norma $1$ y $v_i\cdot v_j=0$ si $i\neq j$. En ese caso, cada vector $u\in lin(v_1,\ldots,v_m)$ se expresa de forma única como $u=(u\cdot v_1)v_1+\cdots+(u\cdot v_m)v_m.$
    \item Dos subespacios $V,W\subset \mathbb{R}^N$ son ortogonales si cada elemento de $V$ es ortogonal a cada uno de $W$, y su suma es directa (y ortogonal), la cual denotamos por $V\oplus^{\perp} W$.
    \item si $v\in \mathbb{R}^N$ y $W$ es un subespacio de $\mathbb{R}^N$, la proyección ortogonal $w$ de $v$ sobre $W$ es el único vector $w\in W$ tal que $v-w$ es ortogonal a $W$ (y coincide con el de mínima distancia en $W$ a $v$).
    \item Una aplicación lineal $T:\mathbb{R}^N\to \mathbb{R}^N$ es una transformación ortogonal si $Tu\cdot Tv=u\cdot v$  para $u,v\in \mathbb{R}^N$) para $u,v\in \mathbb{R}^N$ arbitrarios (equivalente a $\|Tu\|=\|u\|$ para todo $u$). Tamién, que su expresión matricial en una base ortonormal sea una matriz $A$ ortogonal ($A\cdot A^t=I$).
\end{itemize}

Al tener un sistema de vectores ortonormales, tengo la base prefijada y tengamos las coordenadas necesarias. Lo que cambiará es la base que estoy tratando. 

\section{Wavelets de Haar}
Los \textbf{scaling} son
$$
\begin{array}{rcl}
    v_1^1 &=&\left(\dfrac{1}{\sqrt{2}},\dfrac{1}{\sqrt{2}},0,\ldots,0\right)\\\\
    v_2^1 &=& \left(0,0,\dfrac{1}{\sqrt{2}},\dfrac{1}{\sqrt{2}},0,\ldots,0\right)\\\\
    &\vdots&\\\\
    v_{N/2}^1 &=& \left(0,\ldots,0,\dfrac{1}{\sqrt{2}},\dfrac{1}{\sqrt{2}}\right).
\end{array}
$$
Desplazamos dos posiciones.

Los \textbf{wavelets} son
$$
\begin{array}{rcl}
	w_1^1 &=&\left(\dfrac{1}{\sqrt{2}},-\dfrac{1}{\sqrt{2}},0,\ldots,0\right)\\\\
	w_2^1 &=& \left(0,0,\dfrac{1}{\sqrt{2}},-\dfrac{1}{\sqrt{2}},0,\ldots,0\right)\\\\
	&\vdots&\\\\
	w_{N/2}^1 &=& \left(0,\ldots,0,\dfrac{1}{\sqrt{2}},-\dfrac{1}{\sqrt{2}}\right).
\end{array}
$$




\chapter{Detección de roturas en motores de inducción eléctrica}

\section{Muestreo y bandas de frecuencia}
\begin{itemize}
    \item Se realizan muestreos de dos segundos de duración del momento del arranque del motor.
    \item La frecuencia de muestreo es de $5000$ muestras por segundo.
\end{itemize}



% APUNTES TEJIDOS BIOLÓGICOS
%\chapter{Modelos matemáticos para el tratamiento térmico de tejidos biológicos}

\section{Tratamiento térmico de tejidos biológicos}
Consiste en una terapia ablativa. Es decir, que destruye un tejido, gracias a ampliar una energía que hace que haya un calentamiento y se altere este. Es un fin terapéutico. Por ejemplo, la ligadura de trompas, cuando sellamos y ligamos el tejido con calor, para que las trompas se liguen y sea un método conceptivo. El tratamiento será a más de $80\%$ grados.

\section{Ventajas de los modelos matemáticos}
\begin{itemize}
    \item Agiliza los experimentos.
    \item Reduce los costes.
    \item Se puede hacer pruebas.
\end{itemize}

\section{El artículo científico en ingeniería biomédica}
\begin{enumerate}
    \item Introducción.- Interés, hipótesis, objetivo. Estado del arte (Que se haya hecho hasta ahora).
    \item Materiales y métodos.- Formulación del modelo, mecanismos de validación, programas utilizados (Cómo lo hice). Debemos ser específicos para que cualquier pueda replicarlo.
    \item Resultados. 
    \item Discusión.
    \item Conclusiones.
\end{enumerate}

\section{Planteamiento general general de un modelo}
\begin{enumerate}
    \item Interés, hipótesis y objetivo. Interlocución Interdisciplinar.
	\begin{itemize}
	    \item La hipótesis es lo que uno cree que va a pasar.
	    \item El objetivo es lo que se va a hacer. (Realizar un modelo). La acción. 
	    \item Interés es el por qué se hace.
	\end{itemize}
    \item Geometría y abstracción.
    \item Ecuaciones de gobierno.
    \item Condiciones iniciales y de contorno.
    \item Características de los Materiales.
    \item Resolución computacional.
    \item Postprocesado.
    \item Discusión. Comparación con resultados experimentales.
	\begin{itemize}
	    \item Volver al principio y ver si se cumplieron el objetivo y la hipótesis.
	\end{itemize}
    \item Conclusión.
\end{enumerate}

\subsection{Interés, hipótesis y objetivo}
Revisemos el siguiente artículo: \href{https://doi.org/10.3109/02656736.2011.631076}{Relationship between roll-off occurrence and spatial distributionof dehydrated tissue during RF ablation with cooled electrodes}

\begin{itemize}
	\item Interés: Aunque muchos estudios han analizado el efecto térmico de un electrodo refrigerado y la relación exacta entre la distribución del tejido deshidratado y la ocurrencia del "roll-off", todavía no se caracterizo en detalle. Osea, nuestro interés es caracterizarlo en detalle.
	\item Hipótesis: El episodio de "roll-off" coincide aproximadamente con el momento en que el tejido alrededor del centro del electrodo activo se deseca en gran medida (alrededor de 100 °C), es decir, cuando todo el electrodo está completamente rodeado por tejido deshidratado.
	\item Objetivo: Demostrar esta hipótesis mediante un enfoque doble, teórico y experimental.
\end{itemize}

\subsection{Geometría y abstracción}
Dibujos abstractos del tema a estudiar.

\subsection{Ecuaciones de gobierno}
Tenemos que traducir a ecuaciones matemáticas y ver el dominio de donde se va a trabajar. En este caso será un cilindro. Normalmente se heredan modelos de otros trabajos.

Existen dos problemas:
\begin{itemize}
    \item Uno eléctrico.
    \item Uno térmico.
\end{itemize}

En el caso del microondas, 
\begin{itemize}
	\item Uno electromagnético.
	\item Uno térmico.
\end{itemize}

Y en el caso del laser, 
\begin{itemize}
	\item Uno óptico.
	\item Uno térmico.
\end{itemize}

Veamos como se formularía:

Comencemos con la ecuación de calor:
$$\rho c\dfrac{\partial T}{\partial t}=\triangledown(k\triangledown T).$$
Donde:
\begin{center}
    \begin{tabular}{rcl}
	$\rho$ &:& Densidad.\\ 
	$c$ &:& Calor específico.\\
	$T$ &:& Temperatura.\\
	$t$ &:& Tiempo.\\
	$k$ &:& Conductividad térmica.\\
	$\triangledown$ &:& Operador nabla.
    \end{tabular}
\end{center}

A esta ecuación debemos agregar las siguientes variables

$$\rho c\dfrac{\partial T}{\partial t}=\triangledown(k\triangledown T)+Q_b+Q_m+Q_F.$$

Donde:
\begin{center}
    \begin{tabular}{rcl}
	$Q_b$ &:& Calor proveniente de la sangre, llamado calor por perfusión.\\ 
	$Q_m$ &:& Calor metabólico. (Es pequeña por lo que es despreciable, ya que el tiempo de exposición 12 min).\\
	$Q_F$ &:& Calor de la fuente de energía. (Problema óptico o electromagnético).
    \end{tabular}
\end{center}

Veamos el calor por perfusión ($Q_b$).
$$Q_b=\beta \rho_b c_b \omega_b(T_b-T).$$

Donde:
\begin{center}
	\begin{tabular}{rcl}
	$\rho_b$ &:& Densidad de la sangre.\\
	$c_b$ &:& Calor específico de la sangre.\\
	$\omega_b$ &:& Coeficiente de refrigeración de la sangre.\\
	$T_b$ &:& Temperatura de la sangre. $37^{\circ}C$.\\
	$\beta$ &:& Coeficiente de intercambio de calor.\\ 
	\end{tabular}
\end{center}

$$
\beta=\left\{
\begin{array}{ll}
    1 & \Omega(t)<4.6\\
    0 & \Omega(t)\geq 4.6
\end{array}
\right.
\quad \Rightarrow \quad
\Omega(t)=\int_{0}^{t_{final}} Ae^{-\frac{\delta E}{RT(x,t)}}\;dt
$$


Donde:
\begin{center}
    \begin{tabular}{rcl}
	$\Omega$ &:& Daño térmico, en función de la temperatura y el tiempo.\\
	$A$ &:& Factor de frecuencia.\\
	$\triangle E$ &:& Energía de activación.\\
	$R$ &:& Constante de los gases.\\
	$T$ &:& Temperatura.
    \end{tabular}
\end{center}

\begin{itemize}
    \item Los valores de $A$ y $\triangle E$ se obtienen experimentalmente, no es lo mismo los daños en distintos órganos. Se considere $4.6$ es el $99\%$ del tejido muerto.
    \item Manejar una integral tan grande produce errores.
\end{itemize}

$T_b$ que es la Temperatura de la sangre se considera un refrigerante debido a que $T\geq T_b$. Es decir, la Temperatura del objeto a tratar será mayor al de la sangre en si.

\subsubsection{Problema térmico: Bioheat equation}
Ya dijimos que hay vaporización. Por lo que utilizamos un método llamado Entalpía. Lo que hace es que sustituye $pc\dfrac{\partial T}{\partial t}$ de la ecuación que estamos tratando por:
$$
\dfrac{\partial T}{\partial t} \cdot \left\{
    \begin{array}{ll}
	\partial_l c_l & 0<T\leq 99^{\circ}C\\
	h_{fg}C & 99<T\leq 100^{\circ}C\\
	\rho_g c_g & T>100^{\circ}C.
    \end{array}
\right.
$$

Donde:
\begin{center}
	\begin{tabular}{rcl}
	    $l$ & : & Líquido.\\
	    $C$ & : & Contenido de agua del tejido.\\
	    $\partial_l c_l$ &:& Calor latente liquido. (Se produce cambio de fase).\\
	    $h_{fg}C$ &:& Es el producto del  calor latente de vaporización del agua por la densidad del agua a $99^{\circ}$.\\
	    $\rho_g c_g$ &:& Calor específico del vapor.
	\end{tabular}
\end{center}

Considera al parcial de la temperatura con respecto del tiempo.

\subsection{Condiciones iniciales y de contorno}

Veamos las condiciones:
\begin{itemize}
    \item La temperatura es igual a la temperatura ambiente $T=T_a$.
    \item El flujo es nulo. $\vec{n}\cdot(k\triangledown T)=0$. 
    \item El proceso de calentar y enfriar algo, se llama convección natural. $\vec{n}(k\triangledown T)=h(T_r)-T$.
	Donde:
	\begin{center}
	    \begin{tabular}{rcl}
		$h$ &:& Coeficiente de refrigeración. (Acá se nota si es natural o forzada).\\
		$T_r$ &:& Temperatura de refrigeración.
	    \end{tabular}
	\end{center}
\end{itemize}

GRAN PARTE DEL TRABAJO ES AVERIGUAR CADA VARIABLE. 


\subsection{Características de los Materiales}
Los sacamos de artículos o bien de bases de datos.





% APUNTES Deep Learning
%\chapter{Deep Learning}
No podemos usar algoritmos de inteligencia artificial para la sostenibilidad  si los modelos no son interpretables. 

Debemos primero intentar modelos simples.


\section{Tipos de aprendizaje}

\subsection{Aprendizaje supervisado}
A estos modelos les damos etiquetas de entrada,

\subsection{Aprendizaje no supervisado}
No se da etiquetas, simplemente datos para que el modelo aprenda a encontrar patrones, mediante tareas de clustering.

\subsection{Aprendizaje por refuerzo}
En base de ensayo error se va aprendiendo. Por ejemplo si le pedimos que avance de un punto $A$ a un punto $B$, la distancia será su recompensa pero también tendremos que darle penalizaciones para que nuestro fin sea el correcto.

\section{¿Cómo predecir secuencias temporales?}

\subsection{Multi-layer perceptron (MLP)}
Por ejemplo, si queremos predecir el valor del tiempo: La salida sería la temperatura de mañana, la entrada sería la entrada de hoy y los días anteriores, pero la red neuronal no la entenderá debido a que las entradas están ordenadas como una lista atemporal, por lo que el MLP es una arquitectura para establecer un punto de partida.

\subsection{Resumen de una neurona}
Al final de todo tenemos valores de los nodos anteriores. En cada capa tenemos lo que es la matriz de pesos y el vector de sesgos. En la práctica es una secuencia de multiplicaciones matriz por el vector anterior más un vector de sesgos actual. Donde se aplica una función de activación no lineal. 
$$\zeta = g\left(W_i\zeta_{i-1} + b_i\right), \quad g(a)=\frac{1}{1+e^{-a}}.$$
\begin{itemize}
    \item $W_i$ es la matriz de pesos.
    \item $b_i$ es el vector de sesgos.
    \item $g$ es la función de activación.
    \item $\zeta_{i-1}$ es el vector de entrada.
\end{itemize}
Lo que lo hace interesante a las redes neuronales, es la unión de funciones no lineales, donde estas podrán aproximar a cualquier función y me permitirá la capacidad predictiva.

Entrenar una red neuronal es utilizar un algoritmo llamado \textit{Back-propogation}, lo que hace es ajustar los pesos $W_i$. Es decir, se calcula la derivada parcial del error respecto de los pesos de la red neuronal y lo vamos haciendo hacia atrás. Y utilizando el descenso del gradiente estocástica uno puede entrenar de forma efectiva los parámetros de la red neuronal.

\subsection{Entrenamiento de redes neuronales}
Para predecir el modelo, debemos separar los datos en entrenamiento ($80\%$) y validación ($20\%$). Esto para ajustar los parámetros de la red neuronal, donde los datos de validación no son vistos por la red y se utilizan para ver cómo funciona en datos que no se vieron aún. Luego el lost function o la función de coste o error, suele medirse en función de mínimos cuadrados. Así, si el valor es alto podemos decir que la predicción no es tan creíble y viceversa.

Para ello creamos un axis de épocas, donde la recta vertical es la perdida o curva de error, y la recta horizontal será el número de épocas. Donde una época es un paso por todos los datos de entrenamiento. Es decir, cómo el entrenamiento se hace de manera estocástica (descenso de gradiente estocástico) tu no les das los datos de golpe, si no les das pequeños paquetes de datos. Por ejemplo, damos 1000 datos y ajusta los parámetros, otros mil datos y ajusta los parámetros.

Cuando se pasa por esos datos, el error debería bajar. Así mientras le pasas más datos el ajuste debería mejorar.

A esto se tiene una linea azul que se llama linea de entrenamiento. 

Ahora, al final de cada época evaluamos la red en datos de la validación (linea roja). 

El error de validación será más alto que el error de entrenamiento, pero si el error de validación empieza a subir, es que el modelo está sobreajustando. Es decir, el modelo está aprendiendo cosas que no son generales, sino que está aprendiendo cosas que son específicas de los datos de entrenamiento o que tu red neuronal es demasiado grande. Si los valores del error ranto de entrenamiento como de validación son similares y relativamente altos se le llama subajuste.

Cuando el error de validación empieza a subir es cuando debemos parar de entrar, se le llama earling-stopping.

La limpieza de los datos y los ordenes de los datos serán importantes en algunos casos. El $95\%$ del éxito del modelo serán los datos.

\subsection{Redes neuronales recurrentes (RNN)}
Aprovecha la información de la secuencia temporal. Esta diseñada de manera que los ouputs de ayer están concetados con los inputs del siguiente día.

Estas redes tienen un problema, que los descensos gradientes se desvanecen. Cómo vas tomando el cuenta los datos pasados, mientras más avanza el tiempo esos gradientes se hacen más pequeños. Por lo que tenemos el siguiente modelo

\subsubsection{Long short-term memory (LSTM)}
Tiene un mecanismo de fuerzas para de alguna forma ponderar la importancia de los datos en el tiempo. Tiene tres veces más parámetros que un (MPL), donde la información más reciente tiene más peso.

Estos modelos se podrían optimizar combinando días cercanos y a largo plazo.

Todo lo anterior esta basado en estadísticas medias. Pero también se podría realizar con coeficientes instantaneas. Puede ser que seas sus errores similares, pero una pequeña variación podría producir un efecto mariposa (sistemas caóticos). Para ello podemos utilizar el mapa de Poincaré que representa en un plano la densidad de probabilidad de los valores de esos coeficientes en un espacio n-dimensional. Donde nos dará la correlación dentro del sistema.

Si en el sistema original yo meto la misma perturbación ¿desviaré mi trayectoria igualmente?. Si la red neuronal se está yendo a otra trayectoria es porque tiene un pequeño error. Para analizar estos problemas tenemos una herramienta llamada los exponentes de Lyapunov.

Estos exponentes son técnicas de análisis de sistemas caóticos, donde se representa en el eje vertical la desviación es escala logarítmica con respecto a la trayectoria inicial y el eje horizontal el tiempo. La idea es que si meto una perturbación en $t$, lo que sucederá es que de forma exponencial el error va a aumentar hasta que acabe en un lugar diferente. Por lo que veremos que tiene un error muy pequeño. Entonces, la historia es que si se que mi red neuronal tiene un error pequeño y meto manualmente en los sistemas de ecuaciones diferenciales ordinarias y mido la desviación respecto a la trayectoria inicial, es que el crecimiento exponencial viene dado por el exponente de Lyapunov al cavo de una trayectoria diferente y el error de la red neuronal da presente la misma desviación respecto a la trayectoria original. En otras palabras, si yo meto ese primer error manualmente obtengo los mismo resultados con la red neuronal. Es decir, la red neuronal a aprendido la dinámica del sistema original super bien con un pequeño error. El problema de las redes neuronales es que es un sistema caótico.

¿Qué pasa si se tiene multi-escala?. Es decir, algunas actúan más rápido y otras mas lento. Por lo que no podremos modelar todas las frecuencias. Así que cuando utilizamos LSTM, es separar la física de un sistema en física rápida, media y lenta, y utilizar diferentes redes neuronales LSTM para modelizar todos estos rangos de frecuencias. Es básicamente el problema de las redes recurrentes que fuerzan a una estructura en la que el input de un paso temporal está conectado con el output del paso siguiente.

Las ventajas de este sistema es que puedes explorar y explotar las secuencias temporales y las inconvenientes es que las secuencias no siempre tiene que ser tan rígidas como los sistemas que estamos explicando.

\subsection{Transformers}
Si se tiene una frase larga donde el sujeto este al principio o el objeto esta al final de la frase, pero no siempre puede estar al lado. La rede neuronal recurrente asume que las importancias están secuencialmente, el transformer utiliza un mecanismo de atención el cual clasifica a las palabras importantes.

\subsection{Redes neuronales informadas por física (PINNs)}
Estamos utilizando redes neuronales para resolver ecuaciones de derivadas parciales para fenómenos físicos.


\section{Predicciones espaciales}
Básicamente veremos como podemos hacer medidas de flujos.

\subsection{Convoluciones}
Identifica patrones espaciales en los datos.




% Teoría Wavelets
%\begin{multicols}{2}

\chapter{Preliminares}

\section{Espacio vectorial}

Un \textbf{espacio vectorial} sobre un campo $k$ es un conjunto $X$ junto con dos operaciones binarias 
    $$
    \begin{array}{rccl}
	+: &X \times X &\to& X \\
	\cdot:& k \times X &\to& X
    \end{array}
    $$
    que satisfacen los siguientes axiomas:
    \begin{enumerate}[\bfseries (i)]
	    \item Asociatividad de la suma: $x+(y+z) = (x+y)+z$ para todo $x,y,z \in X$.
	    \item Conmutatividad de la suma: $x+y = y+x$ para todo $x,y \in X$.
	    \item Elemento neutro de la suma: existe un elemento $0 \in X$ tal que $x+0 = x$ para todo $x \in X$.
	    \item Elemento opuesto de la suma: para todo $x \in X$ existe un elemento $-x \in X$ tal que $x+(-x) = 0$.
	    \item Asociatividad del producto por escalares: $(\alpha \beta)x = \alpha(\beta x)$ para todo $\alpha,\beta \in k$ y $x \in X$.
	    \item Distributividad del producto por escalares respecto de la suma de vectores: $\alpha(x+y) = \alpha x + \alpha y$ para todo $\alpha \in k$ y $x,y \in X$.
	    \item Distributividad del producto por escalares respecto de la suma de escalares: $(\alpha + \beta)x = \alpha x + \beta x$ para todo $\alpha,\beta \in k$ y $x \in X$.
	    \item Elemento neutro del producto por escalares: $1x = x$ para todo $x \in X$.
    \end{enumerate}

Luego, una \textbf{base} $B$ de un espacio vectorial $X$ sobre un campo $k$ es un subconjunto linealmente independiente de $X$ que genera $X$, tal que
    \begin{enumerate}[\bfseries (i)]
	\item  Si $c_1v_1 +\ldots + c_nv_n = 0$, entonces necesariamente $c_1 =\ldots= c_n = 0$ (es decir, $\left\{v_i : i = 1, 2,..., n\right\}$ es un conjunto linealmente independiente). 
	\item Para cada $x \in X$, $x = c_1v_1 +\ldots+ c_nv_n$. Los números $c_i$ se denominan coordenadas del vector $x$ con respecto a la base $B$ (es decir, $x = \text{span}\left\{v_i : i = 1, 2,..., n\right\})$.
    \end{enumerate}

Ahora, El número de elementos de una base se llama dimensión del espacio, donde pueden ser de dimensión finita o infinita.

\section{Espacio normado}

Un espacio normado $X$ es un espacio vectorial con una norma (es decir, para medir el tamaño del elemento del espacio vectorial) definida en él. La función norma $\|\ldots\| : X \to \mathbb{R}$ satisface los siguientes axiomas:
    \begin{enumerate}[\bfseries (i)]
	\item $\|x\|\geq 0$, $\|x\| = 0 \Leftrightarrow x = 0$.
	\item $\|\alpha x\| = |\alpha|\|x\|$ 
	\item $\|x+y\| \leq \|x\| + \|y\|$
    \end{enumerate}

\vspace{0.5cm}

\begin{ejemplo}
    El espacio vectorial $L^p(\mathbb{R})$, donde $1 \leq p < \infty$, es el conjunto de todas las funciones medibles tales que 
    $$\int_{\mathbb{R}} |f(x)|^p dx < \infty.$$ 

    La norma se define por 
    $$\|f\|_p = \left(\int_{\mathbb{R}} |f(x)|^p dx\right)^{1/p}.$$ 

    Esta norma (definida en (1.2)) también satisfará todos los axiomas mencionados anteriormente.
\label{ejemplo:1.1}
\end{ejemplo}

Las propiedades principales de las normas $\mathcal{L}_P(\mathbb{R}^n)$ son la \textbf{desigualdad de Minkowski}, 
\begin{equation}
    \|f+g\|_p \leq \|f\|_p + \|g\|_p.
\label{eq:Minkowski}
\end{equation}

Y la \textbf{desigualdad de Hölder},
\begin{equation}
    \|fg\|_1 \leq \|f\|_p\|g\|_q, \text{ donde } q=\dfrac{p}{p-1}.
\label{eq:Holder}
\end{equation}

El caso especial $p = q = 2$ da una forma de \textit{desigualdad de Cauchy-Schwarz}. Esto, dará lugar a un espacio normado completo llamado \textbf{espacio de Banach}.


\section{Espacio del producto interior}

Un \textbf{espacio del producto interno} $X$ es un espacio vectorial con un producto interno definido en él. Un producto interno es una función $\langle \cdot , \cdot \rangle : X \times X \to \mathbb{K}$ (donde $\mathbb{K} = \mathbb{R}$ o $\mathbb{C}$) que satisface los siguientes axiomas:
\begin{enumerate}[\bfseries (i)]
	\item $\langle x, x \rangle \geq 0$, $\langle x, x \rangle = 0 \Leftrightarrow x = 0$.
	\item $\langle  \alpha x, y \rangle = \alpha\langle x,y\rangle$.
	\item $\langle x,y \rangle = \overline{\langle y,x \rangle}$.
	\item $\langle x+y, z\rangle = \langle x,z \rangle + \langle y,z \rangle$.
\end{enumerate}

Un producto interno en $X$ define una norma en $X$ dada por $x = \sqrt{\langle x,x\rangle}$. Además, la norma introducida por el espacio del producto interno satisface la importante \textbf{ley del paralelogramo}
\begin{equation}
	\|x+y\|^2 + \|x-y\|^2 = 2\|x\|^2 + 2\|y\|^2.
\label{eq:paralelogramo}
\end{equation}

si una norma no satisface esta ecuación, no se puede obtener a partir de un producto interno. Por tanto, no todos los espacios normados son espacios de producto internos. La \textbf{desigualdad de Cauchy-Schwarz} (con respecto a los espacios $\mathcal{L}_2(\mathbb{R})$ establece que para todos los vectores $x$ e $y$ de un espacio producto interno
\begin{equation}
    |\langle x,y \rangle| \leq \|x\|\|y\|.
\label{eq:Cauchy-Schwarz}
\end{equation}
Además, se dice que dos vectores $x, y \in X$ son \textbf{ortogonales} si 
\begin{equation}
	\langle x, y\rangle = 0.
\label{eq:ortogonal}
\end{equation}


\section{Espacio de Hilbert}
Un espacio de producto interno completo se llama espacio de Hilbert. Podemos representar un elemento en espacio de Hilbert en términos de base como sigue: Dada una \textbf{base} $\{v_k : k \in I\}$ (secuencia de vectores) o $\{v_k (x) : k \in I\}$ (secuencia de funciones) en un espacio de Hilbert $\mathcal{H}$, cada vector o función $f \in \mathcal{H}$ puede ser representado de manera única como
\begin{equation}
    f = \sum_{k \in I} c_k (f) v_k, \quad f(x) = \sum_{k \in I} c_k (f) v_k (x),
\label{eq:baseHilbert}
\end{equation}

donde el conjunto de índices $I$ puede ser finito o infinitamente contable. Supongamos que los elementos del espacio de Hilbert son funciones, entonces un \textbf{marco} también es un conjunto $\{v_k (x) : k \in I\}$ en $\mathcal{H}$ que permite que cada $f$ se escriba como en la ecuación de arriba, pero será linealmente dependiente. Por lo tanto, los coeficientes $c_k (f)$ no son necesariamente únicos y se puede obtener una representación redundante. Más precisamente, una familia $\{v_k (x) : k \in I\}$ en un espacio de Hilbert $\mathcal{H}$ es un \textbf{marco} para $\mathcal{H}$ si existen dos constantes $m, M$ que satisfacen $0 < m \leq M < \infty$ tal que
\begin{equation}
    m \|f\|^2 \leq \sum_{k \in I} | \langle f, v_k \rangle |^2 \leq M \|f\|^2, \;\; \forall f \in \mathcal{H}
\label{eq:marco}
\end{equation}

También, para cada marco existe un \textbf{marco dual} $\{\tilde{v}_k : k \in I\}$ tal que
\begin{equation}
	f = \sum_{k \in I} \langle f, \tilde{v}_k \rangle v_k, \quad \forall f \in \mathcal{H}.
\label{eq:marco_dual}
\end{equation}

Si el marco es un conjunto linealmente independiente para $\mathcal{H}$, entonces el marco proporciona una \textbf{base de Riesz} para $\mathcal{H}$. Por lo tanto, la base de Riesz es una base que satisface \ref{eq:baseHilbert}, y la desigualdad del marco \ref{eq:marco}. En el caso de la base de Riesz, el marco dual también será una base de Riesz y \textbf{biortogonal}. Es decir,
\begin{equation}
	\langle v_k, \tilde{v}_k\rangle = \delta_{k,k'}.
\label{eq:biortogonal}
\end{equation}

\textbf{Observación}\; En el caso de una base ortonormal, se satisface la desigualdad del marco para $m = M = 1$. Por lo tanto, también es una base de Riesz. De hecho, una base ortonormal es biortogonal a sí misma.
\begin{equation}
     \|f\|^2 = \sum_{k \in I} | \langle f, v_k \rangle |^2, \;\; \forall f \in \mathcal{H}
\label{eq:marco11}
\end{equation}




\section{Proyección}
Se dice que un espacio vectorial $X$ es la suma directa ($X = Y \oplus Z$) de dos subespacios $Y$ y $Z$ de $X$, si cada $x \in X$ tiene una representación única
\begin{equation}
	x = y + z, \quad y \in Y \quad \text{y} \quad z \in Z.
\label{eq:suma_directa}
\end{equation}

Además, $Z$ se llama \textbf{complemento algebraico} de $Y$ en $X$ y viceversa. 

Si $X$ es un espacio de Hilbert. Una \textbf{proyección} ($P$) (generaliza la idea de proyección gráfica) es una transformación lineal ($P : X \rightarrow Y$) que tiene las siguientes propiedades básicas:
\begin{itemize}
    \item  $P$ es el \textbf{operador de identidad} $I$ en $Y$: $\forall y \in Y: Py = y$ ($P$ transforma $Y$ en $Y$).
    \item  $P$ es \textbf{idempotente} (es decir, $P^2 = P$).
    \item  Tenemos una suma directa $X = Y \oplus Z$. Cada vector $x$ en $X$ puede descomponerse de manera única como $x = y + z$ con $y = Px$ y $z = x - Px = (I - P)x$, ($Pz = P(x - Px) = 0$, significa que $P$ transforma $Z$ en $\{0\}$).
\end{itemize}


\section{Series de funciones}
En cálculo, una serie de funciones es una serie, donde los sumandos no son solo números reales o complejos sino funciones. Ejemplos de series de funciones incluyen series de potencias, series de Laurent, series de Fourier, etc. En cuanto a las secuencias de funciones, y a diferencia de las series de números, existen muchos tipos de convergencia para una serie de funciones. Mencionamos algunos de ellos a continuación:

\begin{enumerate}[\bfseries (i)]
    \item  La serie infinita $\sum_{n=1}^{\infty} f_n(x)$ converge a $f(x)$ en $a < x < b$ en el sentido de media cuadrática (o sentido $\mathcal{L}_2$. Es decir, en la norma $\mathcal{L}_2(a, b)$), si
    $$\int_a^b |f(x) - s_N(x)|^2 dx \rightarrow 0, \quad \text{cuando} \quad N \rightarrow \infty,$$
    donde $s_N(x) = \sum_{n=1}^{N} f_n(x)$.

    \item  La serie infinita $\sum_{n=1}^{\infty} f_n(x)$ converge a $f(x)$ puntualmente en $(a, b)$ si converge a $f(x)$ para cada $x \in (a, b)$. Es decir,
    $$|f(x) - s_N(x)| \rightarrow 0, \quad \text{cuando} \quad N \rightarrow \infty.$$

    \item  La serie infinita converge uniformemente a $f(x)$ en $a \leq x \leq b$, si
    $$\max_{a \leq x \leq b} |f(x) - s_N(x)| \rightarrow 0, \quad \text{cuando} \quad N \rightarrow \infty.$$

\end{enumerate}





\chapter{Análisis de Fourier}
El análisis de Fourier contiene dos componentes: \textbf{series de Fourier} y \textbf{transformada de Fourier}. La serie de Fourier es más universal que la serie de Taylor porque muchas funciones periódicas discontinuas de interés práctico pueden desarrollarse en forma de series de Fourier.

Sin embargo, muchos problemas prácticos también involucran funciones no periódicas. En ese caso, la serie de Fourier se convierte en integral de Fourier y la forma compleja de la integral de Fourier se denomina transformada de Fourier.

\section{Series de Fourier}
Una serie de Fourier es una expansión de una función periódica $f(x)$ en términos de las funciones del conjunto ortonormal $\{e^{\frac{in\pi x}{l}}\}_{n\in\mathbb{Z}}$. La serie de Fourier es útil como una forma de descomponer una función periódica arbitraria en un conjunto de funciones simples (es decir, polinomios trigonométricos). Estas funciones simples pueden ser insertadas, resueltas individualmente y luego recombinadas para obtener la solución al problema original o una aproximación a ella con cualquier precisión deseada.

Para cualquier $n \in \mathbb{Z}$, $e^{\frac{in\pi x}{l}}$ tiene período $2l$, por lo que la serie de Fourier tiene período $2l$. Ahora, la pregunta importante a responder es si todas las funciones con período $2l$ tienen una expansión en serie de Fourier. Para este propósito, consideramos el espacio $\mathcal{L}_2(0, 2l)$ de funciones cuadrado integrables, $2l$-periódicas.

Una función $f(x) \in \mathcal{L}_2(0, 2l)$ si 
$$f(x + 2l) = f(x)\;\; \forall x \in \mathbb{R}, \text{ y}$$
$$\displaystyle\int_0^{2l} |f(x)|^2 dx < \infty.$$

El espacio $\mathcal{L}_2(0, 2l)$ es un espacio de Hilbert con el producto interno
$$\langle f, g \rangle = \frac{1}{2l} \int_0^{2l} f(x)\overline{g(x)}\;dx,$$

y la norma correspondiente
$$||f||_2^2 = \frac{1}{2l} \int_0^{2l} |f(x)|^2 dx.$$

Resulta que la familia de funciones $\{e^{\frac{in\pi x}{l}}\}_{n\in\mathbb{Z}}$ es una base ortonormal del espacio $\mathcal{L}_2(0, 2l)$ y, por lo tanto, cualquier $f(x) \in \mathcal{L}_2(0, 2l)$ tiene la siguiente representación de serie de Fourier:
\begin{equation}
f(x) = a_0 + \sum_{n=1}^{\infty} \left( a_n \cos\left(\frac{in\pi x}{l}\right) + b_n \sin\left(\frac{in\pi x}{l}\right) \right)
\label{eq:serieFourier}
\end{equation}

donde las constantes $a_0$, $a_n$ y $b_n$ se llaman los coeficientes de Fourier de $f(x)$, definidos por
\begin{equation}
    \left.
	\begin{array}{rcl}
	    a_0 &=&\displaystyle \frac{1}{2l} \int_0^{2l} f(x)dx\\\\
	    a_n &=&\displaystyle \frac{1}{l} \int_0^{2l} f(x)\cos\left(\frac{in\pi x}{l}\right)dx\\\\ 
	    b_n &=&\displaystyle \frac{1}{l} \int_0^{2l} f(x)\sin\left(\frac{in\pi x}{l}\right)dx.
	\end{array}
    \right\}
\end{equation}

La forma compleja de \ref{eq:serieFourier} es
\begin{equation}
    f(x) = \sum_{n=-\infty}^{\infty} c_n e^{\frac{in\pi x}{l}},
\end{equation}

donde las constantes $c_n$ se llaman los coeficientes de Fourier de $f(x)$, definidos por
\begin{equation}
    c_n = \frac{1}{2l} \int_0^{2l} f(x)e^{-\frac{in\pi x}{l}} dx.
\end{equation}

Ahora, la \textbf{convergencia de la serie} \ref{eq:serieFourier} en $\mathcal{L}_2(0, 2l)$ significa
$$\lim_{N\rightarrow\infty} \int_0^{2l} |f(x) - s_N(x)|^2 dx \rightarrow 0,$$

donde $s_N(x)\sum_{n=-N}^{N}c_ne^{\frac{in\pi x}{l}}$ es la \textbf{suma parcial de la serie} hasta el término $N$.

Observamos que a partir de la serie \ref{eq:serieFourier}, la función $f(x)$ (una función periódica con periodo $l = \pi, 2\pi$) se descompone en una suma de infinitos componentes mutuamente ortogonales, $w_n(x) = e^{inx}$, $n \in \mathbb{N}$. Esta base ortonormal se genera por dilataciones enteras ($w_n(x) = w(nx)$, $\forall n \in \mathbb{N}$) de una única función $w(x) = e^{ix}$. Esto significa que $e^{ix}$ (onda sinusoidal) es la única función requerida para generar el espacio $\mathcal{L}_2(0, 2\pi)$. Esta onda sinusoidal tiene alta frecuencia para grandes $|n|$ y baja frecuencia para pequeños $|n|$. Por lo tanto, cada función en el espacio $\mathcal{L}_2(0, 2\pi)$ está compuesta de ondas con diversas frecuencias. Además, la serie de Fourier definida en \ref{eq:serieFourier} satisface la \textbf{identidad de Parseval} dada por
\begin{equation}
\int_0^{2l} |f(x)|^2 dx = 2l \sum_{n=-\infty}^{\infty} |c_n|^2.
\label{eq:identidadParseval}
\end{equation}

Ahora, escribimos resultados para la convergencia de series de Fourier en diferentes modos de convergencia.

\begin{teorema}
Si $f(x)$ y $f'(x)$ son funciones continuas a trozos en $(0, 2l)$, entonces la serie de Fourier converge en cada punto $x$. Además,
$$
\sum_{n=-\infty}^{\infty} c_n e^{in\pi x / l} = \frac{1}{2} \left( f_{\text{ext}}(x+) + f_{\text{ext}}(x-) \right),
$$
para todo $-\infty < x < \infty$, donde $f_{\text{ext}}(x)$, $-\infty < x < \infty$ es la extensión periódica de $f(x)$ definida en $(0, 2l)$.
\end{teorema}

\begin{teorema}
    Si $f(x) \in \mathcal{L}_2(0, 2l)$, entonces la serie de Fourier converge a $f(x)$, $x \in (0, 2l)$ en el sentido $\mathcal{L}_2$.
\end{teorema}

\begin{teorema}
    Si $f(x)$ y $f'(x)$ son continuas en $[0, 2l]$ y el valor de la serie de Fourier coincide con $f(x)$ en los extremos (0 y 2l), entonces la serie de Fourier converge a $f(x)$, $x \in [0, 2l]$ de manera uniforme.
\end{teorema}


\section{Transformada de Fourier}
La transformada de Fourier de una función $f(x) \in \mathcal{L}_1(\mathbb{R})$ (espacio de funciones integrables como $p = 1$ en el ejemplo \ref{ejemplo:1.1}) se define como
\begin{equation}
\hat{f}(\omega) = F(f(x)) = \int_{-\infty}^{\infty} e^{-i\omega x} f(x) dx, \quad \omega \in \mathbb{R}.
\label{eq:transformadaFourier}
\end{equation}

Debe notarse que $|\hat{f}(\omega)| \leq \int_{-\infty}^{\infty} |f(x)| dx < \infty$, y por lo tanto, podemos garantizar la existencia de la transformada de Fourier definida en \ref{eq:transformadaFourier}. Algunas de las \textbf{propiedades} de $\hat{f}(\omega)$, para cada $f(x) \in \textbf{L}_1(\mathbb{R})$ son las siguientes:
\begin{enumerate}[(1)]
    \item  $\hat{f}(\omega) \in \mathcal{L}_\infty(\mathbb{R})$ y $||\hat{f}||_\infty \leq ||f||_1$.
    \item  Si la derivada $f'(x)$ de $f(x)$ existe y $f'(x) \in \mathcal{L}_1(\mathbb{R})$, entonces $\hat{f'}(\omega) = i\omega \hat{f}(\omega)$.
    \item $\hat{f}(\omega) \rightarrow 0$, cuando $\omega \rightarrow -\infty$ o $\infty$.
\end{enumerate}

Debe notarse que, $f \in \mathcal{L}_1(\mathbb{R})$ no implica que $\hat{f} \in \mathcal{L}_1(\mathbb{R})$, lo cual puede ilustrarse con el siguiente ejemplo:
$$
f(x) = 
\begin{cases} 
e^{-x} & \text{si } x \geq 0, \\
0 & \text{si } x < 0.
\end{cases}
$$

La función anterior $f(x)$ está en el espacio $\mathcal{L}_1(\mathbb{R})$ pero su transformada de Fourier $\hat{f}(\omega) = (1 - i\omega)^{-1}$ no está en $\mathcal{L}_1(\mathbb{R})$.

Si $\hat{f} \in L^1(\mathbb{R})$ es la transformada de Fourier de $f \in \mathcal{L}_1(\mathbb{R})$, entonces la \textbf{transformada inversa de Fourier} de $\hat{f}$ se define como
\begin{equation}
(F^{-1}\hat{f})(x) = \frac{1}{2\pi} \int_{-\infty}^{\infty} e^{i\omega x} \hat{f}(\omega) d\omega.
\label{eq:transformadaInversaFourier}
\end{equation}

\textbf{Observación:}\;  La transformada de Fourier definida en \ref{eq:transformadaFourier}, descompone las funciones (señales) en la suma de un número (potencialmente infinito) de componentes de onda sinusoidal y cosinusoidal de frecuencia. El dominio de frecuencia es un término utilizado para describir el dominio para el análisis de funciones matemáticas (señales) con respecto a la frecuencia. Por lo tanto, la transformada de Fourier de una función también se llama el \textbf{espectro de Fourier de esa función}. La transformada inversa de Fourier convierte la transformada de Fourier del dominio de frecuencia a la función original bajo ciertas suposiciones. Si la función original es una señal, entonces la representación física del dominio será el dominio del tiempo.

\begin{tcolorbox}[colframe=white]
Debe notarse que los dos componentes del análisis de Fourier, a saber, la serie de Fourier y la transformada de Fourier son básicamente no relacionados. Veremos más adelante que este no es el caso con el análisis de wavelet.
\end{tcolorbox}

\vspace{0.5cm}

\begin{ejemplo}
    Evalúa la transformada de Fourier de la función gaussiana $f(x) = e^{-ax^2}$, $a > 0$.

    \textbf{Solución.-}\; La transformada de Fourier de $f$ se define como
    \begin{equation}
	\hat{f}(\omega) = \int_{-\infty}^{\infty} e^{-i\omega x} e^{-ax^2} dx,
	\label{eq:transformadaFourierGaussiana}
    \end{equation}

    que se puede reescribir como
    \begin{align*}
	\hat{f}(\omega) &= \int_{-\infty}^{\infty} e^{-a(x+ i\omega/2a)^2 - \omega^2/4a} dx, \\
	&= e^{-\frac{\omega^2}{4a}} \sqrt{a} \int_{-\infty}^{\infty} e^{-x^2} dx, \\
	&= \sqrt{\frac{\pi}{a}} e^{-\omega^2/4a}.
    \end{align*}

    Otro método para evaluar la transformada de Fourier de la función gaussiana usando la regla de leibnitz para la integral impropia y la integración por partes ya que  $f(x) \in \mathcal{L}_1(\mathbb{R})$  es diferenciar ambos lados de \ref{eq:transformadaFourierGaussiana} con respecto a $\omega$:
    \begin{align*}
	\hat{f}'(\omega) &= -i \int_{-\infty}^{\infty} e^{-i\omega x} (x e^{-ax^2}) dx \\
	&= \frac{\omega}{2a} \int_{-\infty}^{\infty} e^{-i\omega x} e^{-ax^2} dx \\
	&= \frac{\omega}{2a} \hat{f}(\omega).
    \end{align*}

    Por lo tanto, 
    $$\hat{f}(\omega) = \hat{f}(0) e^{-\frac{\omega^2}{4a}}.$$ 
    Ahora, 
    $$\hat{f}(0) = \int_{-\infty}^{\infty} e^{-ax^2} dx = \sqrt{\pi/a},$$ 
    Así, $\hat{f}(\omega) = \sqrt{\pi/a} e^{-\frac{\omega^2}{4a}}.$
    \label{ejemplo:1.2}
\end{ejemplo}

Veamos la transformada inversa de Fourier definida en \ref{eq:transformadaInversaFourier} donde devuelve la función original $f(x)$. Para probar esta afirmación, introducimos primero la convolución.

La convolución es una operación matemática en dos funciones $f$ y $g$, que implica la multiplicación de una función por una versión retrasada o desplazada de otra función, integrando o promediando el producto y repitiendo el proceso para diferentes retrasos. Tiene aplicaciones en el área de procesamiento de imágenes y señales, ingeniería eléctrica, probabilidad, estadística, visión por computadora y ecuaciones diferenciales. Más precisamente, \textbf{la convolución de dos funciones} se define por
\begin{equation}
(f * g)(x) = \int_{-\infty}^{\infty} f(\tau) g(x - \tau) d\tau,
\label{eq:convolucion}
\end{equation}

garantizamos la existencia de la convolución si $f, g \in \mathcal{L}_1(\mathbb{R})$. Se puede observar que $f * g \in \mathcal{L}_1(\mathbb{R})$ ya que
\begin{align*}
\int_{-\infty}^{\infty} |(f * g)(x)| dx &\leq \int_{-\infty}^{\infty} \int_{-\infty}^{\infty} |f(\tau)||g(x - \tau)| dx d\tau, \\
&= \int_{-\infty}^{\infty} |f(\tau)| \int_{-\infty}^{\infty} |g(x - \tau)| dx d\tau, \\
&= \int_{-\infty}^{\infty} |f(\tau)| \int_{-\infty}^{\infty} |g(x)| dx d\tau, \\
&= \int_{-\infty}^{\infty} |f(\tau)| d\tau \int_{-\infty}^{\infty} |g(x)| dx.
\end{align*}

Esto implica que $\|f * g\|_1 \leq \|f\|_1 \|g\|_1$ y por lo tanto $f * g \in \mathcal{L}_1(\mathbb{R})$. Un cambio de variable de integración en \ref{eq:convolucion} nos dirá que $f * g = g * f$, $\forall f, g \in L^1(\mathbb{R})$. Es decir, la convolución es conmutativa. 

Luego, dado que $f * g \in \mathcal{L}_1(\mathbb{R})$, podemos definir $(f * g) * h$ y una simple manipulación te dirá que $(f * g) * h = f * (g * h)$, $\forall f, g, h \in \mathcal{L}_1(\mathbb{R})$, es decir, la convolución es asociativa. Sin embargo, ninguna función actúa como una identidad para la convolución excepto en el caso de un concepto modificado de función (función generalizada). 

Para probar que no existe una función de identidad para la convolución, necesitamos el siguiente teorema que establece que la transformada de Fourier de una convolución es el producto puntual de las transformadas de Fourier.\\

\begin{teorema}[Teorema de la convolución]
    Si $f, g \in L^1(\mathbb{R})$, entonces $\hat{(f * g)}(\omega) = \hat{f}(\omega) \hat{g}(\omega)$.

    \textbf{Demosración:}
    \begin{align*}
    \hat{(f * g)}(\omega) &= \int_{-\infty}^{\infty} e^{-i\omega x} (f * g)(x) dx, \\
    &= \int_{-\infty}^{\infty} e^{-i\omega x} \int_{-\infty}^{\infty} f(\tau) g(x - \tau) d\tau dx, \\
    &= \int_{-\infty}^{\infty} f(\tau) \int_{-\infty}^{\infty} e^{-i\omega x} g(x - \tau) dx d\tau, \\
    &= \int_{-\infty}^{\infty} f(\tau) \int_{-\infty}^{\infty} e^{-i\omega (\tau + y)} g(y) dy d\tau, \\
    &= \int_{-\infty}^{\infty} f(\tau) e^{-i\omega \tau} \int_{-\infty}^{\infty} e^{-i\omega y} g(y) dy d\tau, \\
    &= \hat{f}(\omega) \hat{g}(\omega).
    \end{align*}
\label{teorema2.4}
\end{teorema}

Este teorema es extremadamente importante, especialmente útil para implementar una convolución numérica en una computadora.


\textbf{Observación:} \; El algoritmo de convolución estándar tiene una complejidad computacional cuadrática. Con la ayuda del teorema de convolución y la transformada rápida de Fourier, la complejidad de la convolución se puede reducir a O($n \log n$). Esto se puede explotar para construir algoritmos de multiplicación rápidos.

Ahora supongamos que existe una función $e \in \mathcal{L}_1(\mathbb{R})$, que actúa como una \textbf{identidad para el operador de convolución}. Es decir, 
\begin{equation}
	f * e = f,\; \forall f \in \mathcal{L}_1(\mathbb{R}).
\label{eq:identidadConvolucion}
\end{equation}
Ahora aplicando el Teorema \ref{teorema2.4} en \ref{eq:identidadConvolucion}, 
$$\hat{f}(\omega) \hat{e}(\omega) = \hat{f}(\omega), \;\forall f \in L^1(\mathbb{R})$$ 

lo que significa que 
\begin{equation}
	\hat{e}(\omega) = 1,
\label{eq:identidadConvolucion2}
\end{equation}
lo cual no es cierto (como sabemos que $\hat{e}(\omega) \rightarrow 0$, cuando $\omega \rightarrow -\infty$ o $\infty$).


Definamos la \textbf{delta de Dirac} ($\delta$), que es una función generalizada (distribución) en la línea de números reales tal que es distinta de cero solo en cero y cero en todas partes, con una integral igual a uno en toda la línea real. Es decir,
\begin{equation}
	\delta(x) = 0 \quad \forall x \neq 0
\label{eq:deltaDirac}
\end{equation}
y
\begin{equation}
	\int_{-\infty}^{\infty} \delta(x) dx = 1.
\label{eq:deltaDiracIntegral}
\end{equation}

En el contexto del procesamiento de señales, a menudo se refiere como la función de \textbf{impulso unitario}. Es un análogo continuo de la función delta de Kronecker que generalmente se define en un dominio finito y toma valores $0$ y $1$. La distribución delta de Dirac (distribución $\delta$ para $x_0 = 0$) puede actuar como identidad de convolución porque
\begin{equation}
	f * \delta = f, \quad \forall f \in L^1(\mathbb{R}).
\label{eq:deltadirecdistribution}
\end{equation}

siempre que $\hat{\delta}(\omega) = 1$.

Ya hemos visto que no podemos obtener una función de identidad en $\mathcal{L}_1(\mathbb{R})$ para la convolución. Sin embargo, todavía deseamos encontrar una aproximación a la función $\delta$ en \ref{eq:deltadirecdistribution}. Es decir, una aproximación de la identidad de convolución.

Consideremos las \textbf{funciones gaussianas} de la forma $ae^{-(x-b)^2/c^2}$ para algunas constantes reales $a$, $b$, $c$. El parámetro $a$ es la altura del pico de la curva, $b$ es la posición del centro del pico y $c$ controla el ancho del pico de la curva. Luego, tomamos una función gaussiana particular con $c = 2/\sqrt{\alpha}$, $a = 1/c\sqrt{\pi}$ y $b = 0$
\begin{equation}
	e_{\alpha}(x) = \frac{1}{2}\sqrt{\frac{\pi}{\alpha}} e^{-x^2/4\alpha}, \quad \alpha > 0,
\label{eq:gaussianfunction}
\end{equation}

cuya transformada de Fourier se da por
\begin{equation}	
	\hat{e}_{\alpha}(\omega) = e^{-\alpha \omega^2}.
\label{eq:gaussianfunctionFourier}
\end{equation}

Está claro que la función $e_{\alpha}$ satisface \ref{eq:identidadConvolucion2} cuando $\alpha \rightarrow 0^+$. Ahora, enunciamos el siguiente teorema

\begin{teorema}
    Si $f \in L^1(\mathbb{R})$, entonces $\lim_{\alpha \rightarrow 0^+} (f * e_{\alpha})(x) = f(x)$ en cada punto $x$ donde $f$ es continua.
\end{teorema}
Este teorema nos dice que $e_{\alpha} \rightarrow \delta$ en $C$ (colección de funciones continuas en $\mathcal{L}_1(\mathbb{R})$) cuando $\alpha \rightarrow 0^+$. Ahora $\{e_{\alpha}\}$ es una \textbf{aproximación de la identidad de convolución}. Se puede usar para probar el siguiente teorema.

\begin{teorema}
    Si $f \in \mathcal{L}_1(\mathbb{R})$ tal que su transformada de Fourier también está en $\hat{f} \in \mathcal{L}_1(\mathbb{R})$, entonces
    $$f(x) = (F^{-1}\hat{f})(x),$$
    en cada punto $x$ donde $f$ es continua.
\end{teorema}

Hemos definido la transformada de Fourier de las funciones en el espacio $\mathcal{L}_1(\mathbb{R})$, y en este espacio no hay garantía de que exista la transformada inversa de Fourier. Una herramienta no es práctica si no puedes revertir su efecto, por lo que necesitamos movernos a un espacio donde al menos exista la transformada inversa de Fourier. El espacio $\mathcal{L}_2(\mathbb{R})$ (conjunto de funciones cuadrado integrables) es mucho más elegante para definir la transformada de Fourier ya que para $f \in \mathcal{L}_2(\mathbb{R})$ su transformada de Fourier $\hat{f} \in \mathcal{L}_2(\mathbb{R})$, que se puede formular en forma de teorema de la siguiente manera.
\begin{teorema}
    La transformada de Fourier $F$ es un mapa uno a uno de $\mathcal{L}_2(\mathbb{R})$ sobre sí mismo. Es decir, para cada $g \in \mathcal{L}_2(\mathbb{R})$, corresponde una y solo una $f \in \mathcal{L}_2(\mathbb{R})$ tal que $\hat{f} = g$
    $$f(x) = (F^{-1}g)(x),$$
\end{teorema}

Para $f \in \mathcal{L}_2(\mathbb{R})$, se satisface un resultado similar a \ref{eq:identidadParseval} de la siguiente manera:
\begin{equation}
	\|f\|_2^2 = \frac{1}{2\pi} \|\hat{f}\|_2^2 \quad \text{(identidad de Parseval)}.
\label{eq:identidadParseval2}
\end{equation}

La teoría detallada de la transformada de Fourier de funciones en $\mathcal{L}_2(\mathbb{R})$ (discutida anteriormente)  también se conoce como la \textbf{teoría de Plancherel}.



\section{Análisis tiempo-frecuencia}
El análisis de tiempo-frecuencia es un tema de análisis de señales donde estudiamos la señal tanto en los dominios de tiempo como de frecuencia juntos con un par de ventajas sobre la transformada de Fourier como sigue.

\begin{itemize}
    \item Para estudiar una señal en el dominio de la frecuencia a través de su transformada de Fourier, necesitamos una descripción completa del comportamiento de la señal durante todo el tiempo, lo que puede incluir un comportamiento futuro indeterminado de la señal. No podemos predecir la transformada de Fourier basándonos solo en la observación local de la señal.
    \item  Si la señal cambia en un pequeño vecindario de algún tiempo particular, entonces todo el espectro de Fourier se ve afectado.
    \item El análisis de Fourier clásico asume que las señales son infinitas en el tiempo (en el caso de la transformada de Fourier) o periódicas (en el caso de la serie de Fourier). Sin embargo, muchas señales en la práctica son de corta duración (es decir, no están definidas para un tiempo infinito) y no son periódicas. Por ejemplo, los instrumentos musicales tradicionales no producen sinusoides de duración infinita, sino que comienzan con un ataque y luego decaen gradualmente.  
\end{itemize}

El análisis de tiempo-frecuencia es una generalización del análisis de Fourier para aquellas señales cuya frecuencia (estadísticas) varía con el tiempo. Muchas señales de interés, como el habla, la música, las imágenes y las señales médicas, tienen una frecuencia cambiante, lo que nuevamente motiva el análisis de tiempo-frecuencia.

La desventaja de la transformada de Fourier en el análisis de tiempo-frecuencia fue observada inicialmente por D. Gabor. Utilizó el concepto de función de ventana para definir otra transformada. Una función no trivial $w \in \mathcal{L}_2(\mathbb{R})$ se llama función de ventana si $xw(x)$ también está en $\mathcal{L}_2(\mathbb{R})$. Significa que la función decae a cero rápidamente. El centro $t^*$ y el radio $w$ de la función de ventana se definen por
\begin{equation}
    \begin{array}{rcl}
	t^* &=& \displaystyle\frac{1}{||w||_2^2} \int_{-\infty}^{\infty} x|w(x)|^2 dx,\\\\
	w &=& \displaystyle\frac{1}{||w||_2} \left( \int_{-\infty}^{\infty} (x - t^*)^2 |w(x)|^2 dx \right)^{1/2}.
    \end{array}
    \label{eq:ventana}
\end{equation}

por lo tanto, el ancho de la función de ventana será $2w$.

Gabor utilizó esta función gaussiana para definir la transformada de Gabor para $\alpha > 0$, $f \in \mathcal{L}_2(\mathbb{R})$
\begin{equation}
(G_{b,\alpha} f)(\omega) = \int_{-\infty}^{\infty} (e^{-i\omega x} f(x)) e^{\alpha(x - b)} dx. 
\label{eq:2.19}
\end{equation}

En \ref{eq:2.19}, la transformada de Gabor está localizando la transformada de Fourier de $f$ alrededor del punto $x = b$. Además,
\begin{equation}
    \int_{-\infty}^{\infty} e^{\alpha(x - b)} db = \int_{-\infty}^{\infty} e^{\alpha t} dt = 1,
\label{eq:propiedadGabor}
\end{equation}

como $\hat{e}_{\alpha}(0) = 1$ de \ref{eq:gaussianfunctionFourier}. Por lo tanto,
\begin{equation}
\int_{-\infty}^{\infty} (G_{b,\alpha} f)(\omega) db = \hat{f}(\omega), \quad \omega \in \mathbb{R}. 
\end{equation}

Se puede demostrar fácilmente que la función gaussiana discutida en \ref{eq:gaussianfunction} es una función de ventana. Dado que $e_{\alpha}$ es una función par, el centro $t^*$ definido en \ref{eq:ventana} es $0$, y por lo tanto, el radio de la ventana gaussiana será dado por
$$\triangle_{e_{\alpha}} = \frac{1}{||e_{\alpha}||_2} \left( \int_{-\infty}^{\infty} x^2 e^{2\alpha(x)} dx \right)^{1/2} = \sqrt{\alpha}.$$

Podemos reescribir \ref{eq:2.19} de la siguiente manera:
$$(G_{b,\alpha} f)(\omega) = \int_{-\infty}^{\infty} f(x) \overline{(e^{i\omega x} e^{\alpha(x - b)})} dx.$$

Significa que estamos ventaneando la función $f$ utilizando la función de ventana $w(x) = (e^{i\omega x} e^{\alpha(x - b)})$. Por lo tanto
\begin{equation}
(G_{b,\alpha} f)(\omega) = \langle f,w \rangle = \frac{1}{2\pi} \langle \hat{f}, w \rangle,
\label{eq:2.22}
\end{equation}
donde 
$$w(\hat{\eta}) = e^{-ib(\eta-\omega)} e^{-\alpha(\eta-\omega)^2} = e^{ib\omega} e^{-ib\eta} \left( \sqrt{\frac{\pi}{\alpha}} e^{\frac{1}{4\alpha}(\eta - \omega)} \right).$$ 
Por lo tanto
\begin{equation}
    \begin{array}{rcl}
	(G_{b,\alpha} f)(\omega) &=& \displaystyle\frac{1}{2\pi} \int_{-\infty}^{\infty} \hat{f}(\eta) w(\hat{\eta}) d\eta\\\\
			     &=&\displaystyle e^{-ib\omega} \frac{2}{\sqrt{\pi \alpha}} \int_{-\infty}^{\infty} \hat{f}(\eta) e^{ib\eta} e^{\frac{1}{4\alpha}(\eta - \omega)} d\eta.
    \end{array}
\label{eq:2.23}
\end{equation}

Ahora reescribimos la \ref{eq:2.22} de la siguiente manera:
\begin{equation}
	(G_{b,\alpha} f)(\omega) = \langle f,w \rangle = \langle \hat{f}, h \rangle,
\label{eq:2.24}
\end{equation}
donde  
$$h = \frac{1}{2\pi} \hat{w} = e^{ib\omega} e^{-ib\eta} \left( \frac{1}{2} \sqrt{\frac{\pi}{\alpha}} e^{\frac{1}{4\alpha}(\eta - \omega)} \right).$$

Por lo tanto, la información de la $f(x)$ alrededor de $x = b$ utilizando la función de ventana $w$ también se puede obtener observando el espectro $\hat{f}(\eta)$ en el vecindario de $\eta = \omega$ utilizando la función de ventana $h$ definida en \ref{eq:2.24}. La ventana correspondiente a la función de ventana $w$ en \ref{eq:2.24} soportada en $[b - \sqrt{\alpha}, b + \sqrt{\alpha}]$ se llama \textbf{ventana de tiempo}. La ventana correspondiente a la función de ventana $h$ soportada en $[\omega - \frac{1}{2}\sqrt{\alpha}, \omega + \frac{1}{2}\sqrt{\alpha}]$ en \ref{eq:2.24} se llama \textbf{ventana de frecuencia} y el producto cartesiano de estas ventanas se llama \textbf{ventana de tiempo-frecuencia}. Esta ventana de tiempo-frecuencia tiene un área constante $= (2w)(2h) = (2\triangle{e_{\alpha}})(2\triangle_{e_{\frac{1}{\alpha}}}) = 2$.



Por varias razones, como la eficiencia computacional o la conveniencia en la implementación, se pueden usar funciones distintas a la función gaussiana como funciones de ventana. En otras palabras, la transformada de Gabor se puede generalizar a cualquier otra \textbf{transformada de Fourier de ventana (transformada de Fourier de tiempo corto)} utilizando cualquier función de ventana $w_1$ tal que $\hat{w}_1$ también sea una función de ventana. Para $f \in \mathcal{L}_2(\mathbb{R})$, la transformada de Fourier de ventana con $w_1$ como la función de ventana se define como
\begin{equation}
    (W_b f)(\omega) = \int_{-\infty}^{\infty} (e^{-i\omega x} f(x))\overline{w_1(x - b)} dx.
\label{eq:2.25}
\end{equation}

Para $w_1(x) = e^{\alpha(x)}$, la transformada de Fourier de ventana \ref{eq:2.25} se reducirá a la transformada de Gabor \ref{eq:2.19}. Significa que estamos ventaneando la función $f$ utilizando la función de ventana $w(x) = (e^{i\omega x}w_1(x - b))$. Por lo tanto
\begin{equation}\label{eq:2.26}
(W_b f)(\omega) = \langle f,w \rangle = \frac{1}{2\pi} \langle \hat{f}, \hat{w} \rangle = \langle \hat{f}, h \rangle,
\end{equation}

donde $w(\hat{\eta}) = e^{-ib(\eta-\omega)} \hat{w}_1(\eta - \omega)$ y $h = \frac{1}{2\pi}\hat{w}$. La ventana de tiempo correspondiente a la función de ventana $w$ estará soportada en $[x^* + b - w_1 , x^* + b + w_1]$, donde $x^*$ y $w_1$ son el centro y el radio de la función de ventana $w_1$. La ventana de frecuencia correspondiente a la función de ventana $h$ estará soportada en $[\omega^* + \omega - \hat{w}_1 , \omega^* + \omega + \hat{w}_1]$, donde $\omega^*$ y $\hat{w}_1$ son el centro y el radio de la función de ventana $\hat{w}_1$. El área de la ventana es $4w_1\hat{w}_1$, que es nuevamente constante.

\textbf{Observación:}\; El principio de incertidumbre nos dice que no podemos encontrar una ventana más pequeña que la ventana gaussiana.

Ahora la pregunta es cómo recuperar $f$ de su transformada de Fourier de ventana, lo cual se puede hacer usando el siguiente teorema.

\begin{teorema}
    Sea $w_1 \in \mathcal{L}_2(\mathbb{R})$ tal que $||w_1||_2 = 1$ y tanto $w_1$ como $\hat{w}_1$ son funciones de ventana
    $$\int_{-\infty}^{\infty} \int_{-\infty}^{\infty} \langle f,w \rangle \langle g,w \rangle db d\omega = 2\pi \langle f, g \rangle,$$
    para cualquier $f, g \in \mathcal{L}_2(\mathbb{R})$ (note que $w = e^{i\omega x}w_1(x - b)$).
\end{teorema}

Usando el teorema anterior, podemos recuperar una función de su transformada de Fourier de ventana.



\chapter{Wavelet en geometrías planas}

Una wavelet (inventada por Morlet) es una función matemática utilizada para dividir una función dada o señal de tiempo continuo en diferentes componentes de escala. Las wavelets han alcanzado el crecimiento actual debido al análisis matemático de wavelets realizado por Stromberg [1], Grossmann y Morlet [2], Meyer [3] y Daubechies [4]. Por lo tanto, específicamente las wavelets unidimensionales ($\psi(x)$, $x \in \mathbb{R}$) son aquellas funciones que deben cumplir los siguientes requisitos:

• La función y su transformada de Fourier deben estar bien localizadas (es decir, se supone que la función tiene la mayor parte de su energía contenida en una región muy estrecha del dominio).

• $\int_{-\infty}^{\infty} \psi(x) dx = 0$ (oscilando por encima y por debajo del eje x).

Otros requisitos son deseables/técnicos y se necesitan principalmente para algunas aplicaciones específicas y para reducir la complejidad de los algoritmos numéricos para su implementación.



\end{multicols}


%------------------------------------------------------------------------------------------------
% -------------------- HERRAMIENTAS METODOLÓGICAS PARA LA INVESTIGACIÓN MATEMÁTICA --------------
%------------------------------------------------------------------------------------------------
% APUNTES HERRAMIENTAS METODOLÓGICAS
%\chapter{Convexidad y Optimización}

\section{Introducción}

$$
(P)
\left\{
\begin{array}{rl}
    \min & f_0(x)\\\\
    s.a. & f_1(x) \leq b_1\\
	 &f_2(x) \leq b_2\\
	 & \vdots\\
	 & f_m(x) \leq b_m.
\end{array}
\right.
\qquad
\begin{array}{rl}
    f_i: & \mathbb{R}^n \rightarrow \mathbb{R}\\
    f_0 : & \mbox{Función objetivo.}\\
    f_j : & \mbox{Función Restricción donde }j=1,\ldots,m.\\
\end{array}
$$
\begin{itemize}
    \item Las funciones objetivos en economía se les puede llamar función de coste.
    \item Las desigualdades tiene un truco, si multiplicamos por $(-1)$ tenemos en la forma que decidamos.
    \item Maximizar es lo mismo que minimizar. Por lo que minimizaremos las funciones. 
\end{itemize}

El objetivo de (P) es encontrar $x^*$ el optimo ($\arg\min$) que cumple 
$$f_0(x^*)\leq f_0(x), \;\forall x\in \mathbb{R}^n / f_j(x)\leq b_j,\; j=1,\ldots,m.$$
Será en cualquier $x$ que cumple las restricciones. Los puntos que no cumplen las condiciones no sirven para nada.

Al valor $f_0(x^*)$ se le llama valor optimo.

$f_i:  \mathbb{R}^n \rightarrow \mathbb{R}$ Existirá algunas funciones que su dominio sera tramposo.

Los Puntos factibles son los $x\in \mathbb{R}^n / f_j(x)\leq b_j,\; j=1,\ldots,m.$

\begin{itemize}
    \item  Si los problemas son lineales se llama programación lineal.
    \item Cuando es convexa se llama optimización convexa.
    \item La habilidad es de identificar las restricciones y pasarlas a convexas.
\end{itemize}

\begin{ejem}
Sean $A\in \mathcal{M}_{k\times n},\; \textbf{x}\in \mathbb{R}^n,\; \textbf{b}\in \mathcal{R}^k$.
$$
\begin{pmatrix}
    x_1\\
    x_2\\
    \vdots\\
    x_n
\end{pmatrix}
\in \mathbb{R}^n,
\qquad 
\textbf{x}^T=(x_1,x_2,\ldots,x_n).
$$
Diremos que el un vector cualquiera sera vector columna.

Ahora, el problema será una minimización global dada por:
$$
\left\{
\begin{array}{rl}
    \min: &\|A\textbf{x}-b\|^2_2\\
    s.a. & \emptyset.
\end{array}
\right.
$$

El subindice $_2$ significa la normal Euclidea. Que es la distancia normal que existe en $\mathbb{R}^2.$

El objetivo será encontrar la $x$ donde la operación dada será la menor posible.

\textbf{Nota}
Imaginemos que tenemos 
$$
\begin{array}{rl}
    \left\{\min\right. & f(x)\\\\
    \left\{ \min \right. & f_0^2(x)
\end{array}
$$

Si las función $f_0$ es positiva las dos formas son equivalentes. El valor optimo no será el mismo porque lo estoy elevando al cuadrado, pero el punto optimo lo será. Porque las funciones son monótomas crecientes. Si el valor al cuadrado me simplifica entonces podemos utilizarla. Esto nos permite que si no tengamos una función convexa podamos convexificarla.

Por diferenciabilidad:
$$f_0(x)=\|Ax-b\|_2^2 = \langle Ax-b,Ax-b\rangle.$$


\textbf{Notación.-} Podemos escribir $Ax$ como
$$
Ax = 
\underset{A^1}{
\begin{pmatrix}
    a_{11}\\
    a_{21}\\
    \vdots\\
    a_{k1}
\end{pmatrix}}
x_1+
\underset{A^2}{
\begin{pmatrix}
	a_{12}\\
	a_{22}\\
	\vdots\\
	a_{k2}
\end{pmatrix}}
x_2+
\cdots +
\underset{A^n}{
\begin{pmatrix}
	a_{1n}\\
	a_{2n}\\
	\vdots\\
	a_{kn}
\end{pmatrix}}
x_n
=
x_1A^1+x_2A^2+\cdots+x_nA^n.
$$
$A^1$ = A super 1 como columna, y $A_1$ = A super 1 como fila.

Ahora, en términos de filas. Si escribimos los vectores A en columna
$$
A = 
\begin{pmatrix}
	A^T_1\\
	A^T_2\\
	\vdots\\
	A^T_k
\end{pmatrix}
$$

Donde,
$$
Ax = 
\begin{pmatrix}
	A^T_1x\\
	A^T_2x\\
	\vdots\\
	A^T_nx
\end{pmatrix}
=
\begin{pmatrix}
	\langle A^T_1,x\rangle\\
	\langle A^T_2,x\rangle\\
	\vdots\\
	\langle A^T_n,x\rangle
\end{pmatrix}
$$

Intentaremos demostrar el punto donde las parciales de $f_0=0$. Para ello, encontraremos 
$$
\begin{array}{rcl}
    D_if_0&=&D_i\left(\langle Ax-b, Ax-b\rangle\right)\\\\
	  &=&\langle D_i\left(Ax-b\right),Ax-b\rangle+\langle Ax-b,D_i\left(Ax-b\right)\rangle\\\\
	  &=& 2\langle Ax-b,D_i\left(Ax-b\right)\rangle.
\end{array}
$$

Veamos la parcial de $D_i\left(Ax-b\right)$.
$$
\begin{array}{rcl}
    D_i\left(Ax-b\right)&=&D_i\left(x_1A^1+x_2A^2+\cdots+x_nA^n-b\right)\\\\
			&=&A^i.
\end{array}
$$
Dado que $b$ que es constante vale cero, y donde todos los suman que no estén las $x_i$ también valen cero.

Por lo tanto,
$$D_if_0 = 2\langle Ax-b,A^i \rangle.$$

Luego,
$$2\langle Ax-b,A^i \rangle = 0 \quad \forall i=1,\ldots,n \quad \Rightarrow \quad \langle Ax-b,A^i\rangle=0,\quad \forall i = 1,\ldots,n.$$

Observemos que,
$$
\overrightarrow{0} = 
\begin{pmatrix}
    \langle Ax-b,A^1\rangle\\
    \langle Ax-b,A^2\rangle\\
    \vdots\\
    \langle Ax-b,A^n\rangle
\end{pmatrix}
=
\begin{pmatrix}
    (A_1)^T\\
    (A_2)^T\\
    \vdots\\
    (A_n)^T
\end{pmatrix}
(Ax-b)
=A^T(Ax-b).
$$

En funciones convexas el extremo local será el mínimo global.
$$A^T(Ax-b)=\overrightarrow{0}\quad \Leftrightarrow\quad A^TAx=A^Tb.$$
Que es una ecuación normal.\\

\textbf{Argumentos geométricos}

Sera el mismo calcular el mínimo de la distancia, calcular:
$$\min \|Ax-b\|^2_2 = d(b_i,Ax)^2$$
Donde $Ax$ tendrá la forma geométrica, de un subespacio vectorial. en el caso de $\mathbb{R}^3$ será un plano.
\begin{center}
    \begin{tikzpicture}[scale=0.5]
      % Define los vértices del romboide
      \coordinate (A) at (0,0);
      \coordinate (B) at (4,0);
      \coordinate (C) at (9,2);
      \coordinate (D) at (5,2);
      % Dibuja el romboide y etiqueta sus vértices
      \draw (A) -- (B) -- (C) -- (D) -- cycle;
      % Que el plano se llame P
      \node [below right] at (B) {$\left\{Ax:x\in \mathbb{R}^n\right\}$};
      % trazar una linea entrecortada perpedicular al plano y los puntos extremos que tengan un punto negro
      \draw[dashed] (4,2.5)node[above left]{$b$} -- (4,.6)node[left]{$b_0$};
      \fill (4,2.5) circle (2pt);
      \fill (4,.6) circle (2pt);
    \end{tikzpicture}
\end{center}

Si $b\in \left\{Ax:x\in \mathbb{R}^n\right\}\quad \Leftrightarrow \quad x^* \in \mathbb{R}^n : Ax^* =b.$ El valor optimo $f_0(x^*)=0.$

Si $b\notin \left\{Ax:x\in \mathbb{R}^n\right\}$, $f_0(x^*)=d(b,b_0)^2$.

Ahora, cual es el optimo?; es decir cual es el $x^*$

Donde la solución es:
$$x^*\in \mathbb{R}^n : Ax^*=b_0.$$ Aquí, $b_0$ está en el plano, si estamos en $\mathbb{R}^3$. ¿Cómo llegamos algebraicamente?:
$$
\begin{array}{rcl}
b-b_0 \perp \left\{Ax:x\in \mathbb{R}^n\right\}&\Leftrightarrow & b-b_o \perp A^i,\; i=1,\ldots,n\\\\  
						 &\Leftrightarrow & \langle b-b_0,A^i\rangle=0,\; i=1,\ldots,n\\\\
							   &\Leftrightarrow& \langle b-Ax^*,A^i\rangle = 0, \; i=1,\ldots,n\\\\
							   &\Leftrightarrow& \langle Ax^*-b,A^i\rangle = 0, \; i=1,\ldots,n\\\\
								   &\leftrightarrow&A^TAx^*=A^Tb.
\end{array}
$$
\end{ejem}
Las ecuaciones normales vienen dadas por la perpendicularidad.

\section{Conjuntos convexos de \boldmath $\mathbb{R}^n$}

El dominio tendrán que ser conjuntos convexos o Dominio efectivo.

% -------------------- DEFINICIÓN 1 LINEAL
\begin{def.}[Lineal]\,\\ 
$$L(x_0,x_1) := \left\{x_0+\lambda(x_1-x_0):\lambda \in \mathbb{R}\right\}$$
\begin{center}
    \begin{tikzpicture}[scale=0.5]
      \coordinate (A) at (0,0);
      \coordinate (B) at (3,1);
      \coordinate (C) at (6,2);
      \coordinate (D) at (-3,-1);
      \draw[->] (A) -- (B)node[below]{$x_1$};
      \draw[dashed] (B) -- (C);
      \draw[dashed] (D) -- (A);
      \fill (A) circle (2pt) node[below]{$x_0$};
    \end{tikzpicture}
\end{center}
\end{def.}

\begin{itemize}
    \item Cuando $\lambda$ vale $1$ me sale $x_1$ cuando valga cero me sale $x_0$ cuando es positivo va hacia la derecha y cuando es negativo va hacia la izquierda.
    \item Toda la recta nos da un concepto que denominamos Afín. 
    \item Para la convexidad no es necesario tener la linea, solo necesitaremos un segmento. 
	$$\left\{x_0+\lambda(x_1-x_0):\lambda \in \left[a,b\right]\right\}$$
    \item El segmento importante será el intervalo que denotaremos como:
	$$\left[x_0,x_1\right]:=\left\{x_0+\lambda(x_1-x_0):\lambda \in \left[0,1\right]\right\}=\left\{(1-\lambda)x_0+\lambda x_1:\lambda \in \left[0,1\right]\right\}$$
	Es cualquier punto que este entre $x_0$ y $x_1$ del gráfico de arriba. 
\end{itemize}


% -------------------- DEFINICIÓN DE AFÍN
\begin{def.}
    Sea $A\subseteq \mathbb{R}^n$. Se dice \textbf{Afín}, si para todo $x, y\in A$ se tiene que la $L(x,y)\subseteq A$. (Subespacios vectoriales desplazados).
\end{def.}

\begin{itemize}
    \item Un circulo no es afín ya que la linea es infinita y un circulo no.
    \item Un plano podría ser Afín
    \item La recta es afín
    \item Todo $\mathbb{R}^n$ es afín.
    \item Un único punto también es afín, dado que $x=y$.
    \item La diferencia entre espacio vectorial y espacio afín es que el espacio afín esta desplazada; es decir, no necesariamente pasa por el cero como en un subespacio vectorial.
\end{itemize}

Manejar el concepto de afín con lineas es un poco incomodo, entonces se utiliza el concepto de combinación afín  

% -------------------- DEFINICIÓN DE combinación afín
\begin{def.}
    Una \textbf{combinación afín} de los vectores $\left\{x_1,x_2,\ldots,x_k\right\}$ es un vector de la forma 
$$\lambda_1 x_1+\lambda_2x_2+\cdots+\lambda_kx_k.$$
tal que 
$$\sum_{i=1}^k \lambda_i = 1.$$
\end{def.}

\begin{itemize}
    \item Lo que decimos que es una combinación lineal de $x_0$ y $x_1$.
    \item Lo demás puntos fuera del segmento son las combinaciones lineales de $x_0$ y $x_1$.
\end{itemize}

% -------------------- TEOREMA 1
\begin{teo}
    $A$ es afín sii $A$ contiene toda combinación afín de sus puntos.

	Demostración.-\; Primero, tomemos puntos arbitrarios $\left\{x_1,x_2,\ldots,x_k\right\}$ en $A$ tal que 
	$$z=\lambda_1x_1+\lambda_2x_2+\cdots+\lambda_kx_k$$
	donde $\sum\limits_{i=1}^k \lambda_i = 1$. 

	Ahora, consideremos dos puntos $x_i,x_j$ de $z$. Dado que $A$ es afín, entonces $L(x_i,x_j)\subseteq A$, para todo $x_i,x_j$. Esto implica que $z$ está en $A$. Intuitivamente, si 
	$$\lambda_1x_1+\lambda_2 x_2,\quad \lambda_3\lambda_3+\lambda_4\lambda_4,\quad \ldots,\quad  \lambda_{k-1}x_{k-1}+\lambda_kx_k$$ 
	están en $A$. Entonces, $z$ tendrá que estar en $A$.

	Para demostrar la otra implicación, tomemos dos puntos cualesquiera $x$ e $y$ en $A$.  Entonces, $L(x,y)\subseteq A$. Por lo tanto, $A$ es afín.
\end{teo}

\begin{itemize}
    \item Quiere decir que este conjunto es estable para combinaciones lineales muy similar al concepto de subespacio vectorial.
\end{itemize}
\vspace{.5cm}

% -------------------- NOTACIÓN 1
\begin{notacion}
La suma de Minkowski es la operación de conjuntos; es decir, si $A,E\subseteq \mathbb{R}^n$. Entonces,
$$A=x_0+E=\left\{x_0+e:e\in E \right\} \quad \mbox{o}\quad E=A-x_0=\left\{a-x_0:a\in A\right\}.$$ 
Es sencillamente trasladar los puntos del plano y desplazarlos o moverlos.
\end{notacion}

\begin{nota}
La definición de subespacio se refiere a tomar dos escalares y dos vectores, realizar la combinación lineal, donde esta combinación lineal no se saldrá del conjunto dado.
\end{nota}


% -------------------- TEOREMA 2
\begin{teo}
    $A\subseteq \mathbb{R}^n$ es afín sii existe un $E\subseteq \mathbb{R}^n$ subespacio vectorial tal que $A=x_0+E$ para todo $x_0\in A$.\\\\
	Demostración.-\; Supongamos que $A$ es afín y fijamos $x_0\in A$. Intentaremos probar que $E=A-x_0$ es un subespacio de $\mathbb{R}^n$, esto es equivalente a decir que:
	$$\lambda,\mu \in \mathbb{R}, e_i,e_2\subseteq E \quad \Rightarrow \quad \lambda e_i+\mu e_2\in E.$$
	Probemos que $\lambda e_1+\mu e_2\in E$; en otras palabras probaremos que $\lambda e_1+\mu e_2$ es $a-x_0$.
	$$
	\begin{array}{rcl}
	    \lambda e_1+\mu e_2&=&\lambda(a_1-x_0)+\mu(a_2-x_0)\\\\
			       &=&\lambda a_1 + \lambda a_2 - \lambda x_0-\mu x_0\\\\
			       &=&\lambda a_1 + \lambda a_2 - \lambda x_0-\mu x_0 +x_0-x_0\\\\
			       &=&\lambda a_1 + \lambda a_2+(1-\lambda-\mu)x_0-x_0.
	\end{array}
	$$
	Observemos que $\lambda a_1 + \lambda a_2+(1-\lambda-\mu)x_0$ está en $A$, dado a que $\lambda+\mu+(1-\lambda-\mu)=1$. Por lo tanto,
	$$A-x_0=E.$$
	Es un subespacio vectorial.

	Ahora, sabemos que $A\subseteq \mathbb{R}^n$ tal que $A=E+x_0$ para todo $x_0\in A$ es afín. Entonces, $E$ es un subespacio vectorial. Para demostrar que $A$ es afín, probaré que cualquier combinación afín de elementos de $A$ sigue estando en $A$. Sean,
	$$\left\{a_1,a_2,\ldots,a_k\right\},\; \lambda_1,\lambda_2,\cdots,\lambda_k:\sum \lambda_i=1.$$
	De donde,
	$$
	\begin{array}{rcl}
	    \lambda_1a_1+\lambda_2a_2+\cdots+\lambda_ka_k&=&\lambda_1(e_1-x_0)+\lambda_2(e_2-x_0)+\cdots+\lambda_k(e_k-x_0)\\\\
							   &=& \lambda_1e_1+\cdots+\lambda_ke_k+\left(\displaystyle\sum_{i=1}^k \lambda_i\right)x_0
	\end{array}
	$$
	Observemos que $\lambda_1e_1+\cdots+\lambda_ke_k$ es una combinación lineal afín el cual existe en $E$ y por definición, $\left(\displaystyle\sum_{i=1}^k \lambda_i\right)=1$. Por lo tanto,
	$$E+x_0=A.$$
\end{teo}

% -------------------- DEFINICIÓN 3 ENVOLTURA O SPAN AFÍN
\begin{def.}[Envoltura Afín]\,\\\\
    La envolura afín de $B$, $Aff(B)$, es el menor conjunto afín que contiene a $B$. Esto implica que es el conjunto de las combinaciones afines de elementos de $B$ o es la intersección de los conjuntos afines que contienen a $B$ 
\end{def.}

% -------------------- DEFINICIÓN
\begin{def.}
    Si $A$ es $Affin$ se llama "dimensión afín de $A$" a la dimensión de su espacio vectorial.
\end{def.}

\begin{itemize}
    \item Dimensión $0$ un punto.
    \item Dimensión $1$ una recta.
    \item Dimensión $2$ una plano.
\end{itemize}



% -------------------- EJEMPLO 2
\begin{ejem}
    Dado $C\in \mathbb{R}^n$ afín. Siempre existirán una matriz $A\in \mathcal{M}_{p\times n}$ y $b\in \mathbb{R}^p$ tal que
    $$C=\left\{x\in \mathbb{R}^n:Ax=b\right\}.$$\\
	Solución.-\; Cuál es el conjunto lineal asociado?, será el núcleo de la aplicación lineal; es decir,
	$$E=\left\{x\in \mathbb{R}^n:Ax=0\right\}.$$
	Cualquier solución de $x_0\in C$, de modo que $Ax_0=b$. Tomando un punto de $C$ y otro de $E$, tenemos 
	$$A(x_0+e)=Ax_0+Ae=b+0=b.$$
	Por lo tanto,
	$$C=\left\{x\in \mathbb{R}^n:Ax=b\right\}=x_0+E.$$
	Así, el conjunto afín no es más que el traslado del espacio vectorial.
\end{ejem}

% -------------------- Definición 1.6
\begin{def.}[Topología de \boldmath$\mathbb{R}^n$]\,\\\\
    Sean $A\subseteq \mathbb{R}^n$ y $a\in \mathbb{R}^n$.
    \begin{enumerate}[1)]
	\item $a\in A$ está en el interior de $A$ $\left(a\in \interior(A) \mbox{ o } \mathring{A} \right)$, cuando existe $\delta>0$ tal que $B(a,\delta)\subseteq A.$
	$$B(a,\delta)=\left\{x\in \mathbb{R}^n:\|x-a\|\leq \delta.\right\}.$$
	Por ejemplo en $\mathbb{R}^2$ será un circulo y $\mathbb{R}^3$ será una esfera.
	\item $A$ se dice abierto si $A=\interior(A)=\mathring{A}.$ (Los puntos frontera del conjunto $A$ serán los abiertos).
	\item Decimos que $c\in\mathbb{R}^n$ está en el cierre (o clausura) de $A$, cuando $\exists \left\{a_n\right\}\in A | a_n\to c$.
	\item Decimos que $A$ es cerrado cuando $A=\overline{A}$ donde $\left\{x\in \mathbb{R}^n: x \mbox{ está en el cierre de }A\right\}.$ (El cierre se divide en los puntos del interior y los puntos frontera).

	\item Se llama frontera de $A$, $\partial{A}$ a la intersección $\overline{A}\cap \left( \overline{\mathbb{R}\backslash A}\right)=\overline{A} \backslash \interior(A)$.
	\item $a\in \relint(A)$ si existe $\delta>0$ tal que $B(a,\delta)\cap Aff(A)\subseteq A$.\\
	    Lo que decimos es que la bola puede ser muy grande y vivir en $\mathbb{R}^3$ e intersecar en el plano, donde se corta transversalmente para proyectar la imagen.
    \end{enumerate}
\end{def.}

\begin{itemize}
    \item El concepto de punto interior será importante, porque me dice que me puedo acercar al punto $a$ desde todas las direcciones, y si es relativo interior me puedo acerca por todos los lados del conjunto.
    \item El punto de adherencia o clausura es un punto que me puedo acercar de alguna forma pero no de todas formas.
\end{itemize}

% -------------------- EJEMPLO 1.3
\begin{ejem}
    Dibujemos un plano
    \begin{center}
      \begin{tikzpicture}[scale=0.5]
	% Define los vértices del romboide
	\coordinate (A) at (-2,0);
	\coordinate (B) at (6,0);
	\coordinate (C) at (11,3);
	\coordinate (D) at (3,3);
	% Dibuja el romboide y etiqueta sus vértices
	\draw (A) -- (B) -- (C) -- (D) -- cycle;
	% Dibuja el contorno del riñón
	\coordinate (Center) at ($(A)!0.5!(C)$);
	\draw[] ($(Center)+(-1,.5)$) .. controls ($(Center)+(0.2,2)$) and ($(Center)+(3,-.1)$) .. ($(Center)+(1,-.4)$) .. controls ($(Center)+(0,-.5)$) and ($(Center)+(0,-1)$) .. ($(Center)+(-1,-.9)$) .. controls ($(Center)+(-1.5,-1)$) and ($(Center)+(-3,-.5)$) .. ($(Center)+(-1,.5)$);
	% Rellena el interior del riñón con líneas
	\pattern[pattern=north east lines, pattern color=gray!50] ($(Center)+(-1,.5)$) .. controls ($(Center)+(0.2,2)$) and ($(Center)+(3,-.1)$) .. ($(Center)+(1,-.4)$) .. controls ($(Center)+(0,-.5)$) and ($(Center)+(0,-1)$) .. ($(Center)+(-1,-.9)$) .. controls ($(Center)+(-1.5,-1)$) and ($(Center)+(-3,-.5)$) .. ($(Center)+(-1,.5)$);
	\node at ($(Center)+(-.2,.2)$) {\small$B$};
	\node at ($(Center)+(1.9,.7)$) {\small$A$};
      \end{tikzpicture}
    \end{center}
    \begin{itemize}
	\item $B$ es el interior con la frontera.
	\item $A$ es la frontera.
	\item El objetivo será encontrar el punto optimo de un esfera que está proyectada en este plano. 
	\item El conjunto tendrá que ser convexo.
    \end{itemize}
    Veamos algunas propiedades de este conjunto.
    \begin{enumerate}[1)]
	\item $A$ es cerrado.- Cualquier punto que ponga en $B$ me puedo acercar por puntos de $B$.
	\item $B$ cerrado.
	\item $\mathring{A}=\emptyset$.- Si yo ponga una bola, se saldrá del conjunto $A$.
	\item $\mathring{B}=\emptyset$.- Ya que no existirá en el plano ninguna esfera. 
	\item $relint(A)=\emptyset$.
	\item $relint(B)=B\backslash A.$
    \end{enumerate}
\end{ejem}

% -------------------- Definición 1.8
\begin{def.}[Combinación convexa]\,\\\\
    Sean $x_1,x_2,\ldots,x_k\in \mathbb{R}^n$ y $\lambda_1,\lambda_2,\ldots,\lambda_k\in \mathbb{R}$ tales que $\lambda_i\geq 0$ y $\displaystyle\sum_{i=1}^{k}\lambda_i=1$. Al vector
    $$\sum_{i=1}^{k}\lambda_ix_i=\lambda_1x_1+\lambda_2x_2+\ldots+\lambda_kx_k$$
    se le llama combinación convexa de los puntos $\left\{x_1,\ldots,x_k\right\}$.\\

	La única diferencia entre combinación convexa y afín es que sean positivos.
\end{def.}

% -------------------- Observación 1.1
\begin{obs}
    La definición para 2 puntos $\left\{x_1,x_2\right\}$ nos de las combinaciones convexas,
    $$\lambda x_1+(1-\lambda)x_2,\qquad \lambda\geq 0, (1-\lambda)\geq 0 \; \Leftrightarrow \; \lambda \in \left[0,1\right].$$
    Esto es el segmento,
    $$\left\{\lambda x_1+(1-\lambda)x_2: \lambda\in \left[0,1\right]\right\}=\left[x_1,x_2\right].$$
    Nos quedamos con el segmento que los une, eso nos permitirá utilizar las propiedades de los números reales. Por lo que podremos realizar análisis.
\end{obs}

% -------------------- Definición 1.9
\begin{def.}[convexo]\,\\\\
    Un conjunto $C\in \mathbb{R}^n$ se dice convexo cuando $C$ contiene las combinaciones convexas de sus puntos, (Decimos que $C$ es cerrado para las combinaciones convexas), si y sólo si
    $$\forall x_1,x_2\in C \Rightarrow \left[x_1,x_2\right]\subseteq C.$$
    Un conjunto es convexo si dados dos puntos el segmento que los une se queda adentro.
\end{def.}

% -------------------- EJEMPLO 1.4
\begin{ejem}
    \begin{multicols}{3}
	\begin{center}
	  \begin{tikzpicture}[scale=0.8]
	    % Dibuja el contorno del riñón
	    \coordinate (Center) at ($(A)!0.5!(C)$);
	    \draw[] ($(Center)+(-1,.5)$) .. controls ($(Center)+(0.2,2)$) and ($(Center)+(3,-.1)$) .. ($(Center)+(1,-.4)$) .. controls ($(Center)+(0,-.5)$) and ($(Center)+(0,-1)$) .. ($(Center)+(-1,-.9)$) .. controls ($(Center)+(-1.5,-1)$) and ($(Center)+(-3,-.5)$) .. ($(Center)+(-1,.5)$);
	    % Rellena el interior del riñón con líneas
	    \pattern[pattern=north east lines, pattern color=gray!50] ($(Center)+(-1,.5)$) .. controls ($(Center)+(0.2,2)$) and ($(Center)+(3,-.1)$) .. ($(Center)+(1,-.4)$) .. controls ($(Center)+(0,-.5)$) and ($(Center)+(0,-1)$) .. ($(Center)+(-1,-.9)$) .. controls ($(Center)+(-1.5,-1)$) and ($(Center)+(-3,-.5)$) .. ($(Center)+(-1,.5)$);
	    % trazar una linea que una dos puntos del contorno pero este fuere del conjunto
	    \draw[dashed](2.6,0.1)--(4,.55);
	    \node at (2.7,-.4) {\small No convexo};
	  \end{tikzpicture}

	  \begin{tikzpicture}[scale=0.7]
	    % Dibuja un óvalo
		\draw (0,0) ellipse (2.5cm and 1.2cm);
		% Agrega un patrón de líneas diagonales al óvalo
		\pattern[pattern=north east lines, pattern color=gray!50] (0,0) ellipse (2.5cm and 1.2cm);
		\node at (0,1.7) {\small Convexo};
	    \end{tikzpicture}

	  \begin{tikzpicture}[scale=0.6]
		\coordinate (A) at (0,0); % Vértice 1
		\coordinate (B) at (2,0); % Vértice 2
		\coordinate (C) at (3,1.73); % Vértice 3
		\coordinate (D) at (2,3.46); % Vértice 4
		\coordinate (E) at (0,3.46); % Vértice 5
		\coordinate (F) at (-1,1.73); % Vértice 6
		\draw (A) -- (B) -- (C) -- (D) -- (E) -- (F) -- cycle; % Dibuja el hexágono
		\draw[pattern=north east lines, pattern color=gray!50] (A) -- (B) -- (C) -- (D) -- (E) -- (F) -- cycle; % Aplica el patrón al hexágono
		\node at (1,4) {\small Convexo};
	    \end{tikzpicture}
	\end{center}
    \end{multicols}
\end{ejem}

Del gráfico 1) ¿Cuál es el menor conjunto convexo que lo contiene?
\begin{center}
  \begin{tikzpicture}[scale=0.8]
    % Dibuja el contorno del riñón
    \coordinate (Center) at ($(A)!0.5!(C)$);
    \draw[red] ($(Center)+(-1,.8)$) .. controls ($(Center)+(0.2,2)$) and ($(Center)+(3,-.1)$) .. ($(Center)+(1,-.5)$) .. controls ($(Center)+(0,-.8)$) and ($(Center)+(0,-1)$) .. ($(Center)+(-1,-.9)$) .. controls ($(Center)+(-1.5,-1)$) and ($(Center)+(-3,-.5)$) .. ($(Center)+(-1,.8)$);
    % Rellena el interior del riñón con líneas
    \pattern[pattern=north east lines, pattern color=gray!50] ($(Center)+(-1,.5)$) .. controls ($(Center)+(0.2,2)$) and ($(Center)+(3,-.1)$) .. ($(Center)+(1,-.4)$) .. controls ($(Center)+(0,-.5)$) and ($(Center)+(0,-1)$) .. ($(Center)+(-1,-.9)$) .. controls ($(Center)+(-1.5,-1)$) and ($(Center)+(-3,-.5)$) .. ($(Center)+(-1,.5)$);
    % trazar una linea que una dos puntos del contorno pero este fuere del conjunto
  \end{tikzpicture}
\end{center}

% ------------------- Definición 1.10
\begin{def.}
    se llama Envoltura convexa de $A$ al menor conjunto convexo que lo contiene $\Leftrightarrow$ la intersección de todos los convexos que contienen a $A$, denotado por $co(A)$.
    Es equivalente a decir que
    $$co(A)=\left\{\mbox{Combinación convexa de puntos de A.}\right\}$$
\end{def.}

% ------------------- Ejercicio 
\begin{ejer}
    Demostrar que la intersección de conjuntos convexos es convexo.\\\\
	Demostración.-\; 
\end{ejer}


% ------------------- Definición 1.11
\begin{def.}[Cono]\,\\\\
    Un conjunto $C\subseteq \mathbb{R}^n$ se llama cono si y sólo si
    $$\lambda x\in C \mbox{ si } x\in C,\; \lambda \geq 0.$$

    Contiene los rayos que pasan por el cero e intersecan a un punto dado.
    \begin{itemize}
	\item Un cono siempre contiene al origen.
	\item La envoltura cónica de un conjunto es $con(A)=\left\{\lambda : \lambda \geq 0, a\in A\right\}$ la intersección de todos los conos que contiene a $A$.
	\item Un cono $C$ es convexo si y sólo si 
	$$\lambda_1,\lambda_2\in C \Rightarrow \lambda_1x_1+\lambda_2x_2\in C, \; \forall \lambda_1,\lambda_2 \geq 0.$$
    \end{itemize}
\end{def.}

El hiperplano es un caso particular del estudio convexo.

% ------------------- Definición 1.12
\begin{def.}[Hiperplano]\,\\\\
    $H\subseteq \mathbb{R}^n$ es un \textbf{hiperplano} si existe $a\in \mathbb{R}^n \backslash\left\{0\right\}$ tal que
    $H=\left\{x\in \mathbb{R}^n : \langle a,x\rangle = a^T x = 0\right\} = a^\perp$
\end{def.}

% ------------------- 



%------------------------------------------------------------------------------------------------
% ----------------------------------------- WAVELETS --------------------------------------------
%------------------------------------------------------------------------------------------------
% APUNTES 
\chapter{Introducción y wavelets ortogonales }
\columnratio{0.7,0.3}
\setlength{\columnsep}{1cm} % Ajusta el espacio entre las columnas a 1cm

\begin{paracol}{2}
Sea $f=\left(f_1,\ldots,f_N\right)\in \mathbb{R}^N$$, N=2^k$, con $k$ Máximo nivel de descomposición. La energía viene dado por:
$$E(f)=\sum_{j=0}^{N}f_j^2=\|f\|^2.$$
Una aproximación $A$ a primer nivel viene dado por el promedio de la primera y segunda pareja de términos. Es decir, se aproxima a la señal original. Y el detalle $D$ viene dado por la diferencia de la primera y segunda pareja de términos. Es decir, aquello que necesito añadir para recuperar la señal original. 

Cuando se tiene una aproximación de último nivel la señal se queda plana, por ejemplo:
$$A^2(3.75,3.75,3.75,3.75).$$
En este caso, como las entradas son, $k=2$, en consecuencia $2^2=4$, solo se puede realizar dos niveles de aproximación. 

Ahora bien para entender de mejor manera el concepto de wavelets necesitamos recordar las siguientes definiciones de algebra lineal:
\switchcolumn[1]*{\noindent\scriptsize
	\begin{itemize}
	    \item Si tenemos un sistema ortonormal, cualquier elemento que sea combinación lineal es fácil obtener los coeficientes. Es decir, no hay que resolver un sistema de ecuaciones.
	    \item La proyección ortogonal es el punto más cercano a un subespacio. 
	    \item Si yo hago la imagen de dos vectores el producto de las imágenes es igual al producto de las entradas. 
	    \item La norma de partida sea igual a la norma de llegada. Es decir, es transformación ortonormal si conserva la energía.
	\end{itemize}
}
\switchcolumn[0]
\begin{itemize}
    \item Si $\left\{v_1,\ldots,v_m\right\}\subset \mathbb{R}^N$, la envoltura lineal de dicho conjunto de vectores en $lin\left\{v_1,\ldots,v_m\right\}$.
    \item Si $v,w\in\mathbb{R}^N$, su producto escalar es $v\cdot w = v_1w_1+\ldots+v_Nw_N$.
    \item Un sistema de vectores $\left\{v_1,\ldots,v_m\right\}$ es ortonormal si cada uno tiene norma $1$ y $v_i\cdot v_j=0$ para $i\neq j$. En ese caso, cada vector $u\in lin\left\{v_1,\ldots,v_m\right\}$ se expresa de forma única como $u=(u\cdot v_1)v_1+\cdots + (u\cdot v_m)v_m$.
    \item Dos subespacios $V,W\subset \mathbb{R}^N$ son ortogonales si cada elemento de $V$ es ortogonal a cada uno de $W$, y su suma es directa (y ortogonal), la cual denotamos por $V\oplus^\perp W$.
    \item Si $v\in \mathbb{R}^N$ y $W$ es un subespacio de $\mathbb{R}^N$, la proyección ortogonal $w$ de $v$ sobre $W$ es el único vector $w\in W$ tal que $v-w$ es ortogonal a $W$ (y coincide con el de mínima distancia en $W$ a $v$).
    \item Una aplicación lineal $T:\mathbb{R}^N \to \mathbb{R}^N$ es una transformación ortogonal si $Tu\cdot Tv=u\cdot v$ para $u,v\in \mathbb{R}^N$ arbitrarios (equivalencia a $\|Tu\|=\|u\|$ para todo $u$). También, que su expresión matricial en una base ortonormal sea una matriz $A$ ortogonal ($A\cdot A^T=I$).
\end{itemize}

\section{Wavelets de Haar}
Cuando se toma el sistema
$$\left\{v_1^1,\ldots,v_{N/2}^1,w_1^1,\ldots,w_{N/2}^1\right\}$$
de scaling y Wavelets, si se proyecta en el espacio de la scaling la proyección me sale $A^1$, y si proyecto ortogonalmente en el espacio de Wavelets la proyección me sale $D^1$. En otras palabras,
$$
\begin{array}{rcl}
    f&=&\left((f_1\cdot v_1^1)v_1^1+\cdots+(f_{N/2}\cdot v_{N/2}^1)v_{N/2}^1\right)\\\\
     &+& \left((f\cdot w_1^1)w_1^1+\cdots+(f\cdot w_{N/2}^1)w_{N/2}^1\right)\\\\
     &=&A^1+D^1.
\end{array}
$$
De esta manera, el espacio scaling a nivel 1 es la envoltura lineal de $v_1^1,\ldots,v_{N/2}^1$ y el espacio Wavelets a nivel 1 es la envoltura lineal de $w_1^1,\ldots,w_{N/2}^1$. Dado que scaling y wevelets son perpendiculares entre si, la suma directa de estos dos espacios nos da:
$$\mathbb{R}^N = V^1\oplus^\perp W^1.$$
Con $V$, $\dim N/2$ y $W$ $\dim N/2$.

Cuando vamos al siguiente nivel, la parte de detalles no las toco, si no que me fijo en la parte de aproximación. Y ese espacio lo vuelvo a partir en dos.De lo que sale:
$$V^1 = V^2\oplus^\perp W^2.$$
con $V^2$, $\dim N/4$ y $W^2$ $\dim N/4$.
Por lo tanto,
$$\mathbb{R}^N = V^2\oplus^\perp W^2\oplus^\perp D^1.$$

Los coeficientes son los productos escalares. 

Coleccionar las subseñales es de lo que consiste la tranformada de Haar. 

Las señales de aproximación $A$ se le llama también señales de tendencia y a las señales $D$ de detalle se le llama fluctuaciones.


\section{Wavelets de Daubechies (db2)}
En vez de utilizar 2 coeficientes, se utilizan 4 coeficientes en la parte de scaling y 4 coeficientes en la parte de Wavelets.

Ganamos en comparación de Haar, la compactación de energía.


\end{paracol}


\chapter{Análisis de Frecuencia}

\begin{paracol}{2}
Las Transformadas Wavelet tiene un reflejo en la frecuencia por lo que recordaremos como es la Transformada de Fourier.

\section{Análisis de la frecuencia en la Transformada Wavelet (DFT)}

Notación:
$$f=\left(f(0),f(1),\ldots,f(N-1)\right)\in \mathbb{R}^N.$$

\switchcolumn[1]*{\noindent\scriptsize
Geométricamente,
\begin{center}
    \begin{tikzpicture}[scale=.6]
	% Dibujar los ejes
	\draw[thick] (-2.1,0) -- (2.1,0);
	\draw[thick] (0,-2.1) -- (0,2.1);

	% Dibujar el círculo
	\draw (0,0) circle (2cm);

	% Dibujar los ángulos
	\foreach \x in {0,30,...,360} {
	    \draw[gray] (0,0) -- (\x:2cm);
	}

	% Etiquetar los ángulos
	\foreach \x/\xtext in {
	    0/J=0,
	    30/j=1,
	    330/j=N-1
	} {
	    \draw (\x:2.1cm) node[right]{$\xtext$};
	}
    \end{tikzpicture}
    \begin{itemize}
	\item Tomamos como entrada vectores en $\mathbb{R}$ y nos devuelve vectores en $\mathbb{C}$.
    \end{itemize}
\end{center}
}
\switchcolumn[0]
De lo que la transformada discreta de Fourier lo definimos como:
\begin{equation}
\hat{f}(k)=\sum_{j=0}^{N-1}f(j)e^{-i2\pi kj/N}, \quad k=0,1,\ldots,N-1.
\end{equation}

Comenzamos en $0$ debido a que $e^0=1$. Luego voy corriendo una posición cada vez el circulo en $N$ trozos. Cuando tenemos $j=1$. Entonces, $e^{-i2\pi k/N}$.

Ahora, la inversa de la transformada de Fourier está definida como:
\begin{equation}
    f(j)=\frac{1}{N}\sum_{k=0}^{N-1}\hat{f}(k)e^{i2\pi kj/N}, \quad j=0,1,\ldots,N-1.
\end{equation}

Otra de las propiedades de la transformada de Fourier es periódica. Es decir, para $k=N$, el ciclo se repite. Dado que, 
$$e^{i2\pi kj/N}=e^{i2\pi Nj/N}=e^{i2\pi j}=e^0=1.$$
Por la igualdad de Parseval, sabemos que la energía mas o menos se conserva. Es decir,
\switchcolumn[1]*{\noindent\scriptsize
    \begin{itemize}
	\item Cuando tomamos la energía $E(\hat{f})$ que serán números complejos, no tomamos los cuadrados de las entradas, si no los cuadrados de los módulos de las entradas.
    \end{itemize}
}
\switchcolumn[0]\noindent
$$E(f)=\dfrac{1}{N}E(\hat{f}).$$

\subsection{Implementación computacional}
Vemos que (1.1) tiene $N$ multiplicaciones para $N$ entradas $k$. Donde si aplicamos computacionalmente, se tiene un costo de orden de $N^2$. Por lo que, se tiene un costo computacional alto. Por lo tanto, usamos la trasformada rápida de Fourier (FFT). La cual tiene un costo computacional de $N\log N$.

Luego, para visualizar las señales de la transformada de Fourier, usamos el espectro de la señal. Es decir, consideramos los módulos al cuadrado de las entradas $|\hat{f}|^2$. La energía de $f$ resulta ser el valor medio del espectro
$$E(f)=\dfrac{1}{N}\sum_{k=0}^{N-1}|\hat{f}(k)|^2.$$

Que es la media del espectro, dado por:
$$\left(|\hat{f}(0)|^2+|\hat{f}(1)|^2+\ldots+|\hat{f}(N-1)|^2\right).$$

\switchcolumn[1]*{\noindent\scriptsize
	\begin{itemize}
	    \item Al descomponer una señal, lo que estoy realizando son proyecciones ortogonales.
	    \item Sabemos que con aproximación numérica se va promediando la energía.
	    \item En tema de detalle es como fluctúan los coeficientes entre si.
	\end{itemize}
}
A partir del espectro podemos analizar el efecto que produce el tratamiento de señales mediante wavelets a nivel de frecuencia.
\switchcolumn[0]
Ahora bien cuando separamos las señales, mediante promedios y diferencias, y aplicamos la transformada wavelet a estas separaciones de $A^1$ y $D^1$. Entonces, en $A^1$ se queda las frecuencias bajas y en $D^1$ las frecuencias altas. Esto se conoce como filtros pasa bajos y pasa altos. Esto lo podemos verificar en general a partir de las formulas de las señales de promedio y detalle y por la linealidad de la transformada de Fourier discreta:
$$A^1 = (f\cdot v_1^1)v_1^1 + \cdots + (f\cdot v^1_{N/2}v^1_{N/2}.$$
$$D^1 = (f\cdot w_1^1)w_1^1 + \cdots + (f\cdot w^1_{N/2}w^1_{N/2}.$$
Aplicando la transformada de Fourier a $A^1$ y $D^1$ se tiene:
$$\hat{A}^1 = (f\cdot v_1^1)\hat{v}_1^1 + \cdots + (f\cdot v^1_{N/2})\hat{v}^1_{N/2} \quad (\text{Deja pasar frec bajas}).$$
$$\hat{D}^1 = (f\cdot w_1^1)\hat{w}_1^1 + \cdots + (f\cdot w^1_{N/2})\hat{w}^1_{N/2} \quad (\text{Dejan pasar las altas}).$$


Ahora para el nivel 2, hago una aproximación promediando más términos por lo que la meseta lo estrecho, es decir mi filtro es de frecuencias más bajas. Recordando la composición de la señal:
$$f = A^2 + D^2 + D^1.$$
Donde $D^1$ ya se tiene la frecuencias altas, $A^2$ tiene las frecuencias bajas y $D^2$ tiene las frecuencias medias o filtro pasa bandas, con
$$A^2=\sum_{j=0}^{N/2-1}f(2j)\hat{v}^2_j,$$ 
pasa bajas y
$$D^2=\sum_{j=0}^{N/2-1}f(2j+1)\hat{w}^2_j.$$
pasa bandas.

\end{paracol}


\end{document}
