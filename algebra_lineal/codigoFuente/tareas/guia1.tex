
\section*{\center Guía 1}
\vspace{1.2cm}
\setlength{\columnsep}{.7cm}
\setlength{\columnseprule}{0.1pt}
\begin{multicols}{2}

\begin{enumerate}[\large\bfseries 1.]

    %----------1.
    \item  \textbf{\boldmath Suponga que $a,b$ son números reales no nulos simultáneamente. Hallar números reales $c$ y $d$ tales que,
	$$\dfrac{1}{a+bi}=c+di$$\\}
	\textbf{Respuesta.-}\; Hallemos $c\in \mathbb{R}$ como sigue,
	$$\begin{array}{rcl}
	    \dfrac{1}{a+bi}&=&c+di\\
	    \dfrac{1}{a+bi} - di&=&c+di - di\\
	    c&=&\dfrac{1}{a+bi}-di\\
	\end{array}$$

	Ya que, $i^4 = i^3i=(-i)i=-(i^2) = -(-1) = 1\; $
	y $$i^3 = i^2i=(-1)i = -i,$$
	Luego hallamos $d\in \mathbb{R}$,
	$$\begin{array}{rcl}
	    \dfrac{1}{a+bi}&=&c+di\\
			   di&=&\dfrac{1}{a+bi}-c\\
			   di\cdot i^3&=&i^3\left(\dfrac{1}{a+bi}-c\right)\\
			   d&=&\dfrac{-i}{a+bi}+ci\\
	\end{array}$$

	Así, $c=\dfrac{1}{a+bi}-di\;$ ; $\; d=\dfrac{-i}{a+bi}+ci$.\\\\

    %----------2.
    \item \textbf{\boldmath Hallar dos raíces cuadradas distintas de $i$.}\\\\
	\textbf{Respuesta.-}\; $x^2-1 = 0\quad $ y $\quad x^2-4=0$.\\\\  

    %----------3.
    \item \textbf{\boldmath Probar que $\alpha+\beta = \beta +\alpha$, para todo $\alpha,\beta \in \mathbb{C}$.}\\\\
	\textbf{Demostración.-}\; Sea $\alpha = a+bi$ y $\beta = c+di$, entonces por definición de números complejos para la adición, tenemos que,
	$$\begin{array}{rcl}
	    \alpha + \beta &=&(a+bi)+(c+di)\\
			   &=&(c+di)+(a+bi)\\
			   &=&\beta + \alpha\in \mathbb{C}\\
	\end{array}$$
	De donde se demuestra la proposición dada.\\\\

    %----------4.
    \item \textbf{\boldmath Probar que $(\alpha+\beta)+\lambda = \alpha + (\beta + \lambda)$, para todo $\alpha,\beta,\lambda \in \mathbb{C}$.}\\\\
	\textbf{Demostración.-}\; Sea $\alpha = a+bi$,  $\beta = c+di$ y $\lambda = e + fi$ entonces,
	$$\begin{array}{rcl}
	    (\alpha + \beta) + \lambda &=&\left[(a+bi)+(c+di)\right]+(e+fi)\\
				       &=&(a+bi)+\left[(c+di)+(e+fi)\right]\\
				       &=&\beta + (\alpha + \lambda)\\
	\end{array}$$
	Así, $(\alpha + \beta) + \lambda = \beta + (\alpha + \lambda).$\\\\


    %----------5.
    \item \textbf{\boldmath Probar que para todo $\alpha \in \mathbb{C}$, existe un único $\beta \in \mathbb{C}$ tal que $\alpha + \beta = 0$.}\\\\
	\textbf{Demostración.-}\; La existencia queda demostrada por la propiedad identidad para la adición.\\
	Ahora demostremos su unicidad de la siguiente manera:\\
	Supongamos que existen  $\beta^{'},\; \beta \in \mathbb{C}$ tales que $\alpha + \beta = 0\;$ y $\;\alpha + \beta^{'} = 0$ que implica,
	$$\alpha + \beta = \alpha + \beta^{'} \; \Longrightarrow \;\beta=\beta^{'}.$$
	Y por lo tanto, queda demostrada la unicidad.\\\\
	Demostrada la existencia y unicidad concluimos que se cumple la propiedad del inverso aditivo para $\mathbb{C}$.\\\\
	

    %----------6.
    \item \textbf{ \boldmath Probar que para todo $\alpha \in \mathbb{C}-\lbrace0 \rbrace,$ existe un único $\beta \in \mathbb{C}$ tal que $\alpha \beta = 1$.}\\\\
	\textbf{Demostración.-}\; Similar al anterior ejercicio podemos demostrar la existencia de $\beta$ por la propiedad de identidad para la multiplicación. Luego demostremos la unicidad de la siguiente manera:\\\\
	Sean $\beta, \beta^{'}\in \mathbb{C}$ tales que $\alpha \beta = 1\; $ y $\; \alpha \beta = 1$  entonces $$\alpha\beta = \alpha \beta^{'}$$
	 como $\alpha\neq 0$, nos queda que, $\beta=\beta{'}$.\\\\ 
	 Así, queda demostrada la propiedad del inverso multiplicativo para $\mathbb{C}$.\\\\

    %----------7.
     \item \textbf{ \boldmath Hallar $x\in \mathbb{R}^4$ tal que $(4,-3,1,7)+2x=(5,9,-6,8)$.}\\\\
	 \textbf{Respuesta.-}\; se tiene que,
	 $$\begin{array}{rcl}
	     (4,-3,1,7)+2x&=&(5,9,-6,8)\\
			  2x&=&(5,9,-6,8)-(4,-3,1,7)\\
			    2x&=&(5-4,9+3,-6-1,8-7)\\
			    2x&=&(1,12,-7,1)\\
	 \end{array}$$
	 De donde $x=\left(\dfrac{1}{2},6,-\dfrac{7}{2},\dfrac{1}{2}\right).$\\\\

    %----------8.
     \item \textbf{\boldmath Probar que $(x+y)+z=x+(y+z)$ para todo $x,y,z\in \mathbf{F}^n$.}\\\\
	 \textbf{Demostración.-}\; Sea 
	 $$\begin{array}{rcl}
	     x&=&(x_1,\ldots,x_n)\\ 
	     y&=&(y_1,\ldots,y_n)\\
	     z&=&(z_1,\ldots,z_n)
	 \end{array}$$ 
	     entonces,

	 $$\begin{array}{rcl}
	     (x+y)+z&=&\left[(x_1,\ldots,x_n)+(y_1,\ldots,y_n)\right]\\
		    &+&(z_1,\ldots,z_n)\\\\
		    &=&(x_1+y_1,\ldots,x_n+y_n)\\
		    &+&(z_1,\ldots ,z_n)\\\\
		    &=&(x_1+y_1+z_1,\ldots,x_n+y_n+z_n)\\
	 \end{array}$$

	 $$\begin{array}{rcl}
	     x+(y+z)&=&(x_1,\ldots,x_n)\\
		    &+&\left[(y_1,\ldots,y_n)+(z_1,\ldots,z_n)\right]\\\\
		    &=&(x_1,\ldots,x_n)\\
		    &+&(y_1 + z_1,\ldots ,y_n + z_n)\\\\
		    &=&[x_1+(y_1+z_1),\ldots,x_n+(y_n+z_n)]\\
	 \end{array}$$

	 De donde se tiene que,
	 $$(x+y)+z=x+(y+z).$$\\

    %----------9.
     \item \textbf{\boldmath Probar que $(ab)x=a(bx)$ para todo $x\in F^n$ y para todo $a,b \in F$.}\\\\
	 \textbf{Demostración.-}\; Sea $x=(x_1,\ldots,x_n)$ un elemento arbitrario de $\mathbb{F}_n$ y $a,b\in \mathbb{F}$ entonces, por la multiplicación de un escalar en $\mathbb{F}$ tenemos,
	 $$\begin{array}{rcl}
	     (ab)x&=&(ab)(x_1,\ldots,x_n)\\
		  &=&((ab)x_1,\ldots,(ab)x_n)\\
		  &=&(a(bx_1),\ldots,a(bx_n))\\
		  &=&a(bx_1,\ldots,bx_n)\\
		  &=&a(bx).\\
	 \end{array}$$
	 De ésta manera se demuestra que la propiedad dada.\\\\

    %----------10.
     \item \textbf{\boldmath Probar que $1x=x$ para todo $x\in F^n$.}\\\\
	 \textbf{Demostración.-}\; Similar al anterior ejercicio, sea $x=(x_1,\ldots,x_n) \in \mathbb{F}^n$ entonces,
	 $$\begin{array}{rcl}
	     1(x_1,\ldots,x_n)&=&(1\cdot x_1,\ldots,1\cdot x_n)\\
			      &=&(x_1,\ldots,x_n)\\
			      &=&x\\
	 \end{array}$$
	 Así, se demuestra que la propiedad dada.\\\\

    %----------11.
     \item \textbf{\boldmath Probar que $\lambda(x+y)=\lambda x + \lambda y,$ para todo $\lambda \in \mathbb{F}$ y todo $x,y\in \mathbb{F}^n$.}\\\\
	 \textbf{Demostración.-}\; Sea en $\mathbb{F}^n$ $$x=(x_1,\ldots,x_n)\quad \mbox{e}\quad y=(y_1,\ldots,y_n).$$ 
	 Luego por la multiplicación escalar y la adición en $\mathbb{F}$ se tiene,  
    	 $$\begin{array}{rcl}
	     \lambda(x+y)&=&\lambda(x_1+y_1,\ldots,x_n+y_n)\\
			 &=&[\lambda(x_1+y_1),\ldots,\lambda(x_n+y_n)]\\
			 &=&(\lambda x_1 + \lambda y_1,\ldots,\lambda x_n + \lambda y_n)\\
			 &=&(\lambda x_1,\ldots,\lambda x_n)+(\lambda y_1,\ldots,\lambda y_n)\\
			 &=&\lambda(x_1,\ldots,x_n)+\lambda(y_1,\ldots,y_n)\\
			 &=&\lambda x + \lambda y\\
	 \end{array}$$
	Por lo tanto queda demostrado la propiedad.\\\\

    %----------12.
    \item \textbf{\boldmath Probar que $(a+b)x=ax+bx,$ para todo $a,b\in \mathbb{F}$ y todo $x\in F^n$.}\\\\
	\textbf{Demostración.-}\; Sea $x=(x_1,\ldots,x_n)\in \mathbb{F}^n$ y de la definición de multiplicación de un escalar y adición en $\mathbb{F}^n$ se tiene,
	$$\begin{array}{rcl}
	     (a+b)x&=&(a+b)(x_1,\ldots,x_n)\\
		   &=&[(a+b)x_1,\ldots,(a+b)x_n]\\
		   &=&(ax_1+bx_2,\ldots,a x_n + b_n)\\
		   &=&a(x_1,\ldots, x_n)+b( x_1,\ldots, x_n)\\
		   &=&ax + bx\\
       \end{array}$$
       De donde queda demostrado la proposición planteada.\\\\ 

    %----------13.
   \item \textbf{\boldmath Probar que $-(-v)=v$ para todo $v\in V$.}\\\\
       \textbf{Demostración.-}\; Por definición $-(-v)$ es un vector que satisface la ecuación 
       $$-v+(-v)=0$$
       y por el teorema de la unicidad del inverso aditivo sabemos que este vector es único. Luego, añadiendo el vector $v$ a ambos lados de la ecuación tenemos, 
       $$\begin{array}{rcll}
	   v+[-v+-(-v)]&=&v+0&\\
	   (v-v)-(-v)&=&v&\mbox{asoc. y def. de}\; 0\\
       0+[-(-v)]&=&v&\mbox{def. de}\; $-v$\\
       -(-v)+0&=&v&\mbox{conmut.}\\
       -(-v)&=&v&\mbox{def. de}\; 0\\
       \end{array}$$
       por lo tanto el resultado está probado.\\\\

    %----------14.
    \item \textbf{\boldmath Supóngase que $a\in F, v\in V$, y $av=0$. Probar que $a=0$ o $v=0$.}\\\\
	\textbf{Demostración.-}\; Si $a=0$ entonces la demostración es inmediata. Ahora suponga $a\neq 0$, con $a\in F$, de donde $a$ tiene un inverso multiplicativo $a^{-1}$. Sea $v\in V$ por lo tanto
	$$\begin{array}{rcl}
	    av&=&0\\
	    a^{-1}(av)&=&a^{-1}0\\
	    (a\cdot a^{-1})v&=&0\\
	      1v&=&0\\
	      v&=&0\\
	\end{array}$$
	Por lo tanto se demuestra que $a=0$ o $v=0$.\\\\

    %----------15.
    \item \textbf{\boldmath Supóngase que $v,w\in V,$. Explique por que existe un único $x\in V$ tal que $v+3x=w$.}\\\\
	\textbf{Demostración.-}\; Para poder responder a la pregunta planteada tendremos que demostrar la existencia y la unicidad de la proposición dada. \\\\
	 \textbf{Existencia.} Ya que $v,w\in V$ entonces por la adición y multiplicación de vectores se tiene que $x=\dfrac{1}{3}(w-v)$ también pertenece a $V$ por lo que queda demostrado su existencia.\\\\
	 \textbf{Unicidad.} Sean dos ecuaciones,
	$$v+3x_1=w\quad \mbox{;} \quad v+3x_2=w$$
	de donde podemos decir que, $$v+3x_1=v+3x_2$$
	Por lo tanto añadiendo $-v$ a ambos lados nos queda,
	$$\begin{array}{rcl}
	    -v+(v+3x_1)&=&-v+(v+3x_2)\\
	    (-v+v)+3x_1&=&(-v+v)+3x_2\\
	    3x_1&=&3x_2\\
		x_1&=&x_2\\
	\end{array}$$
	de ésta manera, queda demostrada la unicidad. Y por ende respondimos a la pregunta dada.\\\\

    %----------16.
    \item \textbf{El conjunto vacío no es un espacio vectorial. El conjunto vacío satisface solo uno de los requisitos enumerados en 1.19. ¿Cuál?.}\\\\
	\textbf{Respuesta.-}\; El conjunto vacío no es un espacio vectorial porque no contiene ninguna identidad aditiva $0$ (no tiene ningún vector).\\\\

    %----------17.
    \item \textbf{\boldmath Sea $\infty$ y $-\infty$ denota dos objetos distintos, ninguno de los cuales esta en $\mathbb{R}$.}\\\\
	\textbf{Respuesta.-}\; No logro entender la pregunta.\\\\

    %----------18.
    \item \textbf{\boldmath Para cada uno de los siguientes subconjuntos de $F^3$, determinar cuales son subespacios de $F^3$}.\\\\
	Para demostrar que un conjunto $A$ es un subespacio, necesitamos mostrar que contiene $0$, que es cerrado bajo la adición y bajo la multiplicación escalar. Por lo que solo necesitaremos demostrar que una de estas condiciones no se cumple.\\
	$0\in A$
		Supóngase $a=(a_1,a_2,a_3)\in A$ y $b=(b_1,b_2,b_3)\in A$. Es suficiente demostrar que $ja+kb \in A$ para todo escalar $j$ y $k.$ Esto cubre la adición como la multiplicación escalar.
		$$\begin{array}{rcl}
		    &&(ja_1+kb_1)+2(ja_2+kb_2)+3(ja_3+kb_3)\\
		    &=&j(a_1+2a_2+3a_3)+k(b_1+2b_2+3b_3)\\
		    &=&j\cdot 0 + k\cdot 0\\
		    &=&0\\\\
		\end{array}$$

	\begin{enumerate}[\bfseries a)]
	    \item $\lbrace (x_1,x_2,x_3) \in F^3\; : \; x_1+3x_2 + 3x_3=0 \rbrace$;\\\\
		Respuesta.-\; Sea $A=\lbrace (x_1,x_2,x_3) \in F^3\; : \; x_1+2x_2+3x_3=0\rbrace$. $A$ satisface las tres condiciones necesarias para ser un subespacio.\\\\

	    \item $\lbrace (x_1,x_2,x_3) \in F^3\; : \; x_1+2x_2 + 3x_3=4 \rbrace$;\\\\
		Respuesta.-\; Sea $A=\lbrace ()\in F^3\; : \; x_1x_2x_3=0\rbrace$. $A$ no es un subespacio porque no es cerrado bajo la adición. $(1,1,0)$ y $(0,0,1)$ ambos son en $A$, pero $(1,1,1)=(1,1,0)+(0,0,1)$ no es.\\\\

	    \item $\lbrace (x_1,x_2,x_3) \in F^3\; : \; x_1=4x_3 \rbrace$.\\\\
		Respuesta.-\; Sea $A=\lbrace (x_1,x_2,x_3)\in F^3\; : \; x_1=5x_3\rbrace$. $A$ satisface las tres condiciones necesarias para ser un subespacio.\\\\

	\end{enumerate}

    %----------19.
    \item \textbf{\boldmath Probar que el conjunto de funciones diferenciales $f$ con valores reales definida sobre el intervalo $(-4,4)$ tal que $f^{'}(-1)=3f(2)$, es un subespacio de $\mathbb{R}^{(-4,4)}$}.\\\\
	\textbf{Demostración.-}\; Sea $A$ el conjunto de funciones diferenciables de valor real en $(-4,4)$ tal que $f^{'}(-1)=3f(2)$.\\
	$A \in A$.\\
	Sea $f,g\in A$. Entonces por definición $$f^{'}(-1)=3f(2)\qquad g^{'}(-1)=3g(2)$$
	Luego,
	$$\begin{array}{rcl}
	    &&f^{'}(-1) + g^{'}(-1)=3[f(2)+g(2)]\\
	    &\Longrightarrow&(f+g)^{'}(-1)=3(f+g)(2)\\
	    &\Longrightarrow&(f+g)\in A\\
	\end{array}$$
	Ahora sea $\alpha \in \mathbb{R}, \; \alpha f^{'}(-1)=3\alpha f(2)$. De donde $(\alpha f)^{'}(-1)=3(\alpha f)(2)$. Esto implica para $f\in A, \; \alpha f \in A$. Por lo tanto para $f,g\in A,\; f+g\in A$ y $\alpha f \in A.$. Así esto demuestra que $A$ es un subespacio de $(-4,4)$.\\\\
	

    %----------20.
    \item Supóngase que $b\in \mathbb{R}$. Demostrar que el conjunto de funciones continuas $f$ de valores reales definidas sobre el intervalo $[0,1]$ tal que $\int_0^1 f=0$, es un subconjunto de $\mathbb{R}^{[0,1]}$ si y sólo si $b=0$.\\\\
	\textbf{Demostración.-}\;

    %----------21.
    \item \textbf{\boldmath Es $\mathbb{R}^2$ un subespacio de los números complejos $\mathbb{C}^2$}.\\\\
	\textbf{Respuesta.-}\; Si $\mathbb{R}^2$ es un subespacio de $\mathbb{C}^2$ entonces $\mathbb{R}^2$ es cerrado sobre la multiplicación escalar; verificaremos esto.\\
	Sea $\lambda$ un escalar para $\mathbb{C}$. Si $\lambda = i$  y  tenemos $(0,1)$, que pertenece a $\mathbb{R}^2$, entonces $i(0,1)=(i,0)$, el cual no pertence a $\mathbb{R}^2$, de hecho podemos tomar cualquier $(x_1,x_2)$ para $\mathbb{R}^2$ (no necesariamente $(1,0)$) tal que $x_1\neq 0$ o $x_2 \neq 0$, y podemos demostrar que $i(x_1,x_2)=(ix_1,ix_2)$ que no pertenece a $\mathbb{R}^2$. De donde concluimos que $\mathbb{R}^2$ no es un subespacio de $\mathbb{C}^2.$\\\\

    %----------22.
    \item \textbf{\boldmath Probar o refutar los siguientes:}

	\begin{enumerate}[\bfseries a)]

	    \item \textbf{\boldmath Es ${(a,b,c)\in \mathbb{R}^3:a^3=b^3}$ un subespacio de $\mathbb{R}^3$.}\\\\
		\textbf{Respuesta.-}\; 

	    \item \textbf{\boldmath Es ${(a,b,c)\in \mathbb{C}^3:a^3=b^3}$ un subespacio de $\mathbb{C}^3$.}\\\\
		\textbf{Respuesta.-}\; 

	\end{enumerate}

    %----------23.
    \item \textbf{\boldmath Dar un ejemplo de un conjunto no vacío $U$ de $\mathbb{R}^2$ tal que $U$ es cerrado sobre la adición y los inversos aditivos ($-u\in U$, siempre que $u\in U$), pero $U$ no es un subespacio de $\mathbb{R}^2$.}\\\\
	\textbf{Respuesta.-}\;

    %----------24.
    \item \textbf{\boldmath Dar un ejemplo de un conjunto no vacío $U$ de $\mathbb{R}^2$  tal que $U$ es cerrado sobre la multiplicación escalar, pero $U$ no es un subespacio de $\mathbb{R}^2$.}\\\\
	\textbf{Respuesta.-}\;

    %----------25.
    \item \textbf{\boldmath Supóngase que $U_1$ y $U_2$ son subespacios del espacio vectorial $V$. Demostrar que la intresección $U_1 \cap U_2$ es un subespacio de $V$.}\\\\ 
	\textbf{Demostración.-}\;

    %----------26.
    \item \textbf{\boldmath Probar que la unión de dos subespacios de $V$ es un subespacio de $V$ si, y solo si uno de ellos esta contenido en el otro.}\\\\
	\textbf{Demostración.-}\;

    %----------27.
    \item \textbf{\boldmath Probar que la unión de tres subespacios de $V$ es un subespacio de $V$ si, y solo si uno de los subespacios contiene a los otros dos.}\\\\
	\textbf{Demostración.-}\;

    %----------28.
    \item \textbf{\boldmath Supóngase que $U$ es un subespacio de $V$. ¿Es $U+U$ un subespacio de $V$?.}\\\\
	\textbf{Respuesta.-}\;

    %----------29.
    \item \textbf{\boldmath Probar o refutar que si $U,W$ son subespacios de $V$, entonces $U+W=W+U$. }\\\\
	\textbf{Respuesta.-}\;

    %----------30.
    \item \textbf{\boldmath Probar o dar un contraejemplo: si $U_11, U_2, W$ son subespacios de $V$ tal que $U_1+W = U_2+W$, entonces $U_1 = U_2$.}\\\\
	\textbf{Respuesta.-}\;

    %----------31.
    \item \textbf{\boldmath Supóngase que $U=\lbrace (x,x,y,y)\in F^4\; : \; x,y \in F\rbrace$. Hallar un subespacio $W$ de $F^4$ tal que $F^4=U\oplus W$.}\\\\
	\textbf{Respuesta.-}\;

    %----------32.
    \item \textbf{\boldmath Supongase que $U=\lbrace (x,y,x+y,x-y,2x)\in F^5\; : \; x,y\in F\rbrace$. Hallar un subespacio $W$ de $F^5$ tal que $F^5=U\oplus W$.}\\\\
	\textbf{Respuesta.-}\;

    %----------33.
    \item \textbf{\boldmath Probar o dar un contra ejemplo: si $U_1,U_2,W$ son subespacios de $V$ tal que $V=U_1\oplus W$ y $V=U_2\oplus W,$ entonces $U_1=U_2$}\\\\
	\textbf{Respuesta.-}\;

    %----------34.
    \item \textbf{\boldmath Una función $f:{\mathbb{R} \to \mathbb{R}}$ es llamada par si $f(-x)=f(x),$ para todo $x\in \mathbb{R}$. Una función $f:{\mathbb{R}\to \mathbb{R}}$ es llamada impar si $f(-x)=-f(x)$, para todo $x\in \mathbb{R}$. Sea $U_p$ el conjunto de todas las funciones pares de valores reales definidas en $\mathbb{R}$, y sea $U_i$ el conjunto de todas las funciones impares de valores reales definidas en $\mathbb{R}$. Probar que $\mathbb{R}=U_p \oplus U_i$.}\\\\
	\textbf{Demostración.-}\;






\end{enumerate}


\end{multicols}
