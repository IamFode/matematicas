
\section*{\center Guía 1}
\vspace{1.5cm}
\setlength{\columnsep}{1cm}
\setlength{\columnseprule}{0.1pt}
\begin{multicols}{2}

\begin{enumerate}[\bfseries 1.]

    %----------1.
    \item  \textbf{\boldmath Suponga que $a,b$ son números reales no nulos simultáneamente. Hallar números reales $c$ y $d$ tales que,
	$$\dfrac{1}{a+bi}=c+di$$\\}
	\textbf{Respuesta.-}\; Hallemos $c\in \mathbb{R}$ de la siguiente manera,
	$$\begin{array}{rcl}
	    \dfrac{1}{a+bi}&=&c+di\\
	    \dfrac{1}{a+bi} - di&=&c+di - di\\
	    c&=&\dfrac{1}{a+bi}-di\\
	\end{array}$$

	Ya que, $i^4 = i^3i=(-i)i=-(i^2) = -(-1) = 1\; $
	y $$i^3 = i^2i=(-1)i = -i,$$
	podemos hallamos $d\in \mathbb{R}$,
	$$\begin{array}{rcl}
	    \dfrac{1}{a+bi}&=&c+di\\
			   di&=&\dfrac{1}{a+bi}-c\\
			   di\cdot i^3&=&i^3\left(\dfrac{1}{a+bi}-c\right)\\
			   d&=&\dfrac{-i}{a+bi}+ci\\
	\end{array}$$

	Así, $c=\dfrac{1}{a+bi}-di\;$ ; $\; d=\dfrac{-i}{a+bi}+ci$.\\\\

    %----------2.
    \item \textbf{\boldmath Hallar dos raíces cuadradas distintas de $i$.}\\\\
	\textbf{Respuesta.-}\; $x^2-1 = 0\quad $ y $\quad x^2-4=0$.\\\\  

    %----------3.
    \item \textbf{\boldmath Probar que $\alpha+\beta = \beta +\alpha$, para todo $\alpha,\beta \in \mathbb{C}$.}\\\\
	\textbf{Demostración.-}\; Sea $\alpha = a+bi$ y $\beta = c+di$, entonces por definición de números complejos para la adición, tenemos que,
	$$\begin{array}{rcl}
	    \alpha + \beta &=&(a+bi)+(c+di)\\
			   &=&(c+di)+(a+bi)\\
			   &=&\beta + \alpha\in \mathbb{C}\\
	\end{array}$$
	De donde se demuestra la proposición dada.\\\\

    %----------4.
    \item \textbf{\boldmath Probar que $(\alpha+\beta)+\lambda = \alpha + (\beta + \lambda)$, para todo $\alpha,\beta,\lambda \in \mathbb{C}$.}\\\\
	\textbf{Demostración.-}\; Sea $\alpha = a+bi$,  $\beta = c+di$ y $\lambda = e + fi$ entonces,
	$$\begin{array}{rcl}
	    (\alpha + \beta) + \lambda &=&\left[(a+bi)+(c+di)\right]+(e+fi)\\
				       &=&(a+bi)+\left[(c+di)+(e+fi)\right]\\
				       &=&\beta + (\alpha + \lambda)\\
	\end{array}$$
	Así, $(\alpha + \beta) + \lambda = \beta + (\alpha + \lambda).$\\\\


    %----------5.
    \item \textbf{\boldmath Probar que para todo $\alpha \in \mathbb{C}$, existe un único $b\in \mathbb{C}$ tal que $\alpha + \beta = 0$.}\\\\
	\textbf{Demostración.-}\;


\end{enumerate}

\end{multicols}
