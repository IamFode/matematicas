
\section*{\center Guía 1}
\vspace{1.2cm}
\setlength{\columnsep}{.7cm}
\setlength{\columnseprule}{0.1pt}
\begin{multicols}{2}

\begin{enumerate}[\large\bfseries 1.]

    %----------1.
    \item  \textbf{\boldmath Suponga que $a,b$ son números reales no nulos simultáneamente. Hallar números reales $c$ y $d$ tales que,
	$$\dfrac{1}{a+bi}=c+di$$\\}
	\textbf{Respuesta.-}\; Hallemos $c\in \mathbb{R}$ como sigue,
	$$\begin{array}{rcl}
	    \dfrac{1}{a+bi}&=&c+di\\
	    \dfrac{1}{a+bi} - di&=&c+di - di\\
	    c&=&\dfrac{1}{a+bi}-di\\
	\end{array}$$

	Ya que, $i^4 = i^3i=(-i)i=-(i^2) = -(-1) = 1\; $
	y $$i^3 = i^2i=(-1)i = -i,$$
	Luego hallamos $d\in \mathbb{R}$,
	$$\begin{array}{rcl}
	    \dfrac{1}{a+bi}&=&c+di\\
			   di&=&\dfrac{1}{a+bi}-c\\
			   di\cdot i^3&=&i^3\left(\dfrac{1}{a+bi}-c\right)\\
			   d&=&\dfrac{-i}{a+bi}+ci\\
	\end{array}$$

	Así, $c=\dfrac{1}{a+bi}-di\;$ ; $\; d=\dfrac{-i}{a+bi}+ci$.\\\\

    %----------2.
    \item \textbf{\boldmath Hallar dos raíces cuadradas distintas de $i$.}\\\\
	\textbf{Respuesta.-}\; $x^2-1 = 0\quad $ y $\quad x^2-4=0$.\\\\  

    %----------3.
    \item \textbf{\boldmath Probar que $\alpha+\beta = \beta +\alpha$, para todo $\alpha,\beta \in \mathbb{C}$.}\\\\
	\textbf{Demostración.-}\; Sea $\alpha = a+bi$ y $\beta = c+di$, entonces por definición de números complejos para la adición, tenemos que,
	$$\begin{array}{rcl}
	    \alpha + \beta &=&(a+bi)+(c+di)\\
			   &=&(c+di)+(a+bi)\\
			   &=&\beta + \alpha\in \mathbb{C}\\
	\end{array}$$
	De donde se demuestra la proposición dada.\\\\

    %----------4.
    \item \textbf{\boldmath Probar que $(\alpha+\beta)+\lambda = \alpha + (\beta + \lambda)$, para todo $\alpha,\beta,\lambda \in \mathbb{C}$.}\\\\
	\textbf{Demostración.-}\; Sea $\alpha = a+bi$,  $\beta = c+di$ y $\lambda = e + fi$ entonces,
	$$\begin{array}{rcl}
	    (\alpha + \beta) + \lambda &=&\left[(a+bi)+(c+di)\right]+(e+fi)\\
				       &=&(a+bi)+\left[(c+di)+(e+fi)\right]\\
				       &=&\beta + (\alpha + \lambda)\\
	\end{array}$$
	Así, $(\alpha + \beta) + \lambda = \beta + (\alpha + \lambda).$\\\\


    %----------5.
    \item \textbf{\boldmath Probar que para todo $\alpha \in \mathbb{C}$, existe un único $\beta \in \mathbb{C}$ tal que $\alpha + \beta = 0$.}\\\\
	\textbf{Demostración.-}\; La existencia queda demostrada por la propiedad identidad para la adición.\\
	Ahora demostremos su unicidad de la siguiente manera:\\
	Supongamos que existen  $\beta^{'},\; \beta \in \mathbb{C}$ tales que $\alpha + \beta = 0\;$ y $\;\alpha + \beta^{'} = 0$ que implica,
	$$\alpha + \beta = \alpha + \beta^{'} \; \Longrightarrow \;\beta=\beta^{'}.$$
	Y por lo tanto, queda demostrada la unicidad.\\\\
	Demostrada la existencia y unicidad concluimos que se cumple la propiedad del inverso aditivo para $\mathbb{C}$.\\\\
	

    %----------6.
    \item \textbf{ \boldmath Probar que para todo $\alpha \in \mathbb{C}-\lbrace0 \rbrace,$ existe un único $\beta \in \mathbb{C}$ tal que $\alpha \beta = 1$.}\\\\
	\textbf{Demostración.-}\; Similar al anterior ejercicio podemos demostrar la existencia de $\beta$ por la propiedad de identidad para la multiplicación. Luego demostremos la unicidad de la siguiente manera:\\\\
	Sean $\beta, \beta^{'}\in \mathbb{C}$ tales que $\alpha \beta = 1\; $ y $\; \alpha \beta = 1$  entonces $$\alpha\beta = \alpha \beta^{'}$$
	 como $\alpha\neq 0$, nos queda que, $\beta=\beta{'}$.\\\\ 
	 Así, queda demostrada la propiedad del inverso multiplicativo para $\mathbb{C}$.\\\\

    %----------7.
     \item \textbf{ \boldmath Hallar $x\in \mathbb{R}^4$ tal que $(4,-3,1,7)+2x=(5,9,-6,8)$.}\\\\
	 \textbf{Respuesta.-}\; se tiene que,
	 $$\begin{array}{rcl}
	     (4,-3,1,7)+2x&=&(5,9,-6,8)\\
			  2x&=&(5,9,-6,8)-(4,-3,1,7)\\
			    2x&=&(5-4,9+3,-6-1,8-7)\\
			    2x&=&(1,12,-7,1)\\
	 \end{array}$$
	 De donde $x=\left(\dfrac{1}{2},6,-\dfrac{7}{2},\dfrac{1}{2}\right).$\\\\

    %----------8.
     \item \textbf{\boldmath Probar que $(x+y)+z=x+(y+z)$ para todo $x,y,z\in \mathbf{F}^n$.}\\\\
	 \textbf{Demostración.-}\; Sea 
	 $$\begin{array}{rcl}
	     x&=&(x_1,\ldots,x_n)\\ 
	     y&=&(y_1,\ldots,y_n)\\
	     z&=&(z_1,\ldots,z_n)
	 \end{array}$$ 
	     entonces,

	 $$\begin{array}{rcl}
	     (x+y)+z&=&\left[(x_1,\ldots,x_n)+(y_1,\ldots,y_n)\right]\\
		    &+&(z_1,\ldots,z_n)\\\\
		    &=&(x_1+y_1,\ldots,x_n+y_n)\\
		    &+&(z_1,\ldots ,z_n)\\\\
		    &=&(x_1+y_1+z_1,\ldots,x_n+y_n+z_n)\\
	 \end{array}$$

	 $$\begin{array}{rcl}
	     x+(y+z)&=&(x_1,\ldots,x_n)\\
		    &+&\left[(y_1,\ldots,y_n)+(z_1,\ldots,z_n)\right]\\\\
		    &=&(x_1,\ldots,x_n)\\
		    &+&(y_1 + z_1,\ldots ,y_n + z_n)\\\\
		    &=&(x_1+y_1+z_1,\ldots,x_n+y_n+z_n)\\
	 \end{array}$$

	 De donde se tiene que,
	 $$(x+y)+z=x+(y+z).$$\\

    %----------9.
     \item \textbf{\boldmath Probar que $(ab)x=a(bx)$ para todo $x\in F^n$ y para todo $a,b \in F$.}\\\\
	 \textbf{Demostración.-}\; Sea $x=(x_1,\ldots,x_n)$ entonces,
	 $$\begin{array}{rcl}
	     (ab)x&=&(ab)(x_1,\ldots,x_n)\\
		  &=&((ab)x_1,\ldots,(ab)x_n)\\
		  &=&(a(bx_1),\ldots,a(bx_n))\\
		  &=&a(bx_1,\ldots,bx_n)\\
		  &=&a(bx).\\
	 \end{array}$$
	 De ésta manera se demuestra que la propiedad dada.\\\\

    %----------10.
    \item 


\end{enumerate}


\end{multicols}
