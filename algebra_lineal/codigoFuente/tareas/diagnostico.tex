
\begin{center}
    \Large\textbf{Prueba Diagnostico}
\end{center}

\begin{enumerate}[\bfseries 1.]

    %---------- 1.
    \item Dado el conjunto $U=\mathbb{R},\; B=[0,\infty),\; C=[0,1]$ y $D=(0,1)$ determinar los siguientes (si es posible esbozar una idea geométrica):\\

	\begin{enumerate}[\bfseries a)]

	    %----- a)
	    \item $D^c \cap C$.\\\\
		Respuesta.-\; Sea $D^c = (-\infty, 0]\cap [1,\infty)$ entonces 
		$D^c \cap C = 0,\, 1.$\\\\

	    %----- b)
	    \item $B\times U.$\\\\
		Respuesta.-\; $B\times U = \lbrace (a,b): a\in B \land b\in U \rbrace.$\\\\

	    %----- c)
	    \item $B\cap D \cup \lbrace x\in U\; :\; x\leq 1/2\rbrace .$\\\\
		Respuesta.-\; Sea $B\cap D = [0,1]$, entonces,
		$B\cap D \cup \lbrace x\in U\; :\; x\leq 1/2\rbrace = (-\infty,1].$\\\\

	\end{enumerate}

    %---------- 2.
    \item Dada la función $f:[0,5]\rightarrow \mathbb{R}$, con regla de correspondencia $f(t)=\int_0^t (x^2+1)\; dx.$ Determinar el recorrido de la función. ¿Es la función inyectiva?.\\\\
	Respuesta.-\; Sea $f(t) = \displaystyle\int_0^{t} (x^2+1)\; dx = \left(\dfrac{x^3}{3} + x\right)\bigg|_0^t = \dfrac{t^3}{3} + t$ entonces,
	\begin{itemize}
	    \item $f(0) = \dfrac{0^3}{3} + 0 = 0.$
	    \item $f(5) = \dfrac{5^3}{3} + 5 = \dfrac{125}{3} + 5 = \dfrac{28}{3}.$
	\end{itemize}
	El recorrido estará dado por $\left[0,\dfrac{28}{3}\right]$.\\\\
	La función es inyectiva en $[0,5]$ dado que si trazamos lineas horizontales sobre la gráfica de la función la función solo se intersecta en un sólo punto.\\\\

    %---------- 3.
    \item  Definir funciones biyectivas entre los siguientes conjuntos:\\

	\begin{enumerate}[\bfseries a)]

	    %----- a)
	    \item Entre $\mathbb{R}$ y $\mathbb{R}.$\\\\
		Respuesta.-\; 

	    %----- b)
	    \item Entre $\mathbb{R}$ y $(0,1).$\\\\
		Respuesta.-\;


    %---------- 4.
    \item calcular la ecuación de una linea recta que pasa por los puntos $P(3,1)$, $Q(-6,-2)$. Describir la función que caracteriza dicha recta. ¿$(0,0)$ está en la recta?.\\\\
	Respuesta.-\; Haciendo uso de la ecuación de la recta, tenemos por un lado que su pendiente viene dado por,
	$$m=\dfrac{1+2}{3+6} = \dfrac{1}{3}$$
	Luego reemplazamos uno de los puntos, 
	$$y-1 = \dfrac{1}{3}(x-3) \; \Longrightarrow \; y = \dfrac{1}{3}x$$
	Por último vemos que el punto $(0,0)$  está en la recta encontrada dado que $0=0$.\\\\




	\end{enumerate}






\end{enumerate}

