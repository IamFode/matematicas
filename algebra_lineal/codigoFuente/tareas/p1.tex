\begin{enumerate}[\bfseries \mbox{Ejercicio} 1.]

    %-------------------- section 1.2 6
    \item \textbf{Demostrar que si dos sistemas homogéneos de ecuaciones lineales con dos incógnitas tienen las mismas soluciones, son equivalentes.\\\\
	Demostración.-}\; Consideremos los dos sistemas homogéneos con dos incógnitas $(x_1,x_2)$.
	$$\left\{\begin{array}{ccc}
		a_{11} x_1 + a_{12} x_2 &=& 0\\
		a_{21} x_1 + a_{22} x_2 &=& 0\\
		\vdots\\
		a_{m1} x_1 + a_{m2} x_2 &=& 0\\
	\end{array}\right. \qquad \mbox{y} \qquad 
	\left\{\begin{array}{ccc}
		b_{11} x_1 + b_{12} x_2 &=& 0\\
		b_{21} x_1 + b_{22} x_2 &=& 0\\
		\vdots\\
		b_{m1} x_1 + b_{m2} x_2 &=& 0\\
	\end{array}\right.$$

	Sean los escalares $c_1,c_2,\ldots c_m$. De donde multiplicamos $k$ ecuaciones del primer sistemas por $c_k$ y sumamos por columnas,
	$$\left(c_1a_{11}+\ldots + c_m a_{m1}\right)x_1 + \left(c_1 a_{12}+\ldots + c_m a_{m2}\right)x_2=0$$

	Luego comparando esta ecuación con todas las ecuaciones del segundo sistema y utilizando también el hecho de que ambos sistemas tiene las mismas soluciones, obtenemos
	$$c_1a_{11}+c_2a_{21}+\ldots + c_m a_{m1}=b_{11}, b_{21}, \ldots, b_{m1}$$
	$$\mbox{y}$$
	$$c_1a_{12}+c_2a_{22}+\ldots + c_m a_{m2}=b_{12}, b_{22}, \ldots, b_{m2}.$$

	Lo que demuestra que el segundo sistema es una combinación lineal del primer sistema.\\

	De manera similar podemos demostrar que el primer sistema es una combinación lineal del segundo sistema. Sean los escalares $c_1,c_2,\ldots c_m$, entonces
	$$\left(c_1b_{11}+\ldots + c_m b_{m1}\right)x_1 + \left(c_1 b_{12}+\ldots + c_m b_{m2}\right)x_2=0$$
	Después, se tiene
	$$c_1b_{11}+c_2b_{21}+\ldots + c_m b_{m1}=a_{11}, a_{21}, \ldots, a_{m1}$$
	$$\mbox{y}$$
	$$c_1b_{12}+c_2b_{22}+\ldots + c_m b_{m2}=a_{12}, a_{22}, \ldots, a_{m2}.$$

	Así concluimos que ambos sistemas son equivalentes.\\\\


    %-------------------- section 1.2 7
    \item \textbf{Demostrar que todo subcuerpo del cuerpo de los números complejos contiene a todo número racional.\\\\
	Demostración.-}\; Sea $F$ un subcampo en $\mathbb{C}$. De donde tenemos $0\in F$ y $1\in F$. Luego ya que $F$ es un subcampo y cerrado bajo la suma, se tiene
	$$1+1+\ldots + 1 = n\in F.$$
	De este modo $\mathbb{Z}\subseteq F$. Ahora, sabiendo que $F$ es un subcampo, todo elemento tiene un inverso multiplicativo, por lo tanto $\dfrac{1}{n}\in F$. Por otro lado vemos también que $F$ es cerrado bajo la multiplicación. Es decir, para $m,n\in \mathbb{Z}$ y $n\neq 0$ tenemos
	$$m\cdot \dfrac{1}{n}\in F\quad \Rightarrow \quad \dfrac{m}{n}\in F.$$
	Así, concluimos que $\mathbb{Q}\subseteq F.$\\\\


    %-------------------- section 1.3 2
    \item \textbf{\boldmath Si
    $$A=\left[\begin{array}{rrr}
	    3 & -1 & 2 \\
	    2 & 1 & 1 \\
	    1 & -3 & 0
    \end{array}\right]$$
    Hallar todas las soluciones de $AX=0$ reduciendo $A$ por filas.\\\\
    	Respuesta.-}\; Se efectuará una sucesión finita de operaciones elementales de filas en $A$, indicando el tipo de operación realizada.

	$$\begin{array}{ccccc}

	    \begin{bmatrix*}[r]
		3 & -1 & 2 \\
		2 & 1 & 1 \\
		1 & -3 & 0
	    \end{bmatrix*} 
	    &R_1 \to R_3 &
	    \begin{bmatrix*}[r]
		1 & -3 & 0\\
		2 & 1 & 1 \\
		3 & -1 & 2 
	    \end{bmatrix*}
	    &R_2-2R_1 \to R_2&
	    \begin{bmatrix*}[r]
		1 & -3 & 0\\
		0 & 7 & 1 \\
		3 & -1 & 2 
	    \end{bmatrix*}\\\\
	    &R_3-3R_1\to R_3&
	    \begin{bmatrix*}[r]
		1 & -3 & 0\\
		0 & 7 & 1 \\
		0 & 8 & 2 
	    \end{bmatrix*}
	    &R_3-\dfrac{8}{7} R_2 \to R_3&
	    \begin{bmatrix*}[r]
		1 & -3 & 0\\
		0 & 7 & 1 \\
		0 & 0 & \frac{6}{7} 
	    \end{bmatrix*}\\\\

	\end{array}$$

	Por lo tanto tenemos: $\;\dfrac{6}{7}x_3=0,\;\; 7x^2+x^3=0 \; \; \mbox{y} \;\; x_1-3x_2=0.$ De donde 
	$$x_1=x_2=x_3=0.$$\\


    %-------------------- section 1.3 3
    \item \textbf{\boldmath Si
    $$A=\left[\begin{array}{rrr}
	    6 & -4 & 0 \\
	    4 & -2 & 0 \\
	    -1 & 0 & 3 
    \end{array}\right]$$
    Hallar todas las soluciones de $AX=2X$ y todas las soluciones de $AX=3X$ (el símbolo $cX$ representa la matriz, cada elemento de la cual es $c$ veces el correspondiente elemento de $X$).\\\\
	Demostración.-}\; Sea $AX=2X$, entonces
	$$AX=2X\quad \Rightarrow \quad AX-2X=0\quad \Rightarrow \quad (A-2I)X=0.$$
	De donde,

	$$A-2I=\begin{bmatrix*}[r]
	    6 & -4 & 0 \\
	    4 & -2 & 0 \\
	    -1 & 0 & 3
	\end{bmatrix*}-2
	\begin{bmatrix*}[r]
	    1 & 0 & 0 \\
	    0 & 1 & 0 \\
	    0 & 0 & 1 
	\end{bmatrix*} = 
	\begin{bmatrix*}[r]
	    4 & -4 & 0 \\
	    4 & -4 & 0 \\
	    -1 & 0 & 1 
	\end{bmatrix*}$$

	Resolviendo tenemos,

	$$\begin{array}{ccccc}

	    \begin{bmatrix*}[r]
		4 & -4 & 0 \\
		4 & -4 & 0 \\
		-1 & 0 & 1 
	    \end{bmatrix*} 
	    &R_2 - R_1 \to R_2&
	    \begin{bmatrix*}[r]
		4 & -4 & 0 \\
		0 & 0 & 0 \\
		-1 & 0 & 1 
	    \end{bmatrix*} 
	    &R_3 + \dfrac{1}{4} R_1 \to R_3&
	    \begin{bmatrix*}[r]
		4 & -4 & 0 \\
		0 & 0 & 0 \\
		0 & -1 & 1 
	    \end{bmatrix*}\\\\
	    &R_2 \leftrightarrow R_3&
	    \begin{bmatrix*}[r]
		4 & -4 & 0 \\
		0 & -1 & 1 \\
		0 & 0 & 0
	    \end{bmatrix*} & & \\

	\end{array}$$

	Luego, las soluciones estarán dadas por, 
	$$-x_2+x_3=0\quad \Rightarrow \quad x_2=x_3 \qquad \mbox{y}\qquad 4x_1-4x_2=0 \quad \Rightarrow \quad x_1=x_2.$$
	Por lo tanto,
	$$x_1=x_2=x_3.$$\\
	Por otro lado si $AX=3X$, entonces
	$$AX=3X\quad \Rightarrow \quad AX-3X=0\quad \Rightarrow \quad (A-3I)X=0.$$
	De donde,

	$$A-2I=\begin{bmatrix*}[r]
	    6 & -4 & 0 \\
	    4 & -2 & 0 \\
	    -1 & 0 & 3
	\end{bmatrix*}-2
	\begin{bmatrix*}[r]
	    1 & 0 & 0 \\
	    0 & 1 & 0 \\
	    0 & 0 & 1 
	\end{bmatrix*} = 
	\begin{bmatrix*}[r]
	    3 & -4 & 0 \\
	    4 & -5 & 0 \\
	    -1 & 0 & 0 
	\end{bmatrix*}$$

	Resolviendo tenemos,

	$$\begin{array}{ccccc}

	    \begin{bmatrix*}[r]
		3 & -4 & 0 \\
		4 & -5 & 0 \\
		-1 & 0 & 0 
	    \end{bmatrix*} 
	    &R_1 \leftrightarrow R_3&
	    \begin{bmatrix*}[r]
		-1 & 0 & 0 \\
		4 & -5 & 0 \\
		3 & -4 & 0 
	    \end{bmatrix*} 
	    &R_2 + 4 R_1 \to R_2&
	    \begin{bmatrix*}[r]
		-1 & 0 & 0 \\
		0 & -5 & 0 \\
		3 & -4 & 0 
	    \end{bmatrix*}\\\\
	    &R_3 +3R_1 \to R_3&
	    \begin{bmatrix*}[r]
		-1 & 0 & 0 \\
		0 & -5 & 0 \\
		0 & -4 & 0 
	    \end{bmatrix*} 
	    &R_3-\dfrac{4}{5}R_2 \to R_3& 
	    \begin{bmatrix*}[r]
		-1 & 0 & 0 \\
		0 & -5 & 0 \\
		0 & 0 & 0 
	    \end{bmatrix*} 

	\end{array}$$

	Luego,  $0x_3=0\; \Rightarrow \; x_3\in \mathbb{R}$, $x_1=0$ y $x_2=0$ por lo tanto,
	$$x_3\in \mathbb{R}, x_1=x_2=0.$$\\


    %-------------------- section 1.3 6 
    \item \textbf{\boldmath Sea
    $$\left[\begin{array}{rr}
	    a & b  \\
	    c & d  
    \end{array}\right]$$
    es una matriz $2\times 2$ con elementos complejos. Supóngase que $A$ es reducida por filas y también que $a+b+c+d=0$. Demostrar que existen exactamente tres de estas matrices.\\\\
    	Demostración.-}\;

    %-------------------- section 1.3 7
    \item \textbf{\boldmath Demostrar que el intercambio de dos filas en una matriz puede hacerse por medio de un número finito de operaciones elementales con filas de los otros tipos.\\\\
	Demostración.-}\;

    %-------------------- section 1.4 1
    \item \textbf{\boldmath Hallar, mediante reducción por filas de la matriz de coeficientes todas las soluciones del siguiente sistema de ecuaciones:
    $$\begin{array}{rrrrrcc}
	\frac{1}{3}x_1&+&2x_2&-&6x_3&=&0\\
		      -4x_1&&&+&5x_3&=&0\\
			   -3x_1&+&6x_2&-&13x_3&=&0\\
			   -\frac{7}{3}x_1&+&2x_2&-&\frac{8}{3}x_3&=&0\\
   \end{array}.$$\\\\
	Respuesta.-}\; 

    %-------------------- section 1.4 2
    \item \textbf{\boldmath Hallar una matriz escalón reducida por filas que sea equivalente a
    $$\left[\begin{array}{rr}
	1 & -i \\
	2 & 2 \\
	i & 1+i\\
    \end{array}\right]$$\\
    ¿Cuales son las soluciones de $AX=0$?.\\\\
    	Respuesta.-}\;

    %-------------------- section 1.4 5
    \item \textbf{Dar un ejemplo de un sistema de dos ecuaciones lineales con dos incógnitas que no tengan solución.\\\\
	Respuesta.-}\;

    %-------------------- section 1.4 6
    \item \textbf{\boldmath Mostrar que el sistema
    $$\begin{array}{ccccccccc}
	x_1&-&2x_2&+&x_3&+&2x_4&=&1\\
	   x_1&+&x_2&-&x_3&+&x_4&=&2\\
	   x_1&+&7x_2&-&5x_3&-&x_4&=&3\\
    \end{array}$$
    No tiene solución.\\\\
	Demostración.-}\; 

    %-------------------- section 1.4 7
    \item \textbf{\boldmath Hallar todas las soluciones de
    $$\begin{array}{ccccccccccr}
	2x_1&-&3x_2&-&7x_3&+&5x_4&+&2x_5&=&-2\\
	x_1&-&2x_2&-&4x_3&+&3x_4&+&x_5&=&-2\\
	2x_1&&&-&4x_3&+&2x_4&+&x_5&=&3\\
	x_1&-&5x_2&-&7x_3&+&6x_4&+&2x_5&=&-7\\
    \end{array}$$
	Respuesta.-}\;

    %-------------------- section 1.4 8
    \item \textbf{\boldmath Sea 
    $$\left[\begin{array}{rrr}
	3 & -1 & 2 \\
	2 & 1 & 1 \\
	1 & -3 & 0
    \end{array}\right]$$
    ¿Para cuáles ternas $(y_1,y_2,y_3)$ tiene una solución el sistema $AX=Y$?.\\\\
    	Respuesta.-}\;


\end{enumerate}
