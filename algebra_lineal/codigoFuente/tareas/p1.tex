\begin{enumerate}[\bfseries \mbox{Ejercicio} 1.]

    %-------------------- section 1.2 6
    \item \textbf{Demostrar que si dos sistemas homogéneos de ecuaciones lineales con dos incógnitas tienen las mismas soluciones, son equivalentes.\\\\
	Demostración.-}\; Consideremos los dos sistemas homogéneos con dos incógnitas $(x_1,x_2)$.
	$$\left\{\begin{array}{ccc}
		a_{11} x_1 + a_{12} x_2 &=& 0\\
		a_{21} x_1 + a_{22} x_2 &=& 0\\
		\vdots\\
		a_{m1} x_1 + a_{m2} x_2 &=& 0\\
	\end{array}\right. \qquad \mbox{y} \qquad 
	\left\{\begin{array}{ccc}
		b_{11} x_1 + b_{12} x_2 &=& 0\\
		b_{21} x_1 + b_{22} x_2 &=& 0\\
		\vdots\\
		b_{m1} x_1 + b_{m2} x_2 &=& 0\\
	\end{array}\right.$$

	Sean los escalares $c_1,c_2,\ldots c_m$. De donde multiplicamos $k$ ecuaciones del primer sistemas por $c_k$ y sumamos por columnas,
	$$\left(c_1a_{11}+\ldots + c_m a_{m1}\right)x_1 + \left(c_1 a_{12}+\ldots + c_m a_{m2}\right)x_2=0$$

	Luego comparando esta ecuación con todas las ecuaciones del segundo sistema y utilizando también el hecho de que ambos sistemas tiene las mismas soluciones, obtenemos
	$$c_1a_{11}+c_2a_{21}+\ldots + c_m a_{m1}=b_{11}, b_{21}, \ldots, b_{m1}$$
	$$\mbox{y}$$
	$$c_1a_{12}+c_2a_{22}+\ldots + c_m a_{m2}=b_{12}, b_{22}, \ldots, b_{m2}.$$

	Lo que demuestra que el segundo sistema es una combinación lineal del primer sistema.\\

	De manera similar podemos demostrar que el primer sistema es una combinación lineal del segundo sistema. Sean los escalares $c_1,c_2,\ldots c_m$, entonces
	$$\left(c_1b_{11}+\ldots + c_m b_{m1}\right)x_1 + \left(c_1 b_{12}+\ldots + c_m b_{m2}\right)x_2=0$$
	Después, se tiene
	$$c_1b_{11}+c_2b_{21}+\ldots + c_m b_{m1}=a_{11}, a_{21}, \ldots, a_{m1}$$
	$$\mbox{y}$$
	$$c_1b_{12}+c_2b_{22}+\ldots + c_m b_{m2}=a_{12}, a_{22}, \ldots, a_{m2}.$$

	Así concluimos que ambos sistemas son equivalentes.\\\\


    %-------------------- section 1.2 7
    \item \textbf{Demostrar que todo subcuerpo del cuerpo de los números complejos contiene a todo número racional.\\\\
	Demostración.-}\; Sea $F$ un subcampo en $\mathbb{C}$. De donde tenemos $0\in F$ y $1\in F$. Luego ya que $F$ es un subcampo y cerrado bajo la suma, se tiene
	$$1+1+\ldots + 1 = n\in F.$$
	De este modo $\mathbb{Z}\subseteq F$. Ahora, sabiendo que $F$ es un subcampo, todo elemento tiene un inverso multiplicativo, por lo tanto $\dfrac{1}{n}\in F$. Por otro lado vemos también que $F$ es cerrado bajo la multiplicación. Es decir, para $m,n\in \mathbb{Z}$ y $n\neq 0$ tenemos
	$$m\cdot \dfrac{1}{n}\in F\quad \Rightarrow \quad \dfrac{m}{n}\in F.$$
	Así, concluimos que $\mathbb{Q}\subseteq F.$\\\\


    %-------------------- section 1.3 2
    \item \textbf{\boldmath Si
    $$A=\left[\begin{array}{rrr}
	    3 & -1 & 2 \\
	    2 & 1 & 1 \\
	    1 & -3 & 0
    \end{array}\right]$$
    Hallar todas las soluciones de $AX=0$ reduciendo $A$ por filas.\\\\
    	Respuesta.-}\; Se efectuará una sucesión finita de operaciones elementales de filas en $A$, indicando el tipo de operación realizada.

	$$\begin{array}{ccccc}

	    \begin{bmatrix*}[r]
		3 & -1 & 2 \\
		2 & 1 & 1 \\
		1 & -3 & 0
	    \end{bmatrix*} 
	    &R_1 \to R_3 &
	    \begin{bmatrix*}[r]
		1 & -3 & 0\\
		2 & 1 & 1 \\
		3 & -1 & 2 
	    \end{bmatrix*}
	    &R_2-2R_1 \to R_2&
	    \begin{bmatrix*}[r]
		1 & -3 & 0\\
		0 & 7 & 1 \\
		3 & -1 & 2 
	    \end{bmatrix*}\\\\
	    &R_3-3R_1\to R_3&
	    \begin{bmatrix*}[r]
		1 & -3 & 0\\
		0 & 7 & 1 \\
		0 & 8 & 2 
	    \end{bmatrix*}
	    &R_3-\dfrac{8}{7} R_2 \to R_3&
	    \begin{bmatrix*}[r]
		1 & -3 & 0\\
		0 & 7 & 1 \\
		0 & 0 & \frac{6}{7} 
	    \end{bmatrix*}\\\\

	\end{array}$$

	Por lo tanto tenemos: $\;\dfrac{6}{7}x_3=0,\;\; 7x^2+x^3=0 \; \; \mbox{y} \;\; x_1-3x_2=0.$ De donde 
	$$x_1=x_2=x_3=0.$$\\


    %-------------------- section 1.3 3
    \item \textbf{\boldmath Si
    $$A=\left[\begin{array}{rrr}
	    6 & -4 & 0 \\
	    4 & -2 & 0 \\
	    -1 & 0 & 3 
    \end{array}\right]$$
    Hallar todas las soluciones de $AX=2X$ y todas las soluciones de $AX=3X$ (el símbolo $cX$ representa la matriz, cada elemento de la cual es $c$ veces el correspondiente elemento de $X$).\\\\
	Demostración.-}\; Sea $AX=2X$, entonces
	$$AX=2X\quad \Rightarrow \quad AX-2X=0\quad \Rightarrow \quad (A-2I)X=0.$$
	De donde,

	$$A-2I=\begin{bmatrix*}[r]
	    6 & -4 & 0 \\
	    4 & -2 & 0 \\
	    -1 & 0 & 3
	\end{bmatrix*}-2
	\begin{bmatrix*}[r]
	    1 & 0 & 0 \\
	    0 & 1 & 0 \\
	    0 & 0 & 1 
	\end{bmatrix*} = 
	\begin{bmatrix*}[r]
	    4 & -4 & 0 \\
	    4 & -4 & 0 \\
	    -1 & 0 & 1 
	\end{bmatrix*}$$

	Resolviendo tenemos,

	$$\begin{array}{ccccc}

	    \begin{bmatrix*}[r]
		4 & -4 & 0 \\
		4 & -4 & 0 \\
		-1 & 0 & 1 
	    \end{bmatrix*} 
	    &R_2 - R_1 \to R_2&
	    \begin{bmatrix*}[r]
		4 & -4 & 0 \\
		0 & 0 & 0 \\
		-1 & 0 & 1 
	    \end{bmatrix*} 
	    &R_3 + \dfrac{1}{4} R_1 \to R_3&
	    \begin{bmatrix*}[r]
		4 & -4 & 0 \\
		0 & 0 & 0 \\
		0 & -1 & 1 
	    \end{bmatrix*}\\\\
	    &R_2 \leftrightarrow R_3&
	    \begin{bmatrix*}[r]
		4 & -4 & 0 \\
		0 & -1 & 1 \\
		0 & 0 & 0
	    \end{bmatrix*} & & \\

	\end{array}$$

	Luego, las soluciones estarán dadas por, 
	$$-x_2+x_3=0\quad \Rightarrow \quad x_2=x_3 \qquad \mbox{y}\qquad 4x_1-4x_2=0 \quad \Rightarrow \quad x_1=x_2.$$
	Por lo tanto,
	$$x_1=x_2=x_3.$$\\
	Por otro lado si $AX=3X$, entonces
	$$AX=3X\quad \Rightarrow \quad AX-3X=0\quad \Rightarrow \quad (A-3I)X=0.$$
	De donde,

	$$A-2I=\begin{bmatrix*}[r]
	    6 & -4 & 0 \\
	    4 & -2 & 0 \\
	    -1 & 0 & 3
	\end{bmatrix*}-2
	\begin{bmatrix*}[r]
	    1 & 0 & 0 \\
	    0 & 1 & 0 \\
	    0 & 0 & 1 
	\end{bmatrix*} = 
	\begin{bmatrix*}[r]
	    3 & -4 & 0 \\
	    4 & -5 & 0 \\
	    -1 & 0 & 0 
	\end{bmatrix*}$$

	Resolviendo tenemos,

	$$\begin{array}{ccccc}

	    \begin{bmatrix*}[r]
		3 & -4 & 0 \\
		4 & -5 & 0 \\
		-1 & 0 & 0 
	    \end{bmatrix*} 
	    &R_1 \leftrightarrow R_3&
	    \begin{bmatrix*}[r]
		-1 & 0 & 0 \\
		4 & -5 & 0 \\
		3 & -4 & 0 
	    \end{bmatrix*} 
	    &R_2 + 4 R_1 \to R_2&
	    \begin{bmatrix*}[r]
		-1 & 0 & 0 \\
		0 & -5 & 0 \\
		3 & -4 & 0 
	    \end{bmatrix*}\\\\
	    &R_3 +3R_1 \to R_3&
	    \begin{bmatrix*}[r]
		-1 & 0 & 0 \\
		0 & -5 & 0 \\
		0 & -4 & 0 
	    \end{bmatrix*} 
	    &R_3-\dfrac{4}{5}R_2 \to R_3& 
	    \begin{bmatrix*}[r]
		-1 & 0 & 0 \\
		0 & -5 & 0 \\
		0 & 0 & 0 
	    \end{bmatrix*} 

	\end{array}$$

	Luego,  $0x_3=0\; \Rightarrow \; x_3\in \mathbb{R}$, $x_1=0$ y $x_2=0$ por lo tanto,
	$$x_3\in \mathbb{R}, x_1=x_2=0.$$\\


    %-------------------- section 1.3 6 
    \item \textbf{\boldmath Sea
    $$\left[\begin{array}{rr}
	    a & b  \\
	    c & d  
    \end{array}\right]$$
    es una matriz $2\times 2$ con elementos complejos. Supóngase que $A$ es reducida por filas y también que $a+b+c+d=0$. Demostrar que existen exactamente tres de estas matrices.\\\\
    	Demostración.-}\; Ya que $A$ es dado como una matríz reducida por filas, entonces se presentas los siguientes casos.
	\begin{enumerate}[\bfseries \textbf{Caso} i.]
	    \item Sea $a=b=c=d=0$ por lo tanto
		$$\begin{bmatrix*}[r]
		    0 & 0 \\
		    0 & 0
		\end{bmatrix*}$$

	    \item Sea $c=d=0\; \Rightarrow \; a+b=0 \; \Rightarrow \; b=-a$ con $a=1$, entonces
		$$\begin{bmatrix*}[r]
		    1 & -1 \\
		    0 & 0
		\end{bmatrix*}$$

	    Otra posibilidad podría ser que la fila $1$ sea cero. Es decir, $a=b=0\; \Rightarrow \; c+d=0\; \Rightarrow d=-c$ de donde tenemos

		$$\begin{bmatrix*}[r]
		    0 & 0 \\
		    1 & -1 
		\end{bmatrix*}$$

	    \item Dos filas distintas de cero. Es decir, $a\neq 0, d\neq 0$, $c=b=0\; \Rightarrow \; a+d=0 \; \Rightarrow \; d=-a$, de donde $A$ al ser reducido por fila implicará $a=1\; \rightarrow \; d=-1$. Pero $d$ es la entrada principal distinto de cero de la fila $2$ por lo tanto debe ser igual a uno. Así este caso no existe.
	    \end{enumerate}

	    De donde concluimos que existe dos matrices que satisfacen la condición dada.\\\\

    %-------------------- section 1.3 7
    \item \textbf{\boldmath Demostrar que el intercambio de dos filas en una matriz puede hacerse por medio de un número finito de operaciones elementales con filas de los otros tipos.\\\\
	Demostración.-}\; Consideremos la matriz identidad:
	$$\begin{bmatrix*}[r]
	    1 & 0 & 0 \\
	    0 & 1 & 0 \\
	    0 & 0 & 1
	\end{bmatrix*}$$
	Supongamos que se necesita el intercambio de las filas $1$ y $3$, entonces se puede realizar la siguiente secuencia para hacerlo:

	$$\begin{array}{ccccc}

	    \begin{bmatrix*}[r]
		1 & 0 & 0 \\
		0 & 1 & 0 \\
		0 & 0 & 1
	    \end{bmatrix*} 
	    &R_1 \to R_1+R_3&
	    \begin{bmatrix*}[r]
		1 & 0 & 1 \\
		0 & 1 & 0 \\
		0 & 0 & 1
	    \end{bmatrix*} 
	    &R_3 \to R_3-R_1&
	    \begin{bmatrix*}[r]
		1 & 0 & 1 \\
		0 & 1 & 0 \\
		-1 & 0 & 0
	    \end{bmatrix*}\\\\
	    &R_3  \to -R_3&
	    \begin{bmatrix*}[r]
		1 & 0 & 1 \\
		0 & 1 & 0 \\
		1 & 0 & 0
	    \end{bmatrix*} 
	    &R_1 \to R_1-R_3& 
	    \begin{bmatrix*}[r]
		0 & 0 & 1 \\
		0 & 1 & 0 \\
		1 & 0 & 0
	    \end{bmatrix*} 

	\end{array}$$

	Generalizando obtenemos que para realizar el intercambio de $R_i$ y $R_j$, la secuencia de operaciones elementales de los otros tipos es: 

	\begin{enumerate}[a)]
	    \item $R_j\to R_j-R_i$. Restar fila $i$ de la fila $j$, que da como resultado una fila $j$ como el negativo de la fila original $i$.
	    \item $R_j\to -R_j$. Fila negativa $j$ lo que resultará en la fila $j$ cambiando a fila $i$.
	    \item $R_j\to R_i-R_j$. Restar fila $j$ de la fila $i$ que resultará en la fila $i$ cambiando a la fila original $j$.\\\\
	\end{enumerate}



    %-------------------- section 1.4 1
    \item \textbf{\boldmath Hallar, mediante reducción por filas de la matriz de coeficientes todas las soluciones del siguiente sistema de ecuaciones:
    $$\begin{array}{rrrrrcc}
	\frac{1}{3}x_1&+&2x_2&-&6x_3&=&0\\
		      -4x_1&&&+&5x_3&=&0\\
			   -3x_1&+&6x_2&-&13x_3&=&0\\
			   -\frac{7}{3}x_1&+&2x_2&-&\frac{8}{3}x_3&=&0\\
   \end{array}.$$\\\\
	Respuesta.-}\; Consideremos la matriz de la forma $AX=0$ para la respectiva reducción por filas de la siguiente manera:

	$$\begin{array}{ccccc}

	    \begin{bmatrix*}[r]
		\frac{1}{3} & 2 & -6 \\
		-4 & 0 & 5 \\
		-3 & 6 & -13 \\
		-\frac{7}{3} & 2 & -\frac{8}{3}
	    \end{bmatrix*} 
	    &3R_1\to R_1&
	    \begin{bmatrix*}[r]
		1 & 6 & -18 \\
		-4 & 0 & 5 \\
		-3 & 6 & -13 \\
		-\frac{7}{3} & 2 & -\frac{8}{3}
	    \end{bmatrix*} 
	    &\begin{array}{rcl}
		R_2 + 4R_1 &\to& R_2 \\
		R_3 + 3R_1 &\to& R_3 \\
		R_4+\frac{7}{3}R_1&\to&R_4
		\end{array}&
	    \begin{bmatrix*}[r]
		1 & 6 & -18 \\
		0 & 24 & -67 \\
		0 & 24 & -67 \\
		0 & 16 & -\frac{134}{3}
	    \end{bmatrix*}\\\\
	    &\frac{1}{24}R_2  \to R_2&
	    \begin{bmatrix*}[r]
		1 & 6 & -18 \\
		0 & 1 & -\frac{67}{24} \\
		0 & 24 & -67 \\
		0 & 16 & -\frac{134}{3}
	    \end{bmatrix*} 
	    &\begin{array}{rcl}
		R_1-6r_2 &\to&R_1 \\
		R_3-24R_2&\to&R_3 \\
		R_4-16R_2&\to&R_4
	    \end{array}&
	    \begin{bmatrix*}[r]
		1 & 6 & -18 \\
		0 & 1 & -\frac{67}{24} \\
		0 & 0 & 0 \\
		0 & 0 & 0 \\
	    \end{bmatrix*} 

	\end{array}$$

	De donde 
	$$\begin{array}{rcl}
	    x_i-\dfrac{5}{4}x_3=0&\Rightarrow & x_1 = \dfrac{5}{4}x_3\\\\
	    x_2-\dfrac{67}{24}x_3=0&\Rightarrow & x_2 = \dfrac{67}{24}x_3\\\\
	\end{array}$$

	Por lo que la solución está dado por $\left\{\left(\dfrac{5}{4},\dfrac{67}{24},1\right)t\; / \; t\in \mathbb{R}\right\}$.\\\\


    %-------------------- section 1.4 2
    \item \textbf{\boldmath Hallar una matriz escalón reducida por filas que sea equivalente a
    $$\left[\begin{array}{rr}
	1 & -i \\
	2 & 2 \\
	i & 1+i\\
    \end{array}\right]$$\\
    ¿Cuales son las soluciones de $AX=0$?.\\\\
    	Respuesta.-}\; Por la eliminación Gaussiana se tiene,

	$$\begin{array}{ccccc}

	    \begin{bmatrix*}[r]
		1 & -i\\
		2 & 2 \\
		i & 1+i
	    \end{bmatrix*} 
	    &\begin{array}{rcl}
		R_2-2R_1 &\to& R_2 \\
		R_3-iR_1 &\to& R_3 \\
	    \end{array}&
	    \begin{bmatrix*}[r]
		1 & -i \\
		0 & 2+2i \\
		0 & i 
	    \end{bmatrix*} 
	    &\frac{1}{i}R_3 \to R_3&
	    \begin{bmatrix*}[r]
		1 & -i \\
		0 & 2+2i \\
		0 & 1
	    \end{bmatrix*}\\\\
	    &\begin{array}{rcl}
		R_1+iR_3 &\to& R_1 \\
		R_2-(2+2i)R_3&\to&R_2
	    \end{array}&
	    \begin{bmatrix*}[r]
		1 & 0 \\
		0 & 0 \\
		0 & 1 
	    \end{bmatrix*} 
	    &R_2\leftrightarrow R_3&
	    \begin{bmatrix*}[r]
		1 & 0 \\
		0 & 1 \\
		0 & 0
	    \end{bmatrix*} 

	\end{array}$$

	Que es la matriz escalonada reducida por filas equivalente a $A$. Por lo tanto la solución esta dada por,
	$$x_1=x_2=0.$$\\


    %-------------------- section 1.4 5
    \item \textbf{Dar un ejemplo de un sistema de dos ecuaciones lineales con dos incógnitas que no tengan solución.\\\\
	Respuesta.-}\; Consideremos el siguiente sistema de ecuaciones 
	$$\begin{array}{rcccl}
	    x_1&+&x_2&=&1\\
	    2x_1&+&2x_2&=&3\\\\
	\end{array}$$
	Luego el sistema no tiene solución ya que,
	$$\begin{array}{rcl}
	    2x_1+2x_3&=&2(x_1+x_2)\\
		     &=&2\cdot 1\\
		     &=&2\neq 3.
	\end{array}$$
	\vspace{.6cm}\\\\

    %-------------------- section 1.4 6
    \item \textbf{\boldmath Mostrar que el sistema
    $$\begin{array}{ccccccccc}
	x_1&-&2x_2&+&x_3&+&2x_4&=&1\\
	   x_1&+&x_2&-&x_3&+&x_4&=&2\\
	   x_1&+&7x_2&-&5x_3&-&x_4&=&3\\
    \end{array}$$
    No tiene solución.\\\\
	Demostración.-}\; Se tiene la matriz 
	$$AX=b \quad \Rightarrow \quad \begin{bmatrix*}[r]
	    1 & -2 & 1 & 2 \\
	    1 & 1 & -1 & 1 \\
	    1 & 7 & -5 & -1
	\end{bmatrix*}
	\begin{bmatrix*}[r]
	    x_1 \\
	    x_2 \\
	    x_3 \\
	    x_4
	\end{bmatrix*}
	= 
	\begin{bmatrix*}[r]
	    1 \\
	    2 \\
	    3
	\end{bmatrix*}$$
	Entonces,

	$$\begin{array}{ccc}

	    \left[\begin{array}{rrrr|r}
		1 & -2 & 1 & 2 & 1 \\
		1 & 1 & -1 & 1 & 2 \\
		1 & 7 & -5 & -1 & 3
	    \end{array}\right]
	    &\begin{array}{rcl}
		R_2-R_1 &\to& R_2 \\
		R_3-R_1 &\to& R_3 \\
	    \end{array}&
	    \left[\begin{array}{rrrr|r}
		1 & -2 & 1 & 2 & 1 \\
		0 & 3 & -2 & -1 & 1 \\
		0 & 9 & -6 & -3 & 2
	    \end{array}\right]\\\\
	    &\frac{1}{3}R_2\to R_2&
	    \left[\begin{array}{rrrr|r}
		1 & -2 & 1 & 2 & 1 \\
		0 & 1 & -\frac{2}{3} & -\frac{1}{3} & \frac{1}{3} \\
		0 & 9 & -6 & -3 & 2
	    \end{array}\right]\\\\
	    &\begin{array}{rcl}
		R_1+2R_2 &\to& R_1 \\
		R_3-9R_2&\to&R_3
	    \end{array}&
	    \left[\begin{array}{rrrr|r}
		1 & 0 & -\frac{1}{3} & \frac{4}{3} & \frac{5}{3} \\
		0 & 1 & -\frac{2}{3} & -\frac{1}{3} & \frac{1}{3} \\
		0 & 0 & 0 & 0 & -1 
	    \end{array}\right]

	\end{array}$$

	Que es la forma escalonada reducida por filas del sistema. De la última fila, obtenemos $0=-1$ lo cual es absurdo. Por lo tanto, el sistema dado no tiene solución.\\\\


    %-------------------- section 1.4 7
    \item \textbf{\boldmath Hallar todas las soluciones de
    $$\begin{array}{ccccccccccr}
	2x_1&-&3x_2&-&7x_3&+&5x_4&+&2x_5&=&-2\\
	x_1&-&2x_2&-&4x_3&+&3x_4&+&x_5&=&-2\\
	2x_1&&&-&4x_3&+&2x_4&+&x_5&=&3\\
	x_1&-&5x_2&-&7x_3&+&6x_4&+&2x_5&=&-7\\
    \end{array}$$
	Respuesta.-}\; Sea la matriz,
	$$AX=b \quad \Rightarrow \quad \begin{bmatrix*}[r]
	    2 & -3 & -7 & 5 & 2 \\
	    1 & -2 & -4 & 3 & 1 \\
	    2 & 0 & -4 & 2 & 1 \\
	    1 & -5 & -7 & 6 & 2
	\end{bmatrix*}
	\begin{bmatrix*}[r]
	    x_1 \\
	    x_2 \\
	    x_3 \\
	    x_4 \\
	    x_5
	\end{bmatrix*}
	= 
	\begin{bmatrix*}[r]
	    -2 \\
	    -2 \\
	    3 \\
	    -7
	\end{bmatrix*}$$
	Entonces,

	$$\begin{array}{ccl}

	    \left[\begin{array}{rrrrr|r}
		2 & -3 & -7 & 5 & 2 & -2 \\
		1 & -2 & -4 & 3 & 1 & -2 \\
		2 & 0 & -4 & 2 & 1 & 3 \\
		1 & -5 & -7 & 6 & 2 & -7
	    \end{array}\right]
	    &\frac{1}{2}R_1\to R_1&
	    \left[\begin{array}{rrrrr|r}
		1 & -\frac{3}{2} & -\frac{7}{2} & \frac{5}{2} & 1 & -1 \\
		1 & -2 & -4 & 3 & 1 & -2 \\
		2 & 0 & -4 & 2 & 1 & 3 \\
		1 & -5 & -7 & 6 & 2 & -7
	    \end{array}\right]\\\\
	    &\begin{array}{rcl}
		R_2-R_1 &\to& R_2 \\
		R_3-2R_1 &\to& R_3 \\
		R_4-R_1 &\to& R_4
	    \end{array}&
	    \left[\begin{array}{rrrrr|r}
		1 & -\frac{3}{2} & -\frac{7}{2} & \frac{5}{2} & 1 & -1 \\
		0 & -\frac{1}{2} & -\frac{1}{2} & \frac{1}{2} & 0 & -1 \\
		0 & 3 & 3 & -3 & -1 & 5 \\
		0 & -\frac{7}{2} & -\frac{7}{2} & \frac{7}{2} & 1 & -6
	    \end{array}\right]\\\\
	    &-\frac{1}{2}R_2\to R_2&
	    \left[\begin{array}{rrrrr|r}
		1 & -\frac{3}{2} & -\frac{7}{2} & \frac{5}{2} & 1 & -1 \\
		0 & 1 & 1 & -1 & 0 & 2 \\
		0 & 3 & 3 & -3 & -1 & 5 \\
		0 & -\frac{7}{2} & -\frac{7}{2} & \frac{7}{2} & 1 & -6
	    \end{array}\right]\\\\
	    &\begin{array}{rcl}
		R_1-\frac{3}{2}R_2 &\to& R_1 \\
		R_3-3R_2 &\to& R_3 \\
		R_4-\frac{7}{2}R_2 &\to& R_4
	    \end{array}&
	    \left[\begin{array}{rrrrr|r}
		1 & 0 & -2 & 1 & 1 & 2 \\
		0 & 1 & 1 & -1 & 0 & 2 \\
		0 & 0 & 0 & 0 & 1 & 1 \\
		0 & 0 & 0 & 0 & 1 & 1
	    \end{array}\right]\\\\
	    &\begin{array}{rcl}
		R_1-R_3 &\to& R_1 \\
		R_4-R_3 &\to& R_4
	    \end{array}&
	    \left[\begin{array}{rrrrr|r}
		1 & 0 & -2 & 1 & 0 & 1 \\
		0 & 1 & 1 & -1 & 0 & 2 \\
		0 & 0 & 0 & 0 & 1 & 1 \\
		0 & 0 & 0 & 0 & 0 & 0
	    \end{array}\right]

	\end{array}$$

	Por lo tanto las soluciones estarán dadas por:
	$$\begin{array}{rcl}
	    x_1-2x_3+x_4=1&\Rightarrow &x_1=1+2x_3-x_4\\
	    x_2+x_3-x_4=2&\Rightarrow &x_2=2-x_3+x_4\\
	    x_5&=&1.
	\end{array}$$
	Sea $x_3=s\in \mathbb{R}$ y $x_4=t\in \mathbb{R}$ donde el conjunto de soluciones estará dado por:
	$$\left\{(1,2,0,0,1)+(2,-1,1,0,0)s+(-1,1,0,1,0)t\; | \; s,t\in \mathbb{R}\right\}.$$\\


    %-------------------- section 1.4 8
    \item \textbf{\boldmath Sea 
    $$\left[\begin{array}{rrr}
	3 & -1 & 2 \\
	2 & 1 & 1 \\
	1 & -3 & 0
    \end{array}\right]$$
    ¿Para cuáles ternas $(y_1,y_2,y_3)$ tiene una solución el sistema $AX=Y$?.\\\\
    	Respuesta.-}\; Sea el sistema $AX=Y$ de donde,

	$$\begin{array}{ccl}

	    \left[\begin{array}{rrr|r}
		3 & -1 & 2 & y_1 \\
		2 & 1 & 1 & y_2 \\
		1 & -3 & 0 & y_3
	    \end{array}\right]
	    &R_1 \leftrightarrow R_3&
	    \left[\begin{array}{rrr|r}
		1 & -3 & 0 & y_3 \\
		2 & 1 & 1 & y_2 \\
		3 & -1 & 2 & y_1
	    \end{array}\right]\\\\
	    &\begin{array}{rcl}
		R_2-2R_1 &\to& R_2 \\
		R_3-3R_2 &\to& R_3
	    \end{array}&
	    \left[\begin{array}{rrr|r}
		1 & -3 & 0 & y_3 \\
		0 & 7 & 1 & y_2-2y_3 \\
		0 & 8 & 2 & y_1-3y_3
	    \end{array}\right]\\\\
	    &\frac{1}{7}R_2\to R_2&
	    \left[\begin{array}{rrr|r}
		1 & -3 & 0 & y_3 \\
		0 & 1 & \frac{1}{7} & \frac{y_2-2y_3}{7} \\
		0 & 8 & 2 & y_1-3y_3
	    \end{array}\right]\\\\
	    &\begin{array}{rcl}
		R_1+3R_2 &\to& R_1 \\
		R_3+8R_2 &\to& R_3
	    \end{array}&
	    \left[\begin{array}{rrr|r}
		1 & 0 & \frac{3}{7} & \frac{y_3+3y_2}{7} \\
		0 & 1 & \frac{1}{7} & \frac{y_2-2y_3}{7} \\
		0 & 0 & \frac{6}{7} & \frac{7y_1-8y_2-5y_3}{7}
	    \end{array}\right]\\\\
	    &\frac{7}{6}R_3 \to R_3&
	    \left[\begin{array}{rrr|r}
		1 & 0 & \frac{3}{7} & \frac{y_3+3y_2}{7} \\
		0 & 1 & \frac{1}{7} & \frac{y_2-2y_3}{7} \\
		0 & 0 & 1 & \frac{7y_1-8y_2-5y_3}{6}
	    \end{array}\right]\\\\

	\end{array}$$

	Por la forma reducida tenemos $Rango(A)=Rango(A/y)$, así el sistema tiene una única solución $y\in \mathbb{R}^3.$\\\\



\end{enumerate}
