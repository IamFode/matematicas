
\begin{enumerate}[\bfseries \mbox{Ejercicio} 1.]

    %-------------------- section 1.5 3
    \item \textbf{\boldmath Encontrar dos matrices $2\times 2,A$ diferentes tales que $A^2=0$ pero $A\neq 0.$\\\\
	Respuesta.-}\; Consideremos las siguientes matrices
	$$
	A=\begin{bmatrix*}[r]
		0 & 1\\
		0 & 0
	    \end{bmatrix*}\neq 0 \quad \mbox{y}\quad A^2=
	    \begin{bmatrix*}[r]
		0 & 1\\
		0 & 0
	    \end{bmatrix*}
	    \begin{bmatrix*}[r]
		0 & 1\\
		0 & 0
	    \end{bmatrix*} = 
	    \begin{bmatrix*}[r]
		0 & 0\\
		0 & 0
	    \end{bmatrix*} = 0
	    $$

	$$
	A=\begin{bmatrix*}[r]
		0 & 0\\
		1 & 0
	    \end{bmatrix*}\neq 0 \quad \mbox{y}\quad A^2=
	    \begin{bmatrix*}[r]
		0 & 0\\
		1 & 0
	    \end{bmatrix*}
	    \begin{bmatrix*}[r]
		0 & 0\\
		1 & 0
	    \end{bmatrix*} = 
	    \begin{bmatrix*}[r]
		0 & 0\\
		0 & 0
	    \end{bmatrix*} = 0
	    $$

	De este modo encontramos dos matrices tal que $A^2=0$.\\\\


    %-------------------- section 1.5 4
    \item \textbf{\boldmath Para cada $A$ del ejercicio 2, hallar matrices elementales $E_1,E_2,\ldots,E_k$ tal que 
	$$E_k\cdot E_2E_1A=I.$$\\
    Respuesta.-}\; Considere la matriz $A$ dada y reduzca por filas de la siguiente manera y mencione correspondientemente las matrices elementales.
    $$\begin{array}{rcccc}
	\begin{bmatrix*}[r]
	    1 & -1 & 1\\
	    2 & 0 & 1\\
	    3 & 0 & 1
	\end{bmatrix*} && R_2-2R_1 \to R_2 & &
	\begin{bmatrix*}[r]
	    1 & -1 & 1\\
	    0 & 2 & -1\\
	    3 & 0 & 1
	\end{bmatrix*} \\\\
	E_1A&=&\begin{bmatrix*}[r]
	    1 & 0 & 0\\
	    -2 & 1 & 0\\
	    0 & 0 & 1
	\end{bmatrix*} 
	\begin{bmatrix*}[r]
	    1 & -1 & 1\\
	    2 & 0 & 1\\
	    3 & 0 & 1
	\end{bmatrix*} &=& 
	\begin{bmatrix*}[r]
	    1 & -1 & 1\\
	    0 & 2 & -1\\
	    3 & 0 & 1
	\end{bmatrix*} \\\\
    \end{array}$$

    $$\begin{array}{rcccc}
	\begin{bmatrix*}[r]
	    1 & -1 & 1\\
	    0 & 2 & -1\\
	    3 & 0 & 1
	\end{bmatrix*} && R_3-3R_1\to R_3 & &
	\begin{bmatrix*}[r]
	    1 & -1 & 1\\
	    0 & 2 & -1\\
	    0 & 3 & -2
	\end{bmatrix*} \\\\
	E_2A&=&\begin{bmatrix*}[r]
	    1 & 0 & 0\\
	    0 & 1 & 0\\
	    -3 & 0 & 1
	\end{bmatrix*} 
	\begin{bmatrix*}[r]
	    1 & -1 & 1\\
	    0 & 2 & -1\\
	    3 & 0 & 1
	\end{bmatrix*} &=& 
	\begin{bmatrix*}[r]
	    1 & -1 & 1\\
	    0 & 2 & -1\\
	    0 & 3 & -2
	\end{bmatrix*} \\\\
    \end{array}$$

    $$\begin{array}{rcccc}
	\begin{bmatrix*}[r]
	    1 & -1 & 1\\
	    0 & 2 & -1\\
	    0 & 3 & -2
	\end{bmatrix*} && \dfrac{R_2}{2} \to R_2 & &
	\begin{bmatrix*}[r]
	    1 & -1 & 1\\
	    0 & 1 & -\frac{1}{2}\\
	    0 & 3 & -2
	\end{bmatrix*} \\\\
	E_3A&=&\begin{bmatrix*}[r]
	    1 & 0 & 0\\
	    0 & \frac{1}{2} & 0\\
	    0 & 0 & 1
	\end{bmatrix*} 
	\begin{bmatrix*}[r]
	    1 & -1 & 1\\
	    0 & 2 & -1\\
	    0 & 3 & -2
	\end{bmatrix*} &=& 
	\begin{bmatrix*}[r]
	    1 & -1 & 1\\
	    0 & 1 & -\frac{1}{2}\\
	    0 & 3 & -2
	\end{bmatrix*} \\\\
    \end{array}$$

    $$\begin{array}{rcccc}
	\begin{bmatrix*}[r]
	    1 & -1 & 1\\
	    0 & 1 & -\frac{1}{2}\\
	    0 & 3 & -2
	\end{bmatrix*} && R_1+R_2\to R_1 & &
	\begin{bmatrix*}[r]
	    1 & 0 & \frac{1}{2}\\
	    0 & 1 & -\frac{1}{2}\\
	    0 & 3 & -2
	\end{bmatrix*} \\\\
	E_4A&=&\begin{bmatrix*}[r]
	    1 & 1 & 0\\
	    0 & 1 & 0\\
	    0 & 0 & 1
	\end{bmatrix*} 
	\begin{bmatrix*}[r]
	    1 & -1 & 1\\
	    0 & 1 & -\frac{1}{2}\\
	    0 & 3 & -2
	\end{bmatrix*} &=& 
	\begin{bmatrix*}[r]
	    1 & 0 & \frac{1}{2}\\
	    0 & 1 & -\frac{1}{2}\\
	    0 & 3 & -2
	\end{bmatrix*} \\\\
    \end{array}$$

    $$\begin{array}{rcccc}
	\begin{bmatrix*}[r]
	    1 & 0 & \frac{1}{2}\\
	    0 & 1 & -\frac{1}{2}\\
	    0 & 3 & -2
	\end{bmatrix*} && R_3-3R_2\to R_3 & &
	\begin{bmatrix*}[r]
	    1 & 0 & \frac{1}{2}\\
	    0 & 1 & -\frac{1}{2}\\
	    0 & 3 & -\frac{1}{2}
	\end{bmatrix*} \\\\
	E_5A&=&\begin{bmatrix*}[r]
	    1 & 1 & 0\\
	    0 & 1 & 0\\
	    0 & -3 & 1
	\end{bmatrix*} 
	\begin{bmatrix*}[r]
	    1 & 0 & \frac{1}{2}\\
	    0 & 1 & -\frac{1}{2}\\
	    0 & 3 & -2
	\end{bmatrix*} &=& 
	\begin{bmatrix*}[r]
	    1 & 0 & \frac{1}{2}\\
	    0 & 1 & -\frac{1}{2}\\
	    0 & 3 & -\frac{1}{2}
	\end{bmatrix*} \\\\
    \end{array}$$

    $$\begin{array}{rcccc}
	\begin{bmatrix*}[r]
	    1 & 0 & \frac{1}{2}\\
	    0 & 1 & -\frac{1}{2}\\
	    0 & 3 & -\frac{1}{2}
	\end{bmatrix*} && -\dfrac{R_3}{2}\to R_3 & &
	\begin{bmatrix*}[r]
	    1 & 0 & \frac{1}{2}\\
	    0 & 1 & -\frac{1}{2}\\
	    0 & 0 & 1 
	\end{bmatrix*} \\\\
	E_6A&=&\begin{bmatrix*}[r]
	    1 & 1 & 0\\
	    0 & 1 & 0\\
	    0 & 0 & -2 
	\end{bmatrix*} 
	\begin{bmatrix*}[r]
	    1 & 0 & \frac{1}{2}\\
	    0 & 1 & -\frac{1}{2}\\
	    0 & 3 & -\frac{1}{2}
	\end{bmatrix*} &=& 
	\begin{bmatrix*}[r]
	    1 & 0 & \frac{1}{2}\\
	    0 & 1 & -\frac{1}{2}\\
	    0 & 0 & 1 
	\end{bmatrix*} \\\\
    \end{array}$$

    $$\begin{array}{rcccc}
	\begin{bmatrix*}[r]
	    1 & 0 & \frac{1}{2}\\
	    0 & 1 & -\frac{1}{2}\\
	    0 & 0 & 1 
	\end{bmatrix*} && R_1-\dfrac{R_3}{2}\to R_1 & &
	\begin{bmatrix*}[r]
	    1 & 0 & 0\\
	    0 & 1 & -\frac{1}{2}\\
	    0 & 0 & 1 
	\end{bmatrix*} \\\\
	E_7A&=&\begin{bmatrix*}[r]
	    1 & 1 & -\frac{1}{2}\\
	    0 & 1 & 0\\
	    0 & 0 & 1 
	\end{bmatrix*} 
	\begin{bmatrix*}[r]
	    1 & 0 & \frac{1}{2}\\
	    0 & 1 & -\frac{1}{2}\\
	    0 & 0 & 1 
	\end{bmatrix*} &=& 
	\begin{bmatrix*}[r]
	    1 & 0 & 0\\
	    0 & 1 & -\frac{1}{2}\\
	    0 & 0 & 1 
	\end{bmatrix*} \\\\
    \end{array}$$

    $$\begin{array}{rcccc}
	\begin{bmatrix*}[r]
	    1 & 0 & 0\\
	    0 & 1 & -\frac{1}{2}\\
	    0 & 0 & 1 
	\end{bmatrix*} && R_2+\dfrac{R_3}{2}\to R_2 & &
	\begin{bmatrix*}[r]
	    1 & 0 & 0\\
	    0 & 1 & 0\\
	    0 & 0 & 1 
	\end{bmatrix*} \\\\
	E_8A&=&\begin{bmatrix*}[r]
	    1 & 1 & 0\\
	    0 & 1 & \frac{1}{2}\\
	    0 & 0 & 1 
	\end{bmatrix*} 
	\begin{bmatrix*}[r]
	    1 & 0 & 0\\
	    0 & 1 & -\frac{1}{2}\\
	    0 & 0 & 1 
	\end{bmatrix*} &=& 
	\begin{bmatrix*}[r]
	    1 & 0 & 0\\
	    0 & 1 & 0\\
	    0 & 0 & 1 
	\end{bmatrix*} \\\\
    \end{array}$$

    Por lo tanto, la sucesión de matrices elementales son $E_1,E_2,\ldots , E_8$ tal que
    $$E_8 E_7 E_6 E_5 E_4 E_3 E_2 E_1 A = I.$$\\

    %-------------------- section 1.5 7
    \item \textbf{\boldmath Sean $A$ y $B$ matrices $2\times 2$ tales que $AB=I$. Demostrar que $BA=I.$\\\\
	Demostración.-}\; Ya que $AB=I$, entonces $A,B\neq 0.$ Luego
	$$\begin{array}{rcl}
	    AB=I &\Rightarrow & ABA = IA = A\\
		 &\Rightarrow & ABA-A = I\\
		 &\Rightarrow & A(BA-I) = I\\
		 &\Rightarrow & BA-I = 0\\	
		 &\Rightarrow & BA = I.
	\end{array}$$
	\vspace{0.5cm}

    %-------------------- section 1.6 1
    \item \textbf{\boldmath Sea,
	$$A=\begin{bmatrix*}[r]
	    1 & 2 & 1 & 0\\
	    -1 & 0 & 3 & 5\\
	    1 & -2 & 1 & 1
    \end{bmatrix*}$$
    Hallar una matriz escalón reducida por filas $R$ que sea equivalente a $A$, y una matriz inversible $3\times 3$, $P$ tal que $R=PA.$\\\\
	Respuesta.-}\; Reducimos por filas AA aplicando las operaciones elementales por filas y, de manera equivalente, en cada operación determinamos la matriz elemental.

    %-------------------- section 1.6 3
    \item \textbf{\boldmath Repetir el ejercicio 1, pero con 
	$$A=\begin{bmatrix*}[r]
	    2 & 0 & i\\
	    1 & -3 & -i\\
	    i & 1 & 1
	\end{bmatrix*}$$\\
	Respuesta.-}\;

    %-------------------- section 1.6 4
    \item \textbf{\boldmath Sea 
	$$A=\begin{bmatrix*}[r]
	    5&0&0\\
	    1&5&0\\
	    0&1&5
	\end{bmatrix*}$$
	¿Para qué $X$ existe un escalar $c$ tal que $AX=cX$?.\\\\
	Respuesta.-}\;

    %-------------------- section 1.6 5
    \item \textbf{\boldmath Determinar si
	$$A=\begin{bmatrix*}[r]
	    1&2&3&4\\
	    0&2&3&4\\
	    0&0&3&4\\
	    0&0&0&4
	\end{bmatrix*}$$
	es inversible y hallar $A^{-1}$ si existe.\\\\
	Respuesta.-}\;

    %-------------------- section 1.6 6
    \item \textbf{\boldmath Supóngase que $A$ es una matriz $2\times 1$ y que $B$ es una matriz $1\times 2$. Demostrar que $C=AB$ no es inversible.\\\\
	Demostración.-}\; Sea $$A=\begin{bmatrix*}[r]
	    a_1\\
	    a_2
	\end{bmatrix*} \qquad \mbox{y} \qquad B=\begin{bmatrix*}[r]
	b_1 & b_2
    \end{bmatrix*}$$
    Para cualquier matriz $2\times 1$ y $1\times 3$ respectivamente. Luego considere $C$ tal que,
    $$C=AB=\begin{bmatrix*}[r]
	a_1b_1 & a_1b_2\\
	a_2b_1 & a_2b_2
    \end{bmatrix*}$$
    Ya que el determinante de $C$ es cero. Es decir, $|C|=a_1b_1a_2b_2-a_2b_1a_1b_2=0$, entonces $C$ es no invertible.\\\\
    

    %-------------------- section 1.6 9
    \item \textbf{\boldmath Una matriz $n\times n$, $A$, se llama triangular superior si $A_{ij}=0$ para $i>j$; esto es, si todo elemento por debajo de la diagonal principal es $0$. Demostrar que una matriz (cuadrada) triangular superior es inversible si, y sólo si, cada elemento de su diagonal principal es diferente de $0$.\\\\
	Demostración.-}\;

    %-------------------- section 1.6 10
    \item \textbf{\boldmath Demostrar la siguiente generalización del ejercicio 6. Si $A$ es una matriz $m\times n$, $B$ es una matriz $n\times m$ y $n<m$, entonces $AB$ no es inversible.\\\\
	Demostración.-}\; Sea $A'$ una matriz $m\times m$ obtenido al unir $m-n$ columnas cero para $A$. Es decir,  

    %-------------------- section 1.6 12
    \item \textbf{\boldmath El resultado del ejemplo 16 sugiere que tal vez la matriz
	$$A=\begin{bmatrix*}[c]
	    1&\dfrac{1}{2}&\cdots&\dfrac{1}{n}\\\\
	    \dfrac{1}{2}&\dfrac{1}{3}&\cdots&\dfrac{1}{n+1}\\\\
	    \vdots&\vdots&&\vdots\\\\
		  &\dfrac{1}{n}&\cdots&\dfrac{1}{2n-1}\\\\
	\end{bmatrix*}$$
    es inversible y $A^{-1}$ tiene elementos enteros.¿Se puede demostrar esto?.\\\\
	Respuesta.-}\;


\end{enumerate}
