
\begin{enumerate}[\bfseries \mbox{Ejercicio.} 1.]

    %-------------------- section 1.5 3 (1)
    \item \textbf{\boldmath Encontrar dos matrices $2\times 2,A$ diferentes tales que $A^2=0$ pero $A\neq 0.$\\\\
	Respuesta.-}\; Consideremos las siguientes matrices
	$$
	A=\begin{bmatrix*}[r]
		0 & 1\\
		0 & 0
	    \end{bmatrix*}\neq 0 \quad \mbox{y}\quad A^2=
	    \begin{bmatrix*}[r]
		0 & 1\\
		0 & 0
	    \end{bmatrix*}
	    \begin{bmatrix*}[r]
		0 & 1\\
		0 & 0
	    \end{bmatrix*} = 
	    \begin{bmatrix*}[r]
		0 & 0\\
		0 & 0
	    \end{bmatrix*} = 0
	    $$

	$$
	A=\begin{bmatrix*}[r]
		0 & 0\\
		1 & 0
	    \end{bmatrix*}\neq 0 \quad \mbox{y}\quad A^2=
	    \begin{bmatrix*}[r]
		0 & 0\\
		1 & 0
	    \end{bmatrix*}
	    \begin{bmatrix*}[r]
		0 & 0\\
		1 & 0
	    \end{bmatrix*} = 
	    \begin{bmatrix*}[r]
		0 & 0\\
		0 & 0
	    \end{bmatrix*} = 0
	    $$

	De este modo encontramos dos matrices tal que $A^2=0$.\\\\


    %-------------------- section 1.5 4 (2)
    \item \textbf{\boldmath Para cada $A$ del ejercicio 2, hallar matrices elementales $E_1,E_2,\ldots,E_k$ tal que 
	$$E_k\cdot E_2E_1A=I.$$\\
    Respuesta.-}\; Considere la matriz $A$ dada y reduzca por filas de la siguiente manera y mencione correspondientemente las matrices elementales.
    $$\begin{array}{rcccc}
	\begin{bmatrix*}[r]
	    1 & -1 & 1\\
	    2 & 0 & 1\\
	    3 & 0 & 1
	\end{bmatrix*} && R_2-2R_1 \to R_2 & &
	\begin{bmatrix*}[r]
	    1 & -1 & 1\\
	    0 & 2 & -1\\
	    3 & 0 & 1
	\end{bmatrix*} \\\\
	E_1A&=&\begin{bmatrix*}[r]
	    1 & 0 & 0\\
	    -2 & 1 & 0\\
	    0 & 0 & 1
	\end{bmatrix*} 
	\begin{bmatrix*}[r]
	    1 & -1 & 1\\
	    2 & 0 & 1\\
	    3 & 0 & 1
	\end{bmatrix*} &=& 
	\begin{bmatrix*}[r]
	    1 & -1 & 1\\
	    0 & 2 & -1\\
	    3 & 0 & 1
	\end{bmatrix*} \\\\
    \end{array}$$

    $$\begin{array}{rcccc}
	\begin{bmatrix*}[r]
	    1 & -1 & 1\\
	    0 & 2 & -1\\
	    3 & 0 & 1
	\end{bmatrix*} && R_3-3R_1\to R_3 & &
	\begin{bmatrix*}[r]
	    1 & -1 & 1\\
	    0 & 2 & -1\\
	    0 & 3 & -2
	\end{bmatrix*} \\\\
	E_2A&=&\begin{bmatrix*}[r]
	    1 & 0 & 0\\
	    0 & 1 & 0\\
	    -3 & 0 & 1
	\end{bmatrix*} 
	\begin{bmatrix*}[r]
	    1 & -1 & 1\\
	    0 & 2 & -1\\
	    3 & 0 & 1
	\end{bmatrix*} &=& 
	\begin{bmatrix*}[r]
	    1 & -1 & 1\\
	    0 & 2 & -1\\
	    0 & 3 & -2
	\end{bmatrix*} \\\\
    \end{array}$$

    $$\begin{array}{rcccc}
	\begin{bmatrix*}[r]
	    1 & -1 & 1\\
	    0 & 2 & -1\\
	    0 & 3 & -2
	\end{bmatrix*} && \dfrac{R_2}{2} \to R_2 & &
	\begin{bmatrix*}[r]
	    1 & -1 & 1\\
	    0 & 1 & -\frac{1}{2}\\
	    0 & 3 & -2
	\end{bmatrix*} \\\\
	E_3A&=&\begin{bmatrix*}[r]
	    1 & 0 & 0\\
	    0 & \frac{1}{2} & 0\\
	    0 & 0 & 1
	\end{bmatrix*} 
	\begin{bmatrix*}[r]
	    1 & -1 & 1\\
	    0 & 2 & -1\\
	    0 & 3 & -2
	\end{bmatrix*} &=& 
	\begin{bmatrix*}[r]
	    1 & -1 & 1\\
	    0 & 1 & -\frac{1}{2}\\
	    0 & 3 & -2
	\end{bmatrix*} \\\\
    \end{array}$$

    $$\begin{array}{rcccc}
	\begin{bmatrix*}[r]
	    1 & -1 & 1\\
	    0 & 1 & -\frac{1}{2}\\
	    0 & 3 & -2
	\end{bmatrix*} && R_1+R_2\to R_1 & &
	\begin{bmatrix*}[r]
	    1 & 0 & \frac{1}{2}\\
	    0 & 1 & -\frac{1}{2}\\
	    0 & 3 & -2
	\end{bmatrix*} \\\\
	E_4A&=&\begin{bmatrix*}[r]
	    1 & 1 & 0\\
	    0 & 1 & 0\\
	    0 & 0 & 1
	\end{bmatrix*} 
	\begin{bmatrix*}[r]
	    1 & -1 & 1\\
	    0 & 1 & -\frac{1}{2}\\
	    0 & 3 & -2
	\end{bmatrix*} &=& 
	\begin{bmatrix*}[r]
	    1 & 0 & \frac{1}{2}\\
	    0 & 1 & -\frac{1}{2}\\
	    0 & 3 & -2
	\end{bmatrix*} \\\\
    \end{array}$$

    $$\begin{array}{rcccc}
	\begin{bmatrix*}[r]
	    1 & 0 & \frac{1}{2}\\
	    0 & 1 & -\frac{1}{2}\\
	    0 & 3 & -2
	\end{bmatrix*} && R_3-3R_2\to R_3 & &
	\begin{bmatrix*}[r]
	    1 & 0 & \frac{1}{2}\\
	    0 & 1 & -\frac{1}{2}\\
	    0 & 3 & -\frac{1}{2}
	\end{bmatrix*} \\\\
	E_5A&=&\begin{bmatrix*}[r]
	    1 & 1 & 0\\
	    0 & 1 & 0\\
	    0 & -3 & 1
	\end{bmatrix*} 
	\begin{bmatrix*}[r]
	    1 & 0 & \frac{1}{2}\\
	    0 & 1 & -\frac{1}{2}\\
	    0 & 3 & -2
	\end{bmatrix*} &=& 
	\begin{bmatrix*}[r]
	    1 & 0 & \frac{1}{2}\\
	    0 & 1 & -\frac{1}{2}\\
	    0 & 3 & -\frac{1}{2}
	\end{bmatrix*} \\\\
    \end{array}$$

    $$\begin{array}{rcccc}
	\begin{bmatrix*}[r]
	    1 & 0 & \frac{1}{2}\\
	    0 & 1 & -\frac{1}{2}\\
	    0 & 3 & -\frac{1}{2}
	\end{bmatrix*} && -\dfrac{R_3}{2}\to R_3 & &
	\begin{bmatrix*}[r]
	    1 & 0 & \frac{1}{2}\\
	    0 & 1 & -\frac{1}{2}\\
	    0 & 0 & 1 
	\end{bmatrix*} \\\\
	E_6A&=&\begin{bmatrix*}[r]
	    1 & 1 & 0\\
	    0 & 1 & 0\\
	    0 & 0 & -2 
	\end{bmatrix*} 
	\begin{bmatrix*}[r]
	    1 & 0 & \frac{1}{2}\\
	    0 & 1 & -\frac{1}{2}\\
	    0 & 3 & -\frac{1}{2}
	\end{bmatrix*} &=& 
	\begin{bmatrix*}[r]
	    1 & 0 & \frac{1}{2}\\
	    0 & 1 & -\frac{1}{2}\\
	    0 & 0 & 1 
	\end{bmatrix*} \\\\
    \end{array}$$

    $$\begin{array}{rcccc}
	\begin{bmatrix*}[r]
	    1 & 0 & \frac{1}{2}\\
	    0 & 1 & -\frac{1}{2}\\
	    0 & 0 & 1 
	\end{bmatrix*} && R_1-\dfrac{R_3}{2}\to R_1 & &
	\begin{bmatrix*}[r]
	    1 & 0 & 0\\
	    0 & 1 & -\frac{1}{2}\\
	    0 & 0 & 1 
	\end{bmatrix*} \\\\
	E_7A&=&\begin{bmatrix*}[r]
	    1 & 1 & -\frac{1}{2}\\
	    0 & 1 & 0\\
	    0 & 0 & 1 
	\end{bmatrix*} 
	\begin{bmatrix*}[r]
	    1 & 0 & \frac{1}{2}\\
	    0 & 1 & -\frac{1}{2}\\
	    0 & 0 & 1 
	\end{bmatrix*} &=& 
	\begin{bmatrix*}[r]
	    1 & 0 & 0\\
	    0 & 1 & -\frac{1}{2}\\
	    0 & 0 & 1 
	\end{bmatrix*} \\\\
    \end{array}$$

    $$\begin{array}{rcccc}
	\begin{bmatrix*}[r]
	    1 & 0 & 0\\
	    0 & 1 & -\frac{1}{2}\\
	    0 & 0 & 1 
	\end{bmatrix*} && R_2+\dfrac{R_3}{2}\to R_2 & &
	\begin{bmatrix*}[r]
	    1 & 0 & 0\\
	    0 & 1 & 0\\
	    0 & 0 & 1 
	\end{bmatrix*} \\\\
	E_8A&=&\begin{bmatrix*}[r]
	    1 & 1 & 0\\
	    0 & 1 & \frac{1}{2}\\
	    0 & 0 & 1 
	\end{bmatrix*} 
	\begin{bmatrix*}[r]
	    1 & 0 & 0\\
	    0 & 1 & -\frac{1}{2}\\
	    0 & 0 & 1 
	\end{bmatrix*} &=& 
	\begin{bmatrix*}[r]
	    1 & 0 & 0\\
	    0 & 1 & 0\\
	    0 & 0 & 1 
	\end{bmatrix*} \\\\
    \end{array}$$

    Por lo tanto, la sucesión de matrices elementales son $E_1,E_2,\ldots , E_8$ tal que
    $$E_8 E_7 E_6 E_5 E_4 E_3 E_2 E_1 A = I.$$\\\\

    %-------------------- section 1.5 7 (3)
    \item \textbf{\boldmath Sean $A$ y $B$ matrices $2\times 2$ tales que $AB=I$. Demostrar que $BA=I.$\\\\
	Demostración.-}\; Ya que $AB=I$, entonces $A,B\neq 0.$ Luego
	$$\begin{array}{rcl}
	    AB=I &\Rightarrow & ABA = IA = A\\
		 &\Rightarrow & ABA-A = I\\
		 &\Rightarrow & A(BA-I) = I\\
		 &\Rightarrow & BA-I = 0\\	
		 &\Rightarrow & BA = I.
	\end{array}$$
	\vspace{0.5cm}

    %-------------------- section 1.6 1 (4)
    \item \textbf{\boldmath Sea,
	$$A=\begin{bmatrix*}[r]
	    1 & 2 & 1 & 0\\
	    -1 & 0 & 3 & 5\\
	    1 & -2 & 1 & 1
    \end{bmatrix*}$$
    Hallar una matriz escalón reducida por filas $R$ que sea equivalente a $A$, y una matriz inversible $3\times 3$, $P$ tal que $R=PA.$\\\\
	Respuesta.-}\; Reducimos por filas $A$ aplicando las operaciones elementales por filas y, de manera equivalente, en cada operación determinamos la matriz elemental.
	$$\begin{bmatrix*}[r]
	    1 & 2 & 1 & 0\\
	    -1 & 0 & 3 & 5\\
	    1 & -2 & 1 & 1
	\end{bmatrix*}\to 
	\begin{bmatrix*}[r]
	    1 & 0 & 0\\
	    0 & 1 & 0\\
	    0 & 0 & 1
	\end{bmatrix*}$$

	$$\begin{array}{rcccl}
	    &&R_2+R_1\to R_2 &&\\\\
	    \begin{bmatrix*}[r]
		1 & 2 & 1 & 0\\
		0 & 2 & 4 & 5\\
		1 & -2 & 1 & 1
	    \end{bmatrix*}
	    &\to&
	    \begin{bmatrix*}[r]
		1 & 0 & 0\\
		1 & 1 & 0\\
		0 & 0 & 1
	    \end{bmatrix*}
	    &\to&
	    E_1 = \begin{bmatrix*}[r]
		1 & 0 & 0\\
		1 & 1 & 0\\
		0 & 0 & 1
	    \end{bmatrix*}\\\\

	    &&R_3+R_1\to R_2 &&\\\\
	    \begin{bmatrix*}[r]
		1 & 2 & 1 & 0\\
		0 & 2 & 4 & 5\\
		0 & -4 & 0 & 1
	    \end{bmatrix*}
	    &\to&
	    \begin{bmatrix*}[r]
		1 & 0 & 0\\
		1 & 1 & 0\\
		-1 & 0 & 1
	    \end{bmatrix*}
	    &\to&
	    E_2 = \begin{bmatrix*}[r]
		1 & 0 & 0\\
		1 & 1 & 0\\
		-1 & 0 & 1
	    \end{bmatrix*}\\\\

	    &&R_2/2 \to R_2 &&\\\\
	    \begin{bmatrix*}[r]
		1 & 2 & 1 & 0\\
		0 & 1 & 2 & \frac{5}{2}\\
		0 & -4 & 0 & 1
	    \end{bmatrix*}
	    &\to&
	    \begin{bmatrix*}[r]
		1 & 0 & 0\\
		\frac{1}{2} & \frac{1}{2} & 0\\
		-1 & 0 & 1
	    \end{bmatrix*}
	    &\to&
	    E_3 = \begin{bmatrix*}[r]
		1 & 0 & 0\\
		1 & \frac{1}{2} & 0\\
		0 & 0 & 1
	    \end{bmatrix*}\\\\

	    &&R_1-2R_2 \to R_1 &&\\\\
	    \begin{bmatrix*}[r]
		1 & 0 -3 & -5\\
		0 & 1 & 2 & \frac{5}{2}\\
		0 & -4 & 0 & 1
	    \end{bmatrix*}
	    &\to&
	    \begin{bmatrix*}[r]
		1 & -1 & 0\\
		\frac{1}{2} & \frac{1}{2} & 0\\
		-1 & 0 & 1
	    \end{bmatrix*}
	    &\to&
	    E_4 = \begin{bmatrix*}[r]
		1 & -2 & 0\\
		0 & 1 & 0\\
		0 & 0 & 1
	    \end{bmatrix*}\\\\

	\end{array}$$

	$$\begin{array}{rcccl}

	    &&R_3-4R_2 \to R_3 &&\\\\
	    \begin{bmatrix*}[r]
		1 & 0 -3 & -5\\
		0 & 1 & 2 & \frac{5}{2}\\
		0 & 0 & 8 & 11
	    \end{bmatrix*}
	    &\to&
	    \begin{bmatrix*}[r]
		1 & -1 & 0\\
		\frac{1}{2} & \frac{1}{2} & 0\\
		1 & 2 & 1
	    \end{bmatrix*}
	    &\to&
	    E_5 = \begin{bmatrix*}[r]
		1 & 0 & 0\\
		0 & 1 & 0\\
		0 & 4 & 1
	    \end{bmatrix*}\\\\


	    &&R_3/8 \to R_3 &&\\\\
	    \begin{bmatrix*}[r]
		1 & 0 -3 & -5\\
		0 & 1 & 2 & \frac{5}{2}\\
		0 & 0 & 1 & \frac{11}{8}
	    \end{bmatrix*}
	    &\to&
	    \begin{bmatrix*}[r]
		1 & -1 & 0\\
		\frac{1}{2} & \frac{1}{2} & 0\\
		\frac{1}{8} & \frac{1}{4} & \frac{1}{8} 
	    \end{bmatrix*}
	    &\to&
	    E_6 = \begin{bmatrix*}[r]
		1 & 0 & 0\\
		0 & 1 & 0\\
		0 & 0 & \frac{1}{8}
	    \end{bmatrix*}\\\\

	    &&R_2-2R_3 \to R_2 &&\\\\
	    \begin{bmatrix*}[r]
		1 & 0 -3 & -5\\
		0 & 1 & 0 & -\frac{1}{4}\\
		0 & 0 & 1 & \frac{11}{8}
	    \end{bmatrix*}
	    &\to&
	    \begin{bmatrix*}[r]
		0 & -1 & 0\\
		\frac{1}{4} & 0 & -\frac{1}{4}\\
		\frac{1}{8} & \frac{1}{4} & \frac{1}{8} 
	    \end{bmatrix*}
	    &\to&
	    E_7 = \begin{bmatrix*}[r]
		1 & 0 & 0\\
		0 & 1 & -2\\
		0 & 0 & 1
	    \end{bmatrix*}\\\\

	    &&R_1-3R_3 \to R_1 &&\\\\
	    \begin{bmatrix*}[r]
		1 & 0 & 0 & -\frac{7}{8}\\
		0 & 1 & 0 & -\frac{1}{4}\\
		0 & 0 & 1 & \frac{11}{8}
	    \end{bmatrix*}
	    &\to&
	    \begin{bmatrix*}[r]
		\frac{3}{8} & -\frac{1}{4} & \frac{3}{8}\\
		\frac{1}{4} & 0 & -\frac{1}{4}\\
		\frac{1}{8} & \frac{1}{4} & \frac{1}{8}
	    \end{bmatrix*}
	    &\to&
	    E_7 = \begin{bmatrix*}[r]
		1 & 0 & 3\\
		0 & 1 & 0\\
		0 & 0 & 1
	    \end{bmatrix*}\\\\

	\end{array}$$

	Por lo tanto, la matriz escalonada reducida por filas que es equivalente por filas a $A$ es
	$$R=\begin{bmatrix*}[r]
	    1 & 0 & 0 & -\frac{7}{8}\\
	    0 & 1 & 0 & -\frac{1}{4}\\
	    0 & 0 & 1 & \frac{11}{8}
	\end{bmatrix*}$$
	Claramente, de los pasos mencionados anteriormente, $E_1, E_2, \ldots, E_8$ son las matrices elementales que transforman $A$ en $R$. De este modo,
	$$E_8E_7E_6E_5E_4E_3E_2E_1A = R \quad \Rightarrow \quad P = E_8E_7E_6E_5E_4E_3E_2E_1.$$

	Sabemos que las matrices elementales son invertibles y el producto de matrices invertibles es invertible, por lo tanto, $P=E_8E_7E_6E_5E_4E_3E_2E_1P=E$ es invertible.\\

	De los pasos, obtenemos

	$$P=\begin{bmatrix*}[r]
	    \frac{3}{8} & -\frac{1}{4} & \frac{3}{8}\\
	    \frac{1}{4} & 0 & -\frac{1}{4}\\
	    \frac{1}{8} & \frac{1}{4} & \frac{1}{8}
	\end{bmatrix*}$$

	Así,

	$$R=\begin{bmatrix*}[r]
	    1 & 0 & 0 & -\frac{7}{8}\\
	    0 & 1 & 0 & -\frac{1}{4}\\
	    0 & 0 & 1 & \frac{11}{8}
	\end{bmatrix*} \qquad \mbox{y}\qquad 
	P=\begin{bmatrix*}[r]
	    \frac{3}{8} & -\frac{1}{4} & \frac{3}{8}\\
	    \frac{1}{4} & 0 & -\frac{1}{4}\\
	    \frac{1}{8} & \frac{1}{4} & \frac{1}{8}
	\end{bmatrix*}$$

	Tal que $$R=PA.$$\\
	

    %-------------------- section 1.6 3 (5)
    \item \textbf{\boldmath Para cada una de las dos matrices 
	$$\begin{bmatrix*}[r]
	    2 & 0 & -1\\
	    4 & -1 & 2\\
	    6 & 4 & 1
	\end{bmatrix*},\qquad 
	\begin{bmatrix*}[r]
	    1 & -1 & 2\\
	    3 & 2 & 4\\
	    0 & 1 & -2
	\end{bmatrix*}$$\\
	emplear operaciones elementales de fila para determinar cuando es invertible y encontrar la inversa en caso que lo sea.\\\\
	Respuesta.-}\; Primero, considere la matriz $A$. Añadimos la matriz de identidad a $A$ para formar un sistema aumentado como $[A|I]$. Ahora, aplicando operaciones elementales de renglones y reducción de renglones $A$ de la siguiente manera:
	$$\begin{array}{rcccc}
	    &[A|I]&\\\\
	    \begin{bmatrix*}[r]
		2 & 5 & -1 && 1 & 0 & 0\\
		4 & -1 & 2 && 0 & 1 & 0\\
		6 & 4 & 1 && 0 & 0 & 1
	    \end{bmatrix*} & 
	    \begin{array}{rcl}
		R_1/2 &\to & R_1\\
	    \end{array}&
	    \begin{bmatrix*}[r]
		1 & \frac{5}{2} & -\frac{1}{2} && \frac{1}{2} & 0 & 0\\
		4 & -1 & 2 && 0 & 1 & 0\\
		6 & 4 & 1 && 0 & 0 & 1
	    \end{bmatrix*}\\\\ 
	    &\begin{array}{rcl}
		R_2-4R_1 &\to & R_2\\
		R_3-6R_1 &\to & R_3
	    \end{array} &
	    \begin{bmatrix*}[r]
		1 & \frac{5}{2} & -\frac{1}{2} && \frac{1}{2} & 0 & 0\\
		0 & -11 & 4 && -2 & 1 & 0\\
		0 & -11 & 4 && -3 & 0 & 1
	    \end{bmatrix*}\\\\ 
	    &\begin{array}{rcl}
		R_3-R_2 &\to & R_3\\
		\end{array} &
	    \begin{bmatrix*}[r]
		1 & \frac{5}{2} & -\frac{1}{2} && \frac{1}{2} & 0 & 0\\
		0 & -11 & 4 && -2 & 1 & 0\\
		0 & 0 & 0 && -1 & -1 & 1
	    \end{bmatrix*}\\\\ 
	\end{array}$$

	Dado que la última fila es cero. Por lo tanto, $A$ no puede reducirse por filas a la identidad. Es decir, $A$ no es equivalente por filas a la matriz identidad. Así, $A$ no es invertible.\\
	Ahora, considere $B$ y añadimos una matriz identidad para formar un sistema aumentado $[B|I]$. Aplicando operaciones de fila y reducción de fila $B$ de la siguiente manera

	$$\begin{array}{rcl}
	    &[B|I]&\\\\

	    \begin{bmatrix*}[r]
		1 & -1 & 2 && 1 & 0 & 0\\
		3 & 2 & 4 && 0 & 1 & 0\\
		0 & 1 & -2 && 0 & 0 & 1
	    \end{bmatrix*} & 
		R_2-3R_1 \to  R_2
	    &
	    \begin{bmatrix*}[r]
		1 & -1 & 2 && 1 & 0 & 0\\
		0 & 5 & 0 && -3 & 1 & 0\\
		0 & 1 & -2 && 0 & 0 & 1
	    \end{bmatrix*}\\\\ 

	    &R_2/5 \to  R_2&
	    \begin{bmatrix*}[r]
		1 & -1 & 2 && 1 & 0 & 0\\
		0 & 1 & -\frac{2}{5} && -\frac{3}{5} & \frac{1}{5} & 0\\
		0 & 1 & -2 && 0 & 0 & 1
	    \end{bmatrix*}\\\\

	    &\begin{array}{c}
		R_1+R_2 \to  R_1\\\\
		R_3-R_2 \to  R_3
	    \end{array} &
	    \begin{bmatrix*}[r]
		1 & 0 & 1.6 && 0.4 & 0.2 & 0\\
		0 & 1 & -0.4 && -0.6 & 0.2 & 0\\
		0 & 0 & -1.6 && 0.6 & -0.2 & 1
	    \end{bmatrix*}\\\\ 

	    &R_3/(-1.6) \to  R_3&
	    \begin{bmatrix*}[r]
		1 & 0 & 1.6 && 0.4 & 0.2 & 0\\
		0 & 1 & -0.4 && -0.6 & 0.2 & 0\\
		0 & 0 & 1 && -.375 & .125 & -.625 
	    \end{bmatrix*}\\\\ 
	    &\begin{array}{c}
		R_1-1.6R_3 \to  R_1\\\\
		R_2+0.4R_3 \to  R_2
	    \end{array} &
	    \begin{bmatrix*}[r]
		1 & 0 & 0 && 1 & 0 & 1\\
		0 & 1 & 0 && -0.75 & 0.25 & -0.25\\
		0 & 0 & 1 && -0.375 & 0.125 & -0.625
	    \end{bmatrix*}\\\\ 
	    &[I|C]&\\\\
	\end{array}$$

    %-------------------- section 1.6 4 (6)
    \item \textbf{\boldmath Sea 
	$$A=\begin{bmatrix*}[r]
	    5&0&0\\
	    1&5&0\\
	    0&1&5
	\end{bmatrix*}$$
	¿Para qué $X$ existe un escalar $c$ tal que $AX=cX$?.\\\\
	Respuesta.-}\; Para $X=0$, $AX=cX$ se cumple trivialmente para todos los $c$. Ahora, considere $x \neq 0$, entonces
	$$AX=cX\; \Rightarrow \; AX-cX=0\; \Rightarrow \; (A-cI)X=0\; \Rightarrow \; A-cI=0\; \Rightarrow \; A=cI.$$
	Es decir,
	$$\begin{bmatrix*}[r]
	    5&0&0\\
	    1&5&0\\
	    0&1&5
	\end{bmatrix*}=
	\begin{bmatrix*}[r]
	    c&0&0\\
	    0&c&0\\
	    0&0&c
	\end{bmatrix*}.$$

	Por lo de arriba tenemos $c=5$ y $1=0$ lo cual es absurdo. Por lo tanto, para cualquier $x\neq 0$ no existe un $c$ tal que $AX=cX.$ Así, para solo un $x=0$ eiste un $c$ tal que $AX=cX.$\\\\


    %-------------------- section 1.6 5 (7)
    \item \textbf{\boldmath Determinar si
	$$A=\begin{bmatrix*}[r]
	    1&2&3&4\\
	    0&2&3&4\\
	    0&0&3&4\\
	    0&0&0&4
	\end{bmatrix*}$$
	es inversible y hallar $A^{-1}$ si existe.\\\\
	Respuesta.-}\; Se tiene $|A|=1\cdot 2 \cdot 3 \cdot 4 = 24\neq 0$, por lo tanto $A$ es inversible. Para encontrar el inverso de A, agregar a $A$ la matriz identidad $I$ y forme un sistema aumentado $[A|I]$ y reduzcamos en filas usando operaciones elementales de filas de la siguiente manera:
	$$\begin{array}{rcl}
	    &[A|I]&\\\\
	    \begin{bmatrix*}[r]
		1&2&3&4&&1&0&0&0\\
		0&2&3&4&&0&1&0&0\\
		0&0&3&4&&0&0&1&0\\
		0&0&0&4&&0&0&0&1
	    \end{bmatrix*} 
	    &\begin{array}{rcl}
		R_1-R_2 &\to & R_1\\
		R_2-R_3 &\to & R_2\\
		R_3-R_4 &\to & R_3\\
	    \end{array}&
	    \begin{bmatrix*}[r]
		1&0&0&0&&1&-1&0&0\\
		0&2&0&0&&0&1&-1&0\\
		0&0&3&0&&0&0&1&-1\\
		0&0&0&4&&0&0&0&1
	    \end{bmatrix*}\\\\

	    &\begin{array}{rcl}
		R_2/2 &\to & R_2\\
		R_3/3 &\to & R_3\\
		R_4/4 &\to & R_4\\
	    \end{array}&
	    \begin{bmatrix*}[r]
		1&0&0&0&&1&-1&0&0\\
		0&1&0&0&&0&\frac{1}{2}&-\frac{1}{2}&0\\
		0&0&1&0&&0&0&\frac{1}{2}&-\frac{1}{3}\\
		0&0&0&1&&0&0&0&\frac{1}{4}
	    \end{bmatrix*}\\\\
	    &[I|A^{-1}]&
    \end{array}$$

    Del sistema aumentado reducido, obtenemos
	    $$\begin{bmatrix*}[r]
		1&-1&0&0\\
		0&\frac{1}{2}&-\frac{1}{2}&0\\
		0&0&\frac{1}{2}&-\frac{1}{3}\\
		0&0&0&\frac{1}{4}
	    \end{bmatrix*}$$
	    \vspace{0.5cm}


    %-------------------- section 1.6 6
    \item \textbf{\boldmath Supóngase que $A$ es una matriz $2\times 1$ y que $B$ es una matriz $1\times 2$. Demostrar que $C=AB$ no es inversible.\\\\
	Demostración.-}\; Sea $$A=\begin{bmatrix*}[r]
	    a_1\\
	    a_2
	\end{bmatrix*} \qquad \mbox{y} \qquad B=\begin{bmatrix*}[r]
	b_1 & b_2
    \end{bmatrix*}$$
    Para cualquier matriz $2\times 1$ y $1\times 3$ respectivamente. Luego considere $C$ tal que,
    $$C=AB=\begin{bmatrix*}[r]
	a_1b_1 & a_1b_2\\
	a_2b_1 & a_2b_2
    \end{bmatrix*}$$
    Ya que el determinante de $C$ es cero. Es decir, $|C|=a_1b_1a_2b_2-a_2b_1a_1b_2=0$, entonces $C$ es no invertible.\\\\
    

    %-------------------- section 1.6 9
    \item \textbf{\boldmath Una matriz $n\times n$, $A$, se llama triangular superior si $A_{ij}=0$ para $i>j$; esto es, si todo elemento por debajo de la diagonal principal es $0$. Demostrar que una matriz (cuadrada) triangular superior es inversible si, y sólo si, cada elemento de su diagonal principal es diferente de $0$.\\\\
	Demostración.-}\; Sea $A=\left[a_{ij}\right]_{n\times n}$ la matriz triangular superior. Por lo tanto, $a_{ij}=0$ para todo $i>j,i,j=1,2,\ldots, n$. Claramente $|A|=a_{11}a_{22}\cdots a_{nn}$. Ahora
	\begin{center}
	    \begin{tabular}{rcl}
		$A$ es invertible & $\Leftrightarrow$ & $A$ es fila equivalente a matriz de identidad\\\\
				  & $\Leftrightarrow$ & $|A|\neq 0$\\\\
				  & $\Leftrightarrow$ & $a_{11} a_{22} \cdots a_{nn}\neq 0$\\\\
				  & $\Leftrightarrow$ & cada $a_{ii}\neq 0$ para todo $i=1,2,\ldots , n$
	    \end{tabular}
	\end{center}
	Por lo tanto, Una matriz triangular superior $A$ de $n \times n$ es invertible si y solo si cada elemento de la diagonal es distinto de cero.\\\\


    %-------------------- section 1.6 10
    \item \textbf{\boldmath Demostrar la siguiente generalización del ejercicio 6. Si $A$ es una matriz $m\times n$, $B$ es una matriz $n\times m$ y $n<m$, entonces $AB$ no es inversible.\\\\
	Demostración.-}\; Sea $A'$ una matriz $m\times m$ obtenido al unir $m-n$ columnas cero para $A$. Es decir, $A'=\left[A_{m\times n}|0_{m\times (m-n)}\right]_{m\times m}$ y sea $B'$ una matriz $m\times m$ obtenida al unir $m-n$ cero filas de $B$, en otras palabras
	$$B'=\begin{bmatrix*}[c]
	    B_{n\times m}\\
	    ----\\
	    0_{(m-n)\times m}
	\end{bmatrix*}$$
	Ahora,
	$$A'B'=\left[A_{m\times n}|0_{m\times (m-n)}\right]
	\begin{bmatrix*}[c]
	    B_{n\times m}\\
	    ----\\
	    0_{(m-n)\times m}
	\end{bmatrix*} =AB.$$
	Utilizando la multiplicación de matrices. Así, producto $AB$ y $A'B'$ son idénticos. \\
	Por construcción, $|A'|=0$ y $|B'|=0.$ Por lo tanto, $|A'B'|=|A'||B'|=0$. Así, $|AB|=0$ y por ende $AB$ es no invertible.\\\\


    %-------------------- section 1.6 12
    \item \textbf{\boldmath El resultado del ejemplo 16 sugiere que tal vez la matriz
	$$A=\begin{bmatrix*}[c]
	    1&\dfrac{1}{2}&\cdots&\dfrac{1}{n}\\\\
	    \dfrac{1}{2}&\dfrac{1}{3}&\cdots&\dfrac{1}{n+1}\\\\
	    \vdots&\vdots&&\vdots\\\\
		  &\dfrac{1}{n}&\cdots&\dfrac{1}{2n-1}\\\\
	\end{bmatrix*}$$
    es inversible y $A^{-1}$ tiene elementos enteros.¿Se puede demostrar esto?.\\\\
	Respuesta.-}\; Consideremos la matriz $I_n$ dada por:
	$$I_n=\begin{bmatrix*}[r]
	    1&0&\cdots&0\\\\
	    0&1&\cdots&0\\\\
	    \vdots&\vdots&&\vdots\\\\
	    0&0&\cdots&1\\\\
	\end{bmatrix*}$$
	Vemos que la matriz $A$ es una matriz Hilbert, el cual tiene los elementos:
	$$A_{ij}=\dfrac{1}{i+j-1}.$$
	De donde se puede demostrar que su determinante es:
	$$\det A = \dfrac{\left(\displaystyle\prod_{i=1}^{2n-1}i!\right)^4}{\displaystyle \prod_{i=1}^{n-1}i!}$$
	Como el $\det A \neq 0$, la matriz es invertible. Su inversa estará dada por:
	$$\left(A^{-1}\right)_{ij}=\left(-1\right)^{i+j}\left(i+j-1\right){n+i-1\choose n-j}{n+j-1 \choose n-i}{i+j-2 \choose i-1}^2.$$


\end{enumerate}
