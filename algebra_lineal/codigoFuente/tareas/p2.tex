
\begin{enumerate}[\bfseries \mbox{Ejercicio} 1.]

    %-------------------- section 1.5 3
    \item \textbf{\boldmath Encontrar dos matrices $2\times 2,A$ diferentes tales que $A^2=0$ pero $A\neq 0.$\\\\
	Respuesta.-}\;

    %-------------------- section 1.5 4
    \item \textbf{\boldmath Para cada $A$ del ejercicio 2, hallar matrices elementales $E_1,E_2,\ldots,E_k$ tal que 
	$$E_k\cdot E_2E_1A=1.$$\\
    Respuesta.-}\;

    %-------------------- section 1.5 7
    \item \textbf{\boldmath Sean $A$ y $B$ matrices $2\times 2$ tales que $AB=I$. Demostrar que $BA=I.$\\\\
	Demostración.-}\;

    %-------------------- section 1.6 1
    \item \textbf{\boldmath Sea,
	$$A=\begin{bmatrix*}[r]
	    1 & 2 & 1 & 0\\
	    -1 & 0 & 3 & 5\\
	    1 & -2 & 1 & 1
    \end{bmatrix*}$$
    Hallar una matriz escalón reducida por filas $R$ que sea equivalente a $A$, y una matriz inversible $3\times 3$, $P$ tal que $R=PA.$\\\\
	Respuesta.-}\;

    %-------------------- section 1.6 3
    \item \textbf{\boldmath Repetir el ejercicio 1, pero con 
	$$A=\begin{bmatrix*}[r]
	    2 & 0 & i\\
	    1 & -3 & -i\\
	    i & 1 & 1
	\end{bmatrix*}$$\\
	Respuesta.-}\;

    %-------------------- section 1.6 4
    \item \textbf{\boldmath Sea 
	$$A=\begin{bmatrix*}[r]
	    5&0&0\\
	    1&5&0\\
	    0&1&5
	\end{bmatrix*}$$
	¿Para qué $X$ existe un escalar $c$ tal que $AX=cX$?.\\\\
	Respuesta.-}\;

    %-------------------- section 1.6 5
    \item \textbf{\boldmath Determinar si
	$$A=\begin{bmatrix*}[r]
	    1&2&3&4\\
	    0&2&3&4\\
	    0&0&3&4\\
	    0&0&0&4
	\end{bmatrix*}$$
	es inversible y hallar $A^{-1}$ si existe.\\\\
	Respuesta.-}\;

    %-------------------- section 1.6 6
    \item \textbf{\boldmath Supóngase que $A$ es una matriz $2\times 1$ y que $B$ es una matriz $1\times 2$. Demostrar que $C=AB$ no es inversible.\\\\
	Demostración.-}\;

    %-------------------- section 1.6 9
    \item \textbf{\boldmath Una matriz $n\times n$, $A$, se llama triangular superior si $A_{ij}=0$ para $i>j$; esto es, si todo elemento por debajo de la diagonal principal es $0$. Demostrar que una matriz (cuadrada) triangular superior es inversible si, y sólo si, cada elemento de su diagonal principal es diferente de $0$.\\\\
	Demostración.-}\;

    %-------------------- section 1.6 10
    \item \textbf{\boldmath Demostrar la siguiente generalización del ejercicio 6. Si $A$ es una matriz $m\times n$, $B$ es una matriz $n\times m$ y $n<m$, entonces $AB$ no es inversible.\\\\
	Demostración.-}\;

    %-------------------- section 1.6 12
    \item \textbf{\boldmath El resultado del ejemplo 16 sugiere que tal vez la matriz
	$$A=\begin{bmatrix*}[c]
	    1&\dfrac{1}{2}&\cdots&\dfrac{1}{n}\\\\
	    \dfrac{1}{2}&\dfrac{1}{3}&\cdots&\dfrac{1}{n+1}\\\\
	    \vdots&\vdots&&\vdots\\\\
		  &\dfrac{1}{n}&\cdots&\dfrac{1}{2n-1}\\\\
	\end{bmatrix*}$$
    es inversible y $A^{-1}$ tiene elementos enteros.¿Se puede demostrar esto?.\\\\
	Respuesta.-}\;


\end{enumerate}
