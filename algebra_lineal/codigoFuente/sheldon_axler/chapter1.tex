\chapter{Espacios Vectoriales}

\section{\boldmath $R^n$ y $C^n$}

\begin{tcolorbox}[colback=white]
    \begin{def.}[Números complejos] \hfill
	\begin{itemize}
	    \item Un número complejo es un par ordenado $(a,b)$, donde $a,b\in \mathbb{R}$, pero lo escribimos como $a+bi$.
	    \item El conjunto de todos los números complejos es denotado por $\mathbb{C}$:
		$$\mathbb{C}=\lbrace a+bi:a,b\in \mathbb{R}\rbrace.$$

	    \item La adición y la multiplicación en $\mathbb{C}$ esta definida por:
		$$(a+bi)+(c+di)=(a+c)+(b+d)i$$
		$$(a+bi)(c+di)=(ac-bd)+(ad+bc)i$$
	\end{itemize}
    \end{def.}
\end{tcolorbox}

\begin{tcolorbox}[title={Propiedades de la aritmética compleja}, colback=white]
	\textbf{Conmutatividad}
	$$\alpha + \beta = \beta + \alpha\quad \mbox{y}\quad \alpha \beta = \beta \alpha \quad \mbox{para todo }\; \alpha,\beta \in \mathbb{C};$$

	\textbf{Asociatividad}
	$$(\alpha + \beta )+\lambda = \alpha + (\beta + \lambda) \quad \mbox{para todo}\; \alpha,\beta,\lambda \in \mathbb{C};$$

	\textbf{Inverso aditivo}
	\begin{center}
	    Para cada $\alpha \in \mathbb{C}$, existe un único $\beta \in \mathbb{C}$ tal que $\alpha + \beta = 0;$
	\end{center}

	\textbf{Inverso multiplicativo}
	\begin{center}
	    Para cada $\alpha \in \mathbb{C}$ con $\alpha \neq 0$, existe un único $\beta \in \mathbb{C}$ tal que $\alpha \beta = 1;$
	\end{center}

	\textbf{Propiedad distributiva}
	$$\lambda (\alpha + \beta) = \lambda \alpha + \lambda \beta \quad \mbox{para todo}\; \lambda, \alpha, \beta \in \mathbb{C}$$
\end{tcolorbox}

\begin{teo}[Conmutatividad]
    Muestre que $\alpha \beta = \beta \alpha$ para todo $\alpha,\beta \in \mathbb{C}$.\\\\
    Demostración.-\; Supóngase $\alpha = a+bi$ y $\beta = c+di$, donde $a,b,c,d \in \mathbb{R}$. Entonces la definición de multiplicación de números complejos muestra que 
    $$\alpha \beta = (a+bi)(c+di) = (ac-bd)+(ad+bc)i$$
    y
    $$\beta \alpha = (c+di)(a+bi)=(ca-db)+(cb+da)i$$
    Por lo tanto $\alpha \beta = \beta \alpha$.\\\\
\end{teo}


\begin{tcolorbox}[colback=white]
    \begin{def.}[$-\alpha$, sustracción, $1/\alpha$] 
	Sea $\alpha, \beta \in \mathbb{C}$
	\begin{itemize}
	    \item Sea $-\alpha$ que denota el inverso aditivo de $\alpha$. Por lo tanto $-\alpha$ es el único número complejo tal que 
		$$\alpha + (-\alpha) = 0.$$

	    \item \textbf{Sustracción} en $\mathbb{C}$ es definido por:
		$$\beta - \alpha = \beta + (-\alpha).$$

	    \item Para $\alpha\neq 0$, sea $1/\alpha$ denotado por el inverso multiplicativo de $\alpha$. Por lo tanto $1/\alpha$ es el único número complejo tal que
		$$\alpha (1/\alpha)=1.$$

	    \item \textbf{División} en $\mathbb{C}$ es definido por:
		$$\beta/\alpha = \beta(1/\alpha).$$
	\end{itemize}
    \end{def.}
\end{tcolorbox}

\subsection{Listas}

\begin{tcolorbox}[colback=white]
    \begin{def.}[Listas, longitud] 
	Supóngase que $n$ es un entero no negativo. Una lista de longitud $n$ es una colección ordenada de $n$ elementos (el cual podría ser números, otras listas, o mas entidades abstractas) separadas por comas y cerradas por paréntesis. Una lista de longitud $n$ se muestra de la siguiente manera:
	$$(x_1,\ldots,x_n)$$
	Dos listas son iguales si y sólo si tienen la misma longitud y los mismos elementos en el mismo orden.
    \end{def.}
\end{tcolorbox}
Las listas difieren de los conjuntos de dos maneras: en las listas, el orden importa y las repeticiones tienen significado; en conjuntos, el orden y las repeticiones son irrelevantes.\\\\

\subsection{\boldmath $F^n$}

\begin{tcolorbox}[colback=white]
    \begin{def.} 
	$\mathbb{F}^n$ es el conjunto de todas las listas de longitud $n$ de elementos de $\mathbb{F}$
	$$\mathbb{F}^n = \lbrace (x_1,\ldots, x_n)\; : \; x_j \in \mathbb{F} \; \mbox{para cada}\; j = 1,\ldots, n\rbrace.$$
	Para $(x_1,\ldots, x_n)\in \mathbb{F}$ y $j\in \lbrace 1, \ldots, n\rbrace,$ decimos que $x_j$ es la j-enesima coordenada de $(x_1,\ldots, x_n)$.
    \end{def.}
\end{tcolorbox}

\begin{tcolorbox}[colback=white]
    \begin{def.}[\boldmath Adición en $\mathbb{F}^n$]
	La adición en $F^n$ es definido añadiendo las correspondientes coordenadas: 
	$$(x_1,\ldots,x_n)+(y_1,\ldots, y_n)=(x_1+y_1,\ldots, x_n + y_n)$$
    \end{def.}
\end{tcolorbox}

\begin{teo}
    Si $x,y \in F^n,$ entonces $x+y=y+x$.\\\\
    Demostración.-\; $x=(x_1,\ldots, x_n)$ y $y=(y_1,\ldots, y_n)$. Entonces
    $$\begin{array}{rcl}
	x+y&=&(x_1,\ldots,x_n)+(y_1,\ldots,y_n)\\
	   &=&(x_1+y_1,\ldots, x_n + y_n)\\
	   &=&(y_1+x_1,\ldots, y_n x_n)\\
	   &=&(y_1,\ldots,y_n)+(x_1,\ldots,x_n)\\
	   &=&y+x\\
    \end{array}$$
\end{teo}

\begin{tcolorbox}[colback=white]
    \begin{def.}[$0$]
	Sea $0$ la lista de longitud n cuyas coordenadas son todas 0: 
	$$0=(0,\ldots,0)$$
    \end{def.}
\end{tcolorbox}

\begin{tcolorbox}[colback=white]
    \begin{def.}[\boldmath Inverso aditivo en $\mathbb{F}^n$]
	Para $x\in \mathbb{F}^n$, el inverso aditivo de $x$, denota por $-x$, es el vector $-x\in \mathbb{F}^n$ tal que $$x+(-x) =0$$
	En otras palabras, si $x=(x_1,\ldots,x_n)$, entonces $-x=(-x_1,\ldots,-x_n).$
    \end{def.}
\end{tcolorbox}


\begin{tcolorbox}[colback=white]
    \begin{def.}[Multiplicación scalar en $\mathbb{F}^n$]
	El producto de un número $\lambda$ y un vector en $\mathbb{F}^n$ es calculado por la multiplicación de cada coordenada del vector por $\lambda$:
	$$\lambda(x_1,\ldots,x_n) = (\lambda x_1,\ldots, \lambda x_n)$$
	donde $\lambda \in \mathbb{F}$ y $(x_1,\ldots,x_n)\in \mathbb{F}^n$.
    \end{def.}
\end{tcolorbox}

\subsection{Ejercicios 1.A}


\section{Definición de espacio vectorial}
La motivación para la definición de un espacio vectorial proviene de las propiedades de la suma y la multiplicación escalar en $\mathbb{F}^n$: la suma es conmutativa, asociativa y tiene una identidad. Todo elemento tiene un inverso aditivo. La multiplicación escalar es asociativa. La multiplicación escalar por 1 actúa como se esperaba. La suma y la multiplicación escalar están conectadas por propiedades distributivas. Definiremos un espacio vectorial como un conjunto V con una suma y una multiplicación escalar en V que satisfagan las propiedades del párrafo anterior.

\begin{tcolorbox}[colback=white]
    \begin{def.}[Adición y multiplicación escalar]\hfill
	\begin{itemize}
	    \item Una adición en un conjunto $V$ es una función que asigna un elemento $u+v\in V$ para cada par de elementos $u,v\in V$.
	    \item Una multiplicación escalar en un conjunto $V$ es una función que asigna un elemento $\lambda  v\in V$ para cada $\lambda \in \mathbb{F}$ y cada $v\in V$.
	\end{itemize}
    \end{def.}
\end{tcolorbox}

\begin{tcolorbox}[colback=white]
    \begin{def.}[Espacio vectorial] Un espacio vectorial es un conjunto $V$ junto con una suma en $V$ y una multiplicación escalar en $V$ tal que se cumplen las siguientes propiedades: 
	\begin{itemize}
	    \item \textbf{Conmutatividad}
		$$u+v=v+u\; \mbox{para todo}\; u,v\in V;$$
	    \item \textbf{Asociatividad}
		$$(u+v)+w=u+(v+w)\; \mbox{y}\; (ab)v=a(bv)\; \mbox{para todo}\; u,v,w \in V\; \mbox{y todo}\; a,b\in \mathbb{F};$$
	    \item \textbf{Identidad aditiva}
		\begin{center}
		    Existe un elemento $0\in V$ tal que $v+0=v$ para todo $v\in V$;
		\end{center}

	    \item \textbf{Inverso aditivo}
		\begin{center}
		    Para cada $v\in V$, existe $w \in V$ tal que $v+w=0$;
		\end{center}

	    \item \textbf{Identidad Multiplicativa}
		$$1v=v\; \mbox{para todo}\; v\in V$$

	    \item \textbf{Propiedad distributiva}
		\begin{center}
		    $a(u+v)=au+av$ y $(a+b)v=av+bv$ para todo $a,b\in \mathbb{F}$ y todo $u,v \in V$.
		\end{center}
	\end{itemize}
    \end{def.}
\end{tcolorbox}

\begin{tcolorbox}[colback=white]
    \begin{def.}[Vector, punto]
	Elementos de un espacio vectorial son llamados vectores o puntos.
    \end{def.}
\end{tcolorbox}

\begin{tcolorbox}[colback=white]
    \begin{def.}[Espacio vectorial real, espacio vectorial complejo]\hfill
	\begin{itemize}
	    \item Un espacio vectorial sobre $\mathbb{R}$ es llamado un espacio vectorial real.
	    \item Un espacio vectorial sobre $\mathbb{C}$ es llamado un espacio vectorial complejo.
	\end{itemize}
    \end{def.}
\end{tcolorbox}

\begin{tcolorbox}[colback=white]
    \begin{def.}[\boldmath $F^S$]\hfill
	\begin{itemize}
	    \item Si $S$ es un conjunto, entonces $F^S$ se denota como el conjunto de funciones de $S$ para $F$.
	    \item Para $f,g\in F^S$, la suma $f+g\in F_S$ es la función definida por $$(f+g)(x)=f(x)+g(x)$$ para todo $x\in S$
	    \item Para $\lambda \in F$ y $f\in F^S$, el producto $\lambda f \in F^S$ es una función definida por $$(\lambda f)(x)=\lambda f(x)$$
		para todo $x\in S$.
	\end{itemize}
    \end{def.}
\end{tcolorbox}

La definición de un espacio vectorial requiere que tenga una identidad aditiva. El siguiente resultado establece que esta identidad es única.

\begin{teo}
    Un espacio vectorial tiene una única identidad aditiva.\\\\
	Demostración.-\; Supóngase $0$ y $0^{'}$ ambos identidades aditivas para algún espacio vectorial $V$ entonces,
	$$0^{'}=0^{'}+0=0+0^{'}=0$$
	donde se cumple la primera igualdad porque $0$ es una identidad aditiva, la segunda igualdad viene de la conmutatividad, y la tercera igualdad se cumple porque $0^{'}$ es una identidad aditiva. Por lo tanto $0^{'}=0$ y queda probado que $V$ tiene una sola identidad aditiva.\\
\end{teo}

Cada elemento $v$ en un espacio vectorial tiene un inverso aditivo, un elemento $w$ en el espacio vectorial tal que $v + w = 0$. El siguiente resultado muestra que cada elemento en un espacio vectorial tiene solo un inverso aditivo. \\

\begin{teo}
    Cada elemento en un espacio vectorial tiene un único inverso aditivo.\\\\
	Demostración.-\; Supóngase que $V$ es un espacio vectorial. Sea $v\in V$,  $w$ y $w^{'}$ inversos aditivos de $v$. Entonces
	$$w=w+0=w+(v+w^{'}) = (w+v)+w^{'}=0+w^{'}=w^{'}$$
	Así $w=w^{'}$.\\
\end{teo}

\begin{tcolorbox}[colback=white]
    \begin{nota}[$-v,w-v$]
	Sea $v,w \in V$. Entonces
	\begin{itemize}
	    \item $-v$ se denota como el inverso aditivo de $v$;
	    \item $w-v$ es definido como $w+(-v)$.
	\end{itemize}
    \end{nota}
\end{tcolorbox}

\begin{tcolorbox}[colback=white]
    \begin{nota}[$V$]
	Por el resto del libro, $V$ se define como el espacio vectorial sobre $F$.
    \end{nota}
\end{tcolorbox}

\begin{teo}
    $0v=0$ para cada $v\in V$.\\\\
	Demostración.-\; Para $v\in V$, tenemos
	$$0v=(0+0)v=0v+0v$$
	Luego añadiendo el inverso aditivo de $0v$ para ambos lados de la ecuación de arriba tenemos $0=0v$.\\

\end{teo}

Ahora  establecemos que el producto de cualquier escalar y el vector $0$ es igual al vector $0$.\\

\begin{teo}
    $a0=0$ para cada $a\in \mathbb{F}$.\\\\
	Demostración.-\; Para $a\in \mathbb{F}$, tenemos 
	$$a0=a(0+0)=a0+a0$$
	Luego añadiendo el inverso aditivo de $a0$ para ambos lados de la ecuación de arriba tenemos $0=a0$.\\
\end{teo}

Ahora mostramos que si un elemento de $V$ se multiplica por el escalar $-1$, entonces el resultado es el inverso aditivo del elemento de $V$.\\
\begin{teo}
    $(-1)v=-v$ para cada $v\in V$.\\\\
    	Demostración.-\; Para $v\in V$, tenemos
	$$v+(-1)v=1v+(-1)v=[1+(-1)]v=0v=0.$$
	Esta ecuación nos dice que $(-1)v$, cuando se suma a $v$ da $0$. Así $(-1)v$ es el inverso aditivo de $v$.\\
\end{teo}

\subsection{Ejercicio 1.B}

\section{Subespacios}

\begin{tcolorbox}[colback=white]
    \begin{def.}[Subespacio]
	Un subconjunto $U$ de $V$ es llamado un subespacio de $V$ si $U$ es también un espacio vectorial (Usando la misma adición y multiplicación escalar como en $V$).\\\\
    \end{def.}
\end{tcolorbox}
\vspace{.5cm}

El siguiente resultado brinda la forma más fácil de verificar si un subconjunto de un espacio vectorial es un subespacio.\\

\begin{teo}

    Un subconjunto $U$ de $V$ es un subespacio de $V$ si y sólo si $U$ satisface las siguientes tres condiciones:
    \begin{itemize}
	\item \textbf{Identidad aditiva.} $$0\in U;$$
	\item \textbf{Cerrado bajo adición.}$$u,w\in U\; \mbox{implica}\; u+w\in U;$$
	\item \textbf{Cerrado bajo multiplicación escalar.}
	    $$a\in F \; \mbox{y}\; u\in U\; \mbox{implica}\; au\in U;$$
    \end{itemize}

	Demostración.-\; Si $U$ es un subespacio de $V$, entonces $U$ satisface las tres condiciones de arriba por la definición de espacio vectorial.\\
	Por el contrario, suponga que $U$ satisface las tres condiciones anteriores. La primera condición anterior asegura que la identidad aditiva de $V$ está en $U$. La segunda condición de arriba asegura que la adición tenga sentido en $U$. La tercera condición asegura que la multiplicación escalar tenga sentido en $U$.\\
	Si $u\in U$, entonces $-u$ es también en $U$ por la tercera condición de arriba. Por lo tanto cada elemento de $U$ tiene un inverso aditivo en $U$.\\
	Las otras partes de la definición de un espacio vectorial,como la asociatividad y conmutatividad, se satisfacen automáticamente para $U$ porque sostienen el espacio más grande de $V$. Así $U$ es un espacio vectorial y por ende es un subespacio de $V$.\\

\end{teo}

\subsection{Suma de subespacios}

\begin{tcolorbox}[colback=white]
    \begin{def.}[Suma de subespacios]
	Supóngase $U_1,\ldots ,U_m$ son subconjuntos de $V$. La suma de $U_1,\ldots, U_m$, denotado por $U_1 + \ldots+U_m$, es el conjunto de todas los posibles  sumas de los elementos de $U_1,\ldots,U_m$. Más precisamente,
	$$U_1+\ldots+U_m = \lbrace u_1+\ldots +u_m\; :\; u_1\in U_1,\ldots, u_m \in U_m \rbrace$$
    \end{def.}
\end{tcolorbox}
\vspace{.5cm}

El siguiente resultado establece que la suma de subespacios es un subespacio y, de hecho, es el subespacio más pequeño que contiene todos los sumandos.

\begin{teo}[Suma de subespacios es el contenedor más pequeño de subespacios] \hfill

    Supóngase que $U_1,\ldots,U_m$ son subespacios de $V$. Entonces $U_1+\ldots +U_m$ es el subepsacio más pequeño de $V$ que contiene $U_1,\ldots, U_m$.\\\\
	Demostración.-\; Es fácil ver que $0\in U_1+\ldots + U_m$ y que $U_1+\ldots +U_m$ es cerrado sobre la adición y la multiplicación escalar, es decir, por las tres condiciones dadas anteriormente podemos afirmar que $U_1+\ldots + U_m$ es un subespacio de $V$.\\
	Claramente $U_1,\ldots, U_m$ están todos contenidos en $U_1 + \ldots + U_m$ (para ver esto, considere las sumas $u_1+\ldots + u_m$ donde todas menos una de las $u^{'}s$ son $0$). Por el contrario, todo subespacio de $V$ que contenga $U_1,\ldots , U_m$ contiene $U_1 + \ldots + U_m$ (porque los subespacios deben contener todas las sumas finitas de sus elementos). Por lo tanto $U_1+\ldots + U_m$ es el subespacio más pequeño de $V$ que contiene a $U_1,\ldots , U_m$.
\end{teo}

\subsection{Sumas directas}

\begin{tcolorbox}[colback=white]
    \begin{def.}[Suma directa]
	Supóngase que $U_1,\ldots, U_m$ son subespacios de $V$.
	\begin{itemize}
	    \item La suma $U_1+\ldots + U_m$ es llamada una suma directa si cada elemento de $U_1+\ldots + U_m$ se puede escribir de una sola manera como una suma  $u_1 + \ldots + u_m$ donde cada $u_j$ esta en $U_j$.
	    \item Si $U_1 + \ldots + U_m$ es una suma directa, entonces $U_1 \oplus \ldots \oplus U_m$ denota $U_1 + \ldots + U_m$ con la notación $\oplus$ sirviendo como una indicación de que se trata de una suma directa.
	\end{itemize}
    \end{def.}
\end{tcolorbox}
\vspace{.5cm}

El siguiente resultado muestra que al decidir si una suma de subespacios es una suma directa, solo necesitamos considerar si 0 se puede escribir únicamente como una suma apropiada.\\

\begin{teo}[Condición para una suma directa]
    Supóngase que $U_1,\ldots, U_m$ son subespacios de $V$. Entonces $U_1+\ldots + U_m$ es una suma directa si y sólo si la única manera de escribir $0$ como una suma $u_1 + \ldots + u_m$, donde cada $u_j$ está en $U_j$, es tomando cada $u_j$ igual a $0$.\\\\
	Demostración.-\; Supóngase que $U_1+\ldots + U_m$ es una suma directa. Entonces la definición de suma directa implica que la única manera de escribir $0$ como una suma $u_1 + \ldots + u_m$, donde cada $u_j$ está en $U_j$, es tomando cada $u_j$ igual a $0$. Para mostrar que $U_1 + \ldots + U_m$ es una suma directa, sea $v \in U_1 + \ldots + U_m$. Podemos escribir:
	$$v = u_1 + \ldots + u_m$$
	para algún $u_1 \in U_1 , \ldots, u_m \in U_m$. Para mostrar que esta representación es única, supongamos que también tenemos
	$$v = v_1 + \ldots + v_m$$
	donde $v_1 \in U_1,\ldots , v_m \in U_m$. Restando estas dos ecuaciones, tenemos,
	$$0=(u_1-v_1)+\ldots + (u_m - v_m).$$
	Porque $u_1-v_1\in U_1,\ldots U_m$, la ecuación dada implica que cada $u_j - v_j$ es igual a $0$. Por lo tanto $u_1 = v_1, \ldots , u_m = v_m$.\\

\end{teo}

El siguiente resultado da una condición simple para probar qué pares de subespacios dan una suma directa.\\

\begin{teo}[Suma directa de dos subespacios]
    Supóngase que $U$ y $W$ son subespacios de $V$. Entonces $U+W$ es una suma directa si y sólo si $U \cap W = \lbrace 0 \rbrace$.\\\\
    Demostración.-\; Primero supóngase que $U+W$ es una suma directa. Si $v \in U\cap W,$ entonces $0=v+(-v),$ donde $v\in U$ y $-v\in W$. Por la única representación de $0$ como la suma de un vector en $U$ y un vector en $W$, tenemos $v=0$. 
\end{teo}




