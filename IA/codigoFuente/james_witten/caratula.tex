\begin{titlingpage}

\newcommand\nbvspace[1][3]{\vspace*{\stretch{#1}}}
% allow some slack to avoid under/overfull boxes
\newcommand\nbstretchyspace{\spaceskip0.5em plus 0.25em minus 0.25em}
% To improve spacing on titlepages
\newcommand{\nbtitlestretch}{\spaceskip0.6em}
\pagestyle{empty}

\begin{center}
\bfseries
\nbvspace[1]

\Large Gereth James, Daniela Witten, Trevor Hastie, Robert Tibshirani and Jonathan Taylor\\
\Huge
{\nbtitlestretch\Huge Una introducción al Aprendizaje estadístico}\\
\vspace{.5cm}
\large
con applicaciones en Python\\
\nbvspace[1]

APUNTES\\

\nbvspace[1]
\small POR\\
\Large FODE\\[0.5em]
\footnotesize CHRISTIAN LIMBERT PAREDES AGUILERA\\

\nbvspace[2]

\begin{center}
    \begin{tikzpicture}[scale=.55]
	\begin{polaraxis}[
	    xticklabels={,0,$\frac\pi6$,$\frac\pi3$,$\frac\pi2$,$\frac{2\pi}3$,$\frac{5\pi}6$,
	    $\pi$,$\frac{7\pi}6$,$\frac{4\pi}3$,$\frac{3\pi}2$,$\frac{5\pi}3$,$\frac{11\pi}6$}
	    ]
	    \addplot[mark = none, domain = 0:2*pi, data cs = polarrad, samples = 600, fill = gray!30,opacity=.5]{x};
	\end{polaraxis}
    \end{tikzpicture}
\end{center}

\nbvspace[3]
\normalsize

PRIMERA IMPRESIÓN\\
\large
\nbvspace[1]

\end{center}

\break
\bfseries 

\nbvspace[1]
Título de la obra original:\\
An Introduction to Statistical Learning\\

\nbvspace[1]

\begin{center}
Sin ninguna revisión de esta obra.\\


\nbvspace[1]
    Propiedad de esta obra:\\ 

    CHRISTIAN LIMBERT PAREDES AGUILERA\\	

    E-mail: soyfode@gmail.com
\end{center}

\nbvspace[1]

Reservados todos los derechos. La reproducción total o parcial de esta obra, por cualquier medio o procedimiento, comprendidos la reprografía y el tratamiento informático, y la distribución de ejemplares de ella mediante alquiler o préstamo públicos, queda rigurosamente prohibida sin la autorización escrita de los titulares del copyright, bajo las sanciones establecidas por las leyes.\\

\center 2023 

\end{titlingpage}


\pagenumbering{roman}

\tableofcontents								%indice

\pagestyle{fancy}
\fancyhead[LE,RO]{\nouppercase{\truncate{0.5\headwidth}{\rightmark}}}
\fancyhead[LO,RE]{\nouppercase{\truncate{0.5\headwidth}{\leftmark}}}


