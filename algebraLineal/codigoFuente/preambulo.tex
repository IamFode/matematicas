\documentclass[10pt]{book}

% ------------------- Paquetes ----------------------
\usepackage[text=17cm,
	    left=3.4cm,
	    right=2.5cm, 
	    headsep=20pt, 
	    top=2.5cm, 
	    bottom = 2cm,
	    onecolumn,
	    twoside,
	    letterpaper,
	    asymmetric,
	    showframe = false]{geometry} %configuración página
\usepackage{latexsym,xparse,amssymb,amsfonts,amsthm,amsmath} %(símbolos de la AMS)
\parindent = 0cm  %sangria
\usepackage[T1]{fontenc} %acentos en español
\usepackage{graphicx} %gráficos y figuras.
\usepackage[spanish,english]{babel}
\usepackage{mathpazo} % tipo de letra
\usepackage{titlesec} %formato de títulos
\usepackage[backref=page]{hyperref} %hipervinculos
\usepackage{multicol} %columnas
\usepackage{tcolorbox} %cajas
\usepackage{enumerate} %indice enumerado
\usepackage{marginnote}%notas en el margen
\tcbuselibrary{skins,breakable,listings,theorems}
\usepackage[Bjornstrup]{fncychap}%diseño de portada de capitulos
\usepackage[all]{xy}%flechas
\counterwithout{footnote}{chapter}
\usepackage{xcolor}
\spanishdecimal{.}
\usepackage{pgfplots}
\usepgfplotslibrary{polar}
\usepackage{tkz-fct}
\usepackage{mathrsfs}
\usepackage[htt]{hyphenat}
\usepackage[fit]{truncate}
\usepackage{titling,lipsum}
\usepackage{thmtools}

%--------- encabezado y pie de paginas -----------------
\usepackage{fancyhdr}

%--------- configuración tcolorbox -----------------
\tcbset{colback=black!2,colframe=white}

%---------- configuración de separación de sección ------------ 
\titlespacing*{\section}{0pt}{1.7cm}{.3cm}

%------------------------------------------

\declaretheoremstyle[%
    spaceabove=0pt,spacebelow=20pt,%
    headfont=\small\bfseries,% 
    notefont=\bfseries,%
    notebraces={}{. },%
    headpunct={},%
    preheadhook={\hspace{0mm}\newline},%
    postheadhook={},%
    shaded = {
	textwidth =450pt,
	bgcolor = black!2,
	rulecolor=black!4,
	rulewidth=2pt,
	margin=3pt
    },
    postheadspace=0pt,%
    headindent=0pt,%
    bodyfont=\normalfont,%
    headformat={\llap{\smash{\parbox[t]{1.3in}{\raggedleft \NAME \;\;\; \\ \NUMBER\;\;\;}}}\NOTE}%
]{margindef}
%hack to kill some extra space

\makeatletter
\renewcommand\thmt@space{}
\makeatother

\declaretheorem[style=margindef, numberwithin=part, title=Axioma]{axioma}
\declaretheorem[style=margindef, numberwithin=chapter, title=Definición]{def.}
\declaretheorem[style=margindef, numberwithin=chapter, title=Obs]{obs}

\declaretheoremstyle[%
    spaceabove=0pt,spacebelow=10pt,%
    headfont=\small\bfseries,% 
    notefont=\bfseries,%
    notebraces={}{. },%
    headpunct={},%
    preheadhook={\hspace{0mm}\newline},%
    postheadhook={},%
    qed=$\blacksquare$,
    postheadspace=0pt,%
    headindent=0pt,%
    bodyfont=\normalfont,%
    headformat={\llap{\smash{\parbox[t]{1.3in}{\raggedleft \NAME \;\;\; \\ \NUMBER\;\;\;}}}\NOTE}%
]{marginheads}
\makeatletter
\renewcommand\thmt@space{}
\makeatother

\declaretheorem[style=marginheads, numberwithin=chapter, title=Teorema]{teo}
\declaretheorem[style=marginheads, numberwithin=chapter, title=Ejemplo]{ejem}
\declaretheorem[style=marginheads, numberwithin=chapter, title=Postulado]{post}
\declaretheorem[style=marginheads, numberwithin=chapter, title=Corolario]{cor}
\declaretheorem[style=marginheads, numberwithin=chapter, title=Ejercicio]{ej}
\declaretheorem[style=marginheads, numberwithin=part, title=Propiedad]{prop}
\declaretheorem[style=marginheads, numberwithin=chapter, title=Lema]{lema}
\declaretheorem[style=marginheads, numberwithin=chapter, title=Problema]{prob}
\declaretheorem[style=marginheads, numberwithin=chapter, title=Nota]{nota}

%-------------------------------------------------------------------------

\declaretheoremstyle[%
    spaceabove=0pt,spacebelow=20pt,%
    headfont=\small\bfseries,% 
    notefont=\bfseries,%
    notebraces={}{. },%
    headpunct={},%
    preheadhook={\hspace{0mm}\newline},%
    postheadhook={},%
    shaded = {
	textwidth =450pt,
	bgcolor = black!2,
	rulecolor=black!4,
	rulewidth=2pt,
	margin=3pt
    },
    postheadspace=0pt,%
    headindent=0pt,%
    bodyfont=\normalfont,%
    headformat={\llap{\smash{\parbox[t]{1.3in}{\raggedleft \NUMBER\; \NAME\;\;\;}}}\NOTE}%
]{mydef}
%hack to kill some extra space

\makeatletter
\renewcommand\thmt@space{}
\makeatother

\declaretheorem[style=mydef, numberwithin=chapter,title=Definición]{mydef}
\declaretheorem[style=mydef, sibling=mydef,title=Notación]{mynot}

\declaretheoremstyle[%
    spaceabove=0pt,spacebelow=10pt,%
    headfont=\small\bfseries,% 
    notefont=\bfseries,%
    notebraces={}{. },%
    headpunct={},%
    preheadhook={\hspace{0mm}\newline},%
    postheadhook={},%
    qed=$\blacksquare$,
    postheadspace=0pt,%
    headindent=0pt,%
    bodyfont=\normalfont,%
    headformat={\llap{\smash{\parbox[t]{1.3in}{\raggedleft \NUMBER\; \NAME\;\;\;}}}\NOTE}%
]{myteo}
\makeatletter
\renewcommand\thmt@space{}
\makeatother

\declaretheorem[style=myteo, sibling=mydef,title=Teorema]{myteo}
\declaretheorem[style=myteo, sibling=mydef,title=Ejemplo]{myejem}

\newcounter{mysection}
\titleclass{\mysection}{straight}[\chapter]
\titleformat{\mysection}[hang]
  {\normalfont\bfseries}{\themysection}{1em}{}
\titlespacing*{\mysection}{0pt}{3.5ex plus 1ex minus .2ex}{2.3 ex plus .2ex}
\renewcommand{\themysection}{\arabic{chapter}.\Alph{mysection}}
%\counterwithin{section}{mysection}

%-------------------------------------------------------------------

\makeatletter\renewcommand\theenumi{\@roman\c@enumi}\makeatother

\renewcommand\labelenumi{\theenumi)}
\def\sen{\mathop{\mbox{\normalfont sen}}\nolimits}
\def\cotan{\mathop{\mbox{\normalfont cotan}}\nolimits}
\def\cosec{\mathop{\mbox{\normalfont cosec}}\nolimits}
\def\arcsen{\mathop{\mbox{\normalfont arcsen}}\nolimits}
\def\arctan{\mathop{\mbox{\normalfont arctan}}\nolimits}
\def\span{\mathop{\mbox{\normalfont span}}\nolimits}

%----------Formato título de capítulos-------------
%---------------------------------
\titleformat*{\section}{\bfseries}
\titleformat*{\subsection}{\bfseries}
\titleformat*{\subsubsection}{\bfseries}
\titleformat*{\paragraph}{\bfseries}
\titleformat*{\subparagraph}{\bfseries}


\titleformat{\chapter}[display]
{\vspace{4ex}\bfseries\huge}
{\filleft\Huge\thechapter}
{2ex}
{\filleft}

