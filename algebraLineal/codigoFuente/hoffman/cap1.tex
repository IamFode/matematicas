\chapter{Ecuaciones lineales}

\section{Cuerpos}
Se designa por $F$ el conjunto de los números reales o el conjunto de los números complejos.

    \begin{enumerate}[\bfseries 1.]
	%---------- 1.
	\item La adición es conmutativa,
	    $$x+y=y+x$$
	para cualquiera $x$ e $y$ de $F$.

	%---------- 2.
	\item La adición es asociativa,
	    $$x+(y+z)=(x+y)+z$$
	para cualquiera $x,y$ y $z$ de $F$.

	%---------- 3.
	\item Existe un elemento único $0$ (cero) de $F$ tal que $x+0=x$, para todo $x$ en $F$.

	%---------- 4.
	\item A cada $x$ de $F$ corresponde un elemento único $(-x)$ de $F$ tal que $x+(-x)=0$.

	%---------- 5.
	\item La multiplicación es conmutativa,
	    $$xy=yx.$$

	%---------- 6.
	\item La multiplicación es asociativa,
	    $$x(yz)=(xy)z.$$

	%---------- 7.
	\item Existe un elemento no nulo único de $F$ tal que $x1=x$, para todo $x$ en $F$.

	%---------- 8.
	\item A cada elemento no nulo $x$ de $F$ corresponde un único elemento $x^{-1}$ (o ($1/x$)) de $F$ tal que $xx^{-1}=1.$

	%---------- 9.
	\item La multiplicación es distributiva respecto de la adición, esto es, $x(y+z)=xy+xz$, para cualquiera $x,y$ y $z$ de $F$.
	
    \end{enumerate}

% ---------- def. ----------
    El conjunto $F$, junto con  las operaciones de suma y multiplicación, se llama entonces \textbf{cuerpo}.

    Un \textbf{subcuerpo} de un cuerpo $C$ es un conjunto $F$ de números complejos que es a su vez un cuerpo respecto de las operaciones usuales de adición y multiplicación  de números complejos. Esto significa que el $0$ y el $1$ están en el conjunto $F$, y que si $x$ e $y$ son elementos de $F$, también lo son $(x+y), -x,xy-x^{-1}$ si $(x\neq 0)$.


%------------------- ejemplo 1 ----------------------
\begin{ejem}
    El conjunto de los enteros positivos: $1,2,3,\ldots$ no es un subcuerpo de $C$ por varias razones. Por ejemplo, $0$ no es un entero positivo; para ningún entero positivo $n$, es $-n$ un entero positivo; para ningún entero positivo $n$, excepto $1$, es $1/n$ un entero positivo.\\
\end{ejem}

%------------------- ejemplo 2 ----------------------
\begin{ejem}
    El conjunto de los enteros: $\ldots,-2,-1,0,1,2,\ldots$ no es un subcuerpo de $C$, porque para un entero $n$, $1/n$ no es un entero al menos que $n$ sea $1$ o $-1$. Con las operaciones usuales de adición y multiplicación, el conjunto de los enteros satisface todas las condiciones (1)-(9), con excepción de la condición (8).\\
\end{ejem}

%------------------- ejemplo 3 ----------------------
\begin{ejem}
    El conjunto de los números racionales, esto es, números de la forma $p/q$, donde $p$ y $q$ son enteros y $q\neq 0$, es un subcuerpo del cuerpo de los complejos. La división que no es posible en el conjunto de los enteros es posible en el conjunto de los números racionales.\\
\end{ejem}

%------------------- ejemplo 4 ----------------------
\begin{ejem}
    El conjunto de todos los números complejos de la forma $x+y\sqrt{2}$, donde $x$ e $y$ son racionales, es un subcuerpo de $C$.\\\\
	Demostración.-\; La multiplicación y la suma de un número racional y un número irracional siempre da un irracional. Estos números cumplen las condiciones (1)-(9), por lo que podemos concluir que este conjunto es un subcuerpo de $C$.\\
\end{ejem}

\section{Sistema de ecuaciones lineales}
Supóngase que $F$ es un cuerpo. Se considera el problema de encontrar $n$ escalares (elementos de $F$) $x_1,\ldots,x_n$ que satisfagan las condiciones 

\begin{equation}
    \begin{array}{ccc}
	A_{11}x_1+A_{12}x_2+\ldots + A_{1n}x_n&=&y_1\\
	A_{21}x_1+A_{22}x_2+\ldots + A_{2n}x_n&=&y_2\\
	\vdots &&\vdots\\
	A_{m1}x_1+A_{m2}x_2+\ldots + A_{mn}x_n&=&y_m
    \end{array}
\end{equation}

donde $y_1,\ldots , y_{m}$ y $A_{ij}$, $1\leq i \leq m$, $1\leq j \leq n,$ son elementos de $F$. A (1-1) se le llama un \textbf{\boldmath sistema de $m$ ecuaciones lineales con $n$ incógnitas}. Todo n-tuple $(x_1,\ldots,x_n)$ de elementos de $F$ que satisface cada una de las ecuaciones de (1-1) se llama una \textbf{solución} del sistema. Si $y_1=y_2=\ldots = y_m=0$, se dice que el sistema es \textbf{homogéneo}, o que cada una de las ecuaciones es homogénea.\\

Para el sistema general (1-1), supóngase que seleccionamos $m$ escalares $c_1,\ldots, c_m$, que se multiplica la j-ésima ecuación por $c_j$ y que luego se suma. Se obtiene la ecuación
$$(c_1 A_{11}+\ldots +c_m A_{m1})x_1+\ldots+(c_1A_{1n}+\ldots + c_mA_{mn})x_n=c_1y_1+\ldots+c_my_m$$
A tal ecuación se le llama \textbf{combinación lineal} de las ecuaciones (1-1).\\ 

Si se tiene otro sistema de ecuaciones lineales

\begin{equation}
    \begin{array}{ccc}
	B_{11}x_1+\ldots + B_{1n}x_n&=&z_1\\
	\vdots &&\vdots\\
	B_{k1}x_1+\ldots + B_{kn}x_n&=&z_k
    \end{array}
\end{equation}

en que cada una de las $k$ ecuaciones sea combinación lineal de las ecuaciones de (1-1), entonces toda solución de (1-1) es solución de este nuevo sistema.\\ 

Se dirá que dos sistemas de ecuaciones lineales son \textbf{equivalentes} si cada ecuación de cada sistema es combinación lineal de las ecuaciones del otro sistema.\\

\begin{teo}
    Sistemas equivalentes de ecuaciones lineales tiene exactamente las mismas soluciones.
\end{teo}

\section{Ejercicios}

\begin{enumerate}[\bfseries 1.]

    %---------- 1.
    \item Verificar que el conjunto de número complejos descritos en el Ejemplo 4 es un subcuerpo de $\bf{C}$.\\\\
	Demostración.-\; La multiplicación y la suma de un número racional y un número irracional siempre da un irracional. Estos números cumplen las condiciones (1)-(9), por lo que podemos concluir que este conjunto es un subcuerpo de $C$.\\

    %---------- 2.
    \item Sea $F$ el cuerpo de los números complejos. ¿Son equivalentes los dos sistemas de ecuaciones lineales siguientes? Si es así, expresar cada ecuación de cada sistema como combinación lineal de las ecuaciones del otro sistema.

    $$\begin{array}{rcl}
	x_1-x_2&=&0\\
	2x_1+x_2&=&0
    \end{array} \qquad 
    \begin{array}{rcl}
	3x_1+x_2&=&0\\
	x_1+x_2&=&0
    \end{array}$$
    \vspace{.4cm}

	Respuesta.-\; Sí, los sistemas dados son equivalentes ya que cada ecuación en un sistema se puede escribir como una combinación lineal de las ecuaciones del otro sistema.\\

	Sean $c_1=1$ y $c_2=-2$ tal que

	$$\begin{array}{rcl}
	    x_1-x_2&=&(c_1\cdot 3+ c_2)x_1 + (c_1+c_2)x_2\\\\
		   &=&\left[1\cdot 3+(-2)\right]x_1+\left[1+(-2)\right]x_2\\\\
		   &=&x_1-x_2
	\end{array}$$ 

	Sean $c_1=\dfrac{1}{2}$ y $c_2=\dfrac{1}{2}$ tal que 

	$$\begin{array}{rcl}
	    2x_1+x_2&=&(c_1\cdot 3+ c_2)x_1 + (c_1+c_2)x_2\\\\
		    &=&\left[\dfrac{1}{2}\cdot 3+\dfrac{1}{2}\right]x_1+\left[\dfrac{1}{2}+\dfrac{1}{2}\right]x_2\\\\
		    &=&2x_1+x_2
	\end{array}$$ 
	Por lo que podemos decir que la primera ecuación es combinación lineal  de la segunda ecuación. Luego,\\

	Sean $c_1=\dfrac{1}{3}$ y $c_2=\dfrac{4}{3}$ tal que

	$$\begin{array}{rcl}
	    3x_1+x_2&=&\left(c_1+c_2\cdot 2\right)x_1+\left(c_1\cdot(-1)+c_2\right)x_2\\\\
		    &=&\left[\dfrac{1}{3}+\dfrac{4}{3}\cdot 2\right]x_1+\left[\dfrac{1}{3}\cdot(-1)+\dfrac{4}{3}\right]x_2\\\\
		    &=&3x_1+1x_2\\\\
	\end{array}$$

	Sean $c_1=-\dfrac{1}{3}$ y $c_2=\dfrac{2}{3}$ tal que

	$$\begin{array}{rcl}
	    x_1+x_2&=&\left(c_1+c_2\cdot 2\right)x_1+\left(c_1\cdot(-1)+c_2\right)x_2\\\\
		   &=&\left[-\dfrac{1}{3}+\dfrac{2}{3}\cdot 2\right]x_1+\left[-\dfrac{1}{3}\cdot(-1)+\dfrac{2}{3}\right]x_2\\\\
		    &=&1x_1+1x_2\\\\
	\end{array}$$
	Por lo que podemos decir que la segunda ecuación es combinación lineal de la primera ecuación. Así, los sistemas dados son equivalente.\\\\

    %---------- 3.
    \item Examine los siguientes sistemas como en el ejercicio 2.
	$$\begin{array}{rcl}
	    -x_1+x_2+4x_3&=&0\\
	    x_1+3x_2 + 8x_3&=&0\\
	    \frac{1}{2}x_1+x_2+\frac{5}{2}x_3&=&0
	\end{array} \qquad
	\begin{array}{rcl}
	    x_1 \qquad -x_3&=&0\\
	    \qquad x_2+3x_3&=&0
	\end{array}$$
	\vspace{.5cm}

	Respuesta.-\; Sean $c_1=-1$ y $c_2=1$. Entonces,

	$$\begin{array}{rcl}
	    -x_1+x_2+4x_3 &=& (c_1\cdot 1+c_2\cdot 0)x_1+(c_1\cdot 0 + c_2\cdot 1)x_2 + (c_1(-1)+c_23)x_3\\\\
			  &=& \left[(-1)\cdot 1+1\cdot 0\right]x_1+\left[(-1)\cdot 0+1\cdot 1\right]x_2+\left[-1(-1)+1\cdot 3\right]x_3\\\\
			  &=& -x_1+x_2+4x_3\\\\
	\end{array}$$

	Sean $c_1=1$ y $c_2=3$. Entonces,

	$$\begin{array}{rcl}
	    x_1+3x_2+8x_3 &=& (c_1\cdot 1+c_2\cdot 0)x_1+(c_1\cdot 0 + c_2\cdot 1)x_2 + (c_1(-1)+c_23)x_3\\\\
			  &=& \left[1\cdot 1+3\cdot 0\right]x_1+\left[1\cdot 0+3\cdot 1\right]x_2+\left[1(-1)+3\cdot 3\right]x_3\\\\
			  &=& x_1+3x_2+8x_3\\\\
	\end{array}$$

	Sean $c_1=\dfrac{1}{2}$ y $c_2=1$ Entonces,

	$$\begin{array}{rcl}
	    \dfrac{1}{2}x_1+x_2+\dfrac{5}{2}x_3 &=& (c_1\cdot 1+c_2\cdot 0)x_1+(c_1\cdot 0 + c_2\cdot 1)x_2 + (c_1(-1)+c_23)x_3\\\\
	    			  &=& \left[\dfrac{1}{2}\cdot 1+1\cdot 0\right]x_1+\left[\dfrac{1}{2}\cdot 0+1\cdot 1\right]x_2+\left[\dfrac{1}{2}(-1)+1\cdot 3\right]x_3\\\\
				  	    			  &=& \dfrac{1}{2}x_1+x_2+\dfrac{5}{2}x_3\\\\
	\end{array}$$

	Luego. Sean $c_1=-\dfrac{3}{4}$, $c_2=\dfrac{1}{4}$ y $c_3=0$. Entonces, 

	$$\begin{array}{rcl}
	    x_1-x_3 &=& \left[c_1(-1)+c_2\cdot 1 + c_3\cdot \dfrac{1}{2}\right]x_1+(c_1\cdot 1 + c_2\cdot 3 + c_3\cdot 1)x_2+\left(c_1 \cdot 4+c_2\cdot 8 + c_3\cdot \dfrac{5}{2}\right)x_3\\\\
	    		    &=& \left[-\dfrac{3}{4}(-1)+\dfrac{1}{4}\cdot 1 + 0\cdot \dfrac{1}{2}\right]x_1+\left[-\dfrac{3}{4}\cdot 1 + \dfrac{1}{4}\cdot 3 + 0\cdot 1\right]x_2+\left(-\dfrac{3}{4} \cdot 4+\dfrac{1}{4}\cdot 8 + 0\cdot \dfrac{5}{2}\right)x_3\\\\
			    &=&x_1-x_3\\\\
	\end{array}$$ 

	Por último, sean $c_1=c_2=\dfrac{1}{4}$, $c_3=0$. Entonces,

	$$\begin{array}{rcl}
	    x_2+3x_3 &=& \left[c_1(-1)+c_2\cdot 1 + c_3\cdot \dfrac{1}{2}\right]x_1+(c_1\cdot 1 + c_2\cdot 3 + c_3\cdot 1)x_2+\left(c_1 \cdot 4+c_2\cdot 8 + c_3\cdot \dfrac{5}{2}\right)x_3\\\\
	    		    &=& \left[\dfrac{1}{4}(-1)+\dfrac{1}{4}\cdot 1 + 0\cdot \dfrac{1}{2}\right]x_1+\left[\dfrac{1}{4}\cdot 1 + \dfrac{1}{4}\cdot 3 + 0\cdot 1\right]x_2+\left(\dfrac{1}{4} \cdot 4+\dfrac{1}{4}\cdot 8 + 0\cdot \dfrac{5}{2}\right)x_3\\\\
			    &=&x_2+3x_3\\\\
	\end{array}$$ 
	Por lo tanto las dos ecuaciones dadas son equivalentes.\\\\


    %---------- 4.
    \item Examine los siguientes sistemas como en el ejercicio 2.\\\\
    $$\begin{array}{rcl}
	2x_1+(-1+i)x_2 \qquad \quad\;\, + x_4 &=& 0\\\\
	\qquad 3x_2-2ix_3+5x_4 &=& 0\\
    \end{array}\qquad 
    \begin{array}{rcl}
	\left(1+\dfrac{i}{2}\right)x_1+8x_2-ix_3-x_4&=&0\\\\
	\dfrac{2}{3}x_1-\dfrac{1}{2}x_2+x_3+7x_4&=&0\\\\
    \end{array}$$ 
    \vspace{0.4cm}

	Respuesta.-\;Ya que  $c_1$ y $c_2$ no existen, para
	$$\begin{array}{rcl}
	    2x_1+(-1+i)x_2 + x_4 &=& \left[c_1\left(1+\dfrac{i}{2}\right)+c_2\cdot\dfrac{2}{3}\right]x_1 +  \left[c_1\cdot 8+c_2\cdot\left(-\dfrac{1}{2}\right)\right]x_2 \\\\
				 &+& \left[c_1\cdot (-1)+c_2\right]x_3 + \left[c_1\cdot (-1)+c_2\cdot 7\right]x_4\\\\
	\end{array}$$
	Entonces, las ecuaciones no son equivalentes.\\\\

    %---------- 5.
    \item Sea $F$ un conjunto que contiene exactamente dos elementos, $0$ y $1$. Se define una adición y multiplicación por las tablas:
    $$\begin{array}{c|cc}
	+ & 0 & 1 \\
	\hline
	0 & 0 & 1 \\
	1 & 1 & 0 \\
    \end{array} \qquad \qquad 
    \begin{array}{c|cc}
	\cdot & 0 & 1 \\
	\hline
	0 & 0 & 0 \\
	1 & 0 & 1 \\
    \end{array}$$
    Verificar que el conjunto $F$, juntamente con estas operaciones, es un cuerpo.\\\\
	Demostración.-\; Para verificar que $F$ es un cuerpo, demostraremos las distintas operaciones de suma y multiplicación (Pag. 1.).
	\begin{enumerate}[\scriptsize 1.]
	    \item La conmutatividad para la adición es cierta, ya que
		$$\begin{array}{rcl}
		    0+0 &=& 0+0\\
		    0+1 &=& 0+1\\
		    1+0 &=& 1+0\\
		    1+1 &=& 1+1\\
		\end{array}$$

	    \item La adición es asociativa, ya que 
		$$\begin{array}{rcl}
		    0+(0+1) &=& (0+0)+1\\
		    0+(0+0) &=& (0+0)+0\\
		    0+(1+1) &=& (0+1)+1\\
		    0+(1+0) &=& (0+1)+0\\
		    1+(0+1) &=& (1+0)+1\\
		    1+(1+0) &=& (1+1)+0\\
		    1+(1+1) &=& (1+1)+1\\
		    1+(0+0) &=& (1+0)+0\\
		\end{array}$$

	    \item Ya que $0+0=0$ y $0+1=1$. Entonces, existe un elemento único $0$ de $F$ tal que $x+0=x$, para todo $x$ en $F$.

	    \item Ya que el inverso aditivo de $0$ es $0$ y el inverso aditivo de $1$ es $1$. Entonces, a cada $x$ en $F$, corresponde un elemento único $-x$ de $F$ tal que $x+(-x)=0$.

	    \item La multiplicación es conmutativa, ya que
		$$\begin{array}{rcl}
		    0\cdot 0 &=& 0\cdot 0\\
		    0\cdot 1 &=& 1\cdot 0\\
		    1\cdot 0 &=& 0\cdot 1\\
		    1\cdot 1 &=& 1\cdot 1\\
		\end{array}$$

	    \item La multiplicación es asociativa, ya que
		$$\begin{array}{rcl}
		    0\cdot (0\cdot 1) &=& (0\cdot 0)\cdot 1\\
		    0\cdot (0\cdot 0) &=& (0\cdot 0)\cdot 0\\
		    0\cdot (1\cdot 1) &=& (0\cdot 1)\cdot 1\\
		    0\cdot (1\cdot 0) &=& (0\cdot 1)\cdot 0\\
		    1\cdot (0\cdot 1) &=& (1\cdot 0)\cdot 1\\
		    1\cdot (1\cdot 0) &=& (1\cdot 1)\cdot 0\\
		    1\cdot (1\cdot 1) &=& (1\cdot 1)\cdot 1\\
		    1\cdot (0\cdot 0) &=& (1\cdot 0)\cdot 0\\
		\end{array}$$

	    \item Ya que $1\cdot 0=0$ y $1\cdot 1 = 1$. Entonces, existe un elemento no nulo único de $F$ tal que $x1=x,$ para todo $x$ en $F$.

	    \item Ya que $1\neq 0 \in F$. El inverso multiplicativo de $1$ es $1$.

	    \item La multiplicación es distributiva respecto de la adición, ya que
		$$\begin{array}{rcl}
		    0\cdot (0+1) &=& (0\cdot 0)+(0\cdot 1)\\
		    0\cdot (0+0) &=& (0\cdot 0)+(0\cdot 0)\\
		    0\cdot (1+1) &=& (0\cdot 1)+(0\cdot 1)\\
		    0\cdot (1+0) &=& (0\cdot 1)+(0\cdot 0)\\
		    1\cdot (0+1) &=& (1\cdot 0)+(1\cdot 1)\\
		    1\cdot (1+0) &=& (1\cdot 1)+(1\cdot 0)\\
		    1\cdot (1+1) &=& (1\cdot 1)+(1\cdot 1)\\
		    1\cdot (0+0) &=& (1\cdot 0)+(1\cdot 0)\\
		\end{array}$$
	\end{enumerate}
	Por lo tanto, el conjunto $F$ es un cuerpo.\\\\


    %-------------------- 6. --------------------
    \item Demostrar que si dos sistemas homogéneos de ecuaciones lineales con dos incógnitas tienen las mismas soluciones, son equivalentes.\\\\
	Demostración.-\; Consideremos los dos sistemas homogéneos con dos incógnitas $(x_1,x_2)$.
	$$\left\{\begin{array}{ccc}
		a_{11} x_1 + a_{12} x_2 &=& 0\\
		a_{21} x_1 + a_{22} x_2 &=& 0\\
		\vdots\\
		a_{m1} x_1 + a_{m2} x_2 &=& 0\\
	\end{array}\right. \qquad \mbox{y} \qquad 
	\left\{\begin{array}{ccc}
		b_{11} x_1 + b_{12} x_2 &=& 0\\
		b_{21} x_1 + b_{22} x_2 &=& 0\\
		\vdots\\
		b_{m1} x_1 + b_{m2} x_2 &=& 0\\
	\end{array}\right.$$

	Se dirá que dos sistemas de ecuaciones lineales son equivalentes si cada ecuación de cada sistema es combinación lineal de las ecuaciones del otro sistema. Por lo que por definición de combinación lineal se tiene $m$ escalares $c_1,\ldots,c_m$ que se multiplica la j-ésima ecuación por $c_j$ y que luego se suma. 

	Sean los escalares $c_1,c_2,\ldots c_m$. De donde multiplicamos $m$ ecuaciones del primer sistemas por $c_m$ y sumamos por columnas,
	$$\left(c_1a_{11}+\ldots + c_m a_{m1}\right)x_1 + \left(c_1 a_{12}+\ldots + c_m a_{m2}\right)x_2=0$$

	Luego comparando esta ecuación con todas las ecuaciones del segundo sistema y utilizando también el hecho de que ambos sistemas tiene las mismas soluciones, obtenemos
	$$b_{11}x_1+b_{12}x_2=\left(c_1a_{11}+\ldots + c_m a_{m1}\right)x_1 + \left(c_1 a_{12}+\ldots + c_m a_{m2}\right)x_2.$$

	Lo mismo ocurre con las demás ecuaciones del segundo sistema. Sean otros escalares $c_1,c_2,\ldots, c_m$, tal que  
	$$b_{21}x_1+b_{22}x_2=\left(c_1a_{11}+\ldots + c_m a_{m1}\right)x_1 + \left(c_1 a_{12}+\ldots + c_m a_{m2}\right)x_2$$

	Así, sucesivamente hasta 
	$$b_{m1}x_1+b_{m2}x_2=\left(c_1a_{11}+\ldots + c_m a_{m1}\right)x_1 + \left(c_1 a_{12}+\ldots + c_m a_{m2}\right)x_2$$
	con $m$ escalares $c_1,c_2,\ldots, c_m$. Por lo que demostramos que el segundo sistema es una combinación lineal del primer sistema.\\

	De manera similar podemos demostrar que el primer sistema es una combinación lineal del segundo sistema. Sean los escalares $c_1,c_2,\ldots c_m$, entonces
	$$a_{11}x_1+a_{12}x_2=\left(c_1b_{11}+\ldots + c_m b_{m1}\right)x_1 + \left(c_1 b_{12}+\ldots + c_m b_{m2}\right)x_2.$$
	$$\vdots$$
	$$a_{m1}x_1+a_{m2}x_2=\left(c_1b_{11}+\ldots + c_m b_{m1}\right)x_1 + \left(c_1 b_{12}+\ldots + c_m b_{m2}\right)x_2.$$

	Así concluimos que ambos sistemas son equivalentes.\\\\


    %-------------------- section 1.2 7
    \item Demostrar que todo subcuerpo del cuerpo de los números complejos contiene a todo número racional.\\\\
	Demostración.-\; Sea $F$ un subcampo en $\mathbb{C}$, por este hecho,  tenemos $0\in F$ y $1\in F$. Luego ya que $F$ es un subcampo y cerrado bajo la suma, se tiene
	$$1+1+\ldots + 1 = n\in F.$$
	De este modo $\mathbb{Z}\subseteq F$. Ahora, sabiendo que $F$ es un subcampo, todo elemento tiene un inverso multiplicativo, por lo tanto $\dfrac{1}{n}\in F$. Por otro lado vemos también que $F$ es cerrado bajo la multiplicación. Es decir, para $m,n\in \mathbb{Z}$ y $n\neq 0$ tenemos
	$$m\cdot \dfrac{1}{n}\in F\quad \Rightarrow \quad \dfrac{m}{n}\in F.$$
	Así, concluimos que $\mathbb{Q}\subseteq F.$\\\\

    %-------------------- section 1.2 8
    \item Demostrar que todo cuerpo de características cero contiene una copia del cuerpo de los números racionales.\\\\
	Demostración.-\; Sea $F$ cualquier campo de caracterización cero. Ahora, $0\in F$ y $1\in F$. Ya que la caracteristica de $F$ es cero, se tiene
	$$1\neq 1+1\neq 1+1+1\neq1+1+1+1\ldots \neq 0.$$
	Ahora, $F$ es un campo y por ende cerrado bajo la suma, se obtiene
	$$1+1+\ldots + 1 = n\in F \mbox{ con } n\neq 0.$$
	De este modo, $\mathbb{Z}\subseteq F$. Ahora, dado que $F$ es un cuerpo, todo elemento tiene un inversio multiplicativo, por lo tanto $\dfrac{1}{n}\in F$. Además, $F$ es cerrado bajo la multiplicación, así, para $m, n \in \mathbb{Z}$ y $n \neq 0$, tenemos
	$$m\cdot \dfrac{1}{n}\in F\quad \Rightarrow \quad \dfrac{m}{n}\in F.$$
	Por lo que $\mathbb{Q}\in F$. Concluimos que $F$ contiene una copia del campo de número racional.\\\\

\end{enumerate}


\section{Matrices y operaciones elementales de fila}
El sistema (1-1) se abreviará ahora así:
$$AX=Y$$
donde 
$$
\left[\begin{array}{cccc}
    a_{11} & a_{12} & \cdots & a_{1n} \\
    \vdots & \vdots &  & \vdots \\
    a_{m1} & a_{m2} & \cdots & a_{mn}
\end{array}\right]
\left[\begin{array}{c}
	x_1 \\
	\vdots \\
	x_n
\end{array}\right] = 
\left[\begin{array}{c}
	y_1 \\
	\vdots \\
	y_m
\end{array}\right]
$$

$A$ se llama \textbf{matriz de los coeficientes} del sistema.

    Una \textbf{matriz} $m\times n$ \textbf{sobre el cuerpo} $F$ es una función $A$ del conjunto de los pares enteros $(i,j)$, $1\leq i \leq m,$ $1\leq j \leq n,$ en el cuerpo $F$.

    Los \textbf{elementos} de la matriz $A$ son los escalares $A(i,j)=A_{ij}$, con frecuencia, suele ser más conveniente describir la matriz disponiendo sus elementos en un arreglo rectangular con $m$ filas y $n$ columnas.

Deseamos ahora considerar operaciones sobre las filas de la matriz $A$ que corresponden a la formación de combinaciones lineales de las ecuaciones del sistema $AX=Y$. Se limitará nuestra atención a \textbf{tres operaciones elementales de filas} en una matriz $m\times n$, sobre el cuerpo $F$:

    \begin{enumerate}[\bfseries 1.]
	\item Multiplicación de una fila de $A$ por un escalar $c$ no nulo;
	\item Remplazo de la r-ésima fila de $A$ por la fila $r$ más $c$ veces la fila $s$, donde $c$ es cualquier escalar y $r\neq s$;
	\item Intercambio de dos filas de $A$.
    \end{enumerate}

Una operación elemental de filas es, pues, un tipo especial de función (regla) $e$ que asocia a cada matriz $m\times n$, $A$, una matriz $m\times n$, $e(A)$. Se puede describir $e$ en forma precisa en los tres casos como sigue:

    \begin{enumerate}[\bfseries 1.]
	\item $e(A)_{ij}$ si $i\neq r,$ $e(A)_{rj}=cA_{rj}$.
	\item $e(A)_{ij}$ si $i\neq r,$ $e(A)_{rj}=A_{rj}+cA_{sj}.$
	\item $e(A)_{ij}$ si $i$ es diferente de $r$ y $s$, $e(A)_{rj}=A_{sj},$ $e(A)_{sj}=A_{rj}.$
    \end{enumerate}

    Una función $e$ particular esta definida en la clase de todas las matrices sobre $F$ que tiene $m$ filas.

%-------------------- Teorema 1.2
\begin{teo}
    A cada operación elemental de filas $e$ corresponde una operación elemental de filas $e_1$, del mismo tipo de $e$, tal que $e_i(e(A))=e(e_1(A))=A$ para todo $A$. Es decir, existe la operación (función) inversa de una operación elemental de filas y es una operación elemental de filas del mismo tipo.\\\\
    Demostración.-\; (1) Supóngase que $e$ es la operación que multiplica la r-ésima fila de una matriz por un escalar no nulo $c$. Sea $e_1$ la operación que multiplica la fila $r$ por $c^{-1}$. (2) Supóngase que $e$ sea la operación que remplaza la fila $r$ por la misma fila $r$ a la que le sumo la fila $s$ multiplicada por $c$, $r\neq s$. Sea $e_1$ la operación que reemplaza la fila $r$ por la fila $r$ a la que se le ha sumado la fila $s$ multiplicada por $(-c)$. (3) Si $e$ intercambia las filas $r$ y $s$, sea $e_1=e$. En cada uno de estos casos es claro que $e_1(e(A))=e(e_1(A))=A$ para todo $A$.
\end{teo}

%-------------------- Definición 1.1
    \begin{def.}
	Si $A$ y $B$ son dos matrices $m\times n$ sobre el cuerpo $F$, se dice que $B$ es equivalente por filas a $A$ si $B$ se obtiene de $A$ por una sucesión finita de operaciones elementales de filas.
    \end{def.}

Usando el teorema 2, se verificará que: Cada matriz es equivalente por filas a ella misma. Si $B$ es equivalente por filas a $A$, entonces $A$ es equivalente por filas a $B$; si $B$ es equivalente por filas a $A$ y $C$ es equivalente por filas a $B$, entonces $C$ es equivalente por filas a $A$. O sea, que la equivalencia por filas es una relación de equivalencia.

%-------------------- Teorema 1.3
\begin{teo}
    Si $A$ y $B$ son matrices equivalentes por filas, los sistemas homogeneos de ecuaciones lineales $AX=0$ y $BX=0$ tienen exactamente las mismas soluciones.\\\\
	Demostración.-\; Supóngase que se pasa de $A$ a $B$ por una sucesión finita de operaciones elementales de filas:
	$$A=A_0\to A_1 \to \ldots \to A_k = B.$$
	Basta demostrar que los sistemas $A_jX=0$ y $A_{j+1}X=0$ tienen las mismas soluciones, es decir, que una operación elemental por filas no altera el conjunto de soluciones.\\
	así, supóngase que $B$ se obtiene de $A$ por una sola operación elemental de filas. Sin que importe cuál de los tres tipos (1), (2) o (3) de operaciones sea, cada ecuación del sistema $BX=0$ será combinación lineal de las ecuaciones del sistema $AX=0$. Dado que la inversa de una operación elemental de filas es una operación elemental de filas, toda ecuación de $AX=0$ será también combinación lineal de las ecuaciones de $BX=0$. Luego estos dos sistemas son equivalente y, por el teorema 1, tienen las mismas soluciones.
\end{teo}

%-------------------- Definición 1.2
    \begin{def.}
	Una matriz $m\times n$, $R$, se llama \textbf{reducida por filas} si:
	\begin{enumerate}[(a)]
	    \item el primer elemento no nulo de cada fila no nula de $R$ es igual a $1$;
	    \item cada columna de $R$ que tiene el primer elemento no nulo de alguna fila tiene todos sus otros elementos $0$.
	\end{enumerate}
    \end{def.}

%-------------------- Teorema 1.4
\begin{teo}
    Toda matriz $m\times n$ sobre el cuerpo $F$ es equivalente por filas a una matriz reducida por filas.\\\\
    Demostración.-\; Sea $A$ una matriz $m\times n$ sobre $F$. Si todo elemento de la primera fila $A$ es $0$, la condición (a) se cumple en lo que concierne a la fila $1$. Si la fila $1$ tiene un elemento no nulo, sea $k$ el menor entero positivo $j$ para el que $A_{1j}\neq 0$. Multiplicando la fila $1$ por $A_{1k}^{-1}$ la condición (a) se cumple con respecto a esa fila. Luego, para todo $i\geq 2,$ se suma $(-A_{ik})$ veces la fila $1$  a la fila $i$. Y ahora el primer elemento no nulo de la fila $1$ está en la columna $k$, ese elemento es $1$, y todo otro elemento de la columna $k$ es $0$.\\
    Considérese ahora la matriz que resultó de lo anterior. Si todo elemento de la fila $2$ es $0$, se deja tal cual. Si algún elemento de la fila $2$ es diferente de $0$, se multiplica esa fila por un escalar de modo que el primer elemento no nulo sea $1$. En caso de que la fila $1$ haya tenido un primer elemento no nulo en la columna $k$, este primer elemento no nulo de la fila $2$ no puede estar en la columna $k$, supóngase que esté en la columna $k_r\neq k$. Sumando múltiplos apropiados de la fila $2$ a las otras filas, se puede lograr que todos los elementos de la columna $k_r$ sean $0$, excepto el $1$ en la fila $2$. Lo que es importante observar es lo siguiente: Al efectuar etas operaciones, no se alteran los elementos de la fila $1$ el alas columnas $1,\ldots, k$ ni ningún elemento de la columna $k$. Es claro que, si la fila $1$ era idénticamente nula, las operaciones con la fila $2$ no afecta la fila $1$.\\
    Si se opera, como se indicó, con una fila cada vez, es evidente que después de un número finito de etapas se llegará a una matriz reducida por filas.
\end{teo}

\section{Ejercicios}


\begin{enumerate}[\bfseries 1.]

    %---------- 1.
    \item Hallar todas las soluciones del sistema de ecuaciones
    $$\begin{array}{rcrcl}
	(1-i)x_1&-&ix_2 &=& 0\\
	2x_1&+&(1-i)x_2 &=& 0.
    \end{array}$$
    \vspace{.4cm}
	Respuesta.-\; Colocando en su forma matricial tenemos,
	$$\left[\begin{array}{*{2}{c}}
	    1-i & -i\\
	    2 & 1-i
	\end{array}\right]$$

	Reduciendo por filas,

	$$\begin{array}{*{5}{c}}
	    \left[\begin{array}{*{2}{c}}
		1-i & -i\\
		2 & 1-i
	    \end{array}\right]
	    &R_1\leftrightarrow R_2&
	    \left[\begin{array}{*{2}{c}}
		2 & 1-i\\
		1-i & -i
	    \end{array}\right]
	    &\dfrac{R_1}{2}\to R_1&
	    \left[\begin{array}{*{2}{c}}
		1 & \dfrac{1-i}{2}\\\\
		1-i & -i
	    \end{array}\right]\\\\
	    &R_2-(1-i)R_1\to R_2&
	    \left[\begin{array}{*{2}{c}}
		1 & \dfrac{1-i}{2}\\\\
		0 & 0
	    \end{array}\right]
	\end{array}$$
	Sustituyendo se tiene,
	$$x_1+\dfrac{1-i}{2}x_2=0\quad \Rightarrow \quad x_1=\dfrac{-1+i}{2}x_2.$$
	Supongamos $x_2=t\in \mathbb{C}$, por lo que el conjunto de soluciones estará dado por
	$$\left\{\left(\dfrac{-1+i}{2}t,t\right)|t\in\mathbb{C}\right\}.$$\\


    %----------  2.
    \item Si
    $$A=\left[\begin{array}{rrr}
	    3 & -1 & 2 \\
	    2 & 1 & 1 \\
	    1 & -3 & 0
    \end{array}\right]$$
    Hallar todas las soluciones de $AX=0$ reduciendo $A$ por filas.\\\\
    	Respuesta.-\; Se efectuará una sucesión finita de operaciones elementales de filas en $A$, indicando el tipo de operación realizada.

	$$\begin{array}{ccccc}

	    \left[\begin{array}{rrr}
		3 & -1 & 2 \\
		2 & 1 & 1 \\
		1 & -3 & 0
	    \end{array}\right]
	    &R_1 \to R_3 &
	    \left[\begin{array}{rrr}
		1 & -3 & 0\\
		2 & 1 & 1 \\
		3 & -1 & 2 
	    \end{array}\right]
	    &R_2-2R_1 \to R_2&
	    \left[\begin{array}{rrr}
		1 & -3 & 0\\
		0 & 7 & 1 \\
		3 & -1 & 2 
	    \end{array}\right]\\\\
	    &R_3-3R_1\to R_3&
	    \left[\begin{array}{rrr}
		1 & -3 & 0\\
		0 & 7 & 1 \\
		0 & 8 & 2 
	    \end{array}\right]
	    &R_3-\dfrac{8}{7} R_2 \to R_3&
	    \left[\begin{array}{*{3}{r}}
		1 & -3 & 0\\
		0 & 7 & 1 \\
		0 & 0 & \frac{6}{7} 
	    \end{array}\right]\\\\

	\end{array}$$

	Por lo tanto tenemos: $\;\dfrac{6}{7}x_3=0,\;\; 7x^2+x^3=0 \; \; \mbox{y} \;\; x_1-3x_2=0.$ De donde 
	$$x_1=x_2=x_3=0.$$\\


    %---------- 3.
    \item Si
    $$A=\left[\begin{array}{rrr}
	    6 & -4 & 0 \\
	    4 & -2 & 0 \\
	    -1 & 0 & 3 
    \end{array}\right]$$
    Hallar todas las soluciones de $AX=2X$ y todas las soluciones de $AX=3X$ (el símbolo $cX$ representa la matriz, cada elemento de la cual es $c$ veces el correspondiente elemento de $X$).\\\\
	Demostración.-\; Sea $AX=2X$, entonces
	$$AX=2X\quad \Rightarrow \quad AX-2X=0\quad \Rightarrow \quad (A-2I)X=0.$$
	De donde,

	$$A-2I=\left[\begin{array}{*{3}{r}}
	    6 & -4 & 0 \\
	    4 & -2 & 0 \\
	    -1 & 0 & 3
	\end{array}\right]-2
	\left[\begin{array}{*{3}{r}}
	    1 & 0 & 0 \\
	    0 & 1 & 0 \\
	    0 & 0 & 1 
	\end{array}\right] = 
	\left[\begin{array}{*{3}{r}}
	    4 & -4 & 0 \\
	    4 & -4 & 0 \\
	    -1 & 0 & 1 
	\end{array}\right]$$

	Resolviendo tenemos,

	$$\begin{array}{ccccc}

	    \left[\begin{array}{*{3}{r}}
		4 & -4 & 0 \\
		4 & -4 & 0 \\
		-1 & 0 & 1 
	    \end{array} \right]
	    &R_2 - R_1 \to R_2&
	    \left[\begin{array}{*{3}{r}}
		4 & -4 & 0 \\
		0 & 0 & 0 \\
		-1 & 0 & 1 
	    \end{array} \right]
	    &R_3 + \dfrac{1}{4} R_1 \to R_3&
	    \left[\begin{array}{*{3}{r}}
		4 & -4 & 0 \\
		0 & 0 & 0 \\
		0 & -1 & 1 
	    \end{array}\right]\\\\
	    &R_2 \leftrightarrow R_3&
	    \left[\begin{array}{*{3}{r}}
		4 & -4 & 0 \\
		0 & -1 & 1 \\
		0 & 0 & 0
	    \end{array} \right]& & \\

	\end{array}$$

	Luego, las soluciones estarán dadas por, 
	$$-x_2+x_3=0\quad \Rightarrow \quad x_2=x_3 \qquad \mbox{y}\qquad 4x_1-4x_2=0 \quad \Rightarrow \quad x_1=x_2.$$
	Por lo tanto,
	$$x_1=x_2=x_3.$$\\
	Por otro lado si $AX=3X$, entonces
	$$AX=3X\quad \Rightarrow \quad AX-3X=0\quad \Rightarrow \quad (A-3I)X=0.$$
	De donde,

	$$A-2I=\left[\begin{array}{*{3}{r}}
	    6 & -4 & 0 \\
	    4 & -2 & 0 \\
	    -1 & 0 & 3
	\end{array}\right]-2
	\left[\begin{array}{*{3}{r}}
	    1 & 0 & 0 \\
	    0 & 1 & 0 \\
	    0 & 0 & 1 
	\end{array} \right]= 
	\left[\begin{array}{*{3}{r}}
	    3 & -4 & 0 \\
	    4 & -5 & 0 \\
	    -1 & 0 & 0 
	\end{array}\right]$$

	Resolviendo tenemos,

	$$\begin{array}{ccccc}

	    \left[\begin{array}{*{3}{r}}
		3 & -4 & 0 \\
		4 & -5 & 0 \\
		-1 & 0 & 0 
	    \end{array} \right]
	    &R_1 \leftrightarrow R_3&
	    \left[\begin{array}{*{3}{r}}
		-1 & 0 & 0 \\
		4 & -5 & 0 \\
		3 & -4 & 0 
	    \end{array}\right] 
	    &R_2 + 4 R_1 \to R_2&
	    \left[\begin{array}{*{3}{r}}
		-1 & 0 & 0 \\
		0 & -5 & 0 \\
		3 & -4 & 0 
	    \end{array}\right]\\\\
	    &R_3 +3R_1 \to R_3&
	    \left[\begin{array}{*{3}{r}}
		-1 & 0 & 0 \\
		0 & -5 & 0 \\
		0 & -4 & 0 
	    \end{array}\right] 
	    &R_3-\dfrac{4}{5}R_2 \to R_3& 
	    \left[\begin{array}{*{3}{r}}
		-1 & 0 & 0 \\
		0 & -5 & 0 \\
		0 & 0 & 0 
	    \end{array}\right] 

	\end{array}$$

	Luego,  $0x_3=0\; \Rightarrow \; x_3\in \mathbb{R}$, $x_1=0$ y $x_2=0$ por lo tanto,
	$$x_3\in \mathbb{R}, x_1=x_2=0.$$\\

    %---------- 4.
    \item Hallar una matrix reducida por filas que sea equivalente por filas a 
    $$A=\left[\begin{array}{*{3}{c}}
	i & -(1+i) & 0 \\
	1 & -2 & 1 \\
	1 & 2i & -1
    \end{array}\right]$$
    \vspace{0.4cm}
	Respuesta.-\;


    %---------- 5.
    \item 

    %---------- 6.
    \item Sea
    $$\left[\begin{array}{rr}
	    a & b  \\
	    c & d  
    \end{array}\right]$$
    es una matriz $2\times 2$ con elementos complejos. Supóngase que $A$ es reducida por filas y también que $a+b+c+d=0$. Demostrar que existen exactamente tres de estas matrices.\\\\
    	Demostración.-\; Ya que $A$ es dado como una matríz reducida por filas, entonces se presentas los siguientes casos.
	\begin{enumerate}[\bfseries \textbf{Caso} i.]
	    \item Sea $a=b=c=d=0$ por lo tanto
		$$\left[\begin{array}{*{2}{r}}
		    0 & 0 \\
		    0 & 0
		\end{array}\right]$$

	    \item Sea $c=d=0\; \Rightarrow \; a+b=0 \; \Rightarrow \; b=-a$ con $a=1$, entonces
		$$\left[\begin{array}{*{2}{r}}
		    1 & -1 \\
		    0 & 0
		\end{array}\right]$$

	    Otra posibilidad podría ser que la fila $1$ sea cero. Es decir, $a=b=0\; \Rightarrow \; c+d=0\; \Rightarrow d=-c$ de donde tenemos

		$$\left[\begin{array}{*{2}{r}}
		    0 & 0 \\
		    1 & -1 
		\end{array}\right]$$

	    \item Dos filas distintas de cero. Es decir, $a\neq 0, d\neq 0$, $c=b=0\; \Rightarrow \; a+d=0 \; \Rightarrow \; d=-a$, de donde $A$ al ser reducido por fila implicará $a=1\; \rightarrow \; d=-1$. Pero $d$ es la entrada principal distinto de cero de la fila $2$ por lo tanto debe ser igual a uno. Así este caso no existe.
	    \end{enumerate}

	    De donde concluimos que existe dos matrices que satisfacen la condición dada.\\\\

    %---------- 7.
    \item Demostrar que el intercambio de dos filas en una matriz puede hacerse por medio de un número finito de operaciones elementales con filas de los otros tipos.\\\\
	Demostración.-\; Consideremos la matriz identidad:
	$$\left[\begin{array}{*{3}{r}}
	    1 & 0 & 0 \\
	    0 & 1 & 0 \\
	    0 & 0 & 1
	\end{array}\right]$$
	Supongamos que se necesita el intercambio de las filas $1$ y $3$, entonces se puede realizar la siguiente secuencia para hacerlo:

	$$\begin{array}{ccccc}

	    \left[\begin{array}{*{3}{r}}
		1 & 0 & 0 \\
		0 & 1 & 0 \\
		0 & 0 & 1
	    \end{array}\right] 
	    &R_1 \to R_1+R_3&
	    \left[\begin{array}{*{3}{r}}
		1 & 0 & 1 \\
		0 & 1 & 0 \\
		0 & 0 & 1
	    \end{array}\right] 
	    &R_3 \to R_3-R_1&
	    \left[\begin{array}{*{3}{r}}
		1 & 0 & 1 \\
		0 & 1 & 0 \\
		-1 & 0 & 0
	    \end{array}\right]\\\\
	    &R_3  \to -R_3&
	    \left[\begin{array}{*{3}{r}}
		1 & 0 & 1 \\
		0 & 1 & 0 \\
		1 & 0 & 0
	    \end{array}\right] 
	    &R_1 \to R_1-R_3& 
	    \left[\begin{array}{*{3}{r}}
		0 & 0 & 1 \\
		0 & 1 & 0 \\
		1 & 0 & 0
	    \end{array}\right] 

	\end{array}$$

	Generalizando obtenemos que para realizar el intercambio de $R_i$ y $R_j$, la secuencia de operaciones elementales de los otros tipos es: 

	\begin{enumerate}[a)]
	    \item $R_j\to R_j-R_i$. Restar fila $i$ de la fila $j$, que da como resultado una fila $j$ como el negativo de la fila original $i$.
	    \item $R_j\to -R_j$. Fila negativa $j$ lo que resultará en la fila $j$ cambiando a fila $i$.
	    \item $R_j\to R_i-R_j$. Restar fila $j$ de la fila $i$ que resultará en la fila $i$ cambiando a la fila original $j$.\\\\
	\end{enumerate}

    %---------- 8.
    \item 

\end{enumerate}


\section{Matrices escalón reducida por filas}

    \begin{def.}
	Una matriz $m\times n$, $R$, se llama matriz escalón reducida por filas si:
	\begin{enumerate}[(a)]
	    \item $R$ es reducida por filas;
	    \item toda fila de $R$ que tiene todos los elementos $0$ está debajo de todas las filas que tienen elementos no nulos;
	    \item si las filas $1,\ldots, r$ son las filas no nulas de $R$, y si el primer elemento no nulo de la fila $i$ está en la columna $k_i, i=1,\ldots, r$, entonces $k_1<k_2<\ldots,k_r.$
	\end{enumerate}
	Se puede describir también una matriz escalón $R$ reducida por filas como sigue. Todo elemento de $R$ es $0$, o existe un número positivo $r$, $l\leq r \leq m$, y $r$ entero positivo $k_1,\ldots ,k_r$ con $1\leq k_i\leq n$ y
	\begin{enumerate}[(a)]
	    \item $R_{ij}=0$ para $i>r,$ y $R_{ij}=0$ si $j<k_i$.
	    \item $R_{ik_j}=\delta_{ij}$, $1\leq i \leq r,$ $1\leq j\leq r$.
	    \item $k_1<\ldots < k_r$.
	\end{enumerate}
    \end{def.}

% -------------------- teorema 5
\begin{teo}
    Toda matriz $m\times n$, $A$, es equivalente por filas a una matriz escalón por filas.\\\\
	Demostración.-\; Sabemos que $A$ es equivalente por filas a una matriz reducida por filas. Todo lo que se necesita observar es que, efectuando un número finito de intercambios de filas en una matriz reducida por filas, se la puede llevar a la forma escalón reducida por filas.\\
\end{teo}

% -------------------- teorema 6
\begin{teo}
    Si $A$ es una matriz $m\times n$ con $m<n$, el sistema homogéneo de ecuaciones lineales $AX=0$ tiene una solución trivial.\\\\
	Demostración.-\; Sea $R$ una matriz escalón reducida por fila que sea equivalente por fila a $A$. Entonces los sistemas $AX=0$ y $RX=0$ tienen las mismas soluciones por el teorema 3. Si $r$ es el número de filas no nulas de $R$, entonces ciertamente $r\leq m$, y como $m<n$ tenemos que $r<n$. Se sigue inmediatamente de las observaciones anteriores que $AX=0$ tiene una solución trivial.\\
\end{teo}

% -------------------- teorema 7
\begin{teo}
    Si $A$ es una matriz $n\times n$ (cuadrada), $A$ es equivalente por filas a la matriz identidad $n\times n$, si, y sólo si, el sistema de ecuaciones $AX=0$ tiene solamente la solución trivial.\\\\
	Demostración.-\; Si $A$ es equivalentes por filas a $I$, entonces $AX=0$ e $IX=0$ tienen las mismas soluciones. Recíprocamente, supóngase que $AX=0$ tiene solamente la solución trivial $X=0$. Sea $R$ una matriz escalón reducida por filas $n\times n$, que es equivalente por filas a $A$, y sea $r$ el número de filas no nulas de $R$. Entonces $RX=0$ carece de solución no trivial. Con lo que $r\geq n.$ Pero como $R$ tiene un $1$ como primer elemento no nulo en cada una de sus $n$ filas y como estos $1$ están en las diferentes columnas $n$, $R$ debe ser la matriz identidad $n\times n$.\\
\end{teo}

Se construye la matriz aumentada $A'$ del sistema $AX=Y$. Esta es la matrix $m\times (n+1)$ cuyas primeras $n$ columnas son las columnas de $A$ y cuya última columna es $Y$; más precisamente,
$$A'_{ij}=A_{ij},\mbox{ si } j \leq n.$$
$$A'_{i(n+1)}=y_i.$$


\section{Ejercicios}


\begin{enumerate}[1.] 

    %-------------------- 1.
    \item Hallar, mediante reducción por filas de la matriz de coeficientes todas las soluciones del siguiente sistema de ecuaciones:
    $$\begin{array}{rrrrrcc}
	\frac{1}{3}x_1&+&2x_2&-&6x_3&=&0\\
		      -4x_1&&&+&5x_3&=&0\\
			   -3x_1&+&6x_2&-&13x_3&=&0\\
			   -\frac{7}{3}x_1&+&2x_2&-&\frac{8}{3}x_3&=&0\\
   \end{array}.$$\\\\
	Respuesta.-\; Consideremos la matriz de la forma $AX=0$ para la respectiva reducción por filas de la siguiente manera:

	$$\begin{array}{ccccc}

	    \left[\begin{array}{*{3}{r}}
		\frac{1}{3} & 2 & -6 \\
		-4 & 0 & 5 \\
		-3 & 6 & -13 \\
		-\frac{7}{3} & 2 & -\frac{8}{3}
	    \end{array}\right] 
	    &3R_1\to R_1&
	    \left[\begin{array}{*{3}{r}}
		1 & 6 & -18 \\
		-4 & 0 & 5 \\
		-3 & 6 & -13 \\
		-\frac{7}{3} & 2 & -\frac{8}{3}
	    \end{array}\right] 
	    &\begin{array}{rcl}
		R_2 + 4R_1 &\to& R_2 \\
		R_3 + 3R_1 &\to& R_3 \\
		R_4+\frac{7}{3}R_1&\to&R_4
		\end{array}&
	    \left[\begin{array}{*{3}{r}}
		1 & 6 & -18 \\
		0 & 24 & -67 \\
		0 & 24 & -67 \\
		0 & 16 & -\frac{134}{3}
	    \end{array}\right]\\\\
	    &\frac{1}{24}R_2  \to R_2&
	    \left[\begin{array}{*{3}{r}}
		1 & 6 & -18 \\
		0 & 1 & -\frac{67}{24} \\
		0 & 24 & -67 \\
		0 & 16 & -\frac{134}{3}
	    \end{array}\right] 
	    &\begin{array}{rcl}
		R_1-6r_2 &\to&R_1 \\
		R_3-24R_2&\to&R_3 \\
		R_4-16R_2&\to&R_4
	    \end{array}&
	    \left[\begin{array}{*{3}{r}}
		1 & 6 & -18 \\
		0 & 1 & -\frac{67}{24} \\
		0 & 0 & 0 \\
		0 & 0 & 0 \\
	    \end{array}\right] 

	\end{array}$$

	De donde 
	$$\begin{array}{rcl}
	    x_i-\dfrac{5}{4}x_3=0&\Rightarrow & x_1 = \dfrac{5}{4}x_3\\\\
	    x_2-\dfrac{67}{24}x_3=0&\Rightarrow & x_2 = \dfrac{67}{24}x_3\\\\
	\end{array}$$

	Por lo que la solución está dado por $\left\{\left(\dfrac{5}{4},\dfrac{67}{24},1\right)t\; / \; t\in \mathbb{R}\right\}$.\\\\


    %-------------------- 2.
    \item Hallar una matriz escalón reducida por filas que sea equivalente a
    $$\left[\begin{array}{rr}
	1 & -i \\
	2 & 2 \\
	i & 1+i\\
    \end{array}\right]$$\\
    ¿Cuales son las soluciones de $AX=0$?.\\\\
    	Respuesta.-\; Por la eliminación Gaussiana se tiene,

	$$\begin{array}{ccccc}

	    \left[\begin{array}{*{2}{r}}
		1 & -i\\
		2 & 2 \\
		i & 1+i
	    \end{array}\right] 
	    &\begin{array}{rcl}
		R_2-2R_1 &\to& R_2 \\
		R_3-iR_1 &\to& R_3 \\
	    \end{array}&
	    \left[\begin{array}{*{2}{r}}
		1 & -i \\
		0 & 2+2i \\
		0 & i 
	    \end{array}\right] 
	    &\frac{1}{i}R_3 \to R_3&
	    \left[\begin{array}{*{2}{r}}
		1 & -i \\
		0 & 2+2i \\
		0 & 1
	    \end{array}\right]\\\\
	    &\begin{array}{rcl}
		R_1+iR_3 &\to& R_1 \\
		R_2-(2+2i)R_3&\to&R_2
	    \end{array}&
	    \left[\begin{array}{*{2}{r}}
		1 & 0 \\
		0 & 0 \\
		0 & 1 
	    \end{array}\right] 
	    &R_2\leftrightarrow R_3&
	    \left[\begin{array}{*{2}{r}}
		1 & 0 \\
		0 & 1 \\
		0 & 0
	    \end{array}\right] 

	\end{array}$$

	Que es la matriz escalonada reducida por filas equivalente a $A$. Por lo tanto la solución esta dada por,
	$$x_1=x_2=0.$$\\

    %-------------------- 3.
    \item 

    %-------------------- 4.
    \item 

    %-------------------- 5.
    \item Dar un ejemplo de un sistema de dos ecuaciones lineales con dos incógnitas que no tengan solución.\\\\
	Respuesta.-\; Consideremos el siguiente sistema de ecuaciones 
	$$\begin{array}{rcccl}
	    x_1&+&x_2&=&1\\
	    2x_1&+&2x_2&=&3\\\\
	\end{array}$$
	Luego el sistema no tiene solución ya que,
	$$\begin{array}{rcl}
	    2x_1+2x_3&=&2(x_1+x_2)\\
		     &=&2\cdot 1\\
		     &=&2\neq 3.
	\end{array}$$
	\vspace{.6cm}\\\\

    %-------------------- 6.
    \item \boldmath Mostrar que el sistema
    $$\begin{array}{ccccccccc}
	x_1&-&2x_2&+&x_3&+&2x_4&=&1\\
	   x_1&+&x_2&-&x_3&+&x_4&=&2\\
	   x_1&+&7x_2&-&5x_3&-&x_4&=&3\\
    \end{array}$$
    No tiene solución.\\\\
	Demostración.-\; Se tiene la matriz 
	$$AX=b \quad \Rightarrow \quad \left[\begin{array}{*{4}{r}}
	    1 & -2 & 1 & 2 \\
	    1 & 1 & -1 & 1 \\
	    1 & 7 & -5 & -1
	\end{array}\right]
	\left[\begin{array}{r}
	    x_1 \\
	    x_2 \\
	    x_3 \\
	    x_4
	\end{array}\right]
	= 
	\left[\begin{array}{r}
	    1 \\
	    2 \\
	    3
	\end{array}\right]$$
	Entonces,

	$$\begin{array}{ccc}

	    \left[\begin{array}{rrrr|r}
		1 & -2 & 1 & 2 & 1 \\
		1 & 1 & -1 & 1 & 2 \\
		1 & 7 & -5 & -1 & 3
	    \end{array}\right]
	    &\left[\begin{array}{rcl}
		R_2-R_1 &\to& R_2 \\
		R_3-R_1 &\to& R_3 \\
	    \end{array}\right]&
	    \left[\begin{array}{rrrr|r}
		1 & -2 & 1 & 2 & 1 \\
		0 & 3 & -2 & -1 & 1 \\
		0 & 9 & -6 & -3 & 2
	    \end{array}\right]\\\\
	    &\frac{1}{3}R_2\to R_2&
	    \left[\begin{array}{rrrr|r}
		1 & -2 & 1 & 2 & 1 \\
		0 & 1 & -\frac{2}{3} & -\frac{1}{3} & \frac{1}{3} \\
		0 & 9 & -6 & -3 & 2
	    \end{array}\right]\\\\
	    &\left[\begin{array}{rcl}
		R_1+2R_2 &\to& R_1 \\
		R_3-9R_2&\to&R_3
	    \end{array}\right]&
	    \left[\begin{array}{rrrr|r}
		1 & 0 & -\frac{1}{3} & \frac{4}{3} & \frac{5}{3} \\
		0 & 1 & -\frac{2}{3} & -\frac{1}{3} & \frac{1}{3} \\
		0 & 0 & 0 & 0 & -1 
	    \end{array}\right]

	\end{array}$$

	Que es la forma escalonada reducida por filas del sistema. De la última fila, obtenemos $0=-1$ lo cual es absurdo. Por lo tanto, el sistema dado no tiene solución.\\\\


    %-------------------- 7.
    \item Hallar todas las soluciones de
    $$\begin{array}{ccccccccccr}
	2x_1&-&3x_2&-&7x_3&+&5x_4&+&2x_5&=&-2\\
	x_1&-&2x_2&-&4x_3&+&3x_4&+&x_5&=&-2\\
	2x_1&&&-&4x_3&+&2x_4&+&x_5&=&3\\
	x_1&-&5x_2&-&7x_3&+&6x_4&+&2x_5&=&-7\\
    \end{array}$$
	Respuesta.-\; Sea la matriz,
	$$AX=b \quad \Rightarrow \quad \left[\begin{array}{*{5}{r}}
	    2 & -3 & -7 & 5 & 2 \\
	    1 & -2 & -4 & 3 & 1 \\
	    2 & 0 & -4 & 2 & 1 \\
	    1 & -5 & -7 & 6 & 2
	\end{array}\right]
	\left[\begin{array}{r}
	    x_1 \\
	    x_2 \\
	    x_3 \\
	    x_4 \\
	    x_5
	\end{array}\right]
	= 
	\left[\begin{array}{r}
	    -2 \\
	    -2 \\
	    3 \\
	    -7
	\end{array}\right]$$
	Entonces,

	$$\begin{array}{ccl}

	    \left[\begin{array}{rrrrr|r}
		2 & -3 & -7 & 5 & 2 & -2 \\
		1 & -2 & -4 & 3 & 1 & -2 \\
		2 & 0 & -4 & 2 & 1 & 3 \\
		1 & -5 & -7 & 6 & 2 & -7
	    \end{array}\right]
	    &\frac{1}{2}R_1\to R_1&
	    \left[\begin{array}{rrrrr|r}
		1 & -\frac{3}{2} & -\frac{7}{2} & \frac{5}{2} & 1 & -1 \\
		1 & -2 & -4 & 3 & 1 & -2 \\
		2 & 0 & -4 & 2 & 1 & 3 \\
		1 & -5 & -7 & 6 & 2 & -7
	    \end{array}\right]\\\\
	    &\begin{array}{rcl}
		R_2-R_1 &\to& R_2 \\
		R_3-2R_1 &\to& R_3 \\
		R_4-R_1 &\to& R_4
	    \end{array}&
	    \left[\begin{array}{rrrrr|r}
		1 & -\frac{3}{2} & -\frac{7}{2} & \frac{5}{2} & 1 & -1 \\
		0 & -\frac{1}{2} & -\frac{1}{2} & \frac{1}{2} & 0 & -1 \\
		0 & 3 & 3 & -3 & -1 & 5 \\
		0 & -\frac{7}{2} & -\frac{7}{2} & \frac{7}{2} & 1 & -6
	    \end{array}\right]\\\\
	    &-\frac{1}{2}R_2\to R_2&
	    \left[\begin{array}{rrrrr|r}
		1 & -\frac{3}{2} & -\frac{7}{2} & \frac{5}{2} & 1 & -1 \\
		0 & 1 & 1 & -1 & 0 & 2 \\
		0 & 3 & 3 & -3 & -1 & 5 \\
		0 & -\frac{7}{2} & -\frac{7}{2} & \frac{7}{2} & 1 & -6
	    \end{array}\right]\\\\
	    &\begin{array}{rcl}
		R_1-\frac{3}{2}R_2 &\to& R_1 \\
		R_3-3R_2 &\to& R_3 \\
		R_4-\frac{7}{2}R_2 &\to& R_4
	    \end{array}&
	    \left[\begin{array}{rrrrr|r}
		1 & 0 & -2 & 1 & 1 & 2 \\
		0 & 1 & 1 & -1 & 0 & 2 \\
		0 & 0 & 0 & 0 & 1 & 1 \\
		0 & 0 & 0 & 0 & 1 & 1
	    \end{array}\right]\\\\
	    &\begin{array}{rcl}
		R_1-R_3 &\to& R_1 \\
		R_4-R_3 &\to& R_4
	    \end{array}&
	    \left[\begin{array}{rrrrr|r}
		1 & 0 & -2 & 1 & 0 & 1 \\
		0 & 1 & 1 & -1 & 0 & 2 \\
		0 & 0 & 0 & 0 & 1 & 1 \\
		0 & 0 & 0 & 0 & 0 & 0
	    \end{array}\right]

	\end{array}$$

	Por lo tanto las soluciones estarán dadas por:
	$$\begin{array}{rcl}
	    x_1-2x_3+x_4=1&\Rightarrow &x_1=1+2x_3-x_4\\
	    x_2+x_3-x_4=2&\Rightarrow &x_2=2-x_3+x_4\\
	    x_5&=&1.
	\end{array}$$
	Sea $x_3=s\in \mathbb{R}$ y $x_4=t\in \mathbb{R}$ donde el conjunto de soluciones estará dado por:
	$$\left\{(1,2,0,0,1)+(2,-1,1,0,0)s+(-1,1,0,1,0)t\; | \; s,t\in \mathbb{R}\right\}.$$\\


    %-------------------- 8.
    \item Sea 
    $$\left[\begin{array}{rrr}
	3 & -1 & 2 \\
	2 & 1 & 1 \\
	1 & -3 & 0
    \end{array}\right]$$
    ¿Para cuáles ternas $(y_1,y_2,y_3)$ tiene una solución el sistema $AX=Y$?.\\\\
    	Respuesta.-\; Sea el sistema $AX=Y$ de donde,

	$$\begin{array}{ccl}

	    \left[\begin{array}{rrr|r}
		3 & -1 & 2 & y_1 \\
		2 & 1 & 1 & y_2 \\
		1 & -3 & 0 & y_3
	    \end{array}\right]
	    &R_1 \leftrightarrow R_3&
	    \left[\begin{array}{rrr|r}
		1 & -3 & 0 & y_3 \\
		2 & 1 & 1 & y_2 \\
		3 & -1 & 2 & y_1
	    \end{array}\right]\\\\
	    &\begin{array}{rcl}
		R_2-2R_1 &\to& R_2 \\
		R_3-3R_2 &\to& R_3
	    \end{array}&
	    \left[\begin{array}{rrr|r}
		1 & -3 & 0 & y_3 \\
		0 & 7 & 1 & y_2-2y_3 \\
		0 & 8 & 2 & y_1-3y_3
	    \end{array}\right]\\\\
	    &\frac{1}{7}R_2\to R_2&
	    \left[\begin{array}{rrr|r}
		1 & -3 & 0 & y_3 \\
		0 & 1 & \frac{1}{7} & \frac{y_2-2y_3}{7} \\
		0 & 8 & 2 & y_1-3y_3
	    \end{array}\right]\\\\
	    &\begin{array}{rcl}
		R_1+3R_2 &\to& R_1 \\
		R_3+8R_2 &\to& R_3
	    \end{array}&
	    \left[\begin{array}{rrr|r}
		1 & 0 & \frac{3}{7} & \frac{y_3+3y_2}{7} \\
		0 & 1 & \frac{1}{7} & \frac{y_2-2y_3}{7} \\
		0 & 0 & \frac{6}{7} & \frac{7y_1-8y_2-5y_3}{7}
	    \end{array}\right]\\\\
	    &\frac{7}{6}R_3 \to R_3&
	    \left[\begin{array}{rrr|r}
		1 & 0 & \frac{3}{7} & \frac{y_3+3y_2}{7} \\
		0 & 1 & \frac{1}{7} & \frac{y_2-2y_3}{7} \\
		0 & 0 & 1 & \frac{7y_1-8y_2-5y_3}{6}
	    \end{array}\right]\\\\

	\end{array}$$

	Por la forma reducida tenemos $Rango(A)=Rango(A/y)$, así el sistema tiene una única solución $y\in \mathbb{R}^3.$\\\\

    %-------------------- 9.
    \item 

\end{enumerate}


\section{Multiplicación de matrices}

    \begin{def.}
	Sea $A$ una matriz $m\times n$ sobre el cuerpo $F$ y sea $B$ una matriz $n\times p$ sobre $F$. El producto $AB$ es la matriz $m\times p$, $C$, cuyos elementos $i$, $j$ son
	$$C_{ij} = \sum_{r=1}^n A_{ir}B_{rj}.$$
    \end{def.}

El producto está definido si, y sólo si, el número de columnas de la primera matriz coincide con el número de filas de la segunda.\\

\begin{teo}
    Si $A,B, C$ son matrices sobre el cuerpo $F$, tales que los productos $BC$ y $A(BC)$ están definidos, entonces también lo están los productos $AB$, $(AB)C$ y 
    $$A(BC)=(AB)C.$$
	Demostración.-\; Supóngase que $B$ es una matriz $n\times p$. Como $BC$ está definida, $C$ es una matriz con $p$ filas y $BC$ tiene $n$ filas. Como $A(BC)$ está definida, se puede suponer que $A$ es una matriz $m\times n$. Así el producto $AB$ existe y es una matriz $m\times p$, de lo que se sigue que el producto $(AB)C$ existe. para ver que $A(BC)=(AB)C$ se debe demostrar que
	$$\left[A(BC)\right]_{ij}=\left[(AB)C\right]_{ij}$$
	Para todo los $i,j$. Por definición
	$$\begin{array}{rclcl}
	    \left[A(BC)\right]_{ij} &=& \displaystyle\sum_{r} A_{ir}(BC)_{rj} &=& \displaystyle\sum_{r} A_{ir}\sum_{s} B_{rs}C_{sj}\\\\
				    &=& \displaystyle\sum_{s}\sum_{r} A_{ir}B_{rs}C_{sj} &=& \displaystyle\sum_{r}\sum_{s} A_{ir}B_{rs}C_{sj}\\\\
				    &=& \displaystyle\sum_{r}\left(\sum_{s} AB_{ir}\right)C_{sj} &=& \displaystyle\sum_{r} (AB)_{is}C_{sj}\\\\
				    &=& \left[(AB)C\right]_{ij} &&\\\\
	\end{array}$$
\end{teo}

Si $A$ es una matriz cuadrada $n\times n$, el producto $AA$ está definido. Esta matriz se representa por $A^2$. En general el producto $AA\cdots A$ ($k$ veces) está definido y se representará por $A^k$.

    \begin{def.}
	Una matriz $m\times m$ se dice matriz elemental si se puede obtener de la matriz identidad $m\times m$ por medio de una sola operación elemental simple por filas.
    \end{def.}

\begin{teo}
    Sea $e$ una operación elemental de fila y sea $E$ la matriz elemental $m\times m$, $E=e(I)$. Entonces para toda matriz $m\times n$, $A$
    $$e(A)=EA.$$\\
	Demostración.-\; La clave de la demostración radica en que el elemento de la i-ésima fila y la j-ésima columna de la matriz producto $EA$ se obtiene de la i-ésima fila de $E$ y de la j-ésima columna de $A$. Los tres tipos de operaciones elementales de fila deben ser estudiados separadamente. Se dará una demostración detallada para una operación tipo (ii) . Los otros dos casos, ,más fáciles de estudiar se dejan como ejercicios. Supóngase que $r\neq s$ y que $e$ es una operación que remplaza la fila $r$ por la fila $r$ más $c$ veces la fila $s$. Entonces
	$$E_{ik}=\left\{\begin{array}{ll}
		\delta_{ik}, & i\neq r\\
		\delta_{rk} + c\delta sk, & i\neq r.
    \end{array}\right.$$
    Luego
    $$ (EA)_{ij}=\sum_{k=1}^m E_{ik}A_{kj} = \left\{\begin{array}{ll}
	    		A_{ij}, & i\neq r\\
			A_{ij} + cA_{sj}, & i=r.
    \end{array}\right.$$

    Es decir, $EA=e(A).$\\

\end{teo}

\begin{cor}
    Sean $A$ y $B$ dos matrices $m\times n$ sobre el cuerpo $F$. Entonces $B$ es equivalente por filas a $A$ si, y sólo si, $B=PA$, donde $P$ es un producto de matrices elementales $m\times m$.\\\\
	Demostración.-\; Supóngase que $B=PA$, donde $P=E_s\cdots E_2E_1$ y los $E_i$ son matrices elementales $m\times m$. Entonces $E_iA$ es equivalente por filas a $A_1$ y $E_2(E_1A)$ es equivalente por filas a $E_1A$. Luego $E_2E_1A$ es equivalente por filas a $A$. Sean $E_1,E_2,\ldots E_s$a matrices elementales correspondientes a cierta sucesión de operaciones elementales de filas que lleva $A$ a $B$. Entonces $B=(E_s,\cdots , E_1)A.$
\end{cor}


\section{Ejercicios}

\begin{enumerate}[1.]

    %-------------------- 1.
    \item 

    %-------------------- 2.
    \item 

    %-------------------- 3.
    \item Encontrar dos matrices $2\times 2,A$ diferentes tales que $A^2=0$ pero $A\neq 0.$\\\\
	Respuesta.-\;

    %-------------------- 4.
    \item  Para cada $A$ del ejercicio 2, hallar matrices elementales $E_1,E_2,\ldots,E_k$ tal que 
	$$E_k\cdot E_2E_1A=1.$$\\
    Respuesta.-\;

    %-------------------- 5.
    \item 

    %-------------------- 6.
    \item 

    %-------------------- 7.
    \item Sean $A$ y $B$ matrices $2\times 2$ tales que $AB=I$. Demostrar que $BA=I.$\\\\
	Demostración.-\;

    %-------------------- 8.
    \item 

\end{enumerate}


\section{Matrices inversibles}
