\chapter{Espacios vectoriales}

\section{Espacios vectoriales}

%-------------------- Definición 1 espacio vectorial --------------------
\begin{def.}
    Un \textbf{espacio vectorial} (o espacio lineal) consta de lo siguiente:
    \begin{enumerate}[1.]
	\item Un cuerpo $F$ de escalares;
	\item un conjunto $V$ de objetos llamados vectores;
	\item una regla (u operación) llamada adición, que asocia a cada par de vectores $\alpha,\beta $ de $V$ un vector $\alpha+\beta$ de $V$, que se llama suma de $\alpha$ y $\beta$, de tal modo que:
	   \begin{enumerate}[(a)]
	       \item La adición es conmutativa, $\alpha+\beta=\beta+\alpha$;
	       \item la adición es asociativa, $\alpha+(\beta+\gamma)=(\alpha+\beta)+\gamma$;
	       \item existe un único vector $0$ de $V$, llamado vector nulo tal que $\alpha+0=\alpha,$ para todo $\alpha$ de $V$;
	       \item para cada vector $\alpha$ de $V$ existe un vector $-\alpha$ de $V$, tal que $\alpha+(-\alpha)=0$;
	   \end{enumerate}
       \item una regla (u operación) llamada multiplicación escalar, que asocia a cada escalar $c$ de $F$ y cada vector $\alpha$ de $V$ a un vector $c\alpha$ en $V$, llamado producto de $c$ y $\alpha$, de tal modo que:
	   \begin{enumerate}[(a)]
	       \item $1\alpha=\alpha$ para todo $\alpha$ de $V$;
	       \item $(c_1c_2)\alpha=c_1(c_2\alpha)$;
	       \item $c(\alpha+\beta)=c\alpha+c\beta$;
	       \item $(c_1+c_2)\alpha=c_1\alpha+c_2\alpha$.
	   \end{enumerate}
    \end{enumerate}
\end{def.}


%-------------------- ejemplo 1
\begin{ejem}[El espacio de n-tuplas, \boldmath $F_n$]
    Sea $F$ cualquier cuerpo y sea $V$ el conjunto de todos los n-tuples $\alpha=(x_1,x_2,\ldots,x_n)$ de escalares $x_i$ de $F$. Si $\beta = (y_1,y_2,\ldots , y_n)$ con $y_i$ de $F$, la suma de $\alpha$ y $\beta$ se define por
    \begin{equation}
	\alpha+\beta = (x_1+y_1,x_2+y_2,\ldots,x_n+y_n).
    \end{equation}
    El producto de un escalar $c$ y el vector $\alpha$ se define por 
    \begin{equation}
	c\alpha = (cx_1,cx_2,\ldots,cx_n).
    \end{equation}
    Que esta adición vectorial y multiplicación escalar cumplen las condiciones (3) y (4) es fácil de verificar, usando las propiedades semejantes de la adición y multiplicación de elementos de $F$.
\end{ejem}

%-------------------- ejemplo 2
\begin{ejem}[El espacio de matrices \boldmath $m\times n, F^{m\times n}$] Sea F cualquier cuerpo y sean $m$ y $n$ enteros positivos. Sea $F^{m\times n}$ el conjunto de todas las matrices $m\times n$ sobre el cuerpo $F$. La suma de dos vectores $A$ y $B$ en $F^{m\times n}$ se define por 
    \begin{equation}
	(A+B)_{ij}=A_{ij}+B_{ij}.
    \end{equation}
    El producto de un escalar $c$ y de la matriz $A$ se define por 
    \begin{equation}
	(cA)_{ij} = cA_{ij}.
    \end{equation}
    Obsérvece que $F^{i\times n}=F^n$.
\end{ejem}

%------------------- ejemplo 3
\begin{ejem}[El espacio de funciones de un conjunto en un cuerpo]
    Sea $F$ cualquier cuerpo y sea $S$ cualquier conjunto no vacío. Sea $V$ el conjunto de todas las funciones de $S$ en $F$. La suma de dos vectores $f$ y $g$ de $V$ es el vector $f+g$; es decir, la función de $S$ en $F$ defina por
    \begin{equation}
	(f+g)(s)=f(s)+g(s).
    \end{equation}
    El producto del escalar $c$ y la función $f$ es la función $cf$ definida por
    \begin{equation}
	(cf)(s)=cf(s).
    \end{equation}
    Para este tercer ejemplo se indica cómo se puede verificar que las operaciones definidas satisfacen las condiciones (3) y (4). Para la adición vectorial:
    \begin{enumerate}[(a)]
	\item Como la adición en $F$ es conmutativa,
	    $$f(s)+g(s)=g(s)+f(s)$$
	    para todo $s$ de $S$, luego las funciones $f+g$ y $g+f$ son idénticas.
	\item Como la adición en $F$ es asociativa,
	    $$f(s)+[g(s)+h(s)]=[f(s)+g(s)]+h(s)$$
	    para todo $s$, luego $f+(g+h)$ es la misma función que $(f+g)+h.$
	\item El único vector nulo es la función cero, que asigna a cada elemento de $S$ el escalar $0$ de $F$.
	\item Para todo $f$ de $V$, $(-f)$ es la función dada por
	    $$(-f)=-f(s).$$
    \end{enumerate}
    El lector encontrará fácil verificar que la multiplicación escalar satisface las condiciones de (4), razonando como se hizo para la adición vectorial.

\end{ejem}

%-------------------- ejemplo 4.
\begin{ejem}[El espacio de las funciones polinomios sobre el cuerpo $F$] Sea $F$ un cuerpo y sea $V$ el conjunto de todas las funciones $f$ de $F$ en $F$ definidas en la forma
    \begin{equation}
	f(x)=c_0+c_1x+\ldots + c_n x^n
    \end{equation}
    donde $c_0,c_1,\ldots , c_n$ son escalares fijos de $F$ (independiente de $x$). Una función de este tipo se llama \textbf{\boldmath función polinomio sobre $F$}. Sean la adición y la multiplicación escalar definidas sobre en el ejemplo 3. Se debe observar que si $f$ y $g$ son funciones polinomios y $c$ está en $F$, entonces $f+g$ y $cf$ son también funciones polinomios.
\end{ejem}

%-------------------- ejemplo 5
\begin{ejem}
    El cuerpo $C$ de los números complejos puede considerarse como un espacio vectorial sobre el cuerpo $R$ de los números reales. En forma más general, sea $F$ el cuerpo de los números reales y sea $V$ el conjunto de los n-tuples $\alpha=(x_1,\ldots,x_n)$ donde $x_1,\ldots , x_n$ son números complejos. Se define la adición vectorial y la multiplicación escalar por (2.1) y (2-2), como en el ejemplo 1. De este modo se obtiene un espacio vectorial sobre el cuerpo $R$ que es muy diferente del espacio $C^n$ y del espacio $R_n$.
\end{ejem}

\vspace{.5cm}

Hay unos pocos hechos simples que se desprenden, casi inmediatamente, de la definición de espacio vectorial, y procederemos a derivarlos. Si $c$ es un escalar y $0$ es el vector nulo, entonces por 3(c) y 4(c)
$$c0=c(0+0)=c0+c0.$$
Sumando $-(c0)$ y por $3(d)$, se obtiene
\begin{equation}
    c0=0.
\end{equation}

Análogamente, para el escalar $0$ y cualquier vector $\alpha$ se tiene que

\begin{equation}
    0\alpha=0.
\end{equation}

Si $c$ es un escalar no nulo y $\alpha$ un vector tal que $c\alpha=0$, entonces por (2-8), $c^{-1}(c\alpha)=0.$ Pero
$$c^{-1}(c\alpha)=(c^{-1}c)\alpha=1\alpha=\alpha$$
luego, $\alpha=0$. Así se ve que si $c$ es un escalar y $\alpha$ un vector tal que $c\alpha=0$, entonces $c$ es el escalar cero o $\alpha$ es el vector nulo.\\

Si $\alpha$ es cualquier vector de $V$, entonces
$$0=0\alpha=(1-1)\alpha=1\alpha+(-1)\alpha=\alpha+(-1)\alpha$$
de lo que se sigue que

\begin{equation}
    (-1)\alpha=-\alpha.
\end{equation}

Finalmente, las propiedades asociativa y conmutativa de la adición vectorial implican que la suma de varios vectores es independiente de cómo se combinen estos vectores y de cómo se asocien. Por ejemplo, si $\alpha_1,\alpha_2,\alpha_3,\alpha_4$ son vectores de $V$, entonces
$$(\alpha_1,\alpha_2)+(\alpha_3+\alpha_4)=\left[\alpha_2+(\alpha_1+\alpha_3)\right]+\alpha_4$$
y tal suma puede ser escrita, sin confusión,

$$\alpha_1+\alpha_2+\alpha_3+\alpha_4.$$

%-------------------- Definición combinación lineal
\begin{def.}
    Un vector $\beta$ de $V$ se dice \textbf{combinación lineal} de los vectores $\alpha_1,\ldots , \alpha_n$ en $V$, si existen escalares $c_1,\ldots , c_n$ de $F$ tales que
    $$\beta = c_1\alpha_1 + \ldots + c_n\alpha_n = \sum_{i=1}^n c_i\alpha_i.$$
\end{def.}

Otras extensiones de la propiedad asociativa de la adición vectorial y las propiedades distributivas 4(c) y 4(d) de la multiplicación escalar se aplican a las combinaciones lineales:
$$\sum_{i=1}^n c_i\alpha_i + \sum_{i=1}^n d_i\alpha_i=\sum_{i=1}^n (c_i+d_i)\alpha_i$$
$$c\sum_{i=1}^n c_i\alpha_i = \sum_{i=1}^n (cc_i)\alpha_i.$$

\section*{Ejercicios}

\begin{enumerate}[\bfseries 1.]

    %-------------------- 1.
    \item Si $F$ es un cuerpo, verificar que $F^n$ (como se definio en el Ejemplo 1) es un espacio vectorial sobre el cuerpo $F$.\\\\
	Respuesta.-\; Sean $\alpha=(x_1,x_2,\ldots,x_n)$,  $\beta=(y_1,y_2,\ldots,y_n)$ y $\gamma=(z_1,z_2,\ldots,z_n)$ elementos de $F^n$. Como también sean $c,d,c_1,c_2\in F$. Entonces,

	\begin{enumerate}[(1)]
	    \setcounter{enumii}{2}
	    \item 
		\begin{enumerate}[(a)]
		    \item Conmutatividad para la adición. 
			$$
			\begin{array}{rcl}
			    \alpha+\beta & = & (x_1,x_2,\ldots,x_n)+(y_1,y_2,\ldots, y_n) \\
			    & = & (x_1+x_2,\ldots,x_n+y_n)\\
			    & = & (y_1+y_2,\ldots,y_n+x_n)\\
			    & = & \beta+\alpha.
			\end{array}
			$$
		    \item Asociatividad para la adición.
			$$
			\begin{array}{rcl}
			    \alpha+(\beta+\gamma) & = & (x_1,x_2,\ldots,x_n)+(y_1,y_2,\ldots, y_n)+(z_1,z_2,\ldots,z_n) \\
			    & = & (x_1+x_2,\ldots,x_n+y_n+z_n)\\
			    & = & (x_1+x_2+z_1,\ldots,x_n+y_n+z_n)\\
			    & = & (\alpha+\beta)+\gamma.
			\end{array}
			$$
		    \item Existencia del elemento nulo.\\
			$$
			\begin{array}{rcl}
			    \alpha+0 & = & (x_1,x_2,\ldots,x_n)+(0,0,\ldots,0)\\
			    & = & (x_1+0,\ldots,x_n+0)\\
			    & = & \alpha.
			\end{array}
			$$
		    \item Existencia del inverso aditivo.
			$$
			\begin{array}{rcl}
			    \alpha+(-\alpha) & = & (x_1,x_2,\ldots,x_n)+\left[-(x_1,x_2,\ldots,x_n)\right]\\
			    & = & (x_1,x_2,\ldots,x_n)+(-x_1,-x_2,\ldots,-x_n)\\
			    & = & (x_1-x_1,\ldots,x_n-x_n)\\
			    & = & (0,0,\ldots,0)\\
			    & = & 0.
			\end{array}
			$$

		\end{enumerate}

	    \item 
		\begin{enumerate}[(a)]
		    \item Existencia del elemento neutro para la multiplicación escalar.
			$$
			\begin{array}{rcl}
			    1\alpha & = & 1(x_1,x_2,\ldots,x_n)\\
			    & = & (1x_1,1x_2,\ldots,1x_n)\\
			    & = & \alpha.
			\end{array}
			$$
		    \item Asociatividad para la multiplicación escalar.\\
			$$
			\begin{array}{rcl}
			    (c_1c_2)\alpha & = & (c_1c_2)(x_1,x_2,\ldots,x_n)\\
			    & = & (c_1c_2x_1,c_1c_2x_2,\ldots,c_1c_2x_n)\\
			    & = & (c_1(c_2x_1),c_1(c_2x_2),\ldots,c_1(c_2x_n))\\
			    & = & (c_1c_2x_1,c_1c_2x_2,\ldots,c_1c_2x_n)\\
			    & = & c_1(c_2\alpha).
			\end{array}
			$$

		    \item Distributividad para la multiplicación escalar sobre la adición.
			$$
			\begin{array}{rcl}
			    c(\alpha+\beta) & = & c((x_1,x_2,\ldots,x_n)+(y_1,y_2,\ldots, y_n))\\
			    & = & c(x_1+x_2,\ldots,x_n+y_n)\\
			    & = & (cx_1+cx_2,\ldots,cx_n+cy_n)\\
			    & = & (c\alpha+c\beta).
			\end{array}
			$$

		    \item Distributividad para la multiplicación sobre la adición de escalares
			$$
			\begin{array}{rcl}
			    (c+d)\alpha & = & (c+d)(x_1,x_2,\ldots,x_n)\\
			    & = & (cx_1+dx_1,\ldots,cx_n+dx_n)\\
			    & = & c\alpha+d\alpha.
			\end{array}
			$$

		\end{enumerate}
		\vspace{.5cm}

	\end{enumerate}

    %-------------------- 2.
    \item Si $V$ es un espacio vectorial sobre un cuerpo $F$, verificar que
    $$(\alpha_1+\alpha_2)+(\alpha_3+\alpha_4)=\left[\alpha_2+(\alpha_3+\alpha_1)\right]+\alpha_4$$
    para todo los vectores $\alpha_1,\alpha_2,\alpha_3,\alpha_4$ de $v$.\\\\
	Respuesta.-\; Se tiene,
	$$
	\begin{array}{rcl}
	    (\alpha_1+\alpha_2)+(\alpha_3+\alpha_4) &=& (\alpha_2+\alpha_1)+(\alpha_3+\alpha_4)\\
						    &=& \alpha_2+\left[\alpha_1+(\alpha_3+\alpha_4)\right]\\
						    &=& \alpha_2+\left[(\alpha_1+\alpha_3)+\alpha_4\right]\\
						    &=& \alpha_2+\left[(\alpha_3+\alpha_1)+\alpha_4\right]\\
						    &=& \left[\alpha_2+(\alpha_3+\alpha_1)\right]+\alpha_4.
	\end{array}
	$$
	\vspace{.5cm}

    %-------------------- 3.
    \item Si $C$ es el cuerpo de los complejos, ¿qué vectores de $C^3$ son combinaciones lineales de $(1,0,-1),(0,1,1)$ y $(1,1,1)$?.\\\\
	Respuesta.-\; Sea $(x,y,z)\in C^3$ una convinación lineal de los vectores $(1,0,-1),(0,1,1)$ y $(1,1,1)$. Entonces, existen escalares $a,b$ y $c\in C$ tal que
	$$
	\begin{array}{rcl}
	    (x,y,z)&=&a(1,0,-1)+b(0,1,1)+c(1,1,1)\\
		   &=& (a+c,b+c,c-a).
	\end{array}
	$$
	De donde, 
	$$
	\left\{	
	    \begin{array}{rcl}
		a+c & = & x\\
		b+c & = & y\\
		c-a & = & z.
	    \end{array}
	\right.
	$$
	Resolviendo se tiene,
	$$
	\left\{	
	    \begin{array}{rcl}
		a & = & \dfrac{x-z}{2}\\\\
		b & = & \dfrac{2z-x-y}{2}\\\\
		c & = & \dfrac{x+z}{2}.
	    \end{array}
	\right.
	$$

	Por lo tanto, existen escalares $a,b$ y $c\in C$ tal que 
	$$(x,y,z)=a(1,0,-1)+b(0,1,1)+c(1,1,1).$$
	Así, todos los vectores en $C^3$ pueden ser expresados como una combinación lineal de los vectores $(1,0,-1),(0,1,1)$ y $(1,1,1)$.\\\\

    %-------------------- 4.
    \item Sea $V$ el conjunto de los pares $(x,y)$ de números reales, y sea $F$ el cuerpo de los números reales. Se define
    $$
    \begin{array}{rcl}
	(x,y)+(x_1,y_1) &=& (x+x_1,y+y_1)\\
	c(x,y) &=& (cx,y).
    \end{array}
    $$
    ¿Es $V$, con estas operaciones, un espacio vectorial sobre el cuerpo de los números reales?.\\\\
	Respuesta.-\; No es un espacio vectorial ya que, 
	$$(0,2)=(0,1)+(0,1)=2(0,1)=(2\cdot 0,1)=(0,1).$$\\

    %-------------------- 5.
    \item En $\mathbb{R}^n$ se definen dos operaciones
    $$\alpha \oplus \beta = \alpha-\beta$$
    $$c\cdot \alpha=-c\alpha.$$
    Las operaciones del segundo miembro son las usuales. ¿Qué axiomsa de espacio vectorial se cumplen para $\mathbb{R}^n,\oplus,\cdot$?.\\\\
	Respuesta.-\; Sean $\alpha=(x_1,x_2,\ldots,x_n)$,  $\beta=(y_1,y_2,\ldots,y_n)$ y $\gamma=(z_1,z_2,\ldots,z_n)$ elementos de $F^n$. Como también sean $c,d,c_1,c_2\in F$. Entonces,
	\begin{enumerate}[(1)]
	    \setcounter{enumii}{2}
	    \item 
		\begin{enumerate}[(a)]
		    \item No es conmutativa para la adición. 
			$$
			\begin{array}{rcl}
			    \alpha\oplus \beta &=& \alpha-\beta\\
					       & = & (x_1,x_2,\ldots,x_n)-(y_1,y_2,\ldots, y_n) \\
							      &=& (x_1-y_1,x_2-y_2,\ldots,x_n-y_n)\\
							      &=& (-y_1+x_1,-y_2+x_2,\ldots,-y_n+x_n)\\
							      &=& -\beta+ \alpha.\\
							      &=& -(\beta-\alpha)\\
							      &=& -(\beta\oplus \alpha)\\
							      &\neq& \beta\oplus \alpha.
			\end{array}
			$$

		    \item No es asociativa para la adición.
			$$
			\begin{array}{rcl}
			    \alpha\oplus (\beta\oplus \gamma) &=& \alpha-(\beta-\gamma)\\
							      &=& (\alpha-\beta)+\gamma \\
							      &=& (\alpha\oplus \beta)+\gamma\\
							      &\neq& \alpha\oplus (\beta\oplus \gamma).

			\end{array}
			$$
		    \item No existe el elemento nulo.\\
			$$
			\begin{array}{rcl}
			    \alpha\oplus 0 &=& \alpha-0\\
					   &=&\alpha-(0,0,\ldots,0)\\
					   &=& \alpha.
			\end{array}
			$$
			Pero, como $\oplus$ no es conmutativa; es decir, $\alpha\oplus \neq 0\oplus \alpha$ decimos que no existe la identidad aditiva para $\oplus.$\\
			
		    \item No existe el inverso aditivo.
			$$
			\begin{array}{rcl}
			    \alpha\oplus(-\alpha) & = & \alpha - (-\alpha)\\
						  & = & \alpha + \alpha\\
						  &\neq & 0
			\end{array}
			$$

		\end{enumerate}

	    \item 
		\begin{enumerate}[(a)]
		    \item No existe el elemento neutro para la multiplicación escalar.\\
			El elemento $1$ no satisface $1\cdot \alpha=\alpha$ para cualquier $\alpha\neq 0$, ya que de lo contrario tendríamos 
			$$1\cdot(x_1,x_2,\ldots,x_n)=(-x_1,x_2\ldots,x_n)=(x_1,x_2,\ldots,x_n)$$
			sólo si $x_i=0$ para todo $i$.\\

		    \item No es asociativa para la multiplicación escalar.\\
			$$
			\begin{array}{rcl}
			    c_1(c_2\alpha) & = & c_1(-c_2\alpha)\\
					   & = & -c_1(-c_2\alpha)\\
					   & = & -(c_1c_2)\alpha\\
					   & \neq & (c_1c_2)\alpha.
			\end{array}
			$$

		    \item No es distributiva para la multiplicación escalar sobre la adición.
			$$
			\begin{array}{rcl}
			c(\alpha+\beta) & = & -c(\alpha+\beta)\\
					&=& -c\alpha-c\beta\\
					&=& -(c\alpha+c\beta)\\
					&\neq& c\alpha+c\beta.
			\end{array}
			$$

		    \item No es distributiva para la multiplicación sobre la adición de escalares
			$$
			\begin{array}{rcl}
			    c\alpha + d\alpha & = & -c\alpha - d\alpha\\
					      & = & -(c+d)\alpha\\
					      & \neq & (c+d)\alpha.
			\end{array}
			$$

		\end{enumerate}
		\vspace{.5cm}

	\end{enumerate}

    %-------------------- 6.
    \item Sea $V$ el conjunto de todas las funciones que tiene valor complejo sobre el eje real, tales que (para todo $t$ de $R$)
    $$f-(t)=\overline{f(t)}$$.
    La barra indica conjugación compleja. Demostrar que $V$, con las operaciones 
    $$(f+g)(t)=f(t)+g(t)$$
    $$cf(t)=cf(t)$$
    es un espacio vectorial sobre el cuerpo de los números reales. Dar un ejemplo de una función en $V$ que no toma valores reales.\\\\
    Demostración.-\; Sea $f,g,h\in V$. Entonces, para todo $t\in \mathbb{R},$ se tiene

	\begin{enumerate}[(1)]
	    \setcounter{enumii}{2}
	    \item 
		\begin{enumerate}[(a)]
		    \item Conmutatividad para la adición. 
			$$
			\begin{array}{rcl}
			    (f+g)(t) & = & f(t)+g(t)\\
				     & = & g(t) + f(t)\\
				     & = & (g+f)(t).
			\end{array}
			$$
			Por lo tanto, $f+g=g+f$ para todo $f$ y $g\in V$.\\

		    \item Asociatividad para la adición.
			$$
			\begin{array}{rcl}
			    \left[(f+g)+h\right](t) &=& (f+g)(t)+h(t)\\
						    &=& \left[f(t)+g(t)\right]+h(t)\\
						    &=& f(t)+\left[g(t)+h(t)\right].
			\end{array}
			$$
			Por lo tanto, $(f+g)+h=f+(g+h)$ para todo $f,g,h\in V$.\\

		    \item Existencia del elemento nulo.\\
			Considere la función cero $0(t)=0$ para todo $t\in \mathbb{R}$, entonces para todo $f\in V$, tenemos
			$$(f+0)(t)=f(t)+0(t)=f(t)+0=f(t).$$
			Ya que $+$ es conmutativo, se tiene $f=f=0+f.$\\
			
		    \item Existencia del inverso aditivo.\\
			Para $f\in V$, consideremos $g=(-f)$ como $(-f)(t)=-f(t)$. Claramente $g=-f$ existe en $V$. Luego,
			$$\left[f+(-f)\right]=f(t)+(-f)(t)=f(t)-f(t)=0=0(t).$$
			Ya que, $+$ es conmutativo, tenemos $f+(-f)=0=(-f)+f$ para todo $f\in V$. así, el inverso aditivo existe.\\

		\end{enumerate}

	    \item 
		\begin{enumerate}[(a)]
		    \item Existencia del elemento neutro para la multiplicación escalar.\\
			$$1\cdot f =f.$$
			Para $f\in V$, se tiene
			$$(1\cdot f)(t)=1\cdot f(t)=f(t).$$
			Así, $1\cdot f = f$ para todo $f\in V$.\\

		    \item Asociatividad para la multiplicación escalar.\\
			Sean $a,b\in R$ y $f\in V$, entonces
			$$
			\begin{array}{rcl}
			    (ab)f&=& \left[(ab)\cdot f\right](t)\\
				 &=& (ab) f(t)\\
				 &=& a\left[bf(t)\right]\\
				 &=& a\left(b\cdot f\right)\\
				 &=& a\left(bf\right).
			\end{array}
			$$
			Por lo tanto, $(ab)\cdot f=a\left(b\cdot f\right)$ para todo $f\in V$ y $a,b\in R$.\\

		    \item Distributividad para la multiplicación escalar sobre la adición.\\
			Sean $a,b\in R$ y $f,g\in V$, entonces
			$$
			\begin{array}{rcl}
			    \left[a\left(f+g\right)\right](t) &=& a\left[(f+g)(t)\right]\\
							      &=& a\left[f(t)+g(t)\right]\\
							      &=& \left(af\right)(t)+\left(ag\right)(t)\\
			\end{array}
			$$

		    \item Distributividad para la multiplicación sobre la adición de escalares\\
			Sean $a,b\in R$ y $f\in V$, entonces
			$$
			\begin{array}{rcl}
			    \left[(a+b)f\right](t)&=& (a+b)(f(t)\\
						  &=& af(t)+bf(t)\\
						  &=& (af)(t)+(bf)(t).
			\end{array}
			$$

		\end{enumerate}
		De esta manera, $V$ satiface todos las propiedades del espacio vectorial respecto a las operaciones de adición y multiplicación escalar.\\\\

	\end{enumerate}

    %-------------------- 7.
    \item Sea $V$ el conjunto de pares $(x,y)$ de números reales y sea $F$ el cuerpo de los números reales. Se define
    $$(x,y)+(x_1,y_1)=(x+x_1,0)$$
    $$c(x,y)=(cx,0).$$
    ¿Es $V$, con estas operaciones un espacio vectorial?.\\\\
	Respuesta.-\; No es un espacio vectorial. Sea $u=(x_1,y_1)$ y  $0\in R$, $0=(0,0)\in V$. Entonces,
	$$
	\begin{array}{rcl}
	    u+0&=&(x_1,y_1)+(0,0)\\
	      &=&(x_1+0,0)\\
	      &=&(x_1,0)\\
	      &\neq& u.
	\end{array}
	$$
	Por lo tanto, no existe un inverso aditivo para $V$. Así $V$ no es un espacio vectorial.

\end{enumerate}


\section{Subespacios}

%-------------------- Definición 1.1
\begin{def.}
    Sea $V$ un espacio vectorial sobre el cuerpo $F$. Un \textbf{subespacio} de $V$ es un subconjunto $W$ de $V$ que, con las operaciones de adición vectorial y multiplicación escalar sobre $V$, es el mismo un espacio vectorial sobre $F$.
\end{def.}

Esta definición se puede simplificar aún más.

%-------------------- Teorema 1.1
\begin{teo}
    Un subconjunto no vacío $W$ de $V$ es un subespacio de $V$ si, y sólo si, para todo par de vectores $\alpha,\beta$ de $W$ y todo escalar $c$ de $F$, el vector $c\alpha+\beta$ está en $W$.\\\\
	Demostración.-\; Supóngase que $W$ sea un subconjunto no vacío de $V$ tal que $c\alpha+\beta$ pertenezca a $W$ para todos los vectores $\alpha,\beta$ de $W$ y todos los escalares $c$ de $F$. Como $W$ no es vacío, existe un vector $\rho$ en $W$, y por tanto, $(-1)\rho+\rho=0$ está en $W$. Ahora bien, si $\alpha$ es cualquier vector de $W$ y $c$ cualquier escalar, el vector $c\alpha=c\alpha+0$ está en $W$. En particular, $(-1)\alpha=-\alpha$ está en $W$. Finalmente, si $\alpha+\beta$ están en $W$, entonces $\alpha+\beta=1\alpha+\beta$ está en $W$. Así, $W$ es un subespacio de $V$.\\
	Recíprocamente, si $W$ es un subespacio de $V$, $\alpha$ y $\beta$ están en $W$ y $c$ es un escalar, ciertamente $c\alpha+\beta$ está en $W$.
\end{teo}

\setcounter{ejem}{6}
%-------------------- Ejemplo 2.7
\begin{ejem}[El espacio solución de un sistema homogéneo de ecuaciones lineales.]
Sea $A$ un matriz $m\times n$ sobre $F$. Entonces el conjunto de todas las matrices (columna) $n\times 1$, $X$, sobre $F$ tal que $AX=0$ es un subespacio del espacio de todas las matrices $n\times 1$ sobre $F$. Para demostrar esto se necesita probar que $A(cX+Y)=0$ si $AX=0,$ $AY=0$ y $c$ un escalar arbitrario  de $F$. Esto se desprende inmediatamente del siguiente hecho general.
\end{ejem}

%-------------------- Lema 2.1
\begin{lema}
    Si $A$ es una matriz $m\times n$ sobre $F$, y $B$, $C$ son matrices $n\times p$ sobre $F$, entonces
    \begin{equation}
	A(dB+C)=d(AB)+AC
    \end{equation}
    para todo escalar $d$ de $F$.\\\\
	Demostración.-\;
	$$
	\begin{array}{rcl}
	    \left[A(dB+C)\right]_{ij}&=&\displaystyle\sum_k A_{ik}(dB+C)_{kj}\\\\
				     &=&\displaystyle\sum_k \left(dA_{ik}B_{kj}+A_{ik}C_{kj}\right)\\\\
				     &=& d\displaystyle\sum_k A_{ik}B_{kj} + \sum_{k} A_{ik} C_{kj}\\\\
				     &=& d(AB)_{ij} + (AC)_{ij}\\\\
				     &=& \left[d(AB) + AC\right]_{ij}.
	\end{array}
	$$
\end{lema}

En forma semejante se puede ver que $(dB+C)A=d(BA)+CA$, si la suma y el producto de las matrices están definidos.

%-------------------- Teorema 2.2
\begin{teo}
    Sea $V$ un espacio vectorial sobre el cuerpo $F$. La intersección de cualquier colección de subespacios de $V$ es un subespacio de $V$.\\\\
	Demostración.-\; Sea $\left\{W_a\right\}$ una colección de subespacios de $V$, y sea $W=\cap W_a$ su intersección. Recuérdese que $W$ está definido como el conjunto de todos los elementos pertenecientes a cada $W_a$. Dado que todo $W_a$ es un subespacio, cada uno contiene el vector nulo. Así que el vector nulo está en la intersección $W$, y $W$ no es vacío. Sean $\alpha$ y $\beta$ vectores de $W$ y sea $c$ un escalar. Por definición de $W$ ambos. $\alpha$ y $\beta$ pertenecen a cada $W_a$, y por ser cada $W_a$ un subespacio el vector $(c\alpha+\beta)$ está en cada $W_a$. Así $(c\alpha+\beta)$ está también en $W$. Por el teorema 1, $W$ es un subespacio de $V$. 
\end{teo}

De este teorema se deduce que si $S$ es cualquier colección de vectores de $V$, entonces existe un subespacio mínimo de $V$ que contiene a $S$; esto es, un subespacio que contiene a $S$ y que está contenido en cada uno de los otros subespacios que contienen a $S$.


%-------------------- Definición 2.4
\begin{def.}
    Sea $S$ un conjunto de vectores de un espacio vectorial $V$, El \textbf{subespacio generado} por $S$ se define como la intersección $W$ de todos los subespacios de $V$ que contienen a $S$. Cuando $S$ es un conjunto finito de vectores, $S=\left\{\alpha_1,\alpha_2,\ldots,\alpha_n\right\}$ se dice simplemente que $W$ es el \textbf{subespacio generado por los vectores} $\alpha_1,\alpha_2,\ldots,\alpha_n$.
\end{def.}

%-------------------- Teorema 2.3
\begin{teo}
    El subespacio generado por un subconjunto $S$ no vacío de un espacio vectorial $V$ es el conjunto de todas las combinaciones lineales de los vectores de $S$.\\\\
	Demostración.-\; Sea $W$ el subespacio generado por $S$. Entonces, toda combinación lineal 
	$$\alpha=x_1\alpha_1+x_2\alpha_2+\ldots+x_m,\alpha_m$$
	de vectores $\alpha_1,\alpha_2,\ldots,\alpha_m$ de $S$ pertenecen evidentemente a $W$. Así que $W$ contiene el conjunto $L$ de todas las combinaciones lineales de vectores de $S$. El conjunto $L$, entonces, por otra parte, contiene a $S$ y no es, pues, vacío. Si $\alpha$ y $\beta$ pertencen a $L$, entonces $\alpha$ es una combinación lineal,
	$$\alpha=x_1\alpha_1+x_2\alpha_2+\ldots+x_m,\alpha_m$$
	de vectores $\alpha_i$ de $S$, y $\beta$ es una combinación lineal,
	$$\beta=y_1\alpha_1+y_2\alpha_2+\ldots+y_n,\alpha_n$$
	de vectores $\beta_j$ de $S$. Para cada escalar $c$,
	$$c\alpha+\beta=\sum_{i=1}^n (cx_i)\alpha_i+\sum_{j=1}^n y_j\beta_j.$$
	Luego, $c\alpha+\beta$ pertence a $L$. Con lo que $L$ es un subespacio de $V$.\\
	Se demostró que $L$ es un subespacio de $V$ que contiene a $S$, y también que todo subespacio que contiene a $S$ contiene a $L$. Se sigue que $L$ es la intersección de todos los subespacios que contienen a $S$; es decir, que $L$ es el subespacio generado por el conjunto $S$.
\end{teo}

%-------------------- Definición 2.5
\begin{def.}
    Si $S_1,S_2,\ldots,S_k$ son subconjuntos de un espacio vectorial $V$, el conjunto de todas las sumas 
    $$\alpha_1+\alpha_2+\cdots + \alpha_k$$
    de vectores $\alpha_i$ de $S_i$ se llama \textbf{suma} de los subconjuntos $S_1,S_2,\ldots , S_k$ y se representa por
    $$S_1+S_2+\cdots + S_k$$
    o por
    $$\sum_{i=1}^k S_i.$$
    Si $W_1,W_2,\ldots,W_k$ son subespacios de $V$, entonces la suma
    $$W=W_1+W_2+\cdots + W_k$$
    como es fácil ver, es un subespacio de $V$ que contiene cada uno de los subespacios $W_i$. De esto se sigue, como en la demostración del Teorema 3, que $W$ es el subespacio generado por la unión de $W_1,W_2,\ldots , W_k$.
\end{def.}


\section*{Ejercicios}

\begin{enumerate}[\bfseries 1.]

    %-------------------- 1.
    \item 

\end{enumerate}

