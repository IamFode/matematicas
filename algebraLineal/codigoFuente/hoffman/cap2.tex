\chapter{Espacios vectoriales}

\section{Espacios vectoriales}

%-------------------- Definición 1 espacio vectorial --------------------
\begin{def.}
    Un \textbf{espacio vectorial} (o espacio lineal) consta de lo siguiente:
    \begin{enumerate}[1.]
	\item Un cuerpo $F$ de escalares;
	\item un conjunto $V$ de objetos llamados vectores;
	\item una regla (u operación) llamada adición, que asocia a cada par de vectores $\alpha,\beta $ de $V$ un vector $\alpha+\beta$ de $V$, que se llama suma de $\alpha$ y $\beta$, de tal modo que:
	   \begin{enumerate}[(a)]
	       \item La adición es conmutativa, $\alpha+\beta=\beta+\alpha$;
	       \item la adición es asociativa, $\alpha+(\beta+\gamma)=(\alpha+\beta)+\gamma$;
	       \item existe un único vector $0$ de $V$, llamado vector nulo tal que $\alpha+0=\alpha,$ para todo $\alpha$ de $V$;
	       \item para cada vector $\alpha$ de $V$ existe un vector $-\alpha$ de $V$, tal que $\alpha+(-\alpha)=0$;
	   \end{enumerate}
       \item una regla (u operación) llamada multiplicación escalar, que asocia a cada escalar $c$ de $F$ y cada vector $\alpha$ de $V$ a un vector $c\alpha$ en $V$, llamado producto de $c$ y $\alpha$, de tal modo que:
	   \begin{enumerate}[(a)]
	       \item $1\alpha=\alpha$ para todo $\alpha$ de $V$;
	       \item $(c_1c_2)\alpha=c_1(c_2\alpha)$;
	       \item $c(\alpha+\beta)=c\alpha+c\beta$;
	       \item $(c_1+c_2)\alpha=c_1\alpha+c_2\alpha$.
	   \end{enumerate}
    \end{enumerate}
\end{def.}


%-------------------- ejemplo 1
\begin{ejem}[El espacio de n-tuplas, \boldmath F_n]
    Sea $F$ cualquier cuerpo y sea $V$ el conjunto de todos los n-tuples $\alpha=(x_1,x_2,\ldots,x_n)$ de escalares $x_i$ de $F$. Si $\beta = (y_1,y_2,\ldots , y_n)$ con $y_i$ de $F$, la suma de $\alpha$ y $\beta$ se define por
    \begin{equation}
	\alpha+\beta = (x_1+y_1,x_2+y_2,\ldots,x_n+y_n).
    \end{equation}
    El producto de un escalar $c$ y el vector $\alpha$ se define por 
    \begin{equation}
	c\alpha = (c_x1,c_x_2,\ldots,cx_n).
    \end{equation}
    Que esta adición vectorial y multiplicación escalar cumplen las condiciones (3) y (4) es fácil de verificar, usando las propiedades semejantes de la adición y multiplicación de elementos de $F$.
\end{ejem}

%-------------------- ejemplo 2
\begin{ejem}[El espacio de matrices \boldmath $m\times n, F^{m\times n}$] Sea F cualquier cuerpo y sean $m$ y $n$ enteros positivos. Sea $F^{m\times n}$ el conjunto de todas las matrices $m\times n$ sobre el cuerpo $F$. La suma de dos vectores $A$ y $B$ en $F^{m\times n}$ se define por 
    $$(A+B)_{ij}=A_{ij}+B_{ij}.$$
    El producto de un escalar $c$ y de la matriz $A$ se define por 
    \begin{equation}
	(cA)_{ij} = cA_{ij}.
    \end{equation}
    Obsérvece que $F^{i\times n}=F^n$.
\end{ejem}

