\chapter{Espacios vectoriales}

\section{Espacios vectoriales}

%-------------------- Definición 1 espacio vectorial --------------------
\begin{def.}
    Un \textbf{espacio vectorial} (o espacio lineal) consta de lo siguiente:
    \begin{enumerate}[1.]
	\item Un cuerpo $F$ de escalares;
	\item un conjunto $V$ de objetos llamados vectores;
	\item una regla (u operación) llamada adición, que asocia a cada par de vectores $\alpha,\beta $ de $V$ un vector $\alpha+\beta$ de $V$, que se llama suma de $\alpha$ y $\beta$, de tal modo que:
	   \begin{enumerate}[(a)]
	       \item La adición es conmutativa, $\alpha+\beta=\beta+\alpha$;
	       \item la adición es asociativa, $\alpha+(\beta+\gamma)=\alpha+\beta+\gamma$;
	       \item existe un único vector $0$ de $V$, llamado vector nulo tal que $\alpha+0=\alpha,$ para todo $\alpha$ de $V$;
	       \item para cada vector $\alpha$ de $V$ existe un vector $-\alpha$ de $V$, tal que $\alpha+(-\alpha)=0$;
	   \end{enumerate}
       \item una regla (u operación) llamada multiplicación escalar, que asocia a cada escalar $c$ de $F$ y cada vector $\alpha$ de $V$ a un vector $c\alpha$ en $V$, llamado producto de $c$ y $\alpha$, de tal modo que:
	   \begin{enumerate}[(a)]
	       \item $1\alpha=\alpha$ para todo $\alpha$ de $V$;
	       \item $(c_1c_2)\alpha=c_1(c_2\alpha)$;
	       \item $c(\alpha+\beta)=c\alpha+c\beta$;
	       \item $(c_1+c_2)\alpha=c_1\alpha+c_2\alpha$.
	   \end{enumerate}
    \end{enumerate}
\end{def.}
