\chapter{Espacios vectoriales de dimensión finita}

\mysection{Span e independencia lineal}

\setcounter{mydef}{1}
%%%%%%%%%%%%%%%%%%%%%%%%%%%%%%%%%%%%%%%%% 2.A %%%%%%%%%%%%%%%%%%%%%%%%%%%%%%%%%%%%%%%%%%%%%%%

%-------------------- 2.2 Notación 
\begin{mynot}[Lista of vectores]\;\\\\
    Por lo general, escribiremos listas de vectores sin paréntesis alrededor.
\end{mynot}
\vspace{0.5cm}

\subsection*{Combinaciones lineales y generadores}

%-------------------- 2.3 Definición 
\begin{mydef}[Combinación lineal] \;\\\\
    Una \textbf{combinación lineal} de una lista $v_1,\ldots , v_m$ de vectores en $V$ es un vector de la forma 
    $$a_1v_1+\cdots + a_mv_m,$$
    donde $a_1,\ldots a_m \in \textbf{F}.$
\end{mydef}

\setcounter{mydef}{4}
%-------------------- 2.5 Definición
\begin{mydef}[Span o generador] \;\\\\
    El conjunto de todas las combinaciones lineales de una lista de vectores $v_1,\ldots,v_m$ en $V$ se denomina \textbf{generador} de $v_1,\ldots,v_m$, denotado por $\span(v_1,\ldots, v_m)$. En otras palabras,
    $$\span(v_1,\ldots,v_m)=\left\{a_1v_1+\cdots + a_mv_m : a_1,\ldots, a_m \in \textbf{F}\right\}.$$
    El span de la lista vacía $()$ es definida por $\left\{0\right\}.$
\end{mydef}

\setcounter{myteo}{6}
%-------------------- 2.7 Teorema 
\begin{myteo}[Span es el subespacio más pequeño que lo contiene]
    El \textbf{span} de una lista de vectores en $V$ es el subespacio más pequeño de $V$ que contiene todos los vectores de la lista.\\\\
	Demostración.-\; Suponga que $v_1,\ldots,v_m$ es una lista de vectores en $V$. Primero demostraremos que $\span(v_1,\ldots,v_m)$ es un subespacio de $V$. El $0$ está en $\span(v_1,\ldots,v_m)$, porque
	$$0=0v_1+\ldots + 0v_m.$$
	También, $\span(v_1,\ldots,v_m)$ es cerrado bajo la suma, ya que
	$$(a_1v_1+\cdots + a_mv_m)+(c_1v_1+\cdots+c_mv_m)=(a_1+c_1)v_1+\cdots + (a_m+c_m)v_m.$$
	Además, $\span(v_1,\ldots,v_m)$ es cerrado bajo la multiplicación por un escalar, dado que
	$$\lambda(a_1v_1+\cdots + a_mv_m)=\lambda a_1v_1+\cdots + \lambda a_mv_m.$$
	Por lo tanto, $\span(v_1,\ldots, v_m)$ es un subespacio de $V$. Esto por 1.34.\\

	Cada $v_j$ es una combinación lineal de $v_1,\ldots,v_m$  (para mostrar esto, establezca $a_j=1$ y que las otras $a'$s en la definición de combinación lineal sean iguales a $0$). Así, el $\span(v_1,\ldots,v_m)$ contiene a cada $v_j$. Por otra parte, debido a que los subespacios están cerrados bajo la multiplicación de escalares y la suma, cada subespacio de $V$ que contiene a cada $v_j$ contiene a $\span(v_1,\ldots,v_m)$. Por lo tanto, $\span(v_1,\ldots,v_m)$ es el subespacio más pequeño de $V$ que contiene todos los demás vectores $v_1,\ldots,v_m$.
\end{myteo}

%-------------------- 2.8 Definición
\begin{mydef}[Spans]\,\\\\
    Si $\span(v_1,\ldots,v_m)$ es igual a $V$, decimos que $v_1,\ldots, v_m$ se extiende sobre $V$.
\end{mydef}

\setcounter{mydef}{9}
%-------------------- 2.10 Definición
\begin{mydef}[Espacio vectorial de dimensión finita]\,\\\\
    Un espacio vectorial se llama finito-dimensional si alguna lista de vectores en él genera el espacio.
\end{mydef}

%-------------------- 2.11 Definición
\begin{mydef}[Polinomio, $\boldmath \mathcal{P}\left(\textbf{F}\right)$]\,\\
    \begin{itemize}
	\item Una función $p:\textbf{F}\to \textbf{F}$ es llamado polinomio con coeficientes en $\textbf{F}$ si existe $a_0,\ldots,a_m\in \textbf{F}$ tal que
	$$p(z)=a_0+a_1z+a_2z^2+\cdots + a_mz^m$$
	para todo $z\in \textbf{F}$.

	\item $\mathcal{P}(\textbf{P})$ es el conjunto de todos los polinomios con coeficientes en $\textbf{F}$.
    \end{itemize}
\end{mydef}

Con las operaciones usuales de adición y multiplicación escalar, $\mathcal{P}(\textbf{F})$ es un espacio vectorial sobre $\textbf{F}$. En otras palabras, $\mathcal{P}(\textbf{F})$ es un subespacio de $\textbf{F}^{\textbf{F}}$, el espacio vectorial de funciones de $\textbf{F}$ en $\textbf{F}$. \\

Los coeficientes de un polinomio están determinados únicamente por el polinomio. Así, la siguiente definición define de manera única el grado de un polinomio.

%-------------------- 2.12 Definición
\begin{mydef}[Grado de un polinomio, $\mathbold deg\; p$]\,\\
    \begin{itemize}
	\item Un polinomio $p\in \mathcal{P}(F)$ se dice que tiene \textbf{grado} $m$ si existen escalares $a_0,a_1,\ldots,a_m\in \textbf{F}$ con $a_m\neq 0$ tal que
	$$p(z)=a_0+a_1z+\cdots+a_mz^m$$
	para todo $z\in \textbf{F}$. Si $p$ tiene grado $m$, escribimos $deg\; p=m$.
	\item El polinomio que es identicamente $0$ se dice que tiene \textbf{grado} $-\infty$.
    \end{itemize}
\end{mydef}

%-------------------- 2.13 Definición
\begin{mydef}[$\boldmath \mathcal{P}_m\left(\textbf{F}\right)$]\;\\\\
    Para $m$ un entero no negativo, $\mathcal{P}_m(\textbf{F})$ denota el conjunto de todos los polinomios con coeficiente en $\textbf{F}$ y grado no mayor a $m$.
\end{mydef}

Verifiquemos el siguiente ejemplo, tenga en cuenta que $\mathcal{P}_m(\textbf{F})=\span\left(1,z,\ldots,z^m\right)$; aquí estamos abusando ligeramente de la notación al permitir que $z^k$ denote una función.

\setcounter{mydef}{14}
%-------------------- 2.15 definición
\begin{mydef}[Espacio vectorial de dimensión infinita]\,\\\\
    Un espacio vectorial se llama \textbf{infinitamente-dimensional} si no es de dimensión finita.
\end{mydef}

%-------------------- 2.16 ejemplo
\begin{myejem}
    Demuestre que $\mathcal{P}(\textbf{F})$ es infinitamente-dimensional.\\\\
    Demostración.-\; Considere cualquier lista de elementos de $\mathcal{P}(\textbf{F})$. Sea $m$ el grado más alto de los polinomios en esta lista. Entonces, cada polinomio en el generador (span) de esta lista tiene grado máximo $m$. Por lo tanto, $z^{m+1}$ no está en el span de nuestra lista. Así, ninguna lista genera $\mathcal{P}(\textbf{F})$. Concluimos que $\mathcal{P}(\textbf{F})$ es de dimensión infinita.
\end{myejem}
\vspace{.5cm}

\subsection*{Independencia lineal}

Suponga $v_1,\ldots,v_m\in V$ y $v\in \span(v_1,\ldots,v_m)$. Por la definición de span, existe $a_1,\ldots,a_m\in \textbf{F}$ tal que
$$v=a_1v_1+\cdots+a_mv_m.$$
Considere la cuestión de si la elección de escalares en la ecuación anterior es única. Sea $c_1,\ldots,c_m$ otro conjunto de escalares tal que
$$v=c_1v_1+\cdots+c_mv_m.$$
Sustrayendo estas últimas ecuaciones, se tiene
$$0=(a_1-c_1)v_1+\cdots+(a_m-c_m)v_m.$$
Así, tenemos que escribir $0$ como una combinación lineal de $(v_1,\ldots,v_m)$. Si la única forma de hacer esto es la forma obvia (usando $0$ para todos los escalares), entonces cada $a_j-c_j$ es igual a $0$, lo que significa que cada $a_j$ es igual a $c_j$ (y por lo tanto la elección de los escalares fue realmente única). Esta situación es tan importante que le damos un nombre especial, independencia lineal, que ahora definiremos.

%-------------------- 2.17 Definición
\begin{mydef}[Linealmente independiente]\,\\
    \begin{itemize}
	\item Una lista $v_1,\ldots,v_m$ de vectores en $V$ se llama linealmente independiente si la única posibilidad de que  $a_1,\ldots,a_m\in \textbf{F}$ tal que $a_1v_1+\cdots+a_mv_m$ sea igual a $0$ es $a_1=\cdots=a_m=0.$
	\item La lista vacía $()$ también se declara linealmente independiente.
    \end{itemize}
\end{mydef}

El razonamiento anterior muestra que $v_1,\ldots,v_m$ es linealmente independiente si y sólo si cada vector en el $\span(v_1,\ldots,v_m)$ tiene sólo una representación lineal en forma de combinación lineal de $v_1,\ldots,v_m$.

\setcounter{mydef}{18}
%-------------------- 2.19 Definición
\begin{mydef}[Linealmente dependiente]\,\\
    \begin{itemize}
	\item Una lista $v_1,\ldots,v_m$ de vectores en $V$ se llama linealmente dependiente si no es linealmente independiente.
	\item En otras palabras, una lista $v_1,\ldots,v_m$ de vectores en $V$ es linealmente dependiente si existe $a_1,\ldots,a_m\in \textbf{F}$, no todos $0$, tal que $a_1v_1+\cdots+a_mv_m=0$.
    \end{itemize}
\end{mydef}

\setcounter{mylema}{20}
%-------------------- 2.21 Lema
\begin{mylema}
    Suponga $v_1,\ldots,v_m$ es una lista linealmente dependiente en $V$. Entonces, existen $j\in\left\{1,2,\ldots,m\right\}$ tal que se cumple lo siguente:
    \begin{enumerate}[(a)]
	\item $v_j\in \span(v_1,\ldots,v_{j-1});$
	\item Si el $j$ésimo término se elimina de $v_1,\ldots,v_m$, el span de la lista restante es igual a $\span(v_1,\ldots,v_m)$.\\
    \end{enumerate}
    Demostración.-\; Ya que la lista $v_1,\ldots,v_m$ es linealmente dependiente, existe números $a_1,\ldots,a_m\in \textbf{F}$, no todos $0$, tal que
    $$a_1v_1+\cdots + a_mv_m=0.$$
    Sea $j$ el elemento más grande de $\left\{1,\ldots, m\right\}$ tal que $a_j\neq 0$. Entonces,
    $$v_j=-\dfrac{a_1}{a_j}v_1-\cdots-\dfrac{a_{j-1}}{a_j}v_{j-1}\qquad (1).$$
    Lo que prueba (a). \\
    Para probar (b), suponga $u\in \span(v_1,\ldots,v_m)$. Entonces, existe números $c_1,\ldots,c_m\in \textbf{F}$ tal que
    $$u=c_1v_1+\cdots+c_mv_m.$$
    En la ecuación de arriba, podemos reemplazar $v_j$ con el lado derecho de (1), lo que muestra que $u$ está en el span de la lista obtenida al eliminar el j-ésimo término de $v_1,\ldots,v_m$. Así (b) se cumple.
\end{mylema}

Eligir $j=1$ en el lema de dependencia lineal anterior, significa que $v_1=0$, porque si $j=1$ entonces la condición (a) anterior se interpreta como que $v_1\in \span()$. Recuerde que $\span()=\left\{0\right\}$. Tenga en cuenta también que la demostración del inciso (b) debe modificarse de manera obvia si $v_i=0$ y $j=1$.

\setcounter{myteo}{22}
%-------------------- 2.23 Definición
\begin{myteo}[Longitud de la lista linealmente independiente es  $\boldmath \leq$ a la longitud de la lista que genera]\;\\\\
    En un espacio vectorial finito, la longitud de cada lista linealmente independiente de vectores es menor o igual que la longitud de cada lista de vectores.\\\\
    Demostración.-\; Suponga $u_1,\ldots,u_m$ es linealmente independiente en $\textbf{V}$. Suponga también que $w_1,\ldots,w_n$ generan $V$. Necesitamos probar que $m\leq n$. Lo hacemos a través del proceso de pasos que se describe a continuación; tenga en cuenta que en cada paso agregamos una de las $u's$ y eliminamos una de las $w's$.

    \begin{enumerate}[\small\textbf{Paso} 1.]
	\item Sea $B$ la lista $w_1,\ldots,w_n$, que abarca $V$. Por lo tanto, adjuntar cualquier vector en $V$ a esta lista produce una lista linealmente dependiente (porque el nuevo vector adjunto se puede escribir como una combinación lineal de los otros vectores). En particular, la lista
	$$u_1,w_1,\ldots,w_n$$
	es linealmente dependiente. Así, por el lema (2.21), podemos eliminar un de las $w$ para que la nueva lista $B$ (de longitud $n$) que consta de $u_1$ y las $w$ restantes abarquen $V$.
	\item La lista $B$ (de longitud $n$) del paso $j-1$ genera $V$. Así, adjuntar cualquier vector a esta lista produce una lista linealmente dependiente. En particular, la lista de longitud $(n+1)$ obtenida al unir $u_j$ a $B$, colocándola justo después de $u_1,\ldots,u_{j-1}$, es linealmente dependiente. Por el lema de dependencia lineal (2.21)m uno de los vectores de esta lista está en el lapso de los anteriores, y porque $u_1,\ldots,u_j$ es linealmente independientes, este vector es uno de los $w's$, no uno de los $u's$. Podemos eliminar esa $w$ para la nueva lista $B$ (de longitud $n$) que consta de $u_1,\ldots,u_j$ y las $w's$ restantes generan $V$.
    \end{enumerate}
    Después del paso $m$, hemos agregado todas las $u$ y el proceso se detiene. En cada paso, a medida que agregamos un $u$ a $B$, el lema de dependencia lineal implica que hay algo de $w$ que eliminar. Por lo tanto, hay al menos tantas $w$ como $u$.
\end{myteo}

\setcounter{myteo}{25}
%-------------------- 2.26 Teorema
\begin{myteo}[Subespacio de dimensión finita]\; \\\\
    Todo subespacio de un vector de dimensión finita es de dimensión finita.\\\\
	Demostración.-\; Suponga que $V$ es de dimensión finita y $U$ es un subespacio de $V$. Necesitamos demostrar que $U$ es de dimensión finita. Hacemos esto a través de la siguiente construcción de  pasos. 
	\begin{enumerate}[\small\textbf{Paso} 1.]
	    \item Si $u=\left\{0\right\}$, entonces $U$ es de dimensión finita por lo que hemos terminado. Si $U\neq \left\{0\right\}$, entonces elegimos un vector no nulo $v_1\in U$
	    \item Si $U=\span(v_1,\ldots,v_{j-1})$, entonces $U$ es de dimensión finita por lo que hemos terminado. Si $U\neq \span(v_1,\ldots,v_{j-1})$, entonces elegimos un vector $v_j\in U$ tal que 
	    $$v_j\in \span(v_1,\ldots,v_{j-1}).$$
	\end{enumerate}
	Después de cada paso, mientras continúa el proceso, hemos construido una lista de vectores tal que ningún vector en esta lista está en el generador de los vectores anteriores. Así, después de cada paso hemos construido una lista linealmente independiente, por el lema de dependencia lineal (2.21). Esta lista linealmente independiente no puede ser más grande que cualquier lista de expansión de $V$ (por 2,23). Por lo tanto, el proceso eventualmente termina, lo que significa que $U$ es de dimensión finita.
\end{myteo}


\setcounter{mysection}{0}
\mysection{Ejercicios}

\begin{enumerate}[\bfseries 1.]

    %------------------- 1
    \item Suponga $v_1,v_2,v_3,v_4$ se extiende por $V$. Demostrar que la lista 
    $$v_1-v_2,v_2-v_3,v_3-v_4,v_4$$
    también se extiende por $V$.\\\\
	Demostración.-\; Sea $v\in V$, entonces existe $a_1,a_2,a_3,a_4$ tal que 
	$$v=a_1v_1+a_2v_2+a_3v_3+a_4v_4.$$
	Que implica,
	$$
	\begin{array}{rcl}
	    v&=&a_1v_1+a_2v_2+a_3v_3+a_4v_4-a_1v_2+a_1v_2-a_1v_3+a_1v_3-a_2v_3+a_2v_3-a_1v_4+a_1v_4\\
	     &-& a_2v_4+a_2v_4 -a_3v_4 +a_3v_4\\
	\end{array}
	$$

	De donde,

	$$v=a_1(v_1-v_2)+(a_1+a_2)(v_2-v_3)+(a_1+a_2+a_3)(v_3-v_4)+(a_1+a_2+a_3+a_4)v_4.$$
	Por lo tanto, cualquier vector en $V$ puede ser expresado por una combinación lineal de 
	$$v_1-v_2,v_2-v_3,v_3-v_4,v_4.$$
	Así, esta lista se extiende por $V$.\\\\

    %------------------- 2
    \item Verifique las afirmaciones del Ejemplo 2.18.\\

	\begin{enumerate}[(a)]

	    %---------- (a)
	    \item Una lista $v$ de un vector $v\in V$ es linealmente independiente si y sólo si $v\neq 0$.\\\\
		Demostración.-\; Demostremos que si $v$ es linealmente independiente, entonces $v\neq 0$. Supongamos que $v=0$. Sea un escalar $a\neq 0$. De donde, $av=0$ incluso cuando $a\neq 0$. Esto contradice la definición de independencia lineal. Por lo tanto, $v$ debe ser linealmente dependiente. Esto es, $v=0$ implica que $v$ es un vector linealmente dependiente. Por lo que, si $v$ es linealmente independiente, entonces $v$ es un vector distinto de cero.\\
		Por otro lado, debemos demostrar que  $v\neq 0$ implica que $v$ es linealmente independiente. Sea  un escalar $a$ tal que $av=0$. Si $a\neq 0$, entonces $av$ no puede ser $0$. Por eso $a$ debe ser $0$. Por lo tanto, $v\neq 0$ y $av=0$ implica que $a=0$. Así, $v$ es linealmente independiente.\\\\

	    %---------- (b)
	    \item Una lista de dos vectores en $V$ es linealmente independiente si y sólo si ninguno de los vectores es múltiplo escalar del otro. \\\\
		Demostración.-\; El enunciado siguiente es equivalente. Dos vectores son linealmente dependientes si y sólo si uno de los vectores es múltiplo escalar de otro. Supongamos que $v_1,v_2$ son dos vectores linealmente dependientes. Por lo que, existe escalares $a_1,a_2$ tal que 
		$$a_1v_1+a_2v_2=0$$
		y no ambos escalares $a_1,a_2$ son cero. Sea $a_1\neq 0$, entonces la ecuación se podría reescribir como
		$$v_1=-\dfrac{a_2}{a_1}v_2$$
		el cual prueba que $v_1$ es un múltiplo escalar de $v_2$. Por otro lado, si $a_2\neq 0$, entonces $v_2=-\frac{a_1}{a_2}v_1$ de aquí podemos afirmar que $v_2$ es un múltiplo escalar de $v_1$.\\
		Ahora supongamos que que uno de los $v_1$ o $v_2$ es un múltiplo escalar del otro. Podemos decir, sin perdida de generalidad, que $v_1$ es un múltiplo escalar de $v_2$. Esto es, $v_1=cv_2$ para algún escalar $c$. Por lo tanto, la ecuación $v_1-cv_2=0$ se cumple, ya que el multiplicador de $v_1$ es distintos de cero. Esto es precisamente lo que requerimos para la definición de dependencia lineal. Así, $v_1$ y $v_2$ son linealmente dependientes.\\\\

	    %---------- (c)
	    \item $(1,0,0,0),(0,1,0,0),(0,0,1,0)$ es linealmente independiente en $\textbf{F}^4$.\\\\
		Demostración.-\; Utilizaremos la definición de independencia lineal. Sean $a,b,c$ escalares en \textbf{F} tal que
		$$a(1,0,0,0)+b(0,1,0,0)+c(0,0,1,0)=\textbf{0}=(0,0,0,0)$$
		Entonces,
		$$(a,b,c,0)=(0,0,0,0)$$
		Lo que implica,
		$$a,b,c=0.$$
		Esto demuestra que los tres vectores son linealmente independientes.\\\\


	    %---------- (d)
	    \item La lista $1,z,\ldots,z^m$ es linealmente independiente en $\mathcal{P}(\textbf{F})$ para cada entero no negativo $m$.\\\\
		Demostración.-\; Demostremos por contradicción. Supongamos que $1,z,\ldots,z^m$ es linealmente dependiente. Por lo que, existe un escalar $a_0,a_1,\ldots,a_m$ tal que
		$$a_0+a_1z+\ldots+a_mz^m=0.$$
		Sea $k$ el indice más grande tal que $a_k\neq 0$. Esto significa que los escalares desde $a_{k+1}$ hasta $a_m$ son cero. Entonces, se deduce que
		$$a_0+a_1z+\ldots+a_kz^k=0.$$
		Reescribiendo se tiene
		$$z_k=-\dfrac{a_0}{a_k}-\dfrac{a_1}{a_k}z-\cdots-\dfrac{a_{k-1}}{a_k}z^{k-1}.$$
		Aquí, expresamos $z^k$ como un polinomio de grado $k-1$ el cual es absurdo. Por lo que $1,z,z^2,\ldots,z^m$ es un conjunto linealmente independiente.\\\\

	\end{enumerate}


    %------------------- 3.
    \item Encuentre un número $t$ tal que
    $$(3,1,4),(2,-3,5),(5,9,t)$$
    no es linealmente independiente en $\textbf{R}^3$.\\\\
	Respuesta.-\; Sea,
	$$a(3,1,4)+b(2,-3,5)+c(5,9,t)=0.$$
	Si $c=0$. Entonces,
	$$a(3,1,4)+b(2,-3,5)=0.$$
	Lo que implica
	$$
	\begin{array}{rcl}
	    3a+2b&=&0\\
	    a-3b&=&0\\
	    4a+5b&=&0
	\end{array}
	$$
	De donde, resolviendo para $a$ y $b$ se tiene
	$$a=0\quad \mbox{y}\quad b=0.$$
	Pero, no queremos que $a,b,c$ sean cero. Así que debemos forzar que $c\neq 0$, como sigue:
	$$a(3,1,4)+b(2,-3,5)+c(5,9,t)=0\quad \Rightarrow \quad (5,9,t)=-\dfrac{a}{c}(3,1,4)-\dfrac{b}{c}(2,-3,5).$$
	Es decir, estamos expresando $(5,9,t)$ como una combinación lineal de los vectores restantes. Así, sea $-\dfrac{a}{c}=x\;$,$\;-\dfrac{b}{c}=y$  por lo que,
	$$(5,9,t)=x(3,1,4)+y(2,-3,5).$$
	Así, tenemos que
	$$
	\begin{array}{rcl}
	    3x+2y&=&5\\
	    x-3y&=&9\\
	    4x+5y&=&t
	\end{array}
	$$
	Resolviendo para $x$ e $y$ se tiene
	$$x=3\quad \mbox{y}\quad y=-2.$$
	Por lo tanto,
	$$t=2.$$\\

    %------------------- 4.
    \item Verifique la afirmación en el segundo punto del Ejemplo 2.20. Es decir, la lista $(2,3,1),(1,-1,2),(7,3,c)$ es linealmente dependientes en $\textbf{F}^3$ si y sólo si $c=8$, como debes verificar.\\\\
	Respuesta.-\;

\end{enumerate}
