\chapter{Espacios vectoriales de dimensión finita}

\mysection{Span e independencia lineal}

\setcounter{mydef}{1}
%%%%%%%%%%%%%%%%%%%%%%%%%%%%%%%%%%%%%%%%% 2.A %%%%%%%%%%%%%%%%%%%%%%%%%%%%%%%%%%%%%%%%%%%%%%%

%-------------------- 2.2 Notación 
\begin{mynot}[Lista of vectores]\;\\\\
    Por lo general, escribiremos listas de vectores sin paréntesis alrededor.
\end{mynot}
\vspace{0.5cm}

\subsection*{Combinaciones lineales y generadores}

%-------------------- 2.3 Definición 
\begin{mydef}[Combinación lineal] \;\\\\
    Una \textbf{combinación lineal} de una lista $v_1,\ldots , v_m$ de vectores en $V$ es un vector de la forma 
    $$a_1v_1+\cdots + a_mv_m,$$
    donde $a_1,\ldots a_m \in \textbf{F}.$
\end{mydef}

\setcounter{mydef}{4}
%-------------------- 2.5 Definición
\begin{mydef}[Span o generador] \;\\\\
    El conjunto de todas las combinaciones lineales de una lista de vectores $v_1,\ldots,v_m$ en $V$ se denomina \textbf{generador} de $v_1,\ldots,v_m$, denotado por $\span(v_1,\ldots, v_m)$. En otras palabras,
    $$\span(v_1,\ldots,v_m)=\left\{a_1v_1+\cdots + a_mv_m : a_1,\ldots, a_m \in \textbf{F}\right\}.$$
    El span de la lista vacía $()$ es definida por $\left\{0\right\}.$
\end{mydef}

\setcounter{myteo}{6}
%-------------------- 2.7 Teorema 
\begin{myteo}[Span es el subespacio más pequeño que lo contiene]
    El \textbf{span} de una lista de vectores en $V$ es el subespacio más pequeño de $V$ que contiene todos los vectores de la lista.\\\\
	Demostración.-\; Suponga que $v_1,\ldots,v_m$ es una lista de vectores en $V$. Primero demostraremos que $\span(v_1,\ldots,v_m)$ es un subespacio de $V$. El $0$ está en $\span(v_1,\ldots,v_m)$, porque
	$$0=0v_1+\ldots + 0v_m.$$
	También, $\span(v_1,\ldots,v_m)$ es cerrado bajo la suma, ya que
	$$(a_1v_1+\cdots + a_mv_m)+(c_1v_1+\cdots+c_mv_m)=(a_1+c_1)v_1+\cdots + (a_m+c_m)v_m.$$
	Además, $\span(v_1,\ldots,v_m)$ es cerrado bajo la multiplicación por un escalar, dado que
	$$\lambda(a_1v_1+\cdots + a_mv_m)=\lambda a_1v_1+\cdots + \lambda a_mv_m.$$
	Por lo tanto, $\span(v_1,\ldots, v_m)$ es un subespacio de $V$. Esto por 1.34.\\

	Cada $v_j$ es una combinación lineal de $v_1,\ldots,v_m$  (para mostrar esto, establezca $a_j=1$ y que las otras $a'$s en la definición de combinación lineal sean iguales a $0$). Así, el $\span(v_1,\ldots,v_m)$ contiene a cada $v_j$. Por otra parte, debido a que los subespacios están cerrados bajo la multiplicación de escalares y la suma, cada subespacio de $V$ que contiene a cada $v_j$ contiene a $\span(v_1,\ldots,v_m)$. Por lo tanto, $\span(v_1,\ldots,v_m)$ es el subespacio más pequeño de $V$ que contiene todos los demás vectores $v_1,\ldots,v_m$.
\end{myteo}

%-------------------- 2.8 Definición
\begin{mydef}[Spans]\,\\\\
    Si $\span(v_1,\ldots,v_m)$ es igual a $V$, decimos que $v_1,\ldots, v_m$ se extiende sobre $V$.
\end{mydef}

\setcounter{mydef}{9}
%-------------------- 2.10 Definición
\begin{mydef}[Espacio vectorial de dimensión finita]\,\\\\
    Un espacio vectorial se llama finito-dimensional si alguna lista de vectores en él genera el espacio.
\end{mydef}

%-------------------- 2.11 Definición
\begin{mydef}[Polinomio, $\boldmath \mathcal{P}\left(\textbf{F}\right)$]\,\\
    \begin{itemize}
	\item Una función $p:\textbf{F}\to \textbf{F}$ es llamado polinomio con coeficientes en $\textbf{F}$ si existe $a_0,\ldots,a_m\in \textbf{F}$ tal que
	$$p(z)=a_0+a_1z+a_2z^2+\cdots + a_mz^m$$
	para todo $z\in \textbf{F}$.

	\item $\mathcal{P}(\textbf{P})$ es el conjunto de todos los polinomios con coeficientes en $\textbf{F}$.
    \end{itemize}
\end{mydef}

Con las operaciones usuales de adición y multiplicación escalar, $\mathcal{P}(\textbf{F})$ es un espacio vectorial sobre $\textbf{F}$. En otras palabras, $\mathcal{P}(\textbf{F})$ es un subespacio de $\textbf{F}^{\textbf{F}}$, el espacio vectorial de funciones de $\textbf{F}$ en $\textbf{F}$. \\

Los coeficientes de un polinomio están determinados únicamente por el polinomio. Así, la siguiente definición define de manera única el grado de un polinomio.

%-------------------- 2.12 Definición
\begin{mydef}[Grado de un polinomio, $\mathbold deg\; p$]\,\\
    \begin{itemize}
	\item Un polinomio $p\in \mathcal{P}(F)$ se dice que tiene \textbf{grado} $m$ si existen escalares $a_0,a_1,\ldots,a_m\in \textbf{F}$ con $a_m\neq 0$ tal que
	$$p(z)=a_0+a_1z+\cdots+a_mz^m$$
	para todo $z\in \textbf{F}$. Si $p$ tiene grado $m$, escribimos $deg\; p=m$.
	\item El polinomio que es identicamente $0$ se dice que tiene \textbf{grado} $-\infty$.
    \end{itemize}
\end{mydef}

%-------------------- 2.13 Definición
\begin{mydef}[$\boldmath \mathcal{P}_m\left(\textbf{F}\right)$]\;\\\\
    Para $m$ un entero no negativo, $\mathcal{P}_m(\textbf{F})$ denota el conjunto de todos los polinomios con coeficiente en $\textbf{F}$ y grado no mayor a $m$.
\end{mydef}

Verifiquemos el siguiente ejemplo, tenga en cuenta que $\mathcal{P}_m(\textbf{F})=\span\left(1,z,\ldots,z^m\right)$; aquí estamos abusando ligeramente de la notación al permitir que $z^k$ denote una función.

\setcounter{mydef}{14}
%-------------------- 2.15 definición
\begin{def.}[Espacio vectorial de dimensión infinita]\,\\\\
    Un espacio vectorial se llama \textbf{infinitamente-dimensional} si no es de dimensión finita.
\end{def.}

%-------------------- 2.16 ejemplo
\begin{ejem}
    Demuestre que $\mathcal{P}(\textbf{F})$ es infinitamente-dimensional.\\\\
    Demostración.-\; Considere cualquier lista de elementos de $\mathcal{P}(\textbf{F})$. Sea $m$ el grado más alto de los polinomios en esta lista. Entonces, cada polinomio en el generador (span) de esta lista tiene grado máximo $m$. Por lo tanto, $z^{m+1}$ no está en el span de nuestra lista. Así, ninguna lista genera $\mathcal{P}(\textbf{F})$. Concluimos que $\mathcal{P}(\textbf{F})$ es de dimensión infinita.
\end{ejem}
\vspace{.5cm}

\subsection*{Independencia lineal}
