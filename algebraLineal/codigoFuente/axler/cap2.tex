\chapter{Espacios vectoriales de dimensión finita}

\mysection{Span e independencia lineal}

\setcounter{mydef}{1}
%%%%%%%%%%%%%%%%%%%%%%%%%%%%%%%%%%%%%%%%% 2.A %%%%%%%%%%%%%%%%%%%%%%%%%%%%%%%%%%%%%%%%%%%%%%%

%-------------------- 2.2 Notación 
\begin{mynot}[Lista of vectores]\;\\\\
    Por lo general, escribiremos listas de vectores sin paréntesis alrededor.
\end{mynot}
\vspace{0.5cm}

\subsection*{Combinaciones lineales y generadores}

%-------------------- 2.3 Definición 
\begin{mydef}[Combinación lineal] \;\\\\
    Una \textbf{combinación lineal} de una lista $v_1,\ldots , v_m$ de vectores en $V$ es un vector de la forma 
    $$a_1v_1+\cdots + a_mv_m,$$
    donde $a_1,\ldots a_m \in \textbf{F}.$
\end{mydef}

\setcounter{mydef}{4}
%-------------------- 2.5 Definición
\begin{mydef}[Span o generador] \;\\\\
    El conjunto de todas las combinaciones lineales de una lista de vectores $v_1,\ldots,v_m$ en $V$ se denomina \textbf{generador} de $v_1,\ldots,v_m$, denotado por $\span(v_1,\ldots, v_m)$. En otras palabras,
    $$\span(v_1,\ldots,v_m)=\left\{a_1v_1+\cdots + a_mv_m : a_1,\ldots, a_m \in \textbf{F}\right\}.$$
    El span de la lista vacía $()$ es definida por $\left\{0\right\}.$
\end{mydef}

\setcounter{myteo}{6}
%-------------------- 2.7 Teorema 
\begin{myteo}[Span es el subespacio más pequeño que lo contiene]\,\\\\
    El \textbf{span} de una lista de vectores en $V$ es el subespacio más pequeño de $V$ que contiene todos los vectores de la lista.\\\\
	Demostración.-\; Suponga que $v_1,\ldots,v_m$ es una lista de vectores en $V$. Primero demostraremos que $\span(v_1,\ldots,v_m)$ es un subespacio de $V$. El $0$ está en $\span(v_1,\ldots,v_m)$, porque
	$$0=0v_1+\ldots + 0v_m.$$
	También, $\span(v_1,\ldots,v_m)$ es cerrado bajo la suma, ya que
	$$(a_1v_1+\cdots + a_mv_m)+(c_1v_1+\cdots+c_mv_m)=(a_1+c_1)v_1+\cdots + (a_m+c_m)v_m.$$
	Además, $\span(v_1,\ldots,v_m)$ es cerrado bajo la multiplicación por un escalar, dado que
	$$\lambda(a_1v_1+\cdots + a_mv_m)=\lambda a_1v_1+\cdots + \lambda a_mv_m.$$
	Por lo tanto, $\span(v_1,\ldots, v_m)$ es un subespacio de $V$. Esto por 1.34.\\

	Cada $v_j$ es una combinación lineal de $v_1,\ldots,v_m$  (para mostrar esto, establezca $a_j=1$ y que las otras $a'$s en la definición de combinación lineal sean iguales a $0$). Así, el $\span(v_1,\ldots,v_m)$ contiene a cada $v_j$. Por otra parte, debido a que los subespacios están cerrados bajo la multiplicación de escalares y la suma, cada subespacio de $V$ que contiene a cada $v_j$ contiene a $\span(v_1,\ldots,v_m)$. Por lo tanto, $\span(v_1,\ldots,v_m)$ es el subespacio más pequeño de $V$ que contiene todos los demás vectores $v_1,\ldots,v_m$.
\end{myteo}

%-------------------- 2.8 Definición
\begin{mydef}[Spans]\,\\\\
    Si $\span(v_1,\ldots,v_m)$ es igual a $V$, decimos que $v_1,\ldots, v_m$ genera $V$.
\end{mydef}

\setcounter{mydef}{9}
%-------------------- 2.10 Definición
\begin{mydef}[Espacio vectorial de dimensión finita]\,\\\\
    Un espacio vectorial se llama finito-dimensional si alguna lista de vectores en él genera el espacio.\\

    Es decir, si hay una lista de vectores en un espacio vectorial finito-dimensional que puede abarcar todo el espacio, entonces podemos decir que el espacio vectorial es finito-dimensional.
\end{mydef}

%-------------------- 2.11 Definición
\begin{mydef}[Polinomio, $\boldmath \mathcal{P}\left(\textbf{F}\right)$]\,\\
    \begin{itemize}
	\item Una función $p:\textbf{F}\to \textbf{F}$ es llamado polinomio con coeficientes en $\textbf{F}$ si existe $a_0,\ldots,a_m\in \textbf{F}$ tal que
	$$p(z)=a_0+a_1z+a_2z^2+\cdots + a_mz^m$$
	para todo $z\in \textbf{F}$.

	\item $\mathcal{P}(\textbf{P})$ es el conjunto de todos los polinomios con coeficientes en $\textbf{F}$.
    \end{itemize}
\end{mydef}

Con las operaciones usuales de adición y multiplicación escalar, $\mathcal{P}(\textbf{F})$ es un espacio vectorial sobre $\textbf{F}$. En otras palabras, $\mathcal{P}(\textbf{F})$ es un subespacio de $\textbf{F}^{\textbf{F}}$, el espacio vectorial de funciones de $\textbf{F}$ en $\textbf{F}$. \\

Los coeficientes de un polinomio están determinados únicamente por el polinomio. Así, la siguiente definición define de manera única el grado de un polinomio.

%-------------------- 2.12 Definición
\begin{mydef}[Grado de un polinomio, $\mathbold deg\; p$]\,\\
    \begin{itemize}
	\item Un polinomio $p\in \mathcal{P}(F)$ se dice que tiene \textbf{grado} $m$ si existen escalares $a_0,a_1,\ldots,a_m\in \textbf{F}$ con $a_m\neq 0$ tal que
	$$p(z)=a_0+a_1z+\cdots+a_mz^m$$
	para todo $z\in \textbf{F}$. Si $p$ tiene grado $m$, escribimos $deg\; p=m$.
	\item El polinomio que es identicamente $0$ se dice que tiene \textbf{grado} $-\infty$.
    \end{itemize}
\end{mydef}

%-------------------- 2.13 Definición
\begin{mydef}[$\boldmath \mathcal{P}_m\left(\textbf{F}\right)$]\;\\\\
    Para $m$ un entero no negativo, $\mathcal{P}_m(\textbf{F})$ denota el conjunto de todos los polinomios con coeficiente en $\textbf{F}$ y grado no mayor a $m$.
\end{mydef}

Verifiquemos el siguiente ejemplo, tenga en cuenta que $\mathcal{P}_m(\textbf{F})=\span\left(1,z,\ldots,z^m\right)$; aquí estamos abusando ligeramente de la notación al permitir que $z^k$ denote una función.

\setcounter{mydef}{14}
%-------------------- 2.15 definición
\begin{mydef}[Espacio vectorial de dimensión infinita]\,\\\\
    Un espacio vectorial se llama \textbf{infinitamente-dimensional} si no es de dimensión finita.
\end{mydef}

%-------------------- 2.16 ejemplo
\begin{myejem}
    Demuestre que $\mathcal{P}(\textbf{F})$ es infinitamente-dimensional.\\\\
    Demostración.-\; Considere cualquier lista de elementos de $\mathcal{P}(\textbf{F})$. Sea $m$ el grado más alto de los polinomios en esta lista. Entonces, cada polinomio en el generador (span) de esta lista tiene grado máximo $m$. Por lo tanto, $z^{m+1}$ no está en el span de nuestra lista. Así, ninguna lista genera $\mathcal{P}(\textbf{F})$. Concluimos que $\mathcal{P}(\textbf{F})$ es de dimensión infinita.
\end{myejem}
\vspace{.5cm}

\subsection*{Independencia lineal}

Suponga $v_1,\ldots,v_m\in V$ y $v\in \span(v_1,\ldots,v_m)$. Por la definición de span, existe $a_1,\ldots,a_m\in \textbf{F}$ tal que
$$v=a_1v_1+\cdots+a_mv_m.$$
Considere la cuestión de si la elección de escalares en la ecuación anterior es única. Sea $c_1,\ldots,c_m$ otro conjunto de escalares tal que
$$v=c_1v_1+\cdots+c_mv_m.$$
Sustrayendo estas últimas ecuaciones, se tiene
$$0=(a_1-c_1)v_1+\cdots+(a_m-c_m)v_m.$$
Así, tenemos que escribir $0$ como una combinación lineal de $(v_1,\ldots,v_m)$. Si la única forma de hacer esto es la forma obvia (usando $0$ para todos los escalares), entonces cada $a_j-c_j$ es igual a $0$, lo que significa que cada $a_j$ es igual a $c_j$ (y por lo tanto la elección de los escalares fue realmente única). Esta situación es tan importante que le damos un nombre especial, independencia lineal, que ahora definiremos.

%-------------------- 2.17 Definición
\begin{mydef}[Linealmente independiente]\,\\
    \begin{itemize}
	\item Una lista $v_1,\ldots,v_m$ de vectores en $V$ se llama linealmente independiente si la única posibilidad de que  $a_1,\ldots,a_m\in \textbf{F}$ tal que $a_1v_1+\cdots+a_mv_m$ sea igual a $0$ es $a_1=\cdots=a_m=0.$
	\item La lista vacía $()$ también se declara linealmente independiente.
    \end{itemize}
\end{mydef}

El razonamiento anterior muestra que $v_1,\ldots,v_m$ es linealmente independiente si y sólo si cada vector en el $\span(v_1,\ldots,v_m)$ tiene sólo una representación lineal en forma de combinación lineal de $v_1,\ldots,v_m$.

\setcounter{mydef}{18}
%-------------------- 2.19 Definición
\begin{mydef}[Linealmente dependiente]\,\\
    \begin{itemize}
	\item Una lista $v_1,\ldots,v_m$ de vectores en $V$ se llama linealmente dependiente si no es linealmente independiente.
	\item En otras palabras, una lista $v_1,\ldots,v_m$ de vectores en $V$ es linealmente dependiente si existe $a_1,\ldots,a_m\in \textbf{F}$, no todos $0$, tal que $a_1v_1+\cdots+a_mv_m=0$.
    \end{itemize}
\end{mydef}

\setcounter{mylema}{20}
%-------------------- 2.21 Lema
\begin{mylema}
    Suponga $v_1,\ldots,v_m$ es una lista linealmente dependiente en $V$. Entonces, existen $j\in\left\{1,2,\ldots,m\right\}$ tal que se cumple lo siguente:
    \begin{enumerate}[(a)]
	\item $v_j\in \span(v_1,\ldots,v_{j-1});$
	\item Si el j-ésimo término se elimina de $v_1,\ldots,v_m$, el generador de la lista restante es igual a $\span(v_1,\ldots,v_m)$.\\
    \end{enumerate}
    Demostración.-\; Ya que la lista $v_1,\ldots,v_m$ es linealmente dependiente, existe números $a_1,\ldots,a_m\in \textbf{F}$, no todos $0$, tal que
    $$a_1v_1+\cdots + a_mv_m=0.$$
    Sea $j$ el elemento más grande de $\left\{1,\ldots, m\right\}$ tal que $a_j\neq 0$. Entonces,
    $$v_j=-\dfrac{a_1}{a_j}v_1-\cdots-\dfrac{a_{j-1}}{a_j}v_{j-1}\qquad (1).$$
    Lo que prueba (a). \\
    Para probar (b), suponga $u\in \span(v_1,\ldots,v_m)$. Entonces, existe números $c_1,\ldots,c_m\in \textbf{F}$ tal que
    $$u=c_1v_1+\cdots+c_mv_m.$$
    En la ecuación de arriba, podemos reemplazar $v_j$ con el lado derecho de (1). Es decir,
    $$u=c_1v_1+\cdots+a_jv_j+\cdots +c_mv_m.$$
    lo que muestra que $u$ está en el span de la lista obtenida al eliminar el j-ésimo término de $v_1,\ldots,v_m$. Así (b) se cumple.
\end{mylema}

Eligir $j=1$ en el lema de dependencia lineal anterior, significa que $v_1=0$, porque si $j=1$ entonces la condición (a) anterior se interpreta como que $v_1\in \span()$. Recuerde que $\span()=\left\{0\right\}$. Tenga en cuenta también que la demostración del inciso (b) debe modificarse de manera obvia si $v_i=0$ y $j=1$.\\

Ahora llegamos a un resultado importante. Dice que ninguna lista linealmente independiente en $V$ es más extensa que una lista generadora en V.

\setcounter{myteo}{22}
%-------------------- 2.23 teorema
\begin{myteo}[La longitud de una lista linealmente independiente es \boldmath$\leq$ a la longitud de la lista generadora]\;\\\\
    En un espacio vectorial de dimensión finita, la longitud de cada lista linealmente independiente de vectores es menor o igual que la longitud de cada lista generadora de vectores (longitud=n de vectores).\\\\ 
    Demostración.-\; Suponga $u_1,\ldots,u_m$ es linealmente independiente en $\textbf{V}$. Suponga también que $w_1,\ldots,w_n$ generan $V$. Necesitamos probar que $m\leq n$. Lo hacemos a través del proceso de pasos que se describe a continuación; tenga en cuenta que en cada paso agregamos una de las $u's$ y eliminamos una de las $w's$.

    \begin{enumerate}[\small\textbf{Paso} 1.]
	\item Sea $B$ la lista $w_1,\ldots,w_n$, que genera $V$. Por lo tanto, añadir cualquier vector en $V$ a esta lista produce una lista linealmente dependiente (porque el nuevo vector añadido se puede escribir como una combinación lineal de los otros vectores, $u_1=\frac{a_1}{c_1}w_1+\cdots+\frac{a_n}{c_1}$). En particular, la lista
	$$u_1,w_1,\ldots,w_n$$
	es linealmente dependiente. Así, por el lema (2.21), podemos eliminar un de las $w$ para que la nueva lista $B$ (de longitud $n$) que consta de $u_1$ y las $w's$ restantes generen $V$.
	\item[\textbf{Paso} j.] La lista $B$ (de longitud $n$) del paso $j-1$ genera $V$. Así, añadir cualquier vector a esta lista produce una lista linealmente dependiente. En particular, la lista de longitud $(n+1)$ obtenida al unir $u_j$ a $B$, colocándola justo después de $u_1,\ldots,u_{j-1}$, es linealmente dependiente. Por el lema de dependencia lineal (2.21), uno de los vectores de esta lista está en el generador de los anteriores, y ya que $u_1,\ldots,u_j$ es linealmente independientes, este vector es uno de los $w's$, no uno de los $u's$. Podemos eliminar esa $w$ para la nueva lista $B$ (de longitud $n$) que consta de $u_1,\ldots,u_j$ y las $w's$ restantes generan $V$.
    \end{enumerate}
    Después del paso $m$, hemos agregado todas las $u$ y el proceso se detiene. En cada paso, a medida que agregamos un $u$ a $B$, el lema de dependencia lineal implica que hay algo de $w$ que eliminar. Por lo tanto, hay al menos tantas $w$ como $u$.
\end{myteo}

Aclaremos esto con dos ejemplos.

%-------------------- 2.24 Ejemplo
\begin{myejem}
    Demuestre que la lista $(1,2,3),(4,5,8),(9,6,7),(-3,2,8)$ no es linealmente independiente en $\textbf{R}^3$.\\\\
	Demostración.-\; La lista $(1,0,0),(0,1,0),(0,0,1)$ genera $\textbf{R}^3$. Por lo tanto, ninguna lista de longitud superior a $3$ es linealmente independiente en $\textbf{R}^3$.
\end{myejem}

%-------------------- 2.25 Ejemplo
\begin{myejem}
    Demostrar que la lista $(1,2,3,-5),(4,5,8,3),(9,6,7,-1)$ no genera $\textbf{R}^4$.\\\\	
	Demostración.-\; La lista $(1,0,0,0),(0,1,0,0),(0,0,1,0),(0,0,0,1)$ es linealmente independiente en $\textbf{R}^4$. Por lo tanto, ninguna lista de longitud inferior a $4$  genera $\textbf{R}^4$.
\end{myejem}

\setcounter{myteo}{25}
%-------------------- 2.26 Teorema
\begin{myteo}[Subespacio de dimensión finita]\; \\\\
    Todo subespacio de un vector de dimensión finita es de dimensión finita.\\\\
	Demostración.-\; Suponga que $V$ es de dimensión finita y $U$ es un subespacio de $V$. Necesitamos demostrar que $U$ es de dimensión finita. Hacemos esto a través de la siguiente construcción de  pasos. 
	\begin{enumerate}[\small\textbf{Paso} 1.]
	    \item Si $u=\left\{0\right\}$, entonces $U$ es de dimensión finita por lo que hemos terminado. Si $U\neq \left\{0\right\}$, entonces elegimos un vector no nulo $v_1\in U$
	    \item Si $U=\span(v_1,\ldots,v_{j-1})$, entonces $U$ es de dimensión finita por lo que hemos terminado. Si $U\neq \span(v_1,\ldots,v_{j-1})$, entonces elegimos un vector $v_j\in U$ tal que 
	    $$v_j\in \span(v_1,\ldots,v_{j-1}).$$
	\end{enumerate}
	Después de cada paso, mientras continúa el proceso, hemos construido una lista de vectores tal que ningún vector en esta lista está en el generador de los vectores anteriores. Así, después de cada paso hemos construido una lista linealmente independiente, por el lema de dependencia lineal (2.21). Esta lista linealmente independiente no puede ser más grande que cualquier lista de expansión de $V$ (por 2,23). Por lo tanto, el proceso eventualmente termina, lo que significa que $U$ es de dimensión finita.
\end{myteo}


\mysection{Bases}

\setcounter{mydef}{26}
%------------------- Definición 2.27
\begin{mydef}[Base]\;\\\\
    Una base de $V$ es una lista de vectores en $V$ que es linealmente independiente y genera $V$.
\end{mydef}

%------------------- teorema 2.29
\begin{myteo}[Criterio de base]\;\\\\
    Una lista $v_1,\ldots,v_n$ de vectores en $V$ es una base de $V$ si y sólo si cada $v\in V$ puede escribirse unívocamente de la forma 
    $$v=a_1v_1+\cdots+a_nv_n.$$
    donde $a_1,\ldots,a_n\in \textbf{F}$.\\\\
	Demostración.-\; Primero suponga que $v_1,\ldots,v_n$ es una base de $V$. Ya que, $v_1,\ldots,v_n$ genera $V$, existe $a_1,\ldots,a_n\in \textbf{F}$ tal que 
	$$v=a_1v_1+\cdots+a_nv_n$$ 
	se cumple. Para mostrar que esta representación es única, sean $c_1,\ldots,c_n$ escalares tales que, también tenemos
	$$v=c_1v_1+\cdots+c_nv_n.$$
	Sustrayendo la primera ecuación de la segunda, tenemos
	$$0=(a_1-c_1)v_1+\cdots+(a_n-c_n)v_n.$$
	Esto implica que cada $a_j-c_j$ es igual a cero. (Ya que, $v_1,\ldots,v_n$ es linealmente independiente). Por lo tanto $a_1=c_1,\ldots,a_n=c_n$. Así, tenemos la unicidad deseada.\\

	Por otro lado, suponga que cada $v\in V$ puede ser escrita de manera única como la forma $v=a_1v_1+\cdots+a_nv_n$. Claramente esto implica que $v_1,\ldots,v_n$ genera $V$. Para demostrar que $v_1,\ldots,v_n$ es linealmente independiente, suponga que $a_1,\ldots,a_n\in \textbf{F}$ son tales que
	$$0=a_1v_1+\cdots+a_nv_n.$$
	La unicidad de la representación de $v=a_1v_1+\cdots+a_nv_n$ (tomando $v=0$) implica que $a_1=\cdots=a_n=0$. Así, $v_1,\dots,v_n$ es linealmente independiente y por tanto es una base de $V$.
\end{myteo}

Una lista generadora en un espacio vectorial puede no ser una base, ya que no es linealmente independiente. Nuestro próximo resultado dice que dada cualquier lista generadora, algunos (posiblemente ninguno) de los vectores en ella pueden descartarse para que la lista restante sea linealmente independiente y aún genere el espacio vectorial.

\setcounter{myteo}{30}
%------------------- teorema 2.31
\begin{myteo}[La lista generadora contiene un base]\,\\\\
    Cada lista generadora en un espacio vectorial se puede reducir a una base del espacio vectorial.\\\\
	Demostración.-\; Suponga que $v_1,\ldots,v_n$ genera $V$. Queremos eliminar algunos de los vectores de  $v_1,\ldots,v_n$ para que los vectores restantes formen una base de $V$. Sea $B=v_1,\ldots,v_n$, de donde realizamos un bucle con las siguientes condiciones:
	\begin{enumerate}[\textbf{Paso} 1.]
	    \item Si $v_1=0$, eliminamos $v_1$ de $B$. Si $v_1\neq 0$ entonces no cambiamos $B$.
	    \item[\textbf{Paso} J.] Si $v_j$ esta en $\span(v_1,\ldots,v_{j-1})$, eliminamos $v_j$ de $B$. Si $v_j$ no está en $\span(v_1,\ldots,v_j)$, entonces no cambiamos $B$ (lema 2.21). 
	\end{enumerate}
	Paramos el proceso después del paso $n$, obteniendo una lista $B$. Esta lista $B$ genera $V$, ya que nuestra lista original generó $V$ y hemos descartado solo los vectores que ya estaban en el generador de los vectores anteriores. El proceso garantiza que ningún vector de $B$ está en el generador de los anteriores. Así pues, $B$ es linealmente independiente, por el lema de dependencia lineal (2.21). Por tanto, $B$ es una base de $V$.
\end{myteo}

Nuestro siguiente resultado, un corolario fácil del resultado anterior, nos dice que todo espacio vectorial de dimensión finita tiene una base.

\setcounter{mycor}{31}
%------------------- corolario 2.32
\begin{mycor}[Base del espacio vectorial de dimensión finita]\,\\\\
    Cada espacio vectorial de dimensión finita tiene una base.\\\\
	Demostración.-\; Por definición, un espacio vectorial de dimensión finita tiene una lista generadora. El resultado anterior nos dice que cada lista generadora puede ser reducida a una base.
\end{mycor}

Nuestro siguiente resultado es en cierto sentido un derivado de 2.31, que decía que toda puede reducirse a una base. Ahora mostramos que dada cualquier lista linealmente independiente, podemos unir algunos vectores adicionales (esto incluye la posibilidad de no unir ningún vector adicional) de modo que la lista ampliada siga siendo linealmente independiente pero que también genere el espacio.

\setcounter{myteo}{32}
%------------------- teorema 2.33
\begin{myteo}[Una lista linealmente independiente se extiende a una base]\,\\\\
    Cada lista linealmente independiente de vectores en un espacio vectorial se puede extender a una base del espacio vectorial.\\\\
	Demostración.-\; Suponga $u_1,\ldots,u_m$ es linealmente independiente en un espacio vectorial $V$ de dimensión finita. Sea $w_1,\ldots,w_n$ una base de $V$. Por lo que la lista
	$$u_1,\ldots,u_m,\; w_1,\ldots.w_n$$
	genera $V$. Aplicando el procedimiento de la prueba de 2.31 para reducir esta lista a una base de $V$ se obtiene una base formada por los vectores $u_1, \ldots , u_m$ (ninguna de las $u's$ se elimina en este procedimiento porque $u_1,\ldots, u_m$ es linealmente independiente) y algunos de los $w'$.
\end{myteo}

Como aplicación del resultado anterior, mostramos ahora que cada subespacio de un espacio vectorial de dimensión finita puede emparejarse con otro subespacio para formar una suma directa de todo el espacio.

%------------------- teorema 2.34
\begin{myteo}[Cada subespacio de \boldmath$V$ forma parte de una suma directa igual a \boldmath$V$]\;\\\\
    Suponga $V$ es de dimensión finita y $U$ es un subespacio de $V$. Entonces, existe un subespacio $W$ de $V$ tal que $U\oplus W=V$.\\\\
	Demostración.-\; Ya que $V$ es de dimensión finita, también lo es $U$ por 2.26. Por lo que, existe una base $u_1,\ldots,u_m$ de $U$ esto por 2.32. Por su puesto $u_1,\ldots,u_m$ es una lista linealmente independiente de vectores en $V$. Por lo tanto, esta lista puede extenderse a una base $u_1,\ldots,u_n$ de $V$ esto por 2.33. Sea $W=\span(w_1,\ldots,w_n)$. Para probar que $V=U\oplus W,$ por 1.45, solo necesitamos demostrar que
	$$V=U+W \quad \mbox{y}\quad U\cap W=\left\{0\right\}.$$
	Para probar la primera ecuación, suponga $v\in V$. Entonces, ya que la lista $u_1,\ldots,u_m$, $w_1,\ldots,w_n$ genera $V$, existe $a_1,\ldots,a_m$, $b_1,\ldots,b_n\in \textbf{F}$ tal que
	$$v=a_1u_1+\cdots+a_mu_m+b_1w_1+\cdots+b_nw_n.$$
	En otras palabras, tenemos $v=u+w$, donde $u\in U$ y $w\in W$ fueron definidas anteriormente. Así, $v\in U+W$, completando la prueba de $V=W+W$.\\
	Para demostrar que $U\cap W=\left\{0\right\}$, suponga $v\in U\cap W$. Entonces, existe escalares $a_1,\ldots,a_m$, $b_1,\ldots,b_n\in \textbf{F}$ tal que
	$$v=a_1u_1+\cdots+a_mu_m=b_1w_1+\cdots+b_nw_n.$$
	Por lo tanto,
	$$a_1u_1+\cdots+a_mu_m-b_1w_1-\cdots-b_nw_n=0.$$
	Esto, ya que $u_1,\ldots,u_m$, $w_1,\ldots,w_n$ son linealmente independientes, esto implica que $a_1=\cdots=a_m=b_1=\cdots=b_n=0$. Así, $v=0$, de donde $U\cap W= \left\{0\right\}$.
\end{myteo}




\mysection{Dimensión}

%------------------------ 2.35 teorema
\begin{myteo}[La longitud de la base no depende de la base]\,\\\\
    Dos bases cualquier de un espacio vectorial de dimensión finita  tiene la misma longitud.\\\\
	Demostración.-\; Suponga $V$ es de dimensión finita. Sean $B_1$ y $B_2$ dos bases de $V$. Entonces $B_1$ es linealmente independiente en $V$ y $B_2$ genera $V$. Así la longitud de la lista de $B_1$ es como máximo la longitud de la lista de $B_2$ (por 2.23). Intercambiando los roles de $B_1$ y $B_2$. tenemos también que la longitud de la lista de $B_2$ es como máximo la longitud de la lista de $B_1$. Por lo tanto la longitud de $B_1$ es igual a la longitud de $B_2$, como se desea.
\end{myteo}

Ahora que sabemos que dos bases cualesquiera de un espacio vectorial de dimensión finita tienen la misma longitud, podemos definir formalmente la dimensión de tales espacios.

%------------------------ 2.36 definicion
\begin{mydef}[Dimensión, \boldmath $\dim V$]\;\\
    \begin{itemize}
	\item La \textbf{dimensión} de un espacio vectorial de dimensión finita es la longitud de cualquier base del espacio vectorial.
	\item La dimensión de $V$ (si $V$ es de dimensión finita) se denota por $\dim V$.
    \end{itemize}
\end{mydef}

\setcounter{myteo}{37}
%------------------------ 2.37 teorema
\begin{myteo}[Dimensión de un subespacio]\, \\\\
    Si $V$ es de dimensión finita y $U$ es un subespacio de $V$, entonces $\dim U\leq \dim V$.\\\\
	Demostración.-\; Suponga $V$ de dimensión finita y $U$ es un subespacio de $V$. Piense en una base de $U$ como una lista linealmente independiente en $V$, y piense en una base de $V$ como una lista generadora en V. Ahora use $2.23$ para concluir que $\dim U\leq \dim V$.
\end{myteo}

Para comprobar que una lista de vectores en $V$ es una base de $V$, debemos, según la definición, demostrar dos propiedades: debe ser linealmente independiente y debe generar $V$. Los siguientes dos resultados muestran que si la lista en cuestión tiene la longitud correcta, entonces solo necesitamos verificar que satisface una de las dos propiedades requeridas. Primero probamos que toda lista linealmente independiente con la longitud correcta es una base.

%------------------------ 2.39 teorema
\begin{myteo}[La lista linealmente independiente de la longitud correcta es una base]\,\\\\
    Suponga que $V$ es de dimensión finita. Entonces, toda lista linealmente independiente de vectores en $V$ con longitud $\dim V$ es una base de $V$.\\\\
	Demostración.-\; Suponga $\dim V=n$ y $v_1,\ldots,v_n$ es linealmente independiente en $V$. La lista $v_1,\ldots,v_n$ puede ser extenderse a una base de $V$ (por 2.33). Sin embargo, toda base de $V$ tiene una longitud $n$, por lo que en este caso la extensión es trivial, lo que significa que ningún elemento está unido a $v_1,\ldots, v_n$ En otras palabras, $v_1,\ldots, v_n$ es una base de $V$, como deseamos.
\end{myteo}

%------------------------ 2.40 ejemplo
\begin{myejem}
    Demostrar que la lista $(5,7),(4,3)$ es una base de $\textbf{F}^2$.\\\\
	Demostración.-\; Esta lista de dos vectores en $\textbf{F}^2$ es obviamente linealmente independiente (poque ningún vector es un multiplo escalar del otro). Notemos que $\textbf{F}^2$ tiene dimensión $2$. Así, 2.39 implica que la lista linealmente independiente $(5,7),(4,3)$ de longitud $2$ es una base de $\textbf{F}^2$ $\left(\right.$ no necesitamos comprobar que genera $\left.\textbf{F}^2\right)$. 
\end{myejem}

%------------------------ 2.41 ejemplo
\begin{myejem}
    Demostrar que $1,(x-5)^2,(x-5)^3$ es una base del subespacio $U$ de $\mathcal{P}_3(\textbf{R})$ defina por
    $$U=\left\{p\in \mathcal{P}_3(\textbf{R})\; : \; p'(5)=0\right\}.$$
	Solución.-\; Claramente cada los polinomios $1,(x-5)^2$ y $(x-5)^3$ es en $U$. Suponga $a,b,c \in \textbf{R}$ y 
	$$a+b(x-5)^2+c(x-5)^3=0$$
	para cada $x\in \textbf{R}.$ Vemos que el lado izquierdo de la ecuación anterior tiene un termino $cx^3$. Como el lado derecho no tiene un término $x^3$, esto implica que $c = 0$. Como $c = 0$, vemos que el lado izquierdo tiene un término $bx^2$, lo que implica que $b = 0$. Como $b = c = 0$, podemos también concluye que $a = 0$. Así, la ecuación implica que $a=b=c=0$. Por lo que la lista dada es linealmente independiente en $U$.\\

	Observemos que $\dim U\geq 3$, ya que $U$ es un subespacio de $\mathcal{P}_3(\textbf{F})$, sabemos que $\dim U \leq \dim \mathcal{P}_3(\textbf{F})=4$ (por 2.38). Sin embargo, $\dim U$ no puede ser igual a $4$, de lo contrario, cuando extendemos una base de $U$ a una base de $\mathcal{P}_3(\textbf{F})$ obtendríamos una lista con una longitud mayor que $4$. Por lo tanto, $\dim U = 3$. De esto modo, 2.39 implica que la lista linealmente independiente $1,(x-5)^2,(x-5)^3$ es una base de $U$.
\end{myejem}

Ahora probaremos que una lista generadora con longitud correcta es una base.

%------------------------ 2.42 teorema
\begin{myteo}[Lista generadora de la longitud correcta es una base]\,\\\\
	Suponga que $V$ es de dimensión finita. Entonces, cada lista generadora de vectores en $V$ con longitud $\dim V$ es una base de $V$.\\\\
	    Demostración.-\; Suponga $\dim V=n$ y $v_1,\ldots,v_n$ genera $V$. La lista $v_1,\ldots,v_n$ puede ser reducido a una base de $V$ (por 2.31). Sin embargo, cada base de $V$ tiene longitud $n$. En este caso la reducción es trivial, lo que significa que no se eliminan elementos de $v_1,\ldots,v_n$. En otras palabras, $v_1,\ldots,v_n$ es una base de $V$, como deseamos.
\end{myteo}

El siguiente resultado da una fórmula para la dimensión de la suma de dos subespacios de un espacio vectorial de dimensión finita.

%------------------------ 2.43 teorema
\begin{myteo}[Dimensión de una suma]\,\\\\
    Si $U_1$ y $U_2$ son subespacios de un espacio vectorial de dimensión finita, entonces
    $$\dim(U_1+U_2)=\dim U_1+\dim U_2 - \dim(U_1\cap U_2).$$\\
	Demostración.-\; Sea $u_1,\ldots,u_m$ una base de $U_1\cap U_2$; por lo que $(U_1\cap U_2)=m$. Ya que, $u_1,\ldots,u_m$ es una base de $U_1\cap U_2$, es linealmente independiente en $U_1$. En consecuencia esta lista puede ser extendida a una base $u_1,\ldots,u_m,\, v_1,\ldots,v_j$ de $U_1$ (por 2.33). Así, $\dim U_1=m+j$. También extendamos $u_1,\ldots,u_m$ a una base $u_1,\ldots,u_m,\, w_1,\ldots,w_k$ de $U_2$; de donde $U_2=m+k$.\\
	Tendremos que demostrar
	$$u_1,\ldots,u_m,\, v_1,\ldots,v_j,\,w_1,\ldots,w_k$$
	es una base de $U_1+U_2$. Claramente $\span(u_1,\ldots,u_m,\, v_1,\ldots,v_j,\,w_1,\ldots,w_k)$ contiene a $U_1$ y $U_2$, por lo que es igual a $U_1+U_2$. Entonces, para mostrar que la lista es una base de $U_1+U_2$, necesitamos demostrar que es linealmente independiente. Suponga
	$$a_1u_1+\cdots+a_mu_m+b_1v_1+\cdots+b_jv_j+c_1w_1+\cdots+c_kw_k=0,$$
	donde todos los $a's,b's$ y $c's$ son escalares. Podemos reescribir la ecuación anterior como
	$$c_1w_1+\cdots+c_kw_k=-a_1u_1-\cdots-a_mu_m-b_1v_1-\cdots-b_jv_j$$
	el cual demuestra que $c_1w_1+\cdots+c_kw_k\in U_1$. Luego, todos los $w's$ son en $U_2$, esto implica que $c_1w_1+\cdots+c_kw_k\in U_1\cap U_2$. Ya que, $u_1,\ldots,u_m$ es una base de $U_1\cap U_2$, podemos escribir lo siguiente
	$$c_1w_1+\cdots+c_kw_k=d_1u_1+\cdots+d_m u_m$$
	para algunos escalares $d_1,\ldots,d_m$. Pero $u_1,\ldots,u_m,w_1,\ldots,w_k$ es linealmente independiente, así la útlima ecuación implica que todos los $c's$ (y $d's$) son igual a cero. \\
	
	Por lo tanto, nuestra ecuación original que involucra las $a's, b's$ y $c's$ se convierten en 
	$$a_1u_1+\cdots+a_mu_m+b_1v_1+\cdots+b_jv_j=0.$$
	Porque la lista $u_1,\ldots, u_m, v_1, \ldots, v_j$ es linealmente independiente, esta ecuación implica que todas las $a's$ y $b's$ son $0$. Ahora sabemos que todas las $a's$, $b's$ y $c's$ son iguales a $0$, como deseamos.
\end{myteo}


