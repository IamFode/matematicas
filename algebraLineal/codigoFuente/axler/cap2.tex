\chapter{Espacios vectoriales de dimensión finita}

\mysection{Span e independencia lineal}

\setcounter{mydef}{1}
%%%%%%%%%%%%%%%%%%%%%%%%%%%%%%%%%%%%%%%%% 2.A %%%%%%%%%%%%%%%%%%%%%%%%%%%%%%%%%%%%%%%%%%%%%%%

%-------------------- 2.2 Notación 
\begin{mynot}[Lista of vectores]\;\\\\
    Por lo general, escribiremos listas de vectores sin paréntesis alrededor.
\end{mynot}
\vspace{0.5cm}

\subsection*{Combinaciones lineales y generadores}

%-------------------- 2.3 Definición 
\begin{mydef}[Combinación lineal] \;\\\\
    Una \textbf{combinación lineal} de una lista $v_1,\ldots , v_m$ de vectores en $V$ es un vector de la forma 
    $$a_1v_1+\cdots + a_mv_m,$$
    donde $a_1,\ldots a_m \in \textbf{F}.$
\end{mydef}

\setcounter{mydef}{4}
%-------------------- 2.5 Definición
\begin{mydef}[Span o generador] \;\\\\
    El conjunto de todas las combinaciones lineales de una lista de vectores $v_1,\ldots,v_m$ en $V$ se denomina \textbf{generador} de $v_1,\ldots,v_m$, denotado por $\span(v_1,\ldots, v_m)$. En otras palabras,
    $$\span(v_1,\ldots,v_m)=\left\{a_1v_1+\cdots + a_mv_m : a_1,\ldots, a_m \in \textbf{F}\right\}.$$
    El span de la lista vacía $()$ es definida por $\left\{0\right\}.$
\end{mydef}

\setcounter{myteo}{6}
%-------------------- 2.7 Teorema 
\begin{myteo}[Span es el subespacio más pequeño que lo contiene]\,\\\\
    El \textbf{span} de una lista de vectores en $V$ es el subespacio más pequeño de $V$ que contiene todos los vectores de la lista.\\\\
	Demostración.-\; Suponga que $v_1,\ldots,v_m$ es una lista de vectores en $V$. Primero demostraremos que $\span(v_1,\ldots,v_m)$ es un subespacio de $V$. El $0$ está en $\span(v_1,\ldots,v_m)$, porque
	$$0=0v_1+\ldots + 0v_m.$$
	También, $\span(v_1,\ldots,v_m)$ es cerrado bajo la suma, ya que
	$$(a_1v_1+\cdots + a_mv_m)+(c_1v_1+\cdots+c_mv_m)=(a_1+c_1)v_1+\cdots + (a_m+c_m)v_m.$$
	Además, $\span(v_1,\ldots,v_m)$ es cerrado bajo la multiplicación por un escalar, dado que
	$$\lambda(a_1v_1+\cdots + a_mv_m)=\lambda a_1v_1+\cdots + \lambda a_mv_m.$$
	Por lo tanto, $\span(v_1,\ldots, v_m)$ es un subespacio de $V$. Esto por 1.34.\\

	Cada $v_j$ es una combinación lineal de $v_1,\ldots,v_m$  (para mostrar esto, establezca $a_j=1$ y que las otras $a'$s en la definición de combinación lineal sean iguales a $0$). Así, el $\span(v_1,\ldots,v_m)$ contiene a cada $v_j$. Por otra parte, debido a que los subespacios están cerrados bajo la multiplicación de escalares y la suma, cada subespacio de $V$ que contiene a cada $v_j$ contiene a $\span(v_1,\ldots,v_m)$. Por lo tanto, $\span(v_1,\ldots,v_m)$ es el subespacio más pequeño de $V$ que contiene todos los demás vectores $v_1,\ldots,v_m$.
\end{myteo}

%-------------------- 2.8 Definición
\begin{mydef}[Spans]\,\\\\
    Si $\span(v_1,\ldots,v_m)$ es igual a $V$, decimos que $v_1,\ldots, v_m$ genera $V$.
\end{mydef}

\setcounter{mydef}{9}
%-------------------- 2.10 Definición
\begin{mydef}[Espacio vectorial de dimensión finita]\,\\\\
    Un espacio vectorial se llama finito-dimensional si alguna lista de vectores en él genera el espacio.\\

    Es decir, si hay una lista de vectores en un espacio vectorial finito-dimensional que puede abarcar todo el espacio, entonces podemos decir que el espacio vectorial es finito-dimensional.
\end{mydef}

%-------------------- 2.11 Definición
\begin{mydef}[Polinomio, $\boldmath \mathcal{P}\left(\textbf{F}\right)$]\,\\
    \begin{itemize}
	\item Una función $p:\textbf{F}\to \textbf{F}$ es llamado polinomio con coeficientes en $\textbf{F}$ si existe $a_0,\ldots,a_m\in \textbf{F}$ tal que
	$$p(z)=a_0+a_1z+a_2z^2+\cdots + a_mz^m$$
	para todo $z\in \textbf{F}$.

	\item $\mathcal{P}(\textbf{P})$ es el conjunto de todos los polinomios con coeficientes en $\textbf{F}$.
    \end{itemize}
\end{mydef}

Con las operaciones usuales de adición y multiplicación escalar, $\mathcal{P}(\textbf{F})$ es un espacio vectorial sobre $\textbf{F}$. En otras palabras, $\mathcal{P}(\textbf{F})$ es un subespacio de $\textbf{F}^{\textbf{F}}$, el espacio vectorial de funciones de $\textbf{F}$ en $\textbf{F}$. \\

Los coeficientes de un polinomio están determinados únicamente por el polinomio. Así, la siguiente definición define de manera única el grado de un polinomio.

%-------------------- 2.12 Definición
\begin{mydef}[Grado de un polinomio, $\mathbold deg\; p$]\,\\
    \begin{itemize}
	\item Un polinomio $p\in \mathcal{P}(F)$ se dice que tiene \textbf{grado} $m$ si existen escalares $a_0,a_1,\ldots,a_m\in \textbf{F}$ con $a_m\neq 0$ tal que
	$$p(z)=a_0+a_1z+\cdots+a_mz^m$$
	para todo $z\in \textbf{F}$. Si $p$ tiene grado $m$, escribimos $deg\; p=m$.
	\item El polinomio que es identicamente $0$ se dice que tiene \textbf{grado} $-\infty$.
    \end{itemize}
\end{mydef}

%-------------------- 2.13 Definición
\begin{mydef}[$\boldmath \mathcal{P}_m\left(\textbf{F}\right)$]\;\\\\
    Para $m$ un entero no negativo, $\mathcal{P}_m(\textbf{F})$ denota el conjunto de todos los polinomios con coeficiente en $\textbf{F}$ y grado no mayor a $m$.
\end{mydef}

Verifiquemos el siguiente ejemplo, tenga en cuenta que $\mathcal{P}_m(\textbf{F})=\span\left(1,z,\ldots,z^m\right)$; aquí estamos abusando ligeramente de la notación al permitir que $z^k$ denote una función.

\setcounter{mydef}{14}
%-------------------- 2.15 definición
\begin{mydef}[Espacio vectorial de dimensión infinita]\,\\\\
    Un espacio vectorial se llama \textbf{infinitamente-dimensional} si no es de dimensión finita.
\end{mydef}

%-------------------- 2.16 ejemplo
\begin{myejem}
    Demuestre que $\mathcal{P}(\textbf{F})$ es infinitamente-dimensional.\\\\
    Demostración.-\; Considere cualquier lista de elementos de $\mathcal{P}(\textbf{F})$. Sea $m$ el grado más alto de los polinomios en esta lista. Entonces, cada polinomio en el generador (span) de esta lista tiene grado máximo $m$. Por lo tanto, $z^{m+1}$ no está en el span de nuestra lista. Así, ninguna lista genera $\mathcal{P}(\textbf{F})$. Concluimos que $\mathcal{P}(\textbf{F})$ es de dimensión infinita.
\end{myejem}
\vspace{.5cm}

\subsection*{Independencia lineal}

Suponga $v_1,\ldots,v_m\in V$ y $v\in \span(v_1,\ldots,v_m)$. Por la definición de span, existe $a_1,\ldots,a_m\in \textbf{F}$ tal que
$$v=a_1v_1+\cdots+a_mv_m.$$
Considere la cuestión de si la elección de escalares en la ecuación anterior es única. Sea $c_1,\ldots,c_m$ otro conjunto de escalares tal que
$$v=c_1v_1+\cdots+c_mv_m.$$
Sustrayendo estas últimas ecuaciones, se tiene
$$0=(a_1-c_1)v_1+\cdots+(a_m-c_m)v_m.$$
Así, tenemos que escribir $0$ como una combinación lineal de $(v_1,\ldots,v_m)$. Si la única forma de hacer esto es la forma obvia (usando $0$ para todos los escalares), entonces cada $a_j-c_j$ es igual a $0$, lo que significa que cada $a_j$ es igual a $c_j$ (y por lo tanto la elección de los escalares fue realmente única). Esta situación es tan importante que le damos un nombre especial, independencia lineal, que ahora definiremos.

%-------------------- 2.17 Definición
\begin{mydef}[Linealmente independiente]\,\\
    \begin{itemize}
	\item Una lista $v_1,\ldots,v_m$ de vectores en $V$ se llama linealmente independiente si la única posibilidad de que  $a_1,\ldots,a_m\in \textbf{F}$ tal que $a_1v_1+\cdots+a_mv_m$ sea igual a $0$ es $a_1=\cdots=a_m=0.$
	\item La lista vacía $()$ también se declara linealmente independiente.
    \end{itemize}
\end{mydef}

El razonamiento anterior muestra que $v_1,\ldots,v_m$ es linealmente independiente si y sólo si cada vector en el $\span(v_1,\ldots,v_m)$ tiene sólo una representación lineal en forma de combinación lineal de $v_1,\ldots,v_m$.

\setcounter{mydef}{18}
%-------------------- 2.19 Definición
\begin{mydef}[Linealmente dependiente]\,\\
    \begin{itemize}
	\item Una lista $v_1,\ldots,v_m$ de vectores en $V$ se llama linealmente dependiente si no es linealmente independiente.
	\item En otras palabras, una lista $v_1,\ldots,v_m$ de vectores en $V$ es linealmente dependiente si existe $a_1,\ldots,a_m\in \textbf{F}$, no todos $0$, tal que $a_1v_1+\cdots+a_mv_m=0$.
    \end{itemize}
\end{mydef}

\setcounter{mylema}{20}
%-------------------- 2.21 Lema
\begin{mylema}
    Suponga $v_1,\ldots,v_m$ es una lista linealmente dependiente en $V$. Entonces, existen $j\in\left\{1,2,\ldots,m\right\}$ tal que se cumple lo siguente:
    \begin{enumerate}[(a)]
	\item $v_j\in \span(v_1,\ldots,v_{j-1});$
	\item Si el j-ésimo término se elimina de $v_1,\ldots,v_m$, el generador de la lista restante es igual a $\span(v_1,\ldots,v_m)$.\\
    \end{enumerate}
    Demostración.-\; Ya que la lista $v_1,\ldots,v_m$ es linealmente dependiente, existe números $a_1,\ldots,a_m\in \textbf{F}$, no todos $0$, tal que
    $$a_1v_1+\cdots + a_mv_m=0.$$
    Sea $j$ el elemento más grande de $\left\{1,\ldots, m\right\}$ tal que $a_j\neq 0$. Entonces,
    $$v_j=-\dfrac{a_1}{a_j}v_1-\cdots-\dfrac{a_{j-1}}{a_j}v_{j-1}\qquad (1).$$
    Lo que prueba (a). \\
    Para probar (b), suponga $u\in \span(v_1,\ldots,v_m)$. Entonces, existe números $c_1,\ldots,c_m\in \textbf{F}$ tal que
    $$u=c_1v_1+\cdots+c_mv_m.$$
    En la ecuación de arriba, podemos reemplazar $v_j$ con el lado derecho de (1). Es decir,
    $$u=c_1v_1+\cdots+a_jv_j+\cdots +c_mv_m.$$
    lo que muestra que $u$ está en el span de la lista obtenida al eliminar el j-ésimo término de $v_1,\ldots,v_m$. Así (b) se cumple.
\end{mylema}

Eligir $j=1$ en el lema de dependencia lineal anterior, significa que $v_1=0$, porque si $j=1$ entonces la condición (a) anterior se interpreta como que $v_1\in \span()$. Recuerde que $\span()=\left\{0\right\}$. Tenga en cuenta también que la demostración del inciso (b) debe modificarse de manera obvia si $v_i=0$ y $j=1$.\\

Ahora llegamos a un resultado importante. Dice que ninguna lista linealmente independiente en $V$ es más extensa que una lista generadora en V.

\setcounter{myteo}{22}
%-------------------- 2.23 teorema
\begin{myteo}[La longitud de una lista linealmente independiente es \boldmath$\leq$ a la longitud de la lista generadora]\;\\\\
    En un espacio vectorial de dimensión finita, la longitud de cada lista linealmente independiente de vectores es menor o igual que la longitud de cada lista generadora de vectores (longitud=n de vectores).\\\\ 
    Demostración.-\; Suponga $u_1,\ldots,u_m$ es linealmente independiente en $\textbf{V}$. Suponga también que $w_1,\ldots,w_n$ generan $V$. Necesitamos probar que $m\leq n$. Lo hacemos a través del proceso de pasos que se describe a continuación; tenga en cuenta que en cada paso agregamos una de las $u's$ y eliminamos una de las $w's$.

    \begin{enumerate}[\small\textbf{Paso} 1.]
	\item Sea $B$ la lista $w_1,\ldots,w_n$, que genera $V$. Por lo tanto, añadir cualquier vector en $V$ a esta lista produce una lista linealmente dependiente (porque el nuevo vector añadido se puede escribir como una combinación lineal de los otros vectores, $u_1=\frac{a_1}{c_1}w_1+\cdots+\frac{a_n}{c_1}$). En particular, la lista
	$$u_1,w_1,\ldots,w_n$$
	es linealmente dependiente. Así, por el lema (2.21), podemos eliminar un de las $w$ para que la nueva lista $B$ (de longitud $n$) que consta de $u_1$ y las $w's$ restantes generen $V$.
	\item[\textbf{Paso} j.] La lista $B$ (de longitud $n$) del paso $j-1$ genera $V$. Así, añadir cualquier vector a esta lista produce una lista linealmente dependiente. En particular, la lista de longitud $(n+1)$ obtenida al unir $u_j$ a $B$, colocándola justo después de $u_1,\ldots,u_{j-1}$, es linealmente dependiente. Por el lema de dependencia lineal (2.21), uno de los vectores de esta lista está en el generador de los anteriores, y ya que $u_1,\ldots,u_j$ es linealmente independientes, este vector es uno de los $w's$, no uno de los $u's$. Podemos eliminar esa $w$ para la nueva lista $B$ (de longitud $n$) que consta de $u_1,\ldots,u_j$ y las $w's$ restantes generan $V$.
    \end{enumerate}
    Después del paso $m$, hemos agregado todas las $u$ y el proceso se detiene. En cada paso, a medida que agregamos un $u$ a $B$, el lema de dependencia lineal implica que hay algo de $w$ que eliminar. Por lo tanto, hay al menos tantas $w$ como $u$.
\end{myteo}

Aclaremos esto con dos ejemplos.

%-------------------- 2.24 Ejemplo
\begin{myejem}
    Demuestre que la lista $(1,2,3),(4,5,8),(9,6,7),(-3,2,8)$ no es linealmente independiente en $\textbf{R}^3$.\\\\
	Demostración.-\; La lista $(1,0,0),(0,1,0),(0,0,1)$ genera $\textbf{R}^3$. Por lo tanto, ninguna lista de longitud superior a $3$ es linealmente independiente en $\textbf{R}^3$.
\end{myejem}

%-------------------- 2.25 Ejemplo
\begin{myejem}
    Demostrar que la lista $(1,2,3,-5),(4,5,8,3),(9,6,7,-1)$ no genera $\textbf{R}^4$.\\\\	
	Demostración.-\; La lista $(1,0,0,0),(0,1,0,0),(0,0,1,0),(0,0,0,1)$ es linealmente independiente en $\textbf{R}^4$. Por lo tanto, ninguna lista de longitud inferior a $4$  genera $\textbf{R}^4$.
\end{myejem}

\setcounter{myteo}{25}
%-------------------- 2.26 Teorema
\begin{myteo}[Subespacio de dimensión finita]\; \\\\
    Todo subespacio de un vector de dimensión finita es de dimensión finita.\\\\
	Demostración.-\; Suponga que $V$ es de dimensión finita y $U$ es un subespacio de $V$. Necesitamos demostrar que $U$ es de dimensión finita. Hacemos esto a través de la siguiente construcción de  pasos. 
	\begin{enumerate}[\small\textbf{Paso} 1.]
	    \item Si $u=\left\{0\right\}$, entonces $U$ es de dimensión finita por lo que hemos terminado. Si $U\neq \left\{0\right\}$, entonces elegimos un vector no nulo $v_1\in U$
	    \item Si $U=\span(v_1,\ldots,v_{j-1})$, entonces $U$ es de dimensión finita por lo que hemos terminado. Si $U\neq \span(v_1,\ldots,v_{j-1})$, entonces elegimos un vector $v_j\in U$ tal que 
	    $$v_j\in \span(v_1,\ldots,v_{j-1}).$$
	\end{enumerate}
	Después de cada paso, mientras continúa el proceso, hemos construido una lista de vectores tal que ningún vector en esta lista está en el generador de los vectores anteriores. Así, después de cada paso hemos construido una lista linealmente independiente, por el lema de dependencia lineal (2.21). Esta lista linealmente independiente no puede ser más grande que cualquier lista de expansión de $V$ (por 2,23). Por lo tanto, el proceso eventualmente termina, lo que significa que $U$ es de dimensión finita.
\end{myteo}

\vspace{.5cm}


\setcounter{mysection}{0}
\mysection{Ejercicios}

\begin{enumerate}[\bfseries 1.]

    %------------------- 1
    \item Suponga $v_1,v_2,v_3,v_4$ se extiende por $V$. Demostrar que la lista 
    $$v_1-v_2,v_2-v_3,v_3-v_4,v_4$$
    también se extiende por $V$.\\\\
	Demostración.-\; Sea $v\in V$, entonces existe $a_1,a_2,a_3,a_4$ tal que 
	$$v=a_1v_1+a_2v_2+a_3v_3+a_4v_4.$$
	Que implica,
	$$
	\begin{array}{rcl}
	    v&=&a_1v_1+a_2v_2+a_3v_3+a_4v_4-a_1v_2+a_1v_2-a_1v_3+a_1v_3-a_2v_3+a_2v_3-a_1v_4+a_1v_4\\
	     &-& a_2v_4+a_2v_4 -a_3v_4 +a_3v_4\\
	\end{array}
	$$

	De donde,

	$$v=a_1(v_1-v_2)+(a_1+a_2)(v_2-v_3)+(a_1+a_2+a_3)(v_3-v_4)+(a_1+a_2+a_3+a_4)v_4.$$
	Por lo tanto, cualquier vector en $V$ puede ser expresado por una combinación lineal de 
	$$v_1-v_2,v_2-v_3,v_3-v_4,v_4.$$
	Así, esta lista se extiende por $V$.\\\\

    %------------------- 2
    \item Verifique las afirmaciones del Ejemplo 2.18.\\

	\begin{enumerate}[(a)]

	    %---------- (a)
	    \item Una lista $v$ de un vector $v\in V$ es linealmente independiente si y sólo si $v\neq 0$.\\\\
		Demostración.-\; Demostremos que si $v$ es linealmente independiente, entonces $v\neq 0$. Supongamos que $v=0$. Sea un escalar $a\neq 0$. De donde, $av=0$ incluso cuando $a\neq 0$. Esto contradice la definición de independencia lineal. Por lo tanto, $v$ debe ser linealmente dependiente. Esto es, $v=0$ implica que $v$ es un vector linealmente dependiente. Por lo que, si $v$ es linealmente independiente, entonces $v$ es un vector distinto de cero.\\
		Por otro lado, debemos demostrar que  $v\neq 0$ implica que $v$ es linealmente independiente. Sea  un escalar $a$ tal que $av=0$. Si $a\neq 0$, entonces $av$ no puede ser $0$. Por eso $a$ debe ser $0$. Por lo tanto, $v\neq 0$ y $av=0$ implica que $a=0$. Así, $v$ es linealmente independiente.\\\\

	    %---------- (b)
	    \item Una lista de dos vectores en $V$ es linealmente independiente si y sólo si ninguno de los vectores es múltiplo escalar del otro. \\\\
		Demostración.-\; El enunciado siguiente es equivalente. Dos vectores son linealmente dependientes si y sólo si uno de los vectores es múltiplo escalar de otro. Supongamos que $v_1,v_2$ son dos vectores linealmente dependientes. Por lo que, existe escalares $a_1,a_2$ tal que 
		$$a_1v_1+a_2v_2=0$$
		y no ambos escalares $a_1,a_2$ son cero. Sea $a_1\neq 0$, entonces la ecuación se podría reescribir como
		$$v_1=-\dfrac{a_2}{a_1}v_2$$
		el cual prueba que $v_1$ es un múltiplo escalar de $v_2$. Por otro lado, si $a_2\neq 0$, entonces $v_2=-\frac{a_1}{a_2}v_1$ de aquí podemos afirmar que $v_2$ es un múltiplo escalar de $v_1$.\\
		Ahora supongamos que que uno de los $v_1$ o $v_2$ es un múltiplo escalar del otro. Podemos decir, sin perdida de generalidad, que $v_1$ es un múltiplo escalar de $v_2$. Esto es, $v_1=cv_2$ para algún escalar $c$. Por lo tanto, la ecuación $v_1-cv_2=0$ se cumple, ya que el multiplicador de $v_1$ es distintos de cero. Esto es precisamente lo que requerimos para la definición de dependencia lineal. Así, $v_1$ y $v_2$ son linealmente dependientes.\\\\

	    %---------- (c)
	    \item $(1,0,0,0),(0,1,0,0),(0,0,1,0)$ es linealmente independiente en $\textbf{F}^4$.\\\\
		Demostración.-\; Utilizaremos la definición de independencia lineal. Sean $a,b,c$ escalares en \textbf{F} tal que
		$$a(1,0,0,0)+b(0,1,0,0)+c(0,0,1,0)=\textbf{0}=(0,0,0,0)$$
		Entonces,
		$$(a,b,c,0)=(0,0,0,0)$$
		Lo que implica,
		$$a,b,c=0.$$
		Esto demuestra que los tres vectores son linealmente independientes.\\\\


	    %---------- (d)
	    \item La lista $1,z,\ldots,z^m$ es linealmente independiente en $\mathcal{P}(\textbf{F})$ para cada entero no negativo $m$.\\\\
		Demostración.-\; Demostremos por contradicción. Supongamos que $1,z,\ldots,z^m$ es linealmente dependiente. Por lo que, existe un escalar $a_0,a_1,\ldots,a_m$ tal que
		$$a_0+a_1z+\ldots+a_mz^m=0.$$
		Sea $k$ el indice más grande tal que $a_k\neq 0$. Esto significa que los escalares desde $a_{k+1}$ hasta $a_m$ son cero. Entonces, se deduce que
		$$a_0+a_1z+\ldots+a_kz^k=0.$$
		Reescribiendo se tiene
		$$z_k=-\dfrac{a_0}{a_k}-\dfrac{a_1}{a_k}z-\cdots-\dfrac{a_{k-1}}{a_k}z^{k-1}.$$
		Aquí, expresamos $z^k$ como un polinomio de grado $k-1$ el cual es absurdo. Por lo que $1,z,z^2,\ldots,z^m$ es un conjunto linealmente independiente.\\\\

	\end{enumerate}


    %------------------- 3.
    \item Encuentre un número $t$ tal que
    $$(3,1,4),(2,-3,5),(5,9,t)$$
    no es linealmente independiente en $\textbf{R}^3$.\\\\
	Respuesta.-\; Sea,
	$$a(3,1,4)+b(2,-3,5)+c(5,9,t)=0.$$
	Si $c=0$. Entonces,
	$$a(3,1,4)+b(2,-3,5)=0.$$
	Lo que implica
	$$
	\begin{array}{rcl}
	    3a+2b&=&0\\
	    a-3b&=&0\\
	    4a+5b&=&0
	\end{array}
	$$
	De donde, resolviendo para $a$ y $b$ se tiene
	$$a=0\quad \mbox{y}\quad b=0.$$
	Pero, no queremos que $a,b,c$ sean cero. Así que debemos forzar que $c\neq 0$, como sigue:
	$$a(3,1,4)+b(2,-3,5)+c(5,9,t)=0\quad \Rightarrow \quad (5,9,t)=-\dfrac{a}{c}(3,1,4)-\dfrac{b}{c}(2,-3,5).$$
	Es decir, estamos expresando $(5,9,t)$ como una combinación lineal de los vectores restantes. Así, sea $-\dfrac{a}{c}=x\;$,$\;-\dfrac{b}{c}=y$  por lo que,
	$$(5,9,t)=x(3,1,4)+y(2,-3,5).$$
	Así, tenemos que
	$$
	\begin{array}{rcl}
	    3x+2y&=&5\\
	    x-3y&=&9\\
	    4x+5y&=&t
	\end{array}
	$$
	Resolviendo para $x$ e $y$ se tiene
	$$x=3\quad \mbox{y}\quad y=-2.$$
	Por lo tanto,
	$$t=2.$$\\

    %------------------- 4.
    \item Verifique la afirmación en el segundo punto del Ejemplo 2.20. Es decir, la lista $(2,3,1),(1,-1,2),(7,3,c)$ es linealmente dependientes en $\textbf{F}^3$ si y sólo si $c=8$, como debes verificar.\\\\
	Respuesta.-\; Sea los escalares $a,b,c$  no todos cero tal que
	$$r(2,3,1)+s(1,-1,2)+t(7,3,c)=(0,0,0)$$
	De donde, podemos escribir como ecuaciones lineales
	$$
	\begin{array}{rcl}
	    2r+s+7t&=&0\\
	    3r-s+3t&=&0\\
	    r+2s+ct&=&0
	\end{array}
	$$
	De la ecuación 1 y 2 se tiene
	$$5r+10t=0\quad \Rightarrow \quad r=-2t.$$
	Luego sustrayendo la ecuación 1 y 3, 
	$$2r+(c-4)t=0.$$
	Así, tenemos que
	$$2(-2t)+(c-4)t=0\quad \Rightarrow \quad (c-8)t=0$$
	Por lo que,
	$$r=0\quad \mbox{o}\quad c-8=0.$$
	Si $t=0$. Entonces, $r=-2t = 0$, y $s=0.$ Contradiciendo el hecho de que no todos los escalares son cero. Por lo tanto, los tres vectores son linealmente dependientes si y sólo si $c=8$.\\\\

    %------------------- 5.
    \item 
	\begin{enumerate}[(a)]

	    %---------- (a)
	    \item Demuestre que si pensamos en $\textbf{C}$ como un espacio vectorial sobre $\textbf{R}$, entonces la lista $(1+i,1-i)$ es linealmente independiente.\\\\
		Demostración.-\; Sean los escalares $a,b\in \textbf{R}$, tal que
		$$a(1+i)+b(1-i)=0\quad \Rightarrow \quad (a+b)+(a-b)i=0.$$
		Entonces,
		$$a+b=0\quad \mbox{y}\quad a-b=0.$$
		Igualando estas dos ecuaciones se tiene
		$$
		\begin{array}{rcl}
		    a+b=a-b&\Rightarrow& 2b=0\\
			   &\Rightarrow& b=0.
	       \end{array}
		$$
		Reemplazando en $a+b=0$,
		$$a=0.$$
		Por lo tanto, $(1+i,1-i)$ es linealmente independiente sobre $\textbf{R}$.\\\\


	    %---------- (b)
	    \item Demuestre que si pensamos en $\textbf{C}$ como un espacio vectorial sobre $\textbf{C}$, entonces la lista $(1+i,1-i)$ es linealmente dependiente.\\\\
		Demostración.-\; Sean los escalares $i,1\in \textbf{C}$, tal que
		$$i(1+i)+1(1-i)=i+i^2+1-i=0\quad \Rightarrow \quad (i-1)+(1-i)=(i-1)-(i-1)=0.$$
		Donde concluimos que $(1+i,1-i)$ es linealmente dependiente sobre $\textbf{C}$.\\\\

	\end{enumerate}

    %------------------- 6.
    \item Supongamos que $v_1,v_2,v_3,v_4$ es linealmente independiente en $V$. Demostrar que la lista
    $$v_1-v_2,v_2-v_3,v_3-v_4,v_4$$
    es también linealmente independiente.\\\\
	Demostración.-\; Sean los escalares $a,b,c,d\in F$ tal que
	$$a(v_1-v_2)+b(v_2-v_3)+c(v_3-v_4)+d(v_4)=0.$$
	De donde,
	$$av_1-av_2+bv_2-bv_3+cv_3-cv_4+dv_4=0.$$
	Por lo que,
	$$av_1+(b-a)v_2+(c-b)v_3+(d-c)v_4=0$$
	Ya que $v_1,v_2,v_3,v_4$ es linealmente independiente, entonces
	$$
	\begin{array}{rcl}
	    a&=& 0\\
	    b-a&=& 0\\
	    c-b&=& 0\\
	    d-c&=& 0
	\end{array}
	$$
	Resolviendo para $a,b,c,d$ se tiene
	$$a=0,\quad b=0,\quad c=0,\quad d=0.$$
	Esto implica que 
	$$0(v_1-v_2)+0(v_2-v_3)+0(v_3-v_4)+0(v_4)=0.$$
	Por lo tanto, la lista
	$$v_1-v_2,v_2-v_3,v_3-v_4,v_4$$
	es linealmente independiente.\\\\

    %------------------- 7.
    \item Demostrar o dar un contraejemplo: Si $v_1,v_2,\ldots,v_m$ es una lista linealmente independiente  de vectores en $V$, Entonces
    $$5v_1-4v_2,v_2,v_3,\ldots,v_m$$
    es linealmente independiente.\\\\
	Demostración.-\; Por definición de independencia lineal. Sean los escalares $a_i\textbf{F}$ tal que 
	$$a_1(5v_1-4v_2)+a_2v_2+a_3v_3+\ldots+a_mv_m=0$$
	De donde,
	$$5a_1v_1+(a_2-4a_1)v_2+a_3v_3+\ldots+a_m v_m=0$$
	Sabemos que la independencia lineal obliga a todos los escalares de $v_i$ a ser cero. En particular , $5a_1=0$ entonces $a_1=0$ y $a_2-4a_1=0$, implica $a_2=0.$ Por lo tanto, 
	$$0\cdot v_1+(0-0)v_2+a_3v_3+\ldots+a_m v_m=0$$
	Dado que todos $a_i$ son cero, entonces $5v_1-4v_2,v_2,v_3,\ldots,v_m$ es linealmente independiente.\\\\

    %------------------- 8.
    \item Demostrar o dar un contraejemplo: Si $v_1,v_2,\ldots,v_m$ es una lista linealmente independiente  de vectores en $V$ y $\gamma\in \textbf{F}$ con $\gamma\neq 0$, Entonces $\gamma v_1,\gamma v_2,\ldots,\gamma v_m$ es linealmente independiente.\\\\
	Demostración.-\; Por definición de independencia lineal. Sean los escalares $a_i\in \textbf{F}$ tal que
	$$a_1\gamma v_1+a_2\gamma v_2+\ldots+a_m\gamma v_m=0.$$
	De donde,
	$$\gamma\left(a_1 v_1+a_2 v_2+\ldots+a_m v_m\right)=0.$$
	Lo que,
	$$a_1 v_1+a_2 v_2+\ldots+a_m v_m=0.$$
	Ya que, $v_1,v_2,\ldots,v_m$ es linealmente independiente. Entonces, todos los $a_i's$ deben ser cero. Por lo tanto, $a_1\gamma v_1+a_2\gamma v_2+\ldots+a_m\gamma v_m=0.$ es linealmente independiente.\\\\

    %------------------- 9.
    \item Demostrar o dar un contraejemplo: Si $v_1,v_2,\ldots,v_m$ y $w_1,w_2,\ldots,w_m$ son listas linealmente independientes de vectores en $V$, entonces $v_1+w_1,\ldots,v_m+w_m$ es linealmente independiente.\\\\
	Demostración.-\; Sean los escalares $a_i,b_i\in \textbf{F}$ tal que
	$$a_1v_1+a_1v_2+\ldots+a_mv_m=0\quad \mbox{y}\quad b_1w_1+b_2w_2+\ldots+b_mv_m=0.$$
	Entonces,
	$$a_1v_1+a_1v_2+\ldots+a_mv_m+b_1w_1+b_2w_2+\ldots+b_mv_m=0.$$
	De donde,
	$$(a_1-b_1)(v_1+w_1)+(a_2-b_2)(v_2+w_2)+\ldots+(a_m-b_m)(v_m+w_m)=0.$$
	Supongamos $c_i=a_i+b_i\in F$. Luego,
	$$c_1(v_1+w_1)+c_2(v_2+w_2)+\ldots+c_m(v_m+w_m)=0.$$
	Dado que $a_i=b_i=0$, ya que $v_1,v_2,\ldots,v_m$ y $w_1,w_2,\ldots,w_m$ son linealmente independientes. Concluimos que, $v_1+w_1,\ldots,v_m+w_m$ es linealmente independiente.\\\\

    %------------------- 10.
    \item Suponga $v_1,\ldots,v_m$ es linealmente independiente en $V$ y $W\in V$. Demostrar que si $v_1+w,\ldots,v_m+w$ es linealmente dependiente, entonces $w\in \span(v_1,\ldots,v_m)$.\\\\
	Demostración.-\; Por definición de dependencia lineal. Existen $a_1,\ldots a_m \in \textbf{F}$, no todos $0$, tal que 
	$$a_1(v_1+w)+a_2(v_2+w)+\ldots+a_m(v_m+w)=0.$$
	De donde,
	$$a_1v_1+a_2v_2+\ldots+a_mv_m=-(a_1+a_2+\ldots+a_m)w. \qquad \qquad (1)$$
	Dado que $v_1,\ldots,v_m$ es linealmente independiente, entonces existen escalares $t_i,\ldots,t_m\in \textbf{F}$, $\forall t_i=0$, de modo que
	$$t_1v_1+t_2v_2+\ldots+t_mv_m=0$$
	Es único. Así pues notemos, para $a_i\neq 0$ que
	$$a_1v_1+a_2v_2+\ldots+a_mv_m\neq 0$$
	En consecuencia por (1)
	$$-(a_1+a_2+\ldots+a_m)w\neq 0.$$
	Por lo tanto,
	$$w=-\frac{1}{a_1+a_2+\ldots+a_m}(a_1v_1+a_2v_2+\ldots+a_mv_m)\in \span(v_1,\ldots,v_m).$$\\

    %------------------- 11.
    \item Suponga $v_1,\ldots,v_m$ es linealmente independiente en $V$ y $w\in V$. Demostrar que $v_1,\ldots,v_m$, $w$ es linealmente independiente si y sólo si 
    $$w\neq \span(v_1,\ldots,v_m).$$\\
	Demostración.-\; Supongamos que $w\in \span(v_1,\ldots,v_m)$. Entonces,
	$$w=a_1v_1+\ldots+a_mv_m.$$
	De donde,
	$$a_1v_1+\ldots+a_mv_m-w=0\quad \Rightarrow \quad a_1v_1+\ldots+a_mv_m+(-1)w=0.$$
	Por lo tanto, $v_1,\ldots,v_m,w$ es linealmente dependiente.\\

	Por otro lado: $v_1,\ldots,v_m,w$ es linealmente independiente, entonces existe $a_1,\ldots,a_m,b\in \textbf{F}$, $\forall a_i=0$, tal que
	$$a_1v_1+\ldots+a_mv_m+bw=0.$$
	Dado que $b=0$, no se puede escribir $w$ como combinación lineal de $v_1,\ldots,v_m$. Es decir,
	$$w=\dfrac{1}{0}(a_1v_1+\ldots+a_mv_m),$$
	lo que es imposible. De esta manera
	$$w\neq \span(v_1,\ldots,v_m).$$\\


    %------------------- 12.
    \item Explique por qué no existe una lista de seis polinomios que sea linealmente independiente en $\mathcal{P}_4(\textbf{F})$.\\\\
	Respuesta.-\; Notemos que $1,z,z^2,z^3,z^4$ genera $\mathcal{P}_4(\textbf{F})$. Pero por el teorema 1.23 [Axler, Linear Algebra], la longitud de la lista linealmente independiente es menor o igual que la longitud de la lista que genera. Es decir, cualquier lista linealmente independiente no tiene más de $5$ polinomios.\\\\

    %------------------- 13.
    \item Explique por qué ninguna lista de cuatro polinomios genera $\mathcal{P}_4 (\textbf{F})$.\\\\
	Respuesta.-\; Sea $V$ un espacio vectorial de dimensión finita. Si $m$ vectores genera $V$ y si tenemos un conjunto de $n$ vectores linealmente independientes, entonces $n\leq m$. Es decir, el número de vectores en un conjunto linealmente independiente de $V$, no puede ser mayor que el número de vectores en un conjunto generador de $V$.\\
	Por ejemplo, si cuatro polinomios podrían generar $P_4(\textbf{F})$. Entonces, por la definición de arriba, cualquier conjunto de polinomios linealmente independientes en $P_4(\textbf{F})$ podría tener cómo máximo cuatro vectores. Sin embargo, el conjunto $1,z,z^2,z^3,z^4$ tiene cinco polinomio linealmente independientes en $P_4(\textbf{F})$. Por lo tanto, es imposible que cualquier conjunto de cuatro polinomios genere $P_4(\textbf{F})$.\\\\

    %------------------- 14.
    \item Demuestre que $V$ es de dimensión infinita, si y sólo si existe una secuencia $v_1,v_2,\ldots,$ de vectores en $V$ tal que $v_1,\ldots,v_m$ es linealmente independiente para cada entero positivo $m$.\\\\
	Demostración.-\; Supongamos que $V$ es de dimensión infinita. Queremos producir una secuencia de vectores $v_1,v_2,\ldots,$ tal que $v_1,v_2,\ldots,v_m$ es linealmente independiente para cada $m$. Necesitamos mostrar que para cualquier $k\in \textbf{N}$ y un conjunto de vectores linealmente independientes $v_1,v_2,\ldots$ podemos definir un vector $v_{k+1}$ tal que $v_1,v_2,\ldots,v_{k+1}$ es linealmente independiente. Si podemos probar esto, entonces significará que podemos continuar sumando vectores indefinidamente a conjuntos linealmente independientes de modo que los conjuntos resultantes también sean linealmente independientes. Esto nos dará una secuencia de vectores $v_1,v_2,\ldots$, cuyo subconjunto finito es linealmente independiente.\\

	Sea $v_1,v_2,\ldots,v_k$ un conjunto linealmente independiente en $V$. Ya que $V$ es de dimensión finita, no puede generado por un conjunto finito de vectores. Por lo tanto, $V\neq \span(v_1,v_2,\ldots,v_k)$. Sea $v_{k+1}$ tal que $v_{k+1}\notin \span(v_1,v_2,\ldots,v_k)$. Entonces, por el ejercicio 11 $\left[\right.$ Axler, Linear Algebra, que nos dice: Si $v:1, \ldots, v_m$ es linealmente independiente en $V$ y $w \in V$, el conjunto  $v_1,\ldots,v_m, w$ es linealmente independiente si y sólo si $w \neq \span(v_1,\ldots, v_m) \left.\right]$. El conjunto $v_1,v_2,\ldots,v_{k+1}$ es linealmente independiente.\\\\
	Por otro lado, sea $v_1,v_2,\ldots,v_n$ un conjunto generador de $V$. Entonces, por el teorema 1.23 [Axler, Linear Algebra], cualquier conjunto de vectores linealmente independiente en $V$ pueden tener por lo más $n$ vectores. De esto modo, cualquier conjunto que tenga $n+1$ o más vectores es linealmente dependiente. Así, si $V$ es de dimensión finita, entonces no podemos tener una secuencia de vectores $v_1,v_2,\ldots$ tal que, para cada $m$, el subconjunto $v_1,v_2,\ldots,v_m$ es linealmente independiente. Tomando su recíproca, podemos decir que si existe una secuencia de vectores $v_1,v_2,\ldots$ tal que el conjunto $v_1,v_2,\ldots,v_m$ es linealmente independiente para cada $m$. Entonces, $V$ es de dimensión infinita. Lo que completa de demostración.\\\\

    %------------------- 15.
    \item Demostrar que $\textbf{F}^\infty$ es de dimensión infinita.\\\\
	Demostración.-\; Sea un elemento $e_m\in \textbf{F}^\infty$ como el elemento que tiene la coordenada m-ésima igual a $1$ los demás elementos iguala a $0$. Es decir,
	$$(0,1,0,\ldots,0)$$
	Ahora, si varía $m$ sobre el conjunto de los números naturales, entonces tenemos una secuencia $e_1,e_2,\ldots$ en $\textbf{F}^\infty$, si y sólo si podemos probar que $e_1,e_2,\ldots,e_m$ es linealmente independiente para cada $m$. Con este fin, sea
	$$a_1e_1+a_2e_2+\ldots+a_me_m=0$$
	De donde,
	$$(a-1,a_2,\ldots,a_m,0,0,\ldots,0)=(0,0,\ldots,0)$$
	Inmediatamente implica que $a_i's=0$ y por lo tanto, $e_i's$ son linealmente independiente.\\\\

    %------------------- 16.
    \item Demostrar que el espacio vectorial real para todos las funciones de valor real continuas en el intervalo $[0,1]$ es de dimensión infinita.\\\\
	Demostración.-\; Por el ejercicio 14 (Axler, Linear Algebra, 2A), tenemos que encontrar una secuencia linealmente independiente de funciones continuas en $[0,1]$. Observe que los monomiales $1,x,x^2,\ldots,x^n,\ldots$ son funciones continuas en $[0,1]$. Ahora, debemos demostrar que $1,x,x^2,\ldots,x^m$ es linealmente independiente en cada $m$. Para ello, sea $a_0+a_1x+a_2x^2+\ldots+a_mx^m=0$, donde $0$ es el cero polinomial. Lo que significa que $a_0+a_1x+a_2x^2+\ldots+a_mx^m=0$ toma el valor cero en todo el intervalo $[0,1]$. Esto implica que cada punto en $[0,1]$ es una raíz del polinomio. Pero, ya que cada polinomio no trivial tiene como máximo un número finito de raíces, esto es imposible a menos que todos los $a_i'$s sean cero. Lo que muestra que $1,x,x^2,\ldots,x^m$ es linealmente independiente para cada $m\in \textbf{N}$. Por lo tanto, el conjunto de funciones continuas en $[0,1]$ es de dimensión infinita.\\\\

    %------------------- 17.
    \item Suponga $p_0,p_1,\ldots,p_m$ son polinomios en $\mathcal{P}_m(\textbf{F})$ tal que $p_j(2)=0$ para cada $j$. Demostrar que $p_0,p_1,\ldots,p_m$ no es linealmente independiente en $\mathcal{P}_m(\textbf{F})$.\\\\
	Demostración.-\; Supondremos que $p_0,p_1,\ldots.p_m$ es linealmente independiente. Demostraremos que esto  implica que $p_0,p_1,\ldots,p_m$ genera $\mathcal{P}_m(\textbf{F})$. Y que esto a su vez conducirá a una contradicción al construir explícitamente un polinomio que no está en este generador.
	Notemos que la lista $1,z,\ldots,z^{m+1}$ genera $\mathcal{P}_m(\textbf{F})$ y tiene longitud $m+1$. Por lo tanto, cada lista linealmente independiente debe tener una longitud $m+1$ o menos (2.23). Si $\span(p_0,p_1,\ldots,p_m)\neq \mathcal{P}_m(\textbf{F})$, existe algún $p\notin \span(p_0,p_1,\ldots,p_m)$, de donde la lista $p_0,p_1,\ldots,p_m,p$ es linealmente independiente de longitud $m+2$, lo que es una contradicción. Por lo que $\span(p_0,p_1,\ldots,p_m)=\mathcal{P}_m(\textbf{F})$.\\
	Ahora definamos el polinomio $q=1$. Entonces $q\in \span(p_0,p_1,\ldots,p_m)$, de donde existe $a_0,\ldots,a_m\in \textbf{F}$ tal que
	$$q=a_0p_0+a_1p_1+\cdots+a_mP_m,$$
	lo que implica
	$$q(2)=a_0p_0(2)+a_1p_1(2)+\cdots+a_mP_m(2).$$
	Pero esto es absurdo, ya que $1=0$. Por lo tanto, $p_0,p_1,\ldots,p_m$ no puede ser linealmente independiente.\\\\
\end{enumerate}

\vspace{1cm}



\mysection{Bases}

\setcounter{mydef}{26}
%------------------- Definición 2.27
\begin{mydef}[Base]\;\\\\
    Una base de $V$ es una lista de vectores en $V$ que es linealmente independiente y genera $V$.
\end{mydef}

%------------------- teorema 2.29
\begin{myteo}[Criterio de base]\;\\\\
    Una lista $v_1,\ldots,v_n$ de vectores en $V$ es una base de $V$ si y sólo si cada $v\in V$ puede escribirse unívocamente de la forma 
    $$v=a_1v_1+\cdots+a_nv_n.$$
    donde $a_1,\ldots,a_n\in \textbf{F}$.\\\\
	Demostración.-\; Primero suponga que $v_1,\ldots,v_n$ es una base de $V$. Ya que, $v_1,\ldots,v_n$ genera $V$, existe $a_1,\ldots,a_n\in \textbf{F}$ tal que 
	$$v=a_1v_1+\cdots+a_nv_n$$ 
	se cumple. Para mostrar que esta representación es única, sean $c_1,\ldots,c_n$ escalares tales que, también tenemos
	$$v=c_1v_1+\cdots+c_nv_n.$$
	Sustrayendo la primera ecuación de la segunda, tenemos
	$$0=(a_1-c_1)v_1+\cdots+(a_n-c_n)v_n.$$
	Esto implica que cada $a_j-c_j$ es igual a cero. (Ya que, $v_1,\ldots,v_n$ es linealmente independiente). Por lo tanto $a_1=c_1,\ldots,a_n=c_n$. Así, tenemos la unicidad deseada.\\

	Por otro lado, suponga que cada $v\in V$ puede ser escrita de manera única como la forma $v=a_1v_1+\cdots+a_nv_n$. Claramente esto implica que $v_1,\ldots,v_n$ genera $V$. Para demostrar que $v_1,\ldots,v_n$ es linealmente independiente, suponga que $a_1,\ldots,a_n\in \textbf{F}$ son tales que
	$$0=a_1v_1+\cdots+a_nv_n.$$
	La unicidad de la representación de $v=a_1v_1+\cdots+a_nv_n$ (tomando $v=0$) implica que $a_1=\cdots=a_n=0$. Así, $v_1,\dots,v_n$ es linealmente independiente y por tanto es una base de $V$.
\end{myteo}

Una lista generadora en un espacio vectorial puede no ser una base, ya que no es linealmente independiente. Nuestro próximo resultado dice que dada cualquier lista generadora, algunos (posiblemente ninguno) de los vectores en ella pueden descartarse para que la lista restante sea linealmente independiente y aún genere el espacio vectorial.

\setcounter{myteo}{30}
%------------------- teorema 2.31
\begin{myteo}[La lista generadora contiene un base]\,\\\\
    Cada lista generadora en un espacio vectorial se puede reducir a una base del espacio vectorial.\\\\
	Demostración.-\; Suponga que $v_1,\ldots,v_n$ genera $V$. Queremos eliminar algunos de los vectores de  $v_1,\ldots,v_n$ para que los vectores restantes formen una base de $V$. Sea $B=v_1,\ldots,v_n$, de donde realizamos un bucle con las siguientes condiciones:
	\begin{enumerate}[\textbf{Paso} 1.]
	    \item Si $v_1=0$, eliminamos $v_1$ de $B$. Si $v_1\neq 0$ entonces no cambiamos $B$.
	    \item[\textbf{Paso} J.] Si $v_j$ esta en $\span(v_1,\ldots,v_{j-1})$, eliminamos $v_j$ de $B$. Si $v_j$ no está en $\span(v_1,\ldots,v_j)$, entonces no cambiamos $B$ (lema 2.21). 
	\end{enumerate}
	Paramos el proceso después del paso $n$, obteniendo una lista $B$. Esta lista $B$ genera $V$, ya que nuestra lista original generó $V$ y hemos descartado solo los vectores que ya estaban en el generador de los vectores anteriores. El proceso garantiza que ningún vector de $B$ está en el generador de los anteriores. Así pues, $B$ es linealmente independiente, por el lema de dependencia lineal (2.21). Por tanto, $B$ es una base de $V$.
\end{myteo}

Nuestro siguiente resultado, un corolario fácil del resultado anterior, nos dice que todo espacio vectorial de dimensión finita tiene una base.

\setcounter{mycor}{31}
%------------------- corolario 2.32
\begin{mycor}[Base del espacio vectorial de dimensión finita]\,\\\\
    Cada espacio vectorial de dimensión finita tiene una base.\\\\
	Demostración.-\; Por definición, un espacio vectorial de dimensión finita tiene una lista generadora. El resultado anterior nos dice que cada lista generadora puede ser reducida a una base.
\end{mycor}

Nuestro siguiente resultado es en cierto sentido un derivado de 2.31, que decía que toda puede reducirse a una base. Ahora mostramos que dada cualquier lista linealmente independiente, podemos unir algunos vectores adicionales (esto incluye la posibilidad de no unir ningún vector adicional) de modo que la lista ampliada siga siendo linealmente independiente pero que también genere el espacio.

\setcounter{myteo}{32}
%------------------- teorema 2.33
\begin{myteo}[Una lista linealmente independiente se extiende a una base]\,\\\\
    Cada lista linealmente independiente de vectores en un espacio vectorial se puede extender a una base del espacio vectorial.\\\\
	Demostración.-\; Suponga $u_1,\ldots,u_m$ es linealmente independiente en un espacio vectorial $V$ de dimensión finita. Sea $w_1,\ldots,w_n$ una base de $V$. Por lo que la lista
	$$u_1,\ldots,u_m,\; w_1,\ldots.w_n$$
	genera $V$. Aplicando el procedimiento de la prueba de 2.31 para reducir esta lista a una base de $V$ se obtiene una base formada por los vectores $u_1, \ldots , u_m$ (ninguna de las $u's$ se elimina en este procedimiento porque $u_1,\ldots, u_m$ es linealmente independiente) y algunos de los $w'$.
\end{myteo}

Como aplicación del resultado anterior, mostramos ahora que cada subespacio de un espacio vectorial de dimensión finita puede emparejarse con otro subespacio para formar una suma directa de todo el espacio.

%------------------- teorema 2.34
\begin{myteo}[Cada subespacio de \boldmath$V$ forma parte de una suma directa igual a \boldmath$V$]\;\\\\
    Suponga $V$ es de dimensión finita y $U$ es un subespacio de $V$. Entonces, existe un subespacio $W$ de $V$ tal que $U\oplus W=V$.\\\\
	Demostración.-\; Ya que $V$ es de dimensión finita, también lo es $U$ por 2.26. Por lo que, existe una base $u_1,\ldots,u_m$ de $U$ esto por 2.32. Por su puesto $u_1,\ldots,u_m$ es una lista linealmente independiente de vectores en $V$. Por lo tanto, esta lista puede extenderse a una base $u_1,\ldots,u_n$ de $V$ esto por 2.33. Sea $W=\span(w_1,\ldots,w_n)$. Para probar que $V=U\oplus W,$ por 1.45, solo necesitamos demostrar que
	$$V=U+W \quad \mbox{y}\quad U\cap W=\left\{0\right\}.$$
	Para probar la primera ecuación, suponga $v\in V$. Entonces, ya que la lista $u_1,\ldots,u_m$, $w_1,\ldots,w_n$ genera $V$, existe $a_1,\ldots,a_m$, $b_1,\ldots,b_n\in \textbf{F}$ tal que
	$$v=a_1u_1+\cdots+a_mu_m+b_1w_1+\cdots+b_nw_n.$$
	En otras palabras, tenemos $v=u+w$, donde $u\in U$ y $w\in W$ fueron definidas anteriormente. Así, $v\in U+W$, completando la prueba de $V=W+W$.\\
	Para demostrar que $U\cap W=\left\{0\right\}$, suponga $v\in U\cap W$. Entonces, existe escalares $a_1,\ldots,a_m$, $b_1,\ldots,b_n\in \textbf{F}$ tal que
	$$v=a_1u_1+\cdots+a_mu_m=b_1w_1+\cdots+b_nw_n.$$
	Por lo tanto,
	$$a_1u_1+\cdots+a_mu_m-b_1w_1-\cdots-b_nw_n=0.$$
	Esto, ya que $u_1,\ldots,u_m$, $w_1,\ldots,w_n$ son linealmente independientes, esto implica que $a_1=\cdots=a_m=b_1=\cdots=b_n=0$. Así, $v=0$, de donde $U\cap W= \left\{0\right\}$.
\end{myteo}

\vspace{.5cm}


\setcounter{mysection}{1}
\mysection{Ejercicios}

\begin{enumerate}[\bfseries 1.]

    %------------------- 1.
    \item Halle todos los espacios vectoriales que tienen exactamente una base.\\\\ 
	Respuesta.-\; Afirmamos que solo el espacio vectorial trivial tiene exactamente una base. Para ello demostraremos que para espacios vectorial de dimensión finita e infinita se tiene más de una base.\\
	Consideremos un espacio vectorial de dimensión finita. Sea $V$ un espacio vectorial no trivial con base $v_1,\ldots,v_n$. Decimos que para cualquier $c\in \textbf{F}$, la lista $cv_1,\ldots,cv_n$ es también una base. Es decir, la lista es aún linealmente independiente, y es aún generador de $V$. Luego, sea $u\in V$ ya que $v_1,\ldots,v_n$ genera $V$, existe $a_1,\ldots,a_n\in \textbf{F}$ tal que
	$$u=a_1v_1+\cdots+a_nv_n.$$
	De donde, podemos escribir
	$$u=\dfrac{a_1}{c}(cv_1)+\cdots+\dfrac{a_n}{c}(cv_n)$$
	y así $cv_1,\ldots,cv_n$ genera también $V$. Por lo tanto, tendremos más de una base para todo espacio vectorial de dimensión finita.\\
	Por otro lado. Sea $W$ un espacio vectorial de dimensión infinita con base $w_1,w_2,\ldots$. Para cualquier $c\in \textbf{F}$, la lista $cw_1,cw_2,\ldots$ es también una base. Claramente la lista es linealmente independiente, y también genera $V$. Luego, sea $u\in V$, ya que $w_1,w_2,\ldots$ genera $W$, existe $a_1,a_2,\ldots\in \textbf{F}$ tal que
	$$u=a_1w_1+a_2w_2+\cdots$$
	De donde, podemos escribir
	$$u=\dfrac{a_1}{c}cw_1+\dfrac{a_2}{c}cw_2+\cdots$$
	y así $cw_1,cw_2,\ldots$ genera también $W$. Por lo tanto, tendremos más de una base para todo espacio vectorial de dimensión infinita.\\\\


    %------------------- 2.
    \item Verifique todas las afirmaciones del ejemplo 2.28.\\
	\begin{enumerate}[(a)]

	    %---------- (a)
	    \item La lista $(1,0,\ldots,0)$, $(0,1,0,\ldots,0)$, $\ldots$, $(0,\ldots,0,1)$ es una base de $\textbf{F}^n$, llamado la base estándar de $\textbf{F}^n$.\\\\
		Respuesta.-\; Primero demostraremos que la lista genera $\textbf{F}^n$.  Sea, los escalares $x_1,x_2,\ldots,x_n$ en $\textbf{F}$. Podemos escribir
		$$x_1(1,0,\ldots,0)+x_2(0,1,0,\ldots,0)+\cdots+x_n(0,\ldots,0,1)=\left(x_1,x_2,\ldots,x_n\right).$$
		Donde, $(x_1,x_2,\ldots,x_n)$ es un vector cualquier en $\textbf{F}^n$. Esta expresión es una combinación lineal de los $n$ vectores. Por definición, esta lista genera $\textbf{F}^n$.\\
		Ahora, demostraremos que la lista es linealmente independiente. Para ello, aplicaremos la definición. Sea $a_1,a_2,\ldots,a_n\in \textbf{F}$, entonces
		$$a_1(1,0,\ldots,0)+a_2(0,1,0,\ldots,0)+\cdots+a_n(0,\ldots,0,1)=0.$$
		Esto implica que
		$$(a_1,a_2,\ldots,a_n)=(0,0,\ldots,0).$$
		Por lo que $a_1=a_2=\cdots=a_n=0$. Así, la lista es linealmente independiente.\\\\

	    %---------- (b)
	    \item  La lista $(1,2),(2,5)$ es una base de $\textbf{F}^2$.\\\\
		Respuesta.-\; Sea $(x_1,x_2)\in \textbf{F}^2$. Buscaremos escalares $c_1,c_2$ tal que
		$$c_1v_1+c_2v_2=(x_1,x_2).$$
		que implica,
		$$c_1(1,2)+c_2(3,5)=(x_1,x_2)\quad \Rightarrow \quad (c_1+3c_2,2c_1+5c_2)=(x_1,x_2)$$
		De donde, podemos construir un sistema de ecuaciones
		$$
		\begin{array}{rcl}
		    c_1+3c_2&=&x_1\\
		    2c_1+5c_2&=&x_2
		\end{array}
		$$
		Multiplicando por dos la primera ecuación y luego restando la segunda, tenemos
		$$c_2=2x_1-x_2$$
		Luego, reemplazándola a la primera ecuación, se tiene
		$$c_1=-5x_1+3x_2.$$
		Por lo tanto, para cada vector $(x_1,x_2)\in \textbf{F}^2$ podemos encontrar $c_1,c_2$ en función de $x_1$ y $x_2$ tal que $c_1v_1+c_2v_2$ es una combinación lineal el cual genera $\textbf{F}^2$.\\
		Después, sólo nos haría falta reemplazar en
		$$c_2=2x_1-x_2\quad \mbox{y}\quad c_1=-5x_1+3x_2$$
		$(x_1,x_1)=(0,0)$. De donde,
		$$c_2=0\quad \mbox{y}\quad c_1=0.$$
		Esto implica que $(1,2)$ y $(2,5)$ es linealmente independiente Por lo que concluimos que la lista dada es una base de $\textbf{F}^2$.\\\\

	    %---------- (c)
	    \item La lista $(1,2,-4)$, $(7,-5,6)$ es linealmente independiente en $\textbf{F}^3$ pero no es una base en $\textbf{F}^3$, ya que no genera $\textbf{F}^3$.\\\\
		Respuesta.-\; Sean los escalares  $c_1,c_2\in \textbf{F}$ tal que
		$$c_1(1,2,-4)+c_2(7,-5,6)=0\quad \Rightarrow\quad (c_1+7c_2,2c_1-5c_2,-4c_1+6c_2)=(0,0,0)$$
		Por lo que, podemos construir un sistema de ecuaciones
		$$
		\begin{array}{rcl}
		    c_1+7c_2&=&0\\
		    2c_1-5c_2&=&0\\
		    -4c_1+6c_2&=&0
		\end{array}
		$$
		Multiplicando la segunda ecuación y sumando la tercera tenemos
		$$c_2=0$$
		Luego, sustituyendo en la primera ecuación, 
		$$c_1=0$$
		Esto implica que los vectores dados son linealmente independientes.\\
		Ahora, demostraremos que la lista no genera $\textbf{F}^3$, con un contraejemplo. Supongamos que $(1,2,-4),(7,-5,6)$ puede generar $(1,0,0)$ el cual está en $\textbf{F}^3$. Sea los escalares $c_1,c_2\in \textbf{F}$, entonces 
		$$c_1(1,2,-4)+c_2(7,-5,6)=(1,0,0)\quad \Rightarrow\quad (c_1+7c_2,2c_1-5c_2,-4c_1+6c_2)=(1,0,0).$$
		Construimos un sistema de ecuaciones, como sigue
		$$
		\begin{array}{rcl}
		    c_1+7c_2&=&1\\
		    2c_1-5c_2&=&0\\
		    -4c_1+6c_2&=&0
		\end{array}
		$$
		De las ecuaciones 2 y 3 se tiene que
		$$c_1=0,\quad c_2=0$$
		Reemplazando en la primera ecuación, 
		$$0+0=1\quad \Rightarrow\quad 0=1.$$
		Lo que es un absurdo, por lo tanto $(1,2,-4)$, $(7,-5,6)$ no genera $\textbf{F}^3$.\\\\

	    %---------- (d)
	    \item La lista $(1,2),(3,5),(4,13)$ genera $\textbf{F}^2$ pero no es una base de $\textbf{F}^2$, ya que no es linealmente independiente.\\\\
		Respuesta.-\; Demostremos que la lista no es linealmente independiente. Sea $a_1,a_2,a_3\in \textbf{F}$, entonces
		$$a_1(1,2)+a_2(3,5)+a_3(4,13)=0 \quad \Rightarrow \quad (a_1+3a_2+4a_3,2a_1+5a_2+13a_3)=(0,0).$$
		Construimos un sistema de ecuaciones, como sigue
		$$
		\begin{array}{*{7}{r}}
		    a_1&+&3a_2&+&4a_3&=&0\\
		    2a_1&+&5a_2&+&13a_3&=&0
		\end{array}
		$$
		Multiplicando por dos la primera ecuación y luego restando la segunda, tenemos
		$$a_2=5a_3.$$
		Remplazando en la primera ecuación,
		$$a_1=-19a_3.$$
		Sea, $a_3=1$, entonces
		$$a_1=-19\quad \mbox{y}\quad a_2=5.$$
		Por lo tanto, $(1,2),(3,5),(4,13)$ no es linealmente independiente.\\

		Ahora, demostraremos que la lista $(1,2),(3,5),(4,13)$ genera $\textbf{F}^2$.  Sean $a_1,a_2,a_3\in \textbf{F}$ tal que
		$$a_1(1,2)+a_2(3,5)+a_3(4,13)=0$$
		Sabiendo que esta lista es linealmente dependiente, podemos reescribimos la ecuación de modo que $(1,2),(3,5)$ genera $(4,13)$:
		$$(4,13) = \dfrac{a_1}{a_3}(1,2)-\dfrac{a_2}{a_3}(3,5)$$
		Por el lema 2.21 (Axler, Linear Algebra), vemos que el generador de $(1,2),(3,5)$ es igual al generador de $(1,2),(3,5),(4,13)$. Sólo nos faltaría demostrar que $(1,2),(3,5)$ genera $\textbf{F}^2$. Para ello, sea $(x_1,x_2)\in\textbf{F}^2$, de modo que
		$$a_1(1,2)+a_2(3,5)=(x_1,x_2)$$
		Entonces,
		$$
		\begin{array}{rcl}
		    a_1+3a_2&=&x_1\\
		    2a_1+5a_2&=&x_2
		\end{array}
		$$
		Multiplicando la segunda ecuación por dos y restando la primera, tenemos
		$$a_1=2x_1-x_2.$$
		Remplazando en la primera ecuación,
		$$a_2=-(5x_1+3x_2).$$
		Por lo tanto, podemos hallar $a_1$ y $a_2$ en términos de $x_1$ y $x_2$ tal que $a_1(1,2)+a_2(3,5)=(x_1,x_2)$ es una combinación lineal que genera $\textbf{F}^2$.\\

		Siendo más prácticos podemos usar el teorema 2.23.  Para saber que $(1,2),(3,5),(4,13)$ no es linealmente independiente pero genera $\textbf{F}^2$. \\\\

	    %---------- (e)
	    \item La lista $(1,1,0),(0,0,1)$ es una base de $\left\{(x,x,y)\in \textbf{F}^3\; :\; x,y\in \textbf{F}\right\}$.\\\\
		Respuesta.-\; Está claro que la lista es linealmente independiente. Ya que, la única forma de que se cumpla  
		$$c_1(1,1,0)+c_2(0,0,1)=0$$
		es que $c_1,c_2$ sean igual a cero.\\
		Ahora demostraremos que la lista  dada genera $\left\{(x,x,y)\in \textbf{F}^3\; :\; x,y\in \textbf{F}\right\}$. Sea $c_1,c_2\in \textbf{F}$ tal que
		$$c_1(1,1,0)+c_2(0,0,1)=(x,x,y).$$
		De donde, podemos construir un sistema de ecuaciones como sigue:
		$$
		\begin{array}{rcl}
		    c_1+0&=&x\\
		    c_1+0&=&x\\
		    0+c_2&=&y
		\end{array}
		$$
		Por lo que, cualquier  $\textbf{F}^3$ puede ser expresado como una combinación lineal de los vectores $(1,1,0),(0,0,1)$ y por lo tanto generan $\textbf{F}^3$.\\\\ 

	    %---------- (f)
	    \item La lista $(1,-1,0),(1,0,-1)$ es una base de
	    $$\left\{(x,y,z)\in\textbf{F}^3\; : \; x+y+z=0\right\}.$$\\
		Respuesta.-\; Si $x+y+z=0$ para $x,y,z\in \textbf{F}$, entonces podemos escribir 
		$$x=-y-z.$$ 
		Por lo que,
		$$
		\begin{array}{rcl}
		    (x,y,z)&=&(-y-z,y-z)\\
			   &=& (-y,y-0)+(-z,0z)\\
			   &=& -y(1,-1,0)-z(1,0,-1).
		\end{array}
		$$
		Debido a que $y,z$ son escalares, implica que podemos expresar cualquier $(x,y,z)\in\textbf{F}^3$ como una combinación lineal de los vectores $(1,-1,0),(1,0,-1)$. \\

		Es fácil ver que que la lista $(1,-1,0),(1,0,-1)$ es linealmente independiente. Dado que, si $c_1,c_2\in\textbf{F}^n$, entonces 
		$$
		\begin{array}{rcl}
		    c_1(1,-1,0)+c_2(1,0,-1)&=&0\\
		    (c_1+c_2,-c_1,-c_2)&=&0.
		\end{array}
		$$
		De donde,
		$$c_1=c_2=0.$$
		Así, la lista $(1,-1,0),(1,0,-1)$ es linealmente independiente. Por lo tanto,$(1,-1,0),(1,0,-1)$ es una base de $\textbf{F}^3$\\\\

	    %---------- (g)
	    \item La lista $1,z,\ldots,z^m$ es una base de $\mathcal{P}_m(\textbf{F})$.\\\\
		Respuesta.-\; El elemento general de $\mathcal{P}_m(\textbf{F})$ es una combinación lineal de $1,z,z^2,\ldots,z^m$ de la forma:
		$$a_0+a_1z+a_2z^2+\cdots+a_mz^m$$
		donde $a_i\in \textbf{F}$ para $1\leq i \leq m$. Lo que demuestra que genera $\mathcal{P}_m(\textbf{F})$.\\

		Para demostrar que la lista es linealmente independiente, suponemos que la combinación lineal de estos elementos es igual a cero; es decir,
		$$a_0+a_1z+a_2z^2+\cdots+a_mz^m=0.$$
		Donde el $\textbf{0}$ es un polinomio. Esto implica que el polinomio del lado izquierdo toma valor cero para todo los valore de $z$. Esto es posible sólo cuando todos los $a_i's$ son cero, ya que cualquier polinomio no trivial tiene un número finito de raíces. Por lo tanto la lista $1,z,z^2,\ldots,z^m$ es base de $\mathcal{F}_m(\textbf{F})$.\\\\

	\end{enumerate}

    %-------------------- 3.
    \item 
	\begin{enumerate}[a).]

	    %---------- (a)
	    \item Sea $U$ el subespacio de $\textbf{R}^5$ definido por
	    $$U=\left\{(x_1,x_2,x_3,x_4,x_5)\in \textbf{F}^5\; : \; x_1=3x_2\;\mbox{y}\; x_3=7x_4\right\}.$$
	    Encuentre una base de $U$.\\\\
		Respuesta.-\; Dado que se tiene la condición $x_1=3x_2\;\mbox{y}\; x_3=7x_4$. Podemos escribir el vector general, como sigue
		$$
		\begin{array}{rcl}
		    (3x_2,x_2,7x_4,x_4,x_5) &=& (3x_2,x_2,0,0,0)+(0,0,7x_4,x_4,0)+(0,0,0,0,x_5)\\
					  &=& x_2(3,1,0,0,0)+x_4(0,0,7,1,0)+x_5(0,0,0,0,1).
		\end{array}
		$$
		Por lo que $(3,1,0,0,0),(0,0,7,1,0),(0,0,0,0,1)$ forma una base de $U$. Podemos demostrar fácilmente que estos vectores generar $U$, ya que $U$ puede expresarse como una combinación lineal de estos tres vectores. Ahora, demostremos que son linealmente independiente. Sea $c_1,c_2,c_3\in \textbf{F}$. Entonces, 
		$$c_1(3,1,0,0,0)+c_2(0,0,7,1,0)+c_3(0,0,0,0,1)=0$$
		De donde,
		$$(3c_1,c_2,7c_2,c_2,c_3)=(0,0,0,0,0).$$
		Igualando cada componente, tenemos que $c_1=c_2=c_3=0$. Así se demuestra que estos tres vectores son linealmente independientes. Por lo tanto, $(3,1,0,0,0),(0,0,7,1,0),(0,0,0,0,1)$ es una base de $U$.\\\\

	    %---------- (b)
	    \item Extienda la base de la parte (a) a una base de $\textbf{R}^5$.\\\\
		Respuesta.-\; Sean $v_1=(3,1,0,0,0)$, $v_2=(0,0,7,1,0)$, $v_3=c_3(0,0,0,0,1)$. Por el ejercicio 11 del apartado 2A (Axler, Linear Algebra), se sabe que si tenemos $v_4\notin \span(v_1,v_2,v_3)$ entonces $v_1,v_2,v_3,v_4$ son linealmente independientes. Nos preguntamos, ¿que clase de vectores no pueden ser generados por $v_1,v_2,v_3$?. Observemos que las primeras dos coordenadas de $v_2$ y $v_3$ son cero. Por lo que no pueden aportar a otras dos primeras coordenadas de cualquier combinación lineal que consideremos. De hecho, estas coordenadas deben provenir de $v_1$.\\
		Si $av_1+bv_2+cv_3$ es una combinación lineal, entonces las dos primeras coordenadas son $3a$ y $a$. Luego, si escogemos un vector donde sus primeras dos coordenada no son de la forma $3a$ y $a$ para cualquier escalar $a$, entonces no será generados por $v_1,v_2,v_3$. Por ejemplo podemos escoger el vector 
		$$v_4=(0,1,0,0,0)$$
		Ahora, encontremos un $v_5$ tal que $v_5\notin \span(v_1,v_2,v_3,v_4)$. Observemos que las coordenadas cuatro y cinco de los vectores $v_1.v_2,v_3$ y $v_4$ son cero. Por lo que ambas coordenadas deben provenir de $v_2$.\\
		Si  $av_1+bv_2+cv_3+dv_4$ es una combinación lineal, entonces las coordenadas cuatro y cinco son de la forma $7b$ y $b$, para cualquier escalar $b$. Por ejemplo podemos escoger el vector,
		$$v_5=(0,0,0,1,0).$$
		Por último, demostremos que esta lista es base de $\textbf{F}^5$. Por el teorema 2.23 sabemos que, si $m$ vectores generan un espacio vectorial, entonces cualquier lista linealmente independiente en $V$ no puede tener más de $m$ vectores. Así, queda demostrada la independencia lineal de $v_1,v_2,v_3,v_4,v_5$.\\

		Por otro lado, demostremos que esta lista genera $\textbf{F}^5$. Nuestro objetivo será hallar una combinación lineal que incluya a $v_1,v_2,v_3,v_4$ y $v_5$ para $a,b,c,d,e\in \textbf{F}$ tales que
		$$a(1,0,0,0,0)+b(0,0,1,0,0)+c(0,0,0,0,1)+d(0,1,0,0,0)+e(0,0,0,1,0)=(x_1,x_2,x_3,x_4,x_5).$$

		Para ello, notemos que ya se tiene $v_3=(0,0,0,0,5)$, $v_4=(0,1,0,0,0)$ y $v_5=(0,0,0,1,0)$. Ahora, generemos los restantes $(1,0,0,0,0)$ y $(0,0,1,0,0)$, de la siguiente manera 
		$$
		\begin{array}{rcl}
		    \dfrac{1}{3}(v_1-v_4)&=&(1,0,0,0,0)\\\\
		    \dfrac{1}{7}(v_2-v_5)&=&(0,0,1,0,0).
		\end{array}
		$$
		Dado que se incluye a $v_1$ y $v_2$ en combinación lineal con $v_4$ y $v_5$, entonces 
		$$
		\begin{array}{rcl}
		    a&=&x_1\\
		    b&=&x_2\\
		    c&=&x_3\\
		    d&=&x_4\\
		    e&=&x_5.
		\end{array}
		$$
		Encontrando los respectivos escalares $a,b,c,d,e$ en términos de $x_1,x_2,x_3,x_4,x_5$. Decimos que la lista $v_1,v_2,v_3,v_4,v_5$ genera $\textbf{F}^5$. Por lo tanto es una base de $\textbf{F}^5$.\\\\
		    


	    %---------- (c)
	    \item Encuentre un subespacio $W$ de $\textbf{F}^5$ tal que $\textbf{R}^5=U\oplus W$.\\\\
		Respuesta.- Por 1.45 (Axler, Lineal Algebra), demostraremos que $\textbf{R}^5=U+W$ y que $U\cap W=\left\{0\right\}$. Sea $v\in \textbf{R}^5$, ya que $v_1,v_2,v_3,v_4,v_5$ es una base de $\textbf{F}^5$, por el criterio de base (2.28, Axler, Lineal algebra) podemos escribir 
		$$
		\begin{array}{rcl}
		    v &=& c_1v_1+c_2v_2+c_3v_3+c_4v_4+c_5v_5\\
		      &=&(c_1v_1+c_2v_2+c_3v_3)+(c_4v_4+c_5v_5).
		\end{array}
		$$
		Luego, sean $u=c_1v_1+c_2v_2+c_3v_3$ y $w=c_4v_4+c_5v_5$. Entonces, $u\in U$ y $w\in W$. Está claro que $u$ y $w$ generan $U$ y $W$ respectivamente. De este modo, cada vector en $\textbf{R}^5$ puede ser expresado como una suma de vectores en $U$ y $W$. Esto prueba que $\textbf{R}^5=U+W$.\\

		Ahora demostremos que $U\cap W=\left\{0\right\}$. Sea $v\in U\cap W$, ya que $v\in U$ entonces para algunos escalares $a,b,c\in \textbf{F}$ se tiene
		$$v=av_1+bv_2+cv_3.$$
		Lo mismo pasa con $v\in W$, para algunos escalares $d,e\in \textbf{F}$; es decir,
		$$v=dv_4+ev_5.$$
		Dado que queremos encontrar $U\cap W$, se tiene
		$$av_1+bv_2+cv_3=dv_4+ev_5 \quad \Rightarrow \quad av_1+bv_2+cv_3-dv_4-ev_5=0$$
		Por el hecho de que $U$ y $W$ son linealmente independiente, lo que implica $a=b=c=d=e=0$, entonces $v=0$, así $U\cap W = \left\{0\right\}$. Concluimos que 
		$$\textbf{R}^5=U\oplus W.$$\\

	\end{enumerate}

    %-------------------- 4.
    \item 
	\begin{enumerate}[(a)]

	    %---------- (a)
	    \item Sea $U$ el subespacio de $\textbf{C}^5$ definida por 
	    $$U=\left\{(z_1,z_2,z_3,z_4,z_5)\in \textbf{C}^5\; :\; 6z_1=z_2\;\; \mbox{y}\;\; z_3+2z_4+3z_5=0\right\}.$$
	    Encuentre una base de $U$.\\\\
		Respuesta.-\; De las condiciones dadas, podemos escribir el conjunto $U$ como
		$$U=\left\{(6z_1,z_2,-2z_4-3z_5,z_4,z_5)\;:\;z_2,z_4,z_5\in \textbf{C}\right\}$$
		Sea $z\in U$, que implica
		$$
		\begin{array}{rcl}
		    z &=& (6z_1,z_2,-2z_4-3z_5,z_4,z_5)\\
		      &=& z_2(6,1,0,0,0)+z_4(0,0,-2,1,0)+z_5(0,0,-3,0,1).
		\end{array}
		$$
		Entonces, $z_2(6,1,0,0,0)$, $z_4(0,0,-2,1,0)$ y $z_5(0,0,-3,0,1)$ genera $U$. Ahora veamos si esta lista es linealmente independiente. Sean $a,b,c\in \textbf{F}$ tal que 
		$$(6a,a,-2b-3c,b,c)=0.$$
		De donde,
		$$
		\begin{array}{rcl}
		    6a&=&0\\
		    a&=&0\\
		    -2b-3c&=&0\\
		    b&=&0\\
		\end{array}
		\quad \Rightarrow \quad 
		\begin{array}{rcl}
		    a&=&0\\
		    b&=&0\\
		    c&=&0.
		\end{array}
		$$
		Por lo tanto, la lista es linealmente independiente. Así, concluimos que $U$ es generado por $z_2(6,1,0,0,0)$, $z_4(0,0,-2,1,0)$ y $z_5(0,0,-3,0,1)$.\\\\

	    %---------- (b)
	    \item Extienda la base en la parte (a) para una base de $\textbf{C}^5$.\\\\
		Respuesta.-\;

	    %---------- (c)
	    \item Encuentre un subespacio $W$ de $\textbf{C}^5$ tal que $\textbf{C}^5=U\oplus W$.\\\\
		Respuesta.-\; 

	\end{enumerate}

    %-------------------- 5.
    \item Demostrar o refutar: existe una base $p_0,p_1,p_2,p_3$ de $\mathcal{P}_3(\textbf{F})$ tal que ninguno de los polinomios $p_0,p_1,p_2,p_3$ tiene grado $2$.\\\\
	Demostración.-\; Consideremos la lista,
	$$
	\begin{array}{rcl}
	    p_0&=&1,\\
	    p_1&=&X,\\
	    p_2&=&X^3+X^2,\\
	    p_3&=&X^3.
	\end{array}
	$$
	El cual no tiene ningún polinomio de grado $2$. Demostraremos que esta lista es una base. Primero veamos que $\span(p_0,P_1,p_2,p_3)=\mathcal{P}_3(\textbf{F})$. Sea $q\in \mathcal{P}_3(\textbf{F})$. Entonces existe $a_0,a_1,a_2,a_3\in F$ alguno cero, tal que
	$$q=a_0+a_1X+a_2X^2+a_3X^3.$$
	Notemos que
	$$
	\begin{array}{rcl}
	    a_0p_0+a_1p_1+a_2(p_2-p_3)+a_3p_3 &=& a_0p_0+a_1p_1+a_2p_2+a_3p_3-a_2p_3\\\\
					      &=& a_0p_0+a_1p_1+a_2p_2+(a_3-a_2)p_3\\\\
					      &=& a_0+a_1X+a_2\left(X^3+X^2\right)+(a_3-a_2)X^3\\\\
					      &=& a_0+a_1X+a_2X^3+a_2X^2+a_3X^3-a_2X^3\\\\
					      &=& a_0+a_1X+a_2X^2+a_3X^3\\\\
					      &=& p.

	\end{array}
	$$

	Por lo que $p_0,p_1,p_2,p_3$ genera $\mathcal{P}_3(\textbf{F})$. Ahora, demostraremos que la lista es linealmente independiente. Sean $b_0,\ldots,b_3\in \textbf{F}$ tales que

	$$b_0p_0+b_1p_1+b_2p_2+b_3p_3=0.$$

	Se sigue que,

	$$
	\begin{array}{rcl}
	    b_0+b_1X+b_2(X^2+X^3)+b_3X^3&=&0\\\\
	    b_0+b_1X+b_2X^2+b_2X^3 +b_3X^3&=&0\\\\
	    b_0+b_1X+b_2X^2+(b_2+b_3)X^3&=&0.
	\end{array}
	$$
	donde $\textbf{0}$ es el cero polinomial. La lista $\left(1,X,X^2,X^3\right)$ es linealmente independiente, ya que es una base en $\mathcal{P}_3(\textbf{F})$. Por lo que 
	$$
	\begin{array}{rcl}
	    b_0&=&0,\\
	    b_1&=&0,\\
	    b_2&=&0,\\ 
	    b_2+b_3&=&0.
	\end{array}
	$$
	Por lo tanto, existe una base $p_0,p_1,p_2,p_3$ de $\mathcal{P}_3(\textbf{F})$ tal que ninguno de los polinomios tiene grado $2$.\\\\

    %-------------------- 6.
    \item Suponga $v_1,v_2,v_3,v_4$ es una base de $V$. Demostrar que 
    $$v_1+v_2,v_2+v_3,v_3+v_4,v_4$$
    es también una base de $V$.\\\\
	Demostración.-\; Demostremos la independencia lineal. Sean $a_1,a_2,a_3,a_4\in \textbf{F}$ tales que
	$$
	\begin{array}{rcl}
	    a_1(v_1+v_2)+a_2(v_2+v_3)+a_3(v_3+v_4)+a_4v_4&=&0\\\\
	    a_1v_1+a_1v_2+a_2v_2+a_2v_3+a_3v_3+a_3v_4+a_4v_4&=&0\\\\
	    a_1v_1+(a_1+a_2)v_2+(a_2+a_3)v_3+(a_3+a_4)v_4&=&0\\\\
	\end{array}
	$$
	Ya que $v_1,v_2,v_3,v_4$ es linealmente independiente, entonces $a_1=a_1+a_2=a_2+a_3=a_3+a_4=0$. Por lo que  $v_1+v_2,v_2+v_3,v_3+v_4,v_4$ es linealmente independiente.\\

	Luego, demostremos que $v_1+v_2,v_2+v_3,v_3+v_4,v_4$ es una base de $V$. Por definición de generador (span), podemos expresar $v_1,v_2,v_3,v_4$ como combinaciones lineales de $v_1+v_2,v_2+v_3,v_3+v_4+v_4$, de la siguente manera 
	$$
	\begin{array}{rcl}
	    v_3 &=& (v_3+v_4)-v_4\\\\
	    v_2 &=& (v_2+v_3)-(v_3+v_4)+v_4\\\\
	    v_1 &=& (v_1+v_2)-(v_2+v_3)+(v_3+v_4)-v_4\\\\
	\end{array}
	$$
	Por tanto, todos los vectores que pueden expresarnse como combinaciones lineales de $v_1,v_2,v_3,v_4$ también se pueden expresar linealmente por $v_1,+v_2,v_2+v_3,v_3+v_4,v_4$; es decir, $v_1,+v_2,v_2+v_3,v_3+v_4,v_4$ genera $V$.\\\\

    %-------------------- 7.
    \item Demostrar o dar un contraejemplo: Si $v_1,v_2,v_3,v_4$ es una base de $V$ y $U$ es un subespacio de $V$ tal que $v_1,v_2\in U$ y $v_3\notin U$ y $v_4\notin U$, entonces $v_1,v_2$ es una base de $U$.\\\\
	Demostración.-\; Sean,
	$$
	\begin{array}{rcl}
	    v_1 &=& (1,0,0,0)\\
	    v_2 &=& (0,1,0,0)\\
	    v_3 &=& (0,0,1,0)\\
	    v_4 &=& (0,0,0,1).
	\end{array}
	$$
	Luego, definimos 
	$$U=\left\{(x_1,x_2,x_3,x_4)\in \textbf{R}^4 \; :\; x_3=x_4\right\}$$

	Notemos que $v_1,v_2\in U$ y $v_3,v_4\notin U$. Pero ninguna combinación lineal de $v_1,v_2$ produce $(0,0,1,1)$. Entonces, $v_1,v_2$ no genera $U$. Por lo tanto no puede formar una base.\\\\

    %-------------------- 8.
    \item Suponga que $U$ y $W$ son subespacio de $V$ tal que $V=U\oplus W$. Suponga también que $u_1,\ldots,u_m$ es una base de $U$ y $w_1,\ldots,w_n$ es una base de $W$. Demostrar que 
    $$u_1,\ldots,u_m,\; w_1,\ldots,w_n$$
    es una base de $V$.\\\\
	Demostración.-\; Demostremos la independencia lineal. Sean $a_i\in \textbf{F}$ y $c_i\in \textbf{F}$ tal que
	$$a_1u_1+a_2u_2+\cdots+a_mu_m+c_1w_1+c_2w_2+\cdots+c_nw_n=0$$
	Que implica,
	$$a_1u_1+a_2u_2+\cdots+a_mu_m=-(c_1w_1+c_2w_2+\cdots+c_nw_n).$$
	Suponga que,
	$$v=a_1u_1+a_2u_2+\cdots+a_mu_m=-(c_1w_1+c_2w_2+\cdots+c_nw_n).$$
	De donde, $v\in U$ y $v\in W$; esto es $v\in U\cap W.$ Dado que $V=U\oplus W$, debemos tener $U\cap W=\left\{0\right\}$. Sea $v=0$, por lo que
	$$
	\begin{array}{rcl}
	    a_1u_1+a_2v_2+\cdots + a_mu_m &=& 0\\\\
	    -(c_1w_1+c_2w_2+\cdots + c_nw_n) &=& 0.
	\end{array}
	$$
	Ya que, $u_1,u_2,\ldots,u_m$ y $w_1,w_2,\ldots,w_n$ es base de $U$ y $W$, respectivamente. Entonces, ambos son linealmente independiente. Es decir, $a_i=c_i=0$. Por lo tanto, $u_1,\ldots,u_m,\; w_1,\ldots,w_n$ es linealmente independiente.\\
	Ahora, demostremos que $u_1,\ldots,u_m,\; w_1,\ldots,w_n$ genera $V$. Sea $v\in V$, ya que $V=U\oplus W$ podemos escribir $v=u+w$ para algún $u\in U$ y $w\in W$. Luego, por el hecho de que $u_1,u_2,\ldots,u_m$ es base de $U$ y $w_1,w_2,\ldots,w_n$ es base de $W$, entonces
	$$
	\begin{array}{rcl}
	    u &=& a_1u_1+a_2u_2+\cdots+a_mu_m\\\\
	    w &=& c_1w_1+c_2w_2+\cdots+c_nw_n,
	\end{array}
	$$
	respectivamente. Por lo tanto, $v=u+w=a_1u_1+a_2u_2+\cdots+a_mu_m+c_1w_1+c_2w_2+\cdots+c_nw_n$. Así, $u_1,u_2,\ldots,u_m,\, w_1,w_2,\ldots,w_n$ genera $V$. Concluimos que  $u_1,u_2,\ldots,u_m,\, w_1,w_2,\ldots,w_n$ es base de $V$.\\\\

\end{enumerate}


\mysection{Dimensión}

%------------------------ 2.35 teorema
\begin{myteo}[La longitud de la base no depende de la base]\,\\\\
    Dos bases cualquier de un espacio vectorial de dimensión finita  tiene la misma longitud.\\\\
	Demostración.-\; Suponga $V$ es de dimensión finita. Sean $B_1$ y $B_2$ dos bases de $V$. Entonces $B_1$ es linealmente independiente en $V$ y $B_2$ genera $V$. Así la longitud de la lista de $B_1$ es como máximo la longitud de la lista de $B_2$ (por 2.23). Intercambiando los roles de $B_1$ y $B_2$. tenemos también que la longitud de la lista de $B_2$ es como máximo la longitud de la lista de $B_1$. Por lo tanto la longitud de $B_1$ es igual a la longitud de $B_2$, como se desea.
\end{myteo}

Ahora que sabemos que dos bases cualesquiera de un espacio vectorial de dimensión finita tienen la misma longitud, podemos definir formalmente la dimensión de tales espacios.

%------------------------ 2.36 definicion
\begin{mydef}[Dimensión, \boldmath $\dim V$]\;\\
    \begin{itemize}
	\item La \textbf{dimensión} de un espacio vectorial de dimensión finita es la longitud de cualquier base del espacio vectorial.
	\item La dimensión de $V$ (si $V$ es de dimensión finita) se denota por $\dim V$.
    \end{itemize}
\end{mydef}

\setcounter{myteo}{37}
%------------------------ 2.37 teorema
\begin{myteo}[Dimensión de un subespacio]\, \\\\
    Si $V$ es de dimensión finita y $U$ es un subespacio de $V$, entonces $\dim U\leq \dim V$.\\\\
	Demostración.-\; Suponga $V$ de dimensión finita y $U$ es un subespacio de $V$. Piense en una base de $U$ como una lista linealmente independiente en $V$, y piense en una base de $V$ como una lista generadora en V. Ahora use $2.23$ para concluir que $\dim U\leq \dim V$.
\end{myteo}

Para comprobar que una lista de vectores en $V$ es una base de $V$, debemos, según la definición, demostrar dos propiedades: debe ser linealmente independiente y debe generar $V$. Los siguientes dos resultados muestran que si la lista en cuestión tiene la longitud correcta, entonces solo necesitamos verificar que satisface una de las dos propiedades requeridas. Primero probamos que toda lista linealmente independiente con la longitud correcta es una base.

%------------------------ 2.39 teorema
\begin{myteo}[La lista linealmente independiente de la longitud correcta es una base]\,\\\\
    Suponga que $V$ es de dimensión finita. Entonces, toda lista linealmente independiente de vectores en $V$ con longitud $\dim V$ es una base de $V$.\\\\
	Demostración.-\; Suponga $\dim V=n$ y $v_1,\ldots,v_n$ es linealmente independiente en $V$. La lista $v_1,\ldots,v_n$ puede ser extenderse a una base de $V$ (por 2.33). Sin embargo, toda base de $V$ tiene una longitud $n$, por lo que en este caso la extensión es trivial, lo que significa que ningún elemento está unido a $v_1,\ldots, v_n$ En otras palabras, $v_1,\ldots, v_n$ es una base de $V$, como deseamos.
\end{myteo}

%------------------------ 2.40 ejemplo
\begin{myejem}
    Demostrar que la lista $(5,7),(4,3)$ es una base de $\textbf{F}^2$.\\\\
	Demostración.-\; Esta lista de dos vectores en $\textbf{F}^2$ es obviamente linealmente independiente (poque ningún vector es un multiplo escalar del otro). Notemos que $\textbf{F}^2$ tiene dimensión $2$. Así, 2.39 implica que la lista linealmente independiente $(5,7),(4,3)$ de longitud $2$ es una base de $\textbf{F}^2$ $\left(\right.$ no necesitamos comprobar que genera $\left.\textbf{F}^2\right)$. 
\end{myejem}

%------------------------ 2.41 ejemplo
\begin{myejem}
    Demostrar que $1,(x-5)^2,(x-5)^3$ es una base del subespacio $U$ de $\mathcal{P}_3(\textbf{R})$ defina por
    $$U=\left\{p\in \mathcal{P}_3(\textbf{R})\; : \; p'(5)=0\right\}.$$
	Solución.-\; Claramente cada los polinomios $1,(x-5)^2$ y $(x-5)^3$ es en $U$. Suponga $a,b,c \in \textbf{R}$ y 
	$$a+b(x-5)^2+c(x-5)^3=0$$
	para cada $x\in \textbf{R}.$ Vemos que el lado izquierdo de la ecuación anterior tiene un termino $cx^3$. Como el lado derecho no tiene un término $x^3$, esto implica que $c = 0$. Como $c = 0$, vemos que el lado izquierdo tiene un término $bx^2$, lo que implica que $b = 0$. Como $b = c = 0$, podemos también concluye que $a = 0$. Así, la ecuación implica que $a=b=c=0$. Por lo que la lista dada es linealmente independiente en $U$.\\

	Observemos que $\dim U\geq 3$, ya que $U$ es un subespacio de $\mathcal{P}_3(\textbf{F})$, sabemos que $\dim U \leq \dim \mathcal{P}_3(\textbf{F})=4$ (por 2.38). Sin embargo, $\dim U$ no puede ser igual a $4$, de lo contrario, cuando extendemos una base de $U$ a una base de $\mathcal{P}_3(\textbf{F})$ obtendríamos una lista con una longitud mayor que $4$. Por lo tanto, $\dim U = 3$. De esto modo, 2.39 implica que la lista linealmente independiente $1,(x-5)^2,(x-5)^3$ es una base de $U$.
\end{myejem}

Ahora probaremos que una lista generadora con longitud correcta es una base.

%------------------------ 2.42 teorema
\begin{myteo}[Lista generadora de la longitud correcta es una base]\,\\\\
	Suponga que $V$ es de dimensión finita. Entonces, cada lista generadora de vectores en $V$ con longitud $\dim V$ es una base de $V$.\\\\
	    Demostración.-\; Suponga $\dim V=n$ y $v_1,\ldots,v_n$ genera $V$. La lista $v_1,\ldots,v_n$ puede ser reducido a una base de $V$ (por 2.31). Sin embargo, cada base de $V$ tiene longitud $n$. En este caso la reducción es trivial, lo que significa que no se eliminan elementos de $v_1,\ldots,v_n$. En otras palabras, $v_1,\ldots,v_n$ es una base de $V$, como deseamos.
\end{myteo}

El siguiente resultado da una fórmula para la dimensión de la suma de dos subespacios de un espacio vectorial de dimensión finita.

%------------------------ 2.43 teorema
\begin{myteo}[Dimensión de una suma]\,\\\\
    Si $U_1$ y $U_2$ son subespacios de un espacio vectorial de dimensión finita, entonces
    $$\dim(U_1+U_2)=\dim U_1+\dim U_2 - \dim(U_1\cap U_2).$$\\
	Demostración.-\; Sea $u_1,\ldots,u_m$ una base de $U_1\cap U_2$; por lo que $(U_1\cap U_2)=m$. Ya que, $u_1,\ldots,u_m$ es una base de $U_1\cap U_2$, es linealmente independiente en $U_1$. En consecuencia esta lista puede ser extendida a una base $u_1,\ldots,u_m,\, v_1,\ldots,v_j$ de $U_1$ (por 2.33). Así, $\dim U_1=m+j$. También extendamos $u_1,\ldots,u_m$ a una base $u_1,\ldots,u_m,\, w_1,\ldots,w_k$ de $U_2$; de donde $U_2=m+k$.\\
	Tendremos que demostrar
	$$u_1,\ldots,u_m,\, v_1,\ldots,v_j,\,w_1,\ldots,w_k$$
	es una base de $U_1+U_2$. Claramente $\span(u_1,\ldots,u_m,\, v_1,\ldots,v_j,\,w_1,\ldots,w_k)$ contiene a $U_1$ y $U_2$, por lo que es igual a $U_1+U_2$. Entonces, para mostrar que la lista es una base de $U_1+U_2$, necesitamos demostrar que es linealmente independiente. Suponga
	$$a_1u_1+\cdots+a_mu_m+b_1v_1+\cdots+b_jv_j+c_1w_1+\cdots+c_kw_k=0,$$
	donde todos los $a's,b's$ y $c's$ son escalares. Podemos reescribir la ecuación anterior como
	$$c_1w_1+\cdots+c_kw_k=-a_1u_1-\cdots-a_mu_m-b_1v_1-\cdots-b_jv_j$$
	el cual demuestra que $c_1w_1+\cdots+c_kw_k\in U_1$. Luego, todos los $w's$ son en $U_2$, esto implica que $c_1w_1+\cdots+c_kw_k\in U_1\cap U_2$. Ya que, $u_1,\ldots,u_m$ es una base de $U_1\cap U_2$, podemos escribir lo siguiente
	$$c_1w_1+\cdots+c_kw_k=d_1u_1+\cdots+d_m u_m$$
	para algunos escalares $d_1,\ldots,d_m$. Pero $u_1,\ldots,u_m,w_1,\ldots,w_k$ es linealmente independiente, así la útlima ecuación implica que todos los $c's$ (y $d's$) son igual a cero. \\
	
	Por lo tanto, nuestra ecuación original que involucra las $a's, b's$ y $c's$ se convierten en 
	$$a_1u_1+\cdots+a_mu_m+b_1v_1+\cdots+b_jv_j=0.$$
	Porque la lista $u_1,\ldots, u_m, v_1, \ldots, v_j$ es linealmente independiente, esta ecuación implica que todas las $a's$ y $b's$ son $0$. Ahora sabemos que todas las $a's$, $b's$ y $c's$ son iguales a $0$, como deseamos.
\end{myteo}
\vspace{.5cm}

\setcounter{mysection}{2}
\mysection{Ejercicios}

\begin{enumerate}[\bfseries 1.]

    %------------------------ 1.
    \item Suponga que $V$ es de dimensión finita y $U$ es un subespacio de $V$ tal que $\dim U=\dim V$. Demuestre que $U=V$.\\\\
	Demostración.-\; Debemos demostrar que $U\subseteq V$ y $V\subseteq U$. Es fácil ver que $U\subseteq V$, ya que $U$ es un subespacio de $V$.\\
	Por otro lado, sean $v\in V$ y $u_1,u_2,\ldots,u_n$ base de $U$. Entonces, este conjunto es linealmente independiente en $U$, y por lo tanto también en $V$, esto porque $U$ es un subespacio de $V$. Que sea base de $U$ significa que $\dim U =n.$ Sin embargo, $\dim U = \dim V$ implica que $\dim V = n$. Así, $u_1,u_2,\ldots,u_n$ es un conjunto linealmente independiente en $V$ con longitud igual a $\dim V$. Es decir, por 2.39 (Axler, Linear Algebra) $u_1,u_2,\ldots,u_n$ es una base de $V$, por lo que genera $V$. Esto es, por 1.28 (Criterio de base) existe escalares $c_i\in \textbf{F}$ tal que
	$$v=c_1u_1+c_2u_2+\cdots+c_nu_n.$$
	Notemos que $v_1u_1+\cdots+c_nu_n$ es un vector en $U$. Por lo tanto, $v\in U$. Que $V\subseteq U$ y $U\subseteq V$ implica que $U=V$, como queríamos demostrar.\\\\

    %------------------------ 2.
    \item Demostrar que los subespacios de $\textbf{R}^2$ son precisamente: $\left\{0\right\}$, $\textbf{R}^2$, y todas las rectas en $\textbf{R}^2$ que pasan por el origen.\\\\
	Demostración.-\; Claramente $\left\{0\right\}$ es un subespacio de $\textbf{R}^2$, porque contiene el vector cero, que está cerrado bajo la adición y la multiplicación escalar. En particular, cualquier combinación lineal del vector cero sigue siendo el vector cero.\\

	Ahora, supongamos que $U$ es un subespacio de $\textbf{R}^2$ con $\dim U = 1$. En otras palabras, la base de $U$ contiene solo un vector distinto de cero. Esto significa, que la lista contiene un vector distinto de cero que genera $U$. Esto implica que cada vector en $U$ es un múltiplo escalar (combinación lineal) del único vector de la base. Luego,  el conjunto de todos estos vectores en $U$ describe una recta en $\textbf{R}^2$ ; esto es, $U$ es una recta en $\textbf{R}^2$. Además, como $U$ es un subespacio, en particular contiene la identidad aditiva $(0, 0) \in \textbf{R}^2$. Por tanto, $U$ debe ser una recta en $\textbf{R}^2$ que pase por el origen.\\

	Por último, sea $U$ un subespacio de $\textbf{R}^2$ de dimensión $2$. Entonces, por definición $U$ tiene una base de dos vectores, digamos $u_1$ y $u_2$. Estas bases son linealmente independientes en $U$ y por lo tanto linealmente independiente en $\textbf{R}^2$. Sabiendo que $\dim U = \dim \textbf{R}^2=2$,  por 2.39 (Axler, Linear Algebra) $u_1,u_2$ es también una base de $\textbf{R}^2$. Que $U$ y $\textbf{R}^2$ tengan la misma base, por la unicidad del criterio de base 2.28 (Axler, Linear Algebra) significa que $U=\textbf{R}^2$.\\\\

    %------------------------ 3.
    \item Demuestre que los subespacios de $\textbf{R}^3$ son precisamente $\left\{0\right\}$, $\textbf{R}^3$, todas las lineas en $\textbf{R}^3$, y todas las planos en $\textbf{R}^3$ que pasan por el origen.\\\\ 
	Demostración.-\; Suponga
	El conjunto $\left\{0\right\}$ es un subespacio de $\textbf{R}^3$, ya que está cerrado bajo la adición y la multiplicación escalar. Es decir, para cualquier vector $u$ y $v$ en $\left\{0\right\}$ y cualquier escalar $c$, tenemos $u+v=0+0=0$ y $c u =c(0)=0$, ambos también están en $\left\{0\right\}.$\\

	Suponga $U$ un subespacio de $\textbf{R}^2$ de $\dim U = 1$. Entonces la longitud de la base de $U$ es $1$; en otras palabras, la base de $U$ contiene solo un vector no nulo. En particular, la lista contiene un vector no nulo que genera $U$; es decir, cada vector en $U$ es un múltiplo escalar (combinación lineal) del único vector de la base. Luego, el conjunto de todos estos vectores en $U$ describe una recta en $\textbf{R}^3$. Esto es, $U$ es una recta en $\textbf{R}^3$. Además, como $U$ es un subespacio, podemos asegurar que contiene la identidad aditiva $(0,0,0)\in \textbf{R}^3$. Por lo tanto, $U$ debe ser una recta en $\textbf{R}^3$ que pasa por el origen.\\

	Suponga $U$ un subespacio de $\textbf{R}^3$ de $\dim U = 2$. Entonces la longitud de la base de $U$ es $2$. En particular, la base de $U$ contiene dos vectores no nulos. En particular, la lista contiene dos vectores no nulos que genera $U$. Es decir, cada vector en $U$ es una combinación lineal de los dos vectores de la base. Luego, el conjunto de todos estos vectores en $U$ describe un plano en $\textbf{R}^3$. Esto es, $U$ es un plano en $\textbf{R}^3$. Además, como $U$ es un subespacio, podemos asegurar que contiene la identidad aditiva $(0,0,0)\in \textbf{R}^3$. Por lo tanto, $U$ debe ser un plano en $\textbf{R}^3$ que pasa por el origen.\\
	
	Después, sea $U$ un subespacio de $\textbf{R}^3$ de dimensión $3$. Entonces, por definición $U$ tiene una base de dos vectores, digamos $u_1$ $u_2$ y $u_3$. Estas bases son linealmente independientes en $U$ y por lo tanto linealmente independiente en $\textbf{R}^3$. Sabiendo que $\dim U = \dim \textbf{R}^3=3$,  por 2.39 (Axler, Linear Algebra) $u_1,u_2,u_3$ es también una base de $\textbf{R}^3$. Que $U$ y $\textbf{R}^3$ tengan la misma base, por la unicidad del criterio de base 2.28 (Axler, Linear Algebra) significa que $U=\textbf{R}^3$.\\\\


    %------------------------ 4.
    \item 
	\begin{enumerate}[(a)]

	    %---------- (a)
	    \item Sea $U=\left\{p\in \mathcal{P}_4(\textbf{F})\; : \; p(6)=0\right\}.$ Encuentre una base de $U$.\\\\
		Respuesta.-\; Sea $q(x)$ de grado $n-1$. Si $p(x)$ es un polinomio y $p(c)=0$, entonces $c$ se dice que es una raíz de $p(x)$ y $p(x)=(x-c)q(x)$, ya que 
		$$p(6)=(6-6)q(6)=0.$$
		En particular, $p(x)\in \mathcal{P}_4(\textbf{F})$ tal que $p(6)=0$. Que podemos reescribirlo como $p(x)=(x-6)q(x)$, donde $q(x)\in \mathcal{P}_3(\textbf{F})$. Además, que $q(x)\in \mathcal{P}_3(\textbf{F})$, implica que $(x-6)q(x)\in \mathcal{P}_4(\textbf{F})$ y $6$ como raíz de $(x-6)q(x)$. Así,
		$$\left\{p\in \mathcal{P}_4(\textbf{F})\; |\; p(6)=0\right\}=\left\{(x-6)q(x)\; |\; q\in\mathcal{P}_3(\textbf{F})\right\}.$$
		Por el problema 2(g) del apartado 2.B (Axler, Linear Algebra), se sabe que $q(x)=1,x,x^2,x^3$ es una base de $\mathcal{P}_3(\textbf{F})$. De donde, demostremos que 
		$$(x-6),(x-6)x,(x-6)x^2,(x-6)x^3$$
		forma una base de $U$. Si $q(x)=a+bx+cx^2+dx^3\in \mathcal{P}_4(\textbf{F})$ para $a,b,c,d\in \textbf{F}$. Entonces,
		$$
		\begin{array}{rcl}
		    (x-6)q(x)&=&(x-6)(a+bx+cx^2+dx^3)\\
			     &=&a(x-6)+b(x-6)+c(x-6)x^2+d(x-6)x^3.\\
		\end{array}
		$$
		Esto implica que $(x-6),(x-6)x,(x-6)x^2,(x-6)x^3$ genera $U$. Por último, debemos probar que $(x-6),(x-6)x,(x-6)x^2,(x-6)x^3$ es linealmente independiente. Existe $a,b,c,d\in \textbf{F}$ tal que
		$$a(x-6)+b(x-6)x+c(x-6)x^2+d(x-6)x^3=0.$$
		Entonces,
		$$-6a+(a-6b)x+(b-6c)x^2+(c-7d)x^3+dx^4=0.$$
		Para que la lista sea linealmente independiente, cada coeficiente debe ser cero. En consecuencia,
		$$
		\begin{array}{rcl}
		    -6a&=&0\\
		    a-6b&=&0\\
		    b-6c&=&0\\
		    c-7d&=&0\\
		    d&=&0.
		\end{array}
		$$
		Resolviendo la ecuación, se tiene
		$$
		\begin{array}{rcl}
		    a&=&0\\
		    b&=&0\\
		    c&=&0\\
		    d&=&0.
		\end{array}
		$$
		Así, la lista es linealmente independiente. Por lo tanto, concluimos que 
		$$(x-6),(x-6)x,(x-6)x^2,(x-6)x^3$$
		es una base de $U$.\\\\

	    %---------- (b)
	    \item Extienda la base de $U$ en (a) a una base de $\mathcal{P}_4(\textbf{F})$.\\\\
		Respuesta.-\; La condición del inciso (a) hace que 4 polinomios generen $U$. Ahora, por el problema 13 de la sección 2B (Axler, Linear Algebra); observamos que tenemos que tener 5 polinomios para que genera $\mathcal{P}_4(\textbf{F})$. Para ello debemos extender $U$ del inciso (a). Notemos que $U$ tiene todos sus polinomios múltiplos de $(x-6)$, por lo que cualquier combinación lineal de $U$ producirá otro polinomio múltiplo de $(x-6)$. Por el contrario un polinomio el cual no es múltiplo de $(x-6)$ no podrá pertenecer al generador de $U$; así que podemos agregar $1$  a $U$. Está claro por el inciso (a) que los elementos de $U$ son linealmente independientes y por lo dicho anteriormente, ninguna combinación lineal de los elementos de $U$ pueden generar $1$ (definición de independencia). Por lo tanto, 
		$$\left\{1\right\}\cup U = \left\{1,(x-6),(x-6)x,(x-6)x^2,(x-6)x^3\right\}$$
		es linealmente independiente en $\mathcal{P}_4(\textbf{F})$. Observemos que esta lista contiene longitud igual a $\dim \mathcal{P}_4(\textbf{F})$: Entonces por 2.39 (Axler, Linear Algebra), concluimos que 
		$$\left\{1,(x-6),(x-6)x,(x-6)x^2,(x-6)x^3\right\}$$ 
		es una base de $\mathcal{P}_4(\textbf{F})$.\\\\


	    %---------- (c)
	    \item Encuentre un subepacio $W$ de $\mathcal{P}_4(\textbf{F})$ tal que $\mathcal{P}_4(\textbf{F})=U\oplus W$.\\\\
		Respuesta.-\; Supongamos $W=\left\{1\right\}$. Debemos probar que $U+W=\mathcal{P}_4(\textbf{F})$ y que $U\cap W=\left\{0\right\}$. Ya que, $U\cup W$ contiene todos los polinomios base de $\mathcal{P}_4(\textbf{F})$, el espacio vectorial $U+W$ contiene a $\mathcal{P}_4(\textbf{F})$. Esto es,
		$$\mathcal{P}_4(\textbf{F})\subseteq V+W.$$
		Por otro lado, puesto que $U,W\subseteq \mathcal{P}_4(\textbf{F})$, entonces
		$$U+W\subseteq \mathcal{P}_4(\textbf{F}).$$
		Así, 
		$$U+W=\mathcal{P}_4(\textbf{F}).$$

		Ahora, demostremos que $U\cap W=\left\{0\right\}$. Sea $v\in U\cap W$. Como $v\in U$, entonces  existe $c_i\in \textbf{F}$ tal que,
		$$v=c_1(x-6)+c_2(x-6)x+c_3(x-6)x^2+c_4(x-6)x^4.$$
		De la misma forma, ya que $v\in W$ entonces existe $c_0\in \textbf{F}$ tal que,
		$$v=c_0.$$
		Luego,
		$$c_1(x-6)+c_2(x-6)x+c_3(x-6)x^2+c_4(x-6)x^4=c_0.$$
		Esto implica que
		$$-c_0+c_1(x-6)+c_2(x-6)x+c_3(x-6)x^2+c_4(x-6)x^4=0.$$
		Para que la lista linealmente independiente, cada coeficiente debe ser cero. En consecuencia,
		$$
		\begin{array}{rcl}
		    -c_0&=&0\\
		    c_1(x-6)&=&0\\
		    c_2(x-6)&=&0\\
		    c_3(x-6)&=&0\\
		    c_4(x-6)&=&0.
		\end{array}
		\quad \Rightarrow \quad 
		\begin{array}{rcl}
		    c_0&=&0\\
		    c_1&=&0\\
		    c_2&=&0\\
		    c_3&=&0\\
		    c_4&=&0.
		\end{array}
		$$
		Así, $v=0$. Por lo tanto, $U\cap W=\left\{0\right\}$. Otra manera de demostrar que la lista dada es linealmente independiente sería, suponer que
		$$-c_0+c_1(x-6)+c_2(x-6)x+c_3(x-6)x^2+c_4(x-6)x^4\neq0$$
		para todo $x$. Pero esto es absurdo, ya que si $x=6$, entonces 
		$$-c_0+c_1(x-6)+c_2(x-6)x+c_3(x-6)x^2+c_4(x-6)x^4=0.$$
		Concluimos que $W=\left\{1\right\}$ es un subespacio de $\mathcal{P}_4(\textbf{F})$ tal que $\mathcal{P}_4(\textbf{F})=U\oplus W$.\\\\

	\end{enumerate}

    %-------------------- 5.
    \item 
	\begin{enumerate}[(a)]

	    %---------- (a)
	    \item Sea $U=\left\{p\in \mathcal{P}_4(\textbf{F})\; :\; p''(6)\right\}$. Encuentre una base de $U$.\\\\
		Respuesta.-\; Sabemos que si $p(x)\in \mathcal{P}_4(\textbf{F})$, entonces $p''(x)\in \mathcal{P}_2(\textbf{F})$. De ahí, 
		$$\left\{p''(x)\; |\; p(x)\in \mathcal{P}_4(\textbf{F})\right\}=\mathcal{P}_2(\textbf{F}).$$
		Además, si $p''(6)=0$, entonces $p''(x)=(x-6)q(x)$, como se vio en el ejercicio anterior se cumple $p''(6)=(6-6)q(x)=0$, donde $q(x)$ tiene un grado menos que $p''(x)$. Esto es $q(x)\in \mathcal{P}_1(\textbf{F})$.\\

		Queremos encontrar elementos que generen $\mathcal{P}_2(\textbf{F})$. Analicemos un subepacio $S$ de $\mathcal{P}_2(\textbf{F})$ tal que 
		$$S=\left\{u(x)\in \mathcal{P}_2(\textbf{F})\; |\; (x-6)q(x),q(x)\in \mathcal{P}_1(\textbf{F})\right\}.$$ 
		Existe $a,b\in \textbf{F}$ tal que $q(x)=a+bx\in \mathcal{P}_1(\textbf{F})$, por lo que
		$$
		\begin{array}{rcl}
		    (x-6)q(x) &=& (x-6)(a+bx)\\
			      &=& a(x-6)+bx(x-6)\\
			      &=& a(x-6)+b(x-6+6)(x-6)\\
			      &=& a(x-6)+b(x-6)^2+6b(x-6)\\
			      &=& (a+6b)(x-6)+b(x-6)^2.
		\end{array}
		$$
		Lo que demuestra que $(x-6),(x-6)^2$ genera $S$. Sea $c_1,c_2\in \textbf{F}$, tal que
		$$c_1(x-6)+c_2(x-6)^2=0.$$
		De donde,
		$$6(4c_2-6c_1)+(c_1-12c_2)x+c_2x^2=0.$$
		Para que la lista sea linealmente independiente, cada coeficiente debe ser cero. En consecuencia,
		$$
		\begin{array}{rcl}
		    6(4c_2-6c_1)&=&0\\
		    (c_1-12c_2)&=&0\\
		    c_2&=&0.
		\end{array}
		\quad \Rightarrow \quad
		\begin{array}{rcl}
		    c_1&=&0\\
		    c_2&=&0.
		\end{array}
		$$
		Así, $(x-6),(x-6)^2$ es linealmente independiente. Por lo tanto, 
		$$\left\{(x-6),(x-6)^2\right\}$$
		es una base de $U$.\\

		Ahora, encontraremos una base de $p(x)\in \mathcal{P}_4(\textbf{F})$ tal que $p''(x)\in S$. Ya que $(x-6),(x-6)^2$ genera $S$; existe $a,b\in \textbf{F}$, tal que 
		$$p''(x)=a(x-6)^2+b(x-6).$$
		Integando $p''(x)$ dos veces obtenemos
		$$
		\begin{array}{rcl}
		    p(x)&=&\displaystyle\int\left[\int a(x-6)^2+b(x-6)\; dx\right]\; dx\\\\
			&=& \displaystyle\int \left[a\dfrac{(x-6)^3}{3}+b\dfrac{(x-6)^2}{2}+c\right]\;dx\\\\
			&=& a\dfrac{(x-6)^4}{12}+b\dfrac{(x-6)^3}{6}+cx+d\\\\
			&=& a\dfrac{(x-6)^4}{12}+b\dfrac{(x-6)^3}{6}+c(x-6)+(6c+d)\\\\
		\end{array}
		$$
		Sean los escalares $s_1=a/12$, $s_2=b/6$, $s_3=c$ y $s_4=6c+d$. Entonces,
		$$p(x)=s_1(x-6)^4+s_2(x-6)^3+s_3(x-6)+s_4.$$
		Este polinomio cumple con las condiciones iniciales. Es decir, $p''(x)=12s_1(x-6)^2+6s_2(x-6)$ con $p''(6)=0.$ En otras palabras, $p(x)\in \mathcal{P}_4(\textbf{F})$ tal que $p''(6)=0$ siempre que $p(x)$ puede ser expresado como $s_1(x-6)^4+s_2(x-6)^3+s_3(x-6)+s_4$ para algunos escalares $s_1,s_2,s_3,s_4$. Así,
		$$U=\left\{p(x)\in \mathcal{P}_4(\textbf{F})\,|\, p''(6)=0\right\}=\left\{s_1(x-6)^4+s_2(x-6)^3+s_3(x-6)+s_4\,|\, s_1,s_2,s_3,s_4\in \textbf{F}\right\}.$$
		Observe que este conjunto es un espacio vectorial generado por $\left\{1,(x-6),(x-6)^3,(x-6)^4\right\}$, el cual genera $U$. Luego, sea
		$$s_1(x-6)^4+s_2(x-6)^3+s_3(x-6)+s_4=0$$
		$$\Downarrow$$
		$$s_1x^4+(s_2-24s_1)x^3+(216x_1-18x_2)x^2+(108s_2-864s_1+s_3)x+(129s_1-216s_2-6s_3+s_4)=0$$
		de donde,
		$$
		\begin{array}{rcl}
		    s_1&=&0\\
		    s_2-24s_1&=&0\\
		    216s_1-18s_2&=&0\\
		    108s_2-864s_1+s_3&=&0\\
		    129s_1-216s_2-6s_3+s_4&=&0.
		\end{array}
		\quad \Rightarrow \quad 
		\begin{array}{rcl}
		    s_1&=&0\\
		    s_2&=&0\\
		    s_3&=&0\\
		    s_4&=&0.
		\end{array}
		$$
		Así, $\left\{1,(x-6),(x-6)^3,(x-6)^4\right\}$ es linealemente independiente. Por lo tanto, 
		$$\left\{1,(x-6),(x-6)^3,(x-6)^4\right\}$$ 
		es una base de $U$.\\\\

	    %---------- (b)
	    \item Extienda la base en (a) a una base de $\mathcal{P}_4(\textbf{F})$.\\\\
		Respuesta.-\; 

	    %---------- (c)
	    \item Encuentre un subespacio $W$ de $\mathcal{P}_4(\textbf{F})$ tal que $\mathcal{P}_4(\textbf{F})=U\oplus W$.\\\\
		Respuesta.-\;

	\end{enumerate}


\end{enumerate}
