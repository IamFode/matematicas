\chapter{Espacios vectoriales de dimensión finita}

\mysection{Span e independencia lineal}

\setcounter{mydef}{1}
%%%%%%%%%%%%%%%%%%%%%%%%%%%%%%%%%%%%%%%%% 2.A %%%%%%%%%%%%%%%%%%%%%%%%%%%%%%%%%%%%%%%%%%%%%%%

%-------------------- 2.2 Notación 
\begin{mynot}[Lista of vectores]\;\\\\
    Por lo general, escribiremos listas de vectores sin paréntesis alrededor.
\end{mynot}
\vspace{0.5cm}

\subsection*{Combinaciones lineales y Span}

%-------------------- 2.3 Definición 
\begin{mydef}[Combinación lineal] \;\\\\
    Una \textbf{combinación lineal} de una lista $v_1,\ldots , v_m$ de vectores en $V$ es un vector de la forma 
    $$a_1v_1+\cdots + a_mv_m,$$
    donde $a_1,\ldots a_m \in \textbf{F}.$
\end{mydef}

\setcounter{mydef}{1}
%-------------------- 2.5 Definición
\begin{mydef}[Span] \;\\\\
    El conjunto de todas las combinaciones lineales de una lista de vectores $v_1,\ldots,v_m$ en $V$ se denomina \textbf{span} de $v_1,\ldots,v_m$. En otras palabras,
    $$\span(v_1,\ldots,v_m)=\left\{a_1v_1+\cdots + a_mv_m : a_1,\ldots, a_m \in \textbf{F}\right\}.$$
    El span de la lista vacía $()$ es definida por $\left\{0\right\}.$
\end{mydef}

%-------------------- 2.7 Teorema 
\begin{myteo}[Span es el subespacio más pequeño que lo contiene]
    El \textbf{span} de una lista de vectores en $V$ es el subespacio más pequeño de $V$ que contiene todos los vectores de la lista.\\\\
	Demostración.-\; 
\end{myteo}
