\chapter{Espacios vectoriales de dimensión finita}

\mysection{Span e independencia lineal}

\setcounter{mydef}{1}
%%%%%%%%%%%%%%%%%%%%%%%%%%%%%%%%%%%%%%%%% 2.A %%%%%%%%%%%%%%%%%%%%%%%%%%%%%%%%%%%%%%%%%%%%%%%

%-------------------- 2.2 Notación 
\begin{mynot}[Lista of vectores]\;\\\\
    Por lo general, escribiremos listas de vectores sin paréntesis alrededor.
\end{mynot}
\vspace{0.5cm}

\subsection*{Combinaciones lineales y generadores}

%-------------------- 2.3 Definición 
\begin{mydef}[Combinación lineal] \;\\\\
    Una \textbf{combinación lineal} de una lista $v_1,\ldots , v_m$ de vectores en $V$ es un vector de la forma 
    $$a_1v_1+\cdots + a_mv_m,$$
    donde $a_1,\ldots a_m \in \textbf{F}.$
\end{mydef}

\setcounter{mydef}{4}
%-------------------- 2.5 Definición
\begin{mydef}[Span o generador] \;\\\\
    El conjunto de todas las combinaciones lineales de una lista de vectores $v_1,\ldots,v_m$ en $V$ se denomina \textbf{generador} de $v_1,\ldots,v_m$, denotado por $\span(v_1,\ldots, v_m)$. En otras palabras,
    $$\span(v_1,\ldots,v_m)=\left\{a_1v_1+\cdots + a_mv_m : a_1,\ldots, a_m \in \textbf{F}\right\}.$$
    El span de la lista vacía $()$ es definida por $\left\{0\right\}.$
\end{mydef}

\setcounter{myteo}{6}
%-------------------- 2.7 Teorema 
\begin{myteo}[Span es el subespacio más pequeño que lo contiene]
    El \textbf{span} de una lista de vectores en $V$ es el subespacio más pequeño de $V$ que contiene todos los vectores de la lista.\\\\
	Demostración.-\; Suponga que $v_1,\ldots,v_m$ es una lista de vectores en $V$. Primero demostraremos que $\span(v_1,\ldots,v_m)$ es un subespacio de $V$. El $0$ está en $\span(v_1,\ldots,v_m)$, porque
	$$0=0v_1+\ldots + 0v_m.$$
	También, $\span(v_1,\ldots,v_m)$ es cerrado bajo la suma, ya que
	$$(a_1v_1+\cdots + a_mv_m)+(c_1v_1+\cdots+c_mv_m)=(a_1+c_1)v_1+\cdots + (a_m+c_m)v_m.$$
	Además, $\span(v_1,\ldots,v_m)$ es cerrado bajo la multiplicación por un escalar, dado que
	$$\lambda(a_1v_1+\cdots + a_mv_m)=\lambda a_1v_1+\cdots + \lambda a_mv_m.$$
	Por lo tanto, $\span(v_1,\ldots, v_m)$ es un subespacio de $V$. Esto por 1.34.\\

	Cada $v_j$ es una combinación lineal de $v_1,\ldots,v_m$  (para mostrar esto, establezca $a_j=1$ y que las otras $a'$s en la definición de combinación lineal sean iguales a $0$). Así, el $\span(v_1,\ldots,v_m)$ contiene a cada $v_j$. Por otra parte, debido a que los subespacios están cerrados bajo la multiplicación de escalares y la suma, cada subespacio de $V$ que contiene a cada $v_j$ contiene a $\span(v_1,\ldots,v_m)$. Por lo tanto, $\span(v_1,\ldots,v_m)$ es el subespacio más pequeño de $V$ que contiene todos los demás vectores $v_1,\ldots,v_m$.
\end{myteo}

%-------------------- 2.8 Definición
\begin{mydef}[Spans]\,\\\\
    Si $\span(v_1,\ldots,v_m)$ es igual a $V$, decimos que $v_1,\ldots, v_m$ se extiende sobre $V$.
\end{mydef}

\setcounter{mydef}{9}
%-------------------- 2.10 Definición
\begin{mydef}[Espacio vectorial de dimensión finita]\,\\\\
    Un espacio vectorial se llama finito-dimensional si alguna lista de vectores en él genera el espacio.
\end{mydef}

%-------------------- 2.11 Definición
\begin{mydef}[Polinomio, $\boldmath \mathcal{P}\left(\textbf{F}\right)$]\,\\
    \begin{itemize}
	\item Una función $p:\textbf{F}\to \textbf{F}$ es llamado polinomio con coeficientes en $\textbf{F}$ si existe $a_0,\ldots,a_m\in \textbf{F}$ tal que
	$$p(z)=a_0+a_1z+a_2z^2+\cdots + a_mz^m$$
	para todo $z\in \textbf{F}$.

	\item $\mathcal{P}(\textbf{P})$ es el conjunto de todos los polinomios con coeficientes en $\textbf{F}$.
    \end{itemize}
\end{mydef}

Con las operaciones usuales de adición y multiplicación escalar, $\mathcal{P}(\textbf{F})$ es un espacio vectorial sobre $\textbf{F}$. En otras palabras, $\mathcal{P}(\textbf{F})$ es un subespacio de $\textbf{F}^{\textbf{F}}$, el espacio vectorial de funciones de $\textbf{F}$ en $\textbf{F}$. \\

Los coeficientes de un polinomio están determinados únicamente por el polinomio. Así, la siguiente definición define de manera única el grado de un polinomio.

%-------------------- 2.12 Definición
\begin{mydef}[Grado de un polinomio, $\mathbold deg\; p$]\,\\
    \begin{itemize}
	\item Un polinomio $p\in \mathcal{P}(F)$ se dice que tiene \textbf{grado} $m$ si existen escalares $a_0,a_1,\ldots,a_m\in \textbf{F}$ con $a_m\neq 0$ tal que
	$$p(z)=a_0+a_1z+\cdots+a_mz^m$$
	para todo $z\in \textbf{F}$. Si $p$ tiene grado $m$, escribimos $deg\; p=m$.
	\item El polinomio que es identicamente $0$ se dice que tiene \textbf{grado} $-\infty$.
    \end{itemize}
\end{mydef}

%-------------------- 2.13 Definición
\begin{mydef}[$\boldmath \mathcal{P}_m\left(\textbf{F}\right)$]\;\\\\
    Para $m$ un entero no negativo, $\mathcal{P}_m(\textbf{F})$ denota el conjunto de todos los polinomios con coeficiente en $\textbf{F}$ y grado no mayor a $m$.
\end{mydef}

Verifiquemos el siguiente ejemplo, tenga en cuenta que $\mathcal{P}_m(\textbf{F})=\span\left(1,z,\ldots,z^m\right)$; aquí estamos abusando ligeramente de la notación al permitir que $z^k$ denote una función.

\setcounter{mydef}{14}
%-------------------- 2.15 definición
\begin{mydef}[Espacio vectorial de dimensión infinita]\,\\\\
    Un espacio vectorial se llama \textbf{infinitamente-dimensional} si no es de dimensión finita.
\end{mydef}

%-------------------- 2.16 ejemplo
\begin{myejem}
    Demuestre que $\mathcal{P}(\textbf{F})$ es infinitamente-dimensional.\\\\
    Demostración.-\; Considere cualquier lista de elementos de $\mathcal{P}(\textbf{F})$. Sea $m$ el grado más alto de los polinomios en esta lista. Entonces, cada polinomio en el generador (span) de esta lista tiene grado máximo $m$. Por lo tanto, $z^{m+1}$ no está en el span de nuestra lista. Así, ninguna lista genera $\mathcal{P}(\textbf{F})$. Concluimos que $\mathcal{P}(\textbf{F})$ es de dimensión infinita.
\end{myejem}
\vspace{.5cm}

\subsection*{Independencia lineal}

Suponga $v_1,\ldots,v_m\in V$ y $v\in \span(v_1,\ldots,v_m)$. Por la definición de span, existe $a_1,\ldots,a_m\in \textbf{F}$ tal que
$$v=a_1v_1+\cdots+a_mv_m.$$
Considere la cuestión de si la elección de escalares en la ecuación anterior es única. Sea $c_1,\ldots,c_m$ otro conjunto de escalares tal que
$$v=c_1v_1+\cdots+c_mv_m.$$
Sustrayendo estas últimas ecuaciones, se tiene
$$0=(a_1-c_1)v_1+\cdots+(a_m-c_m)v_m.$$
Así, tenemos que escribir $0$ como una combinación lineal de $(v_1,\ldots,v_m)$. Si la única forma de hacer esto es la forma obvia (usando $0$ para todos los escalares), entonces cada $a_j-c_j$ es igual a $0$, lo que significa que cada $a_j$ es igual a $c_j$ (y por lo tanto la elección de los escalares fue realmente única). Esta situación es tan importante que le damos un nombre especial, independencia lineal, que ahora definiremos.

%-------------------- 2.17 Definición
\begin{mydef}[Linealmente independiente]\,\\
    \begin{itemize}
	\item Una lista $v_1,\ldots,v_m$ de vectores en $V$ se llama linealmente independiente si la única posibilidad de que  $a_1,\ldots,a_m\in \textbf{F}$ tal que $a_1v_1+\cdots+a_mv_m$ sea igual a $0$ es $a_1=\cdots=a_m=0.$
	\item La lista vacía $()$ también se declara linealmente independiente.
    \end{itemize}
\end{mydef}

El razonamiento anterior muestra que $v_1,\ldots,v_m$ es linealmente independiente si y sólo si cada vector en el $\span(v_1,\ldots,v_m)$ tiene sólo una representación lineal en forma de combinación lineal de $v_1,\ldots,v_m$.

\setcounter{mydef}{18}
%-------------------- 2.19 Definición
\begin{mydef}[Linealmente dependiente]\,\\
    \begin{itemize}
	\item Una lista $v_1,\ldots,v_m$ de vectores en $V$ se llama linealmente dependiente si no es linealmente independiente.
	\item En otras palabras, una lista $v_1,\ldots,v_m$ de vectores en $V$ es linealmente dependiente si existe $a_1,\ldots,a_m\in \textbf{F}$, no todos $0$, tal que $a_1v_1+\cdots+a_mv_m=0$.
    \end{itemize}
\end{mydef}

\setcounter{mylema}{20}
%-------------------- 2.21 Lema
\begin{mylema}
    Suponga $v_1,\ldots,v_m$ es una lista linealmente dependiente en $V$. Entonces, existen $j\in\left\{1,2,\ldots,m\right\}$ tal que se cumple lo siguente:
    \begin{enumerate}[(a)]
	\item $v_j\in \span(v_1,\ldots,v_{j-1});$
	\item Si el $j$ésimo término se elimina de $v_1,\ldots,v_m$, el span de la lista restante es igual a $\span(v_1,\ldots,v_m)$.\\
    \end{enumerate}
    Demostración.-\; Ya que la lista $v_1,\ldots,v_m$ es linealmente dependiente, existe números $a_1,\ldots,a_m\in \textbf{F}$, no todos $0$, tal que
    $$a_1v_1+\cdots + a_mv_m=0.$$
    Sea $j$ el elemento más grande de $\left\{1,\ldots, m\right\}$ tal que $a_j\neq 0$. Entonces,
    $$v_j=-\dfrac{a_1}{a_j}v_1-\cdots-\dfrac{a_{j-1}}{a_j}v_{j-1}\qquad (1).$$
    Lo que prueba (a). \\
    Para probar (b), suponga $u\in \span(v_1,\ldots,v_m)$. Entonces, existe números $c_1,\ldots,c_m\in \textbf{F}$ tal que
    $$u=c_1v_1+\cdots+c_mv_m.$$
    En la ecuación anterior, podemos reemplazar $v_j$ con el lado derecho de (1), lo que muestra que $u$ está en el span de la lista obtenida al eliminar el j-ésimo término de $v_1,\ldots,v_m$. Así (b) se cumple.
\end{mylema}

Si elegimos $j=1$ en el lema de dependencia lineal anterior, entonces significa que $v_1=0$, ya que si $j=1$ entonces se interpreta que la condición (a) anterior significa que $v_1\in \span()$. Recuerdo que $\span()=\left\{0\right\}$. Tenga en cuenta también que la demostración del inciso (b) debe modificarse de manera obvia si $v_i=0$ y $j=1$.

\setcounter{myteo}{22}
%-------------------- 2.23 Definición
\begin{myteo}[Longitud de la lista linealmente independiente es  $\boldmath \leq$ a la longitud de la lista que abarca]
    En un espacio vectorial finito, la longitud de cada lista linealmente independiente de vectores es menor o igual que la longitud de cada lista de vectores.\\\\
    Demostración.-\; Suponga $u_1,\ldots,u_m$ es linealmente independiente en $\textbf{V}$. Suponga también que $w_1,\ldots,w_1$ spans $V$. Necesitamos probar que $m\leq n$. Lo hacemos a través del proceso de varios pasos que se describe a continuación; tenga en cuenta que en cada paso agregamos una de las $u's$ y eliminamos una de las $w's$.

    \begin{enumerate}[\small\textbf{Paso} 1.]
	\item Sea $B$ la lista $w_1,\ldots,w_n$, que abarca $V$. Por lo tanto, adjuntar cualquier vector en $V$ a esta lista produce una lista linealmente dependiente (porque el nuevo vector adjunto se puede escribir como una combinación lineal de los otros vectores). En particular, la lista
	$$u_1,w_1m\ldots,w_n$$
	es linealmente dependiente. Así, por el lema (2.21), podemos eliminar un de las $w$ para que la nueva lista $B$ (de longitud $n$) que consta de $u_1$ y las $w$ restantes abarquen $V$.
	\item 
    \end{enumerate}
\end{myteo}


\setcounter{mysection}{0}
\mysection{Ejercicios}

\begin{enumerate}[\bfseries 1.]

    %------------------- 1
    \item Suponga $v_1,v_2,v_3,v_4$ se extiende por $V$. Demostrar que la lista 
    $$v_1-v_2,v_2-v_3,v_3-v_4,v_4$$
    también se extiende por $V$.\\\\
	Demostración.-\; Sea $v\in V$, entonces existe $a_1,a_2,a_3,a_4$ tal que 
	$$v=a_1v_1+a_2v_2+a_3v_3+a_4v_4.$$
	Que implica,
	$$
	\begin{array}{rcl}
	    v&=&a_1v_1+a_2v_2+a_3v_3+a_4v_4-a_1v_2+a_1v_2-a_1v_3+a_1v_3-a_2v_3+a_2v_3-a_1v_4+a_1v_4\\
	     &-& a_2v_4+a_2v_4 -a_3v_4 +a_3v_4\\
	\end{array}
	$$

	De donde,

	$$v=a_1(v_1-v_2)+(a_1+a_2)(v_2-v_3)+(a_1+a_2+a_3)(v_3-v_4)+(a_1+a_2+a_3+a_4)v_4.$$
	Por lo tanto, cualquier vector en $V$ puede ser expresado por una combinación lineal de 
	$$v_1-v_2,v_2-v_3,v_3-v_4,v_4.$$
	Así, esta lista se extiende por $V$.\\\\

    %------------------- 2
    \item 

\end{enumerate}
