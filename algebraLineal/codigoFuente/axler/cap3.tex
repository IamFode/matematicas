\chapter{Transformaciones lineales}

%-------------------- 3.1 notación
\begin{mynot}[\boldmath $F,V,W$]\,\\
    \begin{itemize}
	\item $\textbf{F}$ denota $\textbf{R}$ o $\textbf{C}$.
	\item $V$ y $W$ denota espacios vectoriales sobre $\textbf{F}$.
    \end{itemize}
\end{mynot}
\vspace{.5cm}

\mysection{El espacio vectorial de las Transformaciones lineales}
\vspace{.2cm}

\subsection*{Definición y ejemplos de Transformaciones lineales}

%-------------------- 3.2 definición
\begin{mydef}[Transformación lineal]\;\\\\
	Una \textbf{transformación lineal} de $V$ en $W$ es una función $T:V\to W$ con las siguientes propiedades:

	\begin{itemize}
	    \item \textbf{Aditividad}
		$$T(u+ v)=Tu+ Tv \;\mbox{para todo}\; u,v\in V;$$
	    \item \textbf{Homogeneidad}
		$$T(\lambda v)=\lambda(Tv) \;\mbox{para todo}\; \lambda\in \textbf{F}\; \mbox{y todo}\; v\in v.$$
	\end{itemize}
\end{mydef}

%-------------------- 3.3 notación
\begin{mynot}[\boldmath$\mathcal{L}(V,W)$]\; \\\\
    El conjunto de todas las transformaciones lineales de $V$ en $W$ se denota por $\mathcal{L}(V,W)$.
\end{mynot}

