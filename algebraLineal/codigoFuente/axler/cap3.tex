\chapter{Transformaciones lineales}

%-------------------- 3.1 notación
\begin{mynot}[\boldmath $F,V,W$]\,\\
    \begin{itemize}
	\item $\textbf{F}$ denota $\textbf{R}$ o $\textbf{C}$.
	\item $V$ y $W$ denota espacios vectoriales sobre $\textbf{F}$.
    \end{itemize}
\end{mynot}
\vspace{.5cm}

\mysection{El espacio vectorial de las Transformaciones lineales}
\vspace{.2cm}

\subsection*{Definición y ejemplos de Transformaciones lineales}

%-------------------- 3.2 definición
\begin{mydef}[Transformación lineal]\;\\\\
	Una \textbf{transformación lineal} de $V$ en $W$ es una función $T:V\to W$ con las siguientes propiedades:

	\begin{itemize}
	    \item \textbf{Aditividad}
		$$T(u+ v)=Tu+ Tv \;\mbox{para todo}\; u,v\in V;$$
	    \item \textbf{Homogeneidad}
		$$T(\lambda v)=\lambda(Tv) \;\mbox{para todo}\; \lambda\in \textbf{F}\; \mbox{y todo}\; v\in V.$$
	\end{itemize}
\end{mydef}

%-------------------- 3.3 notación
\begin{mynot}[\boldmath $\mathcal{L}\left(V,W\right)$]\; \\\\
    El conjunto de todas las transformaciones lineales de $V$ en $W$ se denota por $\mathcal{L}(V,W)$.
\end{mynot}

\setcounter{myteo}{4}
%-------------------- 3.5 Teorema
\begin{myteo}[Transformaciones lineales y bases del dominio]\,\\\\
    Suponga que $v_1,\ldots,v_n$ es una base de $V$ y $w_1,\ldots,w_n\in W$. Entonces, existe una única transformación lineal $T:V\to W$ tal que 
    $$T(v_j)=w_j$$
    para cada $j=1,\ldots,n$.\\\\
	Demostración.-\; Primero demostremos la existencia de una transformación lineal $T$, con la propiedad deseada. Defina $T:V\to W$ por
	$$T(c_1v_1+\cdots+c_nv_n)=c_1w_1+\cdots+c_nw_n.$$
	donde $c_1,\ldots,c_n$ son elementos arbitrarios de $\textbf{F}$. La lista $v_1,\ldots,v_n$ es una base de $V$, y por lo tanto, la ecuación anterior de hecho define una función $T$ para $V$ en $W$ (porque cada elemento de $V$ puede ser escrito de manera única en la forma $c_1v_1,\ldots,c_nv_n$). Para cada $j$, tomando $c_j=1$ y las otras $c's$ igual a $0$ demostramos la existencia de $T(v_j)=w_j$.\\
	Si $u,v\in V$ con $u=a_1v_1,\ldots,a_nv_n$ y $v=c_1v_1,\ldots,c_nv_n$, entonces
	$$
	\begin{array}{rcl}
	    T(v+u) &=& T\left[(a_1+c_1)v_1+\ldots+(a_n+a_n)v_n\right]\\\\
		   &=& (a_1+c_1)w_1+\ldots+(a_n+c_n)w_n\\\\
		   &=& (a_1w_1+\ldots+a_nw_n)+(c_1w_1+\ldots+c_nw_n)\\\\
		   &=& T(u)+T(v).
	\end{array}
	$$
	Similarmente, si $\lambda \in \textbf{F}$ y $v=c_1v_1+\cdots+c_nv_n$, entonces
	$$
	\begin{array}{rcl}
	    T(\lambda v) &=& T(\lambda c_1v_1+\cdots+\lambda c_nv_n)\\\\
			 &=&\lambda c_1w_1+\cdots+\lambda c_nw_n\\\\
			 &=&\lambda(c_1w_1+\cdots+c_nw_n)\\\\
			 &=&\lambda T(v).
	\end{array}
	$$
	Así, $T$ es una transformación lineal para $V$ en $W$.\\
	Para probar que es único, suponga que $T\in \mathcal{L}(V,W)$ y que $T(v_j)=w_j$ para $j=1,\ldots,n$. Sea $c_1,\ldots,c_n\in \textbf{F}$. La Homogeneidad de $T$ implica que $T(c_jv_j)=c_jw_j$ para $j=1,\ldots,n$. La Aditividad de $T$ implica que 
	$$T(c_1v_1+\cdots+c_nv_n)=c_1w_1+\cdots+c_nw_n.$$
	Por lo tanto, $T$ se determina de forma única en $\span(v_1,\ldots,v_n)$ para la ecuación de arriba. Porque $v_1,\ldots,v_n$ es una base de $V$, esto implica que $T$ es determinado únicamente en $V$.
\end{myteo}

\vspace{.5cm}

\subsection*{Operaciones algebraicas en \boldmath $\mathcal{L}(V,W)$}

\setcounter{mydef}{5}
%-------------------- 3.6 definición
\begin{mydef}[Adición y multiplicación escalar en \boldmath$\mathcal{L}\left(V,W\right)$]\,\\\\
    Suponga que $S,T\in \mathcal{L}(V,W)$ y $\lambda \in \textbf{F}$. La \textbf{suma} $S+T$ y el \textbf{producto} $\lambda T$ son transformaciones lineales para $V$ en $W$ definida por
    $$(S+T)(v)=S(v)+T(v) \quad \mbox{y} \quad (\lambda T)(v)=\lambda(Tv)$$
    para todo $v\in V$.
\end{mydef}

%-------------------- 3.7 teorema
\begin{myteo}[\boldmath$\mathcal{L}\left(V,W\right)$ es un espacio vectorial]\,\\\\
    Con las Operaciones de adición y multiplicación escalar como se definió, $\mathcal{L}\left(V,W\right)$ es un espacio vectorial.
\end{myteo}

Por lo general, no tiene sentido multiplicar dos elementos de un espacio vectorial, pero para algunos pares de combinaciones lineales existe un producto útil. Necesitaremos un tercer espacio vectorial, así que para el resto de esta sección supongamos que $U$ es un espacio vectorial sobre $\textbf{F}$.

%-------------------- 3.8 definición
\begin{mydef}[Producto de combinaciones lineales]\,\\\\
    Si $T\in \mathcal{L}(U,V)$ y $S\in \mathcal{L}(V,W)$, entonces el producto $ST\in \mathcal{L}(U,W)$ es definido por
    $$(ST)(u)=S(Tu)$$
    para $u\in U$.
\end{mydef}

En otras palabras, $ST$ es solo la composición habitual $S\circ T$ de dos funciones, pero cuando ambas funciones son lineales, la mayoría de los matemáticos escriben $ST$ en lugar de $S\circ T$. Debe verificar que $ST$ es de hecho una transformación lineal de $U$ a $W$ siempre que $T\in \mathcal{L}(U,V)$ y $S\in \mathcal{L}(V,W)$. Tenga en cuenta que $ST$ se define solo cuando $T$ se transforma en el dominio de $S$.

%-------------------- 3.9 teorema
\begin{myteo}[Propiedades algebraicas de producto de transformaciones lineales]\,\\\\
    \textbf{Asociatividad}
    $$(T_1T_2)T_3 = T_1(T_2T_3)$$
    siempre que $T_1$, $T_2$ y $T_3$ sean transformaciones lineales tales que los productos tengan sentido (lo que significa que $T_3$ se transforma en el dominio de $T_2$, y $T_2$ se transfora en el dominio de $T_1$).\\

    \textbf{Identidad}
    $$TI=IT=T$$
    siempre que $T\in \mathcal{L}(V,W)$ (el primer $I$ es la transformación de indentidad en $V$, y el segundo $I$ es la transformación de identidad en $W$).\\

    \textbf{Propiedades distributivas}
    \begin{center}
	$(S_1+S_2)T=S_1T+S_2T\quad$ y $\quad S(T_1+T_2)=ST_1+ST_2$
    \end{center}
    siempre que $T,T_1,T_2\in \mathcal{L}(U,V)$ y $S,S_1,S_2\in \mathcal{L}(V,W)$.
\end{myteo}

La multiplicación de aplicaciones lineales no es conmutativa. En otras palabras, no es necesariamente cierto que $ST=TS$, incluso si ambos lados de la ecuación tienen sentido.

%-------------------- 3.10 Ejemplo
\begin{myejem}
    Suponga $D\in \mathcal{L}\left[\mathcal{P}(\textbf{R}),\mathcal{P}(\textbf{R})\right]$ es la transformación de diferenciación definido en el ejemplo 3.4 y $T\in \mathcal{L}\left[\mathcal{P}(\textbf{R}),\mathcal{P}(\textbf{R})\right]$ es la multiplicación por la transformación $x^2$ definida tempranamente en esta sección. Muestre que $TD\neq DT$.\\\\
	Demostración.-\;  Se tiene 
	$$\left[(TD)p\right](x)=x^2p'(x) \quad \mbox{pero}\quad \left[(DT)p\right](x)=x^2p'(x)+2xp(x).$$
	En otras palabras, no es lo mismo derivar y luego multiplicar por  $x^2$ que multiplicar por $x^2$ y luego derivar.
\end{myejem}

%-------------------- 3.11 Teorema
\begin{myteo}[Transformaciones lineales toman \boldmath$0$ a $0$]\,\\\\
    Suponga $T$ es una transformación lineal para $V$ en $W$. Entonces $T(0)=0$.\\\\
	Demostración.-\; Por la aditividad, se tiene
	$$T(0)=T(0+0)=T(0)+T(0).$$
	Agregue el inverso aditivo de $T(0)$ cada lado de la ecuación anterior para concluir que $T(0)=0$.
\end{myteo}


\section*{Ejercicios 3.A}

\begin{enumerate}[\bfseries 1.]

    %--------------------1.
    \item Suponga $b,c\in \textbf{R}$. Defina $T:\textbf{R}^3\to \textbf{R}^2$ por
    $$T(x,y,z)=(2x-4y+3z+b,6x+cxyz).$$
    Demuestre que $T$ es lineal si y sólo si $b=c=0$.\\\\
	Demostración.-\; Por definición de transformación lineal, tendremos que demostrar que se cumple las propiedades de aditividad y homogeneidad.\\

	\textbf{Aditividad.- } Supongamos $(x_1,y_1,z_1),(x_2,y_2,z_2)\in \textbf{R}^3$. Entonces, por la definición de $T$ se tiene
	$$
	\begin{array}{rcl}
	    T\left[(x_1,y_1,z_1)+(x_2,y_2,z_2)\right] &=& T(x_1+x_2,y_1+y_2,z_1+z_2)\\\\
						      &=& \left[2(x_1+x_2)-4(y_1+y_2)+3(z_1+z_2)+b,\right.\\
						      && \left.6(x_1+x_2)+c(x_1+x_2)(y_1+y_2)(z_1+z_2)\right]\\\\
						      &=& (2x_1-4y_1+3z_1+b,6x_1+cx_1y_1z_1)\\
						      &+&(2x_2-4y_2+3z_2+b,6x_2+cx_2y_2z_2)\\\\
						      &=& T(x_1,y_1,z_1)+T(x_2,y_2,z_2).
	\end{array}
	$$
	Esto se cumplirá si $2b=b\Leftrightarrow b=0\;$ y $\;cx_1y_1z_1+cx_1y_2z_1+cx_2y_1z_1+cx_2y_2z_1+cx_1y_1z_2+cx_1y_2z_2+cx_2y_1z_2+cx_2y_2z_2 = cx_1y_1z_1+cx_2y_2z_2 \Leftrightarrow c=0$.\\

	\textbf{Homogeneidad.- } Supongamos $(x,y,z)\in \textbf{R}^3$ y $\lambda \in \textbf{R}$. Entonces, por la definición de $T$ se tiene
	$$
	\begin{array}{rcl}
	    T\left[\lambda(x,y,z)\right] &=& (2\lambda x - 4\lambda y+3\lambda z + \lambda b, 6\lambda x + c\lambda xyz)\\\\
					 &=& \lambda(2x-4y+3z+b,6x+cxyz)\\\\
					 &=& \lambda T\left[(x,y,z)\right].
	\end{array}
	$$

	Notemos que la homogeneidad se cumplirá para todo $b,c\in \textbf{R}$. Por lo tanto, $T$ es lineal si y sólo si $b=c=0$.\\\\

    %--------------------2.
    \item Suponga $b,c\in \textbf{R}$. Defina $T:\mathcal{P}(\textbf{R})\to \textbf{R}^2$  por
    $$Tp=\left[3p(4)+5p'(6)+bp(1)p(2),\int_{-1}^2 x^3p(x)\; dx +c\sen p(0)\right].$$
    Demuestre que $T$ es una transformación lineal si y sólo si $b=c=0$.\\\\
	Demostración.-\; Primero demostraremos que si $T$ es una transformación lineal, entonces $b=c=0$. Por definición para cada $\alpha\in \mathbb{R}$.
	$$\alpha Tp = T(\alpha p)$$
	De esta manera se tiene
	$$
	\begin{array}{rcl}
	    \alpha T p &=& \alpha \left[3p(4)+5p'(6)+bp(1)p(2),\displaystyle\int_{-1}^2 x^3p(x)\; dx +c\sen p(0)\right] \\\\
		       &=& \left[3\alpha p(4)+5\alpha p'(6)+b\alpha p(1)p(2),\alpha \displaystyle\int_{-1}^2 x^3 p(x) \; dx+c\alpha\sen p(0)\right]\quad (1)\\\\
		       T(\alpha p)&=& \left[3(\alpha p)(4) + 5(\alpha p)'(6) + b(\alpha p)(1)(\alpha p)(2),\displaystyle\int_{-1}^2 x^3(\alpha p)(x)\; dx + c\sen (\alpha p)(0)\right].
	\end{array}
	$$

	Es decir, si $p(x)=a_0+a_1x+a_2x^2+\cdots + a_nx^n$, entonces 
	$$(\alpha p)(x)=\left(\alpha a_0\right)+\left(\alpha a_1\right)x+\left(\alpha a_2\right)x^2+\cdots + \left(\alpha a_n\right)x^n.$$

	Por lo tanto, para cada $x$,

	$$
	\begin{array}{rcl}
	    (\alpha p)(x) &=& \left(\alpha a_0\right)+\left(\alpha a_1\right)x+\left(\alpha a_2\right)x^2+\cdots + \left(\alpha a_n\right)x^n\\\\
	    &=& \alpha\left(a_0+a_1x+a_2x^2+\cdots + a_nx^n\right)\\\\
	    &=& \alpha p(x).
	\end{array}
	$$

	Por esta deducción podemos decir que,
	$$
	\begin{array}{rcl}
	    T(\alpha p) &=& \left[3\alpha p(4)+5\alpha p'(6)+b\alpha^2p(1)p(2),\displaystyle\int_{-1}^2 \alpha x^3 p(x)\; dx + c\sen\left(\alpha p(0)\right)\right]\\\\
			&=& \left[\alpha\left(3p(4)+5p'(6)+b\alpha p(1)p(2)\right),\alpha\displaystyle\int_{-1}^2 x^3 p(x)\; dx + c\alpha \sen p(0)\right]\quad (2)\\\\
	\end{array}
	$$

	Ahora, podemos igualar (1) y (2) si y sólo si 
	$$b\alpha^2p(1)p(2)=b\alpha P(1) p(1) \quad \mbox{y} \quad c\sen(\alpha p(0))=c\alpha\sen p(0)$$
	para cada $p(x)$ y cada $\alpha \in \textbf{R}.$\\

	Sean $p(x)=x+\dfrac{\pi}{2}$ y $\alpha=2$. De donde,
	$$b\alpha^2p(1)p(2)=4b\left(1+\dfrac{\pi}{2}\right)\left(2+\dfrac{\pi}{2}\right)\quad \mbox{y}\quad b\alpha p(0)p(1)=2b\left(1+\dfrac{\pi}{2}\right)\left(2+\dfrac{\pi}{2}\right)$$
	Entonces,
	$$
	\begin{array}{rcl}
	    4b\left(1+\dfrac{\pi}{2}\right)\left(2+\dfrac{\pi}{2}\right)=2b\left(1+\dfrac{\pi}{2}\right)\left(2+\dfrac{\pi}{2}\right) &\Leftrightarrow& 4b\left(1+\dfrac{\pi}{2}\right)\left(2+\dfrac{\pi}{2}\right)-2b\left(1+\dfrac{\pi}{2}\right)\left(2+\dfrac{\pi}{2}\right)=0\\\\
																      &\Leftrightarrow& 2b \left(1+\dfrac{\pi}{2}\right)\left(2+\dfrac{\pi}{2}\right)=0\\\\
																      &\Leftrightarrow& b=0.
	\end{array}
	$$

	Por otro lado,

	$$0= c\sen\left(2\cdot \dfrac{\pi}{2}\right)=c\sin\left(\alpha p(0)\right) = c\alpha \sen p(0)=2c\sen\left(\dfrac{\pi}{2}\right)=2c$$

	Por lo tanto, 

	$$c=0.$$

	Ahora, demostremos que si $b=c=0$, entonces $T$ es lineal; es decir, que $T$ es aditiva y Homogénea. Supongamos $p,q\in \mathcal{P}(\textbf{R})$ y $\lambda \in \textbf{R}$, por lo que 

	$$
	\begin{array}{rcl}
	    T(p+q) &=& \left[3(p+q)(4)+5(p+q)'(6),\displaystyle\int_{-1}^2 x^3(p+q)(x)\; dx\right]\\\\
		   &=& \left[3\left(p(4)+q(4)\right)+5\left(p'(6)+q'(6)\right),\displaystyle\int_{-1}^2x^3\left(p(x)+q(x)\right)\; dx\right]\\\\
		   &=& \left[3p(4)+5p'(6)+3q(4)+3q'(6),\displaystyle\int_{-1}^2 x^3 p(x)\; dx +\int_{-1}^2 x^3q(x)\; dx\right]\\\\
		   &=& \left[3p(4)+5p'(6),\displaystyle\int_{-1}^2 p(x)\; dx\right]+\left[3q(4)+3q'(6),\displaystyle\int_{-1}^2 x^3q(x)\; dx\right]\\\\
		   &=& Tp+Tq.
	\end{array}
	$$

	\begin{center}
	y 
	\end{center}

	$$
	\begin{array}{rcl}
	    T(\lambda p) &=& \left[3(\lambda p)(4)+5(\lambda p)'(6),\displaystyle\int_{-1}^2 x^3(\lambda p)(x)\; dx\right]\\\\
			 &=& \left[3\lambda p(4)+5\lambda p'(6),\lambda\displaystyle\int_{-1}^2 x^3 p(x)\; dx\right]\\\\
			 &=& \left[\lambda(3p(4)+5p(6)),\lambda \displaystyle\int_{-1}^2 x^3p(x)\; dx\right]\\\\
			 &=& \lambda\left[3p(4)+5p'(6),\displaystyle\int_{-1}^2 x^3 p(x)\; dx\right]\\\\
			 &=& \lambda Tp.
	\end{array}
	$$

	Por lo tanto, $T$ es lineal. Así, concluimos que $T$ es lineal si y sólo si $b=c=0$.\\\\


    %--------------------3.
    \item Suponga $T\in \mathcal{L}\left(\textbf{F}^n,\textbf{F}^m\right)$. Demostrar que existe escalares $A_{j,k}\in \textbf{F}$ para $j=1,\ldots,m$ y $k=1,\ldots,n$ tal que
    $$T\left(x_1,\ldots,x_n\right)\in \textbf{F}^n.$$
    [El ejercicio demuestra que $T$ tiene la forma prometida en el úmtimo apartado del ejemplo 3.4.]\\\\

    %--------------------4.
    \item Suponga $T\in \mathcal{L}(V,W)$ y $v_1,\ldots, v_m$ una lista de vectores en $V$ tal que $Tv_1,\ldots,Tv_m$ es una lista linealmente independiente en $W$. Demostrar que $v_1,\ldots,v_m$ es linealmente independiente.\\\\
	Demostración.-\; Sea para $c_i\in \textbf{F}$ tal que 
	$$c_1v_1+c_2v_2+\cdots+c_nv_n=0.$$
	Luego, multiplicamos por $T$ a ambos lados de la ecuación anterior,
	$$T(c_1v_1+c_2v_2+\cdots+c_nv_n)=T(0).$$
	Por la definición 1.6 y el torema 3.11 tenemos que
	$$c_1Tv_1+c_2Tv_2+\cdots+c_nTv_n=0.$$
	Entonces, como $Tv_1,\ldots,Tv_m$ es una lista linealmente independiente en $W$.\\\\

	%--------------------5.
	\item  Demostrar la afirmación 3.7.\\\\
	    Demostración.-\; Verificaremos cada propiedad.

	    \begin{itemize}

		\item Conmutatividad.- Sean $S,T\in \mathcal{L}(V,W)$ y $v\in V$, tenemos
		    $$(S+T)(v)=S(v)+T(v)=T(v)+S(v)=(T+S)(v).$$
		    Por lo tanto, la adición es comunutativa.\\

		\item Asociatividad.- Saen $R,S,T\in \mathcal{L}(V,W)$ y $v\in V$, tenemos
		    $$
		    \begin{array}{rcl}
			\left[(R+S)+T\right](v) &=& (R+S)(v)+T(v)\\\\
						&=& R(v)+S(v)+T(v)\\\\
						&=& R(v)+\left[S(v)+T(v)\right]\\\\
		    \end{array}
		    $$
		    Por lo que la adición es asociativa. Luego, sean $a,b\in \textbf{F}$, entonces
		    $$\left[(ab)T\right](v)=(ab)T(v)=a\left[bT(v)\right]=\left[a(bT)\right](v).$$
		    Por lo tanto, la multiplicación es asociativa.\\

		\item Identidad aditiva.- Sea $0\in \mathcal{L}(V,W)$ denotado como transformación cero, sean también $T\in \mathcal{L}(V,W)$ y $v\in V$. Entonces,
		    $$(T+0)(v)=T(v)+0(v)=T(v).$$
		    Por lo tanto, la transformación cero es la identidad aditiva.\\

		\item Inverso aditivo.- Sean $T\in \mathcal{L}(V,W)$ y $v\in V$, y definamos a $(-T)\in \mathcal{L}(V,W)$ por $(-T)(v)=-T(v)$, entonces
		    $$\left[T+(-T)\right](v)=T(v)+(-T)(v)=T(v)-T(v)=0,$$
		    Por lo que, $(-T)$ es el inverso aditivo para cada $T\in \mathcal{L}(V,W)$.\\

		\item Identidad multiplicativa.- Sea $T\in \mathcal{L}(V,W)$. Entonces,
		    $$(1T)(v)=1(T(v))=T(v).$$
		    Así la identidad multiplicativa de $\textbf{F}$ es la identidad multiplicativa de la multiplicación escalar.\\

		\item Propiedad distributiva.- Sean $S,T\in \mathcal{L}(V,W)$, $a,b\in \textbf{F}$ y $v\in V$. Entonces,
		    $$
		    \begin{array}{rcl}
			\left[a(S+T)\right](v) &=& a(Sv+Tv)\\\\
					       &=& aS(v)+aT(v)\\\\
					       &=& (aS)(v)+(aT)(v)\\\\
		    \end{array}
		    $$
		    \begin{center}
			y
		    \end{center}

		    $$
		    \begin{array}{rcl}
			\left[(a+b)T\right](v) &=& (a+b)T(v)\\\\
					       &=& aT(v)+bT(v)\\\\
					       &=& (aT)(v)+(bT)(v).\\\\
		    \end{array}
		    $$
		    
	    \end{itemize}

	    Por lo tanto, $\mathcal{L}(V,W)$ es un espacio vectorial.\\\\

	%--------------------6.
	\item  Demostrar la afirmación 3.9.\\\\
	    Demostración.-\; 
	    \begin{itemize}
		\item Asociatividad.- Para $x$ en el dominio de $T_3$, tenemos
		    $$
		    \begin{array}{rcl}
			\left[(T_1T_2)T_3\right](x) &=& (T_1T_2)\left[T_3(x)\right]\\\\
						    &=& T_1\left[T_2\left[T_3(x)\right]\right]\\\\
						    &=& T_1\left[(T_2T_3)(x)\right]\\\\
						    &=& \left[T_1(T_2T_3)\right](x).
		    \end{array}
		    $$

		\item Identidad.- Para $v\in V$, se tiene
		    $$
		    \begin{array}{rcl}
			TI(v) &=& T\left[I(v)\right]\\\\
					     &=& T(v)\\\\
					     &=& I\left[T(v)\right]\\\\
					     &=& IT(v).\\\\
		    \end{array}
		    $$
		    Por lo tanto, $TI=IT=I$.\\

		\item Proipedad distributiva.- Para $u\in U$, se tiene
		    $$
		    \begin{array}{rcl}
			\left[(S_1+S_2)T\right](u) &=& (S_1+S_2)\left[T(u)\right]\\\\
						   &=& S_1\left[T(u)\right]+S_2\left[T(u)\right]\\\\
						   &=& S_1T(u)+S_2T(u)\\\\
						   &=& (S_1T+S_2T)(u).\\\\
		    \end{array}
		    $$

		    \begin{center}
			y
		    \end{center}

		    $$
		    \begin{array}{rcl}
			\left[T(S_1+S_2)\right](u) &=& T\left[(S_1+S_2)(u)\right]\\\\
						   &=& T\left[S_1(u)+S_2(u)\right]\\\\
						   &=& T\left[S_1(u)\right]+T\left[S_2(u)\right]\\\\
						   &=& (TS_1+TS_2)(u).\\\\
		    \end{array}
		    $$
		    \vspace{.5cm}


	    \end{itemize}

    %--------------------7.
    \item Demostrar que toda transformación lineal de un espacio vectorial unidimensional a sí mismo es multiplicación por un escalar. Más precisamente demostrar que si $\dim V = 1$ y $T\in\mathcal{L}(V,V)$, entonces existe $\lambda \in \textbf{F}$ tal que $Tv=\lambda v$ para todo $v\in V$.\\\\
	Demostración.-\; Supongamos que $\dim V=1$ y $T\in \mathcal{L}(V,V)$. Sea también $u\in V,u\neq 0$. Entonces, todo vector en $V$ es un multiplo escalar de $u$. En particular, $Tu=au$ para algún $a\in \textbf{F}$.\\
	Ahora, consideremos un vector $v\in V$. Ya que, existe $b\in \textbf{F}$ tal que $v=bu$. Entonces,
	$$
	\begin{array}{rcl}
	    Tv&=&T(bu)\\
	      &=& bT(u)\\
	      &=&b(au)\\
	      &=&a(bu)\\
	      &=&av.
	\end{array}
	$$
	\vspace{.5cm}

    %--------------------8.
    \item Dar un ejemplo de una función $\varphi:\textbf{R}^2\to \textbf{R}$ tal que
    $$\varphi(av)=a\varphi(v)$$
    para todo $a\in \textbf{R}$ y para todo $v\in \textbf{R}^2$ pero $\varphi$ no es lineal.\\
    $\left[\right.$\textit{El ejercicio anterior y el siguiente ejercicio muestran que ni la homogeneidad ni la aditividad por sí solas son suficientes para implicar que una función es una transformación lineal.}$\left.\right]$\\\\
	Respuesta.-\; Si $v=(x,y)\in \textbf{R}^2$, entonces definamos $\varphi:\textbf{R}^2\to \textbf{R}$ como,
	$$\varphi(v)=\varphi\left[(x,y)\right]=(x^3+y^3)^{1/3}.$$
	Calculemos ahora $\varphi(av)$;
	$$
	\begin{array}{rcl}
	    \varphi(av) &=& \varphi\left[a(x,y)\right]\\\\
			&=& \varphi\left[(ax,ay)\right]\\\\
			&=& \left[(ax)^3+(ay)^3\right]^{1/3}\\\\
			&=& \left[(a^3x^3)+(a^3y^3)\right]^{1/3}\\\\
			&=& a\left(x^3+y^3\right)^{1/3}\\\\
			&=& a\varphi\left(v\right)\\\\
	\end{array}
	$$
	Ahora, verifiquemos que $\varphi$ no es lineal. Para ello, consideremos $v_1=(1,0)$ y $v_2=(0,1)$, entonces
	$$
	\begin{array}{rcl}
	    \varphi(v_1)&=&\varphi\left[(1,0)\right]=1\\\\
	    \varphi(v_2)&=&\varphi\left[(0,1)\right]=1\\\\
	\end{array}
	$$
	De donde,
	$$\varphi(v_1)+\varphi(v_2)=1+1=2.$$
	Por otro lado,
	$$
	\begin{array}{rcl}
	    \varphi\left(v_1+v_2\right) &=& \varphi\left[(1,0)+(0,1)\right]\\\\
		    &=& \varphi\left[(1,1)\right]\\\\
		    &=& \left[(1)^3+(1)^3\right]^{1/3}\\\\
		    &=& 2^{1/3}.\\\\
	\end{array}
	$$
	Ya que, 
	$$\varphi\left(v_1+v_2\right)\neq \varphi\left(v_1\right)+\varphi\left(v_2\right) \qquad \mbox{(prop. de aditividad)}$$
	entonces $\varphi$ no es lineal.\\\\


    %--------------------9.
    \item Dar un ejemplo de una función $\varphi:\textbf{C}\to \textbf{C}$ tal que
    $$\varphi(w+z)=\varphi(w)+\varphi(z)$$
    para todo $w,z\in \textbf{C}$ pero $\varphi$ no es lineal. (Aquí \textbf{C} se considera como un espacio vectorial complejo.\\
    $\left[\right.$\textit{También existe una función} $\varphi:\textbf{R}\to \textbf{R}$  \textit{tal que} $\varphi$ \textit{satisface la condición de aditividad anterior pero} $\varphi$ \textit{no es lineal. Sin embargo, mostrar la existencia de tal función implica herramientas considerablemente más avanzadas.}$\left.\right]$\\\\
    Respuesta.-\; Sean $x_1+y_1i,x_2+y_2i\in \textbf{C}$ definida por, $\varphi:\textbf{C}\to \textbf{C}$ tal que
    $$\varphi\left[(x+yi)\right]=x-yi.$$
    Entonces,
    $$
    \begin{array}{rcl}
	\varphi\left[(x_1+y_1i)+(x_2+y_2i)\right] &=& \varphi\left[(x_1+x_2)+(y_1+y_2)i\right]\\\\
						  &=& (x_1+x_2)-(y_1+y_2)i\\\\
						  &=& (x_1-y_1i)+(x_2-y_2i)\\\\
						  &=& \varphi\left[(x_1+y_1i)\right]+\varphi\left[(x_2+y_2i)\right].
    \end{array}
    $$

    Esto demuestra la aditividad de $\varphi$. Sin embargo, sean $v=1+2i\in \textbf{C}$ y $a=i\in \textbf{C}$. Entonces,
    $$\varphi(av)=\varphi\left[i(\cdot 1+2i)\right]=\varphi\left(i+2i^2\right)=\varphi(-2+1)=-(2+i)$$
    \begin{center}
    y
    \end{center}
    $$a\varphi(v)=i\cdot \varphi(1+2i)=i(1+2i)=i-2.$$
    Esto demuestra que 
    $$\varphi(av)\neq a\varphi(v)$$
    no es lineal.\\\\


    %--------------------10.
    \item Suponga que $U$ es un subespacio de $V$ con $U\neq V$. Suponga también que $S\in \mathcal{L}(U,W)$ y $S\neq 0$ (lo que significa que $Su\neq 0$ para algún $u\in U$). Defina $T:V\to W$ por
    $$
    Tv = 
    \left\{
	\begin{array}{rcl}
	    Sv &\mbox{si} & v\in U\\\\
	    0 &\mbox{si} & v\in V \;\mbox{ y }\;v\notin U.
	\end{array}
    \right.
    $$
    Demostrar que $T$ no es una transformación lineal en $V$.\\\\
	Demostración.-\; Notemos que $U\neq V$. Luego, podemos elegir $u\in U$ tales que $Su\neq 0$ y $v\in V$, entonces $u+v\in U$. De lo contrario,
	$$v=(u+v)-u\in U$$
	producirá una contradicción. Por eso $T(u+v)=0$ por definición. Por otro lado, $Tu+Tv=SU\neq 0$. Resulta que $T(u+v)\neq Tu+Tv$, por eso $T$ no es una transformación lineal en $V$.\\\\

    %--------------------11.
    \item Suponga que $V$ es de dimensión finita. Demostrar que cada transformación lineal en una subespacio de $V$ puede ser extendida a una transformación lineal en $V$. En otras palabras, demuestra que si $U$ es un subespacio de $V$ y $S\in \mathcal{L}(U,W)$, entonces existe $T\in \mathcal{L}(V,W)$ tal que $Tu=Su$ para todo $u\in U$.\\\\
	Demostración.-\; Sea $v_1,\ldots,v_m$ una base de $U$, entonces por el teorema 2.33 podemos extenderlo a una base de $V$. Esto es, $v_1,\ldots,v_m,v_{m+1}.\ldots,v_n$ es una extensión base de $V$ tal que $v_{m+1},\ldots,v_n$ linealmente independiente.\\
	Para cualquier $z\in V$, existe $a_1,\ldots,a_n\in \textbf{F}$ tal que $z=\sum_{k=1}^n a_kv_k$ definida por

	$$
	\begin{array}{rcrcl}
	    T&:&V&\to & W\\\\
	     &&\displaystyle\sum_{k=1}^n a_kv_k &\mapsto & \displaystyle\sum_{k=1}^m a_kSv_k+\displaystyle\sum_{k=m+1}^n a_kv_k.
	\end{array}
	$$

	Dado que cada $v\in V$ tiene una única representación como una combinación lineal de elementos de nuestra base, la transformación está definida. Primero, demostremos que $T$ existe y  es una transformación lineal. Suponga $z_1,z_2\in V$. Luego, sean $a_1,\ldots,a_n\in \textbf{F}$ y $b_{1},\ldots,b_{n}\in \textbf{F}$ tal que
	$$z_1=a_1v_1+\cdots+a_nv_n \quad \mbox{y}\quad z_2=b_{1}v_{1}+\cdots+b_nv_{n}$$
	se sigue que,
	$$
	\begin{array}{rcl}
	    T(z_1+z_2) &=& T\left(\displaystyle\sum_{k=1}^n a_kv_k+\sum_{1}^n b_kv_k\right)\\\\
		       &=& T\left[\displaystyle\sum_{k=1}^n (a_k+b_k)v_k\right]\\\\
		       &=& \displaystyle\sum_{k=1}^n (a_k+b_k)Sv_k+\sum_{k=m+1}^n (a_k+b_k)v_k\\\\
		       &=& \displaystyle\sum_{k=1}^n a_kSv_k+\sum_{k=m+1}^n a_kv_k+\sum_{k=1}^n b_kSv_k+\sum_{k=m+1}^n b_kv_k\\\\
		       &=& T\left(\displaystyle\sum_{k=1}^n a_kv_k\right)+T\left(\displaystyle\sum_{k=1}^n b_kv_k\right)\\\\
		       &=& T(z_1)+T(z_2).
	\end{array}
	$$
	Lo que demuestra que $T$ es aditiva. Ahora, sean $\sum_{i=1}^n a_kv_k=z\in V$ y $\lambda\in \textbf{F}$, entonces para  $a_1,\ldots,a_n \in \textbf{F}$, se tiene
	$$
	\begin{array}{rcl}
	    T(\lambda z) &=& T\left(\lambda\displaystyle\sum_{k=1}^n a_kv_k\right)\\\\
			 &=& T\left[\displaystyle\sum_{k=1}^n (\lambda a_k)v_k\right]\\\\
			 &=& S\left[\displaystyle\sum_{k=1}^m (\lambda a_k)v_k\right]+\displaystyle\sum_{k=m+1}^n (\lambda a_k)v_k\\\\
			 &=& \lambda S\left[\displaystyle\sum_{k=1}^m a_kv_k\right]+\lambda\displaystyle\sum_{k=m+1}^n a_kv_k\\\\
			 &=& \lambda \left[S\left(\displaystyle\sum_{k=1}^m a_kv_k\right)+\displaystyle\sum_{k=m+1}^n a_kv_k\right]\\\\
			 &=& \lambda T\left(\displaystyle\sum_{k=1}^n a_kv_k\right)=\lambda Tz\\\\
	\end{array}
	$$
	por lo que $T$ es homogénea. Por lo tanto, $T\in \mathcal{L}(V,W)$. Por último, para ver si $T=S$, sea $u\in U$ con $a_1,\ldots,a_m\in \textbf{F}$ tal que $u=\sum_{k=1}^m a_kv_k$ se tiene
	$$Tu=T\left(\displaystyle\sum_{k=1}^m a_kv_k\right)=\displaystyle\sum_{k=1}^m a_kSv_k+\displaystyle\sum_{k=m+1}^n a_kv_k=S\left(\displaystyle\sum_{k=1}^m a_kv_k\right)=Su.$$
	Notemos que $\sum_{k=m+1}^n a_kv_k$ es linealmente independiente. Así, completamos la demostración.\\\\

    %--------------------12.
    \item Suponga que $V$ es de dimensión finita con $\dim(V)>0$, y suponga $W$ es de dimensión infinita. Demostrar que $\mathcal{L}(V,W)$ es de dimensión infinita.\\\\
	Demostración.-\; 

\end{enumerate}
