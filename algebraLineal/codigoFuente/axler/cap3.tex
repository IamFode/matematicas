\chapter{Transformaciones lineales}

%-------------------- 3.1 notación
\begin{mynot}[\boldmath $F,V,W$]\,\\
    \begin{itemize}
	\item $\textbf{F}$ denota $\textbf{R}$ o $\textbf{C}$.
	\item $V$ y $W$ denota espacios vectoriales sobre $\textbf{F}$.
    \end{itemize}
\end{mynot}
\vspace{.5cm}

\mysection{El espacio vectorial de las Transformaciones lineales}
\vspace{.2cm}

\subsection*{Definición y ejemplos de Transformaciones lineales}

%-------------------- 3.2 definición
\begin{mydef}[Transformación lineal]\;\\\\
	Una \textbf{transformación lineal} de $V$ en $W$ es una función $T:V\to W$ con las siguientes propiedades:

	\begin{itemize}
	    \item \textbf{Aditividad}
		$$T(u+ v)=Tu+ Tv \;\mbox{para todo}\; u,v\in V;$$
	    \item \textbf{Homogeneidad}
		$$T(\lambda v)=\lambda(Tv) \;\mbox{para todo}\; \lambda\in \textbf{F}\; \mbox{y todo}\; v\in v.$$
	\end{itemize}
\end{mydef}

%-------------------- 3.3 notación
\begin{mynot}[\boldmath$\mathcal{L}(V,W)$]\; \\\\
    El conjunto de todas las transformaciones lineales de $V$ en $W$ se denota por $\mathcal{L}(V,W)$.
\end{mynot}

\setcounter{myteo}{3}
%-------------------- 3.5 Teorema
\begin{myteo}[Transformaciones lineales y bases del dominio]\,\\\\
    Suponga que $v_1,\ldots,v_n$ es una base de $V$ y $w_1,\ldots,w_n\in W$. Entonces, existe una única transformación lineal $T:V\to W$ tal que 
    $$T(v_j)=w_j$$
    para cada $j=1,\ldots,n$.\\\\
	Demostración.-\; Primero demostremos la existencia de una transformación lineal $T$, con la propiedad deseada. Defina $T:V\to W$ por
	$$T(c_1v_1+\cdots+c_nv_n)=c_1w_1+\cdots+c_nw_n.$$
	donde $c_1,\ldots,c_n$ son elementos arbitrarios de $\textbf{F}$. La lista $v_1,\ldots,v_n$ es una base de $V$, y por lo tanto, la ecuación anterior de hecho define una función $T$ para $V$ en $W$ (porque cada elemento de $V$ puede ser escrito de manera única en la forma $c_1v_1,\ldots,c_nv_n$). Para cada $j$, tomando $c_j=1$ y las otras $c's$ igual a $0$ demostramos la existencia de $T(v_j)=w_j$.\\
	Si $u,v\in V$ con $u=a_1v_1,\ldots,a_nv_n$ y $v=c_1v_1,\ldots,c_nv_n$, entonces
	$$
	\begin{array}{rcl}
	    T(v+u) &=& T\left[(a_1+c_1)v_1+\ldots+(a_n+a_n)v_n\right]\\\\
		   &=& (a_1+c_1)w_1+\ldots+(a_n+c_n)w_n\\\\
		   &=& (a_1w_1+\ldots+a_nw_n)+(c_1w_1+\ldots+c_nw_n)\\\\
		   &=& T(u)+T(v).
	\end{array}
	$$
	Similarmente, si $\lambda \in \textbf{F}$ y $v=c_1v_1+\cdots+c_nv_n$, entonces
	$$
	\begin{array}{rcl}
	    T(\lambda v) &=& T(\lambda c_1v_1+\cdots+\lambda c_nv_n)\\\\
			 &=&\lambda c_1w_1+\cdots+\lambda c_nw_n\\\\
			 &=&\lambda(c_1w_1+\cdots+c_nw_n)\\\\
			 &=&\lambda T(v).
	\end{array}
	$$
	Así, $T$ es una transformación lineal para $V$ en $W$.\\
	Para probar que es único, suponga que $T\in \mathcal{L}(V,W)$ y que $T(v_j)=w_j$ para $j=1,\ldots,n$. Sea $c_1,\ldots,c_n\in \textbf{F}$. La Homogeneidad de $T$ implica que $T(c_jv_j)=c_jw_j$ para $j=1,\ldots,n$. La Aditividad de $T$ implica que 
	$$T(c_1v_1+\cdots+c_nv_n)=c_1w_1+\cdots+c_nw_n.$$
	Por lo tanto, $T$ se determina de forma única en $\span(v_1,\ldots,v_n)$ para la ecuación de arriba. Porque $v_1,\ldots,v_n$ es una base de $V$, esto implica que $T$ es determinado únicamente en $V$.
\end{myteo}

\vspace{.5cm}

\subsection*{Operaciones algebraicas en \boldmath $\mathcal{L}(V,W)$}

%-------------------- 3.6 definición
\begin{mydef}[Adición y multiplicación escalar en \boldmath$\mathcal{L}\left(V,W\right)$]\,\\\\
    Suponga que $S,T\in \mathcal{L}(V,W)$ y $\lambda \in \textbf{F}$. La \textbf{suma} $S+T$ y el \textbf{producto} $\lambda T$ son transformaciones lineales para $V$ en $W$ definida por
    $$(S+T)(v)=S(v)+T(v) \quad \mbox{y} \quad (\lambda T)(v)=\lambda(Tv)$$
    para todo $v\in V$.
\end{mydef}
