\chapter{Transformaciones lineales}

%-------------------- 3.1 notación
\begin{mynot}[\boldmath $F,V,W$]\,\\
    \begin{itemize}
	\item $\textbf{F}$ denota $\textbf{R}$ o $\textbf{C}$.
	\item $V$ y $W$ denota espacios vectoriales sobre $\textbf{F}$.
    \end{itemize}
\end{mynot}
\vspace{.5cm}

\mysection{El espacio vectorial de las Transformaciones lineales}
\vspace{.2cm}

\subsection*{Definición y ejemplos de Transformaciones lineales}

%-------------------- 3.2 definición
\begin{mydef}[Transformación lineal]\;\\\\
	Una \textbf{transformación lineal} de $V$ en $W$ es una función $T:V\to W$ con las siguientes propiedades:

	\begin{itemize}
	    \item \textbf{Aditividad}
		$$T(u+ v)=Tu+ Tv \;\mbox{para todo}\; u,v\in V;$$
	    \item \textbf{Homogeneidad}
		$$T(\lambda v)=\lambda(Tv) \;\mbox{para todo}\; \lambda\in \textbf{F}\; \mbox{y todo}\; v\in V.$$
	\end{itemize}
\end{mydef}

%-------------------- 3.3 notación
\begin{mynot}[\boldmath $\mathcal{L}\left(V,W\right)$]\; \\\\
    El conjunto de todas las transformaciones lineales de $V$ en $W$ se denota por $\mathcal{L}(V,W)$.
\end{mynot}

\setcounter{myteo}{4}
%-------------------- 3.5 Teorema
\begin{myteo}[Transformaciones lineales y bases del dominio]\,\\\\
    Suponga que $v_1,\ldots,v_n$ es una base de $V$ y $w_1,\ldots,w_n\in W$. Entonces, existe una única transformación lineal $T:V\to W$ tal que 
    $$T(v_j)=w_j$$
    para cada $j=1,\ldots,n$.\\\\
	Demostración.-\; Primero demostremos la existencia de una transformación lineal $T$, con la propiedad deseada. Defina $T:V\to W$ por
	$$T(c_1v_1+\cdots+c_nv_n)=c_1w_1+\cdots+c_nw_n.$$
	donde $c_1,\ldots,c_n$ son elementos arbitrarios de $\textbf{F}$. La lista $v_1,\ldots,v_n$ es una base de $V$, y por lo tanto, la ecuación anterior de hecho define una función $T$ para $V$ en $W$ (porque cada elemento de $V$ puede ser escrito de manera única en la forma $c_1v_1,\ldots,c_nv_n$). Para cada $j$, tomando $c_j=1$ y las otras $c's$ igual a $0$ demostramos la existencia de $T(v_j)=w_j$.\\
	Si $u,v\in V$ con $u=a_1v_1,\ldots,a_nv_n$ y $v=c_1v_1,\ldots,c_nv_n$, entonces
	$$
	\begin{array}{rcl}
	    T(v+u) &=& T\left[(a_1+c_1)v_1+\ldots+(a_n+a_n)v_n\right]\\\\
		   &=& (a_1+c_1)w_1+\ldots+(a_n+c_n)w_n\\\\
		   &=& (a_1w_1+\ldots+a_nw_n)+(c_1w_1+\ldots+c_nw_n)\\\\
		   &=& T(u)+T(v).
	\end{array}
	$$
	Similarmente, si $\lambda \in \textbf{F}$ y $v=c_1v_1+\cdots+c_nv_n$, entonces
	$$
	\begin{array}{rcl}
	    T(\lambda v) &=& T(\lambda c_1v_1+\cdots+\lambda c_nv_n)\\\\
			 &=&\lambda c_1w_1+\cdots+\lambda c_nw_n\\\\
			 &=&\lambda(c_1w_1+\cdots+c_nw_n)\\\\
			 &=&\lambda T(v).
	\end{array}
	$$
	Así, $T$ es una transformación lineal para $V$ en $W$.\\
	Para probar que es único, suponga que $T\in \mathcal{L}(V,W)$ y que $T(v_j)=w_j$ para $j=1,\ldots,n$. Sea $c_1,\ldots,c_n\in \textbf{F}$. La Homogeneidad de $T$ implica que $T(c_jv_j)=c_jw_j$ para $j=1,\ldots,n$. La Aditividad de $T$ implica que 
	$$T(c_1v_1+\cdots+c_nv_n)=c_1w_1+\cdots+c_nw_n.$$
	Por lo tanto, $T$ se determina de forma única en $\span(v_1,\ldots,v_n)$ para la ecuación de arriba. Porque $v_1,\ldots,v_n$ es una base de $V$, esto implica que $T$ es determinado únicamente en $V$.
\end{myteo}

\vspace{.5cm}

\subsection*{Operaciones algebraicas en \boldmath $\mathcal{L}(V,W)$}

\setcounter{mydef}{5}
%-------------------- 3.6 definición
\begin{mydef}[Adición y multiplicación escalar en \boldmath$\mathcal{L}\left(V,W\right)$]\,\\\\
    Suponga que $S,T\in \mathcal{L}(V,W)$ y $\lambda \in \textbf{F}$. La \textbf{suma} $S+T$ y el \textbf{producto} $\lambda T$ son transformaciones lineales para $V$ en $W$ definida por
    $$(S+T)(v)=S(v)+T(v) \quad \mbox{y} \quad (\lambda T)(v)=\lambda(Tv)$$
    para todo $v\in V$.
\end{mydef}

%-------------------- 3.7 teorema
\begin{myteo}[\boldmath $\mathcal{L}\left(V,W\right)$ es un espacio vectorial]\,\\\\
    Con las Operaciones de adición y multiplicación escalar como se definió, $\mathcal{L}\left(V,W\right)$ es un espacio vectorial.
\end{myteo}

Por lo general, no tiene sentido multiplicar dos elementos de un espacio vectorial, pero para algunos pares de combinaciones lineales existe un producto útil. Necesitaremos un tercer espacio vectorial, así que para el resto de esta sección supongamos que $U$ es un espacio vectorial sobre $\textbf{F}$.

%-------------------- 3.8 definición
\begin{mydef}[Producto de combinaciones lineales]\,\\\\
    Si $T\in \mathcal{L}(U,V)$ y $S\in \mathcal{L}(V,W)$, entonces el producto $ST\in \mathcal{L}(U,W)$ es definido por
    $$(ST)(u)=S(Tu)$$
    para $u\in U$.
\end{mydef}

En otras palabras, $ST$ es solo la composición habitual $S\circ T$ de dos funciones, pero cuando ambas funciones son lineales, la mayoría de los matemáticos escriben $ST$ en lugar de $S\circ T$. Debe verificar que $ST$ es de hecho una transformación lineal de $U$ a $W$ siempre que $T\in \mathcal{L}(U,V)$ y $S\in \mathcal{L}(V,W)$. Tenga en cuenta que $ST$ se define solo cuando $T$ se transforma en el dominio de $S$.

%-------------------- 3.9 teorema
\begin{myteo}[Propiedades algebraicas de producto de transformaciones lineales]\,\\\\
    \textbf{Asociatividad}
    $$(T_1T_2)T_3 = T_1(T_2T_3)$$
    siempre que $T_1$, $T_2$ y $T_3$ sean transformaciones lineales tales que los productos tengan sentido (lo que significa que $T_3$ se transforma en el dominio de $T_2$, y $T_2$ se transfora en el dominio de $T_1$).\\

    \textbf{Identidad}
    $$TI=IT=T$$
    siempre que $T\in \mathcal{L}(V,W)$ (el primer $I$ es la transformación de indentidad en $V$, y el segundo $I$ es la transformación de identidad en $W$).\\

    \textbf{Propiedades distributivas}
    \begin{center}
	$(S_1+S_2)T=S_1T+S_2T\quad$ y $\quad S(T_1+T_2)=ST_1+ST_2$
    \end{center}
    siempre que $T,T_1,T_2\in \mathcal{L}(U,V)$ y $S,S_1,S_2\in \mathcal{L}(V,W)$.
\end{myteo}

La multiplicación de aplicaciones lineales no es conmutativa. En otras palabras, no es necesariamente cierto que $ST=TS$, incluso si ambos lados de la ecuación tienen sentido.

%-------------------- 3.10 Ejemplo
\begin{myejem}
    Suponga $D\in \mathcal{L}\left[\mathcal{P}(\textbf{R}),\mathcal{P}(\textbf{R})\right]$ es la transformación de diferenciación definido en el ejemplo 3.4 y $T\in \mathcal{L}\left[\mathcal{P}(\textbf{R}),\mathcal{P}(\textbf{R})\right]$ es la multiplicación por la transformación $x^2$ definida tempranamente en esta sección. Muestre que $TD\neq DT$.\\\\
	Demostración.-\;  Se tiene 
	$$\left[(TD)p\right](x)=x^2p'(x) \quad \mbox{pero}\quad \left[(DT)p\right](x)=x^2p'(x)+2xp(x).$$
	En otras palabras, no es lo mismo derivar y luego multiplicar por  $x^2$ que multiplicar por $x^2$ y luego derivar.
\end{myejem}

%-------------------- 3.11 Teorema
\begin{myteo}[Transformaciones lineales toman \boldmath$0$ a $0$]\,\\\\
    Suponga $T$ es una transformación lineal para $V$ en $W$. Entonces $T(0)=0$.\\\\
	Demostración.-\; Por la aditividad, se tiene
	$$T(0)=T(0+0)=T(0)+T(0).$$
	Agregue el inverso aditivo de $T(0)$ cada lado de la ecuación anterior para concluir que $T(0)=0$.
\end{myteo}


\section*{Ejercicios 3.A}

\begin{enumerate}[\bfseries 1.]

    %--------------------1.
    \item Suponga $b,c\in \textbf{R}$. Defina $T:\textbf{R}^3\to \textbf{R}^2$ por
    $$T(x,y,z)=(2x-4y+3z+b,6x+cxyz).$$
    Demuestre que $T$ es lineal si y sólo si $b=c=0$.\\\\
	Demostración.-\; Por definición de transformación lineal, tendremos que demostrar que se cumple las propiedades de aditividad y homogeneidad.\\

	\textbf{Aditividad.- } Supongamos $(x_1,y_1,z_1),(x_2,y_2,z_2)\in \textbf{R}^3$. Entonces, por la definición de $T$ se tiene
	$$
	\begin{array}{rcl}
	    T\left[(x_1,y_1,z_1)+(x_2,y_2,z_2)\right] &=& T(x_1+x_2,y_1+y_2,z_1+z_2)\\\\
						      &=& \left[2(x_1+x_2)-4(y_1+y_2)+3(z_1+z_2)+b,\right.\\
						      && \left.6(x_1+x_2)+c(x_1+x_2)(y_1+y_2)(z_1+z_2)\right]\\\\
						      &=& (2x_1-4y_1+3z_1+b,6x_1+cx_1y_1z_1)\\
						      &+&(2x_2-4y_2+3z_2+b,6x_2+cx_2y_2z_2)\\\\
						      &=& T(x_1,y_1,z_1)+T(x_2,y_2,z_2).
	\end{array}
	$$
	Esto se cumplirá si $2b=b\Leftrightarrow b=0\;$ y $\;cx_1y_1z_1+cx_1y_2z_1+cx_2y_1z_1+cx_2y_2z_1+cx_1y_1z_2+cx_1y_2z_2+cx_2y_1z_2+cx_2y_2z_2 = cx_1y_1z_1+cx_2y_2z_2 \Leftrightarrow c=0$.\\

	\textbf{Homogeneidad.- } Supongamos $(x,y,z)\in \textbf{R}^3$ y $\lambda \in \textbf{R}$. Entonces, por la definición de $T$ se tiene
	$$
	\begin{array}{rcl}
	    T\left[\lambda(x,y,z)\right] &=& (2\lambda x - 4\lambda y+3\lambda z + \lambda b, 6\lambda x + c\lambda xyz)\\\\
					 &=& \lambda(2x-4y+3z+b,6x+cxyz)\\\\
					 &=& \lambda T\left[(x,y,z)\right].
	\end{array}
	$$

	Notemos que la homogeneidad se cumplirá para todo $b,c\in \textbf{R}$. Por lo tanto, $T$ es lineal si y sólo si $b=c=0$.\\\\

    %--------------------2.
    \item Suponga $b,c\in \textbf{R}$. Defina $T:\mathcal{P}(\textbf{R})\to \textbf{R}^2$  por
    $$Tp=\left[3p(4)+5p'(6)+bp(1)p(2),\int_{-1}^2 x^3p(x)\; dx +c\sen p(0)\right].$$
    Demuestre que $T$ es una transformación lineal si y sólo si $b=c=0$.\\\\
	Demostración.-\; Primero demostraremos que si $T$ es una transformación lineal, entonces $b=c=0$. Por definición para cada $\alpha\in \mathbb{R}$.
	$$\alpha Tp = T(\alpha p)$$
	De esta manera se tiene
	$$
	\begin{array}{rcl}
	    \alpha T p &=& \alpha \left[3p(4)+5p'(6)+bp(1)p(2),\displaystyle\int_{-1}^2 x^3p(x)\; dx +c\sen p(0)\right] \\\\
		       &=& \left[3\alpha p(4)+5\alpha p'(6)+b\alpha p(1)p(2),\alpha \displaystyle\int_{-1}^2 x^3 p(x) \; dx+c\alpha\sen p(0)\right]\quad (1)\\\\
		       T(\alpha p)&=& \left[3(\alpha p)(4) + 5(\alpha p)'(6) + b(\alpha p)(1)(\alpha p)(2),\displaystyle\int_{-1}^2 x^3(\alpha p)(x)\; dx + c\sen (\alpha p)(0)\right].
	\end{array}
	$$

	Es decir, si $p(x)=a_0+a_1x+a_2x^2+\cdots + a_nx^n$, entonces 
	$$(\alpha p)(x)=\left(\alpha a_0\right)+\left(\alpha a_1\right)x+\left(\alpha a_2\right)x^2+\cdots + \left(\alpha a_n\right)x^n.$$

	Por lo tanto, para cada $x$,

	$$
	\begin{array}{rcl}
	    (\alpha p)(x) &=& \left(\alpha a_0\right)+\left(\alpha a_1\right)x+\left(\alpha a_2\right)x^2+\cdots + \left(\alpha a_n\right)x^n\\\\
	    &=& \alpha\left(a_0+a_1x+a_2x^2+\cdots + a_nx^n\right)\\\\
	    &=& \alpha p(x).
	\end{array}
	$$

	Por esta deducción podemos decir que,
	$$
	\begin{array}{rcl}
	    T(\alpha p) &=& \left[3\alpha p(4)+5\alpha p'(6)+b\alpha^2p(1)p(2),\displaystyle\int_{-1}^2 \alpha x^3 p(x)\; dx + c\sen\left(\alpha p(0)\right)\right]\\\\
			&=& \left[\alpha\left(3p(4)+5p'(6)+b\alpha p(1)p(2)\right),\alpha\displaystyle\int_{-1}^2 x^3 p(x)\; dx + c\alpha \sen p(0)\right]\quad (2)\\\\
	\end{array}
	$$

	Ahora, podemos igualar (1) y (2) si y sólo si 
	$$b\alpha^2p(1)p(2)=b\alpha P(1) p(1) \quad \mbox{y} \quad c\sen(\alpha p(0))=c\alpha\sen p(0)$$
	para cada $p(x)$ y cada $\alpha \in \textbf{R}.$\\

	Sean $p(x)=x+\dfrac{\pi}{2}$ y $\alpha=2$. De donde,
	$$b\alpha^2p(1)p(2)=4b\left(1+\dfrac{\pi}{2}\right)\left(2+\dfrac{\pi}{2}\right)\quad \mbox{y}\quad b\alpha p(0)p(1)=2b\left(1+\dfrac{\pi}{2}\right)\left(2+\dfrac{\pi}{2}\right)$$
	Entonces,
	$$
	\begin{array}{rcl}
	    4b\left(1+\dfrac{\pi}{2}\right)\left(2+\dfrac{\pi}{2}\right)=2b\left(1+\dfrac{\pi}{2}\right)\left(2+\dfrac{\pi}{2}\right) &\Leftrightarrow& 4b\left(1+\dfrac{\pi}{2}\right)\left(2+\dfrac{\pi}{2}\right)-2b\left(1+\dfrac{\pi}{2}\right)\left(2+\dfrac{\pi}{2}\right)=0\\\\
																      &\Leftrightarrow& 2b \left(1+\dfrac{\pi}{2}\right)\left(2+\dfrac{\pi}{2}\right)=0\\\\
																      &\Leftrightarrow& b=0.
	\end{array}
	$$

	Por otro lado,

	$$0= c\sen\left(2\cdot \dfrac{\pi}{2}\right)=c\sin\left(\alpha p(0)\right) = c\alpha \sen p(0)=2c\sen\left(\dfrac{\pi}{2}\right)=2c$$

	Por lo tanto, 

	$$c=0.$$

	Ahora, demostremos que si $b=c=0$, entonces $T$ es lineal; es decir, que $T$ es aditiva y Homogénea. Supongamos $p,q\in \mathcal{P}(\textbf{R})$ y $\lambda \in \textbf{R}$, por lo que 

	$$
	\begin{array}{rcl}
	    T(p+q) &=& \left[3(p+q)(4)+5(p+q)'(6),\displaystyle\int_{-1}^2 x^3(p+q)(x)\; dx\right]\\\\
		   &=& \left[3\left(p(4)+q(4)\right)+5\left(p'(6)+q'(6)\right),\displaystyle\int_{-1}^2x^3\left(p(x)+q(x)\right)\; dx\right]\\\\
		   &=& \left[3p(4)+5p'(6)+3q(4)+3q'(6),\displaystyle\int_{-1}^2 x^3 p(x)\; dx +\int_{-1}^2 x^3q(x)\; dx\right]\\\\
		   &=& \left[3p(4)+5p'(6),\displaystyle\int_{-1}^2 p(x)\; dx\right]+\left[3q(4)+3q'(6),\displaystyle\int_{-1}^2 x^3q(x)\; dx\right]\\\\
		   &=& Tp+Tq.
	\end{array}
	$$

	\begin{center}
	y 
	\end{center}

	$$
	\begin{array}{rcl}
	    T(\lambda p) &=& \left[3(\lambda p)(4)+5(\lambda p)'(6),\displaystyle\int_{-1}^2 x^3(\lambda p)(x)\; dx\right]\\\\
			 &=& \left[3\lambda p(4)+5\lambda p'(6),\lambda\displaystyle\int_{-1}^2 x^3 p(x)\; dx\right]\\\\
			 &=& \left[\lambda(3p(4)+5p(6)),\lambda \displaystyle\int_{-1}^2 x^3p(x)\; dx\right]\\\\
			 &=& \lambda\left[3p(4)+5p'(6),\displaystyle\int_{-1}^2 x^3 p(x)\; dx\right]\\\\
			 &=& \lambda Tp.
	\end{array}
	$$

	Por lo tanto, $T$ es lineal. Así, concluimos que $T$ es lineal si y sólo si $b=c=0$.\\\\


    %--------------------3.
    \item Suponga $T\in \mathcal{L}\left(\textbf{F}^n,\textbf{F}^m\right)$. Demostrar que existe escalares $A_{j,k}\in \textbf{F}$ para $j=1,\ldots,m$ y $k=1,\ldots,n$ tal que
    $$T\left(x_1,\ldots,x_n\right)\in \textbf{F}^n.$$
    [El ejercicio demuestra que $T$ tiene la forma prometida en el úmtimo apartado del ejemplo 3.4.]\\\\

    %--------------------4.
    \item Suponga $T\in \mathcal{L}(V,W)$ y $v_1,\ldots, v_m$ una lista de vectores en $V$ tal que $Tv_1,\ldots,Tv_m$ es una lista linealmente independiente en $W$. Demostrar que $v_1,\ldots,v_m$ es linealmente independiente.\\\\
	Demostración.-\; Sea para $c_i\in \textbf{F}$ tal que 
	$$c_1v_1+c_2v_2+\cdots+c_nv_n=0.$$
	Luego, multiplicamos por $T$ a ambos lados de la ecuación anterior,
	$$T(c_1v_1+c_2v_2+\cdots+c_nv_n)=T(0).$$
	Por la definición 1.6 y el torema 3.11 tenemos que
	$$c_1Tv_1+c_2Tv_2+\cdots+c_nTv_n=0.$$
	Entonces, como $Tv_1,\ldots,Tv_m$ es una lista linealmente independiente en $W$.\\\\

	%--------------------5.
	\item  Demostrar la afirmación 3.7.\\\\
	    Demostración.-\; Verificaremos cada propiedad.

	    \begin{itemize}

		\item Conmutatividad.- Sean $S,T\in \mathcal{L}(V,W)$ y $v\in V$, tenemos
		    $$(S+T)(v)=S(v)+T(v)=T(v)+S(v)=(T+S)(v).$$
		    Por lo tanto, la adición es comunutativa.\\

		\item Asociatividad.- Saen $R,S,T\in \mathcal{L}(V,W)$ y $v\in V$, tenemos
		    $$
		    \begin{array}{rcl}
			\left[(R+S)+T\right](v) &=& (R+S)(v)+T(v)\\\\
						&=& R(v)+S(v)+T(v)\\\\
						&=& R(v)+\left[S(v)+T(v)\right]\\\\
		    \end{array}
		    $$
		    Por lo que la adición es asociativa. Luego, sean $a,b\in \textbf{F}$, entonces
		    $$\left[(ab)T\right](v)=(ab)T(v)=a\left[bT(v)\right]=\left[a(bT)\right](v).$$
		    Por lo tanto, la multiplicación es asociativa.\\

		\item Identidad aditiva.- Sea $0\in \mathcal{L}(V,W)$ denotado como transformación cero, sean también $T\in \mathcal{L}(V,W)$ y $v\in V$. Entonces,
		    $$(T+0)(v)=T(v)+0(v)=T(v).$$
		    Por lo tanto, la transformación cero es la identidad aditiva.\\

		\item Inverso aditivo.- Sean $T\in \mathcal{L}(V,W)$ y $v\in V$, y definamos a $(-T)\in \mathcal{L}(V,W)$ por $(-T)(v)=-T(v)$, entonces
		    $$\left[T+(-T)\right](v)=T(v)+(-T)(v)=T(v)-T(v)=0,$$
		    Por lo que, $(-T)$ es el inverso aditivo para cada $T\in \mathcal{L}(V,W)$.\\

		\item Identidad multiplicativa.- Sea $T\in \mathcal{L}(V,W)$. Entonces,
		    $$(1T)(v)=1(T(v))=T(v).$$
		    Así la identidad multiplicativa de $\textbf{F}$ es la identidad multiplicativa de la multiplicación escalar.\\

		\item Propiedad distributiva.- Sean $S,T\in \mathcal{L}(V,W)$, $a,b\in \textbf{F}$ y $v\in V$. Entonces,
		    $$
		    \begin{array}{rcl}
			\left[a(S+T)\right](v) &=& a(Sv+Tv)\\\\
					       &=& aS(v)+aT(v)\\\\
					       &=& (aS)(v)+(aT)(v)\\\\
		    \end{array}
		    $$
		    \begin{center}
			y
		    \end{center}

		    $$
		    \begin{array}{rcl}
			\left[(a+b)T\right](v) &=& (a+b)T(v)\\\\
					       &=& aT(v)+bT(v)\\\\
					       &=& (aT)(v)+(bT)(v).\\\\
		    \end{array}
		    $$
		    
	    \end{itemize}

	    Por lo tanto, $\mathcal{L}(V,W)$ es un espacio vectorial.\\\\

	%--------------------6.
	\item  Demostrar la afirmación 3.9.\\\\
	    Demostración.-\; 
	    \begin{itemize}
		\item Asociatividad.- Para $x$ en el dominio de $T_3$, tenemos
		    $$
		    \begin{array}{rcl}
			\left[(T_1T_2)T_3\right](x) &=& (T_1T_2)\left[T_3(x)\right]\\\\
						    &=& T_1\left[T_2\left[T_3(x)\right]\right]\\\\
						    &=& T_1\left[(T_2T_3)(x)\right]\\\\
						    &=& \left[T_1(T_2T_3)\right](x).
		    \end{array}
		    $$

		\item Identidad.- Para $v\in V$, se tiene
		    $$
		    \begin{array}{rcl}
			TI(v) &=& T\left[I(v)\right]\\\\
					     &=& T(v)\\\\
					     &=& I\left[T(v)\right]\\\\
					     &=& IT(v).\\\\
		    \end{array}
		    $$
		    Por lo tanto, $TI=IT=I$.\\

		\item Proipedad distributiva.- Para $u\in U$, se tiene
		    $$
		    \begin{array}{rcl}
			\left[(S_1+S_2)T\right](u) &=& (S_1+S_2)\left[T(u)\right]\\\\
						   &=& S_1\left[T(u)\right]+S_2\left[T(u)\right]\\\\
						   &=& S_1T(u)+S_2T(u)\\\\
						   &=& (S_1T+S_2T)(u).\\\\
		    \end{array}
		    $$

		    \begin{center}
			y
		    \end{center}

		    $$
		    \begin{array}{rcl}
			\left[T(S_1+S_2)\right](u) &=& T\left[(S_1+S_2)(u)\right]\\\\
						   &=& T\left[S_1(u)+S_2(u)\right]\\\\
						   &=& T\left[S_1(u)\right]+T\left[S_2(u)\right]\\\\
						   &=& (TS_1+TS_2)(u).\\\\
		    \end{array}
		    $$
		    \vspace{.5cm}


	    \end{itemize}

    %--------------------7.
    \item Demostrar que toda transformación lineal de un espacio vectorial unidimensional a sí mismo es multiplicación por un escalar. Más precisamente demostrar que si $\dim V = 1$ y $T\in\mathcal{L}(V,V)$, entonces existe $\lambda \in \textbf{F}$ tal que $Tv=\lambda v$ para todo $v\in V$.\\\\
	Demostración.-\; Supongamos que $\dim V=1$ y $T\in \mathcal{L}(V,V)$. Sea también $u\in V,u\neq 0$. Entonces, todo vector en $V$ es un multiplo escalar de $u$. En particular, $Tu=au$ para algún $a\in \textbf{F}$.\\
	Ahora, consideremos un vector $v\in V$. Ya que, existe $b\in \textbf{F}$ tal que $v=bu$. Entonces,
	$$
	\begin{array}{rcl}
	    Tv&=&T(bu)\\
	      &=& bT(u)\\
	      &=&b(au)\\
	      &=&a(bu)\\
	      &=&av.
	\end{array}
	$$
	\vspace{.5cm}

    %--------------------8.
    \item Dar un ejemplo de una función $\varphi:\textbf{R}^2\to \textbf{R}$ tal que
    $$\varphi(av)=a\varphi(v)$$
    para todo $a\in \textbf{R}$ y para todo $v\in \textbf{R}^2$ pero $\varphi$ no es lineal.\\
    $\left[\right.$\textit{El ejercicio anterior y el siguiente ejercicio muestran que ni la homogeneidad ni la aditividad por sí solas son suficientes para implicar que una función es una transformación lineal.}$\left.\right]$\\\\
	Respuesta.-\; Si $v=(x,y)\in \textbf{R}^2$, entonces definamos $\varphi:\textbf{R}^2\to \textbf{R}$ como,
	$$\varphi(v)=\varphi\left[(x,y)\right]=(x^3+y^3)^{1/3}.$$
	Calculemos ahora $\varphi(av)$;
	$$
	\begin{array}{rcl}
	    \varphi(av) &=& \varphi\left[a(x,y)\right]\\\\
			&=& \varphi\left[(ax,ay)\right]\\\\
			&=& \left[(ax)^3+(ay)^3\right]^{1/3}\\\\
			&=& \left[(a^3x^3)+(a^3y^3)\right]^{1/3}\\\\
			&=& a\left(x^3+y^3\right)^{1/3}\\\\
			&=& a\varphi\left(v\right)\\\\
	\end{array}
	$$
	Ahora, verifiquemos que $\varphi$ no es lineal. Para ello, consideremos $v_1=(1,0)$ y $v_2=(0,1)$, entonces
	$$
	\begin{array}{rcl}
	    \varphi(v_1)&=&\varphi\left[(1,0)\right]=1\\\\
	    \varphi(v_2)&=&\varphi\left[(0,1)\right]=1\\\\
	\end{array}
	$$
	De donde,
	$$\varphi(v_1)+\varphi(v_2)=1+1=2.$$
	Por otro lado,
	$$
	\begin{array}{rcl}
	    \varphi\left(v_1+v_2\right) &=& \varphi\left[(1,0)+(0,1)\right]\\\\
		    &=& \varphi\left[(1,1)\right]\\\\
		    &=& \left[(1)^3+(1)^3\right]^{1/3}\\\\
		    &=& 2^{1/3}.\\\\
	\end{array}
	$$
	Ya que, 
	$$\varphi\left(v_1+v_2\right)\neq \varphi\left(v_1\right)+\varphi\left(v_2\right) \qquad \mbox{(prop. de aditividad)}$$
	entonces $\varphi$ no es lineal.\\\\


    %--------------------9.
    \item Dar un ejemplo de una función $\varphi:\textbf{C}\to \textbf{C}$ tal que
    $$\varphi(w+z)=\varphi(w)+\varphi(z)$$
    para todo $w,z\in \textbf{C}$ pero $\varphi$ no es lineal. (Aquí \textbf{C} se considera como un espacio vectorial complejo.\\
    $\left[\right.$\textit{También existe una función} $\varphi:\textbf{R}\to \textbf{R}$  \textit{tal que} $\varphi$ \textit{satisface la condición de aditividad anterior pero} $\varphi$ \textit{no es lineal. Sin embargo, mostrar la existencia de tal función implica herramientas considerablemente más avanzadas.}$\left.\right]$\\\\
    Respuesta.-\; Sean $x_1+y_1i,x_2+y_2i\in \textbf{C}$ definida por, $\varphi:\textbf{C}\to \textbf{C}$ tal que
    $$\varphi\left[(x+yi)\right]=x-yi.$$
    Entonces,
    $$
    \begin{array}{rcl}
	\varphi\left[(x_1+y_1i)+(x_2+y_2i)\right] &=& \varphi\left[(x_1+x_2)+(y_1+y_2)i\right]\\\\
						  &=& (x_1+x_2)-(y_1+y_2)i\\\\
						  &=& (x_1-y_1i)+(x_2-y_2i)\\\\
						  &=& \varphi\left[(x_1+y_1i)\right]+\varphi\left[(x_2+y_2i)\right].
    \end{array}
    $$

    Esto demuestra la aditividad de $\varphi$. Sin embargo, sean $v=1+2i\in \textbf{C}$ y $a=i\in \textbf{C}$. Entonces,
    $$\varphi(av)=\varphi\left[i(\cdot 1+2i)\right]=\varphi\left(i+2i^2\right)=\varphi(-2+1)=-(2+i)$$
    \begin{center}
    y
    \end{center}
    $$a\varphi(v)=i\cdot \varphi(1+2i)=i(1+2i)=i-2.$$
    Esto demuestra que 
    $$\varphi(av)\neq a\varphi(v)$$
    no es lineal.\\\\


    %--------------------10.
    \item Suponga que $U$ es un subespacio de $V$ con $U\neq V$. Suponga también que $S\in \mathcal{L}(U,W)$ y $S\neq 0$ (lo que significa que $Su\neq 0$ para algún $u\in U$). Defina $T:V\to W$ por
    $$
    Tv = 
    \left\{
	\begin{array}{rcl}
	    Sv &\mbox{si} & v\in U\\\\
	    0 &\mbox{si} & v\in V \;\mbox{ y }\;v\notin U.
	\end{array}
    \right.
    $$
    Demostrar que $T$ no es una transformación lineal en $V$.\\\\
	Demostración.-\; Notemos que $U\neq V$. Luego, podemos elegir $u\in U$ tales que $Su\neq 0$ y $v\in V$, entonces $u+v\in U$. De lo contrario,
	$$v=(u+v)-u\in U$$
	producirá una contradicción. Por eso $T(u+v)=0$ por definición. Por otro lado, $Tu+Tv=SU\neq 0$. Resulta que $T(u+v)\neq Tu+Tv$, por eso $T$ no es una transformación lineal en $V$.\\\\

    %--------------------11.
    \item Suponga que $V$ es de dimensión finita. Demostrar que cada transformación lineal en una subespacio de $V$ puede ser extendida a una transformación lineal en $V$. En otras palabras, demuestra que si $U$ es un subespacio de $V$ y $S\in \mathcal{L}(U,W)$, entonces existe $T\in \mathcal{L}(V,W)$ tal que $Tu=Su$ para todo $u\in U$.\\\\
	Demostración.-\; Sea $v_1,\ldots,v_m$ una base de $U$, entonces por el teorema 2.33 podemos extenderlo a una base de $V$. Esto es, $v_1,\ldots,v_m,v_{m+1}.\ldots,v_n$ es una extensión base de $V$ tal que $v_{m+1},\ldots,v_n$ linealmente independiente.\\
	Para cualquier $z\in V$, existe $a_1,\ldots,a_n\in \textbf{F}$ tal que $z=\sum_{k=1}^n a_kv_k$ definida por

	$$
	\begin{array}{rcrcl}
	    T&:&V&\to & W\\\\
	     &&\displaystyle\sum_{k=1}^n a_kv_k &\mapsto & \displaystyle\sum_{k=1}^m a_kSv_k+\displaystyle\sum_{k=m+1}^n a_kv_k.
	\end{array}
	$$

	Dado que cada $v\in V$ tiene una única representación como una combinación lineal de elementos de nuestra base, la transformación está definida. Primero, demostremos que $T$ existe y  es una transformación lineal. Suponga $z_1,z_2\in V$. Luego, sean $a_1,\ldots,a_n\in \textbf{F}$ y $b_{1},\ldots,b_{n}\in \textbf{F}$ tal que
	$$z_1=a_1v_1+\cdots+a_nv_n \quad \mbox{y}\quad z_2=b_{1}v_{1}+\cdots+b_nv_{n}$$
	se sigue que,
	$$
	\begin{array}{rcl}
	    T(z_1+z_2) &=& T\left(\displaystyle\sum_{k=1}^n a_kv_k+\sum_{1}^n b_kv_k\right)\\\\
		       &=& T\left[\displaystyle\sum_{k=1}^n (a_k+b_k)v_k\right]\\\\
		       &=& \displaystyle\sum_{k=1}^n (a_k+b_k)Sv_k+\sum_{k=m+1}^n (a_k+b_k)v_k\\\\
		       &=& \displaystyle\sum_{k=1}^n a_kSv_k+\sum_{k=m+1}^n a_kv_k+\sum_{k=1}^n b_kSv_k+\sum_{k=m+1}^n b_kv_k\\\\
		       &=& T\left(\displaystyle\sum_{k=1}^n a_kv_k\right)+T\left(\displaystyle\sum_{k=1}^n b_kv_k\right)\\\\
		       &=& T(z_1)+T(z_2).
	\end{array}
	$$
	Lo que demuestra que $T$ es aditiva. Ahora, sean $\sum_{i=1}^n a_kv_k=z\in V$ y $\lambda\in \textbf{F}$, entonces para  $a_1,\ldots,a_n \in \textbf{F}$, se tiene
	$$
	\begin{array}{rcl}
	    T(\lambda z) &=& T\left(\lambda\displaystyle\sum_{k=1}^n a_kv_k\right)\\\\
			 &=& T\left[\displaystyle\sum_{k=1}^n (\lambda a_k)v_k\right]\\\\
			 &=& S\left[\displaystyle\sum_{k=1}^m (\lambda a_k)v_k\right]+\displaystyle\sum_{k=m+1}^n (\lambda a_k)v_k\\\\
			 &=& \lambda S\left[\displaystyle\sum_{k=1}^m a_kv_k\right]+\lambda\displaystyle\sum_{k=m+1}^n a_kv_k\\\\
			 &=& \lambda \left[S\left(\displaystyle\sum_{k=1}^m a_kv_k\right)+\displaystyle\sum_{k=m+1}^n a_kv_k\right]\\\\
			 &=& \lambda T\left(\displaystyle\sum_{k=1}^n a_kv_k\right)=\lambda Tz\\\\
	\end{array}
	$$
	por lo que $T$ es homogénea. Por lo tanto, $T\in \mathcal{L}(V,W)$. Por último, para ver si $T=S$, sea $u\in U$ con $a_1,\ldots,a_m\in \textbf{F}$ tal que $u=\sum_{k=1}^m a_kv_k$ se tiene
	$$Tu=T\left(\displaystyle\sum_{k=1}^m a_kv_k\right)=\displaystyle\sum_{k=1}^m a_kSv_k+\displaystyle\sum_{k=m+1}^n a_kv_k=S\left(\displaystyle\sum_{k=1}^m a_kv_k\right)=Su.$$
	Notemos que $\sum_{k=m+1}^n a_kv_k$ es linealmente independiente. Así, completamos la demostración.\\\\

    %--------------------12.
    \item Suponga que $V$ es de dimensión finita con $\dim(V)>0$, y suponga $W$ es de dimensión infinita. Demostrar que $\mathcal{L}(V,W)$ es de dimensión infinita.\\\\
	Demostración.-\; Sea $v_1,v_2,\ldots,v_n\in V$. Cualquier transformación lineal en $V$ está determinado únicamente por su efecto sobre los elementos base. Por lo tanto, es suficiente definir una transformación lineal solo sobre los elementos base. Ahora, 
	Recordemos por el ejercicio por el ejercicio 14A, Capitulo 2, que $W$ es un espacio vectorial de dimensión infinita si y sólo existe una secuencia $w_1,w_2,\ldots \in W$ tal que $w_1,w_2,\ldots,w_m$ es linealmente independiente para cada $m\geq 1.$ Sea $T_i\in \mathcal{L}(V,W)$ tal que $T_i(v_1)=w_i$ donde $v_1,v_2,\ldots,v_n$ es una base de $V$. La existencia de $T_i$ está garatizado por 3.5. Entonces,demostraremos que $T_1,\ldots,T_m$ es linealmente independiente para cada entero positivo $m$. Supongamos que existen $a_1,\ldots,a_m\in \textbf{F}$ tal que
	$$a_1T_1,+\cdots+a_mT_m=0.$$
	Entonces, tenemos $(a_1T_1+\cdots+a_mT_m)(v_1)=0$. Es decir,
	$$a_1w_1+\cdots + a_mw_m=0.$$
	Ya que $W_1,W_2,\ldots,W_m$ es linealmente independiente, se sigue que $a_1=\cdots=a_m=0$. Por lo tanto, $T_1,\ldots,T_m$ es linealmente independiente. Una vez más por el ejercicio 14A, Capitulo 2; se tiene que $\mathcal{L}(V,W)$ es de dimensión infinita.\\\\

    %--------------------13.
    \item Suponga que $v_1,\ldots,v_m$ es una lista de vectores linealmente dependiente en $V$. Suponga también que $W\neq \left\{0\right\}$. Demostrar que existe $w_1,\ldots,w_m\in W$ tal que que ningún $T\in \mathcal{L}(V,W)$ satisface $Tv_k=w_k$ para cada $k=1,\ldots,m$.\\\\
	Demostración.-\; Ya que, $v_1,\ldots,v_m$ es linealmente dependiente, uno de ellos puede escribirse como combinación lineal de los demás. Sin perdida de generalidad, suponga que este es $v_m$. Entonces, existe $a_1,\ldots,a_{m-1}\in \textbf{F}$ tal que 
	$$v_m=a_1v_1 + \cdots + a_{m-1}v_{m-1}$$
    Ya que, $W\neq \left\{0\right\}$, existe algún $z\in W$, $z\neq 0$. Definimos ahora $w_1,\ldots,w_m\in W$ por
	$$
	w_k=\left\{
	    \begin{array}{cl}
		z & \text{si } k=m\\
		0 & \text{en otro caso}
	    \end{array}
	\right.
	$$
	Ahora, supongamos que existe $T\in \mathcal{L}(V,W)$ tal que $Tv_k = w_k$ para $k=1,\ldots,m$. Se sigue que
	$$
	\begin{array}{rcl}
	    T(0) &=& T\left(v_m-a_1v_1-\cdots - a_{m-1}v_{m-1}\right)\\\\	
		 &=& Tv_m -a_1Tv_1-\cdots-a_{m-1}Tv_{m-1}\\\\
		 &=& z.
	\end{array}
	$$
	Pero como $z\neq 0$, entonces $T(0)\neq 0$, lo cual es un absurvo por (3.11). Por lo tanto, no existe tal transformación lineal.\\\\

    %--------------------14.
    \item Suponga que $V$ es de dimensión finita con $\dim V\geq 2$. Demostrar que existe $S,T\in \mathcal{L}(V,V)$ tal que $ST\neq TS$.\\\\
	Demostración.-\; Ya que, $\dim V \geq 2$, existe una base de $V$ formada por al menos dos vectores. Fijemos una base de $V$ y $v_1,v_2$ los primeros dos vectores de esa base. Definamos $T$ y $S$ en la base de $V$ tal que 
	\begin{center}
	    $T(v_1)=v_2,T(v_2)=v_1$ y $S(v_1)=v_2,S(v_2)=0$. 
	\end{center}
	Ahora, calculemos $ST$ y $TS$ para el vector $v_1$:
	$$
	\begin{array}{ccccccc}
	    ST(v_1) &=& S\left[T(v_1)\right] &=& S(v_2) &=& 0\\\\	
	    TS(v_1) &=& T\left[S(v_1)\right] &=& T(v_2) &=& v_1.
	\end{array}
	$$
	Lo que demuestra que 
	$$ST(v_1)\neq TS(v_1).$$
	Y por lo tanto,
	$$ST\neq TS.$$\\

\end{enumerate}


\mysection{Espacios Nulos y Rangos}
\;\\
\subsection*{Espacio Nulo (kernel) e Inyectividad}


%--------------------Definición 3.12
\begin{mydef}[Espacio nulo, null \boldmath $T$]\,\\\\
    Para $T\in \mathcal{L}(V,W)$, el \textbf{espacio nulo} de $T$ denotado por null $T$ es el subconjunto de $V$ formado por aquellos vectores que $T$ transforma a $0$:
    $$\mbox{null } T = \left\{v\in V : Tv=0\right\}.$$
\end{mydef}

Algunos matemáticos usan el termino \textbf{kernel} en lugar de \textbf{espacio nulo}. La palabra "null" significa cero.\\\\

El siguiente resultado demuestra que el espacio nulo o kernel de cada transformación lineal es un subespacio del dominio. En particular, $0$ está en el espacio nulo de cada transformación lineal.

\setcounter{myteo}{13}
%--------------------Teorema 3.14
\begin{myteo}[El espacio nulo es un subespacio]\,\\\\
    Suponga que $T\in \mathcal{L}(V,W)$. Entonces, null $T$ es un subespacio de $V$.\\\\
	Demostración.-\; Ya que $T$ es una transformación lineal, entonces sabemos por 3.11 que $T(0)=0$. Por lo tanto, $0\in \mynull T.$\\
	Supongamos ahora que $u,v\in \mynull T$. Entonces,
	$$T(u+v)=Tu+Tv=0+0=0.$$
	De ahí, $u+v\in \mbox{null } T$. Así null $T$ es cerrado bajo la adición.\\
	Luego, supongamos que $u\in \mynull T$ y $\lambda \in \textbf{F}$
	$$T(\lambda u)=\lambda T u=\lambda 0 = 0.$$
	Por lo que, $\lambda u \in \mynull T$. Así, null $T$ es cerrado bajo la multiplicación por escalares. Por 1.34, null $T$ es un subespacio de $V.$
\end{myteo}

%--------------------Definición 3.15
\begin{mydef}[Inyectiva]\,\\\\
    Una función $T:V\to W$ es llamada \textbf{inyectiva} si $Tu=Tv$ implica $u=v$.\\

    o $T$ es inyectiva si $u\neq v$ implica que $Tu\neq Tv$.\\\\
\end{mydef}

Muchos matemáticos usan el termino \textbf{uno a uno}.\\

El siguiente resultado dice que podemos comprobar si una transformación lineal es inyectiva al verificar si $0$ es el único vector que se asigna a $0$.

%--------------------Teorema 3.16
\begin{myteo}[Inyectividad es equivalente decir que el espacio nulo es igual a \boldmath$\left\{0\right\}$]\,\\\\
    Sea $T\in \mathcal{L}(V,W)$. Entonces, $T$ es inyectiva si y solo si null $T=\left\{0\right\}$.\\\\
	Demostración.-\; Primero suponga que $T$ es inyectiva. Queremos demostrar que null $T$ = $\left\{0\right\}$. Sabemos por 3.11 que $\left\{0\right\}\subset \mbox{null }T$. Para probar que $\mynull T \subset \left\{0\right\}$, suponga $v\in \mbox{null }T$. Entonces,
	$$T(v)=0=T(0).$$
	Ya que $T$ es inyectiva, implica que $v=0$. Así, podemos concluir que null $T=\left\{0\right\}$. Como queriamos.\\

	Para probar la implicación en la otra dirección. Si null $T=\left\{0\right\}$, entonces demostrarmos que $T$ es inyectiva. Para esto, suponga que $u,v\in V$ y $Tu=Tv$, de donde
	$$0=Tu-Tv=T(u-v)$$
	Así, $u-v$ está en null $T$, el cual es igual a $\left\{0\right\}$. Por lo tanto, $u-v=0$, implica que $u=v$. Concluimos que, $T$ es inyectiva.
\end{myteo}
\vspace{.5cm}

\subsection*{Rango y sobreyectividad}

Damos un nombre al conjunto de resultados de una función.

%-------------------- Definición 3.17
\begin{mydef}[Rango]\,\\\\
    Para $T$ una función de $V$ en $W$, el \textbf{rango} de $T$ es el subconjunto de $W$ que consta de aquellos vectores que son de la forma $Tv$ para algún $v\in V$:
$$\range T=\left\{Tv:v\in V\right\}.$$
\end{mydef}

Algunos matemáticos usan la palabra \textbf{imagen} en lugar de rango.\\

EL siguiente resultado muestra que el rango de cada transformación lineal es un subespacio del espacio vectorial en el que se esta transformando.

\setcounter{myteo}{18}
%--------------------Teorema 3.19
\begin{myteo}[El rango es un subespacio]\,\\\\
    Si $T\in \mathcal{L}(V,W)$, entonces el rango de $T$ es un subespacio de $W$.\\\\
	Demostración.-\; Suponga que $T\in \mathcal{L}(V,W)$. Entonces por 3.11, $T(0)=0$, lo que implica que $0\in \range T$.\\
	Si $w_1,w_2\in$ range $T$, entonces existe $v_1,v_2\in V$ tal que $Tv_1=w_1$ y $Tv_2=w_2$. Así,
	$$T(v-1+v-2)=Tv_1+v_2=w_1+w_2.$$
	Ya que $w_1+w_2\in$ rango $T$. Por lo tanto, rango $T$ es cerrado bajo la adición.\\
	Si $w\in$ rango $T$ y $\lambda\in \textbf{F}$, entonces existe $v\in V$ tal que $T_v=w$.
	Por lo que, 
	$$T(\lambda v)=\lambda Tv=\lambda w.$$
	Así, $\lambda w\in \range T$. Por lo tanto, rango $T$ es cerrado bajo la multiplicación por escalares. Por 1.34, el  $\range T$ es un subespacio de $W$.
\end{myteo}

%--------------------Definición 3.20
\begin{mydef}[Sobreyectiva]\,\\\\
	Una función $T:V\to W$ es llamada \textbf{sobreyectiva} si su rango es igual a $W$.\\\\
\end{mydef}

Que una transformación lineal sea sobreyectiva depende del espacio vectorial al que se proyecte.\\\\


\subsection*{Teorema fundamental de las transformaciones lineales}

\setcounter{myteo}{21}
%--------------------Teorema 3.22
\begin{myteo}[Teorema fundamental de las transformaciones lineales]\,\\\\
	Suponga que V es de dimensión finita y $T\in \mathcal{L}(V,W)$. Entonces, $T$ es de dimensión finita y 
	$$\dim V = \dim \mynull T + \dim \range T.$$
	    Demostración.-\; Sea $u_1,\ldots,u_m$ una base de null $T$; en consecuencia $\dim$ null $T=m$. Por 2.33, la lista linealmente independiente $u_1,\ldots,u_m$ puede extenderse a una base
	    $$u_1,\ldots,u_m,v_1,\ldots,v_n$$
	    de $V$. Así la $\dim V=m+n$. Para completar la prueba, sólo necesitamos demostrar que rango $T$ es de dimensión finita y la dim rango $T=n$. Es decir, mostrarmos que  $Tv_1,\ldots,Tv_n$ es una base del rango de $T$.\\

	    Sea $v\in V$. Ya que, $u_1,\ldots,u_m,v_1,\ldots,v_n$ genera $V$. Entonces,
	    $$v=a_1u_1+\cdots+a_mu_m+b_1v_1+\cdots+b_nv_n.$$
	    donde las $a's$ y $b's$ son en $\textbf{F}$. Aplicando $T$ a ambos lados de la ecuación, obtenemos
	    $$T_v=b_1Tv_1+\ldots+b_nTv_n.$$
	    El termino con la forma $Tu_j$ desaparece, porque cada $u_j$ esta en null $T$. De donde, la última ecuación implica que $Tv_1,\ldots,Tv_n$ genera rango $T$. En particular, rango $T$ es de dimensión finita.\\

	    Ahora, demostremos que $Tv_1,\ldots,Tv_n$ es linealmente independiente. Supongamos $c_1,\ldots,c_n\in \textbf{F}$ y 
	    $$c_1Tv_1+\cdots+c_ntv_n=0.$$
	    Entonces,
	    $$T\left(c_1v_1+\cdots +c_nv_n\right)=0.$$
	    Por lo tanto,
	    $$c_1v_1+\cdots c_nv_n \mynull T.$$
	    Luego, ya que $u_1,\ldots,u_m$ genera $\mynull T$, podemos escribir
	    $$c_1v_1+\cdots + c_nv_n = d_1u_1+\cdots + d_mu_m.$$
	    para $d's$ en $\textbf{F}$. Por el hecho de que $u_1,\ldots,u_m,v_1,\ldots,v_n$ es linealmente independiente. Entonces,  todos los $c's$ y $d's$ son $0$. Por lo tanto, $Tv_1,\ldots,Tv_n$ es linealmente independiente y por ende es una base del rango $T$, como queriamos demostrar.

\end{myteo}

Ahora podemos demostrar que ninguna transformación lineal desde un espacio vectorial de dimensión finita hacia un espacio vectorial "más pequeño" puede ser inyectivo, donde "más pequeño" se mide por la dimensión.

%-------------------- teorema 3.23
\begin{myteo}[Una transformación a un espacio de menor dimensión no es inyectiva]\,\\\\
    Suponga que $V$ y $W$ son espacios vectoriales de dimensión finita tales que $\dim V>\dim W$. Entonces, ninguna transformación lineal de $V$ en W es inyectiva.\\\\
	Demostración.-\; Sea $T\in \mathcal{L}(V,W)$. Entonces,
	$$
	\begin{array}{rcl}
	    \dim \mynull T &=& \dim V - \dim \range T\\\\
				 &\geq & \dim V - \dim W\\\\
				 &>& 0.
	\end{array}
	$$
	Donde la ecuación de arriba viene dado por el teorema fundalmental de las transformaciones lineales (3.22). La desigualdad anterior establece que $\dim$ null $T > 0$. Esto significa que $\null T$ contiene vectores distintos a cero. Por 3.16 concluimos que $T$ es no inyectiva.
\end{myteo}

El siguiente resultado muestra que ninguna transformación lineal de un espacio vectorial de dimensión finita a un espacio vectorial "más grande" puede ser sobreyectivo, donde "más grande" se mide por dimensión.

%-------------------- teorema 3.24
\begin{myteo}[Una transformación a un espacio de mayor dimensión no es suryectiva]\,\\\\
    Suponga que $V$ y $W$ son espacios vectoriales de dimensión finita tales que $\dim V<\dim W$. Entonces, ninguna transformación lineal de $V$ en $W$ es sobreyectiva.\\\\
	Demostración.-\; Sea $T\in \mathcal{L}(V,W)$. Entonces,
	$$
	\begin{array}{rcl}
	    \dim \range T &=& \dim V - \dim \mynull T\\\\
				  &\leq & \dim V\\\\
				  &<& \dim W.
	\end{array}
	$$
	Donde la ecuación de arriba viene dado por el teorema fundalmental de las transformaciones lineales (3.22). La desigualdad anterior establece que range $T < \dim W$. Esto significa que range $T$ no puede ser igual a $W$.  Por lo tanto, $T$ es no sobreyectiva.\\\\
\end{myteo}

Como veremos a continuación, 3.23 y 3.24 tienen importantes consecuencias en la teoría de ecuaciones lineales. La idea aquí es expresar cuestiones sobre sistemas de ecuaciones lineales en términos de transformaciones lineales.

%-------------------- ejemplo 3.25
\begin{myejem}
    Reformule en términos de transformaciones lineales la pregunta de si un sistema homogéneo de ecuaciones lineales tiene una solución distinta de cero.\\\\
	Respuesta.-\; Sean los enteros $m$ y $n$ y sea $A_{j,k}\in \textbf{F}$ para $j=1,\ldots,m$ y $k=1,\ldots,n$. Considere el sistema homogéneo de ecuaciones lineales
	$$
	\begin{array}{rcl}
	    \displaystyle\sum_{k=1}^n A_{1,k}x_k &=& 0\\
						 &\vdots&\\
	     \displaystyle\sum_{k=1}^n A_{m,k}x_k &=& 0.
	\end{array}
	$$
	Está claro que $x_1=\cdots=x_n=0$ es la solución al sistema de ecuaciones; la cuestión aquí es si existen otras soluciones.\\

	Defina $T:\textbf{F}^n \to \textbf{F}^m$ por
	$$T(x_1,\ldots,x_n)=\left(\sum_{k=1}^n A_{1,k}x_k,\ldots,\sum_{k=1}^n A_{m,k}x_k\right).$$
	La ecuación $T(x_1,\ldots, x_n)=0$ (el $0$ aquí es la identidad aditiva en $\textbf{F}^m$, es decir, la lista de longitud $m$ de todos los $0$) es la misma que el sistema homogéneo de ecuaciones lineales anterior.\\
	Así pues, queremos saber si null $T$ es estrictamente mayor que $\left\{0\right\}$. En otras palabras, podemos reformular nuestra pregunta sobre las soluciones no nulas de la siguiente manera (por 3.16): ¿Qué condición asegura que $T$ no es inyectiva?
\end{myejem}

%-------------------- teorema 3.26
\begin{myteo}[Sistemas homogéneos de ecuaciones lineales]\,\\\\
    Un sistema homogéneo de ecuaciones lineales con más variables que ecuaciones tienen soluciones distintas de cero.\\\\
	Demostración.-\; Usemos la notación y el resultado de arriba. De donde, $T$ es una transformación lineal de $\textbf{F}^n$ en $\textbf{F}^m$, y tenemos un sistema homogéneo de $m$ ecuaciones lineales con $n$ variables $x_1,\ldots,x_n$. Por 3.23 vemos que $T$ no es inyectiva si $n>m$.
\end{myteo}

\setcounter{myteo}{28}
%-------------------- teorema 3.29
\begin{myteo}[Sistemas no homogéneos de ecuaciones lineales]\,\\\\
    Un sistema no homogéneo de ecuaciones lineales con más ecuaciones que variables no tienen solución para alguna elección de los términos constantes.\\\\
	Demostración.-\; $T$ es una transformación lineal para $\textbf{F}^n$ en $\textbf{F}^m$, y tenemos un sistema de $m$ ecuaciones con $n$ variables $x_1,\ldots,x_n$. Por 3.24 vemos que $T$ es no sobreyectiva si $n<m$.
\end{myteo}


\section*{Ejercicios 3.B}

\begin{enumerate}[\bfseries 1.]

    %--------------------1.
    \item Dar un ejemplo de una transformación lineal tal que $\dim$ null $T=3$ y $\dim \range T=2$.\\\\
	Respuesta.-\; Sea, la transformación lineal $T:\textbf{R}^5 \to \textbf{R}^5$ tal que 
	$$(x_1,x_2,x_3,x_5,x_5)=(0,0,0,x_4,x_5).$$
	Primero demostraremos que es una transformación lineal. Sea $x,y\in \textbf{R}^5$. Entonces,
	$$
	\begin{array}{rcl}
	    T(x+y) &=& T\left[(x1,x_2,x_3,x_4,x_5)+(y_1,y_2,y_3,y_5,y_5)\right]\\\\
		   &=& T(x_1+y_1,x_2+y_2,x_3+y_3,x_4+y_4,x_5+y_5)\\\\
		   &=& (0,0,0,x_4+y_4,x_5+y_5)\\\\
		   &=& (0,0,0,x_4,x_5)+(0,0,0,y_4,y_5)\\\\
		   &=& T(x)+T(y).
	\end{array}
	$$
	Luego, sea $\lambda \in \textbf{R}$. Entonces,
	$$
	\begin{array}{rcl}
	    T(\lambda x) &=& T(\lambda x_1,\lambda x_2,\lambda x_3,\lambda x_4,\lambda x_5)\\\\
			 &=& (0,0,0,\lambda x_4,\lambda x_5)\\\\
			 &=& \lambda(0,0,0,x_4,x_5)\\\\
			 &=& \lambda T(x).
	\end{array}
	$$
	Por lo tanto, $T$ es una transformación lineal.  Notemos ahora que,
	$$\mynull T=\left\{(x_1,x_2,x_3,0,0)\in \textbf{R}^5 | x_1,x_2,x_3\in \textbf{R}\right\}.$$
	Esta espacio claramente tiene como base a $v_1,v_2,v_3\in \textbf{R}^5$, por ejemplo $(1,0,0,0,0),(0,1,0,0,0)$, $(0,0,1,0,0)$, por lo tanto tiene dimensión $3$, y dado que $\dim$ range $T=2$, por el teorema fundamental de transformaciones lineales, se tiene que
	$$\dim \textbf{R}^5=3+\dim \range T.$$
	De esta manera concluimos la demostración.\\\\

    %--------------------2.
    \item Suponga que $V$ es un espacio vectorial y $S,T\in \mathcal{V,V}$ tales que
    $$\range S\subset \mynull T.$$
    Demostrar que $(ST)^2=0.$\\\\
	Demostración.-\; Sea $v\in V$. Por definición 
	$$(ST)^2(v)=S\left(T(S(T(v)))\right)$$

    %--------------------3.
    \item Suponga que $v_1,\ldots,v_m$ es una lista de vectores en $V$. Defina $T\in \mathcal{L}\left(\textbf{F}^m,V\right)$ por
    $$T(z_1,\ldots,z_m)=z_1v_1+\cdots+z_mv_m.$$
	
    \begin{enumerate}[(a)]

	%----------(a)
	\item ¿Qué propiedad de $T$ corresponde a $v_1,\ldots,v_m$ que genere $V$?\\\\
	    Respuesta.-\; El conjunto $v_1,v_2,\ldots,v_n$ es un conjunto generador de $V$ si y sólo si para cualquier $v\in V$, existe $z_1,z_2,\ldots,z_n\in \textbf{F}$ tal que 
	    $$v=z_1v_1+z_2v_2+\cdots+z_nv_n.$$
	    Esto es equivalente a decir que $v=T(z_1,z_2,\ldots,z_n)$ para algunos $(z_1,z_2,\ldots,z_n)\in \textbf{F}^n$. Esto es, $v_1,v_2,\ldots,v_n$ es un conjunto generador si y sólo si cada $v\in V$ se encuentra en el rango $T$. Es decir, si y sólo si $V=\range T$. Por lo tanto, $v_1,v_2,\ldots,v_n$ es un conjunto generador de $V$ si y sólo si $T$ es sobreyectiva.\\\\

	%----------(b)
	\item ¿Qué propiedad de $T$ corresponde a $v_1,\ldots,v_m$ que sea linealmente independiente?\\\\
	    Respuesta.-\; El conjunto $v_1,v_2,\ldots,v_n$ es linealmente independiente si y sólo si
	    $$z_1v_1+\cdots+z_nv_n = 0.$$
	    implica que $z_1,z_2,\ldots,z_n=0.$
	    Esto es equivalente a decir que $T(z_1,z_2,\ldots,z_n)=0$ de donde $(z_1,z_2,\ldots,z_n)=(0,0,\ldots,0)$ lo que significa  que el único elemento que se encuentra en $T$ es $(0,0,\ldots,0).$ Esto es verdadero sólo cuando $T$ es inyectiva. Por lo tanto, $v_1,v_2,\ldots,v_n$ es linealmente independiente si y sólo si $T$ es inyectiva.\\\\

    \end{enumerate}

    %--------------------4.
    \item Demostrar que
    $$\left\{ T\in \mathcal{L}\left(\textbf{R}^5,\textbf{R}^4\right):\dim \mynull T>2 \right\}$$
    no es un subespacio de $\mathcal{L}\left(\textbf{R}^5,\textbf{R}^4\right)$.\\\\
	Demostración.-\; Sea $e_1,\ldots,e_5$ una base de $\textbf{R}^5$ y sea $d_1,\ldots,d_4$ una base para $\textbf{R}^4$. Definamos $S_1$ y $S_2$ por
	$$
	\begin{array}{cccc}
	    S_1e_i=0,&S_1e_4=d_1,&S_1e_5=d_2,&\mbox{para } i=1,2,3;\\\\
	    S_2e_i=0,&S_2e_3=d_3,&S_2e_5=d_4,&\mbox{para } i=1,2,4.
	\end{array}
	$$
	Entonces, $S_1,S_2\in \left\{T\in \mathcal{L}\left(\textbf{R},\textbf{R}^4\right):\dim \null T > 2\right\}$. Sin embargo,
	$$(S_1+S_2)e_1=0,\quad (S_1+S_2)e_2=0$$
	y
	$$(S_1+S_2)e_3=d_3, \quad (S_1+S_2)e_4=d_1, \quad (S_1+S_2)e_5=d_2+d_4.$$
	De donde, podemos verificar que $\dim \mynull(S_1+S_2)=2$. Así, $\left\{T\in \mathcal{L}\left(\textbf{R},\textbf{R}^4\right):\dim \mynull T > 2\right\}$ no es cerrada bajo la adición, esto implica que no es un subespacio de $\mathcal{L}\left(\textbf{R}^5,\textbf{R}^4\right)$.\\\\

    %--------------------5.
    \item Dar un ejemplo de una transformación lineal $T:\textbf{R}^4\to \textbf{R}^4$ tal que
    $$\range T=\mynull T.$$\\
	Respuesta.-\; Definimos a
	$$
	\begin{array}{rcl}
	    T &:& \textbf{R}^4\to \textbf{R}^4\\\\
	    (x_1,x_2,x_3,x_4) &\mapsto & (x_3,x_4,0,0)
	\end{array}
	$$
	Claramente $T$ es una transformación lineal, y 
	$$\mynull T = \left\{(x_1,x_2,x_3,x_4) | x_3=x_4=0 \in \textbf{R}\right\}.$$
	Como también 
	$$\range T = \left\{(x,y,0,0) | x,y \in \textbf{R}\right\}.$$
	Por lo tanto, $\range T = \mynull T$.\\\\

    %--------------------6.
    \item Demostrar que no existe una transformación lineal $T:\textbf{R}^5\to \textbf{R}^5$ tal que
    $$\range T = \mynull T.$$
	Demostración.-\; Por el teorema fundamental de las transformaciones lineales, se tiene
	$$\dim \range T + \dim \mynull T = 5.$$
	Esto es imposible, ya que si $\range T \neq \mynull T$  y la dimensión debe ser entero.\\\\

    %--------------------7.
    \item Suponga que $V$ y $W$ son de dimensión finita con $2\leq \dim V \leq \dim W$ . Demostrar que 
    $$\left\{T\in \mathcal{L}(V,W):T\mbox{ no es inyectivo}\right\}$$ 
    no es un subespacio de $\mathcal{L}(V,W)$.\\\\
    Demostración.-\; Sea $\mathcal{Z}=\left\{T\in \mathcal{L}(V,W):T\mbox{ no es inyectivo}\right\}$.  Sea $v_1,\ldots,v_m$ una base de $V$, donde $m\geq 2$, y sea $w_1,\ldots,w_n$ una base de $W$, donde $n\geq m$. Definamos $T,S\in \left(V,W\right)$ como en la hipótesis:
	$$
	Tv_k = 
	    \left\{
		\begin{array}{rcl}
		    w_1 &\mbox{si}& k=1\\
		    0 &\mbox{si}& k=2,\ldots, m\\
		\end{array}
	    \right.
	\qquad \mbox{y} \qquad
	Sv_k = 
	    \left\{
		\begin{array}{lcl}
		    0 &\mbox{si}& k=1\\
		    w_m &\mbox{si}& k=2,\ldots, m\\
		\end{array}
	    \right.
	$$
	De donde, $T$ y $S$ no son inyectivas, ya que $v_2\in \mynull T$ y $v_1\mynull S$. Ahora, veamos que $T+S$ que actúa en $V$.
	$$(T+S)v_1=Tv_1+Sv_1=w_1+0=w_1$$
	$$\mbox{y}$$
	$$(T+S)v_k = Tv_k+Sv_k=0+w_k=w_k,\; \mbox{ para }2\leq k \leq m.$$
	Esto muestra que todos los $w_k's$ se encuentran en el $\range (T+S)$ para $1\leq k \leq m$. Así, $\dim\range (T+S)=m$. Por el teorema fundamental de las transformaciones lineales, se tiene
	$$\dim\mynull(T+S)=\dim V-\dim \range(T+S)=m-m=0.$$
	Por lo tanto, $T+S$ es inyectiva. Recordemos, que $\mathcal{Z}$ será un subespacio, si $T,S\in \mathcal{Z}$, entonces $T+S\in \mathcal{Z}$, pero dado que $T+S\notin \mathcal{Z}$, entonces concluimos que $\mathcal{Z}$ no es un subespacio de $\mathcal{L}(V,W)$.\\\\

    %--------------------8.
    \item Suponga que $V$ y $W$ son de dimensión finita con $\dim V \geq \dim W\geq 2$. Demostrar que 
	$$\left\{T\in \mathcal{L}(V,W):T\mbox{ no es sobreyectiva}\right\}$$ 
	no es un subespacio de $\mathcal{L}(V,W)$.\\\\
	Demostración.-\; Sea $\mathcal{Z}= \left\{T\in \mathcal{L}(V,W):T\mbox{ no es sobreyectiva}\right\}$. Sean también $v_1,\ldots,v_m$ una base de $V$ y $w_1,\ldots,w_n$ una base de $W$, para $2\leq m \leq n$. Definamos $T,S\in \mathcal{L}(V,W)$ como en la hipótesis:
	$$
	Tv_k = 
	    \left\{
		\begin{array}{rcl}
		    w_k &\mbox{si}& 1\leq k \leq m-1\\
		    0 &\mbox{si}& k\geq m.
		\end{array}
	    \right.
	\qquad \mbox{y} \qquad
	Sv_k = 
	    \left\{
		\begin{array}{lcl}
		    0 &\mbox{si}& 1\leq k \leq m-1\\
		    w_k &\mbox{si}& k\geq m\\
		\end{array}
	    \right.
	$$
	Notemos que $w_m$ no se encuentra en el $\range T$. Por lo tanto, $T$ no es sobreyectiva. De manera similar, $S$ no es sobreyectiva, ya que $w_1,w_2,\ldots,w_{m-1}\notin \range S$. Ahora, veamos que $T+S$ actúa sobre $V$.
	$$(T+S)v_k=Tv_k+Sv_k=w_k+0=w_k,\; \mbox{ para }1\leq k \leq m-1.$$
	$$\mbox{y}$$
	$$(T+S)v_k = Tv_k+Sv_k=0+w_m=w_m,\; \mbox{ para }k\geq m.$$
	Esto muestra que todos los $w_k's$ se encuentran en el $\range (T+S)$ para $1\leq k \leq m$. Esto es, $T+S$ es sobreyectiva. Dado que, $\mathcal{Z}$ será un subespacio, si $T,S\in \mathcal{Z}$, entonces $T+S\in \mathcal{Z}$, pero dado que $T+S\notin \mathcal{Z}$, entonces concluimos que $\mathcal{Z}$ no es un subespacio de $\mathcal{L}(V,W)$.\\\\

    %--------------------9.
    \item Suponga que $T\in \mathcal{L}(V,W)$ es inyectiva y $v_1,\ldots,v_n$ es linealmente independiente en $V$. Demostrar que $Tv_1,\ldots,Tv_n$ es linealmente independiente en $W$.\\\\
	Demostración.-\; Sean $a_1,\ldots, a_n\in \textbf{F}$ tales que
	$$a_1Tv_1+\cdots+a_nTv_n=0.$$
	Ya que, $T$ es una transformación lineal, se sigue
	$$T(a_1v_1+\cdots+a_nv_n)=0$$
	Por 1.16, $\mynull T=\left\{0\right\}$, esto implica que 
	$$a_1v_1+\cdots+a_nv_n=0$$
	Luego, debido a que $v_1,\ldots,v_n$ es linealmente independiente, 
	$$a_1=\cdots = a_n=0.$$
	De donde,
	$$a_1Tv_1+\cdots+a_nTv_n=0$$
	Concluimos que $Tv_1,\ldots,Tv_n$ es linealmente independiente.\\\\

    %--------------------10.
    \item Suponga $v_1,\ldots, v_n$ genera $V$ y $\mathcal{L}(V,W)$. Demostrar que la lista $Tv_1,\ldots,Tv_n$ genera $\range T$.\\\\
	Demostración.-\; Suponga que $(v_1,\ldots,v_n)$ genera $V$ y $T\in \mathcal{L}(V,W)$ es sobreyectiva. Sea $w\in W$. Ya que, $T$ es sobreyectiva existe $v\in V$ tal que $Tv=w$. Luego, por el hecho de que $(v_1,\ldots,v_n)$ genera $V$, existe $a_1,\ldots,a_n\in F$ tal que
	$$v=a_1v_1+\cdots+a_nv_n.$$
	Multiplicando $T$ a ambos lados,
	$$Tv=a_1Tv_1+\cdots+a_nTv_n.$$
	Que $Tv=w$, entonces $w\in \span(Tv_1,\ldots,Tv_n)$. Sabemos que $w$ es un vector arbitrario de $W$, lo implica que $Tv_1,\ldots,Tv_n$ genera $W$.\\\\

    %--------------------11.
    \item Suponga que $S_1,\ldots,S_n$ son transformaciones lineales inyectivas tal que $S_1S_2,\cdots S_n$ toma sentido. Demostrar que $S_1S_2\cdots S_n$ es inyectiva.\\\\
	Demostración.-\; Supongamos que $S_1,\ldots,S_n$ son transformaciones lineales inyectivas tal que $S_1,\ldots,S_n$ tenga sentido; es decir, el domino de $S_1,\cdots , S_n$ tal que $S_1S_2\cdots S_n$ esta bien definido. Suponga $v$ un vector en el dominio de $S_1S_2\cdots S_n$ tal que
	$$(S_1S_2\cdots S_n)v=0.$$
	Para demostrar que $S_1S_2\cdots S_n$ es inyectiva, necesitamos demostrar que $v=0$ (3.16). Ahora bien, podemos reescribir la ecuación anterior como:
	$$S_1(S_2\cdots S_nv)=0.$$
	Ya que $S_1$ es inyectiva por 3.12 y 3.16, se sigue que
	$$(S_2\cdots S_n)v=0.$$
	Si seguimos con esta lógica, obtenemos
	$$S_n v = 0$$
	Esto implica que $v=0$, como queríamos. Por lo tanto, $S_1S_2\cdots S_n$ es inyectiva.\\\\

    %--------------------12.
    \item Suponga que $V$ es de dimensión finita y que $T\in \mathcal{L}(V,W)$. Demostrar que existe un subespacio $U$ de $V$ tal que $U\cap \mynull T = \left\{0\right\}$ y $\range T = \left\{Tu:u\in U\right\}$.\\\\
	Demostración.-\; Por 2.34, existe un subespacio $U$ de $V$ tal que
	$$V=\mynull T\oplus U.$$
	De donde, por definición de suma directa tenemos que 
	$$U \cap \mynull T = \left\{0\right\}.$$
	Por otro lado, observamos que $\left\{Tu:u\in U\right\}\subset \range T$. Para demostrar la inclusión en la otra dirección, supongamos que $v\in V$. Entonces, existe $w\in \mynull T$ y $u'\in U$ tal que
	$$v=w+u'.$$
	Aplicando $T$ a ambos lados, se tiene
	$$Tv=Tw+Tu'=Tu'.$$
	Ya que $v$ es un vector arbitrario de $V$ y $Tv$ es un vector arbitrario de $\range T$, 
	$$Tv\in \left\{Tu:u\in U\right\}$$
	Esto implica que
	$$\range T \subset \left\{Tu:u\in U\right\}$$
	Así, $\range T =\left\{Tu:u\in U\right\}$. Como queríamos demostrar.\\\\

    %--------------------13.
    \item Suponga que $T$ es una transformación lineal de $\textbf{F}^4$ en $\textbf{F}^2$ tal que
    $$\mynull T = \left\{(x_1,x_2,x_3,x_4)\in \textbf{F}^4:x_1=5x_2 \mbox{ y } x_3=7x_4\right\}.$$
    Demostrar que $T$ es sobreyectiva.\\\\
	Demostración.-\; Supongamos que $T\in \mathcal{L}(\textbf{F}^4,\textbf{F}^2)$ es tal que $\mynull T$ definido como en la hipótesis. Entonces, (5,1,0,0),(0,0,7,1) es una base de $\mynull T$, y por lo tanto $\dim\mynull T = 2$. Por el teorema fundalmental de transformaciones lineales
	$$
	\begin{array}{rcl}	
	    \dim \range T &=& \dim \textbf{F}^4 - \dim \mynull T\\ 
			&=& 4-2\\
			&=& 2.
	\end{array}
	$$
	Ya que, $\range T$ es un subespacio de dos dimensiones, entonces $T=\textbf{R}^2$. En otras palabras, $T$ es sobreyectiva.\\\\

    %--------------------14.
    \item Suponga que $U$ es un subespacio de 3 dimensiones de $\textbf{R}^8$ y  $T$ una transformación lineal para $\textbf{R}^8$ en $\textbf{R}^5$ tal que $\mynull T=U$. Demostrar que $T$ es sobreyectiva.\\\\
	Demostración.-\; Esta es una demostración directa. Por el teorema fundamental de transformaciones lineales,
	$$\dim \textbf{R}^8 = \dim \mynull T + \dim \range T.$$
	De donde,
	$$\dim \range T = \dim \textbf{R}^8 - \dim \mynull T.$$
	Dado que $\dim \textbf{8}$ y $\dim\mynull 3$ tienen dimensión 8 y 3, respectivamente, se sigue que
	$$\dim \range T = 8-3=5.$$
	Ahora, es fácil ver que $\textbf{R}^5$ tiene dimensión 5 siempre que $\range T=\textbf{R}^5$. Por lo tanto, $T$ es sobreyectiva.\\\\

    %--------------------15.
    \item Demostrar que no existe una transformación lineal para $\textbf{F}^5$ en $\textbf{F}^2$ cuyo espacio nulo es
    $$\left\{(x_1,x_2,x_3,x_4,x_5)\in \textbf{F}^5 : x=3x_2 \mbox{ y } x_3=x_4=x_5\right\}.$$\\
	Demostración.-\; Sea, $V=\textbf{F}^5$; de donde, $\dim V = 5$. Ya que $T\subseteq \textbf{F}^2$, entonces $\dim T \leq 2$. Por el teorema fundamental de transformaciones lineales,
	$$
	\begin{array}{rcl}
	    \dim \range T &=& \dim V - \dim \mynull T\\
			  &\geq& 5-2\\
			  &=& 3.
	\end{array}
	$$
	Esto es, $\dim \mynull T$ debe ser por lo menos $3$.\\
	Ahora, observamos que
	$$
	\begin{array}{rcl}
	    U &=& \left\{(x_1,x_2,x_3,x_4,x_5) | x_1=3x_2,\; x_3=x_4=x_5\right\}\\\\
	      &=& \left\{(3x_2,x_2,x_3,x_3,x_3) | x_2,x_4\in \textbf{F}\right\}\\\\
	      &=& \left\{x_2(3,1,0,0,0)+x_3(0,0,1,1,1) | x_2,x_3\in \textbf{F}\right\}.
	\end{array}
	$$

	Esto muestra que $U=\span \left\{(3,1,0,0,0),(0,0,1,1,1)\right\}$. Que $U$ es generado por dos vectores implica que la dimensión de $U$ puede ser a lo suma $2$. Esto junto con la conclusión del paso anterior prueba que $U$ nunca puede ser el espacio nulo de ningúna Transformación lineal $T:\textbf{F}^5\to \textbf{F}^2$.\\\\


    %--------------------16.
    \item Suponga que existe una transformación lineal en $V$ cuyo espacio nulo y rango son ambos de dimensión finita. Demuestre que $V$ es de dimensión finita.\\\\
	Demostración.-\; Suponga que $V$ es un espacio vectorial y $T$ es una transformación lineal definido en $V$. Si el rango y el espacio nulo de $T$ son ambos de dimensión finita, entonces el lado derecho de la ecuación 
	$$\dim V = \dim \mynull T + \dim \range T,$$
	es un número finito. Por lo tanto, el lado izquierdo también debe ser un número finito. En otras palabras, $V$ debe ser de dimensión finita.\\\\

    %--------------------17.
    \item Suponga que $V$ y $W$ son ambos de dimensión finita. Demostrar que existe una transformación lineal inyectiva para $V$ en $W$ si y sólo si $\dim V \leq \dim W$.\\\\
	Demostración.-\; Por un lado, suponga que $T\in \mathcal{L}(V,W)$ es inyectiva. Si $\dim V>\dim W$, entonces por el teorema 3.23, ninguna transformación lineal es inyectiva. lo que es una contradicción. Por lo tanto $\dim V\leq \dim W$.\\
	Por otro lado, suponga que $\dim V\leq \dim W$. Si $\dim V=\dim W$ Entonces, la conclusión de la transformación lineal 


    %--------------------18.
    \item Suponga que $V$ y $W$ son ambos de dimensión finita. Demostrar que existe una transformación lineal sobreyectiva para $V$ sobre $W$ si y sólo si $\dim V \geq \dim W$.\\\\

\end{enumerate}

\mysection{Matrices}

\subsection*{Representando una transformación lineal por una matriz}

Sabemos que si $v_1,\ldots,v_n$ es una base de $V$ y $T:V\to W$ es lineal, entonces los valores de $Tv_1,\ldots,Tv_n$ determina los valores de $T$ en vectores arbitrarios en $V$ (3.5). Como veremos, las matrices se usan como un método eficiente para registrar los valor de las $Tv_j's$ en términos de una base de $W$.

%--------------------teorema 3.30
\begin{mydef}[Matriz, \boldmath$A_{j,k}$]\;\\\\
    Sea $m$ y $n$ que denota enteros positivos. Un \textbf{matriz} $A$ de $m$ por $n$ es un arreglo rectangular de elementos de $\textbf{F}$ con $m$ filas y $n$ columnas:
    $$
    A=
    \begin{pmatrix}
	A_{1,1} & \cdots & A_{1,n}\\
	\vdots &  & \vdots\\
	A_{m,1} & \cdots & A_{m,n}
    \end{pmatrix}
    $$
    La notación $A_{j,k}$ denota la entrada en la fila $j$, columna $k$ de $A$. En otras palabras, el primer indice se refiere al número de fila y el segundo indice se refiere al número de columna.
\end{mydef}

Ahora, veamos la definición clave de esta sección

\setcounter{mydef}{31}
%--------------------definición 3.32
\begin{mydef}[Matriz de una transformación lineal, \boldmath$\mathcal{M}\left(T\right)$]\;\\\\
    Suponga que $T\in \mathcal{L}(V,W)$ y $v_1,\ldots,v_n$ es una base de $V$ y $w_1,\ldots,w_m$ es una base de $W$. La matriz de $T$ con respecto a estas bases es la matriz $m$ por $n$, $\mathcal{M}(T)$, cuyas entradas $A_{j,k}$ están definidas por
    $$Tv_k = A_{1,k}w_1+\cdots+A_{m,k}w_m.$$
    Si las bases no están claras por el contexto, entonces la notación $\mathcal{M}\left(T,(v_1,\ldots,v_n),(w_1,\ldots,w_m)\right)$  es usada.
\end{mydef}

Para recordar cómo se construye $\mathcal{M}(T)$ a partir de $T$, puedes escribir en la parte superior de la matriz los vectores base $v_1,\ldots,v_n$ para el dominio y a la izquierda los vectores base $w_1,\ldots,w_m$ para el espacio vectorial al que transforma $T$, como sigue:
$$
\mathcal{M}(T)=
    \begin{array}{cc}
	&v_1 \quad \cdots\quad v_k \quad \cdots \quad v_n\\\\
	\begin{array}{c}
	    w_1\\\\
	    \vdots\\\\
	    w_m
	\end{array}
	&
	\begin{pmatrix}
	    \qquad\qquad\qquad&A_{1,k}&\qquad\qquad\qquad\\\\
		  &\vdots&\\\\
		  &A_{m,k}&
	\end{pmatrix}
    \end{array}
$$

La columna $k$-enésimo de $\mathcal{M}(T)$ consiste en los escalares necesarios para escribir $Tv_k$ como una combinación lineal de $w_1,\ldots,w_m$:
$$Tv_k=\sum_{j=1}^m A_{j,k}w_j.$$

Es decir, $Tv_{k}$ puede calcularse a partir de $\mathcal{M}(T)$ multiplicando cada entrada de la columna $k$ por la correspondiente $w_j$ de la columna de la izquierda, y sumando después los vectores resultantes.\\\\


\subsection*{Adición y multiplicación escalar de matrices}

Para el resto de esta sección, supongamos que V y W son de dimensión finita y que se ha elegido una base para cada uno de estos espacios vectoriales. Así, para cada transformación lineal de V a W, podemos hablar de su matriz (con respecto a las bases elegidas, por supuesto).

\setcounter{mydef}{34}
%--------------------teorema 3.35
\begin{mydef}[Adición de matrices]\;\\\\
    La \textbf{suma de dos matrices de un mismo tamaño} que se obtiene sumando las entradas correspondientes de las matrices:
    $$
    \begin{pmatrix}
	A_{1,1}&\cdots&A_{1,n}\\
	\vdots&&\vdots\\
	A_{m,1}&\cdots&A_{m,n}
    \end{pmatrix}
    +
    \begin{pmatrix}
	C_{1,1}&\cdots&C_{1,n}\\
	\vdots&&\vdots\\
	C_{m,1}&\cdots&C_{m,n}
    \end{pmatrix}
    =
    \begin{pmatrix}
	A_{1,1}+C_{1,1}&\cdots&A_{1,n}+C_{1,n}\\
	\vdots&&\vdots\\
	A_{m,1}+C_{m,1}&\cdots&A_{m,n}+C_{m,n}
    \end{pmatrix}
    $$
    En otras palabras, $(A+C)_{j,k}=A_{j,k}+C_{j,k}$.
\end{mydef}

En el siguiente resultado, se supone que se utilizan las mismas bases para las tres transformaciones lineales $S + T, S ,$ y $T$.

%--------------------teorema 3.36
\begin{myteo}[La matriz de la suma de transformaciones lineales]\;\\\\
    Suponga $S,T\in \mathcal{L}(V,W)$. Entonces, $\mathcal{M}(S+T)=\mathcal{M}(S)+\mathcal{M}(T)$.\\\\
    Demostración.-\; Sean $v_1,\ldots, v_n$ una base de $V$ y $w_1,\ldots,w_m$ una base de $W$. Además las entradas de $\mathcal{M}(S)$ y $\mathcal{M}(T)$ son $A_{j,k}$ y $B_{j,k}$ respectivamente definidas por,
    $$Sv_k=A_{1,k}w_1+\cdots+A_{m,k}w_m$$
    $$\mbox{y}$$
    $$Tv_k=B_{1,k}w_1+\cdots+B_{m,k}w_m$$
    Entonces,
    $$
    \begin{array}{rcl}
	(S+T)v_k&=& Sv_k+Tv_k\\\\
		&=&\left(A_{1,k}w_1+\cdots +A_{m,k}w_m\right)+\left(B_{1,k}w_1+\cdots +B_{m,k}w_m\right)\\\\
		&=&\left(A_{1,k}+B_{1,k}\right) w_1+\cdots +\left(A_{m,k}+B_{m,k}\right)w_m.
    \end{array}
    $$

    Se deduce que las entradas en la fila $k$, columna $k$ de $M(T+S)$ con respecto a estas bases son $A_{j,k}+B_{j,k}$. Luego, por la definición 3.35, se tiene que,
    $$A_{j,k}+B_{j,k}= (A+B)_{j,k}.$$
    De donde, concluimos que $\mathcal{M}(S+T)=\mathcal{M}(S)+\mathcal{M}(T)$.
\end{myteo}

%--------------------teorema 3.37
\begin{mydef}[Multiplicación escalar de una matriz]\;\\\\
    El producto de un escalar y una matriz es la matriz que se obtiene multiplicando cada entrada de la matriz por el escalar:
    $$
    \lambda
	\begin{pmatrix}
	    A_{1,1}&\cdots&A_{1,n}\\
	    \vdots&&\vdots\\
	    A_{m,1}&\cdots&A_{m,n}
	\end{pmatrix}
	=
	\begin{pmatrix}
	    \lambda A_{1,1}&\cdots&\lambda A_{1,n}\\
	    \vdots&&\vdots\\
	    \lambda A_{m,1}&\cdots&\lambda A_{m,n}
	\end{pmatrix}
    $$
    En otras palabras, $(\lambda A)_{j,k}=\lambda A_{j,k}$.
\end{mydef}

En el siguiente resultado, se supone que se utilizan las mismas bases para ambas transformaciones lineales $\lambda T$ y $T$.

%--------------------teorema 3.38
\begin{myteo}[La matriz de un escalar por una transformación lineal]\;\\\\
    Suponga $\lambda \in \textbf{F}$ y $T\in \mathcal{L}(V,W)$. Entonces, $\mathcal{M}(\lambda T)=\lambda \mathcal{M}(T)$.\\\\
	Demostración.-\; Sean $v_1,\ldots, v_n$ una base de $V$ y $\lambda \in \textbf{F}$. Además las entradas de $\mathcal{M}(T)$ son $A_{j,k}$ definida por,
	$$Tv_k=A_{1,k}w_1+\cdots+A_{m,k}w_m$$
	Entonces,
	$$
	\begin{array}{rcl}
	(\lambda T)v_k&=& \lambda(Tv_k)\\\\
		&=&\lambda\left(A_{1,k}w_1+\cdots +A_{m,k}w_m\right)\\\\
		&=&\lambda(A_{1,k}w_1)+\cdots +\lambda(A_{m,k}w_m)\\\\
		&=&(\lambda A_{1,k})w_1+\cdots +(\lambda A_{m,k})w_m.
	\end{array}
	$$

	Se deduce que las entradas en la fila $k$, columna $k$ de $M(T)$ con respecto a esta base y el escalar son $\lambda A_{j,k}$. Luego, por la definición 3.38, 
	$$\lambda A_{j,k}=(\lambda A)_{j,k}.$$
	De donde, concluimos que $\mathcal{M}(T) =  \lambda \mathcal{M}(T)$.
\end{myteo}

Dado que la suma y la multiplicación escalar ya se han definido para las matrices, no debería sorprenderte que esté a punto de aparecer un espacio vectorial. Sólo necesitamos un poco de notación para que este nuevo espacio vectorial tenga un nombre.

%--------------------definición 3.39
\begin{mydef}[\boldmath$\textbf{F}^{m,n}$]\;\\\\
    Para $m$ y $n$ enteros positivos, el conjunto de todos las matrices $m\times n$ con entradas en \textbf{F} se denomina $\textbf{F}^{m,n}$.
\end{mydef}

%--------------------teorema 3.40
\begin{myteo}[\boldmath $\dim \textbf{F}^{m,n}=mn$]\,\\\\
    Suponga que $m$ y $n$ son enteros positivos. Con la adición y la multiplicación escalar ya definidas, $\textbf{F}^{m,n}$ es un espacio vectorial con dimensión $mn$.
\end{myteo}

\section{Multiplicación de matrices}

%--------------------definición 3.41
\begin{mydef}[Multiplicación de matrices]\,\\\\
    Suponga $A$ es una matriz $m\times n$ y $C$ es una matriz $n\times p$. Entonces $AC$ es definida por la matriz $m\times p$ cuyas entradas en la fila $j$, columna $k$, están dadas por la siguiente ecuación:
    $$(AC)_{j,k}=\sum_{r=1}^n A_{j,r}C_{r,k}.$$
    En otras palabras, la entrada en la fila $j$, columna $k$ de $AC$ se calcula tomando la fila $j$ de $A$ y la columna $k$ de $C$, multiplicando las entradas correspondientes y luego sumando.
\end{mydef}

Tenga en cuenta que definimos el producto de dos matrices solo cuando el número de columnas de la primera matriz es igual al número de filas de la segunda matriz.\\

La multiplicación de matrices no es conmutativa. En otras palabras, $AC$ no es necesariamente igual a $CA$. Pero si es distributiva y asociativa.

%--------------------lema 1.1
\begin{lema}
    Demostrar que la propiedad distributiva es cierta para la adición y multiplicación de matrices. En otras palabras, suponga que $A,B$ y $C$ son matrices cuyas dimensiones son tal que $A(B+C)$ tengan sentido. Demostrar que $AB+AC$ tiene sentido y $A(B+C)=AB+AC$.\\\\
	Demostración.-\; Ya que, $A(B+C)$ tiene sentido, $B$ y $C$ tienen el mismo tamaño. Además, el número de columnas $n$ de $A$ debe ser igual al número de filas de $B$ y $C$. Todo esto significa que $AB+AC$ tiene sentido.\\
	Para demostrar que $A(B+C)=AB+AC$, sólo usaremos la definición de adición de matrices, la definición de multiplicación de matrices y la propiedad distributiva de la multiplicación escalar en $\textbf{F}$. En particular, sea $a_{j,k}$, $b_{j,k}$y $c_{j,k}$ denotado como las entrada en la fila $j$, columna $k$ de $A,B$ y $C$, respectivamente. La entrada en la fila $j$, columna $k$ de $B+C$ es $b_{j,k}+c_{j,k}$. Por lo que la entrada en la fila $j$, columna $k$ de $A(B+C)$ es 
	$$\sum_{r=1}^n a_{j,r}(b_{r,k}+c_{r,k})=\sum_{r=1}^n a_{j,r}b_{r,k}+\sum_{r=1}^n a_{j,r}c_{r,k}.$$
	Esto es igual a la entrada en la fila $j$, columna $k$ de $AB+AC$. Por lo tanto, $A(B+C)=AB+AC$.
\end{lema}

%-------------------- lema 1.2
\begin{lema}
    Demostrar que la multiplicación de matrices es asociativa. En otras palabras, suponga que $A,B$ y $C$ son matrices cuyas dimensiones son tales que $(AB)C$ tiene sentido. Demostrar que $A(BC)$ tiene sentido y $(AB)C=A(BC)$.\\\\
    Demostración.-\; Sean $A,B$ y $C$ matrices $m\times n$, $n\times p$ y $p\times q$ matrices respectivamente. Entonces $A(BC)$ y $(AB)C$ tienen sentido. Ahora, por definición de multiplicación de matrices (1.41), se tiene para las entradas $(i,j)$ de $A(BC)$ y $(AB)C$,
    $$A(BC)=\sum_{r=1}^n A_{ij}\left(\sum_{k=1}^p B_{jk}C_{kl}\right) = \sum_{j=1}^n\sum_{k=1}^p A_{ij}B_{jk}C_{kl} = \displaystyle\sum_{j=1}^p\sum_{k=1}^n A_{ij}B_{jk}C_{kl}=\sum_{k=1}^p\left(\sum_{j=1}^n A_{ij}B_{jk}\right)C_{kl}=(AB)C$$
    Por lo tanto, $A(BC)=(AB)C$.
\end{lema}

\setcounter{myteo}{42}
%--------------------teorema 3.43
\begin{myteo}[La matriz del producto de transformaciones lineales]\;\\\\
    Si $T\in \mathcal{L}(U,V)$ y $S\in \mathcal{L}(V,W)$, entonces $\mathcal{M}(ST)=\mathcal{M}(S)\mathcal{M}(T)$.\\\\
	Demostración.-\; Suponga, que $v_1,\ldots,v_n$ es una base de $V$ y $w_1,\ldots,w_m$ es una base de $W$ y que $u_1,\ldots,u_p$ es una base de $U$.\\
	Considere la transformación lineal $T: U\to V$ y $S: V\to W$. La composición $ST$ es una transformación lineal de $U$ en $W$. Ahora, suponga $\mathcal{M}(S)=A$ y $\mathcal{M}(T)=C$. Para $1\leq k\leq p$, tenemos
$$
\begin{array}{rcl}
    (ST)u_k &=& \displaystyle S\left(\sum_{r=1}^n C_{r,k}v_r\right)\\\\
	    &=& \displaystyle \sum_{r=1}^n C_{r,k}Sv_r\\\\
	    &=& \displaystyle\sum_{r=1}^n C_{r,k}\sum_{j=1}^m A_{j,r}w_j\\\\
	    &=& \displaystyle\sum_{j=1}^m\left(\sum_{r=1}^n A_{j,r}C_{r,k}\right)w_j.
\end{array}
$$	

Así, $\mathcal{M}(ST)$ es la matriz $m\times n$ cuyas entradas en la fila $j$, columna $k$, son iguales a
$$\sum_{r=1}^n A_{j,r}C_{r,k}.$$

Debemos tomar encuenta que $\mathcal{M}(ST)$ consiste en los escalares necesarios para escribir $(ST)u_k$ cómo una combinación lineal de $w_k$. Ahora bien, por definción de multiplicación de matraces $1.41$ para las entradas (j,k) de $\mathcal{M}(S)\mathcal{M}(T)$ definida como:
$$\mathcal{M}(S)\mathcal{M}(T)=\sum_{r=1}^n A_{j,r}C_{r,k}.$$
Demostramos que,
$$\mathcal{M}(ST)=\mathcal{M}(S)\mathcal{M}(T).$$
\end{myteo}

%--------------------Notación 3.44
\begin{mynotacion}[\boldmath$A_{(j,\cdot)},A_{(\cdot,k)}$]\,\\\\
    Suponga que $A$ es una matriz $m\times n$.
    \begin{itemize}
	\item Si $1\leq j\leq m$, entonces $A_{j,\cdot}$ denota la matriz $1\times n$ que consiste en la fila $j$ de $A$.
	\item Si $1\leq k\leq n$, entonces $A_{\cdot,k}$ denota la matriz $m\times 1$ que consiste en la columna $k$ de $A$.
    \end{itemize}
\end{mynotacion}

\setcounter{myteo}{46}
%--------------------teorema 3.45
\begin{myteo}[La entrada del producto de la matriz es igual a la fila por la columna]\,\\\\
    Suponga que $A$ es una matriz $m\times n$ y $C$ es una matriz $n\times p$. Entonces, 
    $$(AC)_{j,k}=A_{j,\cdot}C_{\cdot,k}$$
    para $1\leq j\leq m$ y $1\leq k\leq p$.\\\\
	Demostración.-\; La demostración se sigue inmediatamente de las definiciones.
\end{myteo}
\vspace{.5cm}
Otra forma de pensar en las Multiplicaciones de matrices:

\setcounter{myteo}{48}
%--------------------teorema 3.46
\begin{myteo}[La columna del producto de la matriz es igual a la matriz por la columna]\,\\\\
    Suponga que $A$ es una matriz $m\times n$ y $C$ es una matriz $n\times p$. Entonces,
    $$(AC)_{\cdot,k}=AC_{\cdot,k}$$
    para $1\leq k\leq p$.\\\\
	Demostración.-\; 
\end{myteo}

\setcounter{mydef}{51}
%--------------------Definición 3.47
\begin{mydef}[Combinación lineal de columnas]\,\\\\
    Suponga que $A$ es una matriz $m\times n$ y $c=\begin{pmatrix}c_1\\\vdots\\c_n\end{pmatrix}$ es una matriz $n\times 1$. Entonces
    $$Ac=c_1A_{\cdot,1}+\cdots+c_nA_{\cdot,n}.$$
    En otras palabras, $Ac$ es una combinación lineal de las columnas de $A$, con los escalares que multiplican las columnas que viene de $c$.
\end{mydef} 


\mysection{Invertibilidad y espacios vectoriales isomorfos}
\vspace{0.5cm}
\subsection*{Transformaciones lineales invertibles}

