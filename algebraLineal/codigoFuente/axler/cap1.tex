\pagenumbering{arabic} 
\chapter{Espacios Vectoriales}

%%%%%%%%%%%%%%%%%%%%%%%%%%%%%%%%%%%%%%%%% 1.B %%%%%%%%%%%%%%%%%%%%%%%%%%%%%%%%%%%%%%%%%%%%%%%
\mysection{\boldmath $R^n$ y $C^n$}

%--------------------- 1.1
    \begin{mydef}[Números complejos] \hfill
	\begin{itemize}
	    \item Un número complejo es un par ordenado $(a,b)$, donde $a,b\in \bf{R}$, pero lo escribimos como $a+bi$.
	    \item El conjunto de todos los números complejos es denotado por $\bf{C}$:
		$$\bf{C}=\lbrace a+bi:a,b\in \bf{R}\rbrace.$$

	    \item La adición y la multiplicación en $\bf{C}$ esta definida por:
		$$(a+bi)+(c+di)=(a+c)+(b+d)i$$
		$$(a+bi)(c+di)=(ac-bd)+(ad+bc)i$$
	\end{itemize}
    \end{mydef}

Si $a\in \bf{R}$, identificamos $a+0i$ con el número real $a$. Por lo que podemos decir que $\bf{R}$ es un subconjunto de $\bf{C}.$

%--------------------- 1.3
\setcounter{mydef}{2}
    \begin{mydef}[Propiedades aritmeticas de los complejos]\;\\
	\begin{itemize}
	    \item \textbf{Conmutatividad:}\;
		$\alpha + \beta = \beta + \alpha\quad \mbox{y}\quad \alpha \beta = \beta \alpha \quad \mbox{para todo }\; \alpha,\beta \in \bf{C};$

	    \item \textbf{Asociatividad:}\;
		$(\alpha + \beta )+\lambda = \alpha + (\beta + \lambda) \;\;\mbox{y}\;\; (\alpha\beta)\lambda =\alpha(\beta\lambda) \quad \mbox{para todo}\; \alpha,\beta,\lambda \in \bf{C};$

	    \item \textbf{Identidad:}\;
		$\lambda +0=\lambda $ y $\lambda 1 = \lambda$ para todo $\lambda \in \bf{C}$

	    \item \textbf{Inverso aditivo:}\;
		    Para cada $\alpha \in \bf{C}$, existe un único $\beta \in \bf{C}$ tal que $\alpha + \beta = 0;$

	    \item \textbf{Inverso multiplicativo:}\;
		    Para cada $\alpha \in \bf{C}$ con $\alpha \neq 0$, existe un único $\beta \in \bf{C}$ tal que $\alpha \beta = 1;$

	    \item \textbf{Propiedad distributiva:}\;
		$\lambda (\alpha + \beta) = \lambda \alpha + \lambda \beta \quad \mbox{para todo}\; \lambda, \alpha, \beta \in \bf{C}$
	\end{itemize}
    \end{mydef}

%--------------------- 1.4
\begin{myteo}
    Demostrar que $\alpha \beta = \beta \alpha$ para todo $\alpha,\beta \in \bf{C}$.\\\\
	Demostración.-\; Por la definición de multiplicación de números complejos se muestra que
	$$\alpha \beta = (a+bi)(c+di) = (ac-bd)+(ad+bc)i.$$
	y
	$$\beta \alpha = (c+di)(a+bi) = (ca-db)+(ad+bc)i.$$
	Las ecuaciones anteriores, la conmutatividad para la suma y la multiplicación y propiedades de números reales muestran que $$\alpha\beta = \beta \alpha.$$
\end{myteo}

%--------------------- Ejemplo aparte --------------------
\begin{ejem}
   Demostrar que $\lambda +0=\lambda $ y $\lambda 1 = \lambda$ para todo $\lambda \in \bf{C}$.\\\\
	Demostración.-\; Sean $\lambda=(a+bi)$ y $0=(0+0i)$, para $a,b\in \bf{R}$. Entonces,
		$$\begin{array}{rcl}
		    \lambda +0 & = & (a+bi) + (0+0i)\\
			       & = & (a+0) + (b+0)i\\
			       & = & a + bi\\
			       & = & \lambda
		\end{array}$$

		Luego sea $1=(1+0i)$, entonces
		$$\begin{array}{rcl}
		    \lambda 1 & = & (a+bi)(1+0i)\\
			      & = & (a1-b0)+(a0+b1)i\\
			      & = & (1a + 1bi)\\
			      & = & (a+bi)\\
			      & = & \lambda
		\end{array}$$
\end{ejem}


%------------------- 1.5 -------------------
    \begin{mydef}\textbf{\boldmath Definición $\qquad -\alpha$, sustracción, $1/\alpha$ división}\;\\\\ 
	Sea $\alpha, \beta \in \bf{C}$
	\begin{itemize}
	    \item Sea $-\alpha$ que denota el inverso aditivo de $\alpha$. Por lo tanto $-\alpha$ es el único número complejo tal que 
		$$\alpha + (-\alpha) = 0.$$

	    \item \textbf{Sustracción} en $\bf{C}$ es definido por:
		$$\beta - \alpha = \beta + (-\alpha).$$

	    \item Para $\alpha\neq 0$, sea $1/\alpha$ denotado por el inverso multiplicativo de $\alpha$. Por lo tanto $1/\alpha$ es el único número complejo tal que
		$$\alpha (1/\alpha)=1.$$

	    \item \textbf{División} en $\bf{C}$ es definido por:
		$$\beta/\alpha = \beta(1/\alpha).$$
	\end{itemize}
    \end{mydef}

%------------------- 1.6 -------------------
    \begin{mynot}\textbf{F}\;\\\\
	$\bf{F}$ significa cualquier $\bf{R}$ o $\bf{C}$.
    \end{mynot}

Los elementos de $\bf{F}$ son llamados \textbf{escalares.}\\\\


%%%%%%%%%%%%%%%%%%%%%%%%%%%%%%%% Listas %%%%%%%%%%%%%%%%%%%%%%%%%%%%%%%%%%%%%%
\subsection*{Listas}

%-------------------- 1.8 ---------------------
\setcounter{mydef}{7}
    \begin{mydef}[Listas, longitud] 
	Supóngase que $n$ es un entero no negativo. Una lista de longitud $n$ es una colección ordenada de $n$ elementos (el cual podría ser números, otras listas, o mas entidades abstractas) separadas por comas y cerradas por paréntesis. Una lista de longitud $n$ se muestra de la siguiente manera:
	$$(x_1,\ldots,x_n)$$
	Dos listas son iguales si y sólo si tienen la misma longitud y los mismos elementos en el mismo orden.
    \end{mydef}
Las listas difieren de los conjuntos de dos maneras: en las listas, el orden importa y las repeticiones tienen significado; en conjuntos, el orden y las repeticiones son irrelevantes.\\\\


%%%%%%%%%%%%%%%%%%%%%%%%%%%%%%%% F^n %%%%%%%%%%%%%%%%%%%%%%%%%%%%%%%%%%%%%%
\subsection*{\boldmath $F^n$}

\setcounter{mydef}{9}
%------------------- 1.10 -------------------
    \begin{mydef} 
	$\bf{F}^n$ es el conjunto de todas las listas de longitud $n$ de elementos de $\bf{F}$
	$$\bf{F}^n = \lbrace (x_1,\ldots, x_n)\; : \; x_j \in \bf{F} \; \mbox{para cada}\; j = 1,\ldots, n\rbrace.$$
	Para $(x_1,\ldots, x_n)\in \bf{F}$ y $j\in \lbrace 1, \ldots, n\rbrace,$ decimos que $x_j$ es la j-enesima coordenada de $(x_1,\ldots, x_n)$.
    \end{mydef}

%------------------- 1.12 -------------------
\setcounter{mydef}{11}
    \begin{mydef}[\boldmath Adición en $\bf{F}^n$]
	La adición en $\bf{F}^n$ es definido añadiendo las correspondientes coordenadas: 
	$$(x_1,\ldots,x_n)+(y_1,\ldots, y_n)=(x_1+y_1,\ldots, x_n + y_n)$$
    \end{mydef}

%------------------- 1.13 -------------------
\begin{myteo}[Conmutatividad para la adición en $\bf{F}^n$]
    Si $x,y \in \bf{F}^n,$ entonces $x+y=y+x$.\\\\
    Demostración.-\; Suponga $x=(x_1,\ldots, x_n)$ y $y=(y_1,\ldots, y_n)$. Entonces por la definción de adición e $\bf{F}^n$,
    $$\begin{array}{rcl}
	x+y&=&(x_1,\ldots,x_n)+(y_1,\ldots,y_n)\\
	   &=&(x_1+y_1,\ldots, x_n + y_n)\\
	   &=&(y_1+x_1,\ldots, y_n x_n)\\
	   &=&(y_1,\ldots,y_n)+(x_1,\ldots,x_n)\\
	   &=&y+x\\
    \end{array}$$
    donde la segunda y cuarta igualdades anteriores se cumplen debido a la definición de suma en $\bf{F}^n$ y la tercera igualdad se cumple debido a la conmutatividad habitual de la suma en $\bf{F}$.
\end{myteo}
\vspace{0.3cm}

    Si $x \in \bf{F}^n$, entonces hacer que $x$ sea igual a $(x_1,\ldots,x_n)$ es una buena notación, como se muestra en la demostración anterior.

%------------------- 1.14 -------------------
    \begin{mydef}[$0$]
	Sea $0$ la lista de longitud n cuyas coordenadas son todas 0: 
	$$0=(0,\ldots,0)$$
    \end{mydef}

%------------------- 1.16 -------------------
\setcounter{mydef}{15}
    \begin{mydef}[\boldmath Inverso aditivo en $\bf{F}^n$]
	Para $x\in \bf{F}^n$, el inverso aditivo de $x$, denota por $-x$, es el vector $-x\in \bf{F}^n$ tal que $$x+(-x) =0$$
	En otras palabras, si $x=(x_1,\ldots,x_n)$, entonces $-x=(-x_1,\ldots,-x_n).$
    \end{mydef}

%------------------- 1.17 -------------------
    \begin{mydef}[Multiplicación scalar en $\bf{F}^n$]
	El producto de un número $\lambda$ y un vector en $\bf{F}^n$ es calculado por la multiplicación de cada coordenada del vector por $\lambda$:
	$$\lambda(x_1,\ldots,x_n) = (\lambda x_1,\ldots, \lambda x_n)$$
	donde $\lambda \in \bf{F}$ y $(x_1,\ldots,x_n)\in \bf{F}^n$.
    \end{mydef}

%%%%%%%%%%%%%%%%%%%%%%%%%%%%%%%%%% Digresión sobre campos %%%%%%%%%%%%%%%%%%%%%%%%%%%%%%%%%%%%%%
\subsection{Digresión sobre campos}
Un campo es un conjunto que contiene al menos dos elementos distintos llamados $0$ y $1$, junto con operaciones de suma y multiplicación que satisfacen todas las propiedades enumeradas en 1.3. Por lo tanto, $\bf{R}$ y $\bf{C}$ son campos, como lo es el conjunto de números racionales junto con las operaciones habituales de suma y multiplicación. Otro ejemplo de campo es el conjunto $\left\{0,1\right\}$ con las operaciones habituales de suma y multiplicación excepto que $1 + 1$ se define como igual a $0$.

%%%%%%%%%%%%%%%%%%%%%%%%%%%%%%%%%% Ejercicios 1.A %%%%%%%%%%%%%%%%%%%%%%%%%%%%%%%%%%%%%%
\section*{Ejercicios 1.A}

\begin{enumerate}[\bfseries 1.]

    %-------------------- 1. --------------------
    \item Supongase $a$ y $b$ números reales, no ambos $0$. Encuentre números reales $c$ y $d$ tales que
    $$\dfrac{1}{(a+bi)}=c+di.$$\\
	Respuesta.-\; Supongamos $\alpha=a+bi$ y $\beta=c+di$ con $a,b,c,d\in \bf{R}$. Entonces,
	$$\alpha\beta = (a+bi)(c+di) = (ac-bd)+(ad+bc)i=(1+0i)$$
	Luego, por el hecho de que dos números complejos son iguales si sus correspondientes partes reales e imaginarias son iguales, entonces
	$$\left\{\begin{array}{rcl}
	    ac-db & = & 1\\
	    ad+bc & = & 0
	\end{array}\right.$$
	De la segunda ecuación tenemos, $d=-\dfrac{bc}{a}$ y reemplazando en la ecuación 1, 
	$$c=\dfrac{a}{a^2+b^2}.$$
	Luego,
	$$d=-\dfrac{bc}{a}=\dfrac{b\cdot \frac{a}{a^2+b^2}}{a}=-\dfrac{b}{a^2+b^2}.$$\\

    %-------------------- 2. --------------------
    \item Demostrar que 
    $$\dfrac{-1+\sqrt{3}i}{2}$$
    es una raíz cúbica de $1$ (significa que su cubo es igual a $1$).\\\\
	Demostración.-\; Algebraicamente queremos demostrar que,
	$$\dfrac{-1+\sqrt{3}i}{2}=\sqrt[3]{1} \quad \mbox{o}\quad \left(\dfrac{-1+\sqrt{3}i}{2}\right)^3=1$$\\
	Por las propiedades de números complejos, sabemos que
	$$\dfrac{-1+\sqrt{3}i}{2}=\dfrac{-1}{2}+\dfrac{\sqrt{3}i}{2}.$$
	Luego, por el binomio al cubo podemos deducir que
	$$\begin{array}{rcl}
	    \left(\dfrac{-1}{2}+\dfrac{\sqrt{3}i}{2}\right)^3 &=&\left(\dfrac{-1}{2}\right)^3 + 3\cdot \left(\dfrac{-1}{2}\right)^2 \dfrac{\sqrt{3}i}{2} + 3\cdot \dfrac{-1}{2}\left(\dfrac{\sqrt{3}i}{2}\right)^2+\left(\dfrac{\sqrt{3}i}{2}\right)^3\\\\
							      &=&\dfrac{-1}{8}+\dfrac{3\sqrt{3}i}{8}+\dfrac{-3(\sqrt{3})^2(\sqrt{-1})^2}{8}+\dfrac{(\sqrt{3})^2\sqrt{3}(\sqrt{-1})^2\sqrt{-1}}{8}\\\\
							      &=&-\dfrac{1}{8}+\dfrac{3\sqrt{3}i}{8}+\dfrac{9}{8}-\dfrac{3\sqrt{3}i}{8}\\\\
							      &=&1.
	\end{array}$$
	\vspace{0.5cm}

    %-------------------- 3. --------------------
    \item Encuentre dos raíces cuadradas distintas de $i$.\\\\
	Respuesta.-\; Queremos encontrar raíces distintas de $i$ tal que
	$$x^2=i.$$
	Supongamos que $x=a+bi$ para $a,b\in \bf{R}$. Resolvemos la ecuación cuadrática, se tiene 
	$$x^2=(a+bi)^2=(a^2-b^2)+(ab+ba)i=(a^2-b^2)+2abi.$$
	Luego, ya que $x^2=i=0+1i$, entonces
	$$(a^2-b^2)+2abi=(0+1i)\quad \Rightarrow \quad \left\{\begin{array}{rcl}
		a^2-b^2&=&0\\
		2ab&=&1
	    \end{array}\right.$$
	para $a$ y $b$ números reales. Resolvamos este sistema de ecuaciones, se tiene 
	$$\left\{\begin{array}{rcl}
		b&=&\pm \sqrt{\dfrac{1}{2}}\\\\
		a&=&\pm \sqrt{\dfrac{1}{2}}
	\end{array}\right.$$
	Ya que, $x^2=(a+bi)^2=(a^2-b^2)-2abi=i$. Entonces, 
	$$\begin{array}{rcl}
	    \left[\left(\dfrac{1}{\sqrt{2}}\right)^2-\left(\dfrac{1}{\sqrt{2}}\right)^2\right]+2\cdot\dfrac{1}{\sqrt{2}}\cdot\dfrac{i}{\sqrt{2}}i&=&i\\\\
	    \left[\left(-\dfrac{1}{\sqrt{2}}\right)^2-\left(-\dfrac{1}{\sqrt{2}}\right)^2\right]+2\left(-\dfrac{1}{\sqrt{2}}\right)\left(-\cdot\dfrac{i}{\sqrt{2}}\right)i&=&i\\\\
	    \left[\left(-\dfrac{1}{\sqrt{2}}\right)^2-\left(\dfrac{1}{\sqrt{2}}\right)^2\right]+2\left(-\dfrac{1}{\sqrt{2}}\right)\cdot\dfrac{i}{\sqrt{2}}i&\neq&i\\\\
	    \left[\left(\dfrac{1}{\sqrt{2}}\right)^2-\left(-\dfrac{1}{\sqrt{2}}\right)^2\right]+2\cdot \dfrac{1}{\sqrt{2}}\left(-\dfrac{i}{\sqrt{2}}\right)i&\neq&i\\\\
	\end{array}$$
	De las que solo satisface $x^2=i$ las dos primeras ecuaciones. Por lo tanto, las dos raíces distintas de $i$ son:
	$$\dfrac{1}{\sqrt{2}}+\dfrac{1}{\sqrt{2}}i\qquad \mbox{y}\qquad -\dfrac{1}{\sqrt{2}}-\dfrac{1}{\sqrt{2}}i.$$\\



    %-------------------- 4. --------------------
    \item Demostrar que $\alpha+\beta=\beta+\alpha$ para todo $\alpha,\beta\in \bf{C}$.\\\\
	Demostración.-\;  Supóngase $\alpha=a+bi$ y $\beta=c+di$, donde $a,b,c,d \in \bf{R}$. Entonces por definición de suma de números complejos se muestra que
	$$\alpha + \beta = (a+bi) + (c+di) = (a+c) + (b+d)i = (c+a) + (d+b)i = \beta + \alpha.$$\\


    %-------------------- 5. --------------------
    \item Demuestre que $(\alpha+\beta)+\lambda=\alpha+(\beta+\lambda)$ para todo $\alpha,\beta,\lambda\in \bf{C}$.\\\\
	Demostración.-\; Sean,
	$$\begin{array}{rcl}
	    \alpha & = & a+bi\\
	    \beta & = & c+di\\
	    \lambda & = & e+fi
	\end{array}$$
	para $a,b,c,d,e,f \in \bf{R}.$ Entonces,
	$$\begin{array}{rcl}
	    (\alpha + \beta )+\lambda & = & \left[(a+bi)+(c+di)\right]+(e+fi)\\
				      & = & \left[(a+c)+(b+d)i\right] + (e+fi)\\
				      & = & \left[(a+c)+e\right]+\left[(b+d)+f\right]i\\
				      & = & \left[a+(c+e)\right]+\left[b+(d+f)\right]i\\
				      & = & (a+bi) + \left[(c+e)+(d+e)i\right]\\
				      & = & (a+bi) + \left[(c+di)+(e+di)\right]\\
				      & = & \alpha + (\beta + \lambda).
	\end{array}$$
	\vspace{0.5cm}

    %-------------------- 6. --------------------
    \item Demostrar que $(\alpha\beta)\lambda = \alpha(\beta\lambda)$ para todo $\alpha,\beta,\lambda\in \bf{C}$.\\\\
	Demostración.-\; Sean,
	$$\begin{array}{rcl}
	    \alpha & = & a+bi\\
	    \beta & = & c+di\\
	    \lambda & = & e+fi
	\end{array}$$
	para $a,b,c,d,e,f \in \bf{R}.$ Entonces,
	$$\begin{array}{rcl}
	    (\alpha\beta)\lambda &=& \left[(a+bi)(c+di)\right](e+fi)\\
				 &=&\left[(ac-bd)+(ad+bc)i\right](e+fi)\\
				 &=&\left[(ac-bd)e-(ad+bc)f\right]+\left[(ac-bd)f+(ad+bc)e\right]i\\
				 &=&\left[(ac)e-(bd)e-(ad)f-(bc)f\right]+\left[(ac)f-(bd)f+(ad)e+(bc)e\right]i\\
				 &=&\left[a(ce)-a(df)-b(cf)-b(de)\right]+\left[a(cf)+a(de)+b(ce)-b(df)\right]i\\
				 &=&\left[a(ce-df)-b(cf+de)\right]+\left[a(cf+de)+b(ce-df)\right]i\\
				 &=&(a+bi)\left[(ce-df)+(cf+de)i\right]\\
				 &=&(a+bi)\left[(c+di)(e+fi)\right]\\
				 &=&\alpha(\beta\lambda).
	\end{array}$$
	\vspace{0.5cm}

    %-------------------- 7. --------------------
    \item Demostrar que para cada $\alpha \in \bf{C}$, existe un único $\beta\in \bf{C}$ tal que $\alpha+\beta=0$.\\\\
	Demostración.-\; Sean $\alpha=(a+bi)$ y $\beta=(c+di)$ para $a,b,c,d\in \bf{R}$. Por las propiedades de números reales, $a$ y $b$ tienen inversa, las llamaremos $c$ y $d$ respectivamente tales que $a+c=0$ y $b+d=0$. Por lo tanto,
	$$\begin{array}{rcl}
	    \alpha+\beta &=& (a+bi)+(c+di)\\
			 &=& (a+c)+(b+d)i\\
			 &=& 0 + 0i\\
			 &=& 0
	\end{array}$$

	Ahora demostremos la unicidad. Sean $\beta$ y $\beta'$ con $c',d'\in \bf{R}$ tales que $\alpha+\beta=0$ y $\alpha+\beta'=0$. Igualando estas ecuaciones se tiene,
	$$\begin{array}{rcl}
	    \alpha+\beta&=&\alpha+\beta'\\
	    (a+bi)+(c+di)&=&(a+bi)+(c'+d'i)\\
	    (a+c)+(b+d)i&=&(a+c')+(b+d')i\\
	\end{array}$$

	Supongamos $c=c'$ y $d=d'$. Entonces,
	$$(b+d)i=(b+d')i$$
	Dada que la igualdad $\alpha+\beta=\alpha+\beta'$ se cumple siempre que $c=c'$ y $d=d'$, concluimos que,
	$$(c+di)=(c'+d'i)\quad \Rightarrow \quad \beta=\beta'.$$\\

    %-------------------- 8. --------------------
    \item Demostrar que para cada $\alpha \in \bf{C}$ con $\alpha\neq 0$, existe un único $\beta\in \bf{C}$ tal que $\alpha\beta=1$.\\\\
	Demostración.-\; Sean $\alpha=a+bi$ y $\beta=c+di$ con $a,b,c,d\in \bf{R}$. Entonces,
	$$\alpha\beta = (a+bi)(c+di) = (ac-bd)+(ad+bc)i=(1+0i)$$
	Supongamos que:
	$$\left\{\begin{array}{rcl}
	    ac-db & = & 1\\
	    ad+bc & = & 0
	\end{array}\right.$$
	De la segunda ecuación tenemos, $d=-\dfrac{bc}{a}$ y reemplazando en la ecuación 1, 
	$$c=\dfrac{a}{a^2+b^2}.$$
	Luego,
	$$d=-\dfrac{bc}{a}=\dfrac{b\cdot \frac{a}{a^2+b^2}}{a}=-\dfrac{b}{a^2+b^2}.$$
	Por lo tanto, existe un $\beta$ tal que,
	$$\beta=\dfrac{a}{a^2+b^2}-\dfrac{b}{a^2+b^2}i=\dfrac{a}{a^2+b^2}+\dfrac{-bi}{a^2+b^2}=\dfrac{a-bi}{a^2+b^2}=\dfrac{1}{a+bi}.$$\\
	Ahora, demostremos la unicidad. Sean $\beta=c+di$ y $\beta'=c'd'i$ tales que, $(a+bi)(c+di)=1$ y $(a+bi)(c'+d'i)=1$. Entonces, por las propiedades de conmutatividad y asociatividad,
	$$c+di=(c+di)\cdot 1 = (c+di)\cdot(a+bi)(c'+d'i)=\left[(a+bi)(c+di)\right](c'+d'i)=c'+d'i.$$
	Por lo que queda demostrado que $\beta=\beta'.$ Y así concluimos que $\alpha$ tiene inverso multiplicativo.\\\\



    %-------------------- 9. --------------------
    \item Demostrar que $\alpha(\beta+\gamma)=\alpha\beta+\alpha\gamma$ para todo $\alpha,\beta,\gamma\in \bf{C}$.\\\\
	Demostración.-\; Sean $\alpha=a+bi$, $\beta=c+di$ y $\gamma=e+fi$ con $a,b,c,d,e,f\in \bf{R}$. Entonces,
$$\begin{array}{rcl}
    \alpha(\beta+\gamma) & = & (a+bi)\left[(c+e)+(d+f)i\right]\\
			 & = & \left[a(c+e)-b(d+f)\right]+\left[a(d+f)+b(c+e)\right]i\\
			 & = & \left[ac+ae-bd-bf\right]+\left[ad+af+bc+be\right]i\\
			 & = & ac+ae-bd-bf+adi+afi+bci+bei\\
			 & = & \left[(ac-bd)+(ad+bc)i\right] + \left[(ae-bf)+(af+be)i\right]\\
			 & = & (a+bi)(c+di)+(a+bi)(e+fi)\\
			 & = & \alpha\beta + \alpha\gamma.
\end{array}$$
\vspace{0.5cm}

    %-------------------- 10. --------------------
    \item Encuentre $x\in \bf{R}^4$ tal que 
    $$(4,-3,1,7)+2x=(5,9,-6,8).$$\\
	Respuesta.-\; Sea $x = (x_1,x_2,x_3,x_4)$ con $x_1,x_2,x_3,x_4\in \bf{R}$. Por definición de adición y multiplicación por un escalar en $\bf{F}$ propiedades  Entonces,
	$$\begin{array}{rcl}
	    (4,-3,1,7)+2(x_1,x_2,x_3,x_4)&=&(5,9,-6,8)\\
	    (4,-3,1,7)+(2x_1,2x_2,2x_3,2x_4)&=&(5,9,-6,8)\\
	    (4+2x_1,-3+2x_2,1+2x_3,7+2x_4)&=&(5,9,-6,8)\\
	\end{array}$$

	Por la igualdad dada, podemos formar un sistema de ecuaciones como sigue:
	$$\left\{\begin{array}{rcccr}
	    4&+&2x_1&=&5\\
	    -3&+&2x_2&=&9\\
	    1&+&2x_3&=&-6\\
	    7&+&2x_4&=&8	
	\end{array}\right.$$
	Resolviendo, obtenemos
	$$x=(x_1,x_2,x_3,x_4)=\left(\dfrac{1}{2},2,-\dfrac{7}{2},\dfrac{1}{2}\right).$$\\


    %-------------------- 11. --------------------
    \item Explique porque no existe $\lambda\in \bf{C}$ tal que
    $$\lambda(2-3i,5+4i,-6+7i)=(12-5i,7+22i,-32-9i).$$\\\\
	Respuesta.-\; Sean $\lambda=a+bi, \; a,b\in \bf{R}$ y $x=(12-5i,7+22i,-32-9i)$. Entonces, por definición de multiplicación por un escalar en $\bf{C}$ y las propiedades de número complejos,
	$$\begin{array}{rcl}
	    \lambda(2-3i,5+4i,-6+7i) & = & x\\
	    \left[\lambda(2-3i),\lambda(5+4i),\lambda(-6+7i)\right] & = & x\\
	    \left[(a+bi)(2-3i),(a+bi)(5+4i),(a+bi)(-6+7i)\right] & = & x\\
	    \left\{\left[2a-(-3)b\right]+\left(-(3)a+2b\right]i, (5a-4b)+(4a+5b)i,\left[(-6)a-7b\right]+\left[7a+(-6)b\right]i\right\}& = & x\\
	    \left[\left(2a+3b\right)+\left(-3a+2b\right)i, (5a-4b)+(4a+5b)i,\left(-6a-7b\right)+\left(7a-6b\right)i\right]& = & x\\
	\end{array}$$
	Dado que,
	    $$\left[\left(2a+3b\right)+\left(-3a+2b\right)i, (5a-4b)+(4a+5b)i,\left(-6a-7b\right)+\left(7a-6b\right)i\right] = (12-5i,7+22i,-32-9i)$$
	Entonces, por la igualdad tenemos el sistema de ecuaciones:
	$$\left\{\begin{array}{rcccr}
		(2a+3b)&+&(-3a+2b)i&=&12-5i\\
		(5a-4b)&+&(4a+5b)i&=&7+22i\\
		(-6a-7b)&+&(7a-6b)i&=&-32-9i
	\end{array}\right.$$
	Que implica,
	$$\left\{\begin{array}{rcr}
	    2a+3b&=&12\\
	    -3a+2b&=&-5\\
	    5a-4b&=&7\\
	    4a+5b&=&22\\
	    -6a-7b&=&-32\\
	    7a-6b&=&-9
	\end{array}\right.$$
	Resolviendo este último sistema,
	$$3(2a+3b)+2(-3a+2b)=3\cdot 12 + 2(-5)\quad \Rightarrow \quad b=\dfrac{26}{11}.$$
	De donde $b$ estará dado por:
	$$2a+3b=12\quad \Rightarrow \quad 2a+3\cdot \dfrac{26}{11}=12 \quad \Rightarrow \quad a=\dfrac{27}{11}.$$
	Sabemos que $a$ y $b$ debe satisfacer cada una de las ecuaciones del último sistema, Por ejemplo:
	$$\begin{array}{rcl}
	    5a-4b&=&7\\
	    5\cdot \dfrac{27}{11}-4\cdot \dfrac{26}{11}&=&7\\
	    \dfrac{21}{11}=7.
	\end{array}$$
	Claramente, $\dfrac{21}{11}\neq7$, lo que significa que no podemos encontrar números $a$ y $b$ para satisfacer las ecuaciones del último sistema.\\\\


    %-------------------- 12. --------------------
    \item Demostrar que $(x+y)+z=x+(y+z)$ para todo $x,y,z\in \bf{F}^n.$\\\\
	Demostración.-\; Sean $x=(x_1,\ldots,x_n)$, $y=(y_1,\ldots , y_n)$ y $z=(z_1,\ldots , z_n)$. Entonces por definición de adición en $\bf{F}^n$ y los axiomas de números reales,
	$$\begin{array}{rcl}
	    (x+y)+z & = & \left[(x_1+\ldots+x_n)+(y_1+\ldots + y_n)\right]+(z_1,\ldots , z_n)\\
		    & = & (x_1+y_1,\ldots ,x_n+y_n)+(z_1+\ldots, z_n)\\
		    & = & \left[(x_1+y_1)+y_1,\ldots ,(x_n+y_n)+z_n\right]\\
		    & = & \left[x_1+(y_1+y_1),\ldots ,x_n+(y_n+z_n)\right]\\
		    & = & (x_1,\ldots,x_n)+\left[(y_1+z_1,\ldots, y_n+z_n)\right]\\
		    & = & (x_1,\ldots,x_n)+\left[(y_1,\ldots, y_n)+(z_1,\ldots,z_n)\right]\\
		    & = & x+(y+z).
	\end{array}$$
	\vspace{.5cm}

    %-------------------- 13. --------------------
    \item Demostrar que $(ab)x=a(bx)$ para todo $x\in \bf{F}^n$ y todo $a,b\in \bf{F}$.\\\\
	Demostración.-\; Sean $x=(x_1,\ldots,x_n)$ y $a,b\in \bf{F}$. Entonces por definición de multiplicación escalar en $\bf{F}^n$ y los axiomas de números reales,
	$$\begin{array}{rcl}
	    (ab)x&=&(ab)(x_1,\ldots,x_n)\\
		 &=&\left[(ab)x_1,\ldots,(ab)x_n)\right]\\
		 &=&\left[a(bx_1),\ldots,a(bx_n)\right]\\
		 &=&a\left[(bx_1),\ldots,(bx_n)\right]\\
		 &=&a\left[b(x_1,\ldots,x_n)\right]\\
		 &=&a(bx)
	\end{array}$$
	\vspace{.5cm}

    %-------------------- 14. --------------------
    \item Demostrar que $1x=x$ para todo $x\in \bf{F}^n$.\\\\
	Demostración.-\; Sea $x=(x_1,\ldots,x_n)$. Entonces, por definición de multiplicación escalar en $\bf{F}^n$ y por el hecho de que $1a=a$ para $a\in \bf{F}$
	$$\begin{array}{rcl}
	    1x&=&1(x_1,\ldots,x_n)\\
		 &=&\left[1x_1,\ldots,1x_n\right]\\
		 &=&\left[x_1,\ldots,x_n\right]\\
		 &=&x.
	\end{array}$$
	\vspace{.5cm}

    %-------------------- 15. --------------------
    \item Demostrar que $\lambda (x+y)=\lambda x+\lambda y$ para todo $\lambda\in \bf{F}$ y todo $x,y\in \bf{F}^n$.\\\\
	Demostración.-\; Sean $x=(x_1,\ldots,x_n)$, $y=(y_1,\ldots , y_n)$ y $\lambda\in \bf{F}$. Entonces, por definición de adición y multiplicación escalar en $\bf{F}^n$ 
	$$\begin{array}{rcl}
	    \lambda (x+y)&=&\lambda\left[(x_1,\ldots,x_n)+(y_1,\ldots , y_n)\right]\\
		 &=&\lambda\left[(x_1+y_1,\ldots ,x_n+y_n)\right]\\
		 &=&\left[\lambda(x_1+y_1),\ldots ,\lambda(x_n+y_n)\right]\\
		 &=&\left[\lambda x_1+\lambda y_1,\ldots ,\lambda x_n+\lambda y_n\right]\\
		 &=&\left[\lambda x_1,\ldots ,\lambda x_n\right]+\left[\lambda y_1,\ldots ,\lambda y_n\right]\\
		 &=&\lambda x+\lambda y.
	\end{array}$$
	\vspace{.5cm}

    %-------------------- 16. --------------------
    \item Demostrar que $(a+b)x=ax+bx$  para todo $a,b\in \bf{F}$ y todo $x\in \bf{F}^n$.\\\\
	Demostración.-\; Sea $x=(x_1,\ldots,x_n)$ y $a,b\in \bf{F}$. Entonces, por definición de multiplicación escalar en $\bf{F}^n$
	$$\begin{array}{rcl}
	    (a+b)x&=&(a+b)(x_1,\ldots,x_n)\\
		 &=&\left[(a+b)x_1,\ldots,(a+b)x_n\right]\\
		 &=&\left[(ax_1+bx_1),\ldots,(ax_n+bx_n)\right]\\
		 &=&\left[ax_1,\ldots,ax_n\right]+\left[bx_1,\ldots,bx_n\right]\\
		 &=&ax+bx.
	\end{array}$$
	\vspace{.5cm}

\end{enumerate}

%%%%%%%%%%%%%%%%%%%%%%%%%%%%%%% Definición de espacio vectorial %%%%%%%%%%%%%%%%%%%%%%%%%%%%%%%
\mysection{Definición de espacio vectorial}

La motivación para la definición de un espacio vectorial proviene de las propiedades de la suma y la multiplicación escalar en $\bf{F}^n$: la suma es conmutativa, asociativa y tiene una identidad. Todo elemento tiene un inverso aditivo. La multiplicación escalar es asociativa. La multiplicación escalar por 1 actúa como se esperaba. La suma y la multiplicación escalar están conectadas por propiedades distributivas. Definiremos un espacio vectorial como un conjunto V con una suma y una multiplicación escalar en V que satisfagan las propiedades del párrafo anterior.

%-------------------- 1.18 --------------------
    \begin{mydef}[Adición y multiplicación escalar]\hfill
	\begin{itemize}
	    \item Una adición en un conjunto $V$ es una función que asigna un elemento $u+v\in V$ para cada par de elementos $u,v\in V$.
	    \item Una multiplicación escalar en un conjunto $V$ es una función que asigna un elemento $\lambda  v\in V$ para cada $\lambda \in \bf{F}$ y cada $v\in V$.
	\end{itemize}
    \end{mydef}

%-------------------- 1.19 ---------------------
    \begin{mydef}[Espacio vectorial] Un espacio vectorial es un conjunto $V$ junto con una suma en $V$ y una multiplicación escalar en $V$ tal que se cumplen las siguientes propiedades: 
	\begin{itemize}
	    \item \textbf{Conmutatividad}
		$$u+v=v+u\; \mbox{para todo}\; u,v\in V;$$
	    \item \textbf{Asociatividad}
		$$(u+v)+w=u+(v+w)\; \mbox{y}\; (ab)v=a(bv)\; \mbox{para todo}\; u,v,w \in V\; \mbox{y todo}\; a,b\in \bf{F};$$
	    \item \textbf{Identidad aditiva}
		\begin{center}
		    Existe un elemento $0\in V$ tal que $v+0=v$ para todo $v\in V$;
		\end{center}

	    \item \textbf{Inverso aditivo}
		\begin{center}
		    Para cada $v\in V$, existe $w \in V$ tal que $v+w=0$;
		\end{center}

	    \item \textbf{Identidad Multiplicativa}
		$$1v=v\; \mbox{para todo}\; v\in V$$

	    \item \textbf{Propiedad distributiva}
		\begin{center}
		    $a(u+v)=au+av$ y $(a+b)v=av+bv$ para todo $a,b\in \bf{F}$ y todo $u,v \in V$.
		\end{center}
	\end{itemize}
    \end{mydef}

%-------------------- 1.20 --------------------
    \begin{mydef}[Vector, punto]
	Elementos de un espacio vectorial son llamados vectores o puntos.
    \end{mydef}

%-------------------- 1.21 --------------------
    \begin{mydef}[Espacio vectorial real, espacio vectorial complejo]\hfill
	\begin{itemize}
	    \item Un espacio vectorial sobre $\bf{R}$ es llamado un espacio vectorial real.
	    \item Un espacio vectorial sobre $\bf{C}$ es llamado un espacio vectorial complejo.
	\end{itemize}
    \end{mydef}

    El espacio vectorial más simple contiene sólo un punto. En otras palabras, $\left\{0\right\}$ es un espacio vectorial.

% -------------------- 1.23 --------------------
\setcounter{mynot}{22}
    \begin{mynot}[\boldmath $\bf{F}^S$]\hfill
	\begin{itemize}
	    \item Si $S$ es un conjunto, entonces $\bf{F}^S$ se denota como el conjunto de funciones de $S$ para $\bf{F}$.
	    \item Para $f,g\in \bf{F}^S$, la suma $f+g\in \bf{F}^S$ es la función definida por $$(f+g)(x)=f(x)+g(x)$$ para todo $x\in S$
	    \item Para $\lambda \in \bf{F}$ y $f\in \bf{F}^S$, el producto $\lambda f \in \bf{F}^S$ es una función definida por $$(\lambda f)(x)=\lambda f(x)$$
		para todo $x\in S$.
	\end{itemize}
    \end{mynot}

La definición de un espacio vectorial requiere que tenga una identidad aditiva. El siguiente resultado establece que esta identidad es única.\\

%-------------------- 1.25 --------------------
\setcounter{myteo}{24}
\begin{myteo}[Identidad aditiva única]
    Un espacio vectorial tiene una única identidad aditiva.\\\\
	Demostración.-\; Supóngase $0$ y $0^{'}$ ambos identidades aditivas para algún espacio vectorial $V$ entonces,
	$$0^{'}=0^{'}+0=0+0^{'}=0$$
	donde se cumple la primera igualdad porque $0$ es una identidad aditiva, la segunda igualdad viene de la conmutatividad, y la tercera igualdad se cumple porque $0^{'}$ es una identidad aditiva. Por lo tanto $0^{'}=0$ y queda probado que $V$ tiene una sola identidad aditiva.
\end{myteo}
\vspace{0.3cm}

Cada elemento $v$ en un espacio vectorial tiene un inverso aditivo, un elemento $w$ en el espacio vectorial tal que $v + w = 0$. El siguiente resultado muestra que cada elemento en un espacio vectorial tiene solo un inverso aditivo. \\

% -------------------- 1.26 --------------------
\begin{myteo}
    Cada elemento en un espacio vectorial tiene un único inverso aditivo.\\\\
	Demostración.-\; Supóngase que $V$ es un espacio vectorial. Sea $v\in V$,  $w$ y $w^{'}$ inversos aditivos de $v$. Entonces
	$$w=w+0=w+(v+w^{'}) = (w+v)+w^{'}=0+w^{'}=w^{'}$$
	Así $w=w^{'}$. 
\end{myteo}

%-------------------- 1.27 --------------------
\begin{mynot}[$\bf{-v,w-v}$]\,\\\\
	Sea $v,w \in V$. Entonces
	\begin{itemize}
	    \item $-v$ se denota como el inverso aditivo de $v$;
	    \item $w-v$ es definido como $w+(-v)$.
	\end{itemize}
    \end{mynot}

%-------------------- 1.28 --------------------
    \begin{mynot}[$\bf{V}$]\;\\\\
	Por el resto del libro, $V$ se define como el espacio vectorial sobre $F$.
    \end{mynot}

% -------------------- 1.29 --------------------
\begin{myteo}[El número 0 por un vector]
    $0v=0$ para cada $v\in V$.\\\\
	Demostración.-\; Para $v\in V$, tenemos
	$$0v=(0+0)v=0v+0v$$
	Luego añadiendo el inverso aditivo de $0v$ para ambos lados de la ecuación de arriba tenemos $0=0v$. 
\end{myteo}
\vspace{0.3cm}

Ahora  establecemos que el producto de cualquier escalar y el vector $0$ es igual al vector $0$.\\

% -------------------- 1.30 --------------------
\begin{myteo}[Un número por el vector $\bf{0}$]
    $a0=0$ para cada $a\in \bf{F}$.\\\\
	Demostración.-\; Para $a\in \bf{F}$, tenemos 
	$$a0=a(0+0)=a0+a0$$
	Luego añadiendo el inverso aditivo de $a0$ para ambos lados de la ecuación de arriba tenemos $0=a0$. 
\end{myteo}
\vspace{0.3cm}

Ahora mostramos que si un elemento de $V$ se multiplica por el escalar $-1$, entonces el resultado es el inverso aditivo del elemento de $V$.\\

% -------------------- 1.31 --------------------
\begin{myteo}[El número $\bf{-1}$ por un vector]
    $(-1)v=-v$ para cada $v\in V$.\\\\
    	Demostración.-\; Para $v\in V$, tenemos
	$$v+(-1)v=1v+(-1)v=[1+(-1)]v=0v=0.$$
	Esta ecuación nos dice que $(-1)v$, cuando se suma a $v$ da $0$. Así $(-1)v$ es el inverso aditivo de $v$. 
\end{myteo}

%%%%%%%%%%%%%%%%%%%%%%%%%%%%%%%%%%%%% Ejercicios 1.B %%%%%%%%%%%%%%%%%%%%%%%%%%%%%%%%%%%%%
\section*{Ejercicio 1.B}

\begin{enumerate}[\bfseries 1.]

    %-------------------- 1.
    \item Demostrar que $-(-v)=v$ para cada $v\in V$.\\\\
	Demostración.-\; Ya que $(-1)v=-v$ para cada $v\in V$ (apartado 1.31, Axler, Linear algebra), $(-a)(-a)=a, \forall a \in \bf{R}$ y la identidad multiplicativa de espacio vectorial tenemos que 
	$$\begin{array}{rcl}
	v &=& 1v \\
	&=& (-1)(-1)v \\
	&=& (-1)(-v) \\
	&=& -(-v)
	\end{array}$$
	\vspace{0.5cm}

    %-------------------- 2.
    \item Supóngase $a\in \bf{F}$, $v\in V$ y $av=0$. Demostrar que $a=0$ o $v=0$.\\\\
	Demostración.-\; Por propiedades de lógica ($p \Rightarrow q \lor r \equiv p\land \sim q \Rightarrow r$). Entonces será suficiente demostrar que,
	$$av=0 \land a\neq 0 \Rightarrow v=0 \qquad a\in \bf{F}, v\in V.$$
	Sea $av=0$. Dado que $a\neq 0$ y $a\in \bf{F}$, podemos multiplicar ambos lados por $\dfrac{1}{a}$,
	$$\dfrac{1}{a}(av) = 0\dfrac{1}{a}.$$
	Luego por propiedades en $\bf{F}$ tenemos que
	$$\left(\dfrac{1}{a}\cdot a\right)v=0\quad \Rightarrow \quad 1v=0 \quad \Rightarrow \quad v=0.$$\\

    %-------------------- 3.
    \item Supóngase $v,w\in V$. Explique porque existe un único $x\in V$ tal que $v+3x=w$.\\\\
	Respuesta.-\; Supongamos $x=\dfrac{1}{3}(w-v)$. Dado $v+(-w)\in V$, entonces  por la asociatividad de espacio vectorial, tenemos
	$$\begin{array}{rcl}
	    v+3\left[\dfrac{1}{3}(w-v)\right] &=& v + \left(3\cdot\dfrac{1}{3}\right)(w-v)\\\\
	    &=& v+(w-v)\\\\
	    &=& w
	\end{array}$$
	Por lo que demostramos su existencia. Ahora, demostremos la unicidad. Sea $x,x'\in V$ tales que $v+3x=w$ y $v+3x'=w$. Por el inverso aditivo, 
	$$w+(-w)=0 \quad \Rightarrow w-w=0.$$
	Reemplazando tenemos,
	$$\begin{array}{rcl}
	    v+3x-(v+3x') &=&0\\
	    (v-v) + 3x-3x' &=&0\\
	    0+3(x-x')&=&0\\
	    x-x'&=&0
	\end{array}$$
	Veamos que esta última igualdad se cumplirá si y sólo si $x=x'$. Por lo tanto queda demostrado el ejercicio.\\\\


    %-------------------- 4.
    \item El conjunto vacío no es un espacio vectorial. El conjunto vacío deja de satisfacer solo uno de los requisitos enumerados en 1.19. ¿Cuál?.\\\\
	Respuesta.-\; Un espacio vectorial debe contener la identidad aditiva $0$, el cual no está presente en el conjunto vacío.\\\\

    %-------------------- 5.
    \item Demuestre que en la definición de un espacio vectorial (1.19), la condición inversa aditiva se puede sustituir por la condición de que
    $$0v=0\mbox{ para todo } v\in V.$$
    Aquí, el $0$ del lado izquierdo es el número $0$, y el $0$ en el lado derecho es la identidad aditiva de $V$. (La frase $"$ una condición puede ser sustituida $"$ en una definición significa que la colección de objetos que satisfacen la definición no cambia si la condición original se sustituye con la nueva condición).\\\\
	Demostración.-\;  El teorema 1.29 (Axler, Linear algebra, capítulo 1), nos dice que $0v=0$ para todo $v\in V$, satisface la definición 1.19. Por lo tanto, debemos demostrar que esta propiedad junto con todas las demás que componen la Definición 1.19, excepto la condición inversa aditiva, implican la condición inversa aditiva. Es decir, debemos demostrar que para todo $v\in V$, existe $w\in V$ tal que $v+w=0.$\\ 
	Suponemos que $v$ es algún vector arbitrario en $V$. Entonces,
	$$\begin{array}{rcl}
	    0 &=& 0v \\
	      &=& (-1+1)v\\
	      &=& (-1)v + 1v\\
	      &=& (-1)v + v\\
	\end{array}$$
	Pongamos $w=(-1)v$ para cualquier vector $v\in V$, donde vemos que existe un vector $w$ que satisface la ecuación $v+w=0$, demostrando nuestro resultado.\\\\


    %-------------------- 6.
    \item Sea $\infty$ y $-\infty$ dos objetos distintos, ninguno de los cuales está en $\bf{R}$. Definir una suma y una multiplicación escalar en $\bf{R}\cup \left\{\infty\right\}\cup\left\{-\infty\right\}$ como se puede adivinar la notación. Específicamente, la suma y el producto de dos números reales son los habituales, y  para $t\in \bf{R}$ definir
    $$
    t\infty = \left\{\begin{array}{rcl}
	    -\infty & \mbox{si} & t<0,\\
	    0 & \mbox{si} & t=0,\\
	    \infty & \mbox{si} & t>0
    \end{array}\right.,\qquad 
    t(-\infty) = \left\{\begin{array}{rcl}
	    \infty & \mbox{si} & t<0,\\
	    0 & \mbox{si} & t=0,\\
	    -\infty & \mbox{si} & t>0
    \end{array}\right.
    $$
    $$t+\infty=\infty+t=\infty,\qquad t+(-\infty)=(-\infty)+t=-\infty,\qquad \infty+\infty=\infty,$$
    $$(-\infty)+(-\infty)=-\infty,\qquad \infty+(-\infty)=0.$$
    Es $\bf{R}\cup \left\{\infty\right\}\cup \left\{-\infty\right\}$ un espacio vectorial sobre $\bf{R}$?. Explique.\\\\
    Respuesta.-\; Para que $\bf{R}\cup \left\{\infty\right\}\cup\left\{-\infty\right\}$ se convierta en un espacio vectorial sobre $\bf{R}$,  debe satisfacer todas las propiedades dadas en la definición 1.19. Ahora bien, este conjunto satisface algunas de estas propiedades. Por ejemplo, la adición es conmutativa en este conjunto. Sin embargo, no las satisface todas.\\
    Una propiedad que no se cumple es la asociatividad de la suma. Por ejemplo, observe que $1,\infty$ y $-\infty$ son todos vectores en $\bf{R}\cup \left\{\infty\right\}\cup\left\{-\infty\right\}$ y por tanto deberíamos tener lo siguiente:
    $$1+\left[\infty+(-\infty)\right]=(1+\infty)+(-\infty)$$
    Sin embargo, cuando evaluamos los lados izquierdo y derecho de esta ecuación usando las definiciones dadas en el enunciado del ejercicio, encontramos que
    $$\begin{array}{rcl}
    1+\left[\infty+(-\infty)\right] &=& (1+\infty)+(-\infty)\\
    1+0&=& \infty+(-\infty)\\
    1&=&0
    \end{array}$$
    Pero esto es claramente falso. Así que nuestra suposición de que $1 + \left[\infty + (-\infty)\right] = (1+\infty) + (-\infty)$ también es falsa y, por tanto, este conjunto no satisface la asociatividad aditiva.\\\\

\end{enumerate}


%%%%%%%%%%%%%%%%%%%%%%%%%%%%%%%%%%%%% Subespacios %%%%%%%%%%%%%%%%%%%%%%%%%%%%%%%%%%%%%

%-------------------- 1.32
\mysection{Subespacios}
    \begin{mydef}[Subespacio]\;\\
	Un subconjunto $U$ de $V$ es llamado un subespacio de $V$ si $U$ es también un espacio vectorial (Usando la misma adición y multiplicación escalar como en $V$).
    \end{mydef}
\vspace{.5cm}

El siguiente resultado brinda la forma más fácil de verificar si un subconjunto de un espacio vectorial es un subespacio.\\

%-------------------- 1.34
\setcounter{myteo}{33}
\begin{myteo}[Condiciones para un subespacio]\;\\\\
    Un subconjunto $U$ de $V$ es un subespacio de $V$ si y sólo si $U$ satisface las siguientes tres condiciones:\\
    \begin{itemize}
	\item \textbf{Identidad aditiva.} $$0\in U;$$
	\item \textbf{Cerrado por adición.}$$u,w\in U\; \mbox{implica}\; u+w\in U;$$
	\item \textbf{Cerrado por multiplicación escalar.}
	    $$a\in F \; \mbox{y}\; u\in U\; \mbox{implica}\; au\in U;$$
    \end{itemize}
    \vspace{.4cm}
	Demostración.-\; Si $U$ es un subespacio de $V$, entonces $U$ satisface las tres condiciones de arriba por la definición de espacio vectorial.\\
	A la inversa, suponga que $U$ satisface las tres condiciones anteriores. La primera condición anterior asegura que la identidad aditiva de $V$ está en $U$. La segunda condición de arriba asegura que la adición tenga sentido en $U$. La tercera condición asegura que la multiplicación escalar tenga sentido en $U$.\\
	Si $u\in U$, entonces $-u$ [el cual es igual a $(-1)u$ por 1.31] es también en $U$ por la tercera condición de arriba. Por lo tanto cada elemento de $U$ tiene un inverso aditivo en $U$.\\
	Las otras partes de la definición de un espacio vectorial,como la asociatividad y conmutatividad, se satisfacen automáticamente para $U$ porque sostienen el espacio más grande de $V$. Así $U$ es un espacio vectorial y por ende es un subespacio de $V$.
\end{myteo}
\vspace{.5cm}

%%%%%%%%%%%%%%%%%%%% Suma de Subespacios %%%%%%%%%%%%%%%%%%%%%%%
\subsection{Suma de subespacios}

La unión de subespacios rara vez es un subespacio (ver Ejercicio 12), por lo que solemos trabajar con sumas en lugar de uniones.

%-------------------- 1.36
\setcounter{mydef}{35}
\begin{mydef}[Suma de subespacios]\,\\\\
    Supóngase $U_1,\ldots ,U_m$ son subconjuntos de $V$. La suma de $U_1,\ldots, U_m$, denotado por $U_1 + \ldots+U_m$, es el conjunto de todas los posibles  sumas de elementos de $U_1,\ldots,U_m$. Más precisamente,
    $$U_1+\ldots+U_m = \lbrace u_1+\ldots +u_m\; :\; u_1\in U_1,\ldots, u_m \in U_m \rbrace$$
\end{mydef}
\vspace{.5cm}

El siguiente resultado afirma que la suma de subespacios es un subespacio, y es de hecho el subespacio más pequeño que contiene todos los sumandos.\\

%-------------------- 1.39
\setcounter{mydef}{38}
\begin{myteo}[La suma de subespacios es el subespacio más pequeño que lo contiene]\,\\\\ 
    Supóngase que $U_1,\ldots,U_m$ son subespacios de $V$. Entonces $U_1+\ldots +U_m$ es el subespacio más pequeño de $V$ que contiene $U_1,\ldots, U_m$.\\\\
	Demostración.-\; Es fácil ver que $0\in U_1+\ldots + U_m$ y que $U_1+\ldots +U_m$ es cerrado por la adición y la multiplicación escalar, es decir, por las tres condiciones dadas anteriormente podemos afirmar que $U_1+\ldots + U_m$ es un subespacio de $V$.\\
	Claramente $U_1,\ldots, U_m$ están todos contenidos en $U_1 + \ldots + U_m$ (para ver esto, considere las sumas $u_1+\ldots + u_m$ donde todas menos una de las $u^{'}s$ son $0$). A la vez, cada subespacio de $V$ que contenga $U_1,\ldots , U_m$ contiene $U_1 + \ldots + U_m$ (porque los subespacios deben contener todas las sumas finitas de sus elementos). Por lo tanto $U_1+\ldots + U_m$ es el subespacio más pequeño de $V$ que contiene a $U_1,\ldots , U_m$.\\
	Esta segunda parte, podríamos demostrarla también de la siguiente manera: Supongamos que $W\subseteq V$ es algún subespacio que contiene a $U_1,\ldots,U_n$, para todo $u_1,\ldots,u_n\in W$. Ya que $W$ es un subespacio se sigue también $u_1+\ldots+u_n\in W$ (porque $W$ es cerrado por la suma finita). Por lo tanto
	$$U_1+\ldots+U_n = \left\{u_i+\ldots + u_n : u_1\in U_i,\ldots , u_n \in U_n\right\}\subseteq W.$$
\end{myteo}
Las sumas de subespacios en la teoría de espacios vectoriales son análogas a las uniones de subconjuntos en la teoría de conjuntos. Dados dos subespacios de un espacio vectorial, el subespacio más pequeño que los contiene es su suma. Análogamente, dados dos subconjuntos de un conjunto, el subconjunto más pequeño que los contiene es su unión.\\\\


\subsection{Sumas directas}

%-------------------- 1.40
    \begin{mydef}[Suma directa]\,\\\\
	Supóngase que $U_1,\ldots, U_m$ son subespacios de $V$.
	\begin{itemize}
	    \item La suma $U_1+\ldots + U_m$ es llamada una suma directa si cada elemento de $U_1+\ldots + U_m$ se puede escribir de una sola manera como una suma  $u_1 + \ldots + u_m$ donde cada $u_j$ esta en $U_j$.
	    \item Si $U_1 + \ldots + U_m$ es una suma directa, entonces $U_1 \oplus \ldots \oplus U_m$ denota $U_1 + \ldots + U_m$ con la notación $\oplus$ sirviendo como una indicación de que se trata de una suma directa.
	\end{itemize}
    \end{mydef}
\vspace{.5cm}

La definición de suma directa requiere que cada vector de la suma tenga una representación única como suma adecuada. El siguiente resultado muestra que al decidir si una suma de subespacios es una suma directa, sólo tenemos que considerar si 0 puede escribirse de forma única como una suma correspondiente.\\

%-------------------- 1.44
\setcounter{myteo}{43}
\begin{myteo}[Condición para una suma directa]\,\\\\
    Supóngase que $U_1,\ldots, U_m$ son subespacios de $V$. Entonces $U_1+\ldots + U_m$ es una suma directa si y sólo si la única manera de escribir $0$ como una suma $u_1 + \ldots + u_m$, donde cada $u_j$ está en $U_j$, es tomando cada $u_j$ igual a $0$.\\\\
	Demostración.-\; Supóngase que $U_1+\ldots + U_m$ es una suma directa. Entonces la definición de suma directa implica que la única manera de escribir $0$ como una suma $u_1 + \ldots + u_m$, donde cada $u_j$ está en $U_j$, es tomando cada $u_j$ igual a $0$. Para mostrar que $U_1 + \ldots + U_m$ es una suma directa, sea $v \in U_1 + \ldots + U_m$. Podemos escribir:
	$$v = u_1 + \ldots + u_m$$
	para algún $u_1 \in U_1 , \ldots, u_m \in U_m$. Para mostrar que esta representación es única, supongamos que también tenemos
	$$v = v_1 + \ldots + v_m$$
	donde $v_1 \in U_1,\ldots , v_m \in U_m$. Restando estas dos ecuaciones, tenemos,
	$$0=(u_1-v_1)+\ldots + (u_m - v_m).$$
	Porque $u_1-v_1\in U_1,\ldots U_m$, la ecuación dada implica que cada $u_j - v_j$ es igual a $0$. Por lo tanto $u_1 = v_1, \ldots , u_m = v_m$.
\end{myteo}
\vspace{.5cm}

El siguiente resultado da una condición simple para probar qué pares de subespacios dan una suma directa.

%-------------------- 1.45
\begin{myteo}[Suma directa de dos subespacios]\,\\\\
    Supóngase que $U$ y $W$ son subespacios de $V$. Entonces $U+W$ es una suma directa si y sólo si $U \cap W = \lbrace 0 \rbrace$.\\\\
    Demostración.-\; Primero supóngase que $U+W$ es una suma directa. Si $v \in U\cap W,$ entonces $0=v+(-v),$ donde $v\in U$ y $-v\in W$. Por la única representación de $0$ como la suma de un vector en $U$ y un vector en $W$, tenemos $v=0$. 
\end{myteo}


\section*{Ejercicios 1.C}

\begin{enumerate}[\bfseries 1.]

    %-------------------- 1.
    \item Para cada de los siguientes subconjuntos de $\bf{F}^3$, determinar si es un subespacio de $\bf{F}^3:$
	\begin{enumerate}[(a)]

	    %---------- (a)
	    \item $\left\{(x_1,x_2,x_3)\in \bf{F}^3 : x_1+2x_2+3x_3=0\right\}$.\\\\
		Respuesta.-\; 

	\end{enumerate}
\end{enumerate}


