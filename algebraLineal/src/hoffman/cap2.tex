\chapter{Espacios vectoriales}

\section{Espacios vectoriales}

%-------------------- Definición 1 espacio vectorial --------------------
\begin{def.}
    Un \textbf{espacio vectorial} (o espacio lineal) consta de lo siguiente:
    \begin{enumerate}[1.]
	\item Un cuerpo $F$ de escalares;
	\item un conjunto $V$ de objetos llamados vectores;
	\item una regla (u operación) llamada adición, que asocia a cada par de vectores $\alpha,\beta $ de $V$ un vector $\alpha+\beta$ de $V$, que se llama suma de $\alpha$ y $\beta$, de tal modo que:
	   \begin{enumerate}[(a)]
	       \item La adición es conmutativa, $\alpha+\beta=\beta+\alpha$;
	       \item la adición es asociativa, $\alpha+(\beta+\gamma)=(\alpha+\beta)+\gamma$;
	       \item existe un único vector $0$ de $V$, llamado vector nulo tal que $\alpha+0=\alpha,$ para todo $\alpha$ de $V$;
	       \item para cada vector $\alpha$ de $V$ existe un vector $-\alpha$ de $V$, tal que $\alpha+(-\alpha)=0$;
	   \end{enumerate}
       \item una regla (u operación) llamada multiplicación escalar, que asocia a cada escalar $c$ de $F$ y cada vector $\alpha$ de $V$ a un vector $c\alpha$ en $V$, llamado producto de $c$ y $\alpha$, de tal modo que:
	   \begin{enumerate}[(a)]
	       \item $1\alpha=\alpha$ para todo $\alpha$ de $V$;
	       \item $(c_1c_2)\alpha=c_1(c_2\alpha)$;
	       \item $c(\alpha+\beta)=c\alpha+c\beta$;
	       \item $(c_1+c_2)\alpha=c_1\alpha+c_2\alpha$.
	   \end{enumerate}
    \end{enumerate}
\end{def.}


%-------------------- ejemplo 1
\begin{ejem}[El espacio de n-tuplas, \boldmath $F_n$]
    Sea $F$ cualquier cuerpo y sea $V$ el conjunto de todos los n-tuples $\alpha=(x_1,x_2,\ldots,x_n)$ de escalares $x_i$ de $F$. Si $\beta = (y_1,y_2,\ldots , y_n)$ con $y_i$ de $F$, la suma de $\alpha$ y $\beta$ se define por
    \begin{equation}
	\alpha+\beta = (x_1+y_1,x_2+y_2,\ldots,x_n+y_n).
    \end{equation}
    El producto de un escalar $c$ y el vector $\alpha$ se define por 
    \begin{equation}
	c\alpha = (cx_1,cx_2,\ldots,cx_n).
    \end{equation}
    Que esta adición vectorial y multiplicación escalar cumplen las condiciones (3) y (4) es fácil de verificar, usando las propiedades semejantes de la adición y multiplicación de elementos de $F$.
\end{ejem}

%-------------------- ejemplo 2
\begin{ejem}[El espacio de matrices \boldmath $m\times n, F^{m\times n}$] Sea F cualquier cuerpo y sean $m$ y $n$ enteros positivos. Sea $F^{m\times n}$ el conjunto de todas las matrices $m\times n$ sobre el cuerpo $F$. La suma de dos vectores $A$ y $B$ en $F^{m\times n}$ se define por 
    \begin{equation}
	(A+B)_{ij}=A_{ij}+B_{ij}.
    \end{equation}
    El producto de un escalar $c$ y de la matriz $A$ se define por 
    \begin{equation}
	(cA)_{ij} = cA_{ij}.
    \end{equation}
    Obsérvese que $F^{1\times n}=F^n$.
\end{ejem}

%------------------- ejemplo 3
\begin{ejem}[El espacio de funciones de un conjunto en un cuerpo]
    Sea $F$ cualquier cuerpo y sea $S$ cualquier conjunto no vacío. Sea $V$ el conjunto de todas las funciones de $S$ en $F$. La suma de dos vectores $f$ y $g$ de $V$ es el vector $f+g$; es decir, la función de $S$ en $F$ defina por
    \begin{equation}
	(f+g)(s)=f(s)+g(s).
    \end{equation}
    El producto del escalar $c$ y la función $f$ es la función $cf$ definida por
    \begin{equation}
	(cf)(s)=cf(s).
    \end{equation}
    Para este tercer ejemplo se indica cómo se puede verificar que las operaciones definidas satisfacen las condiciones (3) y (4). Para la adición vectorial:
    \begin{enumerate}[(a)]
	\item Como la adición en $F$ es conmutativa,
	    $$f(s)+g(s)=g(s)+f(s)$$
	    para todo $s$ de $S$, luego las funciones $f+g$ y $g+f$ son idénticas.
	\item Como la adición en $F$ es asociativa,
	    $$f(s)+[g(s)+h(s)]=[f(s)+g(s)]+h(s)$$
	    para todo $s$, luego $f+(g+h)$ es la misma función que $(f+g)+h.$
	\item El único vector nulo es la función cero, que asigna a cada elemento de $S$ el escalar $0$ de $F$.
	\item Para todo $f$ de $V$, $(-f)$ es la función dada por
	    $$(-f)=-f(s).$$
    \end{enumerate}
    El lector encontrará fácil verificar que la multiplicación escalar satisface las condiciones de (4), razonando como se hizo para la adición vectorial.

\end{ejem}

%-------------------- ejemplo 4.
\begin{ejem}[El espacio de las funciones polinomios sobre el cuerpo $F$] Sea $F$ un cuerpo y sea $V$ el conjunto de todas las funciones $f$ de $F$ en $F$ definidas en la forma
    \begin{equation}
	f(x)=c_0+c_1x+\ldots + c_n x^n
    \end{equation}
    donde $c_0,c_1,\ldots , c_n$ son escalares fijos de $F$ (independiente de $x$). Una función de este tipo se llama \textbf{\boldmath función polinomio sobre $F$}. Sean la adición y la multiplicación escalar definidas sobre en el ejemplo 3. Se debe observar que si $f$ y $g$ son funciones polinomios y $c$ está en $F$, entonces $f+g$ y $cf$ son también funciones polinomios.
\end{ejem}

%-------------------- ejemplo 5
\begin{ejem}
    El cuerpo $C$ de los números complejos puede considerarse como un espacio vectorial sobre el cuerpo $R$ de los números reales. En forma más general, sea $F$ el cuerpo de los números reales y sea $V$ el conjunto de los n-tuples $\alpha=(x_1,\ldots,x_n)$ donde $x_1,\ldots , x_n$ son números complejos. Se define la adición vectorial y la multiplicación escalar por (2.1) y (2-2), como en el ejemplo 1. De este modo se obtiene un espacio vectorial sobre el cuerpo $R$ que es muy diferente del espacio $C^n$ y del espacio $R_n$.
\end{ejem}

\vspace{.5cm}

Hay unos pocos hechos simples que se desprenden, casi inmediatamente, de la definición de espacio vectorial, y procederemos a derivarlos. Si $c$ es un escalar y $0$ es el vector nulo, entonces por 3(c) y 4(c)
$$c0=c(0+0)=c0+c0.$$
Sumando $-(c0)$ y por $3(d)$, se obtiene
\begin{equation}
    c0=0.
\end{equation}

Análogamente, para el escalar $0$ y cualquier vector $\alpha$ se tiene que

\begin{equation}
    0\alpha=0.
\end{equation}

Si $c$ es un escalar no nulo y $\alpha$ un vector tal que $c\alpha=0$, entonces por (2-8), $c^{-1}(c\alpha)=0.$ Pero
$$c^{-1}(c\alpha)=(c^{-1}c)\alpha=1\alpha=\alpha$$
luego, $\alpha=0$. Así se ve que si $c$ es un escalar y $\alpha$ un vector tal que $c\alpha=0$, entonces $c$ es el escalar cero o $\alpha$ es el vector nulo.\\

Si $\alpha$ es cualquier vector de $V$, entonces
$$0=0\alpha=(1-1)\alpha=1\alpha+(-1)\alpha=\alpha+(-1)\alpha$$
de lo que se sigue que

\begin{equation}
    (-1)\alpha=-\alpha.
\end{equation}

Finalmente, las propiedades asociativa y conmutativa de la adición vectorial implican que la suma de varios vectores es independiente de cómo se combinen estos vectores y de cómo se asocien. Por ejemplo, si $\alpha_1,\alpha_2,\alpha_3,\alpha_4$ son vectores de $V$, entonces
$$(\alpha_1,\alpha_2)+(\alpha_3+\alpha_4)=\left[\alpha_2+(\alpha_1+\alpha_3)\right]+\alpha_4$$
y tal suma puede ser escrita, sin confusión,

$$\alpha_1+\alpha_2+\alpha_3+\alpha_4.$$

%-------------------- Definición combinación lineal
\begin{def.}
    Un vector $\beta$ de $V$ se dice \textbf{combinación lineal} de los vectores $\alpha_1,\ldots , \alpha_n$ en $V$, si existen escalares $c_1,\ldots , c_n$ de $F$ tales que
    $$\beta = c_1\alpha_1 + \ldots + c_n\alpha_n = \sum_{i=1}^n c_i\alpha_i.$$
\end{def.}

Otras extensiones de la propiedad asociativa de la adición vectorial y las propiedades distributivas 4(c) y 4(d) de la multiplicación escalar se aplican a las combinaciones lineales:
$$\sum_{i=1}^n c_i\alpha_i + \sum_{i=1}^n d_i\alpha_i=\sum_{i=1}^n (c_i+d_i)\alpha_i$$
$$c\sum_{i=1}^n c_i\alpha_i = \sum_{i=1}^n (cc_i)\alpha_i.$$

\section*{Ejercicios}

\begin{enumerate}[\bfseries 1.]

    %-------------------- 1.
    \item Si $F$ es un cuerpo, verificar que $F^n$ (como se definio en el Ejemplo 1) es un espacio vectorial sobre el cuerpo $F$.\\\\
	Respuesta.-\; Sean $\alpha=(x_1,x_2,\ldots,x_n)$,  $\beta=(y_1,y_2,\ldots,y_n)$ y $\gamma=(z_1,z_2,\ldots,z_n)$ elementos de $F^n$. Como también sean $c,d,c_1,c_2\in F$. Entonces,

	\begin{enumerate}[(1)]
	    \setcounter{enumii}{2}
	    \item 
		\begin{enumerate}[(a)]
		    \item Conmutatividad para la adición. 
			$$
			\begin{array}{rcl}
			    \alpha+\beta & = & (x_1,x_2,\ldots,x_n)+(y_1,y_2,\ldots, y_n) \\
			    & = & (x_1+x_2,\ldots,x_n+y_n)\\
			    & = & (y_1+y_2,\ldots,y_n+x_n)\\
			    & = & \beta+\alpha.
			\end{array}
			$$
		    \item Asociatividad para la adición.
			$$
			\begin{array}{rcl}
			    \alpha+(\beta+\gamma) & = & (x_1,x_2,\ldots,x_n)+(y_1,y_2,\ldots, y_n)+(z_1,z_2,\ldots,z_n) \\
			    & = & (x_1+x_2,\ldots,x_n+y_n+z_n)\\
			    & = & (x_1+x_2+z_1,\ldots,x_n+y_n+z_n)\\
			    & = & (\alpha+\beta)+\gamma.
			\end{array}
			$$
		    \item Existencia del elemento nulo.\\
			$$
			\begin{array}{rcl}
			    \alpha+0 & = & (x_1,x_2,\ldots,x_n)+(0,0,\ldots,0)\\
			    & = & (x_1+0,\ldots,x_n+0)\\
			    & = & \alpha.
			\end{array}
			$$
		    \item Existencia del inverso aditivo.
			$$
			\begin{array}{rcl}
			    \alpha+(-\alpha) & = & (x_1,x_2,\ldots,x_n)+\left[-(x_1,x_2,\ldots,x_n)\right]\\
			    & = & (x_1,x_2,\ldots,x_n)+(-x_1,-x_2,\ldots,-x_n)\\
			    & = & (x_1-x_1,\ldots,x_n-x_n)\\
			    & = & (0,0,\ldots,0)\\
			    & = & 0.
			\end{array}
			$$

		\end{enumerate}

	    \item 
		\begin{enumerate}[(a)]
		    \item Existencia del elemento neutro para la multiplicación escalar.
			$$
			\begin{array}{rcl}
			    1\alpha & = & 1(x_1,x_2,\ldots,x_n)\\
			    & = & (1x_1,1x_2,\ldots,1x_n)\\
			    & = & \alpha.
			\end{array}
			$$
		    \item Asociatividad para la multiplicación escalar.\\
			$$
			\begin{array}{rcl}
			    (c_1c_2)\alpha & = & (c_1c_2)(x_1,x_2,\ldots,x_n)\\
			    & = & (c_1c_2x_1,c_1c_2x_2,\ldots,c_1c_2x_n)\\
			    & = & (c_1(c_2x_1),c_1(c_2x_2),\ldots,c_1(c_2x_n))\\
			    & = & (c_1c_2x_1,c_1c_2x_2,\ldots,c_1c_2x_n)\\
			    & = & c_1(c_2\alpha).
			\end{array}
			$$

		    \item Distributividad para la multiplicación escalar sobre la adición.
			$$
			\begin{array}{rcl}
			    c(\alpha+\beta) & = & c((x_1,x_2,\ldots,x_n)+(y_1,y_2,\ldots, y_n))\\
			    & = & c(x_1+x_2,\ldots,x_n+y_n)\\
			    & = & (cx_1+cx_2,\ldots,cx_n+cy_n)\\
			    & = & (c\alpha+c\beta).
			\end{array}
			$$

		    \item Distributividad para la multiplicación sobre la adición de escalares
			$$
			\begin{array}{rcl}
			    (c+d)\alpha & = & (c+d)(x_1,x_2,\ldots,x_n)\\
			    & = & (cx_1+dx_1,\ldots,cx_n+dx_n)\\
			    & = & c\alpha+d\alpha.
			\end{array}
			$$

		\end{enumerate}
		\vspace{.5cm}

	\end{enumerate}

    %-------------------- 2.
    \item Si $V$ es un espacio vectorial sobre un cuerpo $F$, verificar que
    $$(\alpha_1+\alpha_2)+(\alpha_3+\alpha_4)=\left[\alpha_2+(\alpha_3+\alpha_1)\right]+\alpha_4$$
    para todo los vectores $\alpha_1,\alpha_2,\alpha_3,\alpha_4$ de $v$.\\\\
	Respuesta.-\; Se tiene,
	$$
	\begin{array}{rcl}
	    (\alpha_1+\alpha_2)+(\alpha_3+\alpha_4) &=& (\alpha_2+\alpha_1)+(\alpha_3+\alpha_4)\\
						    &=& \alpha_2+\left[\alpha_1+(\alpha_3+\alpha_4)\right]\\
						    &=& \alpha_2+\left[(\alpha_1+\alpha_3)+\alpha_4\right]\\
						    &=& \alpha_2+\left[(\alpha_3+\alpha_1)+\alpha_4\right]\\
						    &=& \left[\alpha_2+(\alpha_3+\alpha_1)\right]+\alpha_4.
	\end{array}
	$$
	\vspace{.5cm}

    %-------------------- 3.
    \item Si $C$ es el cuerpo de los complejos, ¿qué vectores de $C^3$ son combinaciones lineales de $(1,0,-1),(0,1,1)$ y $(1,1,1)$?.\\\\
	Respuesta.-\; Sea $(x,y,z)\in C^3$ una convinación lineal de los vectores $(1,0,-1),(0,1,1)$ y $(1,1,1)$. Entonces, existen escalares $a,b$ y $c\in C$ tal que
	$$
	\begin{array}{rcl}
	    (x,y,z)&=&a(1,0,-1)+b(0,1,1)+c(1,1,1)\\
		   &=& (a+c,b+c,c-a).
	\end{array}
	$$
	De donde, 
	$$
	\left\{	
	    \begin{array}{rcl}
		a+c & = & x\\
		b+c & = & y\\
		c-a & = & z.
	    \end{array}
	\right.
	$$
	Resolviendo se tiene,
	$$
	\left\{	
	    \begin{array}{rcl}
		a & = & \dfrac{x-z}{2}\\\\
		b & = & \dfrac{2z-x-y}{2}\\\\
		c & = & \dfrac{x+z}{2}.
	    \end{array}
	\right.
	$$

	Por lo tanto, existen escalares $a,b$ y $c\in C$ tal que 
	$$(x,y,z)=a(1,0,-1)+b(0,1,1)+c(1,1,1).$$
	Así, todos los vectores en $C^3$ pueden ser expresados como una combinación lineal de los vectores $(1,0,-1),(0,1,1)$ y $(1,1,1)$.\\\\

    %-------------------- 4.
    \item Sea $V$ el conjunto de los pares $(x,y)$ de números reales, y sea $F$ el cuerpo de los números reales. Se define
    $$
    \begin{array}{rcl}
	(x,y)+(x_1,y_1) &=& (x+x_1,y+y_1)\\
	c(x,y) &=& (cx,y).
    \end{array}
    $$
    ¿Es $V$, con estas operaciones, un espacio vectorial sobre el cuerpo de los números reales?.\\\\
	Respuesta.-\; No es un espacio vectorial ya que, 
	$$(0,2)=(0,1)+(0,1)=2(0,1)=(2\cdot 0,1)=(0,1).$$\\

    %-------------------- 5.
    \item En $\mathbb{R}^n$ se definen dos operaciones
    $$\alpha \oplus \beta = \alpha-\beta$$
    $$c\cdot \alpha=-c\alpha.$$
    Las operaciones del segundo miembro son las usuales. ¿Qué axiomsa de espacio vectorial se cumplen para $\mathbb{R}^n,\oplus,\cdot$?.\\\\
	Respuesta.-\; Sean $\alpha=(x_1,x_2,\ldots,x_n)$,  $\beta=(y_1,y_2,\ldots,y_n)$ y $\gamma=(z_1,z_2,\ldots,z_n)$ elementos de $F^n$. Como también sean $c,d,c_1,c_2\in F$. Entonces,
	\begin{enumerate}[(1)]
	    \setcounter{enumii}{2}
	    \item 
		\begin{enumerate}[(a)]
		    \item No es conmutativa para la adición. 
			$$
			\begin{array}{rcl}
			    \alpha\oplus \beta &=& \alpha-\beta\\
					       & = & (x_1,x_2,\ldots,x_n)-(y_1,y_2,\ldots, y_n) \\
							      &=& (x_1-y_1,x_2-y_2,\ldots,x_n-y_n)\\
							      &=& (-y_1+x_1,-y_2+x_2,\ldots,-y_n+x_n)\\
							      &=& -\beta+ \alpha.\\
							      &=& -(\beta-\alpha)\\
							      &=& -(\beta\oplus \alpha)\\
							      &\neq& \beta\oplus \alpha.
			\end{array}
			$$

		    \item No es asociativa para la adición.
			$$
			\begin{array}{rcl}
			    \alpha\oplus (\beta\oplus \gamma) &=& \alpha-(\beta-\gamma)\\
							      &=& (\alpha-\beta)+\gamma \\
							      &=& (\alpha\oplus \beta)+\gamma\\
							      &\neq& \alpha\oplus (\beta\oplus \gamma).

			\end{array}
			$$
		    \item No existe el elemento nulo.\\
			$$
			\begin{array}{rcl}
			    \alpha\oplus 0 &=& \alpha-0\\
					   &=&\alpha-(0,0,\ldots,0)\\
					   &=& \alpha.
			\end{array}
			$$
			Pero, como $\oplus$ no es conmutativa; es decir, $\alpha\oplus \neq 0\oplus \alpha$ decimos que no existe la identidad aditiva para $\oplus.$\\
			
		    \item No existe el inverso aditivo.
			$$
			\begin{array}{rcl}
			    \alpha\oplus(-\alpha) & = & \alpha - (-\alpha)\\
						  & = & \alpha + \alpha\\
						  &\neq & 0
			\end{array}
			$$

		\end{enumerate}

	    \item 
		\begin{enumerate}[(a)]
		    \item No existe el elemento neutro para la multiplicación escalar.\\
			El elemento $1$ no satisface $1\cdot \alpha=\alpha$ para cualquier $\alpha\neq 0$, ya que de lo contrario tendríamos 
			$$1\cdot(x_1,x_2,\ldots,x_n)=(-x_1,x_2\ldots,x_n)=(x_1,x_2,\ldots,x_n)$$
			sólo si $x_i=0$ para todo $i$.\\

		    \item No es asociativa para la multiplicación escalar.\\
			$$
			\begin{array}{rcl}
			    c_1(c_2\alpha) & = & c_1(-c_2\alpha)\\
					   & = & -c_1(-c_2\alpha)\\
					   & = & -(c_1c_2)\alpha\\
					   & \neq & (c_1c_2)\alpha.
			\end{array}
			$$

		    \item No es distributiva para la multiplicación escalar sobre la adición.
			$$
			\begin{array}{rcl}
			c(\alpha+\beta) & = & -c(\alpha+\beta)\\
					&=& -c\alpha-c\beta\\
					&=& -(c\alpha+c\beta)\\
					&\neq& c\alpha+c\beta.
			\end{array}
			$$

		    \item No es distributiva para la multiplicación sobre la adición de escalares
			$$
			\begin{array}{rcl}
			    c\alpha + d\alpha & = & -c\alpha - d\alpha\\
					      & = & -(c+d)\alpha\\
					      & \neq & (c+d)\alpha.
			\end{array}
			$$

		\end{enumerate}
		\vspace{.5cm}

	\end{enumerate}

    %-------------------- 6.
    \item Sea $V$ el conjunto de todas las funciones que tiene valor complejo sobre el eje real, tales que (para todo $t$ de $R$)
    $$f-(t)=\overline{f(t)}$$.
    La barra indica conjugación compleja. Demostrar que $V$, con las operaciones 
    $$(f+g)(t)=f(t)+g(t)$$
    $$cf(t)=cf(t)$$
    es un espacio vectorial sobre el cuerpo de los números reales. Dar un ejemplo de una función en $V$ que no toma valores reales.\\\\
    Demostración.-\; Sea $f,g,h\in V$. Entonces, para todo $t\in \mathbb{R},$ se tiene

	\begin{enumerate}[(1)]
	    \setcounter{enumii}{2}
	    \item 
		\begin{enumerate}[(a)]
		    \item Conmutatividad para la adición. 
			$$
			\begin{array}{rcl}
			    (f+g)(t) & = & f(t)+g(t)\\
				     & = & g(t) + f(t)\\
				     & = & (g+f)(t).
			\end{array}
			$$
			Por lo tanto, $f+g=g+f$ para todo $f$ y $g\in V$.\\

		    \item Asociatividad para la adición.
			$$
			\begin{array}{rcl}
			    \left[(f+g)+h\right](t) &=& (f+g)(t)+h(t)\\
						    &=& \left[f(t)+g(t)\right]+h(t)\\
						    &=& f(t)+\left[g(t)+h(t)\right].
			\end{array}
			$$
			Por lo tanto, $(f+g)+h=f+(g+h)$ para todo $f,g,h\in V$.\\

		    \item Existencia del elemento nulo.\\
			Considere la función cero $0(t)=0$ para todo $t\in \mathbb{R}$, entonces para todo $f\in V$, tenemos
			$$(f+0)(t)=f(t)+0(t)=f(t)+0=f(t).$$
			Ya que $+$ es conmutativo, se tiene $f=f=0+f.$\\
			
		    \item Existencia del inverso aditivo.\\
			Para $f\in V$, consideremos $g=(-f)$ como $(-f)(t)=-f(t)$. Claramente $g=-f$ existe en $V$. Luego,
			$$\left[f+(-f)\right]=f(t)+(-f)(t)=f(t)-f(t)=0=0(t).$$
			Ya que, $+$ es conmutativo, tenemos $f+(-f)=0=(-f)+f$ para todo $f\in V$. así, el inverso aditivo existe.\\

		\end{enumerate}

	    \item 
		\begin{enumerate}[(a)]
		    \item Existencia del elemento neutro para la multiplicación escalar.\\
			$$1\cdot f =f.$$
			Para $f\in V$, se tiene
			$$(1\cdot f)(t)=1\cdot f(t)=f(t).$$
			Así, $1\cdot f = f$ para todo $f\in V$.\\

		    \item Asociatividad para la multiplicación escalar.\\
			Sean $a,b\in R$ y $f\in V$, entonces
			$$
			\begin{array}{rcl}
			    (ab)f&=& \left[(ab)\cdot f\right](t)\\
				 &=& (ab) f(t)\\
				 &=& a\left[bf(t)\right]\\
				 &=& a\left(b\cdot f\right)\\
				 &=& a\left(bf\right).
			\end{array}
			$$
			Por lo tanto, $(ab)\cdot f=a\left(b\cdot f\right)$ para todo $f\in V$ y $a,b\in R$.\\

		    \item Distributividad para la multiplicación escalar sobre la adición.\\
			Sean $a,b\in R$ y $f,g\in V$, entonces
			$$
			\begin{array}{rcl}
			    \left[a\left(f+g\right)\right](t) &=& a\left[(f+g)(t)\right]\\
							      &=& a\left[f(t)+g(t)\right]\\
							      &=& \left(af\right)(t)+\left(ag\right)(t)\\
			\end{array}
			$$

		    \item Distributividad para la multiplicación sobre la adición de escalares\\
			Sean $a,b\in R$ y $f\in V$, entonces
			$$
			\begin{array}{rcl}
			    \left[(a+b)f\right](t)&=& (a+b)(f(t)\\
						  &=& af(t)+bf(t)\\
						  &=& (af)(t)+(bf)(t).
			\end{array}
			$$

		\end{enumerate}
		De esta manera, $V$ satiface todos las propiedades del espacio vectorial respecto a las operaciones de adición y multiplicación escalar.\\\\

	\end{enumerate}

    %-------------------- 7.
    \item Sea $V$ el conjunto de pares $(x,y)$ de números reales y sea $F$ el cuerpo de los números reales. Se define
    $$(x,y)+(x_1,y_1)=(x+x_1,0)$$
    $$c(x,y)=(cx,0).$$
    ¿Es $V$, con estas operaciones un espacio vectorial?.\\\\
	Respuesta.-\; No es un espacio vectorial. Sea $u=(x_1,y_1)$ y  $0\in R$, $0=(0,0)\in V$. Entonces,
	$$
	\begin{array}{rcl}
	    u+0&=&(x_1,y_1)+(0,0)\\
	      &=&(x_1+0,0)\\
	      &=&(x_1,0)\\
	      &\neq& u.
	\end{array}
	$$
	Por lo tanto, no existe un inverso aditivo para $V$. Así $V$ no es un espacio vectorial.

\end{enumerate}


\section{Subespacios}

%-------------------- Definición 1.1
\begin{def.}
    Sea $V$ un espacio vectorial sobre el cuerpo $F$. Un \textbf{subespacio} de $V$ es un subconjunto $W$ de $V$ que, con las operaciones de adición vectorial y multiplicación escalar sobre $V$, es el mismo un espacio vectorial sobre $F$.
\end{def.}

Esta definición se puede simplificar aún más.

%-------------------- Teorema 1.1
\begin{teo}
    Un subconjunto no vacío $W$ de $V$ es un subespacio de $V$ si, y sólo si, para todo par de vectores $\alpha,\beta$ de $W$ y todo escalar $c$ de $F$, el vector $c\alpha+\beta$ está en $W$.\\\\
	Demostración.-\; Supóngase que $W$ sea un subconjunto no vacío de $V$ tal que $c\alpha+\beta$ pertenezca a $W$ para todos los vectores $\alpha,\beta$ de $W$ y todos los escalares $c$ de $F$. Como $W$ no es vacío, existe un vector $\rho$ en $W$, y por tanto, $(-1)\rho+\rho=0$ está en $W$. Ahora bien, si $\alpha$ es cualquier vector de $W$ y $c$ cualquier escalar, el vector $c\alpha=c\alpha+0$ está en $W$. En particular, $(-1)\alpha=-\alpha$ está en $W$. Finalmente, si $\alpha+\beta$ están en $W$, entonces $\alpha+\beta=1\alpha+\beta$ está en $W$. Así, $W$ es un subespacio de $V$.\\
	Recíprocamente, si $W$ es un subespacio de $V$, $\alpha$ y $\beta$ están en $W$ y $c$ es un escalar, ciertamente $c\alpha+\beta$ está en $W$.
\end{teo}

\setcounter{ejem}{6}
%-------------------- Ejemplo 2.7
\begin{ejem}[El espacio solución de un sistema homogéneo de ecuaciones lineales]
Sea $A$ un matriz $m\times n$ sobre $F$. Entonces el conjunto de todas las matrices (columna) $n\times 1$, $X$, sobre $F$ tal que $AX=0$ es un subespacio del espacio de todas las matrices $n\times 1$ sobre $F$. Para demostrar esto se necesita probar que $A(cX+Y)=0$ si $AX=0,$ $AY=0$ y $c$ un escalar arbitrario  de $F$. Esto se desprende inmediatamente del siguiente hecho general.
\end{ejem}

%-------------------- Lema 2.1
\begin{lema}
    Si $A$ es una matriz $m\times n$ sobre $F$, y $B$, $C$ son matrices $n\times p$ sobre $F$, entonces
    \begin{equation}
	A(dB+C)=d(AB)+AC
    \end{equation}
    para todo escalar $d$ de $F$.\\\\
	Demostración.-\;
	$$
	\begin{array}{rcl}
	    \left[A(dB+C)\right]_{ij}&=&\displaystyle\sum_k A_{ik}(dB+C)_{kj}\\\\
				     &=&\displaystyle\sum_k \left(dA_{ik}B_{kj}+A_{ik}C_{kj}\right)\\\\
				     &=& d\displaystyle\sum_k A_{ik}B_{kj} + \sum_{k} A_{ik} C_{kj}\\\\
				     &=& d(AB)_{ij} + (AC)_{ij}\\\\
				     &=& \left[d(AB) + AC\right]_{ij}.
	\end{array}
	$$
\end{lema}

En forma semejante se puede ver que $(dB+C)A=d(BA)+CA$, si la suma y el producto de las matrices están definidos.

%-------------------- Teorema 2.2
\begin{teo}
    Sea $V$ un espacio vectorial sobre el cuerpo $F$. La intersección de cualquier colección de subespacios de $V$ es un subespacio de $V$.\\\\
	Demostración.-\; Sea $\left\{W_a\right\}$ una colección de subespacios de $V$, y sea $W=\cap W_a$ su intersección. Recuérdese que $W$ está definido como el conjunto de todos los elementos pertenecientes a cada $W_a$. Dado que todo $W_a$ es un subespacio, cada uno contiene el vector nulo. Así que el vector nulo está en la intersección $W$, y $W$ no es vacío. Sean $\alpha$ y $\beta$ vectores de $W$ y sea $c$ un escalar. Por definición de $W$ ambos,  $\alpha$ y $\beta$ pertenecen a cada $W_a$, y por ser cada $W_a$ un subespacio el vector $(c\alpha+\beta)$ está en cada $W_a$. Así $(c\alpha+\beta)$ está también en $W$. Por el teorema 1, $W$ es un subespacio de $V$. 
\end{teo}

De este teorema se deduce que si $S$ es cualquier colección de vectores de $V$, entonces existe un subespacio mínimo de $V$ que contiene a $S$; esto es, un subespacio que contiene a $S$ y que está contenido en cada uno de los otros subespacios que contienen a $S$.


%-------------------- Definición 2.4
\begin{def.}
    Sea $S$ un conjunto de vectores de un espacio vectorial $V$, El \textbf{subespacio generado} por $S$ se define como la intersección $W$ de todos los subespacios de $V$ que contienen a $S$. Cuando $S$ es un conjunto finito de vectores, $S=\left\{\alpha_1,\alpha_2,\ldots,\alpha_n\right\}$ se dice simplemente que $W$ es el \textbf{subespacio generado por los vectores} $\alpha_1,\alpha_2,\ldots,\alpha_n$.
\end{def.}

%-------------------- Teorema 2.3
\begin{teo}
    El subespacio generado por un subconjunto $S$ no vacío de un espacio vectorial $V$ es el conjunto de todas las combinaciones lineales de los vectores de $S$.\\\\
	Demostración.-\; Sea $W$ el subespacio generado por $S$. Entonces, por definición de subespacio; toda combinación lineal 
	$$\alpha=x_1\alpha_1+x_2\alpha_2+\ldots+x_m,\alpha_m$$
	de vectores $\alpha_1,\alpha_2,\ldots,\alpha_m$ de $S$ pertenecen evidentemente a $W$. Así que $W$ contiene el conjunto $L$ de todas las combinaciones lineales de vectores de $S$. El conjunto $L$, entonces, por otra parte, contiene a $S$ y no es, pues, vacío. Si $\alpha$ y $\beta$ pertencen a $L$, entonces $\alpha$ es una combinación lineal,
	$$\alpha=x_1\alpha_1+x_2\alpha_2+\ldots+x_m,\alpha_m$$
	de vectores $\alpha_i$ de $S$, y $\beta$ es una combinación lineal,
	$$\beta=y_1\alpha_1+y_2\alpha_2+\ldots+y_n,\alpha_n$$
	de vectores $\beta_j$ de $S$. Para cada escalar $c$,
	$$c\alpha+\beta=\sum_{i=1}^n (cx_i)\alpha_i+\sum_{j=1}^n y_j\beta_j.$$
	Luego, $c\alpha+\beta$ pertence a $L$. Con lo que $L$ es un subespacio de $V$.\\
	Se demostró que $L$ es un subespacio de $V$ que contiene a $S$, y también que todo subespacio que contiene a $S$ contiene a $L$ (definición de subespacio). Se sigue que $L$ es la intersección de todos los subespacios que contienen a $S$; es decir, que $L$ es el subespacio generado por el conjunto $S$.
\end{teo}

%-------------------- Definición 2.5
\begin{def.}
    Si $S_1,S_2,\ldots,S_k$ son subconjuntos de un espacio vectorial $V$, el conjunto de todas las sumas 
    $$\alpha_1+\alpha_2+\cdots + \alpha_k$$
    de vectores $\alpha_i$ de $S_i$ se llama \textbf{suma} de los subconjuntos $S_1,S_2,\ldots , S_k$ y se representa por
    $$S_1+S_2+\cdots + S_k$$
    o por
    $$\sum_{i=1}^k S_i.$$
    Si $W_1,W_2,\ldots,W_k$ son subespacios de $V$, entonces la suma
    $$W=W_1+W_2+\cdots + W_k$$
    como es fácil ver, es un subespacio de $V$ que contiene cada uno de los subespacios $W_i$. De esto se sigue, como en la demostración del Teorema 3, que $W$ es el subespacio generado por la unión de $W_1,W_2,\ldots , W_k$.
\end{def.}


\section*{Ejercicios}

\begin{enumerate}[\bfseries 1.]

    %-------------------- 1.
    \item ¿Cuál de los siguientes conjuntos de vectores $\alpha=(a_1,\ldots , a_n)$ de $\mathbb{R}^n$ son subespacios de $\mathbb{R}^n$ $(n\geq 3)$?

	\begin{enumerate}[\bfseries (a)]

	    %---------- (a)
	    \item todos los $\alpha$ tal que $a_1\geq 0$;\\\\
		Respuesta.-\; Sea $A=\left\{\alpha = (a_1,a_2,\ldots,a_n)\in F^n | a_1\geq 0\right\}$. De donde, podemos elegir un vector $\alpha'=(1,0,\ldots,0)\in A$ tal que $a_1\geq 0$. Ahora notemos que
		$$(-1)\alpha'=(-1)(1,0,\ldots,0)=(-1,0,\ldots,0)\notin A.$$
		lo que contradice el la condición $a_1\geq 0$. Por lo tanto, $A$ no es cerrado con respecto a la multiplicación escalar, por lo que no es un subespacio.\\\\

	    %---------- (b)
	    \item todos los $\alpha$ tal que $a_1+3a_2=a_3$;\\\\
		Respuesta.-\; Sean $\alpha=(a_1,a_2,\ldots,a_n)\in A$ y $\beta=(b_1,b_2,\ldots,b_n)\in A$. Ahora, consideremos $c\alpha+\beta$ de donde tenemos 

		$$
		\begin{array}{rcl}
		    c\alpha+\beta & = & (ca_1,c a_2,\ldots,c a_n)+ (b_1,b_2,\ldots,b_n)\\
		    & = & (ca_1+b_1,c a_2+b_2,\ldots,c a_n+b_n)\\
		\end{array}
		$$

		Después, usando la condición $a_1+3a_2=a_3$ se tiene, 

		$$
		\begin{array}{rcl}
		    (ca_1+b_1)+3(ca_2+b_2) &=& c(a_1+3a_2)+(b_1+3b_2)\\
					   &=& ca_3+b_3 
		\end{array}
		$$
		Por lo tanto, $c\alpha+\beta$ está en el subconjunto, entonces es un subconjunto por el teorema 1.\\\\

	    %---------- (c)
	    \item todos los $\alpha$ tal que $a_2=a_1^2$;\\\\
		Respuesta.-\; Tomemos los vectores $\alpha=(-1,1,0,\ldots,0)$ y $\beta=(1,1,0,\ldots,0)$. Luego,

		$$\alpha+\beta=(-1,1,0,\ldots, 0)+(1,1,0,\ldots)=(0,2,0,\ldots,0)$$

		Claramente $2\neq 0^2$, por lo que $\alpha+\beta$ no pertenece al subconjunto dado. Por lo tanto, no es un subespacio.\\\\

	    %---------- (d)
	    \item todos los $\alpha$ tal que $a_1a_2=0$;\\\\
		Respuesta.-\; Sea $\alpha=(0,1,0,\ldots,0)$ y $\beta=(1,0,0,\ldots,0)$. Luego,

		$$\alpha+\beta=(0,1,0,\ldots,0)+(1,0,0,\ldots,0)=(1,1,0,\ldots,0)$$

		Donde, claramente $1\cdot 1 = 1\neq 0$. Por lo tanto, $\alpha+\beta$ no pertenece al subconjunto dado. Así, no es un subespacio.\\\\

	    %---------- (e)
	    \item todos los $\alpha$ tal que $a_2$ es racional.\\\\
		Respuesta.-\; Sea el conjunto $A=\left\{\alpha=(a_1,a_2,\ldots,a_n)\in \mathbb{R}^n | a_2\in \mathbb{Q}\right\}$. Luego sea $\beta=(0,1,,0,\ldots,0)$. Entonces para $c=\sqrt{2}\in\mathbb{R}$, tenemos
		$$c\beta=\sqrt{2}(0,1,0,\ldots,0)=\left(0,\sqrt{2},0,\ldots,0\right)$$

		Claramente $\sqrt{2}\notin \mathbb{Q}$, por lo que $c\beta$ no pertenece al subconjunto dado. Por lo tanto, no es un subespacio.\\\\

	\end{enumerate}

    %-------------------- 2.
    \item Sea $V$ el espacio vectorial (real) de todas las funciones $f$ de $R$ en $R$. ¿Cuàl de los siguientes conjuntos de funciones son subespacios de $V$?.

	\begin{enumerate}[(a)]

	    %---------- (a)
	    \item Todas las $f$ tales que $f\left(x^2\right)=f^2(x)$;\\\\
		Respuesta.-\; Por el teorema 1, para que cada una de las funciones sea un subespacio, debe ser cerrado respecto a la suma y al multiplicación escalar en $V$ definida como: $f,g\in V$, con $c\in R$, tal que

		$$(cf+g)(x)=f(x)+g(x).$$

		Entonces, para este caso en particular. Sea $f(x)=x$ y $g(x)=x^2$, de donde ambos satisfacen la condición inicial; Es decir, 
		$$f\left(x^2\right)=x^2=f^2(x)\qquad \mbox{y}\qquad g\left(x^2\right)=\left(x^2\right)^2=g^2(x).$$
		Pero para algún $c\in R$,

		$$(cf+g)\left(x^2\right)=cf(x)+g(x)=cx^2+x^4$$

		Y 

		$$\left(cf+g\right)^2(x)=\left[cf(x)+g(x)\right]^2=(cx+x^2)^2=x^4+2cx^3+(cx)^2.$$

		Dado que no se cumple la igualdad de la condición $f\left(x^2\right)=f^2(x)$ para $(f+g)=f(x)+g(x)$. Entonces, el conjunto no es un subespacio de $V$.\\\\


	    %---------- (b)
	    \item Todas las $f$ tales que $f\left(0\right)=f(1)$;\\\\
		Respuesta.-\; Sean $f$ y $g$  dos funciones en $V$ que satisfacen la condición inicial. Entonces, para algún $c\in R$, se tiene

		$$(cf+g)(0)=cf(0)+g(0)=cf(1)+g(1)=(cf+g)(1).$$
		
		Por lo tanto, por el teorema 1, el conjunto dado es un subespacio de $V$.\\\\

	    %---------- (c)
	    \item Todas las $f$ tales que $f\left(3\right)=1+f(-5)$;\\\\
		Respuesta.-\; Por hipótesis, sean $f(3)=1+f(-5)$ y $g(3)=1+g(-5)$ en $V$. Entonces, para $c=1$, se tiene
		$$
		\begin{array}{rcl}
		    (cf+g)(3) &=& 1\cdot f(3)+g(3)\\
			     &=& 1+f(-5)+1+g(-5)\\
			     &=& 2+f(-5)+g(-5)\\
			     &=& 2+(f+g)(-5)\\
			     &\neq& 1+(f+g)(-5).

		\end{array}
		$$
		Por lo tanto, el conjunto dado no es un subespacio de $V$.\\\\

	    %---------- (d)
	    \item Todas las $f$ tales que $f\left(-1\right)=0$;\\\\
		Respuesta.-\; Sean, $f$ y $g$ dos funciones en $V$ que satisfacen la condición inicial; Es decir,

		$$f(-1)=0=g(-1).$$

		Entonces, para algún $c\in R$, se tiene

		$$
		\begin{array}{rcl}
		    (cf+g)(-1) &=& cf(-1)+g(-1)\\
			      &=& c\cdot 0+0\\
			      &=& 0.
		\end{array}
		$$

		Por lo tanto, el conjunto dado es un subespacio de $V$.\\\\

	    %---------- (e)
	    \item Todas las $f$ que son continuas.\\\\
		Respuesta.-\; Sea $f$ y $g$ dos funciones continuas en $V$. Entonces, para algún $c\in R$, $cf$ es continua. Lo mismo pasa para $f+c$, ya que la suma de funciones continuas es también continua.\\\\

	\end{enumerate}

    %-------------------- 3.
    \item ¿Pertenece el vector $(3,-1,0,-1)$ al subespacio de $R^5$ generado por los vectores $(2,-1,3,2)$, $(-1,1,1,-3)$ y $(1,1,9,-5)$?\\\\
	Respuesta.-\; Por el teorema 3, para que un vector pertenezca a un subespacio generado por un conjunto de vectores, debe ser una combinación lineal de los mismos. Entonces, para el vector $(3,-1,0,-1)$, tenemos

	$$
	\begin{array}{rcl}
	    (3,-1,0,-1) &=& x(2,-1,3,2)+y(-1,1,1,-3)+z(1,1,9,-5)\\\\
		        &=& (2x,-x,3x,2x)+(-y,y,y,-3y)+(z,z,9z,-5z)\\\\
			&=& (2x-y+z,-x+y+z,3x+y+9z,2x-3y-5z).
	\end{array}
	$$

	Resolviendo el sistema de ecuaciones mediante matrices, tenemos

	$$
	\begin{array}{rcl}
	    \left[
		\begin{array}{rrr|r}
		    2 & -1 & 1 & 3\\
		    -1 & 1 & 1 & -1\\
		    3 & 1 & 9 & 0\\
		    2 & -3 & -5 & -1
		\end{array}
	    \right]
	    &\dfrac{R_1}{2}\to R_1&
	    \left[
		\begin{array}{rrr|r}
		    1 & -\frac{1}{2} & \frac{1}{2} & \frac{3}{2}\\
		    -1 & 1 & 1 & -1\\
		    3 & 1 & 9 & 0\\
		    2 & -3 & -5 & -1
		\end{array}
	    \right]\\\\
	    &
	    \begin{array}{rcl}
		R_2 + R_1 &\to R_2\\
		R_3 - 3R_1 &\to R_3\\
		R_4 - 2R_1 &\to R_4
	    \end{array}
	    &
	    \left[
		\begin{array}{rrr|r}
		    1 & -\frac{1}{2} & \frac{1}{2} & \frac{3}{2}\\
		    0 & \frac{1}{2} & \frac{3}{2} & \frac{1}{2}\\
		    0 & \frac{5}{2} & \frac{15}{2} & -\frac{9}{2}\\
		    0 & -2 & -6 & -4
		\end{array}
	    \right]\\\\
	    &2R_2\to R_2&
	    \left[
		\begin{array}{rrr|r}
		    1 & -\frac{1}{2} & \frac{1}{2} & \frac{3}{2}\\
		    0 & 1 & 3 & 1\\
		    0 & \frac{5}{2} & \frac{15}{2} & -\frac{9}{2}\\
		    0 & -2 & -6 & -4
		\end{array}
	    \right]\\\\
	    &
	    \begin{array}{rcl}
		R_3 - \dfrac{5}{2}R_2 &\to R_3\\\\
		R_4 + 2R_2 &\to R_4\\\\
		R_1 + \dfrac{1}{2}R_2 &\to R_1
	    \end{array}
	    &
	    \left[
		\begin{array}{rrr|r}
		    1 & 0 & 2 & 2\\
		    0 & 1 & 3 & 1\\
		    0 & 0 & 0 & -7\\
		    0 & 0 & 0 & -2
		\end{array}
	    \right]

	\end{array}
	$$

	Ya que $0\neq -7$ y $0\neq -2$, el vector $(3,-1,0,-1)$ no pertenece al subespacio generado por los vectores $(2,-1,3,2)$, $(-1,1,1,-3)$ y $(1,1,9,-5)$.\\\\

    %-------------------- 4.
    \item Sea $W$ el conjunto de todos los $(x_1,x_2,x_3,x_4,x_5)$ de $R^5$ que satisfacen
    $$
    \begin{array}{ccccccccccc}
	2x_1 &-& x_2 &+& \frac{4}{3}x_3 &-& x_4 && &=& 0\\\\
	x_1 &&  &+& \frac{2}{3}x_3 &&  &-& x_5 &=& 0\\\\
	9x_1 &-&3x_2&+&6x_3&-&3x_4&-&3x_5&=&0 
    \end{array}
    $$
    Encontrar un conjunto finito de vectores que genera $W$.\\\\
	Respuesta.-\; Podemos escribir el sistema dado mediante la matriz y luego reducirla mediante la forma escalonada por filas, como sigue:
	$$
	\begin{array}{rcl}
	    \left[
		\begin{array}{rrrrr}
		    2 & -1 & 4/3 & -1 & 0\\
		    1 & 0 & 2/3 & 0 & -1\\
		    9 & -3 & 6 & -3 & -3
		\end{array}
	    \right]
	    & R_1 \leftrightarrow R_2 &
	    \left[
		\begin{array}{rrrrr}
		    1 & 0 & 2/3 & 0 & -1\\
		    2 & -1 & 4/3 & -1 & 0\\
		    9 & -3 & 6 & -3 & -3
		\end{array}
	    \right]\\\\
	    & R_2 - 2R_1 \to R_2 &
	    \left[
		\begin{array}{rrrrr}
		    1 & 0 & 2/3 & 0 & -1\\
		    0 & -1 & 0 & -1 & 2\\
		    9 & -3 & 6 & -3 & -3
		\end{array}
	    \right]\\\\
	    &R_3-9R_1 \to R_3& 
	    \left[
		\begin{array}{rrrrr}
		    1 & 0 & 2/3 & 0 & -1\\
		    0 & -1 & 0 & -1 & 2\\
		    0 & -3 & 0 & -3 & 6
		\end{array}
	    \right]\\\\
	    &-R_2\to R_2&
	    \left[
		\begin{array}{rrrrr}
		    1 & 0 & 2/3 & 0 & -1\\
		    0 & 1 & 0 & 1 & -2\\
		    0 & -3 & 0 & -3 & 6
		\end{array}
	    \right]\\\\
	    &R_3+3R_2\to R_3&
	    \left[
		\begin{array}{rrrrr}
		    1 & 0 & 2/3 & 0 & -1\\
		    0 & 1 & 0 & 1 & -2\\
		    0 & 0 & 0 & 0 & 0 
		\end{array}
	    \right]

	\end{array}
	$$

	Por lo tanto, 

	$$
	\begin{array}{rcl}
	    x_1 &=& x_5-\frac{2}{3}x_3\\\\
	    x_2&=& x_4 + 2x_5.
	\end{array}
	$$

	Así, el conjunto que genera $W$ estará dada por la forma:

	$$W=\left\{\left(x_5-\dfrac{2}{3}x_3,x_4+2x_5,x_3,x_4,x_5\right)\in R^5\right\}$$

	De lo que, un conjunto generado para $W$ viene dado por los vectores:

	$$(-2,0,3,0,0),\quad (0,-1,0,1,0)\quad y \quad (1,2,0,0,1).$$\\

    %-------------------- 5.
    \item Sean $F$ un cuerpo y $n$ un entero postivo $(n\geq 2)$. Sea $V$ el espacio vectorial de todas las matrices $n\times n$ sobre $F$. ¿Cuáles de los siguientes conjuntos de matrices $A$ de $V$ son subespacios de $V$?\\

	\begin{enumerate}[(a)]

	    %---------- (a)
	    \item todas las $A$ inversibles;\\\\
		Respuesta.-\; Supongamos que $A\in W_1$ es invertible. Entonces, para $c=0\in F$ tenemos $0\cdot A = 0$, donde $0$ es la matriz y este no es invertible. Por lo tanto, $0\cdot A \notin W_1$. Es decir, $W_1$ no es cerrado respecto a la multiplicación escalar.\\\\

	    %---------- (b)
	    \item todas las $A$ no inversibles;\\\\
		Respuesta.-\; Sean dos matrices $2\times 2$ sobre $F$, tal que no son invertibles:
		$$
		A=
		\left[
		    \begin{array}{rr}
			1 & 0\\
			0 & 0
		    \end{array}
		\right]
		\quad y \quad B=
		\left[
		    \begin{array}{rr}
			0 & 0\\
			0 & 1
		    \end{array}
		\right]
		$$

		Pero la suma de estas dos matrices es la matriz identidad, es decir, $A+B=I$, es invertible. Por lo que la suma no es cerrada respecto a la adición.\\\\


	    %---------- (c)
	    \item todas las $A$ para las que $AB=BA,$ donde $B$ es cierta matriz dada de $V$;\\\\
		Respuesta.-\; Sean, $A_1$ y $A_2$ dos matrices en $V$ tal que $A_1B=BA_2$ y $A_2B=BA_2$. Entonces, por el definición de subespacio, el lema 1 y el teorema 1, para algún $c\in F$:
		$$
		\begin{array}{rcl}
		    (cA_1+A_2)B &=& cA_1B+A_2B\\\\
				&=& cBA_1+BA_2\\\\
				&=& B(cA_1)+BA_2\\\\
				&=& B(cA_1+A_2).
		\end{array}
		$$

		Por lo tanto, este conjunto es un subespacio de $V$.\\\\

	    %---------- (d)
	    \item todas las $A$ para las que $A^2=A$.\\\\
		Respuesta.-\; Sea, $c\in F$ tal que $c\neq 1$. Entonces,
		$$(cA)^2=c^2A^2=c^2A\neq cA.$$
		Ya que no es cerrado bajo la multiplicación escalar, concluimos que el conjunto dado no es un subespacio de $V$.\\\\

	\end{enumerate}

    %-------------------- 6.
    \item 
	\begin{enumerate}[(a)]

	    %---------- (a)
	    \item Demostrar que los únicos subespacios de $R^1$ son $R^1$ y el subespacio nulo.\\\\
		Demostración.-\; Supongamos que $A$ está en el subespacio de $R^1$, si $A=\left\{0\right\}$, entonces hemos terminado la demostración, Ahora, si $A$ no es un subespacio cero, sea $x\in R^1$, tal que $x\in A$. Entonces, $\alpha x\in A$ para cada $\alpha\in R$. Luego, elija cualquier $y\in R^1$. Existe $\alpha'$ tal que $y=\alpha'x$. de donde $y\in A$. Así, concluimos que $A$ es igual a $R^1$.\\\\

	    %---------- (b)
	    \item Demostrar que un subespacio de $R^2$ es $R^2$ o el subespacio nulo o consta de todos los múltiplos escalares de algún vector fijo de $R^2$. (El último tipo de subespacio es, intuitivamente, una recta por el origen.)\\\\
		Demostración.-\; Sea $B$ un subespacio de $R^2$. Si $B=\left\{0,0\right\}$, entonces hemos terminado la demostración. Ahora, si $B$ no es un subespacio cero, sea $R^2$, tales que $x_1,x_2\neq 0$ y $x\in A$. Entonces, $\alpha x=(\alpha x_1,\alpha x_2)\in A$ para cada $\alpha\in R$. Concluimos que $B$ consiste en todos los multiplos scalares del vector fijo $x\in R^2$.\\\\

	    %---------- (c)
	    \item ¿Puede usted describir los subespacios de $R^3$?.\\\\
		Demostración.-\; Los subespacios de $R^3$ son el subespacio cero, el conjunto de todos los múltiplos escalares de un vector fijo distinto de cero (es decir, una línea que pasa por el origen), el conjunto de todas las combinaciones lineales de dos vectores linealmente independientes (es decir, un plano que pasa por el origen) y el propio $R^3$.\\\\

	\end{enumerate}

    %-------------------- 7.
    \item Sean $W_1$ y $W_2$, subespacios de un espacio vectorial $V$ tal que la unión conjuntista de $W_1$ y $W_2$ sea también un subespacio. Demostrar que uno de los espacios $W_i$ está contenido en el otro.\\\\
	Demostración.-\; Sea $W_1$ y $W_2$ ninguno contenido en el otro. De donde, existe un vector $u\in W_1$ tal que $u\notin W_2$ y existe un vector $v\in W_2$ tal que $v\notin W_1$. Ya que $W_1\cup W_2$ es un subespacio, entonces $u+v\in W_1\cup W_2$. Ahora, $u+v\in W_1$ o $u+v\in W_2$. En el primer caso, como $-u\in W_1$ debemos tener
	$$(u+v)+(-u)=v\in W_1,$$
	lo que es una contradicción. En el segundo caso, como $-v\in W_2$ debemos tener
	$$(u+v)+(-v)=u\in W_2,$$
	lo que es también una contradicción. Por lo tanto, uno de los subespacios $W_i$ debe estar contenido en el otro.\\\\

    %-------------------- 8.
    \item Sea $V$ el espacio vectorial de todas las funciones de $R$ en $R$; sea $V_p$ el subconjunto de las funciones pares, $f(-x)=f(x)$; sea $V_i$ el subconjunto de las funciones impares $f(-x)=-f(x)$.\\
	\begin{enumerate}[(a)]

	    %---------- (a)
	    \item Demostrar que $V_p$ y $V_i$ son subespacios de $V$.\\\\
		Demostración.-\; Sean $f$ y $g$ funciones pares. Entonces, para cualquier escalar $c$,

		$$
		\begin{array}{rcl}
		    (cf+g)(-x)&=&cf(-x)+g(-x)\\
			      &=&cf(x)+g(x)\\
			      &=&(cf+g)(x).\\
		\end{array}
		$$

		Así, $cf+g$ es también una función par y por lo tanto $V_p$ es un subespacio de $V$. Similarmente, si $f$ y $g$ funciones impares. Entonces,

		$$
		\begin{array}{rcl}
		    (cf+g)(-x)&=&cf(-x)+g(-x)\\
			      &=&-cf(x)-g(x)\\
			      &=&-(cf+g)(x).\\
		\end{array}
		$$

		Así, $V_i$ es también un subespacio de $V$.\\\\

	    %---------- (b)
	    \item Demostrar que $V_p + V_i=V$.\\\\
		Demostración.-\; Sea $f\in V$ una función cualquiera. Sean también $g$ una función en $V$ definida por 
		$$g(x)=\dfrac{f(x)+f(-x)}{2}$$
		y $h$ una función en $V$ definida por
		$$h(x)=\dfrac{f(x)-f(-x)}{2}.$$
		Es claro que $g\in V_p$ y $h\in V_i$. Ya que, $f(x)=g(x)+h(x)$ para todo $x$, entonces $V=V_p+V_i$.\\\\


	    %---------- (c)
	    \item Demostrar que $V_p \cap V_i=\left\{0\right\}$.\\\\
		Demostración.-\; Sea $f\in V_p\cap V_i$ y sea $x\in R$. Ya que $f$ es par, entonces $f(x)=f(-x)$. Y ya que $f$ es impar, entonces $f(-x)=-f(x)$. Por lo tanto, tenemos $f(x)=-f(x)$, el cual es posible si $f(x)=0.$ Ya que se tiene cualquier $x$, $f$ debe ser la función cero. Esto muestra que $V_p\cap V_i$ es el espacio cero.\\\\

	\end{enumerate}

    %-------------------- 9.
    \item Sea $W_1$ y $W_2$ subespacios de un espacio vectorial $V$ tales que $W_1+W_2=V$ y $W_1\cap W_2=\left\{0\right\}.$ Demostrar que para todo vector $\alpha$ de $V$ existen únicos vectores $\alpha_1$ e $W_1$ y $\alpha_2$ en $W_2$ tales que $\alpha=\alpha_1+\alpha_2$.\\\\
	Demostración.-\; Ya que $W_1+W_2=V,$ podríamos definir $\alpha_1\in W_1$ y $\alpha_2\in W_2$ tal que $\alpha=\alpha_1+\alpha_2$. Ahora, suponga que existe también $\alpha_3\in W_1$ y $\alpha_4\in W_2$ con $\alpha=\alpha_3+\alpha_4$. Entonces,
	$$\alpha_1+\alpha_2=\alpha_3+\alpha_4 \quad \Rightarrow \quad \alpha_1-\alpha_3=\alpha_4-\alpha_2.$$
	Pero el vector del lado izquierdo debe pertenecer a $W_1$, y el vector del lado derecho debe pertenecer a $W_2$. Por lo tanto, $\alpha_1-\alpha_3$ pertenece a la intersección de $W_1$ y $W_2$, lo que implica que $\alpha_1-\alpha_3$ o $\alpha_1=\alpha_3$. Y $\alpha_4=\alpha_2$ también. Esto muestra que los vectores $\alpha_1$ y $\alpha_2$ son únicos.\\\\

\end{enumerate}


\section{Bases y dimensión}

Pasamos, ahora a la tarea de dotar de dimensión a ciertos espacios vectoriales. Debemos encontrar una definición algebraica apropiada para la dimensión de espacio vectorial. Esto se hará mediante el concepto de base de un espacio vectorial.

%-------------------- definición 1.6.
\begin{def.}
    Sea $V$ un espacio vectorial sobre $F$. Un subconjunto $S$ de $V$ se dice $\textbf{linealmente dependientes}$ (o simplemente, $\textbf{dependiente}$) si existen vectores distintos $\alpha_1,\alpha_2,\ldots,\alpha_n$ de $S$ y escalares $c_1,c_2,\ldots,c_n$ de $F$, no todos nulos, tales que
    $$c_1\alpha_1+c_2\alpha_2+\cdots+c_n\alpha_n=0.$$
    Un conjunto que no es linealmente dependiente se dice $\textbf{linealmente independiente}$. Si el conjunto $S$ tiene solo un número finito de vectores $\alpha_1,\alpha_2,\ldots,\alpha_n$ se dice, a veces, que los $\alpha_1,\alpha_2,\ldots,\alpha_n$ son dependientes (o independientes), en vez de decir que $S$ es dependiente (o independiente).
\end{def.}

Las siguientes son fáciles consecuencias de la definición.

\begin{enumerate}[1.]

    \item Todo conjunto que contiene un conjunto linealmente dependiente es linealmente dependiente.
    \item Todo subconjunto de un conjunto linealmente independiente es linealmente independiente.
    \item Todo conjunto que contiene el vector $0$ es linealmente dependiente; en efecto, $1\cdot 0=0$.
    \item Un conjunto $S$ de vectores es linealmente independiente si, y sólo si, todo subconjunto finito de $S$ es linealmente independiente; es decir, si, y sólo si, para vectores diferentes $\alpha_1,\ldots,\alpha_n$ de $S$, arbitrariamente elegido $c_1\alpha_1+\cdots+c_n\alpha_n=0$ implica que todo $c_i=0$.

\end{enumerate}

%-------------------- definición 1.7.
\begin{def.}
    Sea $V$ un espacio vectorial. Una \textbf{base} de $V$ es un conjunto de vectores linealmente independiente de $V$ que genera el espacio $V$. El espacio $V$ es de \textbf{dimensión finita} si tiene una base finita.
\end{def.}


\setcounter{ejem}{12}
%-------------------- Ejemplo 13
\begin{ejem}
    Sea $F$ un cuerpo, y en $F^n$ sea $S$ el subconjunto que consta de los vectores $\epsilon_1,\epsilon_2,\ldots,\epsilon_n$ definidos por
    $$
    \begin{array}{ccc}
	\epsilon_1&=&(1,0,\ldots,0)\\
	\epsilon_2&=&(0,1,\ldots,0)\\
	\vdots&&\vdots\\
	\epsilon_n&=&(0,0,\ldots,1).
    \end{array}
    $$
    Sean $x_1,x_2,\ldots,x_n$ escalares de $F$, y hágase $\epsilon=x_1\epsilon_1+x_2\epsilon_2+\cdots+x_n\epsilon_n$. Entonces
    $$\epsilon=(x_1,x_2,\ldots,x_n).$$
    Esto muestra que $\epsilon_1,\cdots,\epsilon_n$ genera $F^n$. Como $\epsilon=0$ si, y sólo si, $x_1=x_2=\ldots = x_n=0$, los vectores $\epsilon_1,\ldots,\epsilon_n$ son linealmente independientes. El conjunto $S=\left\{\epsilon_1,\ldots,\epsilon_n\right\}$ es, por tanto, una base de $F^n$. Esta base particular se llamará \textbf{base canónica} de $F^n$.

\end{ejem}


%-------------------- ejemplo 14
\begin{ejem}
    Sea $P$ una matriz $n\times n$ inversible con elementos en el cuerpo $F$. Entonces $P_1,\ldots,P_n$, las columnas de $P$, forman una base del espacio de las matrices columnas $F^{n\times 1}$. Eso se verá como sigue. Si $X$ es una matriz columna, entonces 
    $$PX=x_1P_1+\cdots + x_nP_n.$$

    Como $PX=0$ tiene solo la solución trivial $X=0$, se sigue que $\left\{P_1,\ldots,P_n\right\}$ es un conjunto linealmente independiente. ¿Por qué generan $F^{n\times 1}$? Sea $Y$ cualquier matriz columna. Si $X=P^{-1}Y$, entonces $Y=PX$, esto es
    $$Y=x_1P_1+\cdots + x_nP_n.$$

    Así, $\left\{P_1,\ldots,P_n\right\}$ es una base de $F^{n\times 1}$.
\end{ejem}

%-------------------- ejemplo 15
\begin{ejem}
    Sea $A$ una matriz $m\times n$ y sea $S$ el espacio solución del sistema homogéneo $AX=0$ (Ejemplo 7). Sea $R$ una matriz escalón reducida por filas que es equivalente por filas a $A$. Entonces $S$ es también el espacio solución del sistema $RX=0$. Si $R$ tiene $r$ filas no nulas, entonces el sistema de ecuaciones $RX=0$ simplemente expresa $r$ de las incógnitas $x_1,\ldots,x_n$ en términos de las $(n-r)$ incógnitas $x_j$ restantes. 
\end{ejem}

%-------------------- teorema 4
\begin{teo}
    Sea $V$ un espacio vectorial generado por un conjunto finito de vectores $\beta_1,\beta_2,\ldots,\beta_m$. Entonces, todo conjunto independientes de vectores de $V$ es finito y no contiene más de $m$ elementos.\\\\
    Demostración.-\; Para demostrar este teorema es suficiente ver que todo subconjunto $S$ de $V$ que contiene más de $m$ vectores es linealmente dependiente. Sea $S$ un tal conjunto. En $S$ hay vectores diferentes $\alpha_1,\alpha_2,\ldots,\alpha_n$ donde $n>m$. Como $\beta_1,\beta_2,\ldots,\beta_m$ generan $V$, existen escalares $A_{ij}$ en $F$ tales que
    $$\alpha_j=\sum_{i=1}^m A_{ij}\beta_i.$$
    Para cualquier $n$ escalares $x_1,x_2,\ldots,x_n$ se tiene
    $$
    \begin{array}{rcl}
	x_1\alpha_1+\cdots+x_n\alpha_n &=& \displaystyle\sum_{j=1}^n x_j\alpha_j\\\\
				       &=& \displaystyle\sum_{j=1}^n x_j\left(\sum_{i=1}^m A_{ij}\beta_i\right)\\\\
				       &=&\displaystyle\sum_{i=1}^n\sum_{j=1}^m \left(A_{ij}x_j\right)\beta_i\\\\
	&=&\displaystyle\sum_{i=1}^m\left(\sum_{j=1}^n A_{ij}x_j\right)\beta_i.
    \end{array}
    $$
    Como $n>m$, el teorema 6 del capítulo 1 implica que existen escalares $x_1,x_2,\ldots,x_n$, no todos $0$, tales que
    $$\sum_{j=1}^nA_{ij}x_j=0,\quad 1\leq i \leq m.$$
    Luego $x_1\alpha_1+x_2\alpha_2+\cdots+x_n\alpha_n=0$. Ello demuestra que $S$ es un conjunto linealmente dependiente.
\end{teo}

%-------------------- corolario 1
\begin{cor}
    Si $V$ es un espacio vectorial de dimensión finita, entonces dos bases cualesquiera de $V$ tiene el mismo número (finito) de elementos.\\\\
	Demostración.-\; Como $V$ es de dimensión finita, tiene una base finita
	$$\left\{\beta_1,\beta_2,\ldots,\beta_m\right\}.$$
	Por el teorema 4 toda base de $V$ es finita y contiene no más de $m$ elementos. Así, si $\left\{\alpha_1,\alpha_2,\ldots,\alpha_n\right\}$ es una base, $n\leq m$. Por el mismo razonamiento, $m\leq n$. Luego $m=n$.
\end{cor}

Este corolario permite definir la dimensión de un espacio vectorial de dimensión finita como el número de elementos de una base cualquier de $V$. Se indicará la dimensión de un espacio $V$ de dimensión finita por $\dim V$. Ello nos permite volver a enunciar el Teorema 4 como sigue:

%-------------------- corolario 2
\begin{cor}
    Sea $V$ un espacio vectorial de dimensión finita y sea $n=\dim V$. Entonces,
    \begin{enumerate}[(a)]
	\item cualquier subconjunto de $V$ que contenga más de $n$ vectores es linealmente dependiente;
	\item ningún subconjunto de $V$ que contenga menos de $n$ vectores puede generar $V$.
    \end{enumerate}
\end{cor}

%-------------------- Lema 2
\begin{lema}
    Sea $S$ un subconjunto linealmente independiente de un espacio vectorial $v$. Supóngase que $\beta$ es un vector de $V$ que no pertenece al subespacio generado por $S$. Entonces, el conjunto que se obtiene agregando $\beta$ a $S$, es linealmente independiente.\\\\
	Demostración.-\; Supóngase que $\alpha_1,\ldots,\alpha_m$ son vectores distintos de $S$ y que 
	$$c_1\alpha_1+\cdots+c_m\alpha_m+b\beta=0.$$
	Entonces $b=0$; de otra manera
	$$\beta=\left(-\dfrac{c_1}{b}\right)\alpha_1+\cdots+\left(-\dfrac{c_m}{b}\right)\alpha_m$$
	y $\beta$ está en el subespacio generado por $S$; lo que contradice con la hipótesis. Así, $c_1\alpha_1+\cdots+c_m\alpha_m=0$, y como $S$ es un conjunto linealmente independientes, todo $c_i=0$.
\end{lema}

%-------------------- teorema 5
\begin{teo}
    Si $W$ es un subespacio de un espacio vectorial de dimensión finita $V$, todo subconjunto linealmente independiente de $W$ es finito y es parte de un base (finita) de $W$.\\\\
	Demostración.-\; Supóngase que $S_0$ es un conjunto linealmente independientes de $W$. Si $S$ es un subconjunto linealmente independiente de $W$ que contiene a $S_0$, entonces $S$ también es un subconjunto linealmente independiente de $V$; como $V$ es de dimensión finita, $S$ no tiene más de $\dim V$ elementos.\\
	Se extiende $S_0$ a una base de $W$, como sigue. Si $S_0$ genera $W$, entonces $S_0$ es una base de $S_0$ y está demostrado. Si $S_0$ no genera $W$, por el lema anterior se halla un vector $\beta_1$ en $W$ tal que el conjunto $S_1=S_0\cup \left\{\beta_1\right\}$ es independiente. Si $S_1$ genera $W$, está demostrado. Si no, se aplica el lema para obtener un vector $\beta_2$ en $W$ tal que $S_2=S_1\cup \left\{\beta\right\}$ es independiente. Si se continúa de este modo entonces (y en no más de $\dim V$ de etapas) se llega a un conjunto
	$$S_m=S_0\cup \left\{\beta_1,\ldots,\beta_m\right\}$$
	que es una base de $W$.
\end{teo}

%-------------------- corolario 3
\begin{cor}
    Si $W$ es un subespacio propio de un espacio vectorial de dimensión finita $V$, entonces $W$ es de dimensión finita y $\dim W<\dim V$.\\\\
	Demostración.-\; Podemos suponer que $W$ contiene un vector $\alpha\neq 0$. Por el teorema 5 y su demostración existe una base de $W$ que, conteniendo a $\alpha$, no contiene más que $\dim V$ elementos. Luego $W$ es de dimensión finita y $\dim W\leq \dim V$. Como es un subespacio propio existe un vector $\beta$ en $V$ que no está en $W$. Agregando $\beta$ a cualquier base de $W$ se obtiene un subconjunto linealmente independiente de $V$. Así $\dim V < \dim V$.
\end{cor}

%-------------------- corolario 4
\begin{cor}
    En un espacio vectorial $V$ de dimensión finita todo conjunto linealmente independiente de vectores es parte de una base.
\end{cor}

%-------------------- corolario 5
\begin{cor}
    Sea $A$ una matriz $n\times n$ sobre el cuerpo $F$, y supóngase que los vectores fila de $A$ forman un conjunto linealmente independiente de vectores de $F^n$. Entonces $A$ es inversible.\\\\
	Demostración.-\; Sean $\alpha_1,\alpha_2,\ldots,\alpha_n$ vectores fila de $A$, y supóngase que $W$ es un subespacio de $F^n$ generado por $\alpha_1,\alpha_2,\ldots,\alpha_n$. Como los $\alpha_1,\alpha_2,\ldots.\alpha_n$ son linealmente independientes, la dimensión de $W$ es $n$. Por el corolario 3 se tiene que $W=F^n$. Luego existen escalares $\beta_{ij}$ tales que
	$$\epsilon_i=\sum_{j=1}^n B_{ij}\alpha_j, \quad 1\leq i \leq n$$
	donde $\left\{\epsilon_1,\epsilon_2,\ldots,\epsilon_n\right\}$ es la base canónica de $F^n$. Así, para la matriz $B$ de elementos $B_{ij}$ se tiene 
	$$BA=I.$$
\end{cor}

%-------------------- teorema 6
\begin{teo}
    Si $W_1$ y $W_2$ son subespacios de dimensión finita de un espacio vectorial, entonces $W_1+W_2$ es de dimensión finita y
    $$\dim W_1+\dim W = \dim (W_1\cap W_2)+\dim (W_1+W_2).$$\\
	Demostración.-\; Por el teorema 5 y sus corolarios, $W_1\cap W_2$ tiene una base finita $\left\{\alpha_1,\ldots,\alpha_k\right\}$ qie es parte de la base
	$$\left\{\alpha_1,\ldots,\alpha_k,\; \beta_1,\ldots,\beta_m\right\}\quad \mbox{para }W_1$$
	y parte ed la base
	$$\left\{\alpha_1,\ldots,\alpha_k,\; \gamma_1,\ldots,\gamma_m\right\}\quad \mbox{para }W_2.$$
	El subespacio $W_1+W_2$ es generado por los vectores
	$$\alpha_1,\ldots,\alpha_k,\quad \beta_1,\ldots,\beta_m,\quad \gamma_1,\ldots,\gamma_n.$$
	y estos vectores forman un conjunto independiente. En efecto, supóngase que 
	$$\sum x_i\alpha_i + \sum y_j\beta_j + \sum z_r\gamma_r=0.$$
	Entonces 
	$$-\sum z_r\gamma_r = \sum x_i\alpha_i+\sum y_j\beta_j$$
	que muestra que $\sum z_r\gamma_r$ pertences a $W_1$. Como $\sum z_r\gamma_r$ también pertence a $W_2$ se sigue que
	$$\sum z_r\gamma_r = \sum c_i\alpha_i$$
	para ciertos escalares $c_1,\ldots,c_k$. Como el conjunto 
	$$\left\{\alpha_1,\ldots,\alpha_k,\quad \gamma_1,\ldots,\gamma_n\right\}$$
	es independiente, cada uno de los escalares $z_r=0$. Así, 
	$$\sum x_i\alpha_i + \sum y_j\beta_j=0$$
	y como 
	$$\left\{\alpha_1,\ldots,\alpha_k,\quad \beta_1,\ldots,\beta_m\right\}$$
	es también un conjunto independiente, dada $x_i=0$ y cada $y_i=0$. Así
	$$\left\{\alpha_1,\ldots,\alpha_k,\quad \beta_1,\ldots,\beta_m,\quad \gamma_1,\ldots,\gamma_n\right\}$$
	es una base para $W_1+W_2$. Finalmente,
	$$
	\begin{array}{rcl}
	    \dim W_1+\dim W_2 &=& (k+m)+(k+n)\\
			      &=& k+(m+k+n)\\
			      &=& \dim(W_1\cap W_2)+\dim(W_1+W_2).
	\end{array}
	$$
\end{teo}


\section*{Ejercicios}

\begin{enumerate}[\bfseries 1.]

    %-------------------- 1.
    \item Demostrar que si dos vectores son linealmente dependientes, uno de ellos es un múltiplo escalar del otro.\\\\
	Demostración.-\; Sean $v_1$ and $v_2$ don vectores linealmente dependientes en $V$. Por definición, existen escalares $a_1$ y $a_2$ ambos no cero tal que
	$$a_1v_1+a_2v_2=0.$$
	Si $a_1\neq 0$, entonces podemos escribir la ecuación anterior como
	$$v_1=-\frac{a_2}{a_1}v_2.$$
	Por lo que $v_1$ es un múltiplo escalar de $v_2$.\\\\

    %-------------------- 2.
    \item ¿Son los vectores
    $$\alpha_1=(1,1,2,4),\quad \alpha_2=(2,-1,-5,2),\quad \alpha_3=(1,-1,-4,0),\quad \alpha_4=(2,1,1,6)$$
    linealmente independientes?\\\\
	Respuesta.-\; Sean, los escalares $a,b,c,d\in \textbf{F}$, entonces
	$$
	\begin{array}{rcl}
	    a(1,1,2,4)+b(2,-1,-5,2)+c(1,-1,-4,0)+d(2,1,1,6) &=& 0\\\\
	    (a+2b+c+2d,a-b-c+d,2a-5b-4c+d,4a+2b+6d) &=& 0.
	\end{array}
	$$
	De donde,
	$$
	\left\{
	    \begin{array}{rcl}
		a+2b+c+2d &=& 0\\\\
		a-b-c+d &=& 0\\\\
		2a-5b-4c+d &=& 0\\\\
		4a+2b+6d &=& 0.
	    \end{array}
	\right.
	$$
	Resolviendo, tenemos 
	$$a=\dfrac{c-4d}{3},\quad b=\dfrac{-2c-d}{3}.$$
	Supongamos que $c=d=1$. Entonces, 
	$$a=b=-1.$$
	Dado que $a,b,c,d\in \textbf{F}$ son distintos de cero, concluimos que los vectores dados no son linealmente independientes.\\\\ 

    %-------------------- 3.
    \item Hallar una base para el subespacio de $R^4$ generado por los cuatro vectores del Ejercicio 2.\\\\
	Respuesta.-\; 

\end{enumerate}
