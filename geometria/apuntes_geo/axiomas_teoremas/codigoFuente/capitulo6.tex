\chapter{El axioma de los paralelos}

%--------------------axioma 5
\begin{axioma}
    Para un punto fuera de la recta $m$, se puede trazar una sola recta paralela a la recta $m$.\\\\    
\end{axioma}

	    %---------- proposición 6.1
	    \begin{proposicion}
		Si la recta $m$ es paralela a las rectas $n_1$ y $n_2$, entonces $n_1$ y $n_2$ son paralelas o coincidentes\\
	    \end{proposicion}

	%----------corolario 6.1
	\begin{cor}
	    Si una recta corta uno de dos paralelos, también corta otro.\\
	\end{cor}

	    %----------proposición 6.2
	    \begin{proposicion}
		Sean $m$, $n$, $\widehat{1}$ y $\widehat{2}$ como en la figura $(6.1)$. Si $\widehat{1} = \widehat{2}$, entonces las rectas $m$ y $n$ son paralelas.\\
	    \end{proposicion}

	    %----------proposición 6.3
	    \begin{proposicion}
		Si, al cortar dos rectas con una transversal, obtenemos $\widehat{3} + \widehat{2} = 180^{\circ}$ entonces las rectas son paralelas.\\
	    \end{proposicion}

	    %----------proposición 6.4
	    \begin{proposicion}
		Si, cuando cortamos dos rectas con una transversal, los ángulos correspondientes son iguales, entonces las rectas son paralelas.\\
	    \end{proposicion}

	    %----------proposición 6.5
	    \begin{proposicion}
		Si dos rectas paralelas están cortadas por una transversal, entonces los ángulos correspondientes son iguales.\\
	    \end{proposicion}

    %----------teorema 6.2
    \begin{teo}
	La suma de los ángulos internos de un triángulo es $180^{\circ}$.\\
    \end{teo}

	%----------corolario 1.2
	\begin{cor}
	    \begin{enumerate}[\bfseries a)]
		\item La suma de las medidas de los ángulos agudos de un triángulo rectángulo es $90^{circ}$.
		\item Cada ángulo de un triángulo equilátero mide $60^{\circ}$.
		\item a medida de un ángulo externo de un triángulo es igual a la suma de las medidas de los ángulos internos que no son adyacentes a él.
		\item La suma de los ángulos internos de una cuadrilátero es $360^{\circ}$.\\
	    \end{enumerate}
	\end{cor}

    %----------teorema 6.3
    \begin{teo}
	Si $m$ y $n$ son rectas paralelas, entonces todos los puntos de $m$ están a la misma distancia de la recta $n$.\\
    \end{teo}

	    %----------proposición 6.6
	    \begin{proposicion}
		En un paralelogramo, los lados y ángulos opuestos son congruentes.\\
	    \end{proposicion}

    %----------definición 1.1
    \begin{def.}
	Un paralelogramo es un cuadrilátero cuyos lados opuestos son paralelos.\\
    \end{def.}

	    %----------proposición 6.7
	    \begin{proposicion}
		En un paralelogramo, los lados y ángulos opuestos son congruentes.\\
	    \end{proposicion}

	    %----------proposición 6.8
	    \begin{proposicion}
		Las diagonales de un paralelogramo se cruzan en un punto que es el punto medio de las dos diagonales.\\
	    \end{proposicion}

	    %----------proposición 6.8
	    \begin{proposicion}
		Si dos lados opuestos de un cuadrilátero son congruentes y paralelos, entonces el cuadrilátero es un paralelogramo.\\
	    \end{proposicion}

    %----------teorema 6.4
    \begin{teo}
	El segmento que conecta los puntos medios en dos lados de un triángulo es paralelo al tercer lado y tiene la mitad de su longitud.\\
    \end{teo}

	%----------proposición 6.9
	\begin{proposicion}
	    Suponga que tres rectas paralelas, $a,b$ y $c$ cortan las rectas $m$ y $n$ en los puntos $A,B$ y $C$ y en los puntos $A^{'}, B^{'}$ y $C^{'}$ respectivamente. Si el punto $B$ está entre $A$ y $C$, entonces el punto $B^{'}$ también está entre $A^{'}$ y $C^{'}$. Si $AB=BC$, entonces también hay $A^{'} B^{'} = B^{'} C^{'}$\\ 
	\end{proposicion}

    %----------corolario 1.3
    \begin{cor}
	Suponga que $k$ rectas paralelas $a_1,a_2,...,a_k$ cortan dos rectas $m$ y $n$ en los puntos $A_1, A_2,...,A_k$ y en los puntos $A_1^{'},A_2^{'},..., A_k^{'}$, respectivamente. Si $A_1A_2,...,A_2 A_3 = A_{k-1} A_k$ entonces $A_1^{'}A_2^{'}=A_2^{'}A_3^{'}=A_{k-1}^{'}A_k^{'}$
    \end{cor}

    %----------teorema 6.5
    \begin{teo}
	Si una recta, paralela a un lado de un triángulo, corta los otros dos lados, entonces se divide en la misma proporción.\\
    \end{teo}
