\chapter{El eje de la incidencia y el orden}

%-------------------axioma I.1
\begin{axioma}
    Cualquiera que sea la recta, hay puntos que pertenecen a la recta y puntos que no pertenecen a la recta.\\
\end{axioma}

%-------------------axioma I.2
\begin{axioma}
    Dados dos puntos distintos, hay una sola recta que contiene estos puntos.\\
\end{axioma}

	%----------proposición 1.1.
	\begin{proposicion}
	    Dos líneas distintas no se cruzan o se cruzan en un solo punto.\\
	\end{proposicion}

\setcounter{part}{2}
%--------------------axioma II.1
\begin{axioma}
   Dados tres puntos de una recta, uno y solo uno de ellos se ubica entre los otros dos.\\
\end{axioma}

    %----------definición 1.2
    \begin{def.}
	El conjunto que consta de dos puntos $A$ y $B$ y todos los puntos entre $A$ y $B$ se llama segmento $AB$. Los puntos $A$ y $B$ se denominan extremos o extremos del segmento.\\
    \end{def.}

    %----------definición 1.3
    \begin{def.}
	Si $A$ y $B$ son puntos distintos, el conjunto que consta de los puntos del segmento $AB$ y todos los puntos $C$ tales que $B$ está entre $A$ y $C$, se denomina semi-recta de origen $A$ que contiene el punto $B$ y está representado por $S_{AB}$. El punto $A$ se llama entonces el origen del $S_{AB}$ semi-recto.\\
    \end{def.}

	%----------proposición 1.2
	\begin{proposicion}.\\
	    \begin{enumerate}[\bfseries a)] 
		\item $S_{AB} \cup S_{BA}$ y la recta determinada por $A$ y $B$.
		\item $S_{AB} \cap S_{BA} = AB$.\\
	    \end{enumerate}
	\end{proposicion}

%--------------------axioma II_2
\begin{axioma}
    Dados dos puntos $A$ y $B$ siempre existe un punto $C$ entre $A$ y $B$ y un punto $D$ tal que $B$ está entre $A$ y $D$.\\
\end{axioma}

    %----------definición 1.4
    \begin{def.}
	Sea $m$ una recta y $A$ un punto que no pertenece a $m$. El conjunto que consta de los puntos de $m$ y todos los puntos $B$ tales que $A$ y $B$ están en el mismo lado de la recta $m$ es llamado  semiplano  determinado por $m$ que contiene  a $A$, y estará representado por $P_mA$.\\
    \end{def.}

%--------------------axiomaII_3
\begin{axioma}
   Una recta $m$ determina exactamente dos semiplanos distintos cuya intersección es la recta $m$.\\ 
\end{axioma}
