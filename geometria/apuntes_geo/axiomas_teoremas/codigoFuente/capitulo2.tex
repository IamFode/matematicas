\chapter{Axiomas sobre medición de segmentos}

%--------------------axioma III1
\begin{axioma}
    Cada par de puntos en el plano corresponde a un número mayor o igual a cero. Este número es cero si y solo si los puntos coinciden.\\
\end{axioma}

%--------------------axioma III2
\begin{axioma}
    Los puntos de una línea siempre se pueden colocar en correspondencia bidireccional con los números reales, de modo que la diferencia entre estos números mida la distancia entre los puntos correspondientes.\\
\end{axioma}

%--------------------axioma III3
\begin{axioma}
   Si el punto $C$ está entre $A$ y $B$, entonces:  $$\overline{AC} + \overline{CB} = \overline{AB}$$\\
\end{axioma}

	%--------------------proposición 2.1
	\begin{proposicion}
	    Si en una semirecta $S_{AB}$ consideramos un segmento un segmento $AC$ con $\overline{AB} < \overline{AC}$, entonces el punto $C$ estará entre $A$ y $B$.\\
	\end{proposicion}

    %--------------------teorema 2.1
    \begin{teo}
	Sean $A$, $B$ y $C$ puntos en la misma línea cuyas coordenadas son, respectivamente, $a$, $b$ y $c$. El punto $C$ está entre $A$ y $B$ si y solo si el número $c$ está entre $a$ y $b$.\\
    \end{teo}

    %----------definición 2.1
    \begin{def.}
	Llamamos al punto medio del segmento $AB$  un punto $C$ de este segmento tal que $\overline{AC} = \overline{CB}$.\\
    \end{def.}

    %----------teorema 2.2.
    \begin{teo}
	Un segmento tiene exactamente un punto medio.\\
    \end{teo}

    %----------definición 2.2
    \begin{def.}
	Sea $A$ un punto en el plano y $r$ un número positivo real. El círculo del centro $A$ y el radio $r$ es el conjunto que consta de todos los puntos $B$ del plano tales que $AB = r$.\\
    \end{def.}
