\chapter{Ángulos}

\begin{axioma}
holta
\end{axioma}

\begin{def.}
Una semirecta divide un semiplano si ella está contenida en el semiplano y su origen es un punto de la recta que lo determina.
\end{def.}

\begin{tcolorbox}
\begin{axioma}
Es posible colocar en correspondencia biunívoca los números reales entre $0$ y $180$ y las semirectas del mismo origen que dividen un semiplano dado, de modo que la diferencia entre estos números sea la medida del ángulo formado o las semirectas correspondientes.\\
\textbf{obs:} Dado el semiplano $P_{mr}$
$$\lbrace S_{OA} / S_{OA} \; divide \; a \; P_{mR} \rbrace \rightarrow [0,180] \subset \mathbf{R}$$
$$S_{OA} \longmapsto coordenada de S_{OA}$$
\end{axioma}
\end{tcolorbox}

