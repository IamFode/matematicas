\chapter{Congruencia}

%----------definición 4.1
\begin{tcolorbox}[colframe=white]
    \begin{def.}
	dos segmentos son congruentes si tienen la misma longitud. dos ángulos son congruentes si tiene igual medida.\\\\
	como $\overline{ab}=\overline{cd} \quad \rightarrow \quad ab$ es congruente con $cd$.\\\\
	si $\alpha = \beta \quad \rightarrow \quad a\widehat{o}b$ es congruente con $a^{'}\widehat{o^{'}}b^{'}$\\\\
    \end{def.}
\end{tcolorbox}

%----------definición 4.1
\begin{tcolorbox}[colframe=white]
    \begin{def.}
	Dos triángulos son congruentes si fuese posible establecer una correspondencia biunívoca entre sus vértices tal que los lados y ángulos correspondientes sean congruentes.
    \end{def.}
\end{tcolorbox}

%----------axioma 4.1
\begin{tcolorbox}[colframe=white]
    \begin{axioma}[Lado ángulo lado (LAL)]
	Dados los triángulos $ABC$ y $EFG$, si $AB=EF, \; AC=EG,\; \widehat{A}=\widehat{E}$ entonces $EFG$. 
    \end{axioma}
\end{tcolorbox}

    %----------teorema 4.1
    \begin{teo}[Ángulo lado angulo (ALA)] Dados $ABC$ y $EFG$ si $AB=EF, \; \widehat{A}=\widehat{E} \widehat{B}=\widehat{F}$ entonces $DEF$.
	Demostración.-\; Sean $ABC$ y $EFG$ tal que, $AB=EF$, $\widehat{A}=\widehat{E}$ y $\widehat{B}=\widehat{F}$. Sea $D$ en $S_{AC}$ tal que $AD=EG$. Luego, por el axioma 4.1, $DAB=GEB$. Así, $A\widehat{B}D=\widehat{F},$ 
    \end{teo}

%----------definición 4.2
\begin{tcolorbox}[colframe=white]
    \begin{def.}
	Un triángulo se dice isósceles si tiene dos lados congruentes.
    \end{def.}
\end{tcolorbox}

    %----------proposición 4.1
    \begin{proposicion}
	En un triángulo isósceles los ángulos de la base son congruentes.\\\\
	    Demostración.-\;
    \end{proposicion}

    %----------proposición 4.1
    \begin{proposicion}
	Si en un triángulo $ABC$, se tienen dos ángulos congruentes, entonces el triángulo es isósceles.\\\\
	    Demostración.-\;
    \end{proposicion}

%----------definición 4.2
\begin{tcolorbox}[colframe=white]
    \begin{def.}
	Sea $ABC$ un triángulo y $D$ un punto en la recta $BC$. La mediana del triángulo relativa al vértice  $A$ y lado $BC$, (va de $A$ hasta $BC$) es el segmento que une $A$ con $D$ el punto medio de $BC$. El segmento $AD$ se llama bisectriz de $\widehat{A}$ si $S_{AD}$ divide al ángulo en $A$ en dos ángulos congruentes. El segmento $AD$ se llama altura del triángulo relativa al vértice $A$ y lado $BC$ si $AD$ es perpendicular a la recta $BC$. Si $AD$ es altura entonces $BC$ se dice base.
    \end{def.}
\end{tcolorbox}

    %----------proposición 4.1
    \begin{proposicion}
	En un triángulo isósceles la mediana relativa a la base es también bisectriz y altura.\\\\
	    Demostración.-\;
    \end{proposicion}

    %----------teorema 4.2
    \begin{teo}[Criterio (LLL) de congruencia] Si dos triángulos tienen tres lados correspondientes congruentes, entonces los triángulos son congruentes.
	Demostración.-\;
    \end{teo}

