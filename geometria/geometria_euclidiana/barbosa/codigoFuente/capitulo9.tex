\chapter{Funciones trigonométricas}

Sean: $Circ$ un círculo de centro $O$, $AB$ diámetro de $C$, $C \in Circ$, $CD \perp AB$ $D \in AB$ y $\alpha = C\widehat{O}D$

%----------definición 8.1
\begin{tcolorbox}[colframe = white]
    \begin{def.}[Seno de un ángulo]
	El seno del ángulo $\alpha$ es $\dfrac{\overline{CD}}{\overline{DC}}$\\\\
	Notación: $\sin \alpha$\\\\
	Obs:
	\begin{itemize}
	    \item 
	\end{itemize}

    \end{def.}
\end{tcolorbox}

%----------definición 8.2
\begin{tcolorbox}[colframe = white]
    \begin{def.}[Coseno de un ángulo]
	$$\cos \alpha = \left\{\begin{array}{rcl}
		\dfrac{\overline{OD}}{\overline{OC}}&si&0 \leq \alpha \leq 90\\\\
		\\%%\\
	    \end{array}\right.$$	
    \end{def.}
\end{tcolorbox}
    
    %----------proposición 8.1
    \begin{proposicion}
	Los valores de $\sin$ y $\cos$ de un ángulo no dependen del semicirculo utilizado.\\\\
	    Demostración.-\; $$\dfrac{DC}{OC}=\dfrac{d^{'}C^{'}}{O^{'}C^{'}}$$ $$\dfrac{\overline{OD}}{\overline{OC}}=\dfrac{\overline{O^{i}D^{'}}}{\overline{O^{'}C^{i}}}$$
	    Por AA tenemos $ODC=O^{'}D^{'}C^{'}$ luego, $$\dfrac{\overline{OD}}{\overline{O^{'}D^{'}}}=\dfrac{\overline{CD}}{\overline{C^{'}D^{'}}}=\dfrac{\overline{OC}}{\overline{O^{'}C^{'}}}$$ de donde la primera parte es cierta. Esto para $0\leq \alpha \leq 90$. 
    \end{proposicion}
