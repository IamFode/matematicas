\chapter{El círculo}

%----------definición 8.1
\begin{tcolorbox}[colframe=white]
\begin{def.}
    Dados $0$ y $r>0$. El círculo de centro $0$ y radio $r$ es $$C=\lbrace P/\overline{OP} = r \rbrace$$ Además:
    \begin{enumerate}[\bfseries 1)]
	
	%----------1)
	\item Al segmento $OP$ también se le llama radio de $C$.

	%----------2)
	\item Dados $A,B \in C$. El segmento $AB$ se llama cuerda de $C$. Si la cuerda $CD$ pasa por $0$ se llama diametro. También llamamos diámetro al valor $2r$.\\

    \end{enumerate}
\end{def.}
\end{tcolorbox}

    %----------proposición 8.1
    \begin{proposicion}
	Un radio es perpendicular a una cuerda (que no es un diámetro) si y sólo si lo divide en dos segmentos congruentes.
	$$OP \perp AB \Leftrightarrow \overline{AC} = \overline{CB}$$ donde $C \in AB \cap OP$\\
    \end{proposicion}


%----------definición 8.2
\begin{tcolorbox}[colframe=white]
\begin{def.}
    Sea $C$ un círculo de centro $0$ y radio $r$. Si la recta $l$ corta a $C$ en un sólo punto, decimos que $l$ es tangente a $C$.\\
\end{def.}
\end{tcolorbox}

    %---------- proposición 8.2
    \begin{proposicion}
	Si una recta es tangente a un círculo, entonces es perpendicular al radio que conecta el centro con el punto de tangencia.\\	
	ó\\
	Sea $C$ un círculo de centro $0$. Si $l$ es tangente a $C$ y $T$ es punto de tangencia, entonces $$OT \perp l$$
    \end{proposicion}

    %----------proposición 8.3
    \begin{proposicion}
	Si una recta es perpendicular a un radio en su extremo (que no es el centro), entonces la recta es tangente al círculo.
	$$Si \; OT \perp l \Rightarrow \; es \; tangente \; a \; C \; (en \; T)$$
    \end{proposicion}

%----------definición 8.3
\begin{tcolorbox}[colframe=white]
    \begin{def.}
	Sea $C$ un círculo de centro $0$ y radio $r$. Sean $A,B$ dos puntos en $C$ tal que $AB$ no es un diámetro. La recta $AB$ separa al plano en dos semi-planos.\\
	Los puntos de $C$ que están en $P_{OL}$ forman el arco mayor $AB$. Los puntos de $C$ que están en $P_{CL}$ forman el arco menor $AB$. Si $AB$ es un diámetro, la recta $AB$ determina dos semi-planos $S_1$, $S_2$.\\
	Los conjuntos $C \cap S_1$, $C\cap S_2$ se llaman semi-círculos.\\
	El ángulo $A\widehat{O}B$, se llama ángulo central. La medida en grados del arco menor $AB$ la medida del ángulo $AB$
    \end{def.}
\end{tcolorbox}

%----------proposición 8.3
\begin{proposicion} En el mismo círculo, o en círculos del mismo radio, las coordenadas congruentes determinan ángulos centrales congruentes y viceversa.\\\\
    Demostración.-\;
\end{proposicion}

%----------proposición 8.4
\begin{proposicion} Cada ángulo inscrito en un círculo tiene la mitad del tamaño del arco correspondiente.\\\\
    Demostración.-\;
\end{proposicion}

    %----------corolario 8.1
    \begin{cor} Todos los ángulos inscritos que subtienden el mismo arco tienen la misma medida. En particular, todos los ángulos que subtienden un semicírculo son rectos.\\\\
	Demostración.-\;
    \end{cor}

    %----------proposición 8.5
    \begin{proposicion} Sean $AB$ y $CD$ coordenadas diferentes del mismo círculo que se cruzan en un punto $P$. Entonces $\overline{AP} $\\\\
	Demostración.-\;
    \end{proposicion}

    %-----------proposición 8.6
    \begin{proposicion}
	Si los dos lados de un ángulo de vértice $P$ son tangentes a un círculo en los puntos $A$ y $B$, entonces:
	\begin{enumerate}[\bfseries a)]
	    \item la medida del ángulo $\widehat{P}$ es igual a $180^{\circ}$ menos la medida del arco menor determinada por $A$ y $B$;
	    \item $PA=PB$\\\\
	\end{enumerate}
	    Demostración.-\;
    \end{proposicion}

%----------definición 8.3
\begin{tcolorbox}[colframe=white]
    \begin{def.} Un círculo está inscrito en un polígono si todos los lados del polígono son tangentes al círculo´.
    \end{def.}
\end{tcolorbox}

    %----------proposición 8.7
    \begin{proposicion} Todo triángulo posee un círculo inscrito.\\\\
	Demostración.-\;
    \end{proposicion}

