\chapter{Semejanza de triángulos}

    %-----------definición 7.1
    \begin{def.}
	Los triángulos $ABC$ y $DEF$ son semejantes si se establece una correspondencia donde los lados correspondientes son proporcionales y en la misma proporción, y los ángulos correspondientes congruentes. Es decir, si $ABC \Longleftrightarrow DEF$ es tal que
	$$\dfrac{\overline{AB}}{\overline{DE}} = $$
    \end{def.}

    %----------teorema 7.1
    \begin{teo}[AA]
	Dados los triángulos $ABC$ y $EFG$, tal que $\widehat{A}=\widehat{E}$ y $\widehat{B} = \widehat{F}$. Entonces los triángulos son semejantes.\\
    \end{teo}
    
    %----------teorema 7.2
    \begin{teo}
	Si en dos triángulos $ABC$ y $EFG$ tenemos $\widehat{A} = \widehat{E}$ y $(\overline{AB}/\overline{EF}) = (\overline{AC} / \overline{EG})$, entonces los triángulos son semejantes.\\
    \end{teo}

    %----------teorema 7.3
    \begin{teo}
	Si en dos triángulos $ABC$ y $EFG$ se tiene que $$\dfrac{\overline{AB}}{\overline{EF}}=\dfrac{\overline{BC}}{\overline{FG}}=\dfrac{\overline{CA}}{\overline{GE}}$$ entonces los dos triángulos son similares.\\	
    \end{teo}

	%----------proposición 1.1
	\begin{proposicion}
	    En todo triángulo rectángulo, la altura del vértice del ángulo recto es la media proporcional entre las proyecciones de los catetos en la hipotenusa.\\
	\end{proposicion}

	%----------teorema 7.4
	\begin{teo}[Pitágoras]
	    En cada triángulo un rectángulo, el cuadrado de la longitud de la hipotenusa es igual a la suma de los cuadrados de las longitudes de los catetos.\\ En términos de la notación establecida anteriormente, el teorema de Pitágoras establece que $$a^2=b^2+c^2$$\\
	\end{teo}

	%----------proposición 1.2
	\begin{proposicion}
	    Un triángulo tiene lados que miden $a$, $b$ y $c$. Si $a^2 = b^2 + c^2$, entonces el triángulo es un rectángulo y su hipotenusa es el lado que mide $a$.\\
	\end{proposicion}
