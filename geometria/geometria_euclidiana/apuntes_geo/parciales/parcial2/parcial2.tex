\documentclass[10pt]{article} 						
\usepackage[text=17cm,left=2.5cm,right=2.5cm, headsep=20pt, top=2.5cm, bottom = 2cm,letterpaper,showframe = false]{geometry} %configuración página
\usepackage{latexsym,amsmath,amssymb,amsfonts} %(símbolos de la AMS).7
\parindent = 0cm  %sangria
\usepackage[T1]{fontenc} %acentos en español
\usepackage[spanish]{babel} %español capitulos y secciones
\usepackage{graphicx} %gráficos y figuras.
%-----------------------------------------%
\usepackage{multicol}
\usepackage{titlesec}
\usepackage[rflt]{floatflt}
\usepackage{wrapfig} 
\usepackage{tikz}
\usepackage{tkz-euclide}
\usetikzlibrary{decorations.markings,arrows}
\usetikzlibrary{matrix,arrows, positioning,shadows,shadings,backgrounds,
calc, shapes, tikzmark}
\usepackage{tcolorbox, empheq}
\tcbuselibrary{skins,breakable,listings,theorems}
\usepackage{xparse}							
\usepackage{pstricks}							
\usepackage[Bjornstrup]{fncychap}			
\usepackage{rotating}
\usepackage{enumerate}
\usepackage{booktabs}
\usepackage{synttree} 
\usepackage{chngcntr}
\usepackage{venndiagram}
\usepackage[all]{xy}
\usepackage{xcolor}
\usepackage{tikz}
\usetikzlibrary{datavisualization.formats.functions}
\usepackage{marginnote}										
\usepackage{fancyhdr}

%------------------------------------------
\renewcommand{\labelenumi}{\Roman{enumi}.}		%primer piso II) enumerate
\renewcommand{\labelenumii}{\arabic{enumii}$)$ }%segundo piso 2)
\renewcommand{\labelenumiii}{\alph{enumiii}$)$ }%tercer piso a)
\renewcommand{\labelenumiv}{$\bullet$}			%cuarto piso (punto)


\pagestyle{fancy}
\fancyhead[R]{Geometría I}
\fancyhead[L]{Parcial II}

\begin{document}
\begin{tabular}{r l }
Universidad: & \textbf{Mayor de San Ándres.}\\
Asignatura: & \textbf{Geometría I.}\\
 Parcial: & II.\\ Alumno: & \textbf{PAREDES AGUILERA CHRISTIAN LIMBERT.}
\end{tabular}
\begin{flushleft}
\begin{tikzpicture}
\draw(0,1)--(16.5,1);
\end{tikzpicture}
\end{flushleft}

\section*{\center Soluciones}

\begin{enumerate}[\Large \bfseries 1.]

%--------------------1.
\item Sabemos por el teorema de ángulo externo que $E\widehat{C}D > \widehat{B}, \widehat{A}.$ Luego como $\triangle ABC$ por lo tanto $E\widehat{C}D=B\widehat{C}A$ de donde $\overline{AC}=\overline{EC}$ y por el teorema de desigualdad triangular se tiene que:
$$\overline{AC}+\overline{CB}\geq \overline{AB}$$ como $\overline{AC}=\overline{EC}$ y $\overline{CB}=\overline{CD}$ entonces $$\overline{EC} + \overline{CD} > \overline{AB}$$ luego $\overline{AD}=\overline{AC} + \overline{CD}$ así tenemos que:  $$\overline{AD}=\overline{EC} + \overline{CD} > \overline{AB}$$ que implica $\overline{AD}>\overline{AB}$\\\\

%--------------------2.
\item Consideremos un cuadrilátero de vértices $A, B, C$ y $D$, con $A, C$ y $B, D$ opuestos. Ahora trazamos  la diagonal $BD$, para obtener los triángulos $ABD$ y $BCD$.  Sean $M$ y $N$ los puntos medios de los lados $AB$ y $AD$, respectivamente. Luego trazamos el segmento $MN$. Tengamos en cuenta que es la base promedio del triángulo $ABD$ y, por lo tanto, debe ser paralelo a $BD$: $$MN \parallel BD$$
Ahora tracemos los segmentos $P$ y $Q$, siendo $P$ y $Q$ los respectivos puntos medios de $BC$ y $CD$. Tengamos en cuenta que $PQ$ es la base promedio en el triángulo $BCD$, lo que implica que es paralelo a la base:
$$PQ \parallel BD$$
Luego por  transitividad del paralelismo de recta, si dos rectas son paralelas a una tercera, entonces son paralelas entre sí. De donde: $$MN \parallel BD; \; PQ \parallel BD \quad \Rightarrow \quad MN \parallel PQ$$
Tengamos en cuenta que $MNPQ$ es el cuadrilátero con vértices en los puntos medios a los lados del cuadrilátero original. \\
Ya hemos probado que dos de los lados opuestos son paralelos. Si probamos que los otros dos también lo son, entonces, por definición, $MNPQ$ es un paralelogramo. Tal prueba es análogo a la demostración dada, considerando la diagonal $AC$.\\\\

%--------------------3.
\item La gráfica se verá de la siguiente manera:\\
\begin{center}
    \begin{tikzpicture}
	\draw(0,0)node[below]{$A$}--(3,0)node[below]{$C$}--(5,3)node[right]{$P$}--(0,0);	
	\draw(3,0)--(2.5,1.5)node[above]{$B$}--(2,0)node[below]{$D$};
    \end{tikzpicture}
\end{center}
Siendo $BD$ la bisectriz interna de $\widehat{B}$, tracemos una paralela que pase por el vértice $C$ la bisectriz $BD$, donde se encontrará una extensión de $BA$ en un punto $P$. Luego por el axioma de los paralelos, el triángulo $BPC$ es isosceles es decir que uno de sus ángulos alterna internamente con una de las mitades del ángulo $B$ y el otro correspondiente a la otra mitad. Por tanto según el teorema de Tales se sigue que $\dfrac{AD}{DC}=\dfrac{AB}{AP}$, como $BC=AP$ tenemos que $$\dfrac{AD}{DC}=\dfrac{AB}{BC}$$\\\\


%--------------------4.
\item 

\end{enumerate}

\end{document}
