\documentclass[10pt]{article} 						
\usepackage[text=17cm,left=2.5cm,right=2.5cm, headsep=20pt, top=2.5cm, bottom = 2cm,letterpaper,showframe = false]{geometry} %configuración página
\usepackage{latexsym,amsmath,amssymb,amsfonts} %(símbolos de la AMS).7
\parindent = 0cm  %sangria
\usepackage[T1]{fontenc} %acentos en español
\usepackage[spanish]{babel} %español capitulos y secciones
\usepackage{graphicx} %gráficos y figuras.
%-----------------------------------------%
\usepackage{multicol}
\usepackage{titlesec}
\usepackage[rflt]{floatflt}
\usepackage{wrapfig} 
\usepackage{tikz}\usetikzlibrary{shapes.misc}
\usepackage{tikz,tkz-tab}						
\usetikzlibrary{matrix,arrows, positioning,shadows,shadings,backgrounds,
calc, shapes, tikzmark}
\usepackage{tcolorbox, empheq}
\tcbuselibrary{skins,breakable,listings,theorems}
\usepackage{xparse}							
\usepackage{pstricks}							
\usepackage[Bjornstrup]{fncychap}			
\usepackage{rotating}
\usepackage{enumerate}
\usepackage{booktabs}
\usepackage{synttree} 
\usepackage{chngcntr}
\usepackage{venndiagram}
\usepackage[all]{xy}
\usepackage{xcolor}
\usepackage{tikz}
\usetikzlibrary{datavisualization.formats.functions}
\usepackage{marginnote}										
\usepackage{fancyhdr}

%------------------------------------------
\renewcommand{\labelenumi}{\Roman{enumi}.}		%primer piso II) enumerate
\renewcommand{\labelenumii}{\arabic{enumii}$)$ }%segundo piso 2)
\renewcommand{\labelenumiii}{\alph{enumiii}$)$ }%tercer piso a)
\renewcommand{\labelenumiv}{$\bullet$}			%cuarto piso (punto)


\pagestyle{fancy}
\fancyhead[R]{Geometría I}
\fancyhead[L]{Práctica III}

\begin{document}
\begin{tabular}{r l }
Universidad: & \textbf{Mayor de San Ándres.}\\
Asignatura: & \textbf{Geometría I.}\\
 Parcial: & I.\\ 
Alumno: & \textbf{PAREDES AGUILERA CHRISTIAN LIMBERT.}
\end{tabular}
\begin{flushleft}
\begin{tikzpicture}
\draw(0,1)--(16.5,1);
\end{tikzpicture}
\end{flushleft}



    \begin{enumerate}[\Large\bfseries 1.]

	%----------1.
	\item Muestre que la intersección de $3$ semiplanos es un conjunto convexo.\\\\
	Demostración.-\; Imagine los semiplanos $S_1$, $S_2$,  $S_3$ y $S_4$ tales que $S_4 = S_1 \cap S_2 \cap S_3$, tomando tres puntos $P_1, P_2$ y $P_3$ pertenecientes a $S_4$ entonces: $P_1, P_2, P_3 \in S_1, S_2, S_3$ Sea $S_1,S_2$ y $S_2$ convexos entonces $P_1, P_2, P_3 in S_1, S_2, S_3$ y por lo tanto pertenece a la intersección, entonces $S_4$ también es convexo.\\\\

	%----------2.
	\item Dado un segmento $AB$ muestre que existe, y es único, un punto $C$ entre $A$ y $B$ tal que $$\dfrac{\overline{AC}}{\overline{BC}}=5$$\\\\
	Demostración.-\; Supongamos que  $C$ está entre $A$ y $B$. Entonces probemos la existencia del punto $C$.\\
	Por el axioma $III_2$ existe $x$, $b$ y $c$ en los reales que representan las coordenadas de los puntos A, B y C respectivamente, y por lo tanto, $$\dfrac{m(AC)}{m(BC)}=\dfrac{c-x}{b-c}$$ pero como $\dfrac{m(AC)}{m(BC)}=5$ entonces $$\dfrac{c-x}{b-c}=5$$ lo que implica que $$c=\dfrac{5b+x}{1+5}\,\,\, (1)$$
	De donde vemos que $c$ existe para cualquier valor de $x$ y $b$ por lo tanto verificamos la existencia del punto $C$.\\
	Para probar la unicidad de $C$ supondremos que hay una $C^{'}$ entonces, $$\dfrac{m(AC^{'})}{m(BC^{'})}=5 \,\, y \, por \, lo \, tanto \dfrac{c^{'}-x}{b-c^{'}}=5$$ así nos queda, $$c^{'}=\dfrac{5b+x}{1+5}\,\,\, (2)$$ 
	Por el axioma $III_2$ en $(1)$ y $(2)$, podemos decir que los puntos $C$ y $C^{'}$ tienen una distancia igual a cero lo que por el axioma $III_1$, concluimos que $C$ y $C^{'}$ son el mismo punto, de donde tenemos una contradicción.\\\\ 

	%----------3.
	\item Muestre que las bisectrices de un ángulo y de su suplemento son perpendiculares.\\\\
	Demostración.-\; Sea
	\begin{center}
	    \begin{tikzpicture}
		\draw[<->](0,0)--(4,0);
		\draw[->](2,0)node[below]{$O$}--(4,2);
		\draw[->](2,0)--(2.5,2.5);
		\draw[->](2,0)--(0.4,2.2);
		\filldraw[black](.5,0) circle(1.5pt) node[above]{$C$};
		\filldraw[black](3.5,0) circle(1.5pt) node[above]{$A$};
		\filldraw[black](3.5,1.5) circle(1.5pt) node[left]{$D$};
		\filldraw[black](2.4,2) circle(1.5pt) node[left]{$B$};
		\filldraw[black](.7,1.8) circle(1.5pt) node[left]{$E$};
	    \end{tikzpicture}
	\end{center}

	Sea $A\widehat{O}B$ un ángulo y $B\widehat{O}C$ su suplemento, entonces: $$A\widehat{O}B + B\widehat{O}C = 180^{\circ}\,\,\, (1)$$
	Demostremos que $B\widehat{O}D + B\widehat{O}E = 90^{\circ}$ para esto, tenga en cuenta que $B\widehat{O}D = \dfrac{A\widehat{O}B}{2},$ por lo tanto, $S_{OD}$ es la bisectriz de $A\widehat{O}B$ y $B\widehat{O}E = \dfrac{B\widehat{O}C}{2}$ luego, $B\widehat{O}E + B\widehat{O}D = \dfrac{A\widehat{O}B}{2} + \dfrac{B\widehat{O}C}{2}$ de donde
	$$2(B\widehat{O}E + B\widehat{O}D) = A\widehat{O}B + B\widehat{O}C \,\,\, (2)$$ 
	Igualando $(1)$ y $(2)$, 
	\begin{center}
	    \begin{tabular}{rcl}
		$2(B\widehat{O}E + B\widehat{O}D)$ & $=$  & $180^{\circ}$\\
		$B\widehat{O}E + B\widehat{O}D$ & $=$ & $90^{\circ}$\\ 
	    \end{tabular}
	\end{center}

	Lo que implica que $S_{OE} \perp S_{OD}$ como queríamos demostrar.\\\\

	%----------4.
	\item En la figura, se tiene $AD=DE$, $\widehat{A}=D\widehat{E}C$ y $A\widehat{D}E=B\widehat{D}C$. Muestre que los triángulos $ADB$ y $EDC$ son congruentes.
	
	\begin{center}
	    \begin{tikzpicture}
		\draw(0,0)node[below]{$A$}--(4,0)node[below]{$B$}--(5.5,3)node[above]{$C$}--(1.5,3)node[above]{$D$}--(3,0)node[below]{$E$}--(5.5,3);
		\draw(4,0)--(1.5,3)--(0,0);
	    \end{tikzpicture}
	\end{center}

	Demostración.-\;

	\begin{center}
	    \begin{tabular}{crc}
		&$\overline{AD}\cong \overline{DE}$&\\
		&$D\widehat{A}B \cong D\widehat{E}C$&\\		
		$(a)$&$A\widehat{D}E \cong C\widehat{D}B$&\\	
		&$\triangle ABD \cong \triangle EDC$&\\
		$(I)$&$A\widehat{D}B=A\widehat{D}E + E\widehat{D}B \, y \, C\widehat{D}E=C\widehat{D}B+B\widehat{D}E$&\\
		$(II)$&$A\widehat{D}B =A\widehat{D}E + E\widehat{D}B=C\widehat{D}B + B\widehat{D}E=C\widehat{D}E \Rightarrow A\widehat{D}B \cong C\widehat{D}E$&Por $(a)$ y $(I)$ \\
		$(A)$&$D\widehat{A}B \cong D\widehat{E}C$&\\
		$(L)$&$\overline{AD}\cong \overline{DE}$&\\
		$(A)$&$ A\widehat{D}B \cong C\widehat{D}E$&De $(I)$ y $(II)$\\\\
	    \end{tabular}
	\end{center}
	Luego por $ALA$ tenemos $\triangle ABD \cong \triangle EDC$\\\\

    \end{enumerate}

\end{document}
