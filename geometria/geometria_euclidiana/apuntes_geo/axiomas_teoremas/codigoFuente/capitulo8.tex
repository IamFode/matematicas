\chapter{El círculo}

    %----------definición 8.1
    \begin{def.}
	Dados $0$ y $r>0$. El círculo de centro $0$ y radio $r$ es $$C=\lbrace P/\overline{OP} = r \rbrace$$ Además:
	\begin{enumerate}[\bfseries 1)]
	    
	    %----------1)
	    \item Al segmento $OP$ también se le llama radio de $C$.

	    %----------2)
	    \item Dados $A,B \in C$. El segmento $AB$ se llama cuerda de $C$. Si la cuerda $CD$ pasa por $0$ se llama diametro. También llamamos diámetro al valor $2r$.\\

	\end{enumerate}
    \end{def.}

	%----------proposición 8.1
	\begin{proposicion}
	    Un radio es perpendicular a una cuerda (que no es un diámetro) si y sólo si lo divide en dos segmentos congruentes.
	    $$OP \perp AB \Leftrightarrow \overline{AC} = \overline{CB}$$ donde $C \in AB \cap OP$\\
	\end{proposicion}


    %----------definición 8.2
    \begin{def.}
	Sea $C$ un círculo de centro $0$ y radio $r$. Si la recta $l$ corta a $C$ en un sólo punto, decimos que $l$ es tangente a $C$.\\
    \end{def.}

	%---------- proposición 8.2
	\begin{proposicion}
	    Si una recta es tangente a un círculo, entonces es perpendicular al radio que conecta el centro con el punto de tangencia.\\	
	    ó\\
	    Sea $C$ un círculo de centro $0$. Si $l$ es tangente a $C$ y $T$ es punto de tangencia, entonces $$OT \perp l$$
	\end{proposicion}

	%----------proposición 8.3
	\begin{proposicion}
	    Si una recta es perpendicular a un radio en su extremo (que no es el centro), entonces la recta es tangente al círculo.
	    $$Si \; OT \perp l \Rightarrow \; es \; tangente \; a \; C \; (en \; T)$$
	\end{proposicion}

    %----------definición 8.3
    \begin{def.}
	Sea $C$ un círculo de centro $0$ y radio $r$. Sean $A,B$ dos puntos en $C$ tal que $AB$ no es un diámetro. La recta $AB$ separa al plano en dos semi-planos.\\
	Los puntos de $C$ que están en $P_{OL}$ forman el arco mayor $AB$. Los puntos de $C$ que están en $P_{CL}$ forman el arco menor $AB$. Si $AB$ es un diámetro, la recta $AB$ determina dos semi-planos $S_1$, $S_2$.\\
	Los conjuntos $C \cap S_1$, $C\cap S_2$ se llaman semi-círculos.\\
	El ángulo $A\widehat{O}B$, se llama ángulo central. La medida en grados del arco menor $AB$ la medida del ángulo $AB$

    \end{def.}
