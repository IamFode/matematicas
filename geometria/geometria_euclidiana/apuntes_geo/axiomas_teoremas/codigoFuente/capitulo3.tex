\chapter{Axiomas sobre la medición de ángulos}

    %--------------------definición 3.1
    \begin{def.}
	Llamamos figura ángulo a la figura formada por dos semi-rectas con el mismo origen.\\
    \end{def.}

%--------------------axioma III_4
\begin{axioma}
    Cada ángulo tiene una medida mayor o igual a cero. La medida del ángulo es cero si y sólo si consta de dos líneas coincidentes.
\end{axioma}

    %--------------------definición 3.2
    \begin{def.}
	Diremos que una semi-recta divide un semi-plano si está contenido en el semi-plano y su origen es un punto de la recta que lo determina.\\
    \end{def.}

%---------------------axiomaIII_5
\begin{axioma}
    Es posible colocar, en correspondencia bidireccional, los números reales entre cero y 180 y las semi-rectas de un mismo origen que dividen un semi-plano dado, de modo que la diferencia entre este número sea la medida del ángulo formado por el correspondientes semi-rectas.\\
\end{axioma}

    %----------definición 3.3
    \begin{def.}
	Sea $S_{OA}, S_{OB}$ y $S_{OC}$ semi-rectas del mismo origen. Si el segmento $AB$ intercepta $S_{OC}$ diremos que $S_{OC}$ divide al ángulo $A\widehat{O}B$.\\
    \end{def.}

%----------axioma III_6 
\begin{axioma}
    Una semi-recta $S_{OC}$ divide un ángulo $A\widehat{O}B$, entonces $$A\widehat{O}B=A\widehat{O}C + C\widehat{O}B$$\\
\end{axioma}


    %----------definición 3.4
    \begin{def.}
	Se dice que dos ángulos son suplementarios si la suma de sus medidas es $180^{◦}$. El suplemento de un ángulo es el ángulo adyacente al ángulo  dado obtenido por la prolongación de uno de sus lados.\\
    \end{def.}
