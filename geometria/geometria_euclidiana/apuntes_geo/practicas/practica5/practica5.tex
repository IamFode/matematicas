\documentclass[10pt]{article} 						
\usepackage[text=17cm,left=2.5cm,right=2.5cm, headsep=20pt, top=2.5cm, bottom = 2cm,letterpaper,showframe = false]{geometry} %configuración página
\usepackage{latexsym,amsmath,amssymb,amsfonts} %(símbolos de la AMS).7
\parindent = 0cm  %sangria
\usepackage[T1]{fontenc} %acentos en español
\usepackage[spanish]{babel} %español capitulos y secciones
\usepackage{graphicx} %gráficos y figuras.
%-----------------------------------------%
\usepackage{multicol}
\usepackage{titlesec}
\usepackage[rflt]{floatflt}
\usepackage{wrapfig} 
\usepackage{tikz}\usetikzlibrary{shapes.misc}
\usepackage{tikz,tkz-tab}						
\usetikzlibrary{matrix,arrows, positioning,shadows,shadings,backgrounds,
calc, shapes, tikzmark}
\usepackage{tcolorbox, empheq}
\tcbuselibrary{skins,breakable,listings,theorems}
\usepackage{xparse}							
\usepackage{pstricks}							
\usepackage[Bjornstrup]{fncychap}			
\usepackage{rotating}
\usepackage{enumerate}
\usepackage{booktabs}
\usepackage{synttree} 
\usepackage{chngcntr}
\usepackage{venndiagram}
\usepackage[all]{xy}
\usepackage{xcolor}
\usepackage{tikz}
\usetikzlibrary{datavisualization.formats.functions}
\usepackage{marginnote}										
\usepackage{fancyhdr}

%------------------------------------------
\renewcommand{\labelenumi}{\Roman{enumi}.}		%primer piso II) enumerate
\renewcommand{\labelenumii}{\arabic{enumii}$)$ }%segundo piso 2)
\renewcommand{\labelenumiii}{\alph{enumiii}$)$ }%tercer piso a)
\renewcommand{\labelenumiv}{$\bullet$}			%cuarto piso (punto)


\pagestyle{fancy}
\fancyhead[R]{Geometría I}
\fancyhead[L]{Práctica IV}

\begin{document}
\begin{tabular}{r l }
Universidad: & \textbf{Mayor de San Ándres.}\\
Asignatura: & \textbf{Geometría I.}\\
 Práctica: & IV.\\ Alumno: & \textbf{PAREDES AGUILERA CHRISTIAN LIMBERT.}
\end{tabular}
\begin{flushleft}
\begin{tikzpicture}
\draw(0,1)--(16.5,1);
\end{tikzpicture}
\end{flushleft}

\section*{\center Soluciones}

\begin{enumerate}[\Large\bfseries 1.]
    
%-------------------1.
\item Sea el triángulo $ABC$, como sigue 

\begin{center}
    \begin{tikzpicture}
	\draw(0,0)--(6,0); 
	\draw(1,0)node[below]{$A$}--(3,3)node[above]{$C$}--(5,0)node[below]{$B$}; 
	\draw [color=black](40:.8cm) node[rotate=0] {$\widehat{e}$};
	\draw [red!50!black,very thick](0:.6cm) arc (180:50:.4cm);
	\draw [color=black](8:5.3cm) node[rotate=0] {$\widehat{f}$};
	\draw [red!50!black,very thick](4:4.8cm) arc (115:0:.4cm);
    \end{tikzpicture}
\end{center}

    Como $\widehat{e}$ y $C\widehat{A}B$ son adyacentes y están bajo el mismo semirecta entonces: $$\widehat{e} + C\widehat{A}B = 180^{\circ},$$ de igual forma se concluye que $$C\widehat{B}A + \widehat{f} = 180^{\circ}$$ o que implica que $\widehat{e} + C\widehat{A}B = C\widehat{B}A + \widehat{f} \qquad (1)$, luego por hipótesis $\widehat{e} = \widehat{f}$ entonces por $(1)$ tenemos que $$C\widehat{A}B = C\widehat{B}A$$ Por lo tanto, el triángulo $ABC$ es isosceles de base $AB$\\\\

%--------------------2.
\item 
\begin{enumerate}[\bfseries a)]
    
    %----------a)
    \item $5,8,3,10$\\\\

    %----------b)
    \item $7,6,4,2,9,1$\\\\

    %----------c)
    \item $1,2,3,5,6,7,8,9,10$\\\\

\end{enumerate}

%--------------------3.
\item Por $TAE$ es necesario: $$A\widehat{C}E > B\widehat{A}C, A\widehat{B}C$$ Como por hipótesis $$A\widehat{B}C < A\widehat{C}E < A \widehat{B}D$$ Que implica que $A\widehat{B}C< A\widehat{B}D$\\\\ 

%--------------------4.
\item Dado el triángulo $ABC$ 
\begin{center}
    \begin{tikzpicture}
	\draw(0,0)node[below]{$A$}--(2,0)node[below]{$B$}--(2,2)node[above]{$C$}--(0,0);
    \end{tikzpicture}
\end{center}
Sea $\widehat{A}+\widehat{B}+\widehat{C} = 180^{\circ}$. Como $\widehat{B} = 90^{\circ}$, entonces $widehat{A},\widehat{C} < 90^{\circ}$. Luego el ángulo externo a $\widehat{A}$ y $\widehat{C} > 90^{\circ}$ ya que son suplementarios.\\\\

%--------------------5.
\item Notemos que $A\widehat{D}E$ es un ángulo externo al triángulo $DBC$ y por $TAE$ se tiene: $$A\widehat{D}E > D\widehat{B}C, D\widehat{C}B \qquad (1)$$ De la misma manera, $A\widehat{E}C$ es externo a $\triangle ADE$ y una vez más por $TAE$:
$$A\widehat{E}C = A\widehat{D}E \qquad (2),$$ luego por $(2)$ y $(1)$ se tiene que $A\widehat{E}C > D\widehat{B}C$.\\\\ 

%--------------------6.
\item El ángulo $E\widehat{C}D > \widehat{B}, \widehat{A}$ por $TAE$. Como $\triangle ABC = \triangle ECD$ así $E\widehat{C}D=B\widehat{C}A,$ entonces $\overline{AC} = \overline{EC}$ por el teorema de desigualdad triangular tenemos que: $$\overline{AC} + \overline{CB} \geq \overline{AB},$$ como $\overline{AC} = \overline{EC}$ y $\overline{CB} = \overline{CD}$ entonces $$\overline{EC} + \overline{CD} > \overline{AB},$$ luego $\overline{AD} = \overline{AC} + \overline{CD}$ se tiene $$\overline{AD} = \overline{EC} + \overline{CD} > \overline{AB}$$ que implica que $\overline{AD} > \overline{AB}$\\\\

%--------------------7.
\item Simplemente debemos trazar el segmento $AC$ e introducirlo en  $\triangle ADC = \triangle ABC$ según el criterio del catéto de hipotenusa que implica que $AD = BC$.\\\\

%--------------------8.
\item Los triángulos serán congruentes con el caso del caso LAA o con el cateto opuesto.\\\\

%--------------------9.
\item 
\begin{enumerate}[\bfseries a)]
    
    %----------a)
    \item Si son congruentes ya que los ángulos lo son.\\\\ 

    %----------b)
    \item El lado $AB$.\\\\

    %----------c)
    \item El lado $AC$\\\\

\end{enumerate}

%--------------------10.
\item 
\begin{enumerate}[\bfseries a)]
    
    %----------a)
    \item No son congruentes.\\\\

    %----------b)
    \item El lado $AB$.\\\\

    %----------c)
    \item El lado $AC$\\\\

\end{enumerate}




\end{enumerate}
\end{document}
