\documentclass[10pt]{article} 						
\usepackage[text=17cm,left=2.5cm,right=2.5cm, headsep=20pt, top=2.5cm, bottom = 2cm,letterpaper,showframe = false]{geometry} %configuración página
\usepackage{latexsym,amsmath,amssymb,amsfonts} %(símbolos de la AMS).7
\parindent = 0cm  %sangria
\usepackage[T1]{fontenc} %acentos en español
\usepackage[spanish]{babel} %español capitulos y secciones
\usepackage{graphicx} %gráficos y figuras.
%-----------------------------------------%
\usepackage{multicol}
\usepackage{titlesec}
\usepackage[rflt]{floatflt}
\usepackage{wrapfig} 
\usepackage{tikz}
\usepackage{tkz-euclide}
\usetikzlibrary{decorations.markings,arrows}
\usetikzlibrary{matrix,arrows, positioning,shadows,shadings,backgrounds,
calc, shapes, tikzmark}
\usepackage{tcolorbox, empheq}
\tcbuselibrary{skins,breakable,listings,theorems}
\usepackage{xparse}							
\usepackage{pstricks}							
\usepackage[Bjornstrup]{fncychap}			
\usepackage{rotating}
\usepackage{enumerate}
\usepackage{booktabs}
\usepackage{synttree} 
\usepackage{chngcntr}
\usepackage{venndiagram}
\usepackage[all]{xy}
\usepackage{xcolor}
\usepackage{tikz}
\usetikzlibrary{datavisualization.formats.functions}
\usepackage{marginnote}										
\usepackage{fancyhdr}
\usepackage{yhmath}

%------------------------------------------
\renewcommand{\labelenumi}{\Roman{enumi}.}		%primer piso II) enumerate
\renewcommand{\labelenumii}{\arabic{enumii}$)$ }%segundo piso 2)
\renewcommand{\labelenumiii}{\alph{enumiii}$)$ }%tercer piso a)
\renewcommand{\labelenumiv}{$\bullet$}			%cuarto piso (punto)


\pagestyle{fancy}
\fancyhead[R]{Geometría I}
\fancyhead[L]{Práctica IV}

\begin{document}
\begin{tabular}{r l }
Universidad: & \textbf{Mayor de San Ándres.}\\
Asignatura: & \textbf{Geometría I.}\\
 Práctica: & IX.\\ Alumno: & \textbf{PAREDES AGUILERA CHRISTIAN LIMBERT.}
\end{tabular}
\begin{flushleft}
\begin{tikzpicture}
\draw(0,1)--(16.5,1);
\end{tikzpicture}
\end{flushleft}


\begin{enumerate}[\Large\bfseries 1.]

\section*{Funciones trigonométricas}

%--------------------1.
\item Cuando el sol está $20^\circ$ encima del horizonte, ¿cuál es la longitud de la sombra proyectada por un edificio de $50$ metros?\\\\
    Respuesta.-\; 
    \begin{center}
	\begin{tikzpicture}
	    \draw (0,0)--(3,2)--(3,0)--(0,0);
	    \draw(3,2)--(4,2)--(4,0)--(3,0);
	    \draw(3,2)--(5,3.3)node[]{$\circ$};
	    \draw(2,0)node[below]{$Sombra$};
	    \draw(5,3.3)node[right]{$Sol$};
	    \draw(4,1)node[right]{$Edificio$};
	\end{tikzpicture}
    \end{center}
	$$\tan 20^\circ = \dfrac{50}{\overline{AB}} \quad \Longrightarrow \quad \overline{AB} = 137.38$$\\

%--------------------2.
\item Un árbol de $10$ metros de altura proyecta una sombra de $12$ metros. ¿Cuál es la altura angular del sol?\\\\
    Respuesta.-\; 
    \begin{center}
	\begin{tikzpicture}
	    \draw (0,0)--(3,2)--(3,0)--(0,0);
	    \draw(2,0)node[below]{$12\; m$};
	    \draw(3,1)node[right]{$10\; m$};
	\end{tikzpicture}
    \end{center}
	$$\tan \theta = \dfrac{10}{12} \quad \Longrightarrow \quad \theta = 39.8^\circ$$\\

%--------------------3.
\item Los lados de un triángulo $ABC$ son los siguientes: $AB = 5$, $AC = 8$ y $BC = 5$. Determine el seno del ángulo $\widehat{A}$\\\\ 
	Respuesta.-\; 
	\begin{center}
	    \begin{tikzpicture}
		\draw(0,0)node[left]{$A$}--(4,0)node[right]{$C$}--(2,2)node[above]{$B$}--(0,0);
		\draw(2,0)node[below]{$8$};
		\draw(.9,1)node[left]{$5$};
		\draw(3.1,1)node[right]{$5$};
	    \end{tikzpicture}
	\end{center}
	Por la ley de cosenos se tiene $$\overline{AB}=\overline{AC}^2 + \overline{BC}^2 - 2\overline{AC}^2 \overline{BC}^2 \cos \widehat{C} \quad \Longrightarrow \quad \cos \widehat{C} = \dfrac{64}{80} \quad \Longrightarrow \quad \widehat{C} = 36^\circ 87^{'}$$
	Como $\widehat{C}=\widehat{A}$ por ser triangulo isosceles, entonces $\sin \widehat{C}=\sin \widehat{A} = 0.6$\\\\

%--------------------4.
\item De la punta de un faro, $40$ metros encima del nivel del mar, el farolero ve un barco en un ángulo (de depresión) de $15^\circ$. ¿Cuál la distancia del barco al faro?\\\\
    Respuesta.-\; 
	\begin{center}
	    \begin{tikzpicture}
		\draw(0,0)node[left]{$A$}--(3,0)node[right]{$B$}--(3,2)node[above]{$C$}--(0,0);
		\draw(2.7,1.5)node[]{$15^\circ$};
		\draw(3,1)node[right]{$40$};
	    \end{tikzpicture}
	\end{center}
	$$\tan \theta = \dfrac{\overline{AC}}{\overline{BC}} \quad \Longrightarrow \quad \overline{AC} = \tan 15^\circ \cdot 40 = 10.72 \; m$$\\

%--------------------5.
\item Un carro sube $500$ metros sobre una pista recta de $20^\circ$ de inclinación. ¿Cuántos metros el punto de llegada está por encima del punto de partida?\\\\
    Respuesta.-\; 
	\begin{center}
	    \begin{tikzpicture}
		\draw(0,0)node[left]{$A$}--(3,0)node[right]{$B$}--(3,2)node[above]{$C$}--(0,0);
		\draw(.8,.2)node[]{$20^\circ$};
		\draw(.2,1.2)node[right]{$500 \; m$};
	    \end{tikzpicture}
	\end{center}
	Aplicando la ley de senos se tiene $$\dfrac{\sen 90^\circ}{500}=\dfrac{\sen 20^\circ}{\overline{BC}} \quad \Longrightarrow \quad \overline{BC} = 171\; m$$


%--------------------6.
    \item En un triángulo $ABC$ se tiene $\overline{AC} = 23$, $\widehat{A} = 20^\circ$ y $\widehat{C} = 140^\circ$. Determine la altura desde el vértice $B$.\\\\
	Respuesta.-\; Este triángulo no existe. Lo que invalida la pregunta. \\\\

%--------------------7.
    \item Las funciones secantes, cosecantes y cotangentes de un ángulo $\widehat{A}$ son definidas por $\sec \widehat{A}=1/\cos \widehat{A}$, $\csc \widehat{A}=1/\sen \widehat{A}$ y $ctan \; \widehat{A}=1/\tan \widehat{A}$, siempre que $\cos \widehat{A}$, $\sen \widehat{A}$ y $\tan \widehat{A}$ estén definidas y sean distintas de cero. Pruebe que:
	\begin{enumerate}[\bfseries a)]
	    \item $1+\tan^2 \widehat{A} = \sec^2 \widehat{A}$\\\\
		Demostración.-\; $$\sec^2 \widehat{A}=\dfrac{1}{\cos^2 \widehat{A}} = \dfrac{\sen^2 \widehat{A} + \cos^2 \widehat{A}}{\cos^2 \widehat{A}} = \dfrac{\sen^2 \widehat{A}}{\cos^2 \widehat{A}} + 1 = \tan^2 \widehat{A} + 1$$\\

	    \item $1+ctan^2 \; \widehat{A}=\csc^2 \widehat{A}$\\\\
		Demostración.-\; $$csec^2 \;\widehat{A} =\dfrac{1}{\sen^2 \widehat{A}} = \dfrac{\sen^2\widehat{A} + \sen^2 \widehat{A}}{\sen^2 \widehat{A}} = 1+ \dfrac{\cos^2 \widehat{A}}{\sen^2\widehat{A}} = 1 + ctan^2 \; \widehat{A} = 1 + \dfrac{1}{\tan^2 A}$$\\
	\end{enumerate}

\end{enumerate}



\section*{Área}

\begin{enumerate}[\Large\bfseries 1.]

%--------------------1.
\item Determine el área de un triángulo equilátero de lado $s$.\\\\
    Respuesta.-\; Partamos de $Área = \dfrac{s\cdot h}{2}$ por el teorema de Pitágoras se tiene que $$s^2 = h^2 + \left(\dfrac{s}{2}\right)^2 \quad \Longrightarrow \quad h=\dfrac{\sqrt{3}s}{2}$$
    Remplazado tenemos $$ Area = \dfrac{s \cdot \dfrac{\sqrt{3}s}{2}}{2} = \dfrac{\sqrt{3}}{4} \cdot s^2$$

%--------------------2.
\item ¿Qué relación satisfacen las áreas de dos triángulos rectángulos semejantes?\\\\
    Respuesta.-\; La relación se denomina razón de las áreas de dos triángulos similares viene dada por el cuadrado de la razón de similitud entre ellos.\\\\

%--------------------3.
\item El radio del círculo inscrito en un polígono regular es llamado apotema del polígono regular. Pruebe que el área de un polígono regular es igual a la mitad del producto de su perímetro y su apotema.\\\\
    Demostración.-\; Sea,
	\begin{center}
	    \begin{tikzpicture}[scale = .8]
		\draw(0,0) circle (2cm);
		\draw(0,2)node[above]{$C$}--(1.8,.8)node[right]{$D$}--(0,0)--(0,2)--(-1.8,.8)node[left]{$B$}--(0,0);
		\draw(1.8,.8)--(1.3,-1.5)node[right]{$E$}--(0,0)node[above right]{$O$}--(-1.3,-1.5)node[left]{$A$}--(-1.8,.8);
		\draw(-1.3,-1.5)--(1.3,-1.5);
		\draw(0,0)--(0,-1.5);
		\draw(-1.5,-.4)node[right]{$s$};
		\draw(0,-.8)node[left]{$a$};
	    \end{tikzpicture}
	\end{center}
	Como se muestra en la figura al dibujar los radios que van hasta los vértices de un polígono de $n$ lados y perímetro $p=ns$, se divide en $n$ triángulos, cada uno de área $\frac{1}{2} as$. Como el área de un polígono regular es $n\left(\frac{1}{2}as\right)=\frac{1}{2}nsa=\frac{1}{2}pr$\\\\

%--------------------4.
\item Determine el área de un hexágono regular inscrito en un círculo de radio $R$.\\\\
    Respuesta.-\;
	\begin{center}
	    \begin{tikzpicture}
		\draw(0,0) circle (2cm);
		\draw(-1,1.7)--(1,1.7)--(2,0)--(1,-1.7)--(-1,-1.7)--(-2,0)--(-1,1.7);
		\draw(1,-1.7)--(0,0)--(-1,-1.7);
		\draw(0,0)--(0,-1.7);
		\draw(-.9,-.7)node[right]{$R$};
		\draw(0,-1)node[right]{$a$};
		\draw(-.5,-1.5)node[]{$L/2$};
	    \end{tikzpicture}
	\end{center}
	Sea $R^2 = \left(\dfrac{L}{2}\right)^2 + a^2$, si $L=R$ entonces $R^2 = \dfrac{R^2}{4} + a^2$ de donde $a=\dfrac{R \sqrt{3}}{2}$. Luego Área$ = \dfrac{P\cdot a}{2} $, por ser un hexágono se tiene $P=6L=6R$, por lo tanto Área$=\dfrac{3R^2\cdot \sqrt{3}}{2}$\\\\ 

%--------------------5.
\item Pruebe que la razón entre las longitudes de dos círculos es igual a la razón entre sus radios.\\\\
    Demostración.-\; Debemos probar que $\dfrac{C_1}{C_2}=\dfrac{r_1}{r_2} \qquad (1)$.\\
	Sea $C_1=2\pi r_1$  y $C_2=2\pi r_2$. Remplazado a $(1)$ tenemos que $$\dfrac{2\pi r_1}{2\pi r_2}=\dfrac{r_1}{r_2} \quad \Longrightarrow \quad 2\pi = 2\pi$$\\

%--------------------6.
\item Pruebe que la razón entre las áreas de dos círculos es igual a la razón entre los cuadrados de sus radios.\\\\
    Demostración.-\; Debemos demostrar que $\dfrac{S_1}{S_2}=\dfrac{r_1^2}{r_2^2}$.\\
	Sea $S_1=\pi r_1^2$ y $S_2 = \pi r_2^2$, entonces $$\dfrac{\pi r_1^2}{\pi r_2^2}=\dfrac{r_1}{r_2} \quad \Longrightarrow \quad \pi=\pi$$\\

%--------------------7.
\item Si los diámetros de dos círculos son $3$ y $6$, ¿cuál la relación entre sus áreas?\\\\
    Respuesta.-\; 
	\begin{center}
	    \begin{tabular}{r|l}
		$C_1=2\pi r_1$&$C_2 = 2\pi r_2$\\\\
		$3=2\pi r_1 \Longrightarrow r_1=\dfrac{3}{2\pi}$&$6=2\pi r_2 \Longrightarrow r_2=\dfrac{3}{\pi}$\\\\
		\hline \\
		$A_1=\pi r_1^2$&$A_2=\pi r_2^2$\\\\
		$A_1=\pi \left(\dfrac{3}{2\pi}\right)^2$&$A_2 = \pi \left(\dfrac{3}{\pi}\right)^2$\\\\
		$A_1=\dfrac{9}{4\pi}$&$A_2=\dfrac{9}{\pi}$\\\\
	    \end{tabular}
	\end{center}
	De donde la relación viene dada por $$4A_1=A_2$$\\

%--------------------8.
\item ¿Cuál es el área de un cuadrado inscrito en un círculo cuyo radio mide $5 \;cm$?\\\\
    Respuesta.-\;
    \begin{center}
	\begin{tikzpicture}[scale=1]
	    \draw(0,0) circle (2cm);
	    \draw(-2,0)--(0,2)--(2,0)--(0,-2)--(-2,0)--(0,0)--(0,2)(0,1)node[right]{$5$};
	    \draw(-1,0)node[below]{$5$};
	\end{tikzpicture}
    \end{center}
    Por el problema $(1)$ se tiene que el área de un triangulo isosceles es dado por $\dfrac{\sqrt{3}}{4}\cdot 5^2$ por lo tanto $$Area = \sqrt{3}\cdot 5^2 = 43.3\; cm^2$$\\

%--------------------9.
\item Dos hexágonos regulares tienen lados de medida $2\; cm$ y $3\; cm$. ¿Cuál es la relación entre sus áreas?\\\\
    Respuesta.-\; Por el problema $(4)$
    \begin{center}
	\begin{tabular}{r|l}
	    $A_1=\dfrac{3\cdot 3^2 \cdot \sqrt{3}}{2}$&$A_2=\dfrac{3\cdot 2^2 \cdot \sqrt{3}}{2}$\\\\
	    $A_1=\dfrac{27\sqrt{3}}{2}$&$A_2=6\sqrt{3}$\\\\
	    $\dfrac{2A_1}{27}=\sqrt{3}$&$\dfrac{A_2}{6}=\sqrt{3}$\\\\
	\end{tabular}
    \end{center}
    Igualando se tiene que $$12A_1 = 27A_2 \quad o \quad \dfrac{A_2}{A_1}=\dfrac{12}{27}$$\\

%--------------------10.
\item La longitud de un círculo vale dos veces la longitud de otro círculo. ¿Qué relación satisfacen sus áreas?\\\\
    Respuesta.-\; Sea $L_1=2\pi r_1$ \; y \; $L_2=\dfrac{2\pi r_2}{2}=\pi r_2 \quad \Longrightarrow \quad  r_1=\dfrac{L_1}{2\pi}$ \; y \; $r_2=\dfrac{L_2}{\pi}$ \; respectivamente. Luego $A_1 = \pi \cdot r_1^2$ \; y \; $A_2 =  \pi \cdot r_2^2$ \; de donde \; $A_1 = \pi \dfrac{L_1^2}{4\pi^2}$ \; y \; $A_2 = \pi \dfrac{L_2^2}{\pi^2}$ \; por lo tanto \; $$A_1 = \dfrac{L_1^2}{4\pi} \quad y \quad A_2=\dfrac{L_2^2}{\pi}$$
    Así la relación viene dada por $$\dfrac{4A_1}{L_1^2}=\dfrac{A_2}{L_2^2}$$

%--------------------11.
\item El área de un círculo vale cinco veces el área de otro círculo. ¿Qué relación satisfacen sus radios?\\\\
    Respuesta.-\;

%--------------------12.
\item ¿Cuánto papel seria necesario para cubrir la cara externa de una lata cilíndrica cuya altura es $15\; cm$ y cuyo radio de la base es $5\; cm$?\\\\
    Respuesta.-\;

%--------------------13.
\item  Muestre que si dos triángulos son semejantes entonces la razón entre sus áreas es igual a la razón entre los cuadrados de cualesquier dos de sus pares de lados correspondientes.\\\\
    Demostración.-\; Sea dos triángulos semejantes $ABC$ de área = $S_1$ y $A^{'} B^{'} C^{'}$ de área = $S_2$. Consideremos loa lados $a_1$ y $a_2$ (lados similares) de los dos triángulos y $h_1$ como $h_2$ las alturas semejantes de estos triángulos. Como son semejantes podemos escribir: $$\dfrac{a_1}{a_2}=\dfrac{h_1}{h_2}=k$$ luego $$\dfrac{S_1}{S_2}=\dfrac{1/2 (a_1 h_1)}{1/2 (a_2 h_2)} \quad \Longrightarrow \quad \dfrac{a_1}{a_2}\cdot \dfrac{h_1}{h_2}=K^2 \quad \Longrightarrow \quad \dfrac{S_1}{S_2}=K^2$$\\

%--------------------14.
\item Muestre que la razón entre las áreas de dos polígonos semejantes es igual a la razón entre los cuadrados de cualesquier dos de sus lados correspondientes.\\\\
    Demostración.-\;

%--------------------15.
\item Tres polígonos semejantes son construidos teniendo cada uno de ellos, como lado, unos de los lados de un triángulo rectángulo. Pruebe que el área del mayor de ellos es igual a la suma de las áreas de los dos menores.\\\\
    Demostración.-\;

%--------------------16.
\item La región limitada por dos radios y un arco de un círculo es llamado sector del círculo. Muestre que el área de un sector es $\frac{1}{2}RS$ donde $R$ es el radio del círculo y $S$ es la longitud del arco.\\\\
    Demostración.-\;

%--------------------17.
\item Pruebe que, en dos círculos de radios $R$ y $kR$, cuerdas de longitud $c$ y $kc$, respectivamente, subtienden arcos de longitudes $S$ y $kS$\\\\

%--------------------18.
\item Determine el área de la región limitada por una cuerda y por el arco del círculo que ella subtiende.\\\\

%--------------------19.
\item Determine el área de la parte de la región circular que queda entre dos de la cuerdas del círculo.\\\\

%--------------------20.
\item  En la figura siguiente se sugiere una manera de demostrar el Teorema de Pitágoras. Para hacer la demostración, exprese el área del cuadrado mayor de dos maneras diferentes: como producto de los lados y como la suma de las áreas de los cuatro triángulos y el área del cuadrado menor.\\\\

%--------------------21.
\item Otra prueba del Teorema de Pitágoras es sugerida por la figura siguiente. Determine el área del trapecio de dos maneras diferentes: directamente y como suma de áreas.\\\\
    Demostración.-\; Trazamos una un segmento como se vera a continuación en  la figura, obteniendo un trapezoide rectangular formado por los dos triángulos rectangulares iniciales (iguales) y otro triángulo que, como demostraremos, también es un triángulo rectangular.
    \begin{center}
	\begin{tikzpicture}
	    \draw(0,0)--(2,0)node[below]{$c$}--(4,0)--(4,.5)node[right]{$b$}--(4,3)node[right]{$c$}--(4,5)--(3,5)node[above]{$b$}--(2,5)--(4,1)--(0,0)--(2,5);
	    \draw(2.7,3)node[]{$a$};
	    \draw(2,0.8)node[]{$a$};
	    \draw(3.2,4)node[]{$A$};
	    \draw(3.2,.4)node[]{$A$};
	    \draw(2,2.2)node[]{$B$};
	    \draw(2.4,4.7)node[]{$\alpha$};
	    \draw(3.8,.7)node[]{$\alpha$};
	    \draw(3.8,1.9)node[]{$\beta$};
	    \draw(1.4,.15)node[]{$\beta$};
	\end{tikzpicture}
    \end{center}
    Necesitamos mostrar que el ángulo $\theta$ tiene una medida de $90$, para confirmar la afirmación de que el tercer triángulo $(B)$ también es un rectángulo.\\
    Como el triángulo inicial $(A)$ es un rectángulo, tenemos que los ángulos $\alpha$ y $\beta$ suman $90^\circ$ (según la ley angular de Tales). Así, mirando los tres ángulos formados alrededor del punto $P$, y en el mismo lado de una línea recta, tendremos que $\alpha + \beta + \theta = 180^\circ$, lo que nos lleva a concluir que $\theta$ también mide $90^\circ$  y el triángulo $B$ es también rectángulo.\\
    Nótese también que las tres partes unidas generaron un TRAPECIO DE ÁNGULO RECTO, cuya altura es $b + c$ cuyas bases son $b$ y $c$. Podemos calcular el área de este trapecio de dos formas:
    \begin{enumerate}[\bfseries a)]
	\item Directamente por la fórmula de área del trapecio.
	\item Suma de las áreas de los tres triángulos rectangulares ($2A$ y $1B$)\\
    Por supuesto, no importa la forma del cálculo, estos dos resultados deben ser iguales. Veamos:
	    Por a) (la mitad de la suma de las bases) x altura o $\dfrac{b+c}{2}(b+c)$\\
	    Por b) suma de las áreas de las partes: $2A+B$ o $2\left(\\dfrac{bc}{2}\right)+\left(\dfrac{a^2}{2}\right)=bc+\dfrac{a^2}{2}$\\
	    Igualando las dos expresiones obtenidas, tendremos:
	    $$bc+\dfrac{a^2}{2}=\dfrac{(b+c)^2}{2} \quad \Longrightarrow \quad 2bc+a^2=b^2+2bc+c^2$$
	    y finalmente $a^2=b^2+ c^2$\\\\
    \end{enumerate}

%--------------------22.
\item Blaskara, un matemático hindú del siglo doce, creo una prueba del Teorema de Pitágoras basada en la figura siguiente. Haga esta demostración.\\\\

%--------------------23.
\item  Cualesquier dos cuadrados puede ser cortados en cinco pedazos de tal forma que estos cinco pedazos pueden ser reagrupados para formar un nuevo cuadrado. La manera de hacer los cortes se indica en la figura siguiente, para el caso particular en que los cuadrados considerados tengan lados uno el doble del otro. Después de verificar como se construye el cuadrado con los cinco pedazos, diga como determinar los cortes en el caso en que los cuadrados tengan lados $12$ y $8$.\\\\

\end{enumerate}


\end{document}
