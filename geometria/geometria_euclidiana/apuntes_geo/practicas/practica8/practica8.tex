\documentclass[10pt]{article} 						
\usepackage[text=17cm,left=2.5cm,right=2.5cm, headsep=20pt, top=2.5cm, bottom = 2cm,letterpaper,showframe = false]{geometry} %configuración página
\usepackage{latexsym,amsmath,amssymb,amsfonts} %(símbolos de la AMS).7
\parindent = 0cm  %sangria
\usepackage[T1]{fontenc} %acentos en español
\usepackage[spanish]{babel} %español capitulos y secciones
\usepackage{graphicx} %gráficos y figuras.
%-----------------------------------------%
\usepackage{multicol}
\usepackage{titlesec}
\usepackage[rflt]{floatflt}
\usepackage{wrapfig} 
\usepackage{tikz}
\usepackage{tkz-euclide}
\usetikzlibrary{decorations.markings,arrows}
\usetikzlibrary{matrix,arrows, positioning,shadows,shadings,backgrounds,
calc, shapes, tikzmark}
\usepackage{tcolorbox, empheq}
\tcbuselibrary{skins,breakable,listings,theorems}
\usepackage{xparse}							
\usepackage{pstricks}							
\usepackage[Bjornstrup]{fncychap}			
\usepackage{rotating}
\usepackage{enumerate}
\usepackage{booktabs}
\usepackage{synttree} 
\usepackage{chngcntr}
\usepackage{venndiagram}
\usepackage[all]{xy}
\usepackage{xcolor}
\usepackage{tikz}
\usetikzlibrary{datavisualization.formats.functions}
\usepackage{marginnote}										
\usepackage{fancyhdr}
\usepackage{yhmath}

%------------------------------------------
\renewcommand{\labelenumi}{\Roman{enumi}.}		%primer piso II) enumerate
\renewcommand{\labelenumii}{\arabic{enumii}$)$ }%segundo piso 2)
\renewcommand{\labelenumiii}{\alph{enumiii}$)$ }%tercer piso a)
\renewcommand{\labelenumiv}{$\bullet$}			%cuarto piso (punto)


\pagestyle{fancy}
\fancyhead[R]{Geometría I}
\fancyhead[L]{Práctica IV}

\begin{document}
\begin{tabular}{r l }
Universidad: & \textbf{Mayor de San Ándres.}\\
Asignatura: & \textbf{Geometría I.}\\
 Práctica: & IV.\\ Alumno: & \textbf{PAREDES AGUILERA CHRISTIAN LIMBERT.}
\end{tabular}
\begin{flushleft}
\begin{tikzpicture}
\draw(0,1)--(16.5,1);
\end{tikzpicture}
\end{flushleft}


\begin{enumerate}[\Large\bfseries 1.]
    
%-------------------1.
\item Pruebe que, en un mismo círculo o en círculos de mismo radio, cuerdas congruentes son equidistantes del centro.\\\\
    Demostración.-\;
    \begin{center}
	\begin{tikzpicture}
	    \draw (0,0)node[right]{$O$} circle (1cm);
	    \draw (-.6,.8)node[above]{$A$}--(.6,.8)node[above]{$B$}--(-.6,-.8)node[below]{$C$}--(.6,-.8)node[below]{$D$}--(-.6,.8);
	    \draw(0,.8)node[above]{$E$}--(0,-.8)node[below]{$F$};
	\end{tikzpicture}
    \end{center}
    Construyamos una circunferencia de centro $O$ con segmentos  $AB = CD$. Por $O$ trazamos los segmentos $OA = OB = OC = OD = \; Radio$. Entonces $\triangle ABO$ es isosceles, lo mismo para $\triangle COD$. Como $A\widehat{O}B = C\widehat{O}D$, ya que están opuestos por el vértice, entonces $\triangle ABO = \triangle COD$ por el caso $LAL$. Trazando los segmentos $OE$ y $OF$ de manera que $OE$ y $EF$ son alturas de los triángulos, por lo tanto perpendiculares a $AB$ y $CD$ respectivamente. Como $\triangle AOB = \triangle COD$ entonces $EO = OF$ por lo tanto $AB$ y $CD$ son equidistantes.\\\\

%--------------------2.
\item Pruebe que, en un mismo círculo o en círculos de mismo radio, cuerdas equidistantes del centro son congruentes.\\\\
    Demostración.-\; 
    \begin{center}
	\begin{tikzpicture}[scale=0.6]
	    \draw (0,0)node[left]{$O$} circle (2cm);
	    \draw (-.9,-1.8)node[below]{$C$}--(0,0)--(.6,1.9)node[above]{$A$}--(1.8,.8)node[above right]{$B$}--(0,0)--(.8,-1.8)node[below]{$D$}--(-.9,-1.8);
	    \draw(0,-1.8)node[below]{$F$}--(0,0)--(1.2,1.3)node[above]{$E$};
	\end{tikzpicture}
    \end{center}
    Sabemos por el problema anterior que si dos arcos son equidistantes, entonces hay una perpendicular a cada arco que es congruente, es decir: $OE=OF$ por hipótesis tenemos que $OE$, $OF \perp AB,$ $CD$ respectivamente. Así mismo $\triangle OBE = \triangle OCF =\triangle OFD$ por el caso cateto hipotenusa. Por lo tanto $AE=EB=CF=FD$ así, $$AE+EB=CF+FD$$ $$AB=CD.$$\\\\

%--------------------3.
\item Pruebe que, en un mismo círculo o en círculos de mismo radio, si dos cuerdas tienen longitudes diferentes, la más corta es la más alejada del centro.\\\\
    Demostración.-\; 
    \begin{center}
	\begin{tikzpicture}[scale=0.6]
	    \draw (0,0)node[left]{$O$} circle (2cm);
	    \draw (-1.8,-1)node[below]{$C$}--(0,0)--(.6,1.9)node[above]{$A$}--(1.8,.8)node[above right]{$B$}--(0,0)--(1.8,-.9)node[below]{$D$}--(-1.8,-1);
	    \draw(0,-.9)node[below]{$F$}--(0,0)--(1.2,1.3)node[above]{$E$};
	\end{tikzpicture}
    \end{center}
    Como $A,B,C$ y $D$ pertenece al circulo entonces: $$OC=OD=OA=OB=Radio$$ Luego $\triangle COD, \triangle AOB$ son isosceles.\\
    Los segmentos $OE$ y $OF$ para tener $\triangle AOB=\triangle COD$ ambos rectángulos. Entonces por el teorema de Pitágoras: $$OA^2 = OF^2 + AF^2 \quad o \quad OC^2 = OE^2 = CE^2$$ Luego como $OA=OC=Radio$ $$OF^2 + AF^2 = OE^2 + CE^2 \quad (1)$$
    Como $AB<CD$ por hipótesis o $F$ o $E$ son puntos medios de $AB$ y $CD$ respectivamente, debido a que $\triangle AOB, \triangle COD$ son isosceles, entonces $AF<CE$ que obliga a la desigualdad $OF> OE$ a mantener la igualdad en $(1)$.\\\\

%--------------------4.
\item Muestre que la mediatriz de una cuerda pasa por el centro del círculo.\\\\
    Demostración.-\; Dada una cuerda $AB$ y una mediatriz $m$ que corta a $AB$ en el punto $E$ de manera que $AE = EB$ y $m\perp AB$, como se ve a continuación:
    \begin{center}
	\begin{tikzpicture}[scale=0.6]
	    \draw (0,0)node[below left]{$O$} circle (2cm);
	    \draw (0,0)--(-.8,1.8)node[above]{$A$}--(1.7,1)node[above]{$B$}--(0,0);
	    \draw[gray](-.5,-2)--(.7,2.3)node[above]{$m$};
	\end{tikzpicture}
    \end{center}
    Luego $AO=OB=Radio$ como también $\triangle AOB$ es isosceles de base $AB$.\\
    Sea $OE$ la mediana relativa a la base $AB$ del $\triangle AOB$ entonces, $OE \perp AB$. Cuando un punto pasa por una sola línea perpendicular, entonces $OE$ es la propia mediana que pasa por el punto $O$.\\\\

%--------------------5.
\item Explique porque el reflejo de un círculo relativo a una recta que pasa por su centro es el mismo círculo.\\\\
    Demostración.-\; Recordando las propiedades de reflexión tenemos: $$F_m (A) = A \; si \; A\in m,$$luego 
    \begin{center}
	\begin{tikzpicture}[scale=.6]
	    \draw (0,0)node[left]{$O$} circle (2cm);
	    \draw (.7,-1.9)node[below]{$A{'}$}--(.7,1.9)node[above]{$A$}--(0,0)--(.7,-1.9);
	    \draw (-2.3,0)--(2,0)node[right]{$m$};
	    \draw (.7,0)node[below right]{$E$};
	\end{tikzpicture}
    \end{center}
    Al dibujar una línea recta $m$ que pasa por el centro del círculo, la reflexión del centro es el centro mismo. Sea $A$ cualquier punto que pertenezca al círculo, entonces hay un segmento $AA^{'}$ que se cruza con $m$ en el punto $E$ tal que $AE = A^{'}E$ y $AA^{'} \perp  m$. Luego trazamos $\triangle AOE = \triangle EOA^{'}$ que son congruentes en el caso $LAL$. Entonces $OA = OA^{'}$ y por lo tanto la reflexión de $A$ también pertenece al círculo.\\\\

%--------------------6.
\item En la figura, existen tres rectas que son tangentes simultaneamente a los dos círculos. Estas rectas se dicen tangentes comunes a los círculos. Diga si se puede diseñar dos círculos que tengan:

    \begin{center}
	\begin{tikzpicture}
	    \draw(0,0) circle (1.1cm);
	    \draw(.3,-2) circle (.9cm);
	    \draw[<->] (-1.3,.9)--(-.4,-3.2);
	    \draw[<->] (-1.6,-1.2)--(2,-1);
	    \draw[<->] (1.05,1)--(1.3,-3.2);
	\end{tikzpicture}
    \end{center}

    \begin{enumerate}[\bfseries a)]
	
	%----------a)
	\item Cuatro tangentes comunes,\\\\
	    Respuesta.-\; Se puede diseñar 4 tangentes comunes si dos las cruzamos por en medio de los circulos.\\\\

	%----------b)
	\item Exactamente dos tangentes comunes,\\\\
	    Respuesta.-\; Se puede diseñar sobreponiendo un circulo con el otro.\\\\

	%----------c)
	\item Solamente una tangente común,\\\\
	    Respuesta.-\; Se puede graficar solamente una tangente si hacemos que el circulo este contenido en el otro.\\\\

	%----------d)
	\item ninguna tangente en común,\\\\
	    Respuesta.-\; Se podría siempre y cuando uno de los círculos fuese más pequeño y no tocara con la circunferencia del otro.\\\\  

	%----------e)
	\item más de cuatro tangentes en común.\\\\
	    Respuesta.-\; No se puede graficar mas de 4 tangentes en común.\\\\

    \end{enumerate}


%--------------------7.
\item En la figura $AE$ es tangente común y $JS$ une los centros de los círculos. Los puntos $E$ y $A$ son puntos de tangencia y $M$ es el punto de intersección de los segmentos $JS$ y $AE$. Pruebe que $\widehat{J} = \widehat{S}.$\\\\
    Respuesta.-\; 
    \begin{center}
	\begin{tikzpicture}
	    \draw[<->](0,0)--(3,-3.5);
	    \draw(1.45,-.2) circle (1cm);
	    \draw(1.4,-2.7) circle (.7cm);
	    \draw(1.9,-2.2)node[right]{$A$}--(1.4,-2.7)node[below]{$J$}--(1.45,-.2)node[above]{$S$}--(.7,-.8)node[left]{$E$};
	    \draw(1.4,-1.5)node[right]{$M$};
	\end{tikzpicture}
    \end{center}
    Si un radio tiene una línea tangente, la circunferencia en su extremo es entonces perpendicular a la línea tangente.\\
    Según el teorema $\triangle ESB$ y $\triangle JBA$ son rectángulos y $E\widehat{B}S = J\widehat{B}A$, ya que están colocados por el vértice. Como $\triangle ESB$ y $\triangle JBA$ tienen dos ángulos congruentes, entonces son similares y por lo tanto $\widehat{S} = \widehat{J}$.\\\\

%--------------------8.
\item En la figura siguiente a izquierda, $M$ es el centro de los dos círculos y $AK$ es tangente al círculo menor en el punto $R$. Muestre que $AR = RK$.\\\\
    Demostración.-\;
    \begin{center}
	\begin{tikzpicture}
	    \draw(0,0)node[above]{$M$} circle (.8cm);
	    \draw(0,0) circle (1.7cm);
	    \draw(0,0)--(-1.5,-.8)node[left]{$A$}--(1.5,-.8)node[right]{$K$}--(0,0)--(0,-.8);
	    \draw(0,-.8)node[below]{$R$};
	\end{tikzpicture}
    \end{center}
    Construyamos $\triangle MAR$ y $\triangle MRK$ tal que $AM=MK=Radio$\\
    Con $m$ trazamos una línea que interseca a $AK$ en el punto $R$. Ahora bien, si un rayo corta a una línea en su punto de tangencia, es perpendicular a la línea. En base a esto tendremos $A\widehat{R}M = M\widehat{R}K = 90^{\circ}$. Por lo tanto, $\triangle AMR y \triangle MRK$son rectángulos y por el criterio de hipotenusa de los triángulos rectangulares $\triangle MRK = \triangle AMR$. Luego $AR = RK$.\\\\

%--------------------9.
\item En la figura anterior a derecha, $L$ es el centro del círculo, $UK$ es tangente al círculo en el punto $U$ y $UE = LU$. Muestre que $LE = EK.$\\\\
    Demostración.-\; 
    \begin{center}
	\begin{tikzpicture}[scale=.7]
	    \draw(0,0) circle (1.6cm);
	    \draw(0,0)node[left]{$L$}--(0,-1.6)node[below]{$U$}--(3,-1.6)node[right]{$K$}--(0,0);
	    \draw(0,-1.6)--(1.4,-.75)node[right]{$E$};
	\end{tikzpicture}
    \end{center}
    Si $UK$ es tangente al círculo en el punto $U$, entonces $UK \perp LU$ entonces $L\widehat{U}K = 90^{\circ}$. Por hipótesis $UE = LU$ y como $LE$ es radio, $LE = LU = UE$. Entonces, $\triangle LUE$  es equilátero y $L\widehat{E}U = L\widehat{U}E = U\widehat{L}E = 60^{\circ}$. Como $L\widehat{E}U$ es el ángulo externo $\triangle EUK$, entonces: $$L\widehat{E}U = E\widehat{K}U + E\widehat{U}K$$
    Sin embargo como $L\widehat{U}K = L\widehat{U}E + E\widehat{U}K$ entonces $L\widehat{U}E + E\widehat{U}K = 90^{\circ} \qquad (1)$\\
    Como $L\widehat{U}E = 60^{\circ}$ por $(1)$ $E\widehat{U}K=30^{\circ}$\\
    Así $L\widehat{E}U = E\widehat{U}K + E\widehat{K}U$ implica que: $$60^{\circ} = 30^{\circ} + E\widehat{K}U$$ $$E\widehat{K}U = 30^{\circ}$$ Luego $\triangle E\widehat{U}K$ es isosceles de base $UK$, porque tienen dos ángulos de $30^{\circ}$, por lo que $EK = EU \qquad (2)$. Como $UE=LU=LE \qquad (3)$. \\
    Por $(2)$ y $(3)$ tenemos $LE=EK$.\\\\

%--------------------10.
\item En la figura siguiente a izquierda, $MO = IX$. Pruebe que $MI = OX$.\\
Dos puntos en un círculo determinan dos arcos. Si los puntos son $A$ y $B$ denotamos por $AB$ al arco menor determinado por estos dos puntos. Si $P$ también pertenece al círculo usaremos la notación $APB$ para representar al arco que contiene al punto $P$.\\\\
    Demostración.-\; Trazando una cuerda $MX$ es posible notar que $M\widehat{O}X = M\widehat{I}X$, porque ambos tienen la misma cuerda.
    \begin{center}
	\begin{tikzpicture}
	    \draw(0,0)node[]{$B$} circle (1cm);
	    \draw(-.9,.5)node[left]{$M$}--(.85,.5)node[right]{$X$}--(-.4,-0.9)node[left]{$O$}--(-.9,.5)--(.4,-.9)node[right]{$I$}--(.85,.5);
	\end{tikzpicture}
    \end{center}
    Como $\triangle MOB$ y $\triangle BXI$ tienen dos ángulos congruentes estos son similares por lo tanto: $$\dfrac{MO}{XI}=\dfrac{OB}{BI}$$
    Como $MO=XI$ por hipótesis: $$\dfrac{OB}{BI} = 1 \quad \Rightarrow \quad OB=BI$$
    Así $\triangle MOB = \triangle BXI$ por el caso $LAL$ y $MB=BX$, por tanto:
    $$MB+BI=BX+OB$$ $$MI=XO$$\\

%--------------------11.
\item En la figura anterior a derecha, $H$ es el centro del círculo y $CI$ es un diámetro. Si $CA$ y $HN$ son paralelos, muestre que $\wideparen{AN}$ y $\wideparen{IN}$ tienen la misma medida.\\\\
    Demostración.-\; 
    \begin{center}
	\begin{tikzpicture}
	    \draw(0,0)node[above]{$H$} circle (1cm);
	    \draw(1,0)node[right]{$I$}--(-1,0)node[left]{$C$}--(.4,.9)node[above]{$A$}--(0,0)--(.9,.4)node[right]{$N$};
	\end{tikzpicture}
    \end{center}
    Tenemos $AH$ de donde $\widehat{C}=0.5 (A\widehat{H}I) \qquad (1)$\\
    Notese que $CA=HA=HN=HI=Radio$ por el paralelismo entre $CA$ y $HN$. También podemos ver que $C\widehat{A}H=A\widehat{H}N$, por los ángulos alternos internos.\\
    Como $\triangle ACH$ es equilátero $AH=CA=CH=Radio$ entonces $\widehat{C}=C\widehat{A}H = A\widehat{H}N$.\\
    Luego de $(1)$ vemos: $$\widehat{C} = 0.5 A\widehat{H}I = 0.5(A\widehat{H}N + N\widehat{H}I)$$ $$\widehat{C} = 0.5(A\widehat{H}N + N\widehat{H}I)$$
    Como $A\widehat{H}N=\widehat{C}$ $$\widehat{C} - 0.5 \widehat{C}=0.5N\widehat{H}I$$ $$0.5\widehat{C} = 0.5N\widehat{H}I$$
    $$N\widehat{H}I=\widehat{C}=A\widehat{H}N$$ Concluyendo que $N\widehat{H}I=A\widehat{H}N$. Como ángulos centrales iguales dan como resultado cuerdas congruentes, completamos la demostración concluyendo que $\wideparen{AN} = \wideparen{IN}$\\\\

%--------------------12.
\item  En la figura siguiente a izquierda, $O$ es el centro del círculo y $TA$ es un diámetro. Si $PA = AZ$, muestre que los triángulos $PAT$ y $ZAT$ son congruentes.
    \begin{center}
	\begin{tikzpicture}
	    \draw(0,0)node[below]{$O$} circle (1cm);
	    \draw(.95,.3)node[right]{$T$}--(-.95,.4)node[left]{$P$}--(-.8,-.5)node[left]{$A$}--(0,-.95)node[below]{$Z$}--(.95,.3)--(-.8,-.5);
	\end{tikzpicture}
    \end{center}
    Notemos que $T\widehat{P}A$ y $T\widehat{Z}A$ ambos se refieren a arcos formados por semicírculos de modo que $$T\widehat{P}A=T\widehat{Z}A=90^{\circ}$$ 
    Luego los triángulos $PAT$ y $ZAT$ son congruentes por el caso $PA=AZ$ y $TA$.\\\\

%--------------------13.
\item En la figura anterior a derecha, se sabe que $Y$ es el centro del círculo y que $BL = ER$. Muestre que $BE$ es paralelo a $LR$.\\\\

%--------------------14.
\item En la figura siguiente, el cuadrilátero $DIAN$ es un paralelogramo y son colineales los puntos $I$, $A$ y $M$. Muestre que $DI = DM$.\\\\
    Demostración.-\; Tenemos $D\widehat{N}A=D\widehat{M}A,$ porque se someten al mismo arco  $\wideparen{DA}$. Como $DIAN$ es un paralelogramo, $D\widehat{N}A = D\widehat{I}A$, así: $$D\widehat{M}A=D\widehat{I}A$$ 
    Como $I,A$ y $M$ son colineales y $\triangle DMI$ es isosceles de base $MI$ que implica que $DM=DI$.\\\\

%--------------------15.
\item En la figura siguiente, ¿cuál de los dos arcos,  $AH$ o $MY$ tiene mayor medida en grados? Se sabe que los dos círculos son concéntricos.\\\\
    Respuesta.-\; 
    \begin{center}
	\begin{tikzpicture}
	    \draw(0,0)node[right]{$o$} circle (.7cm);
	    \draw(0,0) circle (1.5cm);
	\end{tikzpicture}
    \end{center}
    Note que $A\widehat{O}H$ es un ángulo central de la misma circunferencia que $A\widehat{T}H$ está inscrita y por tanto:
    $$A\widehat{T}H=\dfrac{A\widehat{O}H}{2}$$ 
    Como $A\widehat{T}H$ relativo al arco $\wideparen{AH}$ y $M\widehat{O}Y$ relativo al arco $\wideparen{MY}$.\\
    Como $M\widehat{O}Y>A\widehat{O}H$ que implica directamente que $\wideparen{MY}=\wideparen{AH}$\\\\

%--------------------16.

%--------------------17.

%--------------------18.

%--------------------19.

%--------------------20.

%--------------------21.

%--------------------22.

%--------------------23.

%--------------------24.

%--------------------25.

%--------------------26.

%--------------------27.

%--------------------28.

%--------------------29.

%--------------------30.

%--------------------31.

%--------------------32.

\end{enumerate}
\end{document}
