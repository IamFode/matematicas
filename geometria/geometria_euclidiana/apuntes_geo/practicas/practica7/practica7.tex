\documentclass[10pt]{article} 						
\usepackage[text=17cm,left=2.5cm,right=2.5cm, headsep=20pt, top=2.5cm, bottom = 2cm,letterpaper,showframe = false]{geometry} %configuración página
\usepackage{latexsym,amsmath,amssymb,amsfonts} %(símbolos de la AMS).7
\parindent = 0cm  %sangria
\usepackage[T1]{fontenc} %acentos en español
\usepackage[spanish]{babel} %español capitulos y secciones
\usepackage{graphicx} %gráficos y figuras.
%-----------------------------------------%
\usepackage{multicol}
\usepackage{titlesec}
\usepackage[rflt]{floatflt}
\usepackage{wrapfig} 
\usepackage{tikz}
\usepackage{tkz-euclide}
\usetikzlibrary{decorations.markings,arrows}
\usetikzlibrary{matrix,arrows, positioning,shadows,shadings,backgrounds,
calc, shapes, tikzmark}
\usepackage{tcolorbox, empheq}
\tcbuselibrary{skins,breakable,listings,theorems}
\usepackage{xparse}							
\usepackage{pstricks}							
\usepackage[Bjornstrup]{fncychap}			
\usepackage{rotating}
\usepackage{enumerate}
\usepackage{booktabs}
\usepackage{synttree} 
\usepackage{chngcntr}
\usepackage{venndiagram}
\usepackage[all]{xy}
\usepackage{xcolor}
\usepackage{tikz}
\usetikzlibrary{datavisualization.formats.functions}
\usepackage{marginnote}										
\usepackage{fancyhdr}

%------------------------------------------
\renewcommand{\labelenumi}{\Roman{enumi}.}		%primer piso II) enumerate
\renewcommand{\labelenumii}{\arabic{enumii}$)$ }%segundo piso 2)
\renewcommand{\labelenumiii}{\alph{enumiii}$)$ }%tercer piso a)
\renewcommand{\labelenumiv}{$\bullet$}			%cuarto piso (punto)


\pagestyle{fancy}
\fancyhead[R]{Geometría I}
\fancyhead[L]{Práctica IV}

\begin{document}
\begin{tabular}{r l }
Universidad: & \textbf{Mayor de San Ándres.}\\
Asignatura: & \textbf{Geometría I.}\\
 Práctica: & IV.\\ Alumno: & \textbf{PAREDES AGUILERA CHRISTIAN LIMBERT.}
\end{tabular}
\begin{flushleft}
\begin{tikzpicture}
\draw(0,1)--(16.5,1);
\end{tikzpicture}
\end{flushleft}

\section*{\center Soluciones}

\begin{enumerate}[\Large\bfseries 1.]
    
%-------------------1.
\item Cuánto mide la altura de un triángulo equilátero cuyos lados miden un centímetro.\\\\
    Respuesta.-\; Para encontrar la altura, debemos usar el teorema de Pitágoras $A^2=B^2+C^2$. Sabemos que la hipotenusa ($A$) es el lado más grande que será $1$ cm, y un lado será $0.5$ de altura $h$ ya que dividiremos el triángulo, transformando el triángulo equilátero en un triángulo rectángulo y la altura estará dado por: $$1^2=0.5^2+h2 \quad \Rightarrow \quad h=0.86$$\\\\

%-------------------2.
\item En el triángulo $ABC$, $\overline{AB} = 5$, $\overline{BC} = 12$ y $\overline{CA} = 13$. ¿Cuál es la medida del ángulo en $\widehat{B}$.\\\\
    Respuesta.-\; Sea $CA^2 = BC^2 + AB^2$ entonces $169=169$ por lo tanto el ángulo con vértice en $B$ mide $90^{\circ}$ ya que el segmento $CA$ es opuesto a $B$.\\\\ 

%--------------------3.
\item Muestre que los triángulos equiláteros siempre son semejantes.\\\\
    Demostración.-\; Sean $\triangle ABC$ y $\triangle DEF$ dos triángulos equiláteros, las medidas de los ángulos $\widehat{A}, \widehat{B}$ y $\widehat{C}$ son iguales a $60^{\circ}$, de la misma forma para los ángulos de triangulo $DEF$, entonces $$\widehat{A}=\widehat{D} \qquad \widehat{B}=\widehat{E} \qquad \widehat{C}=\widehat{F}$$ Luego por la razón de semejanza se tiene que $\dfrac{\overline{AB}}{\overline{DE}}=\dfrac{\overline{AC}}{\overline{DF}}=\dfrac{\overline{BC}}{\overline{EF}}=\alpha$ donde $\alpha$ es una constante. Por lo tanto Dos triángulos equiláteros son siempre semejantes.\\\\

%-------------------4.
\item Muestre que son semejantes dos triángulos isósceles que tienen iguales los ángulos opuestos a su base.\\\\
    Demostración.-\; Los ángulos $A\widehat{C}E=A\widehat{B}D$ son ángulos inscritos en la circunferencia con respecto a $AD$, por lo tanto $A\widehat{C}D=A\widehat{B}D$ Sea $G$ el punto de intersección del segmento $AC$ y la recta perpendicular al segmento $AC$ que pasa por $E$. Entonces, el triángulo $CEG$ y el rectángulo con ángulo recto en el vértice $G$. Como $C, D$ y $E$ son colineales y $A$ $C$ e $G$ son colineales, por lo que $G\widehat{C}E = ACD$. Como CEG y rectángulo en ángulo recto en el vértice $G$ y $G\widehat{C}E = A\widehat{C}D$, 
    entonces $C\widehat{E}G=90^{\circ} - G\widehat{C}E = 90^{\circ} - A\widehat{C}D$. Luego $C\widehat{E}G + C\widehat{E}B + B\widehat{E}F = 180^{\circ}$, $C\widehat{E}G=90^{\circ}$ y $C\widehat{E}G=90^{\circ}-A\widehat{C}D,$
    de donde $B\widehat{E}F=A\widehat{C}D$. Luego $B\widehat{E}F=A\widehat{C}D$ y $A\widehat{C}D=A\widehat{B}D$, entonces $B\widehat{E}F=A\widehat{B}D$. Sea $A,B$ y $E$ colineales como también $B,D$ y $F$, entonces $E\widehat{B}F=A\widehat{B}D$. 
    Por último $B\widehat{E}F=A\widehat{B}D$ y $E\widehat{B}F=A\widehat{B}D$, entonces $B\widehat{E}F=E\widehat{B}F$ y luego el triángulo $BEF$ es isosceles  siendo los lados $BF$ y $EF$ congruentes. Los ángulos $B\widehat{A}C=B\widehat{D}C$  son ángulos inscritos en la circunferencia con respecto a $BC$, por lo tanto $B\widehat{A}C=B\widehat{D}C$. \\
    Procediendo de manera completamente análoga lo que se hizo anteriormente, concluimos que $D\widehat{E}F=E\widehat{D}F$ y luego el triángulo $DEF$ es isosceles, como los lados $DF$ y $EF$ siendo congruentes. Así los segmentos de la recta $BF$ y $EF$ son congruentes y los segmentos de la recta $DF$ y $EF$ también son congruentes. Entonces las rectas $BF$ y $DF$ son congruentes, luego $F$ y el punto medio del segmento de la recta $BD$.\\\\

%--------------------5.
\item Pruebe que las alturas correspondientes en triángulos semejantes están en la misma razón que los lados correspondientes.\\\\
    Demostración.-\;
    \begin{multicols}{2}
    \begin{center}
	\begin{tikzpicture}[scale=0.9]
	   \draw(0,0)node[below]{$B$}--(4,0)node[below]{$C$}--(1.5,2.5)node[above]{$A$}--(0,0); 
	   \draw(1.5,2.5)--(1.5,0)node[below]{$M$};
	   \draw(2.5,0)node[below]{$a$};
	   \draw(2.8,1.5)node[right]{$b$};
	   \draw(1.5,1.2)node[left]{$h$};
	\end{tikzpicture}
    \end{center}

    \begin{center}
	\begin{tikzpicture}[scale=0.7]
	   \draw(0,0)node[below]{$B^{'}$}--(4,0)node[below]{$C^{'}$}--(1.5,2.5)node[above]{$A^{'}$}--(0,0); 
	   \draw(1.5,2.5)--(1.5,0)node[below]{$M^{'}$};
	   \draw(2.5,0)node[below]{$a^{'}$};
	   \draw(2.8,1.5)node[right]{$b^{'}$};
	   \draw(1.5,1.2)node[left]{$h^{'}$};
	\end{tikzpicture}
    \end{center}

    Considerando dos triángulos rectángulos $AMC$ y $A^{'} M^{'}C^{'}$ que tienen $\widehat{M}  = \widehat{M^{'}}$ y $\widehat{C}=\widehat{C^{'}}$ ya que $\triangle ABC$ es semejante a $\triangle A^{'}B^{'}C^{'}$. Resulta $\triangle AMC \sim \triangle A^{'}M^{'}C^{'}$, luego por definición de semejanza de triángulos $\dfrac{h}{h^{'}}=\dfrac{b}{b^{'}}$ y como $\dfrac{a}{a^{'}}=\dfrac{b}{b^{'}}=\dfrac{c}{c^{'}}$ por ser $\triangle ABC \sim \triangle A^{'}B^{'}C^{'}$ se tiene $\dfrac{h}{h^{'}}=\dfrac{a}{a^{'}}$. Análogamente se demuestra que $\dfrac{h}{h^{'}}= \dfrac{b}{b^{'}}$ y $\dfrac{h}{h^{'}}=\dfrac{c}{c^{'}}$ 

    \end{multicols}

%--------------------6.
\item  Pruebe que, si un triángulo rectángulo tiene sus ángulos agudos de $30^{\circ}$ y $60^{\circ}$, entonces su menor cateto mide la mitad de la longitud de la hipotenusa.\\\\
    Demostración-.\; Sea $ABC$ un triángulo rectángulo con hipotenusa $BC$ y angulos agudos $\widehat{C}=30^{\circ}$ y $\widehat{B}=60^{\circ}$. El Cateto más pequeño es $AB$, opuesto al ángulo interno más pequeño $\widehat{C}$. Luego sea $D \in S_{BA}$, tal que $\overline{DA}=\overline{AB}$, así trazamos un segmento $DC$. Como $DA=BA$, entonces por ángulos suplementarios $C\widehat{A}D=90^{\circ}=C\widehat{A}B$ y $AC=AC$, por $LAL$ queda $ADC=ABC$. En particular, $A\widehat{D}C=A\widehat{B}C=60^{'}, \; D\widehat{C}A=B\widehat{C}A=30^{\circ}$, esto es $D\widehat{C}B=60^{\circ}$ y el triángulo $DBC$ es equilátero. Luego $$\overline{AB}=\dfrac{\overline{BD}}{2} = \overline{CB}$$\\\\

%--------------------7.
\item En la figura, $D$ es el punto medio de $AB$ y $E$ el punto medio de $AC$. Muestre que los triángulos $ADE$ y $ABC$ son semejantes.\\\\

%--------------------8.
\item En la figura se tiene que $BDA \sim ABC$. Muestre que el triángulo $BDA$ es isósceles.\\\\

%--------------------9.
\item Muestre que todo triángulo de lados $p^2 - q^2$, $2pq$ y $p^2 + q^2$ es un triángulo rectángulo. Aquí, $p$ y $q$ son números enteros positivos arbitrarios tal que $p > q$.\\\\
    Demostración.-\; Si el triángulo es un rectángulo, debe valer el teorema de Pitágoras; de lo contrario, el triángulo no sería rectángulo. Demostraremos que este teorema es válido.
    $$(p^2 + q^2)^2 = (2pq)^2 + (p^2 - q^2)^2$$ $$(p^2+q^2)^2=4p^2q^2 + p^4 + q^4 - 2p^2q^2$$ 
    $$(p^2+q^2)^2=p^4 + 2p^2q^2 + q^4$$ $$(p^2 + q^2)^2=(p^2 + q^2)^2$$\\\\

%--------------------10.
\item Todos los triángulos indicados en la figura son rectángulos. Determine $a, b, c, d$ y $e$.\\\\

%--------------------11.
\item  Pruebe que la bisectriz de un ángulo de un triángulo divide al lado opuesto en segmentos proporcionales a los otros dos lados. Esto es, si $ABC$ es un triángulo y $BD$ es la bisectriz del ángulo $\widehat{B}$ siendo $D$ un punto del lado $AC$, entonces $$\dfrac{AD}{DC}=\dfrac{AB}{BC}$$
    Demostración.-\; La gráfica se verá de la siguiente manera:\\
\begin{center}
    \begin{tikzpicture}
	\draw(0,0)node[below]{$A$}--(3,0)node[below]{$C$}--(5,3)node[right]{$P$}--(0,0);	
	\draw(3,0)--(2.5,1.5)node[above]{$B$}--(2,0)node[below]{$D$};
    \end{tikzpicture}
\end{center}
Siendo $BD$ la bisectriz interna de $\widehat{B}$, tracemos una paralela que pase por el vértice $C$ la bisectriz $BD$, donde se encontrará una extensión de $BA$ en un punto $P$. Luego por el axioma de los paralelos, el triángulo $BPC$ es isosceles es decir que uno de sus ángulos alterna internamente con una de las mitades del ángulo $B$ y el otro correspondiente a la otra mitad. Por tanto según el teorema de Tales se sigue que $\dfrac{AD}{DC}=\dfrac{AB}{AP}$, como $BC=AP$ tenemos que $$\dfrac{AD}{DC}=\dfrac{AB}{BC}$$\\\\

%--------------------12.
\item Enuncie y pruebe la recíproca del ejercicio anterior.\\\\

%--------------------13.
\item 
    Demostración.-\; Sea $x$ el cateto más pequeño y $y, z$ los ángulos agudos y la medida de la hipotenusa. Tenemos $x = h / 2.$ Sabemos que en un triángulo la suma de los ángulos internos resulta en $180^{\circ}. Entonces:
$$$90^{\circ} + y + z = 180^{\circ} \quad \Rightarrow \quad y + z = 90^{\circ} \quad \Rightarrow \quad y = 90^{circ} -z$$
Suponga la medida del otro lado. Según el Teorema de Pitágoras, tenemos: $h^2 = (h / 2)^2 + t^2 \quad \Rightarrow \quad h^2 = h^2 / 4 + t^2 \quad \Rightarrow \quad 4h^2 = h^2 + 4t^2 \quad \Rightarrow \quad t = (h\sqrt{3}) / 2$
Suponga $t$ es lado opuesto de $y$. Entonces:
$sen (y) = ( (h\sqrt{3}) / 2) / h = \sqrt{3}/2 \quad \Rightarrow \quad y = 60^{\circ}$
Luego hallar $z$:
$y = 90^{\circ} -z \quad \Rightarrow \quad z = 90^{\circ} -60^{\circ} = 30^{\circ}$ 
Para $x$ lado opuesto de $y$, encuentre $y = 30^2$ y $z = 60^{\circ}.$\\\\

\end{enumerate}
\end{document}
