\begin{enumerate}

%--------------------A1
\item [\large\bfseries 1)] \textbf{\boldmath $\forall \;\vec{u}, \vec{v} \in V_n:\; \vec{u} + \vec{v} \in V_n$.}\\\\
    Demostración.-\; Sean $\vec{u} = (u_1,u_2,...,u_n)$ y $\vec{v} = (v_1,v_2,...,v_n)$ $\in V_n$ para $u_i ,v_i \in \mathbb{R}, \; i = 1,2,3,...,n$ entonces, $$\vec{u}+\vec{v} = (u_1+v_1,u_2+v_2,...,u_n+v_n)$$
    de donde por axioma de cerradura de los números reales se tiene $u_i+v_i \in \mathbb{R}$ y por lo tanto $\vec{u}+\vec{v} \in V_n$.\\\\

%--------------------A2
\item [\large\bfseries 2)] \textbf{\boldmath $\forall\; \vec{u},\vec{v}\in V_n\; : \; \vec{u}+\vec{v} = \vec{v}+\vec{u}.$}\\\\
    Demostración.-\; Sean $\vec{u} = (u_1,u_2,...,u_n)$ y $\vec{v} = (v_1,v_2,...,v_n)$ $\in V_n$, entonces
    $$\begin{array}{rcl}
	\vec{u}+\vec{v}&=&(u_1,u_2,...,u_n)+(v_1,v_2,...,v_n)\\
		       &=& (u_1+v_1,u_2+v_2,\ldots,u_n+v_n)\\
		       &=&(v_1+u_1,v_2+v_2,\ldots,v_n+u_n)\\
		       &=&(v_1,v_2,\ldots,v_n)+(u_1,u_2,\ldots,u_n)\\
		       &=&\vec{v}+\vec{u}.\\\\
    \end{array}$$

%--------------------A3
\item [\large\bfseries 3)] \textbf{\boldmath$\forall \; \vec{u},\vec{v},\vec{w} \in V_n:\; \vec{u} + (\vec{v}+\vec{w}) = (\vec{u} + \vec{v}) + \vec{w}$.}\\\\
    Demostración.-\; Sean $\vec{u} = (u_1,u_2,...,u_n)$, $\vec{v} = (v_1,v_2,...,v_n)$ y $\vec{w} = (w_1,w_2,...,w_n)$ entonces, 
    \begin{center}
	\begin{tabular}{rcll}
	    $\vec{u} + (\vec{v}+\vec{w})$&$=$&$(u_1,u_2,...,u_n) + \left[(v_1,v_2,...,v_n) + (w_1,w_2,...,w_n)\right]$&\\
	    &$=$&$\left[u_1+(v_1+w_1),u_2+(v_2+w_2),...,u_n+(v_n+w_n)\right]$&D2\\
	    &$=$&$\left[(u_1+v_1)+w_1,(u_2+v_2)+w_2,...,(u_n+v_n)+w_n\right]$& axioma asociativa en $\mathbb{R}$\\
	    &$=$&$(\vec{u}+\vec{v})+\vec{w}$&\\\\
	\end{tabular}
    \end{center)}

%--------------------A5
\item [\large\bfseries 5)] \textbf{\boldmath $\forall\; \vec{u} \in V_n, \;\exists! \; (-\vec{u}) \in V_n : \; \vec{u} + (-\vec{u}) = \vec{o}$}\\\\
    Demostración.-\;\\\\ \textbf{Existencia}.  Sea $\vec{u} = (u_1,u_2,...,u_n)$ entonces,
    \begin{center}
	\begin{tabular}{rcll}
	    $\vec{u}+(-\vec{u})$&$=$&$(u_1-u_1,u_2-u_2,...,u_n-u_n)$& D2 y definición de sustracción $(-1)\vec{u}$\\
	    &$=$&$(0,0,...,0)$&\\
	    &$=$&$\vec{o}$&\\
	\end{tabular}
    \end{center}

    \textbf{Unicidad}. Supongamos que $\vec{u},\vec{u}^{'} \in V_n$ tal que $\vec{u}\neq \vec{u}^{'}$ entonces $\vec{u}+(-\vec{u}) = \vec{o}\; $ y $\; \vec{u}+(-\vec{u}^{'}) = \vec{o}$ de donde \begin{center} $-\vec{u} = -\vec{u} + \vec{o}\; $ y $\; -\vec{u}^{'} = -\vec{u} + \vec{o}$,\end{center} luego por 4 tenemos \begin{center} $-\vec{u} = -\vec{u}\;$ y $\; -\vec{u} = -\vec{u}^{'}$ \end{center}
    por lo tanto se comprueba la unicidad de $-\vec{u}$.\\\\ 

\end{enumerate}
