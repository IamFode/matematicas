\begin{itemize}
    \item \textbf{\boldmath Hallar la ecuación del plano que se encuentra sobre las rectas paralelas
    $$\mathscr{L}_1: \dfrac{x-1}{2}=\dfrac{y+2}{3}=z-1$$
    $$\mathscr{L}_2: \dfrac{x+1}{2}=\dfrac{y-1}{3}=z+5.$$\\\\
Respuesta.-}\; Ya que las ecuaciones son simétricas, el vector director de $\mathscr{L}_1$ es igual a 
$$\vec{v}_{\mathscr{L}_1}=(2,3,1)$$ 
con un punto que está contenido en la recta $\mathscr{L}_1$, $$P_{\mathscr{L}_1}(1,-2,1).$$
Y el vector director de $\mathscr{L}_2$ es igual a 
$$\vec{v}_{\mathscr{L}_2}=(2,3,1)$$ 
con un punto que está contenido en la recta $\mathscr{L}_1$, $$P_{\mathscr{L}_2}(-1,1,-5).$$
\end{itemize}
Dado que $\mathscr{L}_1$ y $\mathscr{L}_2$, son paralelos, entonces podemos construir un vector que une a las rectas mediante los puntos $P_{\mathscr{L}_1}$ y $P_{\mathscr{L}_2}$, como sigue

$$\vec{v}_{\mathscr{L}_1\mathscr{L}_2}=P_{\mathscr{L}_2}(-1,1,-5)-P_{\mathscr{L}_1}(1,-2,1)=(-2,3,-6).$$\\
Luego, hallamos el vector normal, calculando el producto vectorial de $\vec{v}_{\mathscr{L}_1}$ y $\vec{v}_{\mathscr{L}_1\mathscr{L}_2}$, de la siguiente manera
$$\vec{n}=\vec{v}_{\mathscr{L}_1}\times \vec{v}_{\mathscr{L}_1\mathscr{L}_2}=\begin{bmatrix*}[c]
    i&j&k\\
    2&3&1\\
    -2&3&-6
\end{bmatrix*} = \left[3\cdot(-6)-1\cdot3\;,\;1\cdot(-2)-2\cdot(-6)\;,\;2\cdot3-3\cdot(-2)\right]=(21,10,12).$$
Por lo tanto, la ecuación del plano que pasa por las rectas paralelas $\mathscr{L}_1$ y $\mathscr{L}_2$ es
$$21(x-1)+10(y+2)+12(z-1)=0\quad \Rightarrow \quad 21x+10y+12z-13=0.$$
