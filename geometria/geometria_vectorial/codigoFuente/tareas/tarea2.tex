\begin{enumerate}

%--------------------M1
\item[\large\bfseries 1.] \textbf{\boldmath $\forall\; \vec{u} \in V_n:\;  1\cdot \vec{u} = \vec{u}$.}\\\\
    Demostración.-\; Sea $\vec{u} = (u_1,u_2,...,u_n)$, entonces $1\cdot \vec{u} = 1\cdot (u_1,u_2,...,u_n)$, luego por D3 se tiene, $$(1\cdot u_1,1\cdot u_2,...,1\cdot u_n)$$
    luego por existencia de una identidad para la multiplicación en $\mathbb{R}$ obtenemos $(u_1,u_2,...,u_n)$ de donde concluimos $$1\cdot \vec{u} = \vec{u}.$$\\

%--------------------M2
\item[\large\bfseries 2.] \textbf{\boldmath $\forall\; \vec{u} \in \mathbb{R}\; :\; 1\vec{u}=\vec{u}$.}\\\\
    Demostración.-\; Sea $\vec{u}=(u_1,u_2,\ldots , u_n)$, entonces
    $$1\vec{u}=1(u_1,u_2,\ldots , u_n)=(1u_1, 1u_2, \ldots , 1u_n) = (u_1,U_2,\ldots , u_n)=\vec{u}.$$\\

%--------------------M5
    \item[\large\bfseries 5.] \textbf{\boldmath $\forall\; \vec{u},\vec{v} \in V_n, \; \forall\; r \in \mathbb{R}:\; r(\vec{u}+\vec{v}) = r\vec{u} + r\vec{v}$.}\\\\
    Demostración.-\; 
    \begin{center}
	\begin{tabular}{rcll}
	    $r(\vec{u}+\vec{v})$&$=$&$r\left[(u_1+v_1,u_2+v_2,...,u_n+v_n)\right]$&A2\\
	    &$=$&$\left[r(u_1+v_1),r(u_2+v_2),...,r(u_n+v_n)\right]$&M3\\
	    &$=$&$(ru_1+rv_1,ru_2+rv_2,...,ru_n+rv_n)$&Axioma asociativa en $\mathbb{R}$\\
	    &$=$&$r(u_1+u_2,...,u_n)+r(v_1,v_2,...,v_n)$&D2 y D3\\
	    &$=$&$r\vec{u} + r\vec{v}$&\\
	\end{tabular}
    \end{center}
    así, la proposición queda demostrada.\\\\

\end{enumerate}
