\begin{enumerate}[\Large\bfseries 1.]

%----------1.
\item $\vec{u} \times \vec{v} = -(\vec{v}\times \vec{u})$\\\\
    Demostración.-\;  Sea $\vec{u}, \vec{v} \in V_3$, por definición se tiene, $$\vec{u}\times \vec{v} = \left(u_2 v_3 - u_3v_2, u_3v_1-u_1v_3,u_1v_2 - u_2v_1\right)$$
	Ya que $u_i,v_i \in \mathbb{R}$ para $i=1,2,...,n$ entonces obtenemos, $$\vec{u}\times \vec{v} = \left[-1(v_2u_3 - v_3u_2),-1(v_3u_1 - v_1u_3),-1(v_1u_2 - v_2u_1)  \right],$$
	luego por la multiplicación de un número real por un vector, $$\vec{u}\times \vec{v} =-1\left(v_2u_3 - v_3u_2 ,v_3u_1 - v_1u_3,v_1u_2 - v_2u_1 \right),$$
	de donde, $$\vec{u}\times \vec{v} =-\left(\vec{v}\times \vec{u}\right)$$\\\\

%----------2.
\item Refute $\vec{u}\times\left(\vec{v}\times \vec{w}\right) = \left(\vec{u}\times \vec{v}\right)\times \vec{w}$\\\\
    Respuesta.-\; Sea $\vec{u}=(1,2,-1)$, $\vec{v}=(2,-1,1)$ y $\vec{w} = (0,1,2)$ entonces el producto escalar de $\vec{u}\times\left(\vec{v}\times \vec{w}\right)$ vendrá dado por:
	$$(1,2,-1)\times\left[(2,-1,1)\times (0,1,2)\right] = (1,2,-1)\times (-3,-4,2) = (0,1,2).$$
	Por otro lado se tiene,
	$$\left(\vec{u}\times \vec{v}\right)\times \vec{w} = \left[(1,2,-1)\times (2,-1,1)\right]\times (0,1,2) = (1,-3,-5)x(0,1,2) = (-1,-2,1),$$
	de  donde $$\vec{u}\times\left(\vec{v}\times \vec{w}\right) \neq \left(\vec{u}\times \vec{v}\right)\times \vec{w},$$\\
	y por lo tanto NO se cumple la igualdad dada al principio de la proposición.\\\\

	
	

\end{enumerate}
