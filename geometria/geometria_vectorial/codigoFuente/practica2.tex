\begin{enumerate}[\bfseries \textbf{Ejercicio} 1.]

    %--------------------1.
    \item \textbf{\boldmath Hallar la ecuación vectorial de la recta que pasa por el punto con vector direccional $\vec{v}$, cuando}
	\begin{enumerate}[\bfseries a)]
	    
	    %----------a)
	    \item \textbf{\boldmath $P_0=(5,3,-2); \; \vec{v}=(2,-3,3)$.\\\\
		Respuesta.-}\; La ecuación vectorial estará dada por, $$ X = (5,3,-2) + t(2,-3,3) ;\;  t \in \mathbb{R}$$ 
		Luego la ecuación cartesiana estará dada por, 
		$$\left\{\begin{array}{rcr}
		    x&=&5+2t\\
		    y&=&3-3t\\
		    z&=&-2+3t\\
		\end{array}\right.$$\\

	    %----------b)
	    \item $P_0=(-3,2,-1); \; \vec{v}=(-2,5,1)$.\\\\
		Respuesta.-\; La ecuación vectorial es, $$X= (-1,3,4) + t(-2,5-1) ;\;  t\in \mathbb{R} $$
		se sigue,
		$$\left\{\begin{array}{rcr}
		    x&=&-3-2t\\
		    y&=&2+5t\\
		    z&=&-2-t\\
		\end{array}\right.$$\\\\

	\end{enumerate}

    %--------------------2.
    \item \textbf{Determinar la ecuación de la recta que pasa por los pares de puntos dados y proporcionar sus ecuaciones paramétricas.}

	\begin{enumerate}[\bfseries a)]
	    
	    %----------a)
	    \item \textbf{\boldmath $(8,3,2)$ y $(5,0,1)$.\\\\
		Respuesta.-}\; La ecuación de la recta viene dada por 
		$$\mathcal{L} = \lbrace (8,3,2) + t(-3,-3,-1) / t\in \mathbb{R} \rbrace$$
		Luego las ecuaciones paramétricas son, 
		$$\left\{\begin{array}{rcr}
		    x&=&8-3t\\
		    y&=&3-3t\\
		    z&=&2-t\\
		\end{array}\right.$$\\

	    %----------b)
	    \item \textbf{\boldmath $(-3,2,-1)$ y $(-2,7,-5)$\\\\
		Respuesta.-}\; La ecuación de la recta será,
		$$\mathcal{L} = \lbrace (-3,2,-1) + t(1,5,-4) / t\in \mathbb{R} \rbrace$$
		Luego las ecuaciones paramétricas son, 
		$$\left\{\begin{array}{rcr}
		    x&=&-3+t\\
		    y&=&2+5t\\
		    z&=&-1-4t\\
		\end{array}\right.$$
		\vspace{0.5cm}

	\end{enumerate}

    %--------------------3.
    \item \textbf{¿Son colineales los puntos dados?.}\\
	\begin{enumerate}[\bfseries a)]
	    
	    %----------a)
	    \item \textbf{\boldmath $(2,-3,2),(0,0,0),(3,-2,0)$\\\\
		Respuesta.-}\; Sea $\vec{v}=(0,0,0)-(2,-3,2)=(-2,3,-2)$ y $\vec{u}=(3,-2,0)-(2,-3,2)=(1,1,-2)$ de donde nos faltará comprobar que $\vec{v}=r\vec{u}$ para $r\in \mathbb{R}$.
		$$(-2,-3,-2)\neq r(1,1,-2)$$
		por lo tanto los puntos dados no son colineales.\\\\

	    %----------b)
	    \item \textbf{\boldmath $(1,2,0),(5,-7,8),(4,3,-1)$\\\\
		Respuesta.-}\; análogamente al anterior ejercicio $(4,-9,8)\neq r(3,1,-1)$ para $r\in \mathbb{R}$ y por lo tanto los puntos dados no son colineales.\\\\

	\end{enumerate}

    %--------------------4.
    \item \textbf{\boldmath Calcular la distancia del punto $P_0$ a la recta que pasa por los puntos $P_1$ y $P_2$.\\ $P_0 = (5,-3,1); \; P_1=(4,0,2);\; P_2=(5,0,0)$.\\\\
	Respuesta.-}\; Primero encontramos la recta asociada a los dos puntos dados de la siguiente forma, $$\mathscr{L} = \left\{P_1 +(P_2-P_1)t | t \in \mathbb{R}\right\} \quad \Rightarrow \quad \mathscr{L} = \lbrace (4,0,2) + t(1,0,-2) | t \in \mathbb{R} \rbrace.$$
    Luego,
    $$(5,-3,1)=(4,0,2)+t(1,0,-2),$$
    de donde,
    $$\left\{\begin{array}{rcl}
	5&=&4+1t\\
	-3&=&0+0t\\
	1&=&2-2t
    \end{array}\right.$$
    Dado que no existe un $t$ tal que $(5,-3,1)=(4,0,2)+t(1,0,-2)$, entonces $P_0$ no pertenece a $\mathscr{L}$. Por lo que podemos calcular la distancia del punto $P_0$ a la recta $\mathscr{L}$ como sigue, 
    $$\begin{array}{rcl}
	d(P_0,\mathscr{L}) &=& \bigg\| (P_0 - P_1) - \dfrac{(P_0-P_1)\circ \vec{v}}{\|\vec{v}\|^2}\cdot \vec{a}\bigg\| \\\\
			   &=& \bigg\| [(5,-3,1)-(4,0,2)] - \dfrac{\left[(5,-3,1)-(4,0,2)\right]\circ (1,0,-2)}{\|(1,0,-2)\|^2} \cdot (1,0,-2)\bigg\|
    \end{array}$$

	    de donde $$d(P_0,\mathcal{L}) = \sqrt{\dfrac{230}{25}} = 3.0331$$\\

    %--------------------5.
    \item \textbf{\boldmath Hallar la ecuación paramétrica de la recta que pasa por $(-1,2,-4)$ y que es paralela a $3i-4j+k$.\\\\
	Respuesta.-}\; Sean, el punto $P(-1,2,-4)$ y el vector $\vec{a}=3i-4j+k=(3,-4,1)$. Podemos construir la recta de la siguiente manera,
	$$\mathscr{L}=\left\{P-t\vec{a} | t\in \mathbb{R}\right\}$$

	Luego, la a ecuación paramétrica paralela a $(3i-4j+k)$ estará dada por,
	$$\left\{\begin{array}{rcrcr}
		x &=& -1 &+& 3t\\
		y &=& 2 &+& 4t\\
		z &=& -4 &+& t
	\end{array}\right.$$
	\vspace{0.5cm}

	
    %--------------------6.
    \item \textbf{\boldmath Hallar la ecuación vectorial de la recta que pasa por $(-2,0,5)$ y que es paralela a la recta $x=1+2t$, $y=4-t$, $z=6+2t$.\\\\
	Respuesta.-}\; Sea, 
	$$\left\{\begin{array}{rcrcr}
		x&=&1&+&2t\\
		y&=&4&-&t\\
		z&=&6&+&2t\\
	\end{array}\right.$$
De donde,
$$\mathscr{L}_1 = \left\{(1,4,6)+t(2,-1,2)|t\in \mathbb{R}\right\}.$$ 
Así, podemos construir otra recta paralela a $\mathscr{L}$, de la siguiente forma:
$$\mathscr{L}_2 = {(-2,0,5)+r(2,-1,2)/s\in \mathbb{R}}.$$\\\\


    %--------------------7.
    \item \textbf{\boldmath ¿Dónde interseca la recta $x=-1+2t,y=3+t,z=4-t$?}
	\begin{enumerate}[\bfseries a)]
	    \item \textbf{\boldmath al plano $xy$.}
	    \item \textbf{\boldmath al plano $xz$.}
	    \item \textbf{\boldmath al plano $yz$.}\\\\
	\end{enumerate}
	\textbf{Respuesta.-}\; Lo que debemos hallar son puntos que pasa por $xy$, $xz$ ó $yz$.\\
	Sean,
	$$
	\mathscr{L} =\left\{\begin{array}{rcr}
		x&=&-1+2t\\
		y&=&3+t\\
		z&=&4-t
	\end{array}\right.\quad t\in \mathbb{R}
	$$
	Y las ecuaciones cartesianas del plano $xy$, $xz$ y $yz$, respectivamente:
	$$ 
	\begin{array}{rrcr}
	    xy:&z&=&0\\
	    xz:&y&=&0\\
	    yz:&x&=&0
	\end{array}
	$$
	Entonces, igualando $\mathscr{L}$ en cada una de las ecuaciones anteriores, obtenemos:
	$$
	\begin{array}{rrcr}
	    xy:&4-t_1&=&0\\
	    xz:&3+t_2&=&0\\
	    yz:&-1+2t_3&=&0
	\end{array}
	\quad \Rightarrow \quad
	\begin{array}{rrcr}
	    xy:&t_1&=&4\\
	    xz:&t_2&=&-3\\
	    yz:&t_3&=&\frac{1}{2}
	\end{array}
	$$
	Reemplazando cada $t_i$ en la ecuación de $\mathscr{L}$, obtenemos:
	$$
	\begin{array}{rrcccr}
	    xy:&x&=&-1+2\cdot 4&=&7\\
	       &y&=&3+4&=&7\\
	       &z&=&4-\cdot4&=&0\\\\
	    xz:&x&=&-1+2\cdot (-3)&=&-7\\
	       &y&=&3-3&=&0\\
	    &z&=&4-(-3)=7\\\\
	    yz:&x&=&-1+2\cdot \frac{1}{2}&=&0\\
	       &y&=&3+\frac{1}{2}&=&3.5\\
	       &z&=&4-\frac{1}{2}&=&3.5
	\end{array}
	$$
	Por lo tanto, la recta $\mathscr{L}$ interseca al plano $xy$ en el punto $(7,7,0)$, al plano $xz$ en el punto $(-7,0,7)$ y al plano $yz$ en el punto $(0,3.5,3.5)$.\\\\

    %--------------------8.
    \item \textbf{\boldmath Encontrar las ecuaciones paramétricas de la recta que pasa por $(x_1, y_1, z_1)$ y que es paralela a la recta $x = x_0 + at, y = y_0 + bt, z = z_0 + ct$.\\\\
	    Respuesta.-}\; Primero encontramos el vector director de la recta,
	    $$\vec{v}=(a,b,c).$$
	    luego, definimos el vector normal de la siguiente manera,
	    $$\vec{n}=(n_1,n_2,n_3)$$
	    De donde podemos calcular su producto escalar, doescalarnde nos indica que la normal del plano donde pasa el punto $(x_1,y_2,z_1)$ es perpendicular a la recta dada.
	    $$\vec{n}\cdot \vec{v}=0 \quad \Rightarrow \quad n_1x_1+n_2y_1+n_3z_1=0$$
	    Ahora, fijamos dos valores arbitrarios para 
	


    %--------------------22.
    \item \textbf{\boldmath Encontrar la ecuación del plano que pasa por $(-1,4-2)$ y que contiene a la recta de intersección de los planos.
    $$4x-y+z-2=0\qquad \mbox{y}\qquad 2x+y-2z-3=0.$$
	Respuesta.-}\; La ecuación del plano que pasa por la línea de intersección de los planos: $4x-y+z-2=0$ y $2x+y-2z-3=0$ está dada por:
	$$4x-y+z-2+\lambda(2x+y-2z-3)=0$$
	Dado que el plano anterior pasa por el punto dado $(-1,4,2)$, satisfará la ecuación del plano $4x-y+z-2+\lambda(2x+y-2z-3)=0$ de la siguiente manera
	$$4(-1)-4+2-2+\lambda(2(-1)+4-2\cdot 2-3)=0 \quad \Rightarrow \quad \lambda =-\dfrac{8}{5}.$$
	Luego sustituyendo $\lambda$ en la ecuación del plano $4x-y+z-2+\lambda(2x+y-2z-3)=0$ se concluye que:
	$$4x-y+z-2-\dfrac{8}{5}(2x+y-2z-3)=0 \quad \Rightarrow \quad 4x-13y+21z+14=0.$$\\

    %--------------------23.
    \item \textbf{\boldmath Demuestre que las rectas $x-1=\dfrac{y+1}{2}=z$ y $x-2=\dfrac{y-2}{3}=\dfrac{z-4}{2}$ se intersecan. Determine una ecuación del (único) plano que las contiene.\\\\
	Demostración.-}\; Sean, $\mathscr{L}_1 : x-1=\dfrac{y+1}{2}=z\;$ y $\;\mathscr{L}:x-2=\dfrac{y-2}{3}=\dfrac{z-4}{2}$. Convirtiendo es su forma parametrica se tiene:
	$$
	\mathscr{L}_1 : \left\{\begin{array}{rcl}
		x &=& 1+t\\
		y &=& -1+2t\\
		z &=& t
	\end{array}\right.;t\in \mathbb{R}\qquad \mbox{y}\qquad
	\mathscr{L}_2 : \left\{\begin{array}{rcl}
		x &=& 2+\lambda\\
		y &=& 2+3\lambda\\
		z &=& 4+2\lambda
	\end{array}\right.;\lambda\in \mathbb{R}
	$$
	Para el punto de intersercción, igualamos las rectas $\mathscr{L}_1$ y $\mathscr{L}_2$, como sigue:
	$$
	\begin{array}{rcl}
		1+t &=& 2+\lambda\\
		-1+2t &=& 2+3\lambda\\
		t &=& 4+2\lambda
	\end{array}
	\quad \Rightarrow \quad 
	1+(4+2\lambda)=2+\lambda
	\quad \Rightarrow \quad 
	\lambda = -3 \; \mbox{ y } \; t = -2
	$$
	Luego, reemplazamos $\lambda$ y $t$ en la ecuación de $\mathscr{L}_1$,
	$$
	\left\{\begin{array}{rcl}
		x &=& 1+(-2)\\
		y &=& -1+2(-2)\\
		z &=& (-2)
	\end{array}\right.
	$$
	Así el punto de intersección entre $\mathscr{L}_1$ y $\mathscr{L}_2$, estará dado por:
	$$P_I(-1,-5,-2).$$
	Ahora, hallemos las normales de cada recta. Para ello, primero tomamos los escalares que multiplican al vector de cada recta
	$$\vec{v}_1 = (1,2,1)\quad \mbox{y}\quad \vec{v}_2 = (1,3,2).$$
	Luego calculemos el producto vectorial de cada uno de los vectores anteriores:
	$$
	\vec{v}_1\times \vec{v}_2 = \left|\begin{array}{rcl}
		i & j & k\\
		1 & 2 & 1\\
		1 & 3 & 2
	\end{array}\right| = 1i+(-2)j+(-1)k = (1,-1,1)=\vec{n}
	$$
	Así, la ecuación del plano estará dado por el vector normal y el punto de la recta $\mathscr{L}_1$.
	$$\begin{array}{rcl}
	    (1,-1,1)\circ(x-1,y+1,z)=0 &\Rightarrow& x-1-y-1+z=0\\
				       &\Rightarrow& x-y+z=2
	\end{array}$$
	\vspace{0.5cm}

    %--------------------24.
    \item \textbf{\boldmath Demuestre que la recta de intersección de los planos $x + 2y - z = 2$ y $3x + 2y + 2z = 7$ es paralela a la recta $x = 1 + 6t$, $y = 3 - 5t$, $z = 2 - 4t$. Determine una ecuación del plano determinado por estas dos rectas.\\\\
	Demostración.-}\; Sean,
	$$
	\left\{\begin{array}{rrcl}
		\mathscr{P}_1:&x+2y-z &=& 2\\
		\mathscr{P}_2:&3x+2y+2z &=& 7
	\end{array}\right. \qquad \mbox{y}\qquad
	\mathscr{L}_1:\left\{\begin{array}{rcl}
		x &=& 1+6t\\
		y &=& 3-5t\\
		z &=& 2-4t
	\end{array}\right.
	$$
	Primero, parametricemos las ecuaciones de los planos $\mathscr{P}_1$ y $\mathscr{P}_2$, poniendo a $x=\lambda$,
	$$
	\begin{array}{rcr} 
		\lambda+2y-z &=& 2\\
		3\lambda+2y+2z &=& 7\\
		\hline
		-2\lambda + 0 - 3z &=& -5
	    \end{array} \quad \Rightarrow \quad z=\dfrac{5-2\lambda}{3}.
	$$
	Luego reemplazos $z$ en el plano de $\mathscr{P}_1$, 
	$$x+2y-z=2\quad \Rightarrow \quad x+2y-\dfrac{5-2\lambda}{3}=2\quad \Rightarrow \quad y=\dfrac{11}{6}-\dfrac{5}{6}\lambda.$$
	De donde, la ecuación paramétrica de la recta $\mathscr{L}_1$ que se interseca en $\mathscr{P}_1$ y $\mathscr{P}_2$ estará dada por:
	$$
	\mathscr{L}_2:\left\{\begin{array}{rcl}
		x &=& \lambda\\\\
		y &=& \dfrac{11}{6}-\dfrac{5}{6}\lambda\\\\
		z &=& \dfrac{5}{3}-\dfrac{2}{3}\lambda
	\end{array}\right. \; \mbox{para}\; \lambda\in \mathbb{R}
	$$
	Para determinar la ecuación del plano que pase por las dos rectas, es suficiente un punto de cada recta $\mathscr{L}_1$ y $\mathscr{L}_2$.
	$$P_1(1,3,2)\quad \mbox{y}\quad P_2(0,\frac{11}{6},\dfrac{5}{3})$$
	Luego, hallemos su vector
	$$\vec{P_1 P_2}=P_1-P-2=\left(1,\dfrac{7}{6},\dfrac{1}{3}\right)=(6,7,2).$$
	Dado que $\vec{v}$ de la recta $\mathscr{L}_1$ es $(6,-5,-4)$, entonces
	$$
	\vec{n}=\vec{P_1 P_2}\times \vec{v} =
	\left[\begin{array}{rrr}	
		i & j & k\\
		6 & 7 & 2\\
		6 & -5 & -4
	\end{array}\right]
	\quad \Rightarrow \quad \vec{n}=(-3,4,-8)
	$$
	Por lo tanto, la ecuación del plano que pasa por las dos rectas $\mathscr{L}_1$ y $\mathscr{L}_2$ estará dada por:
	$$\begin{array}{rcl}
	   \mathscr{P}: (-3,4,-8)\circ(x,y,z)=0 &\Rightarrow& -3x+3+4y-12-8z+16=0\\\\
				       &\Rightarrow& 3x-4y+8z=7
	\end{array}$$
	\vspace{0.5cm}



    %-------------------- 26.
    \item \textbf{\boldmath Demostrar que la distancia $D$ entre los planos paralelos $ax+by+cz+d=d_1$ y $ax+by+cz+d=d_2$ está dada por:
    $$D=\dfrac{|d_2-d_1|}{\sqrt{a^2+b^2+c^2}}.$$\\
	Demostración.-}\; Sabemos que  para $\mathscr{P}:ax+by+cz=d$ con $P_0=(x_0,y_0)$, la distancia está dada por:
	$$d(P_0,\mathscr{P})\dfrac{|ax_0,by_0+cz_0-d|}{\sqrt{a^2+b^2+c^2}}.$$
	Primero, hallemos un punto $P_0$ del plano $ax+by+cz+d=d_1$.
	$$\left\{\begin{array}{rcl}
	    x&=&0\\
	    y&=&0\\
    \end{array}\right.\quad \Rightarrow \quad z=\dfrac{d_1-d}{c}.$$
    Así $P_0=\left(0,0,\dfrac{d_1-d}{c}\right)$
    Por último, hallemos $D=d(P_0,ax+by+cz+d=d_2)$ de la siguiente manera,
    $$\begin{array}{rcl}
	D&=&\dfrac{|a\cdot 0+b\cdot 0 +c\left(\frac{d_1-d}{c}\right)-d_2+d|}{\sqrt{a^2+b^2+c^2}}\\\\
	 &=&\dfrac{|d_1-d-d_2+d}{\sqrt{a^2+b^2+c^2}|}\\\\
	 &=&\dfrac{|d_1-d_2|}{\sqrt{a^2+b^2+c^2}}.\\\\
    \end{array}$$

    \end{enumerate}
