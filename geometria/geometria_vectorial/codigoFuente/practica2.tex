\begin{enumerate}[\bfseries \textbf{Ejercicio} 1.]

    %--------------------1.
    \item \textbf{\boldmath Hallar la ecuación vectorial de la recta que pasa por el punto con vector direccional $\vec{v}$, cuando}
	\begin{enumerate}[\bfseries a)]
	    
	    %----------a)
	    \item \textbf{\boldmath $P_0=(5,3,-2); \; \vec{v}=(2,-3,3)$.\\\\
		Respuesta.-}\; La ecuación vectorial estará dada por, $$ X = (5,3,-2) + t(2,-3,3) ;\;  t \in \mathbb{R}$$ 
		Luego la ecuación cartesiana estará dada por, 
		$$\left\{\begin{array}{rcr}
		    x&=&5+2t\\
		    y&=&3-3t\\
		    z&=&-2+3t\\
		\end{array}\right.$$\\

	    %----------b)
	    \item $P_0=(-3,2,-1); \; \vec{v}=(-2,5,1)$.\\\\
		Respuesta.-\; La ecuación vectorial es, $$X= (-1,3,4) + t(-2,5-1) ;\;  t\in \mathbb{R} $$
		se sigue,
		$$\left\{\begin{array}{rcr}
		    x&=&-3-2t\\
		    y&=&2+5t\\
		    z&=&-2-t\\
		\end{array}\right.$$\\\\

	\end{enumerate}

    %--------------------2.
    \item Determinar la ecuación de la recta que pasa por los pares de puntos dados y proporcionar sus ecuaciones paramétricas.

	\begin{enumerate}[\bfseries a)]
	    
	    %----------a)
	    \item $(8,3,2)$ y $(5,0,1)$.\\\\
		Respuesta.-\; La ecuación de la recta viene dada por 
		$$\mathcal{L} = \lbrace (8,3,2) + t(-3,-3,-1) / t\in \mathbb{R} \rbrace$$
		Luego las ecuaciones paramétricas son, 
		$$\left\{\begin{array}{rcr}
		    x&=&8-3t\\
		    y&=&3-3t\\
		    z&=&2-t\\
		\end{array}\right.$$\\

	    %----------b)
	    \item $(-3,2,-1)$ y $(-2,7,-5)$\\\\
		Respuesta.-\; La ecuación de la recta será,
		$$\mathcal{L} = \lbrace (-3,2,-1) + t(1,5,-4) / t\in \mathbb{R} \rbrace$$
		Luego las ecuaciones paramétricas son, 
		$$\left\{\begin{array}{rcr}
		    x&=&-3+t\\
		    y&=&2+5t\\
		    z&=&-1-4t\\
		\end{array}\right.$$
		\vspace{0.5cm}

	\end{enumerate}

    %--------------------3.
    \item ¿Son colineales los puntos dados?.\\\\

	\begin{enumerate}[\bfseries a)]
	    
	    %----------a)
	    \item $(2,-3,2),(0,0,0),(3,-2,0)$\\\\
		Respuesta.-\; Sea $\vec{v}=(0,0,0)-(2,-3,2)=(-2,3,-2)$ y $\vec{u}=(3,-2,0)-(2,-3,2)=(1,1,-2)$ de donde nos faltará comprobar que $\vec{v}=r\vec{u}$ para $r\in \mathbb{R}$.
		$$(-2,-3,-2)\neq r(1,1,-2)$$
		por lo tanto los puntos dados no son colineales.\\\\

	    %----------b)
	    \item $(1,2,0),(5,-7,8),(4,3,-1)$\\\\
		Respuesta.-\; análogamente al anterior ejercicio $(4,-9,8)\neq r(3,1,-1)$ para $r\in \mathbb{R}$ y por lo tanto los puntos dados no son colineales.\\\\

	\end{enumerate}

    %--------------------4.
    \item Calcular la distancia del punto $P_0$ a la recta que pasa por los puntos $P_1$ y $P_2$.\\ $P_0 = (5,-3,1); \; P_1=(4,0,2);\; P_2=(5,0,0)$.\\\\
    Respuesta.-\; Primero encontramos la recta asociada a los dos puntos dados de la siguiente forma, $$\mathscr{L} = \left\{P_1 +(P_2-P_1)t | t \in \mathbb{R}\right\} \quad \Rightarrow \quad \mathscr{L} = \lbrace (4,0,2) + t(1,0,-2) | t \in \mathbb{R} \rbrace.$$
    Luego,
    $$(5,-3,1)=(4,0,2)+t(1,0,-2),$$
    de donde,
    $$\left\{\begin{array}{rcl}
	5&=&4+1t\\
	-3&=&0+0t\\
	1&=&2-2t
    \end{array}\right.$$
    Dado que no existe un $t$ tal que $(5,-3,1)=(4,0,2)+t(1,0,-2)$, entonces $P_0$ no pertenece a $\mathscr{L}$. Por lo que podemos calcular la distancia del punto $P_0$ a la recta $\mathscr{L}$ como sigue, 
    $$\begin{array}{rcl}
	d(P_0,\mathscr{L}) &=& \bigg\| (P_0 - P_1) - \dfrac{(P_0-P_1)\circ \vec{v}}{\|\vec{v}\|^2}\cdot \vec{a}\bigg\| \\\\
			   &=& \bigg\| [(5,-3,1)-(4,0,2)] - \dfrac{\left[(5,-3,1)-(4,0,2)\right]\circ (1,0,-2)}{\|(1,0,-2)\|^2} \cdot (1,0,-2)\bigg\|
    \end{array}$$

	    de donde $$d(P_0,\mathcal{L}) = \sqrt{\dfrac{230}{25}} = 3.0331$$\\

    %--------------------5.
    \item Hallar la ecuación paramétrica de la recta que pasa por $(-1,2,-4)$ y que es paralela a $3i-4j+k$.\\\\
	Respuesta.-\; Sean, el punto $P(-1,2,-4)$ y el vector $\vec{a}=3i-4j+k=(3,-4,1)$. Podemos construir la recta de la siguiente manera,
	$$\mathscr{L}=\left\{P-t\vec{a} | t\in \mathbb{R}\right\}$$

	Luego, la a ecuación paramétrica paralela a $(3i-4j+k)$ estará dada por,
	$$\left\{\begin{array}{rcrcr}
		x &=& -1 &+& 3t\\
		y &=& 2 &+& 4t\\
		z &=& -4 &+& t
	\end{array}\right.$$
	\vspace{0.5cm}

	
    %--------------------6.
    \item Hallar la ecuación vectorial de la recta que pasa por $(-2,0,5)$ y que es paralela a la recta $x=1+2t$, $y=4-t$, $z=6+2t$.\\\\
	Respuesta.-\; Sea, 
	$$\left\{\begin{array}{rcrcr}
		x&=&1&+&2t\\
		y&=&4&-&t\\
		z&=&6&+&2t\\
	\end{array}\right.$$
De donde,
$$\mathscr{L}_1 = \left\{(1,4,6)+t(2,-1,2)|t\in \mathbb{R}\right\}.$$ 
Así, podemos construir otra recta paralela a $\mathscr{L}$, de la siguiente forma:
$$\mathscr{L}_2 = {(-2,0,5)+r(2,-1,2)/s\in \mathbb{R}}.$$\\\\


    %--------------------7.
    \item \textbf{\boldmath ¿Dónde interseca la recta $x=-1+2t,y=3+t,z=4-t$?}
	\begin{enumerate}[\bfseries a)]
	    \item \textbf{\boldmath al plano $xy$.}
	    \item \textbf{\boldmath al plano $xz$.}
	    \item \textbf{\boldmath al plano $yz$.}\\\\
	\end{enumerate}
	\textbf{Respuesta.-}\; 



\end{enumerate}
