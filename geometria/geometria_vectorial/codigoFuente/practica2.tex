\begin{enumerate}[\bfseries \textbf{Ejercicio} 1.]

    %--------------------1.
    \item Hallar la ecuación vectorial de la recta que pasa por el punto con vector direccional $\vec{v}$, cuando 
	\begin{enumerate}[\bfseries a)]
	    
	    %----------a)
	    \item $P_0=(5,3,-2); \; \vec{v}=(2,-3,3)$.\\\\
		Respuesta.-\; La ecuación vectorial estará dada por, $$ X = (5,3,-2) + t(2,-3,3) ;\;  t \in \mathbb{R}$$ 
		Luego la ecuación cartesiana estará dada por, 
		$$\left\{\begin{array}{rcr}
		    x&=&5+2t\\
		    y&=&3-3t\\
		    z&=&-2+3t\\
		\end{array}\right.$$\\

	    %----------b)
	    \item $P_0=(-3,2,-1); \; \vec{v}=(-2,5,1)$.\\\\
		Respuesta.-\; La ecuación vectorial es, $$X= (-1,3,4) + t(-2,5-1) ;\;  t\in \mathbb{R} $$
		se sigue,
		$$\left\{\begin{array}{rcr}
		    x&=&-3-2t\\
		    y&=&2+5t\\
		    z&=&-2-t\\
		\end{array}\right.$$\\\\

	\end{enumerate}

    %--------------------2.
    \item Determinar la ecuación de la recta que pasa por los pares de puntos dados y proporcionar sus ecuaciones paramétricas.

	\begin{enumerate}[\bfseries a)]
	    
	    %----------a)
	    \item $(8,3,2)$ y $(5,0,1)$.\\\\
		Respuesta.-\; La ecuación de la recta viene dada por 
		$$\mathcal{L} = \lbrace (8,3,2) + t(-3,-3,-1) / t\in \mathbb{R} \rbrace$$
		Luego las ecuaciones paramétricas son, 
		$$\left\{\begin{array}{rcr}
		    x&=&8-3t\\
		    y&=&3-3t\\
		    z&=&2-t\\
		\end{array}\right.$$\\

	    %----------b)
	    \item $(-3,2,-1)$ y $(-2,7,-5)$\\\\
		Respuesta.-\; La ecuación de la recta será,
		$$\mathcal{L} = \lbrace (-3,2,-1) + t(1,5,-4) / t\in \mathbb{R} \rbrace$$
		Luego las ecuaciones paramétricas son, 
		$$\left\{\begin{array}{rcr}
		    x&=&-3+t\\
		    y&=&2+5t\\
		    z&=&-1-4t\\
		\end{array}\right.$$
		\vspace{0.5cm}

	\end{enumerate}

    %--------------------3.
    \item ¿Son colineales los puntos dados?.\\\\

	\begin{enumerate}[\bfseries a)]
	    
	    %----------a)
	    \item $(2,-3,2),(0,0,0),(3,-2,0)$\\\\
		Respuesta.-\; Sea $\vec{v}=(0,0,0)-(2,-3,2)=(-2,3,-2)$ y $\vec{u}=(3,-2,0)-(2,-3,2)=(1,1,-2)$ de donde nos faltará comprobar que $\vec{v}=r\vec{u}$ para $r\in \mathbb{R}$.
		$$(-2,-3,-2)\neq r(1,1,-2)$$
		por lo tanto los puntos dados no son colineales.\\\\

	    %----------b)
	    \item $(1,2,0),(5,-7,8),(4,3,-1)$\\\\
		Respuesta.-\; análogamente al anterior ejercicio $(4,-9,8)\neq r(3,1,-1)$ para $r\in \mathbb{R}$ y por lo tanto los puntos dados no son colineales.\\\\

	\end{enumerate}

    %--------------------4.
    \item Calcular la distancia del punto $P_0$ a la recta que pasa por los puntos $P_1$ y $P_2$.\\ $P_0 = (5,-3,1); \; P_1=(4,0,2);\; P_2=(5,0,0)$.\\\\
	Respuesta.-\; Primero encontramos la recta asociada a los dos puntos dados de la siguiente forma, $$\mathcal{L} = \lbrace (4,0,2) + t(1,0,-2) / t \in \mathbb{R} \rbrace.$$
	Luego calculamos la distancia del punto a la recta como sigue, 
	$$d(P_0,\mathcal{L}) = \bigg| (P_0 - P_1) - \dfrac{(P_0-P_1)\circ \vec{v}}{\vec{v}^2}\bigg| = \bigg| [(5,-3,1)-(4,0,2)] - \dfrac{\left[(5,-3,1)-(4,0,2)\right]\circ (1,0,-2)}{|(1,0,-2)|^2} \cdot (1,0,-2)\bigg|$$
	    de donde $$d(P_0,\mathcal{L}) = \sqrt{\dfrac{300}{25}} = 3\mbox{.}0331$$\\

    %--------------------5.
    \item Hallar la ecuación paramétrica de la recta que pasa por $(-1,2-4)$ y que es paralela a $3i+4j+k$.\\\\
	Respuesta.-\; Ya que la recta que pasa por $(-1,2,-4$ es paralela a $3i+4j+k$ entonces,  $$\mathcal{L}={(-1,2,-4)+t(3,4,1)/t\in\mathbb{R}}$$\\

	
    %--------------------6.
    \item Hallar la ecuación vectorial de la recta que pasa por $(-2,0,5)$ y que es paralela a la recta $x=1+2t$, $y=4-t$, $z=6+2t$.\\\\
	Respuesta.-\; Sea $\left\{\begin{array}{rcr}
		x&=&1+2t\\
		y&=&4-t\\
		z&=&6+2t\\
	\end{array}\right.$
Intuitivamente tenemos que $$\mathcal{L}_1 = {(1,4,6)+t(2,-1,2)/t\in \mathbb{R}}$$, luego como $L_1 || L_2$ entonces $$\mathcal{L}_2 = {(-2,0,5)+r(2,-1,2)/s\in \mathbb{R}}$$\\\\


    %--------------------7.
    \item 


\end{enumerate}
