
\begin{def.}
    Dado un plano $\Pi$ se puede establecer un sistema de coordenadas formado por dos ejes perpendiculares con origen común.\\
    Si $OX$, $OY$ son los ejes, denotamos por $OXY$ al sistema de coordenadas
    \begin{center}
	\begin{tikzpicture}
	    \draw[->] (-1,0) -- (2,0);
	    \draw[->] (0,-1) -- (0,2);
	\end{tikzpicture}
    \end{center}
\end{def.}

\begin{def.}
    Si tenemos un sistema de coordenadas $OXY$ sobre el plano $\Pi$, se establece una correspondencia biunívoca entre $\pi$ y $\mathbb{R}^2$, de la siguiente manera
    \begin{enumerate}[\bfseries (1)]
	\item $A\; P\in OX \longmapsto (x,0)\in \mathbb{R}^2$
	\item $A\quad P\in OY \longmapsto (0,y)\in \mathbb{R}^2$ donde $y$ es la coordenada de $P$ sobre $OY$
	\item Al punto $P\in \Pi,\; P\notin OX,\; P\notin OY$ le corresponde el par $(x,y)\in \mathbb{R}^2$. Donde $P_1 \longmapsto (x,0)$ y $P_2 \longmapsto (0,y)$.\\
	    \begin{itemize}
		\item A $P_1$ se le llama la proyección de $P$ sobre $OX$
		\item A $P_2$ se le llama la proyección de $P$ sobre $OY$
	    \end{itemize}
    \end{enumerate}
\end{def.}

Así podemos sobreponer el plano $\Pi$ con $\mathbb{R}^2$\\\\

\textbf{OBS.-} Para simplificar la expresión tenemos $P=(x,y)$.\\\\

\textbf{Objetivo:} Tenemos $\Pi,\; OXY \longleftrightarrow \mathbb{R}^2$
\begin{itemize}
    \item Estudiamos $\Pi$ a partir de $\mathbb{R}^2$.
    \item Estudiamos $\mathbb{R}^2$ a partir de $\Pi$.
\end{itemize}

\begin{def.}[Ángulos]
    Dado el sistema de $OXY$ sobre $\Pi$ destacamos $\longvec{OX}$
\end{def.}


\chapter{Segmentos en el plano}

Para $t\in [0,1]$, $x_t = (1-t)a+tb$
