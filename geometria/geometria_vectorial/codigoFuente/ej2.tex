\begin{enumerate}[\large \bfseries 3)]
\item $\|\vec{u}\|=0 \Leftrightarrow \vec{u} = \vec{o}$\\\\
    Demostración.-\; Se podría demostrar intuitivamente como sigue,
    $$\|\vec{u}\|=\sqrt{0^2+0^2+...+0^2} = (0,0,...,0) = \vec{u} \quad \vec{u}\in V_n$$ 
    pero demostraremos  estableciendo la definición de producto escalar.
    Si $\vec{u}$ y $\vec{v}$ son dos vectores de $V_n$ entonces $$\vec{u}\cdot \vec{v} = \sum_{k=1}^n u_k v_k$$
    Sea $\vec{u}=\vec{o}$ entonces $$\vec{u}\cdot \vec{u} = \sum\limits_{k=1}^n u_k^2 = 0^2+0^2 + ... +0^2 = 0 $$
   se sigue, $$\vec{u}\cdot\vec{u}=0\; \mbox{si} \; \vec{u}=0$$\\
   Luego podemos definir la longitud de un vector de la siguiente forma, $$\|\vec{u}\| = (\vec{u}\cdot\vec{u})^{1/2}$$ de donde concluimos que, $$\|\vec{u}\|=0.$$\\\\
   Por otro lado sabemos que $\|\vec{u}\|=0$ de donde, $$\|\vec{u}\| = (\vec{o}\cdot\vec{o})^{1/2} =  \sqrt{0^2 + 0^2 + ... + 0^2}$$
   así, por el teorema de Pitágoras se tiene, $$\mbox{Norma de} \; (\vec{u})^2 = 0^2+ 0^2 = 0$$
   y en consecuencia, $$\vec{u}=0$$\\
   
\end{enumerate}
