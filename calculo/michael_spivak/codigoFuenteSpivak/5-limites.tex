\chapter{Limites}

%--------------------definición de límite
\begin{tcolorbox}[colframe=white]
    \begin{def.}
	La función $f$ tiende hacia el límite $l$ en $a$ $\left(\lim\limits_{x \to a} f(x) = l \right)$ significa: para todo $\epsilon > 0$ existe algún $\delta > 0$ tal que, para todo $x$, si $0<|x-a|<\delta$, entonces $|f(x)-l|<\epsilon$.\\\\
	Existe algún $\epsilon >0$ tal que para todo $\delta > 0$ existe algún $x$ para el cual es $0<|x-a|<\delta$, pero no $|f(x)-l|<\epsilon$.
    \end{def.}
\end{tcolorbox}
\vspace{.5cm}

%--------------------teorema 1.1
\begin{teo}
Una función no puede tender hacia dos límites diferentes en $a$. En otros términos si $f$ tiende hacia $l$ en $a$, y $f$ tiende hacia $m$ en $a$, entonces $l=m$.\\\\
    Demostración.-\; Puesto que $f$ tiende hacia $l$ en $a$, sabemos que para todo $\epsilon>0$ existe algún número $\delta_1 >0$ tal que, para todo $x$, si $0<|x-a|<\delta_1$, entonces $|f(x)-l|<\epsilon$.\\
    Sabemos también, puesto que $f$ tiende hacia $m$ en $a$, que existe algún $\delta_2 >0$ tal que, para todo $x$, si $0<|x-a|<\delta_2$, entonces $|f(x)-m|<\epsilon$.\\
    Hemos empleado dos números $delta_1$ y $\delta_2$, ya que no podemos asegurar que el $\delta$ que va bien en una definición irá bien en la otra. Sin embargo, de hecho, es ahora fácil concluir que para todo $\epsilon>0$ existe algún $\delta >0$ tal que, para todo $x$, $$si \; 0<|x-a|<\delta=\min(\delta_1,\delta_2), \; entonces \; |f(x)-l|<\epsilon \;\; y \;\; |f(x)-m|<\epsilon$$  
    Para completar la demostración solamente nos queda tomar un $\epsilon>0$ particular para el cual las dos condiciones $|f(x)-l|<\epsilon$ y $|f(x)-m|<\epsilon$ no puedan cumplirse a la vez si $l\neq m$\\
    Si $l\neq m$, de modo que $|m-l|>0$ podemos tomar como $\epsilon$ a $|l-m|/2$. Se sigue que existe un  $\delta > 0$ tal que, para todo $x$, $$si \; 0<|x-a|<\delta, \; entonces \; |f(x)-l|<\dfrac{|l-m|}{2} \; \; y \; \; |f(x)-m|<\dfrac{|l-m|}{2}$$
    Esto implica que para $0<|x-a|<\delta$ tenemos $$|l-m| = |l - f(x) + f(x) -m|\leq |l-f(x)| + |f(x)-m|<\dfrac{|l-m|}{2}+\dfrac{|l-m|}{2}=|l-m|$$ El cual es una contradicción.\\\\
\end{teo}
\vspace{.5cm}

    %-------------------lema1.1
    \begin{lema}
	Si $x$ está cerca de $x_0$ e $y$ está cerca de $y_0$, entonces $x+y$ estará cerca de $x_0 + y_0$, $xy$ estará cerca de $x_0y_0$ y $1/y$ estará cerca de $1/y_0$.\\\\
	\begin{enumerate}[\bfseries (1)]

	    %----------(1)
	    \item Si $|x-x_0|<\dfrac{\epsilon}{2}$ y $|y-y_0|<\dfrac{\epsilon}{2}$ \; entonces \; $|(x+y) - (x_0+y_0)|<\epsilon$.\\\\
		Demostración.-\;$$|(x+y)-(x_0+y_0)| = |(x-x_0)+(y-y_0)| \leq |x-x_0| + |y-y_0| < \dfrac{\epsilon}{2} + \dfrac{\epsilon}{2} = \epsilon$$\\\\

	    %----------(2)
	    \item Si $|x-x_0| < \min\left(1,\dfrac{\epsilon}{2(|y_0|+1)}\right)$ \; y\; $|y-y_0| < \dfrac{\epsilon}{2(|x_0| + 1)}$\; entonces \;$|xy-x_0y_0|<\epsilon$.\\\\
		Demostración.-\; Puesto que $|x-x_0|<1$ se tiene $$|x| - |x_0| \leq |x-x_0|<1,$$
		de modo que $$|x| < 1 + |x_0|$$ así pues 
		\begin{center}
		    \begin{tabular}{rcl}
			$|xy-x_0y_0|$&$=$&$|x(y-y_0) + y_0(x-x_0)|$\\\\
			&$\leq$&$|x|\cdot |y-y_0| + |y_0|\cdot |x-x_0|$\\\\
			&$<$&$(1+|x_0|)\cdot \dfrac{\epsilon}{2(|x_0|+1)} + |y_0|\cdot \dfrac{\epsilon}{2(|y_0|+1)}$\\\\
			&$<$&$\dfrac{\epsilon}{2} + \dfrac{\epsilon}{2} = \epsilon$\\\\
		    \end{tabular}
		\end{center}
		Notemos que $\dfrac{|y_0|}{|y_0|-1}<1$, por lo tanto $\dfrac{|y_0|}{|y_0|-1}\cdot \dfrac{\epsilon}{2}< \dfrac{\epsilon}{2}$.\\\\
		\vspace{.5cm}

	    %----------(3)
	    \item Si $y_0 \neq 0$ y $|y-y_0| < \min\left(\dfrac{|y_0|}{2},\dfrac{\epsilon|y_0|^2}{2}\right)$ \; entonces \; $y\neq 0$ y $\left| \dfrac{1}{y} - \dfrac{1}{y_0} \right|< \epsilon$.\\\\
		Demostración.-\; Se tiene $$|y_0|-|y|<|y-y_0|<\dfrac{|y_0|}{2},$$
		de modo que $-|y|< -\dfrac{|y_0|}{2} \Longrightarrow |y|>|y_0|/2$. En particular. $y\neq 0$, y $$\dfrac{1}{|y|}<\dfrac{2}{|y_0|}$$ Así pues 
		$$\left|\dfrac{1}{y}-\dfrac{1}{y_0}\right| = \dfrac{|y_0-y|}{|y|\cdot |y_0|} = \dfrac{1}{|y|}\cdot \dfrac{|y_0-y|}{|y_0|} < \dfrac{2}{|y_0|}\cdot \dfrac{1}{|y_0|} \cdot \dfrac{\epsilon |y_0|^2}{2} = \epsilon$$\\\\
	\end{enumerate}
    \end{lema}
\vspace{1cm}

%--------------------teorema 2
\begin{teo}
    Si $\lim\limits_{x \to a} f(x) = l$ y $\lim\limits_{x \to a} g(x) = m$, entonces 
    \begin{enumerate}[\bfseries (1)]
	\item  $\lim\limits_{x \to a} (f+g)(x) = l+m$
	\item  $\lim\limits_{x \to a} (f\cdot g)(x) = l\cdot m$\\\\
	Además, si $m\neq 0$, entonces
	\item  $\lim\limits_{x \to a} (\dfrac{1}{g})(x) = \dfrac{1}{m}$
    \end{enumerate}
	Demostración.-\; La hipótesis significa que para todo $\epsilon > 0$ existen $\delta_1, \delta_2>0$ tales que, para todo $x$, $$si \; 0<|x-a|<\delta_1, \; entonces \; |f(x)-l|<\epsilon$$ 
	$$y \quad si \; 0<|x-a|<\delta_2, \; entonces \; |g(x)-m < \epsilon|$$
	Esto significa ( ya que después de todo, $\epsilon/2$ es también un número positivo) que existen $\delta_1,\delta_2>0$ tales que, para todo $x$,
	$$si \; 0<|x-a|<\delta_1, \; entonces \; |f(x)-l|<\dfrac{\epsilon}{2}$$
	$$y \quad si \; 0<|x-a|<\delta_2, \; entonces \; |g(x)-m|<\dfrac{\epsilon}{2}$$
	Sea ahora $\delta = \min(\delta_1,\delta_2)$. Si $0<|x-a|<\delta$, entonces $0<|x-a|<\delta_1$ y $0<|x-a|<\delta_2$ se cumplen las dos, de modo que es a la vez $$|f(x)-l|<\dfrac{\epsilon}{2} \quad y \quad |g(x)-m|<\dfrac{\epsilon}{2}$$
	pero según la parte $(1)$ del lema anterior esto implica que $|(f+g)(x) - (l+m)|< \epsilon$.\\\\
	Para demostrar $(2)$ procedemos de la misma manera, después de consultar la parte $(2)$ del lema. Si $\epsilon>0$ existen $\delta_1,\delta_2>0$ tales que, para todo $x$ $$si\; 0<|x-a|<\delta_1, \; entonces \; |f(x)-l|<\min\left(1,\dfrac{\epsilon}{2(|m|+1)}\right),$$ $$y\quad si\; 0<|x-a|<\delta_2,\; entonces \; |g(x)-m|<\dfrac{\epsilon}{2(|l|)+1}$$ Pongamos de nuevo $\delta = \min(\delta_1,\delta_2)$. Si $0<|x-a|<\delta,$ entonces $$|f(x)-l|<\min\left(1,\dfrac{\epsilon}{2(|m|+1)}\right) \qquad y \qquad |g(x)-m|<\dfrac{\delta}{2(|l|+1)}$$
	Así pues, según el lema, $|(f\cdot g)(x)-l\cdot m|<\epsilon$, y esto demuestra $(2)$.\\\\
	Finalmente, si $\epsilon>0$ existe un $\delta>0$ tal que, para todo $x$, $$si\; 0<|x-a|<\delta, \; entonces \; |g(x)-m|< \min\left(\dfrac{|m|}{2},\dfrac{\epsilon|m|^2}{2}\right)$$
	Pero según la parte $(3)$ del lema, esto significa, en primer lugar que $g(x)\neq 0$, de modo que $(1/g)(x)$ tiene sentido, y en segundo lugar que $$\left|\left(\dfrac{1}{g}\right)(x)-\dfrac{1}{m}\right|<\epsilon$$
	Esto demuestra $(3)$.\\\\
\end{teo}

%--------------------definición 5.2
\begin{tcolorbox}[colframe=white]
    \begin{def.}
	$\lim\limits_{x \to a^+} f(x)=l$ significa que para todo $\epsilon>0$, existe un $\delta>0$ tal que, para todo $x$, $$si \; 0<x-a<\delta, \; entonces \; |f(x)-l|<\epsilon$$	
	La condición $0<x-a<\delta$ es equivalente a $0<|x-a|<\delta$ y $x>a$
    \end{def.}
\end{tcolorbox}

%--------------------definición 5.3
\begin{tcolorbox}[colframe=white]
    \begin{def.}
	$\lim\limits_{x \to a^-} f(x)=l$ significa que para todo $\epsilon>0$, existe un $\delta>0$ tal que, para todo $x$, $$si \; 0<a-x<\delta, \; entonces \; |f(x)-l|<\epsilon$$	
    \end{def.}
\end{tcolorbox}

%--------------------definición 5.4
\begin{tcolorbox}[colframe=white]
    \begin{def.}
	$\lim\limits_{x \to \infty} f(x)=l$ significa que para todo $\epsilon>0$, existe un número $N$ grande, que, para todo $x$, $$si \; x>N, \; entonces \; |f(x)-l|<\epsilon$$	
    \end{def.}
\end{tcolorbox}

\section{Problemas}
\begin{enumerate}[\Large \bfseries 1.]

%--------------------1.
\item Hallar los siguientes limites (Estos limites se obtienen todos, después de algunos cálculos, de las distintas partes del teorema 2; téngase cuidado en averiguar cuáles son las partes que se aplican, pero sin preocuparse de escribirlas.)
\begin{enumerate}[\bfseries (i)]

    %----------(i)
    \item $\lim\limits_{x \to 1}\dfrac{x^2-1}{x+1} = \dfrac{1^2 - 1}{1 + 1} = \dfrac{0}{2} = 0$\\\\

    %----------(ii)
    \item $\lim\limits_{x \to 2} \dfrac{x^3 - 8}{x - 2} = \dfrac{(x-2)(x^2+2x+4)}{x-2} = 2^2+4+4 = 12$\\\\

    %----------(iii)
    \item $\lim\limits_{x \to 3} \dfrac{x^3-8}{x-2} = \dfrac{3^3-8}{3-2} =19$\\\\

    %----------(iv)
    \item $\lim\limits_{x\to y}\dfrac{x^n - y^n}{x-y} = \dfrac{(x-y)(x^{n-1}+x^{n-2}y + ... + xy^{n-2}+y^{n-1})}{x-y} = x^{n-1}+x^{n-2}y + ... + xy^{n-2}+y^{n-1}=$ $$= ny^{n-1}$$

    %----------(v)
    \item $\lim\limits_{y\to x}\dfrac{x^n - y^n}{x-y} = x^{n-1}+x^{n-2}y + ... + xy^{n-2}+y^{n-1}= nx^{n-1}$\\\\

    %----------(vi)
    \item $\lim\limits_{h\to 0}  = \dfrac{\sqrt{a+h}-\sqrt{a}}{h} = \dfrac{\sqrt{a+h}+\sqrt{a}}{h}\cdot \dfrac{\sqrt{a+h}+\sqrt{a}}{\sqrt{a+h}+\sqrt{a}} = \dfrac{(\sqrt{a+h})^2 - (\sqrt{a})^2}{h(\sqrt{a+h} + \sqrt{a})}=\dfrac{1}{\sqrt{a+h}+\sqrt{a}} = $ $$=\dfrac{1}{2\sqrt{a}}$$\\\\
    \vspace{1cm}

\end{enumerate}

%--------------------2.
\item Hallar los límites siguientes:
\begin{enumerate}[\bfseries (i)]
    
    %----------(i)
    \item $\lim\limits_{x \to 1} \dfrac{1-\sqrt{x}}{1-x} =\lim\limits_{x \to 1} \dfrac{1-\sqrt{x}}{1-x}\cdot \dfrac{1+\sqrt{x}}{1+\sqrt{x}} =\lim\limits_{x \to 1} \dfrac{1^2 - (\sqrt{x})^2}{(1-x)(1+\sqrt{x})} =\lim\limits_{x \to 1} \dfrac{1}{1+\sqrt{x}} = \dfrac{1}{2}$\\\\
    
    %----------(iii)
    \item $\lim\limits_{x \to 0} \dfrac{1-\sqrt{1-x^2}}{x} =\lim\limits_{x \to 0} \dfrac{1-\sqrt{1-x^2}}{x} \cdot \dfrac{1+\sqrt{1-x^2}}{1+\sqrt{1-x^2}} =\lim\limits_{x \to 0} \dfrac{x}{1+\sqrt{1-x^2}} = 0$\\\\
    
    %----------(iii)
    \item $\lim\limits_{x \to 0} \dfrac{1-\sqrt{1-x^2}}{x^2} = \lim\limits_{x \to 0} \dfrac{1-\sqrt{1-x^2}}{x^2}\cdot \dfrac{1+\sqrt{1-x^2}}{1+\sqrt{1-x^2}} = \lim\limits_{x \to 0} \dfrac{1}{1+\sqrt{1+x^2}} = \dfrac{1}{2}$\\\\

\end{enumerate}

%--------------------3.
\item En cada uno de los siguientes casos, encontrar un $\delta$ tal que, $|f(x)-l|<\epsilon$ para todo $x$ que satisface $0<|x-a|<\delta$\\
\begin{enumerate}[\bfseries (i)]

    %----------(i).
    \item $f(x)=x^4; \; l=a^4$\\\\
	Respuesta.-\; Por la parte $(2)$ del lema anterior se tiene $$|x^2-a^2| < \min\left(1,\dfrac{\epsilon}{2(|a|^2+1)}\right).$$ Si aplicamos una vez mas la parte $(2)$ del lema obtenemos $$|x-a|<\min\left(1,\dfrac{\min\left(1,\dfrac{\epsilon}{2(|a|^2+1)}\right)}{2(|a|+1)}\right)=\min \left(1,\dfrac{\epsilon}{4(|a|^2 + 1)(|a|+1)}\right)=\delta$$\\

    %----------(ii).
    \item $f(x) =\dfrac{1}{x}; \; a=1,\; l=1$\\\\
	Respuesta.-\; Por la parte $(3)$ del lema se tiene $\left|\dfrac{1}{x} - 1\right|<\epsilon$ por lo tanto $|y-1|< \min\left(\dfrac{1}{2},\dfrac{\epsilon}{2}\right)$\\\\

    %----------(iii).
    \item $f(x)=x^4 + \dfrac{1}{x}; \; a=1, \; l=2$\\\\
	Respuesta.-\; Por la primera parte del lema se tiene $\left|\left(x^4+\dfrac{1}{x}\right)-(1+1)\right|<\epsilon$ de donde $$|x^4-1|<\dfrac{\epsilon}{2} \quad y \quad \left|\dfrac{1}{x}-1\right|<\dfrac{\epsilon}{2}$$
	Luego por el inciso $(i)$ y $(ii)$ $$\left|x-1\right|<\min\left(\dfrac{1}{2},\dfrac{\dfrac{\epsilon}{2}}{2}\right) \;\; y \;\; \left|x-1\right|<\min\left(1,\dfrac{1,\min\left(\dfrac{\dfrac{\epsilon}{2}}{2(1+1)}\right)}{2(1+1)}\right) \; \Longrightarrow \;|x-1|<\min\left(\dfrac{1}{2},\dfrac{\epsilon}{4},1,\dfrac{\epsilon}{32}\right)$$
	y por lo tanto $$|x-1|<\min\left(\dfrac{1}{2},\dfrac{\epsilon}{32}\right)=\delta$$\\


    %----------(iv).
    \item $f(x)=\dfrac{x}{1 + \sen^2 x}; \; a=0, \; l=0$\\\\
	Respuesta.-\; Sea $\left|\dfrac{x}{1+\sen^2 x}\right|<\epsilon \quad y \quad  |x|<\delta$ pero $\left|\dfrac{x}{1+\sen^2 x}\right|\leq |x|$ por lo tanto \\ $\left|\dfrac{x}{1+\sen^2 x}\right|\leq |x|<\delta=\epsilon$\\\\

    %----------(v).
    \item $f(x)=\sqrt{|x|};\; a=0,\; l=0$\\\\
	Respuesta.-\; Sea $\left|\sqrt{|x|}\right|<\epsilon$ entonces $\left|(|x|)^{1/2}\right| = \left(\sqrt{x^2}\right)^{1/2} = \left[(x^2)^{1/2}\right]^{1/2}=\sqrt{x}<\epsilon$, luego sabemos que la raíz cuadrada de $x$ debe ser siempre mayor o igual a $0$ por lo tanto $|x|<\epsilon^2$, de donde concluimos que $\delta=\epsilon^2$\\\\

    %----------(vi).
    \item $f(x)=\sqrt{x};\; a=1,\; l=1$\\\\
	Respuesta.-\; Si $\epsilon > 1$,  póngase $\delta = 1$. Entonces $|x-1|<\delta$ implica que $0<x<2$ con lo que $0<\sqrt{x}<2$ y $|\sqrt{x}-1|<1$. Si $\epsilon < 1$, entonces $(1-\epsilon)^2<x<(1+\epsilon)^2$ implica que $|\sqrt{x}-1|<\epsilon$, de modo que podemos elegir  un $\delta$ tal que $(1-\epsilon)^2 \leq 1 - \delta$ y $1+\delta\leq (1-\epsilon)^2$. Podemos elegir, pues $\delta = 2\epsilon - \epsilon^2$\\\\

\end{enumerate}

%--------------------4.
\item Para cada una de las funciones del problema $4-17$, decir para qué números $a$ existe el límite $\lim\limits_{x\to a}f(x)$

\begin{enumerate}[\bfseries (i)]
    
    %----------(i)
    \item Existe el límite si $a$ no es un entero, ya que en los puntos enteros la función tiene un salto.\\\\
    
    %----------(ii)
    \item Existe el límite si $a$ no es un entero.\\\\
    
    %----------(iii)
    \item De la misma forma que el inciso $(ii)$.\\\\
    
    %----------(iv)
    \item Existe para todo $a$.\\\\
    
    %----------(v)
    \item Existe para todo $a$ si sólo si sea $a=0$ y $a=\dfrac{1}{n}, \; n \in \mathbb{Z}, n\neq 0$.\\\\
    
    %----------(vi)
    \item El límite no existe para los puntos $|a| < 1 \; y \; a\neq \dfrac{1}{n}$\\\\

\end{enumerate}

%--------------------5.
\item 
\begin{enumerate}[\bfseries (a)]
    
    %----------(a)
    \item Hágase lo mismo para cada una de las funciones del problema $4-19$
    \begin{enumerate}[\bfseries (i)]
	
	%----------(i).
	\item Existe para cualquier número que tenga la forma $n+\dfrac{k}{10}, \; n,k\in \mathbb{Z}$\\\\
	
	%----------(ii).
	\item Existe para cualquier número que tenga la forma $n + \dfrac{k}{100},\; n,k \in \mathbb{Z}$\\\\
	
	%----------(iii).
	\item No es posible para ningún $a$.\\\\
	
	%----------(iv).
	\item De la misma forma que el anterior inciso.\\\\
	
	%----------(v).
	\item Existe para todo $a$ excepto para los que terminan en $7999...$\\\\
	
	%----------(vi).
	\item Existe para todo $a$ excepto para los que terminan en $1999...$.\\\\

    \end{enumerate}

    %----------(b)
    \item El mismo problema usando decimales infinitos que terminen en una fila de ceros en lugar de los que terminan en una fila de nueves.
    \begin{enumerate}[\bfseries (i)]
	
	%----------(i).
	\item De igual forma de la parte $(a)$ inciso $(i)$.\\\\
	
	%----------(ii).
	\item De igual forma de la parte $(a)$ inciso $(ii)$.\\\\
	
	%----------(iii).
	\item De igual forma de la parte $(a)$ inciso $(iii)$.\\\\
	
	%----------(iv).
	\item De igual forma de la parte $(a)$ inciso $(iii)$.\\\\
	
	%----------(v).
	\item Existe para todo $a$ excepto para los que terminan en $8000...$\\\\
	
	%----------(vi).
	\item Existe para todo $a$ excepto para los que terminan en $2000...$\\\\

    \end{enumerate}

\end{enumerate}

%--------------------6.
\item Supóngase que las funciones $f$ y $g$ tiene n la siguiente propiedad: Para todo $\epsilon>0$ y todo $x$, $$si \; 0<|x-2|<\sen^2\left(\dfrac{\epsilon^2}{9}\right) + \epsilon, \; entonces \; |f(x)-2|<\epsilon,$$ $$si \; 0<|x-2|<\epsilon^2, \; entonces \; |g(x)-4| < \epsilon.$$\\
Para cada $\epsilon>0$ hallar un $\delta>0$ tal que, para todo $x$,\\
\begin{enumerate}[\bfseries (i)]
    
    %----------(i)
    \item Si $0<|x-2|<\delta$, entonces $|f(x)+g(x)-6|<\epsilon$.\\\\
	Respuesta.-\; Por la primera parte del lema se tiene $|f(x)-2|<\dfrac{\epsilon}{2}$ y $|g(x)-4|<\dfrac{\epsilon}{2}$, luego remplazamos $\epsilon$ por $\epsilon/2$ de donde nos queda $$0<|x-2|<\sen^2 \left[\dfrac{\left(\dfrac{\epsilon}{2}\right)^2}{9}\right] \quad y \quad |x-2|<\left(\dfrac{\epsilon}{2}\right)^2$$
	Por último, solo hace verificar para todo $\epsilon>0$ existe  algún $\delta>0$. En este caso solo hace falta elegir $$0<|x-2|<\min\left[\sen^2 \left(\dfrac{\epsilon^2}{36}\right)+\epsilon,\dfrac{\epsilon^2}{4}\right] = \delta$$\\

    %----------(ii)
    \item Si $0<|x-2|<\delta$, entonces $|f(x)g(x) - 8|<\epsilon$\\\\\
	Respuesta.-\; Por la segunda parte del lema demostrado tenemos que $$|f(x)-2|<\min\left(1,\dfrac{\epsilon}{2(|4|+1)}\right) \quad y \quad |g(x)-4|<\dfrac{\epsilon}{2(|2|+1)}$$ ya que $|f(x)g(x)-2\cdot 4|<\epsilon$.\\ 
	Luego reemplazando en $\epsilon$ a cada parte obteniendo, 
	$$0<|x-2|<\min\left\{\sen^2 \left[\dfrac{\min\left(\dfrac{\epsilon}{10}\right)^2}{9}\right] + \min\left(1,\dfrac{\epsilon}{10}\right),\left[\min\left(1,\dfrac{\epsilon}{6}\right)\right]^2\right\}=\delta$$\\\\

    %----------(iii)
    \item Si $0<|x-2|<\delta$, entonces $\left|\dfrac{1}{g(x)}-\dfrac{1}{4}\right|<\epsilon$\\\\
	Respuesta.-\; Por la tercera parte del lema se tiene que $|g(x) - 4|<\min\left(\dfrac{|4|}{2},\dfrac{\epsilon|4|^2}{2}\right)$, luego remplazando en $\epsilon$ obtenemos $$|x-2|<\left[\min\left(2,8\epsilon \right)\right]^2 = \delta$$.\\

    %----------(iv)
    \item Si $0<|x-2|<\delta$, entonces $\left|\dfrac{f(x)}{g(x)}-\dfrac{1}{2}\right|<\delta$\\\\
	Respuesta.-\; Sea $\left|f(x)\dfrac{1}{g(x)}-2dfrac{1}{4}\right|$ entonces $$|f(x)-2|<\min\left(1,\dfrac{\epsilon}{2(|1/4|+1)}\right) \quad y \quad \dfrac{1}{g(x)} - \dfrac{1}{4}<\dfrac{\epsilon}{2(|2|+1)}$$, de donde $$0<|x-2|\min\left\{\sen^2\left[\dfrac{\left(\min(1,2\epsilon/5)\right)^2}{9}\right] + \min(1,2\epsilon/5),\left[\min\left(2,\dfrac{8\epsilon}{2(|2|+1)}\right)\right]^2\right\}=\delta$$\\\\ 

\end{enumerate}

%--------------------7.
\item 

\end{enumerate}
