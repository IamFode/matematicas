\chapter{Funciones}


    %definición 1.1
\begin{tcolorbox}[colframe=white]
    \begin{def.}
	El conjunto de los números a los cuales se aplica una función recibe el nombre de \textbf{dominio} de la función.
    \end{def.}
\end{tcolorbox}

    %definición 1.2
\begin{tcolorbox}[colframe=white]
    \begin{def.}
	Si $f$ \; y \; $g$ son dos funciones cualesquiera, podemos definir una nueva función $f+g$ denominada \textbf{suma} de $f+g$ mediante la ecuación:
	$$(f+g)(x)=f(x)+g(x)$$
	Para el conjunto de todos los $x$ que están a la vez en el dominio de $f$ y en el dominio de $g$, es decir: $$dominio \; (f+g)=dominio \; f \; \cap \; dominio \; g$$\\
    \end{def.}
\end{tcolorbox}

    %definición 1.3
\begin{tcolorbox}[colframe=white]
    \begin{def.}
	El dominio de $f \cdot g$ es $dominio \; f \cap \; dominio \; g$ $$(f \cdot g)(x)=f(x)\cdot g(x)$$\\
    \end{def.}
\end{tcolorbox}

    %definición 1.4 
\begin{tcolorbox}[colframe=white]
    \begin{def.}
	Se expresa por dominio $f$ $\cap$ dominio $g$ $\cap$ $\lbrace x:g(x)\neq 0 \rbrace$
	$$\left( \dfrac{f}{g}\right) (x)=\dfrac{f(x)}{g(x)}$$ \\
    \end{def.}
\end{tcolorbox}

%definición 1.5
\begin{tcolorbox}[colframe=white]
    \begin{def.}[Función constante]
	$$(c \cdot g)(x)=c \cdot g(x)$$\\
    \end{def.}
\end{tcolorbox}

%teorema 1.1
\begin{teo}
    $(f+g)+h=f+(g+h)$\\\\
    Demostración.- \; \textbf{La demostración es característica de casi todas las demostraciones que prueban que dos funciones son iguales: se debe hacer ver que las dos funciones tienen el mismo dominio y el mismo valor para cualquier número del dominio.} Obsérvese que al interpretar la definición de cada lado se obtiene:
    \begin{center}
	\begin{tabular}{r c l}
	    $\left[ (f+g) + h \right](x)$&=&$(f+g)(x)+h(x)$\\
	    &=&$\left[ f(x) +g(x) \right] +h(x)$\\\\
	    &y&\\\\
	    $\left[ f+(g+h) \right](x)$&=&$f(x)+(g+h)(x)$\\
	    &=&$f(x)+\left[ g(x)+h(x) \right]$\\\\
	\end{tabular}
    \end{center}
    Es esta demostración no se ha mencionado la igualdad de los dos dominios porque esta igualdad parece obvia desde el momento en que empezamos a escribir estas ecuaciones: el dominio de $(f+g)+h$ y el de $f+(g+h)$ es evidentemente dominio $f$ $\cap$ dominio $g$ $\cap$ dominio $h$. Nosotros escribimos, naturalmente $f+g+h$ por $(f+g)+h=f+(g+h)$\\\\
\end{teo}

%teorema 1.2
\begin{teo}
    Es igual fácil demostrar que $(f\cdot g)\cdot g=f\cdot (g \cdot h)$ y ésta función se designa por $f \cdot g \cdot h$. Las ecuaciones $f+g=g+f$ \; y \; $f\cdot g=g \cdot f$ no deben presentar ninguna dificultad.\\\\
\end{teo}

%definición 1.6
\begin{tcolorbox}[colframe=white]
    \begin{def.}[Composición de función]
	$$(f \circ g)(x)=f(g(x))$$
	El dominio de $f\circ g$ es $\lbrace $ $x$ : $x$ está en el dominio de $g$ \: y \; $g(x)$ está en el dominio de $f$ $\rbrace$
	$$D_{f \circ g}= \lbrace x \; / \; x \in D_g \; \land \; g(x)\in D_f \rbrace$$
    \end{def.}
\end{tcolorbox}

    %propiedad 1.1
\begin{tcolorbox}[colframe=white]
    \begin{prop}
    $(f \circ g) \circ h = f \circ (g \circ h)$   La demostración es una trivalidad.
    \end{prop}
\end{tcolorbox}

%definición  1.7 
\begin{tcolorbox}
    \begin{def.} 
	Una \textbf{función} es una colección de pares de números con la siguiente propiedad: Si $(a,b)$ \; y \; $(a,c)$ pertenecen ambos a la colección, entonces $b=c$; en otras palabras, la colección no debe contener dos pares distintos con el mismo primer elemento.\\
    \end{def.}
\end{tcolorbox}

%definición 1.8
\begin{tcolorbox}
    \begin{def.} 
	Si $f$ es una función, el \textbf{dominio} de $f$ es el conjunto de todos los $a$ para los que existe algún $b$ tal que $(a,b)$ está en $f$. Si $a$ está en el dominio de $f$, se sigue de la definición de función que existe, en efecto, un número $b$ único tal que $(a,b)$ está en $f$. Este $b$ único se designa por $f(a)$.\\  
    \end{def.}
\end{tcolorbox}


\section{Problemas}

    \begin{enumerate}[\Large \bfseries 1.]

	%--------------------1.
	\item Sea $f(x)=1/(1+x)$. Interpretar lo siguiente:
	    \begin{enumerate}[\bfseries (i)]

		%----------(i)
		\item $f(f(x))$ (¿Para que $x$ tiene sentido?)\\\\
		Respuesta.- \; Sea $f\left( \dfrac{1}{1+x} \right)$ entonces $\dfrac{1}{1 + \dfrac{1}{1+x}}$, por lo tanto $\dfrac{1-x}{x+2}$ de donde llegamos a la conclusión de que $x$ se cumple para todo número real de $1$ y $-2$\\\\

		%----------(ii)
		\item $f\left( \dfrac{1}{x} \right)$\\\\
		Respuesta.- \; $\dfrac{1}{1 + \dfrac{1}{x}}=\dfrac{1}{\dfrac{x+1}{x}}=\dfrac{x}{x+1}$ por lo tanto se cumple para todo $x\neq -1, 0$\\\\

		%----------(iii)
		\item $f(cx)$\\\\
		Respuesta.- \; $\dfrac{1}{1+cx}$ donde se cumple para todo $x\neq -1$ si $c\neq 0$\\\\

		%----------(iv)
		\item $f(x+y)$\\\\
		Respuesta.- \; $\dfrac{1}{1+x+y}$ donde se cumple para todo $x+y\neq-1$\\\\

		%----------(v)
		\item $f(x) + f(y)$\\\\
		Respuesta.- \; $\dfrac{1}{1+x} + \dfrac{1}{1+y}=\dfrac{x+y+2}{(1+x)(1+y)}$ siempre y cuando $x\neq-1$ y $y \neq -1$\\\\

		%----------(vi)
		\item ¿Para que números $c$ existe un número $x$ tal que $f(cx)=f(x)$?\\\\
		Respuesta.- \; Para todo $c$ ya que $f(c\cdot 0)=f(0)$\\\\

		%----------(vii)
		\item ¿Para que números $c$ se cumple que $f(cx)=f(x)$ para dos números distintos $x$?\\\\
		Respuesta.- \;  Solamente $c=1$ ya que $f(x)=f(cx)$ implica que $x=cx$, y esto debe cumplirse por lo menos para un $x \geq 1$\\\\

	    \end{enumerate}

	%--------------------2.  
	\item Sea $g(x)=x^2$ y sea 
	\begin{equation*}
	    h(x) = \left\lbrace
		\begin{array}{rl}
		    0, & x \; racional\\
		    1, & x \; irracional
		\end{array}
	    \right.
	\end{equation*}

	\begin{enumerate}[\bfseries (i)]

	    %----------(i)
	    \item ¿Para cuáles $y$ es $h(y) \leq y$?\\\\
	    Respuesta-. \; Se cumple para $y\geq 0$ si $y$ es racional, o para todo $y\geq 1$\\\\

	    %----------(ii)
	    \item ¿Para cuáles $y$ es $h(y) \leq g(y)$?\\\\
	    Respuesta-. \; Para $-1\leq y \leq 1$ siempre que $y$ sea racional y para todo $y$ tal que $|y|\leq1$\\\\

	    %----------(iii)
	    \item ¿Qué es $g(h(z)) - h(z)$?\\\\
	    Respuesta-. \; 
	    \begin{equation*}
		g(h(z)) = \left\lbrace
		    \begin{array}{rl}
			0, & z^2 \; racional\\
			1, & z^2 \; irracional
		    \end{array}
		\right.
	    \end{equation*}

	    Por lo tanto el resultado es $0$\\\\

	    %----------(iv)
	    \item ¿Para cuáles $w$ es $g(w)\leq w $?\\\\
	    Respuesta-. \; Para todo $w$ tal que $0\leq w\leq 1$\\\\

	    %----------(v)
	    \item ¿Para cuáles $\epsilon$ es $g(g(\epsilon)) = g(\epsilon)$?\\\\
	    Respuesta-. \; Para $-1,0,1$\\\\ 

	\end{enumerate}

	%--------------------3.
	\item Encontrar el dominio de las funciones definidas por las siguientes fórmulas:
	    \begin{enumerate}[\bfseries (i)]

	    %----------(i)
	    \item $f(x)=\sqrt{1-x^2}$\\\\
	    Respuesta.- \; Por la propiedad de raíz cuadrada, se tiene  $1-x^2 \geq 0$ entonces $x^2 \leq 1$ por lo tanto el dominio son todos los $x$ tal que $|x| \leq 1$\\\\

	    %-----------(ii)
	    \item $f(x)=\sqrt{1-\sqrt{1-x^2}}$\\\\
	    Respuesta.- \; Se observa claramente que el dominio es $-1\leq x \leq 1$\\\\

	    %-----------(iii)
	    \item $f(x)=\dfrac{1}{x-1} + \dfrac{1}{x-2}$\\\\
	    Respuesta.- \; Operando un poco tenemos $$f(x) = \dfrac{2x-3}{(x-1)(x-2)},$$ sabemos que el denominador no puede ser $0$ por lo tanto el $D_{f} = \lbrace x\; / \; x \neq 1, \; x\neq  2 \rbrace$\\\\ 
    
	    %-----------(iv)
	    \item $f(x)=\sqrt{1-x^2} + \sqrt{x^2-1}$\\\\
	    Respuesta.- \; Claramente notamos que el dominio de $f$ son $-1$ y $1$ ya que si se toma otros números daría un número imaginario.\\\\

	    %-----------(v)
	    \item $f(x)=\sqrt{1-x}+\sqrt{x-2}$\\\\
	    Respuesta.- \; Notamos que no se cumple para ningún $x$ ya que si $0\leq x \leq 1$ entonces no se cumple para $\sqrt{x-2} $ y si $x\geq 2$ no se cumple para $\sqrt{1-x}$\\\\ 

	    \end{enumerate}

	%--------------------4.
	\item Sean $S(x)=x^2,$ $P(x)=2^x$ y $s(x)=sen x$. Determinar los siguientes valores. En cada caso la solución debe ser un número.\\\\

	\begin{enumerate}[\bfseries (i)]

	    %----------(i)
	    \item $(S \circ P)(y)$\\\\
	    Respuesta.- \; Por definición se tiene que $(S \circ P)(y)=S(P(y))$ entonces $S(2^y)=2^{2y}$ siempre y cuando $D_{S \circ P}=\lbrace y / y \in D_P \land P(y) \in D_S\rbrace$\\\\
	    
	    %----------(ii)
	    \item $(S \circ s)(y)$\\\\
	    Respuesta.- \; Por definición tenemos que $(S \circ s)(y)=S(s(y))$ entonces $S(\sen y)=\sen^2 y$ siempre y cuando $D_{S \circ s}=\lbrace y / y \in D_s \land S(y) \in D_S \rbrace$\\\\

	    %----------(iii)
	    \item $(S \circ P \circ s)(t)+(s \circ P)(t)$\\\\
	    Respuesta.- \; $(S \circ P \circ s)(t)+(s \circ P)(t) = S((P \circ s)(t))+s(P(t)) = S(P(s(t))) + s(P(t))=S(P(\sen t)) + s(2^t)=S(2^{\sen t}) +  \sen 2^t = 2^{2 \sen t} +\sen 2^t$\\\\  

	    %----------(iv)
	    \item $s(t^3)$\\\\
	    Respuesta.- \; $s(t^3)=\sen t^3$\\\\

	\end{enumerate}

	    %--------------------5.
	\item Expresar cada una de las siguientes funciones en términos de $S,P,s$ usando solamente $+,\cdot , \circ$\\\\
	\begin{enumerate}[\bfseries (i)]
	    
	    %----------(i)
	    \item $f(x)=2^{\sen x}$\\\\
	    Respuesta.- \; Claramente vemos que $P \circ s$\\\\ 

	    %----------(ii)
	    \item $f(x) = \sen 2^x$\\\\
	    Respuesta.- \; $s \circ P$\\\\

	    %----------(iii)
	    \item $f(x) = \sen x^2$\\\\
	    Respuesta.- \; $s \circ S$\\\\

	    %----------(iv)
	    \item $f(x) = \sen x$\\\\
	    Respuesta.- \; $S \circ s$\\\\

	    %----------(v)
	    \item $f(t) = 2^{2t}$\\\\
	    Respuesta.- \; $P \circ P$\\\\

	    %----------(vi)
	    \item $f(u)=\sen (2^u + 2^{u^2})$\\\\
	    Respuesta.- \; $s \circ (P + P \circ S)$\\\\

	    %----------(vii)
	    \item $f(y) = \sen (\sen (\sen (2^{2^{2^{\sen y}}})))$\\\\
	    Respuesta.- \; $s \circ s \circ s \circ P \circ P \circ P \circ s$\\\\

	    %----------(viii)
	    \item $f(a)= 2^{\sen^2 a} +  \sen(a^2) + 2^{sen(a^2 + \sen a)}$\\\\
	    Respuesta.- \; $P \circ S \circ s  + s \circ S + P \circ s \circ (S + s)$\\\\

	\end{enumerate}

	%--------------------6.
	\item 
	\begin{enumerate}[\bfseries (a)]

	    %----------(a)
	    \item Si $x_1, ... , x_n$ son números distintos, encontrar una función polinómica $f_i$ de grado $n-1$ que tome el valor $1$ en $x_i$ y $0$ en $x_j$ para $j \neq i.$ Indicación: El producto de todos los $(x-x_j)$ para $j \neq i$ es $0$ en $x_j$ si $j \neq i.$ Este producto es designado generalmente por $$\prod\limits_{j=1_{j \neq i}}^n (x-x_j)$$ donde el símbolo $\prod$ (pi mayúscula) desempeña para productos el mismo papel que $\sum$ para sumas.\\\\

	Respuesta.- \; Una forma de pensar sobre esta pregunta es considerar una solución fija $n$ y elegir un conjunto de distintas $x_1, x_2, ..., x_n$. Por ejemplo supongamos que elegimos $n=3$ $x_1=1$, $x_2=2$, $x_3 = 3.$ Entonces supongamos que queremos encontrar un polinomio $f_i(x_1)=f_1(1)=1,$ pero $f_1(x_2)=f_1(2)=f_1(3)=0.$ Es decir, $F_1$ es un cuadrático que tiene ceros en $x=2$ \; y\; $x=3$, pero es igual a $1$ en $x=1.$ Naturalmente, esto sugiere mirar un polinomio de la forma $$a(x-2)(x-3),$$ para que la igualdad sea igual a $1$ por alguna constante $a.$ Pero, ¿Qué es esta constante? Bueno, si nos conectamos con $x=1$, debemos tener $$f_1(1)=1=a(x-2)(x-3)=2a,$$ por lo tanto $a=1/2$ y la solución deseada es $$f_1(x)=\dfrac{1}{2}(x-2)(x-3).$$ Del mismo modo, si tratamos de encontrar un polinomio $f_2(x)$ tal que $f_2(2)=1$ con raíces en $x=1,3$ tendríamos que resolver la ecuación $1=a(2-1)(2-3),$ lo que da $a=-1$ por lo tanto $f_2(x)=-(x-1)(x-3)$\\
	    Ahora veamos el caso general. El polinomio $f_i(x)$ satisface $f_i(x_i)$ \; y \; $f_i(x_j)=0$ para todo $j \neq i$, entonces debe tomar la forma 
	    \[ f_i(x)=a \prod_{j \neq i} (x-x_j)  \]
	    Para alguna constante $a$. Para encontrar esta constante, aplicamos $x=x_1$:
	    \[ f_i(x_i)=1=a\prod_{j \neq i} (x_i-x_j), \]
	    por lo tanto:
	    \[ a= \dfrac{1}{\displaystyle\prod_{j \neq i} (x_i-x_j)}  \]  
	    Así queda
	    \[ f_i(x)= \prod_{j \neq i} \dfrac{(x-x_j)}{(x_i-x_j)} \]\\\\

	    %----------(b)
	    \item Encontrar ahora una función polinómica de grado $n-1$ tal que $f(x_1)=a_1,$ donde $a_1,...,a_n$ son números dados. (Utilícense las Funciones $f_1$ de la parte $(a)$.) La fórmula que se obtenga es la llamada \textbf{Fórmula de interpolación de Lagrange}\\\\

	    Respuesta.- \;Sea \[ f(x) = \sum_{j=1} a_i f_i(x) \] entonces  \[ f(x) = \sum_{j=1} a_i \prod_{j \neq i} \dfrac{(x-x_j)}{(x_i-x_j)} \]\\\\ 

	\end{enumerate}

	%--------------------7.
	\item
	\begin{enumerate}[\bfseries (a)]

	    %----------(a)
	    \item Demostrar que para cualquier función polinómica $f$ y cualquier número $a$ existe función polinómica $g$ y un número $b$ tales que $f(x)=(x-a)g(x)+b$ para todo $x$. (La idea es esencialmente dividir $f(x)$ por $(x-a)$ mediante la división larga hasta encontrar un resto constante.)\\
	    Demostración.- \; Si el grado de $f$ es $1$, entonces $f$ es de la forma $$f(x)=cx+d=cx+d +ac-ac=c(x-a)+(d+ac)$$ de tal modo que $g(x)=c$ y $b=d+ac$. Por inducción supongamos que el resultado es válido para polinomios de grado $\leq k.$ Si $f$ tiene grado $k+1$, entonces $f$ tiene la forma $$f(x)=a_{k+1}x^{k+1} + ... + a_1x + a_0$$ luego para grados $\leq k$ se tiene $$f(x)-a_{k+1}x^{k+1} =(x-a)g(x)+b$$ así $$f(x) = (x-a)\left[g(x) + a_{k+1}(x-a)^k\right]+b$$\\\\

	    %----------(b)
	    \item Demostrar que si $f(a)=0$, entonces $f(x)=(x-a)g(x)$ para alguna función polinómica $g$. (La reciproca es evidente)\\\\
	    Demostración.- \; Por la parte $(a)$, podemos poner que $f(x)=(x-a)g(x)+b$, entonces $$0=f(a)=(a-a)g(a)+b=b$$ de modo que $f(x)=(x-a)g(x)$\\\\

	    %----------(c)
	    \item Demostrar que si $f$ es una función polinómica de grado $n$, entonces $f$ tiene a lo sumo $n$ raíces, es decir, existen a lo sumo $n$ números $a$ tales que $f(a)=0$\\\\
	    Demostración.- \; Supóngase que $f$ tiene $n$ raíces $a_1,...,a_n$. Entonces según la parte $(b)$ podemos poner $f(x)(x-a)g_1(x)$ donde el grado de $g_1(x)$ es $n-1$. Pero $$0=f(a_2)=(a_2-a_1)g_1(a_2)$$ de modo que $g_1(a_2)=0$, ya que $a_2 \neq a_1$. Podemos pues escribir $$f(x)(x-a_2)g_2(x),$$ donde el grado de $g_2$ es $n-2$. Prosiguiendo de esta manera, obtenemos que $$f(x)=(x-a_1)(x-a_2)\cdot ... \cdot (x-a_n)c$$ para algún número $c \neq 0.$ Está claro que $f(a)\neq 0$ si $a \neq a_1,...,a_n.$ Así pues, $f$ puede tener a lo sumo $n$ raíces.\\\\

	    %----------(d)
	    \item Demostrar que para todo $n$ existe una función polinómica de grado $n$ con raíces. Si $n$ es par, encontrar una función polinómica de grado $n$ sin raíces, y si $n$ es impar, encontrar una con una sola raíz\\\\
	    Demostración.- \; Si $f(x)=(x-1)(x-2)\cdot ... \cdot (x-n),$ entonces $f$ tiene $n$ raíces. Si $n$ es par, entonces $f(x)=x^n + 1$ no tiene raíces. Si $n$ es impar, entonces $f(x)=x^n$ tiene una raíz única, que es $0.$\\\\

	\end{enumerate}

	%-----------------8.
	\item ¿Para qué números $a,b,c$ y $d$ la función 
	$$f(x)=\dfrac{ax+d}{cx+b}$$ 
	satisface $f(f(x))=x$ para todo $x$?\\\\
	Respuesta.-\; Si $$x=f(f(x))=\dfrac{a\left(\dfrac{ax+b}{cx+d}\right)+b}{c\left(\dfrac{ax+b}{cx+d}\right)+d}$$ para todo $x$, entonces $$x=\dfrac{a^2x+ab+bcx+bd}{acx+bc+cdx+d^2}$$ y por lo tanto $$\left(ac+cd\right)x^2 + \left(d^2-a^2\right)x - ab - bd = 0$$ para todo $x$, de modo que 
	\begin{center}
	    \begin{tabular}{rcl}
		$ac+cd$&=&$0$\\
		$ab+bd$&=&$0$\\
		$d^2-a^2$&=&$0$\\
	    \end{tabular}
	\end{center}
	Se sigue que $a=d$ ó $a=-d.$ Una posibilidad es $a=d=0$, en cuyo caso $f(x)=\dfrac{b}{cx}$ que satisface $f(f(x))=x$ para todo $x\neq 0$. Si $a=d\neq 0$, entonces $b=c=0$ con lo que $f(x)=x$. La tercera posibilidad es $a+d=0$, de modo que $f(x)=\dfrac{ax+b}{cx-a}$, la cual satisface $f(f(x))=x$ para todo $x\neq \dfrac{a}{c}$ la cual satisface $f(f(x))=x$ para todo $x\neq \dfrac{a}{c}$. Estrictamente hablando, podemos añadir la condición $f(x)\neq \dfrac{a}{c}$ para $x\neq \dfrac{a}{c}$, lo que significa que $$\dfrac{ax+b}{cx-a}\neq \dfrac{a}{c}, \; ó \; a^2+bc\neq 0.$$\\\\

	%--------------------9.
	\item 
	\begin{enumerate}[\bfseries (a)]

	    %----------(a)
	    \item Si $A$ es un conjunto cualquiera de números reales, defínase una función $C_A$ como sigue:  
		   $$C_A(x) = \left\lbrace
			\begin{array}{c l}
			    $1$,&si \; $x$ \; está \; en \;$A$\\
			    $0$,&si \; $x$\; no\; está \; en \;  $A$
			\end{array}
			    \right.$$
		    Encuéntrese expresiones para $C_{A\cap B}, \; C_{A \cup B}$ y $C_{\mathbb{R}-A}$, en términos de $C_A$ y $C_B$. \\\\
	    Respuesta.-\; Según la definición de teoría de conjunto tenemos,
	    \begin{center}
		\begin{tabular}{rcl}
		    $C_{A\cap B}$&$=$&$C_A \cdot C_B$\\
		    $C_{A\cup B}$&$=$&$C_A + C_B - C_A \cdot C_B$\\
		    $C_{\mathbb{R} - A}$&$=$&$1 - C_A$\\\\
		\end{tabular}
	    \end{center}
	    
	    %----------(b)
	    \item Supóngase que $f$ es una función tal que $f(x)=0$ o $1$ para todo $x$. Demostrar que existe un conjunto $A$ tal que $f=C_A$\\\\
	    Demostración.-\; Sea $A=\lbrace x\in \mathbb{R}: f(x)=1\rbrace$ , entonces $f=C_A$.\\\\

	    %----------(c)
	    \item Demostrar que $f=f^2$ si y sólo si $f=C_A$ para algún conjunto $A$\\\\
	    Demostración.-\; Sea $f=f^2$, entonces para cada real $x$, $f(x)=f[f(x)]^2$, así $f(x)=0$ó $f(x)=1$, luego por la parte $b)$, $f=C_A$ para algún $A$.\\
	    Por otro lado sea $f=C_A$ para algún $A$. Entonces si $x\in A$, $f(x)=1=1^2=f(x)^2$, mientras si $x\notin A,$ $f(x)=0=0^2=f(x)^2$, así en cualquier caso $f(x)=[f(x)]^2$ y $f=f^2$\\\\

	\end{enumerate}

	%--------------------10.
	\item 
	\begin{enumerate}[\bfseries (a)]

	    %----------(a)
	    \item ¿Para qué funciones $f$ existe una función $g$ tal que $f=g^2$?\\\\
	    Respuesta.-\; Debido a que algún número elevado al cuadrado siempre será no negativo podemos afirmar que las funciones $f$ satisfacen a todo $x$ tal que $f(x)\geq 0$\\\\

	    %----------(b)
	    \item ¿Para qué función $f$ existe una función $g$ tal que $f=1/g$?\\\\
	    Respuesta.-\; Dado a que un número divido entre cero es indeterminado se ve claramente que satisfacen a todo $x$ tal que $f(x)\neq 0$\\\\

	    %----------(c)
	    \item ¿Para qué funciones $b$ y $c$ podemos encontrar una función $x$ tal que $$(x(t))^2 + b(t)x(t)+c(t)=0$$ para todos los números $t$?\\\\
	    Respuesta.-\; Por teorema se observa que para las funciones $b$ \; y \; $c$ que satisfacen $(b(t))^2 - 4c(t) \geq 0$ para todo $t$\\\\ 

	    %----------(d)
	    \item ¿Qué condiciones deben satisfacer las funciones $a$ y $b$ si ha de existir una función $x$ tal que $$a(t)x(t)+b(t)=0$$ para todos los números $t$? ¿Cuántas funciones $x$ de éstas existirán?\\\\
	    Respuesta.-\; Es facil notar que $b(t)$ tiene que ser igual a $0$ siempre que $a(t) = 0$. Si $a(t) \neq 0$ para todo $t$, entonces existe una función única con esta condición, que es $x(t) = a(t)/b(t)$. Si $a(t)=0$ para algún $t$, entonces puede elegirse arbitrariamente $x(t)$, de modo que existen infinitas funciones que satisfacen la condición.\\\\

	\end{enumerate}

	%--------------------11.
	\item 

	\begin{enumerate}[\bfseries (a)]

	    %----------(a)
	    \item Supóngase que $H$ es una función $e$ y un número tal que $H(H(y))=y$. ¿Cuál es el valor de $$H(H(H...(H(y))))?$$\\\\
	    Respuesta.-\; Si aplicamos la hipótesis, tendremos que aplicar $78$ veces la función, luego $76$ y así, hasta llegar a $2$, donde la función sera $H(H(y))$, y una vez más por hipótesis tenemos como resultado $y$.\\\\

	    %----------(b)
	    \item La misma pregunta sustituyendo $80$ por $81$\\\\
	    Respuesta.-\; Sea $H(H(y))$ la $78ava$ vez de la función, entonces la $81ava$ vez será $H(H(H(y))$, por lo tanto queda como resultado $H(y)$.\\\\ 

	    %----------(c)
	    \item La misma pregunta si $H(H(y))=H(y)$\\\\
	    Respuesta.-\; Análogamente a la parte $a)$ si la $80ava$ vez es $y$ entonces por hipótesis nos queda $H(y)$.\\\\ 

	    %----------(d)
	    \item Encuéntrese una función $H$ tal que $H(H(x))=H(x)$ para todos los números $x$ y tal que $H(1)=36$, $H(2)=\dfrac{\pi}{3}$, $H(13)=47$, $H(36)36$, $H(\pi / 3)\dfrac{\pi}{3}$, $H(47)=47$\\\\
	    Respuesta.-\; Dar a $H(l)$, $H(2)$, $H(13)$, $H(36)$, $H(\pi /3)$, y $H(47)$ los valores especificados y hágase $H(x) = 0$ para $x \neq 1, 2, 13, 36, \pi /3, 47.$ Al ser, en particular, $H(0) = 0$, la condición $H(H(x)) = H(x)$ se cumple para todo $x$.\\\\

	    %----------(e)
	    \item Encontrar una función $H$ tal que $H(H(x))=H(x)$ para todo $x$ y tal que $H(1)=7$, $H(17)=18$\\\\
	    Respuesta.-\; Hágase $H(1) = 7$, $H(7) = 7$, $H(17) = 18$, $H(18) = 18$, y $H(x) = 0$ para $x \neq l , 7, 17, 18$.\\\\

	\end{enumerate}

	%--------------------12.
	\item Una función $f$ es par si $f(x)=f(-x),$ e impar si $f(x)=-f(-x)$. Por ejemplo, $f$ es par si $f(x)=x^2$ ó $f(x)=|x|$ ó $f(x)=\cos x$, mientras que $f$ es impar si $f(x)=x$ ó $f(x)=\sen x$.

	\begin{enumerate}[\bfseries (a)]

	    %----------(a)
	    \item Determinar si $f+g$ es par, impar o no necesariamente ninguna de las dos cosas, en los cuatro casos obtenidos al tomar $f$ par o impar y $g$ par o impar. (Las soluciones pueden ser convenientemente dispuestas en una tabla $2$ x $2$)\\\\
	    Respuesta.-\; Sea $f(x)=x^2$ y $g(x)=|x|$ entonces $f(-x)+g(-x)=(-x)^2 + |-x| = x^2 + |x| = f(x) + g(x)$ por lo tanto par y par es par.\\
	    Sea $f(x) = x$ y $g(x)=x$ entonces $-f(-x) + (-g(-x)) = -(-x) + [-x(-x)] = x + x = f(x) + g(x)$, por lo tanto impar e impar es impar.\\
	    Los otros dos últimos se prueba fácilmente y se llega a la conclusión de que ni uno ni lo otro.
	    \begin{center}
		\begin{tabular}{c|cc}
		    &\textbf{Par}&\textbf{Par}\\
		    \hline\\
		    \textbf{Par}&Par&Ninguno\\\\
		    \textbf{Par}&Ninguno&Par\\
		\end{tabular}
	    \end{center}
	    \vspace{1cm}

	    %----------(b)
	    \item Hágase lo mismo para $f\cdot g$\\\\
	    Respuesta.-\; Sea $f(x)=x^2$ \; y \; $g(x)=|x|$, entonces $f(-x) \cdot g(-x) = x^2 \cdot |x| = f(x) \cdot g(x)$, por lo tanto se cumple para par y par.\\
	    Sea $f(x)=x$ \; y \; $g(x)=x$, entonces $-f(-x) \cdot -g(-x) = -(-x) \cdot -(-x) = x \cdot x = f(x) \cdot g(x)$, por lo tanto impar impar da impar\\
	    Sea $f(x)=x^2$ \; y \; $g(x)=x$, podemos crear otra función llamada $h$ que contiene a $x^2 \cdot x$ por lo tanto $h(x)=x^3 = -(-x)^2$ y así demostramos que par e impar es impar.\\
	    De igual forma al anterior se puede probar que impar y par es impar.
	    \begin{center}
		\begin{tabular}{c|cc}
		    &\textbf{Par}&\textbf{Par}\\
		    \hline\\
		    \textbf{Par}&Par&Impar\\\\
		    \textbf{Par}&Impar&Par\\\\
		\end{tabular}
	    \end{center}
	    \vspace{1cm}

	    %----------(c)
	    \item Hágase lo mismo para $f\circ g$\\\\
	    Respuesta.-\; Sea $f(x)=x$ \; y \; $g(x)=x$, luego $h(x)=(f \circ g)(x)$ entonces $h(x) = x$ luego $-f(-x) = x$, por lo tanto impar e impar da impar.\\
	    De similar manera se puede encontrar para los demás problemas y queda:
	    \begin{center}
		\begin{tabular}{c|cc}
		    &\textbf{Par}&\textbf{Par}\\
		    \hline\\
		    \textbf{Par}&Par&Par\\\\
		    \textbf{Par}&Par&Impar\\\\
		\end{tabular}
	    \end{center}
	    \vspace{1cm}

	    %----------(d)
	    \item Demostrar que para toda función par $f$ puede escribirse $f(x)=g(|x|)$, para una infinidad de funciones $g$.\\\\
	    Demostración.-\; Sea $g(x)=f(x)$ sabemos que $f$ es par si $f(x)=f(-x)$, de donde $g(x)=f(-x)$, luego  por definición de valor absoluto se tiene $g(|x|)=f(|-x|)$, y por lo tanto $f(x)=g(|x|)$\\\\

	\end{enumerate}

	%--------------------13.
	\item 
	\begin{enumerate}[\bfseries (a)]

	    %----------(a)
	    \item Demostrar que para toda función $f$ con dominio $\mathbf{R}$ puede ser puesta en la forma $f=E+O,$ con $E$ par y $O$ impar.\\\\
	    Demostración.-\; Por la parte $(b)$ y resolviendo en $E(x)$ y $O(x)$ se tiene $$E(x)\dfrac{f(x)+f(-x)}{2}, \;\;\; O(x)\dfrac{f(x)-f(-x)}{2}$$\\\\

	    %----------(b)
	    \item Demuéstrese que esta manera de expresar $f$ es única. (Si se intenta resolver primero la parte $(b)$ despejando $E$ y $O$, se encontrará probablemente la solución a la parte $(a)$)\\\\
	    Demostración.-\; Si $f=E+O$, siendo $E$ par y $O$ impar, entonces $$f(x)=E(x)+O(x)$$ $$f(-x)=E(x)-O(x)$$\\\\

	\end{enumerate}

	%--------------------14.
	\item Si $f$ es una función cualquiera, definir una nueva función $|f|$ mediante $|f|(x)=|f(x)|$. Si $f$ y $g$ son funciones, definir dos nuevas funciones, $max(f,g)$ y $min(f,g)$ mediante $$max(f,g)(x)=max(f(x),g(x)),$$ $$min(f,g)(x)=min(f(x),g(x))$$ Encontrar una expresión para $max(f,g)$ y $min(f,g)$ en términos de $| |$.\\\\
	Respuesta.-\; Por problema $1.13$ se tiene que $$max(f,g)=\dfrac{f+g+|f-g|}{2};$$ $$min(f,g)=\dfrac{f+g-|f-g|}{2}$$\\\\

	%--------------------15.
	\item 

	\begin{enumerate}[\bfseries (a)]
	    
	    %----------(a)
	    \item Demostrar que $f=max(f,0)+min(f,0)$. Esta manera particular de escribir $f$ es bastante usada; las funciones $max(f,0)$ y $min(f,0)$ se llaman respectivamente parte positiva y parte negativa de $f$\\\\
	    Demostración.-\; Esta proposición mostrará que se puede dividir una función en sus partes no negativas y no positivas. Es decir para todo los elementos $x$ de algún dominio, es cierto que el valor de la función $f$ en un punto $x$ es igual a la suma dada, que consiste en la parte no negativa de $max(f(x),0)$ y la parte no positiva de $f$, $min(f(x),0)$.\\
	    Para probarlo, lo dividiremos en dos casos. Sabemos que ó $f(x)\geq 0$ ó $f(x)\leq 0$. Si $f(x)\geq 0$ entonces $max(f(x),0)=f(x)$ y $min(f(x),0)=0$ por lo que nuestra ecuación se reduce a $f(x)=f(x)+0$. Por otro lado si $f(x)\leq 0,$ entonces $max(f(x),0)=0$ y $min(f(x),0)=f(x)$, por lo que nuestra ecuación se reduce a $f(x)=0+f(x)$.\\
	    En cualquier caso, nuestro lado derecho se reduce a $f(x)$ y sabemos que al menos uno de estos dos casos es verdadero; por lo tanto concluimos que $\forall x, f(x)= max(f(x),0)+min(f(x),0)$ ó $f=max (f,0)+min(f,0)$\\\\

	    %----------(b)
	    \item Una función $f$ se dice que es no negativa si $f(x)\geq 0$ para todo $x$. Demostrar que para cualquier función $f$ puede ponerse $f=g-h$ de infinitas maneras con $g$ y $h$ no negativas. (La manera corriente es $g=max(f,0)$ y $h=-min(f,0)$. Cualquier número puede ciertamente expresarse de infinitas maneras como diferencia de dos números no negativos.)\\\\
	    Demostración.-\; Comenzamos con la observación de que, para cualquier número real no negativo $r$, hay infinitos números reales no negativos $s, t$ tales que $$r=s-t$$ De hecho, para cada $n \in \mathbb{N}$, tomamos $s_n = 2r + n$ y $t_n = r + n$. Entonces, dado que $r\geq 0$, tanto $s_n$ como $t_n$ son no negativos. Además, $$s_n=t_n=2r+n-r-n=r$$ Ahora, para cada número real $x$, tenemos que $f(x)\geq  0$. Por lo tanto, a partir de la observación anterior, vemos que hay infinitos números reales no negativos $s_x$ y $t_x$ tales que $$f(x)=s_x-t_x$$ para cada $x \in \mathbb{R}$. Así que definimos funciones no negativas $g$ y $h$ como sigue $$g(x)=s_x\;\; y \;\; h(x)=t_x$$. Entonces hemos demostrado que hay infinitas opciones de tales funciones. Además, tenemos que $$f(x)=g(x)-h(x)$$. Por lo tanto, hemos demostrado que hay infinitas funciones no negativas $g$ y $h$ tales que $$f = g-h$$\\\\

	\end{enumerate}

	%--------------------16.
	\item Supongase que $f$ satisface $f(x+y)=f(x)+f(y)$ para todo $x$ e $y$. 

	\begin{enumerate}[\bfseries (a)]

	    %----------(a)
	    \item Demostrar que $f(x_1,+...+x_n)=f(x_1)+...+f(x_n)$\\\\
	    Demostración.-\; El resultado se cumple para $n=1$, $f(x_1)=f(x_1)$. Luego si $f(x_1 + .... + x_n)=f(x_1)+ ... + f(x_n)$ para todo $x_1,...,x_n$, entonces
	    \begin{center}
		\begin{tabular}{rcll}
		    $f(x_1 + ... + x_{n+1})$ & $=$ & $f(\left[x_1 + ... + x_n\right]+x_{n+1})$&\\
		       & $=$ & $f(x_1 + ... + x_n) + f(x_{n+1})$&por hipótesis\\
		       & $=$ & $f(x_1)+...+f(n)+f(x_{n+1})$&\\\\
		\end{tabular}
	    \end{center}

	    %----------(b)
	    \item Demostrar que existe algún número $c$ tal que $f(x)=cx$ para todos los números racionales $x$ (en este punto no intentamos decir nada acerca de $f(x)$ cuando $x$ es irracional). Indicación: Piénsese primero en cómo debe ser $c$. Demostrar luego que $f(x)=cx$, primero cuando $x$ es un entero, después cuando $x$ es el reciproco de un entero, y finalmente para todo racional $x$.\\\\
	    Demostración.-\; Sea $c=f(1)$. Luego para cualquier número natural $n$ y el inciso $(a)$,  $$f(n)=f(1+...+1)=f(1)+...+f(1)=n\cdot f(1)=cn \,\,\,\, (1)$$ 
	    Al ser $$f(x)+f(0)=f(x+0)=f(x),$$ entonces $f(0)=0$. Ahora, puesto que $$f(x)+f(-x)=f(x+(-x))=f(0)=0,$$ 
	    resulta que $f(-x)=-f(x)$. En particular, para cualquier número natural $n$ y por $(1)$, $$f(-n)=-f(n)=-cn=c\cdot (-n)$$
	    Además $$f\left(\dfrac{1}{n}\right) + ... + f\left(\dfrac{1}{n}\right)=f\left(\dfrac{1}{n} + ... + \dfrac{1}{n}\right)=f\left( \dfrac{n}{n}\right)=f(1)=c$$ de modo que, $$f\left(\dfrac{1}{n}\right)=c\cdot \dfrac{1}{n},$$
	    y en consecuencia $$f\left(\dfrac{1}{-n}\right)=f\left(- \dfrac{1}{n}\right)=-f\left(\dfrac{1}{n}\right)=-c \cdot \dfrac{1}{n} = c \left(\dfrac{1}{n}\right)$$
	    Por último, cualquier número racional puede escribirse en la forma $m/n$, siendo $m$ un número natural y $n$ un entero;
	    $$f\left(\dfrac{m}{n}\right)=f\left(\dfrac{1}{n} + ... + \dfrac{1}{n}\right)=f\left(\dfrac{1}{n}\right) + ... + f\left(\dfrac{1}{n}\right)=mc\cdot \dfrac{1}{n}=c\cdot \dfrac{m}{n}$$\\\\

	\end{enumerate}

	%--------------------17
	\item Si $f(x)=0$ para todo $x$, entonces $f$ satisface $f(x+y)=f(x) + f(y)$ para todo $x$ e $y$ también $f(x\cdot y)=f(x)\cdot f(y)$ para todo $x$ e $y$. Supóngase ahora que $f$ satisface estas dos propiedades, pero que $f(x)$ no es siempre $0$. Demostrar que 

	\begin{enumerate}[\bfseries (a)]

	    %----------(a)
	    \item Demostrar que $f(1)=1$\\\\
		Demostración.-\; Al ser $f(a)=f(a\cdot 1)=f(a)\cdot f(1)$ y $f(a)\neq 0$ para algún $a$, resulta ser $f(1)=1$\\\\

	    %----------(b)
	    \item Demostrar que $f(x)=x$ si $x$ es racional \\\\
		Demostración.-\; Por el problema 16, $f(x)=f(1)\cdot x = x$ para todo número racional $x$. \\\\ 

	    %----------(c)
	    \item Demostrar que $f(x)>0$ si $x>0$. (Esta parte es artificiosa, pero habiendo puesto atención a las observaciones filosóficas que van con los problemas de los dos últimos capítulos, se sabrá lo que hacer.)\\\\
		Demostración.-\; Si $c>0$ entonces $c=d^2$ para algún $d$, de modo que $f(c)=f(d^2)=(f(d))^2\geq 0$. Por otro lado, no podemos tener $f(c)=0,$ ya que esto implicaría que $$f(a)=f\left( c\cdot \dfrac{a}{c}\right) = f(c)\cdot f\left(\dfrac{a}{c}\right) = 0 \qquad para \; todo \; a$$\\\\

	    %----------(d)
	    \item Demostrar que $f(x)>f(y)$ si $x>y$\\\\
		Demostración.-\; Si $x>y$, entonces $x-y>0$, luego por la parte $(c)$ tenemos que $f(x)-f(y)>0$. \\\\ 

	    %----------(e)
	    \item Demostrar que $f(x)=x$ para todo $x$. Indicación: Hágase uso del hecho de que entre dos números calesquiera existe un número racional\\\\
		Demostración.-\; Sea $f(x)>x$ para algún $x$. Elíjase un número racional $r$ con $x<r<f(x)$. Entonces, según las partes $(b)$ y $(d)$, $$f(x)<f(r)=r<f(x),$$ 
		lo cual constituye una contradicción. Análogamente, es imposible que $f(x)<x$ ya que si $f(x)<r<x$ entonces $$f(x)<r=f(r)<f(x).$$\\\\

	\end{enumerate}

	%--------------------18.
	\item  ¿Qué condiciones precisas deben satisfacer $f,g,h$ y $k$ para que $f(x)g(y)=h(x)k(y)$ para todo $x \; e \; y$? \\\\
	    Respuesta.-\; Se satisface la ecuación si $f=0$ ó $g=0$ y $h=0$ ó $k=0$. De no ocurrir esto, existirá algún $x$ con $f(x)\neq 0$ y algún $y$ con $g(y)\neq 0$, entonces $0 \neq ff(x)g(y) = h(x)k(y)$, de modo que también se tendrá $h(x) \neq 0$ y $k(y)\neq 0$. Haciendo $\alpha = h(x)/f(x)$, tenemos también $h(x^{'} = \alpha f(x^{'}$ para todo $x^{'}$ para todo $x^{'}$. Tenemos pues. que $g=\alpha k$ y $h=\alpha f$ para cierto número $\alpha=0$. \\\\
	
	%--------------------19.
	\item 

	\begin{enumerate}[\bfseries (a)]

	    %----------(a)
	    \item Demostrar que no existen funciones $f$ y $g$ con alguna de las propiedades siguientes:\\\\

	    \begin{enumerate}[\bfseries (i)]

		%-----(i)
		\item $f(x)+g(y)=xy$ para todo $x$ e $y$.\\\\
		    Demostración.-\; Si $f(x)+g(y)=xy \; \forall x,y$ entonces para $y=0$ tenemos $f(x)+g(0)=0 \; \forall x.$, de donde $f(x)=-g(0)$, e implica que $f$ es una función constante. Luego $$xy=f(x)+g(y)=-g(0) +g(y)  \; \forall y$$ porque $f(x)$ es constante para cualquier $x$. Por otro lado sabemos que  $g(0)$ es una constante y $g(y)$  no depende de $x$, sin embargo su diferencia está dada por $g(y) - g(0) = xy$. Y finalmente sea $x=0$ entonces $g(y)=g(0) \; \forall y$,  por lo tanto se concluye que $$xy=f(x)+g(x) = -g(0) + g(0) = 0 \; \forall x,y$$  ya que si tomamos $x=y=1$ implica que $1=0$ donde llegamos a un absurdo.\\\\  

		%-----(ii)
		\item $f(x) \cdot g(y) = x+y$ para todo $x$ e $y$.\\\\
		    Demostración.-\; Sea $y=0$, obtenemos $f(x)=x/g(0)$. De la misma forma si $x=0$, entonces $g(y)=y/f(0)$. Por lo tanto $$  f(x) \cdot g(y) = x+y \Longrightarrow \dfrac{x}{g(0)}\cdot \dfrac{y}{f(0)}=x+y \qquad \forall x, \; e \; \forall y$$ Supongamos que $y=0$, entonces $\dfrac{x}{g(0)}\cdot \dfrac{0}{f(0)}=x \quad \forall x \Longrightarrow 0=x \quad \forall x$, lo cual es absurdo.\\\\

	    \end{enumerate}

	    %----------(b)
	    \item Hallar funciones $f$ y $g$ tales que $f(x+y) = g(xy)$ para todo $x$ e $y$. \\\\
		Respuesta.-\; Sean $f$ y $g$ la misma función constante. Argumentos similares a los utilizados en la parte $(a)$ muestran que estas son las únicas opciones posibles. \\\\ 

	\end{enumerate}

	%--------------------20.
	\item 
	\begin{enumerate}[\bfseries (a)]

	    %----------(a)
	    \item Hallar una función $f$ que no sea constante y tal que $|f(y) - f(x)| \leq |y-x|$.\\\\
		Respuesta.-\; Podemos ver que la función $f(x)=x$ satisface la condición $|f(y) - f(x)| \leq |y-x|$\\\\

	    %----------(b)
	    \item Supóngase que $f(y) - f(x) \leq (y-x)^2$ para todo $x$ e $y.$ (¿ Por qué esto implica $|f(y) - f(x)| \leq (y-x)^2$?) Demostrar que $f$ es una constante. Indicación: Divídase el intervalo $[x,y]$ en $n$ partes iguales.\\\\
		Demostración.-\; Supongamos, que puede probar que la siguiente desigualdad es cierta para todos $x , y \in \mathbb{R}$, y $n \in \mathbb{N}$: $$|f(y) - f(x)| \leq \dfrac{(y-x)^2}{n}$$ Ahora mantengamos los valores de $x$ e $y$ constantes. Podemos suponer $x\neq y$ (porque si $x=y$ entonces $f(x)=f(y)$ y así terminaríamos la demostración). Entonces, en el lado derecho, el numerador $(y-x)^2$ es distinto de $0$, y mayor a cero. Por lo tanto, podemos dividir por $(y-x)^2$, de donde: $$\dfrac{|f(y) - f(x)|}{(y-x)^2} \leq \dfrac{1}{n}$$ En el lado izquierdo tenemos un número no negativo que es constante (ya que $x$ e $y$ se mantienen constantes, el numerador no es negativo y el denominador es positivo). Este número es menor que cada fracción $\dfrac{1}{n}$ para todos los números naturales $n\geq 1$. Esto implica que el lado izquierdo es igual a cero:
		$$\dfrac{|f(y) - f(x)|}{(y-x)^2} = 0$$ una vez mas multiplicamos por $(y-x)^2$ entonces $$|f(y)-f(x)|=0,$$ de donde $$|f(y)-f(x)|=0 \Longrightarrow f(y)=f(x)$$
		Dado que esto es cierto para todos los valores $x$, $y$  terminamos la demostración.\\\\

	\end{enumerate}

	%--------------------21.
	\item  Demostrar o dar un contraejemplo de las siguientes proposiciones:

	    \begin{enumerate}[\bfseries (a)]

		%----------(a)
		\item $f \circ (g+h) = f \circ g + f \circ h.$\\\\
		    Demostración.-\; Esto es falso en general ya que si designamos a $g$ y $h$ la función identidad y  $f$ sea $x^2$ entonces $$\left[ f \circ (g+h) \right](x) = f\left(  g + h\right)(x)=f \left[ g(x) +  h(x) \right] = f(x+x)= f(2x) = 4x^2.$$ 
		    luego por la parte derecha de la ecuación se tendra:
		    $$\left[ \left( f\circ g \right) + \left( f \circ h \right)\right]\left(x \right) = \left(f \circ g \right) \left(x\right) + \left( f \circ h \right)\left(x\right) = f\left[g\left(x\right)\right] + f\left[h \left(x\right)\right] = f\left(x\right) + g\left(x\right) = x^2 + x$$
		    De donde $4x^2 \neq x^2 + x$\\\\

		%----------(b)
		\item $(g+h) \circ f = g\circ f + h \circ f.$\\\\
		    Demostración.-\; Por definición de composición de función tenemos 
		    \begin{center}
			\begin{tabular}{rcll}
			    $\left[ \left( g+ h\right) \circ f \right] \left(x\right)$ & $=$ & $\left(g+h\right)\left[f(x)\right]$ & \\ 
			    & $=$ & $g\left[ f(x)\right] + h\left[f(x)\right]$ & por definición\\
			    & $=$ & $\left(g\circ f\right)(x) + \left(h \circ f\right)(x)$ & \\
			    & $=$ & $\left[ \left(g\circ f\right)  + \left(h \circ f\right)\right](x)$ &\\\\
			\end{tabular} 
		    \end{center}
		    Así  $(g+h)\circ f = (g\circ f)+(h \circ f)$\\\\

		%----------(c)
		\item $\dfrac{1}{f\circ g} = \dfrac{1}{f} \circ g.$\\\\
		    Demostración.-\; Por definición se tiene,
		    \begin{center}
			\begin{tabular}{rcl}
			    $\left( \dfrac{1}{f\circ g}\right)(x)$ & $=$ & $\dfrac{1}{(f\circ g)(x)}$\\\\
			    & $=$ & $\dfrac{1}{f\left[ g(x)\right] }$\\\\
			    & $=$ & $\left( \dfrac{1}{f}\right) \left[ g(x)\right]$\\\\
			    & $=$ & $\left( \dfrac{1}{f} \circ g\right)(x)$\\\\
			\end{tabular}
		    \end{center}
		    Así, $1/(f\circ g)=(1/f)\circ g$\\\\

		%----------(d)
		\item $\dfrac{1}{f \circ g} = f \circ \left( \dfrac{1}{g}\right).$\\\\
		    Demostración.-\;  Esto es falso ya que si consideramos $f(x)=x+1$ y $g(x)=x^2$, entonces 
		    $$\left( \dfrac{1}{f \circ g}\right) (x) = \dfrac{1}{(f\circ g)(x)} = \dfrac{1}{f\left[ g(x) \right]} = \dfrac{1}{f(x^2)} = \dfrac{1}{x^2+1}$$ y por otro lado 
		    $$\left[f\circ \left( \dfrac{1}{g}\right)\right](x)  = f\left[ \left( \dfrac{1}{g}\right) (x)\right] = f\left( \dfrac{1}{g(x)}\right) = f \left( \dfrac{1}{x^2}\right)= \dfrac{1}{x^2} + 1$$ 
		    de  donde $\dfrac{1}{x^2 + 1} \neq \dfrac{1}{x^2} + 1$\\\\

	    \end{enumerate}

	%--------------------22.
	\item  
	\begin{enumerate}[\bfseries (a)]
	    
	    %----------(a)
	    \item Supóngase que $g=h \circ f$. Demostrar que si $f(x)=f(y)$, entonces $g(x)=g(y)$.\\\\
		Demostración.-\;  $g(x)=h\left(f(x)\right) =h\left(f(y)\right) = g(y) $ esto por definición  e hipótesis.\\\\ 
	    
	    %----------(b)
	    \item Recíprocamente, supóngase que $f$ y $g$ son dos funciones tales que $g(x)=g(y)$ siempre que $f(x)=f(y)$. Demostrar que $g = h\circ f$ para alguna función $h$. Indicación: Inténtese definir $h(z)$ cuando $z$ es de la forma $z=f(x)$ (Éstos son los únicos $z$ que importan) y aplicar la hipótesis para demostrar que la definicón es consistente.\\\\
		Demostración.-\; Si $z=f(x)$, defínase $h(z)=g(x)$. Esta definición tiene sentido, ya que si $z=f(x^{'},$ entonces $g(x)=g(x^{'})$ según la parte $(a)$. Tenemos entonces, para todo $x$ del dominio de $f$, $g(x)=h(f(x))$.\\\\

	\end{enumerate}

	%--------------------23.
	\item Supóngase que $f\circ g = I$ donde $I(x)=x$. demostrar que 

	\begin{enumerate}[\bfseries (a)]

	    %----------(a)
	    \item Si $x\neq y$, entonces $g(x)\neq g(y)$\\\\
		Demostración.-\; Supongamos que $x\neq y$ y $g(x)=g(y)$ esto implica que $x=I(x)=f(g(x))=f(g(y))=y$. Donde vemos una contradicción.\\\\

	    %-----------(b)
	    \item Todo número $b$ puede escribirse $b=f(a)$ para algún número $a$.\\\\
		Demostración.-\; Por hipótesis $b=f(g(b))$ donde basta con poner $a=g(b)$.\\\\

	\end{enumerate}

	%--------------------24.
	\item 
	
	\begin{enumerate}[\bfseries (a)]

	    %----------(a)
	    \item Supóngase que $g$ es una función con la propiedad de ser $g(x)\neq g(y)$ si $x\neq y$. Demuéstrese que existe una función $f$ tal que $f\circ g = I$\\\\
		Demostración.-\; Es equivalente enunciar que si $x=y$, entonces $g(x)=g(y)$. en consecuencia del problema $22b$.\\\\

	    %----------(b)
	    \item Supóngase que $f$ es una función tal que todo número $b$ puede escribirse en la forma $b=f(a)$ para algún número $a$. Demostrar que existe una función $g$ tal que $f\circ g=I$\\\\
		Demostración.-\; Para cada $x$, elíjase un número $a$ tal que $x=f(a)$. Llámese a este número $g(x)$. Entonces $f(g(x))=x=I(x)$ para todo $x$.\\\\ 

	\end{enumerate}

	%--------------------25.
	\item Hallar una función $f$ tal que $g\circ f=I$ para alguna función $g$, pero tal que no exista ninguna función $h$ con $f\circ h = I$\\\\
	    Respuesta.-\; Basta hallar una función $f$ tal que $f(x)\neq f(y)$ si $x\neq y$, pero tal que no todo número sea de la forma $f(x)$, pues entonces según el problema $24(a)$ existirá una función $g$ con $g\circ f=I$, y según el problema $23(b)$ no existiría ninguna función $g$ con $f\circ g=I$. Una función que reúne estas condiciones es:
	    $$f(x) = \left\{ \begin{array}{lc} 
		x,&x\leq 0\\
		\\ x+1,& x>0\\
	    \end{array}\right.$$
	    ningún número  de los comprendidos entre $0$ y $1$ es de la forma $f(x)$.\\\\

	%--------------------26.
	\item Supóngase $f\circ g = I$ y $h\circ f = I$. Demostrar que $g=h$. Indicación: Aplíquese el hecho de que la composición es asociativa.\\\\
	    Demostración.-\; Sea $h\circ f\circ g$ entonces $h\circ (f \circ g)=h\circ I = h,$ como también $h\circ f \circ g = (h \circ f) \circ g = I \circ g = g.$\\\\ 

	%--------------------27.
	\item 
	    \begin{enumerate}[\bfseries (a)]

		%----------(a)
		\item Supóngase $f(x)=x+1.$ ¿Existen funciones $g$ tales que $f\circ g = g\circ f$?\\\\
		    Respuesta.-\; La condición $f\circ g = f \circ g$ significa que $f(x) + 1 = g(x+1)$ para todo $x$. Existen muchas funciones $g$ que satisfacen esta condición. La función $g$ puede en efecto definirse arbitrariamente para $0\leq x < 1$ y para otros $x$ pueden determinarse sus valores mediante esta ecuación.\\\\

		%----------(b)
		\item Supóngase que $f$ es una función constante. ¿Para qué funciones $g$ se cumple $f\circ g = g\circ f$?\\\\
		    Respuesta.-\; Si $f(x)=c$ para todo $x$, entonces $f\circ g = g\circ f$ si y sólo si $c=f(g(x)) = g(f(x))=g(c)$, es decir, $c=g(c)$\\\\ 

		%----------(c)
		\item Supóngase que $f\circ g = g\circ f$ para todas las funciones $g$. Demostrar que $f$ es la función identidad $f(x)=x$\\\\
		    Respuesta.-\; Si $f\circ g = g  \circ f$ para todo $g$, entonces se cumple esto en particular para todas las funciones constantes $g(x)=c$. Se sigue de la parte $(b)$ que $f(c)=c$ para todo $c.$\\\\

	    \end{enumerate}

	%--------------------28.
	\item
	\begin{enumerate}[\bfseries (a)]

	    %----------(a)
	    \item Sea $F$ el conjunto de todas las funciones cuyo dominio es $\mathbb{R}$. Demuéstrese que con las definiciones de $+$ y $\cdot $ dadas en este capítulo, se cumplen todas las propiedades $P1-P9$, excepto $P7$, siempre que $0$ y $1$ se interpreten como funciones constantes.\\\\
	    Demostración.-\; Se comprueba fácilmente.\\\\

	    %----------(b)
	    \item Demostrar que $P7$ no se cumple.\\\\
		Demostración.-\; Sea $f$ una función con $f(x)=0$ para algún $x$, pero no para todo $x$. Entonces $f\neq 0$, pero claramente no existe ninguna función $g$ con $f(x)\cdot g(x) = 1$ para todo $x$.\\\\

	    %----------(c)
	    \item Demostrar que no pueden cumplirse $P10-P12.$ En otros términos, demostrar que no existe ninguna colección $P$ de funciones en $F$, tales que $P10-P12$ se cumplen para $P$. (Es suficiente, y esto simplificará las cosas, considere sólo funciones que sean $0$, excepto en dos puntos $x_0$ y $x_1$).\\\\
		Demostración.-\; Sean $f$ y $g$ dos funciones cuyos valores son todos $0$ excepto en $x_0$ y $x_1$, siendo $f(x_0)=1$, $f(x_1)=0$, $gx_0 = 0$, $g(x_1)=1.$ Ninguna de ellas es $0$, de modo que o bien $f$ o bien $-f$ tendría que estar en $P$ y lo mismo podría decirse de $g$ o $-g$. Pero $(\pm f)(\pm g) = 0$, en contradicción con $P12$.\\\\

	    %----------(d)
	    \item Supóngase que se ha definido $f<g$ en el sentido de que $f(x)<g(x)$ para todo $x$. ¿Cuáles de las propiedades $P^{'}-P^{'}13$ (del problema $1-8$)? se cumplen ahora?\\\\
		Respuesta.-\; $P^{'} 11, P^{'} 12$ y $P^{'}13$ se cumplen. $P^{'}10$ es falso; si bien se cumple a lo sumo una de las condiciones, o es necesariamente cierto que se tenga que cumplir por lo menos una de ellas. Por ejemplo, si $f(x)>0$ para algún $x$ y $y<0$ para otro $x$, entonces ninguna de las condiciones $f=0,$ $f<0$ ó $f>0$ para todo $x$.\\\\

	    %----------(e)
	    \item si $f<g$, ¿Se cumple $h\circ f < h \circ g$? ¿Es $f\circ h<g\circ h$?\\\\
		Respuesta.-\; No para el primer ejemplo; si $h(x)=-x,$ entonces $f<g$ implica, en realidad , que $h\circ f > h \circ g$. Si para el segundo, ya que $f(h(x)<g(h(x)$ para todo $x$.\\\\

	\end{enumerate}

    \end{enumerate}

\section{Pares ordenados}
    
    %--------------------definición 1.9
    \begin{tcolorbox}[colframe = white]
	\begin{def.}
	    $(a,b) = \lbrace {a},{a,b} \rbrace$\\
	\end{def.}
    \end{tcolorbox}

    %--------------------teorema 
    \begin{teo}
	Si $(a,b) = (c,d)$ entonces $a=c$ y $b=d$\\\\
	    
	    Demostración.-\; La hipótesis significa que $$\lbrace {a},{a,b} \rbrace \lbrace {c},{a,d}\rbrace,$$
	    Ahora bien, $\lbrace {a},{a,b} \rbrace$ contiene justamente dos elementos ${a}$ y ${a,b}$ y $a$ es el único elemento común a estos dos elementos de $\lbrace{a},{a,b}\rbrace$. Por lo tanto, $a=c$. Así pues tenemos $$\lbrace {a},{a,b}\rbrace = \lbrace {a},{a,d} \rbrace,$$ y solamente queda por demostrar $b=d.$ Conviene distinguir dos casos.\\
	    \textbf{Caso 1 $b=a$.} En este caso, ${a,b}={a}$, de modo que el conjunto $\lbrace {a},{a,b}\rbrace$ tiene en realidad un solo elemento que es ${a}$. Lo mismo vale para $\lbrace {a},{a,b}\rbrace$, de modo que ${a,d}={a},$ lo cual implica $d=a=b.$\\
	    \textbf{Caso 2. $b\neq a$}. En este caso, $b$ pertenece a uno de los elementos de $\lbrace {a}, {a,b}\rbrace$, pero no al otro. Debe, por lo tanto cumplirse que $b$ pertenece a uno de los elementos de $\lbrace {a},{a,b}\rbrace$, pero no al otro. Esto solamente puede ocurrir si $b$ pertenece a ${a,d}$, pero no a ${a}$; así pues, $b=a$ o $b=d$, pero $b\neq a$, con lo que $b=d$.\\\\

    \end{teo}

