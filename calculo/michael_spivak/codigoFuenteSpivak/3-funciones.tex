\chapter{Funciones}


    %definición 1.1
\begin{tcolorbox}[colframe=white]
    \begin{def.}
	El conjunto de los números a los cuales se aplica una función recibe el nombre de \textbf{dominio} de la función.
    \end{def.}
\end{tcolorbox}

    %definición 1.2
\begin{tcolorbox}[colframe=white]
    \begin{def.}
	Si $f$ \; y \; $g$ son dos funciones cualesquiera, podemos definir una nueva función $f+g$ denominada \textbf{suma} de $f+g$ mediante la ecuación:
	$$(f+g)(x)=f(x)+g(x)$$
	Para el conjunto de todos los $x$ que están a la vez en el dominio de $f$ y en el dominio de $g$, es decir: $$dominio \; (f+g)=dominio \; f \; \cap \; dominio \; g$$\\
    \end{def.}
\end{tcolorbox}

    %definición 1.3
\begin{tcolorbox}[colframe=white]
    \begin{def.}
	El dominio de $f \cdot g$ es $dominio \; f \cap \; dominio \; g$ $$(f \cdot g)(x)=f(x)\cdot g(x)$$\\
    \end{def.}
\end{tcolorbox}

    %definición 1.4 
\begin{tcolorbox}[colframe=white]
    \begin{def.}
	Se expresa por dominio $f$ $\cap$ dominio $g$ $\cap$ $\lbrace x:g(x)\neq 0 \rbrace$
	$$\left( \dfrac{f}{g}\right) (x)=\dfrac{f(x)}{g(x)}$$ \\
    \end{def.}
\end{tcolorbox}

%definición 1.5
\begin{tcolorbox}[colframe=white]
    \begin{def.}[Función constante]
	$$(c \cdot g)(x)=c \cdot g(x)$$\\
    \end{def.}
\end{tcolorbox}

%teorema 1.1
\begin{teo}
    $(f+g)+h=f+(g+h)$\\\\
    Demostración.- \; \textbf{La demostración es característica de casi todas las demostraciones que prueban que dos funciones son iguales: se debe hacer ver que las dos funciones tienen el mismo dominio y el mismo valor para cualquier número del dominio.} Obsérvese que al interpretar la definición de cada lado se obtiene:
    \begin{center}
	\begin{tabular}{r c l}
	    $\left[ (f+g) + h \right](x)$&=&$(f+g)(x)+h(x)$\\
	    &=&$\left[ f(x) +g(x) \right] +h(x)$\\\\
	    &y&\\\\
	    $\left[ f+(g+h) \right](x)$&=&$f(x)+(g+h)(x)$\\
	    &=&$f(x)+\left[ g(x)+h(x) \right]$\\\\
	\end{tabular}
    \end{center}
    Es esta demostración no se ha mencionado la igualdad de los dos dominios porque esta igualdad parece obvia desde el momento en que empezamos a escribir estas ecuaciones: el dominio de $(f+g)+h$ y el de $f+(g+h)$ es evidentemente dominio $f$ $\cap$ dominio $g$ $\cap$ dominio $h$. Nosotros escribimos, naturalmente $f+g+h$ por $(f+g)+h=f+(g+h)$\\\\
\end{teo}

%teorema 1.2
\begin{teo}
    Es igual fácil demostrar que $(f\cdot g)\cdot g=f\cdot (g \cdot h)$ y ésta función se designa por $f \cdot g \cdot h$. Las ecuaciones $f+g=g+f$ \; y \; $f\cdot g=g \cdot f$ no deben presentar ninguna dificultad.\\\\
\end{teo}

%definición 1.6
\begin{tcolorbox}[colframe=white]
    \begin{def.}[Composición de función]
	$$(f \circ g)(x)=f(g(x))$$
	El dominio de $f\circ g$ es $\lbrace $ $x$ : $x$ está en el dominio de $g$ \: y \; $g(x)$ está en el dominio de $f$ $\rbrace$
	$$D_{f \circ g}= \lbrace x \; / \; x \in D_g \; \land \; g(x)\in D_f \rbrace$$
    \end{def.}
\end{tcolorbox}

    %propiedad 1.1
\begin{tcolorbox}[colframe=white]
    \begin{prop}
    $(f \circ g) \circ h = f \circ (g \circ h)$   La demostración es una trivalidad.
    \end{prop}
\end{tcolorbox}

%definición  1.7 
\begin{tcolorbox}
    \begin{def.} 
	Una \textbf{función} es una colección de pares de números con la siguiente propiedad: Si $(a,b)$ \; y \; $(a,c)$ pertenecen ambos a la colección, entonces $b=c$; en otras palabras, la colección no debe contener dos pares distintos con el mismo primer elemento.\\
    \end{def.}
\end{tcolorbox}

%definición 1.8
\begin{tcolorbox}
    \begin{def.} 
	Si $f$ es una función, el \textbf{dominio} de $f$ es el conjunto de todos los $a$ para los que existe algún $b$ tal que $(a,b)$ está en $f$. Si $a$ está en el dominio de $f$, se sigue de la definición de función que existe, en efecto, un número $b$ único tal que $(a,b)$ está en $f$. Este $b$ único se designa por $f(a)$.\\  
    \end{def.}
\end{tcolorbox}


\section{Problemas}

    \begin{enumerate}[\Large \bfseries 1.]

	%--------------------1.
	\item Sea $f(x)=1/(1+x)$. Interpretar lo siguiente:
	    \begin{enumerate}[\bfseries (i)]

		%----------(i)
		\item $f(f(x))$ (¿Para que $x$ tiene sentido?)\\\\
		Respuesta.- \; Sea $f\left( \dfrac{1}{1+x} \right)$ entonces $\dfrac{1}{1 + \dfrac{1}{1+x}}$, por lo tanto $\dfrac{1-x}{x+2}$ de donde llegamos a la conclusión de que $x$ se cumple para todo número real de $1$ y $-2$\\\\

		%----------(ii)
		\item $f\left( \dfrac{1}{x} \right)$\\\\
		Respuesta.- \; $\dfrac{1}{1 + \dfrac{1}{x}}=\dfrac{1}{\dfrac{x+1}{x}}=\dfrac{x}{x+1}$ por lo tanto se cumple para todo $x\neq -1, 0$\\\\

		%----------(iii)
		\item $f(cx)$\\\\
		Respuesta.- \; $\dfrac{1}{1+cx}$ donde se cumple para todo $x\neq -1$ si $c\neq 0$\\\\

		%----------(iv)
		\item $f(x+y)$\\\\
		Respuesta.- \; $\dfrac{1}{1+x+y}$ donde se cumple para todo $x+y\neq-1$\\\\

		%----------(v)
		\item $f(x) + f(y)$\\\\
		Respuesta.- \; $\dfrac{1}{1+x} + \dfrac{1}{1+y}=\dfrac{x+y+2}{(1+x)(1+y)}$ siempre y cuando $x\neq-1$ y $y \neq -1$\\\\

		%----------(vi)
		\item ¿Para que números $c$ existe un número $x$ tal que $f(cx)=f(x)$?\\\\
		Respuesta.- \; Para todo $c$ ya que $f(c\cdot 0)=f(0)$\\\\

		%----------(vii)
		\item ¿Para que números $c$ se cumple que $f(cx)=f(x)$ para dos números distintos $x$?\\\\
		Respuesta.- \;  Solamente $c=1$ ya que $f(x)=f(cx)$ implica que $x=cx$, y esto debe cumplirse por lo menos para un $x \geq 1$\\\\

	    \end{enumerate}

	%--------------------2.  
	\item Sea $g(x)=x^2$ y sea 
	\begin{equation*}
	    h(x) = \left\lbrace
		\begin{array}{rl}
		    0, & x \; racional\\
		    1, & x \; irracional
		\end{array}
	    \right.
	\end{equation*}

	\begin{enumerate}[\bfseries (i)]

	    %----------(i)
	    \item ¿Para cuáles $y$ es $h(y) \leq y$?\\\\
	    Respuesta-. \; Se cumple para $y\geq 0$ si $y$ es racional, o para todo $y\geq 1$\\\\

	    %----------(ii)
	    \item ¿Para cuáles $y$ es $h(y) \leq g(y)$?\\\\
	    Respuesta-. \; Para $-1\leq y \leq 1$ siempre que $y$ sea racional y para todo $y$ tal que $|y|\leq1$\\\\

	    %----------(iii)
	    \item ¿Qué es $g(h(z)) - h(z)$?\\\\
	    Respuesta-. \; 
	    \begin{equation*}
		g(h(z)) = \left\lbrace
		    \begin{array}{rl}
			0, & z^2 \; racional\\
			1, & z^2 \; irracional
		    \end{array}
		\right.
	    \end{equation*}

	    Por lo tanto el resultado es $0$\\\\

	    %----------(iv)
	    \item ¿Para cuáles $w$ es $g(w)\leq w $?\\\\
	    Respuesta-. \; Para todo $w$ tal que $0\leq w\leq 1$\\\\

	    %----------(v)
	    \item ¿Para cuáles $\epsilon$ es $g(g(\epsilon)) = g(\epsilon)$?\\\\
	    Respuesta-. \; Para $-1,0,1$\\\\ 

	\end{enumerate}

	%--------------------3.
	\item Encontrar el dominio de las funciones definidas por las siguientes fórmulas:
	    \begin{enumerate}[\bfseries (i)]

	    %----------(i)
	    \item $f(x)=\sqrt{1-x^2}$\\\\
	    Respuesta.- \; Por la propiedad de raíz cuadrada, se tiene  $1-x^2 \geq 0$ entonces $x^2 \leq 1$ por lo tanto el dominio son todos los $x$ tal que $|x| \leq 1$\\\\

	    %-----------(ii)
	    \item $f(x)=\sqrt{1-\sqrt{1-x^2}}$\\\\
	    Respuesta.- \; Se observa claramente que el dominio es $-1\leq x \leq 1$\\\\

	    %-----------(iii)
	    \item $f(x)=\dfrac{1}{x-1} + \dfrac{1}{x-2}$\\\\
	    Respuesta.- \; Operando un poco tenemos $$f(x) = \dfrac{2x-3}{(x-1)(x-2)},$$ sabemos que el denominador no puede ser $0$ por lo tanto el $D_{f} = \lbrace x\; / \; x \neq 1, \; x\neq  2 \rbrace$\\\\ 
    
	    %-----------(iv)
	    \item $f(x)=\sqrt{1-x^2} + \sqrt{x^2-1}$\\\\
	    Respuesta.- \; Claramente notamos que el dominio de $f$ son $-1$ y $1$ ya que si se toma otros números daría un número imaginario.\\\\

	    %-----------(v)
	    \item $f(x)=\sqrt{1-x}+\sqrt{x-2}$\\\\
	    Respuesta.- \; Notamos que no se cumple para ningún $x$ ya que si $0\leq x \leq 1$ entonces no se cumple para $\sqrt{x-2} $ y si $x\geq 2$ no se cumple para $\sqrt{1-x}$\\\\ 

	    \end{enumerate}

	%--------------------4.
	\item Sean $S(x)=x^2,$ $P(x)=2^x$ y $s(x)=sen x$. Determinar los siguientes valores. En cada caso la solución debe ser un número.\\\\

	\begin{enumerate}[\bfseries (i)]

	    %----------(i)
	    \item $(S \circ P)(y)$\\\\
	    Respuesta.- \; Por definición se tiene que $(S \circ P)(y)=S(P(y))$ entonces $S(2^y)=2^{2y}$ siempre y cuando $D_{S \circ P}=\lbrace y / y \in D_P \land P(y) \in D_S\rbrace$\\\\
	    
	    %----------(ii)
	    \item $(S \circ s)(y)$\\\\
	    Respuesta.- \; Por definición tenemos que $(S \circ s)(y)=S(s(y))$ entonces $S(\sen y)=\sen^2 y$ siempre y cuando $D_{S \circ s}=\lbrace y / y \in D_s \land S(y) \in D_S \rbrace$\\\\

	    %----------(iii)
	    \item $(S \circ P \circ s)(t)+(s \circ P)(t)$\\\\
	    Respuesta.- \; $(S \circ P \circ s)(t)+(s \circ P)(t) = S((P \circ s)(t))+s(P(t)) = S(P(s(t))) + s(P(t))=S(P(\sen t)) + s(2^t)=S(2^{\sen t}) +  \sen 2^t = 2^{2 \sen t} +\sen 2^t$\\\\  

	    %----------(iv)
	    \item $s(t^3)$\\\\
	    Respuesta.- \; $s(t^3)=\sen t^3$\\\\

	\end{enumerate}

	    %--------------------5.
	\item Expresar cada una de las siguientes funciones en términos de $S,P,s$ usando solamente $+,\cdot , \circ$\\\\
	\begin{enumerate}[\bfseries (i)]
	    
	    %----------(i)
	    \item $f(x)=2^{\sen x}$\\\\
	    Respuesta.- \; Claramente vemos que $P \circ s$\\\\ 

	    %----------(ii)
	    \item $f(x) = \sen 2^x$\\\\
	    Respuesta.- \; $s \circ P$\\\\

	    %----------(iii)
	    \item $f(x) = \sen x^2$\\\\
	    Respuesta.- \; $s \circ S$\\\\

	    %----------(iv)
	    \item $f(x) = \sen x$\\\\
	    Respuesta.- \; $S \circ s$\\\\

	    %----------(v)
	    \item $f(t) = 2^{2t}$\\\\
	    Respuesta.- \; $P \circ P$\\\\

	    %----------(vi)
	    \item $f(u)=\sen (2^u + 2^{u^2})$\\\\
	    Respuesta.- \; $s \circ (P + P \circ S)$\\\\

	    %----------(vii)
	    \item $f(y) = \sen (\sen (\sen (2^{2^{2^{\sen y}}})))$\\\\
	    Respuesta.- \; $s \circ s \circ s \circ P \circ P \circ P \circ s$\\\\

	    %----------(viii)
	    \item $f(a)= 2^{\sen^2 a} +  \sen(a^2) + 2^{sen(a^2 + \sen a)}$\\\\
	    Respuesta.- \; $P \circ S \circ s  + s \circ S + P \circ s \circ (S + s)$\\\\

	\end{enumerate}

	%--------------------6.
	\item 
	\begin{enumerate}[\bfseries (a)]

	    %----------(a)
	    \item Si $x_1, ... , x_n$ son números distintos, encontrar una función polinómica $f_i$ de grado $n-1$ que tome el valor $1$ en $x_i$ y $0$ en $x_j$ para $j \neq i.$ Indicación: El producto de todos los $(x-x_j)$ para $j \neq i$ es $0$ en $x_j$ si $j \neq i.$ Este producto es designado generalmente por $$\prod\limits_{j=1_{j \neq i}}^n (x-x_j)$$ donde el símbolo $\prod$ (pi mayúscula) desempeña para productos el mismo papel que $\sum$ para sumas.\\\\

	Respuesta.- \; Una forma de pensar sobre esta pregunta es considerar una solución fija $n$ y elegir un conjunto de distintas $x_1, x_2, ..., x_n$. Por ejemplo supongamos que elegimos $n=3$ $x_1=1$, $x_2=2$, $x_3 = 3.$ Entonces supongamos que queremos encontrar un polinomio $f_i(x_1)=f_1(1)=1,$ pero $f_1(x_2)=f_1(2)=f_1(3)=0.$ Es decir, $F_1$ es un cuadrático que tiene ceros en $x=2$ \; y\; $x=3$, pero es igual a $1$ en $x=1.$ Naturalmente, esto sugiere mirar un polinomio de la forma $$a(x-2)(x-3),$$ para que la igualdad sea igual a $1$ por alguna constante $a.$ Pero, ¿Qué es esta constante? Bueno, si nos conectamos con $x=1$, debemos tener $$f_1(1)=1=a(x-2)(x-3)=2a,$$ por lo tanto $a=1/2$ y la solución deseada es $$f_1(x)=\dfrac{1}{2}(x-2)(x-3).$$ Del mismo modo, si tratamos de encontrar un polinomio $f_2(x)$ tal que $f_2(2)=1$ con raíces en $x=1,3$ tendríamos que resolver la ecuación $1=a(2-1)(2-3),$ lo que da $a=-1$ por lo tanto $f_2(x)=-(x-1)(x-3)$\\
	    Ahora veamos el caso general. El polinomio $f_i(x)$ satisface $f_i(x_i)$ \; y \; $f_i(x_j)=0$ para todo $j \neq i$, entonces debe tomar la forma 
	    \[ f_i(x)=a \prod_{j \neq i} (x-x_j)  \]
	    Para alguna constante $a$. Para encontrar esta constante, aplicamos $x=x_1$:
	    \[ f_i(x_i)=1=a\prod_{j \neq i} (x_i-x_j), \]
	    por lo tanto:
	    \[ a= \dfrac{1}{\displaystyle\prod_{j \neq i} (x_i-x_j)}  \]  
	    Así queda
	    \[ f_i(x)= \prod_{j \neq i} \dfrac{(x-x_j)}{(x_i-x_j)} \]\\\\

	    %----------(b)
	    \item Encontrar ahora una función polinómica de grado $n-1$ tal que $f(x_1)=a_1,$ donde $a_1,...,a_n$ son números dados. (Utilícense las Funciones $f_1$ de la parte $(a)$.) La fórmula que se obtenga es la llamada \textbf{Fórmula de interpolación de Lagrange}\\\\

	    Respuesta.- \;Sea \[ f(x) = \sum_{j=1} a_i f_i(x) \] entonces  \[ f(x) = \sum_{j=1} a_i \prod_{j \neq i} \dfrac{(x-x_j)}{(x_i-x_j)} \]\\\\ 

	\end{enumerate}

	%--------------------7.
	\item
	\begin{enumerate}[\bfseries (a)]

	    %----------(a)
	    \item Demostrar que para cualquier función polinómica $f$ y cualquier número $a$ existe función polinómica $g$ y un número $b$ tales que $f(x)=(x-a)g(x)+b$ para todo $x$. (La idea es esencialmente dividir $f(x)$ por $(x-a)$ mediante la división larga hasta encontrar un resto constante.)\\
	    Demostración.- \; Si el grado de $f$ es $1$, entonces $f$ es de la forma $$f(x)=cx+d=cx+d +ac-ac=c(x-a)+(d+ac)$$ de tal modo que $g(x)=c$ y $b=d+ac$. Por inducción supongamos que el resultado es válido para polinomios de grado $\leq k.$ Si $f$ tiene grado $k+1$, entonces $f$ tiene la forma $$f(x)=a_{k+1}x^{k+1} + ... + a_1x + a_0$$ luego para grados $\leq k$ se tiene $$f(x)-a_{k+1}x^{k+1} =(x-a)g(x)+b$$ así $$f(x) = (x-a)\left[g(x) + a_{k+1}(x-a)^k\right]+b$$\\\\

	    %----------(b)
	    \item Demostrar que si $f(a)=0$, entonces $f(x)=(x-a)g(x)$ para alguna función polinómica $g$. (La reciproca es evidente)\\\\
	    Demostración.- \; Por la parte $(a)$, podemos poner que $f(x)=(x-a)g(x)+b$, entonces $$0=f(a)=(a-a)g(a)+b=b$$ de modo que $f(x)=(x-a)g(x)$\\\\

	    %----------(c)
	    \item Demostrar que si $f$ es una función polinómica de grado $n$, entonces $f$ tiene a lo sumo $n$ raíces, es decir, existen a lo sumo $n$ números $a$ tales que $f(a)=0$\\\\
	    Demostración.- \; Supóngase que $f$ tiene $n$ raíces $a_1,...,a_n$. Entonces según la parte $(b)$ podemos poner $f(x)(x-a)g_1(x)$ donde el grado de $g_1(x)$ es $n-1$. Pero $$0=f(a_2)=(a_2-a_1)g_1(a_2)$$ de modo que $g_1(a_2)=0$, ya que $a_2 \neq a_1$. Podemos pues escribir $$f(x)(x-a_2)g_2(x),$$ donde el grado de $g_2$ es $n-2$. Prosiguiendo de esta manera, obtenemos que $$f(x)=(x-a_1)(x-a_2)\cdot ... \cdot (x-a_n)c$$ para algún número $c \neq 0.$ Está claro que $f(a)\neq 0$ si $a \neq a_1,...,a_n.$ Así pues, $f$ puede tener a lo sumo $n$ raíces.\\\\

	    %----------(d)
	    \item Demostrar que para todo $n$ existe una función polinómica de grado $n$ con raíces. Si $n$ es par, encontrar una función polinómica de grado $n$ sin raíces, y si $n$ es impar, encontrar una con una sola raíz\\\\
	    Demostración.- \; Si $f(x)=(x-1)(x-2)\cdot ... \cdot (x-n),$ entonces $f$ tiene $n$ raíces. Si $n$ es par, entonces $f(x)=x^n + 1$ no tiene raíces. Si $n$ es impar, entonces $f(x)=x^n$ tiene una raíz única, que es $0.$\\\\

	\end{enumerate}

	%------------------8.
	\item ¿Para qué números $a,b,c$ y $d$ la función 
	$$f(x)=\dfrac{ax+d}{cx+b}$$ 
	satisface $f(f(x))=x$ para todo $x$?\\\\
	Respuesta.-\; Si $$x=f(f(x))=\dfrac{a\left(\dfrac{ax+b}{cx+d}\right)+b}{c\left(\dfrac{ax+b}{cx+d}\right)+d}$$ para todo $x$, entonces $$x=\dfrac{a^2x+ab+bcx+bd}{acx+bc+cdx+d^2}$$ y por lo tanto $$\left(ac+cd\right)x^2 + \left(d^2-a^2\right)x - ab - bd = 0$$ para todo $x$, de modo que 
	\begin{center}
	    \begin{tabular}{rcl}
		$ac+cd$&=&$0$\\
		$ab+bd$&=&$0$\\
		$d^2-a^2$&=&$0$\\
	    \end{tabular}
	\end{center}
	Se sigue que $a=d$ ó $a=-d.$ Una posibilidad es $a=d=0$, en cuyo caso $f(x)=\dfrac{b}{cx}$ que satisface $f(f(x))=x$ para todo $x\neq 0$. Si $a=d\neq 0$, entonces $b=c=0$ con lo que $f(x)=x$. La tercera posibilidad es $a+d=0$, de modo que $f(x)=\dfrac{ax+b}{cx-a}$, la cual satisface $f(f(x))=x$ para todo $x\neq \dfrac{a}{c}$ la cual satisface $f(f(x))=x$ para todo $x\neq \dfrac{a}{c}$. Estrictamente hablando, podemos añadir la condición $f(x)\neq \dfrac{a}{c}$ para $x\neq \dfrac{a}{c}$, lo que significa que $$\dfrac{ax+b}{cx-a}\neq \dfrac{a}{c}, \; ó \; a^2+bc\neq 0.$$\\\\

	%--------------------9.
	\item 
	\begin{enumerate}[\bfseries (a)]

	    %----------(a)
	    \item Si $A$ es un conjunto cualquiera de números reales, defínase una función $C_A$ como sigue:  
		   $$C_A(x) = \left\lbrace
			\begin{array}{c l}
			    $1$,&si \; $x$ \; está \; en \;$A$\\
			    $0$,&si \; $x$\; no\; está \; en \;  $A$
			\end{array}
			    \right.$$
		    Encuéntrese expresiones para $C_{A\cap B}, \; C_{A \cup B}$ y $C_{\mathbb{R}-A}$, en términos de $C_A$ y $C_B$. \\\\
	    Respuesta.-\; Según la definición de teoría de conjunto tenemos,
	    \begin{center}
		\begin{tabular}{rcl}
		    $C_{A\cap B}$&$=$&$C_A \cdot C_B$\\
		    $C_{A\cup B}$&$=$&$C_A + C_B - C_A \cdot C_B$\\
		    $C_{\mathbb{R} - A}$&$=$&$1 - C_A$\\\\
		\end{tabular}
	    \end{center}
	    
	    %----------(b)
	    \item Supóngase que $f$ es una función tal que $f(x)=0$ o $1$ para todo $x$. Demostrar que existe un conjunto $A$ tal que $f=C_A$\\\\
	    Demostración.-\; Sea $A=\lbrace x\in \mathbb{R}: f(x)=1\rbrace$ , entonces $f=C_A$.\\\\

	    %----------(c)
	    \item Demostrar que $f=f^2$ si y sólo si $f=C_A$ para algún conjunto $A$\\\\
	    Demostración.-\; Sea $f=f^2$, entonces para cada real $x$, $f(x)=f[f(x)]^2$, así $f(x)=0$ó $f(x)=1$, luego por la parte $b)$, $f=C_A$ para algún $A$.\\
	    Por otro lado sea $f=C_A$ para algún $A$. Entonces si $x\in A$, $f(x)=1=1^2=f(x)^2$, mientras si $x\notin A,$ $f(x)=0=0^2=f(x)^2$, así en cualquier caso $f(x)=[f(x)]^2$ y $f=f^2$\\\\

	\end{enumerate}

	%--------------------10.
	\item 
	\begin{enumerate}[\bfseries (a)]

	    %----------(a)
	    \item ¿Para qué funciones $f$ existe una función $g$ tal que $f=g^2$?\\\\
	    Respuesta.-\; Debido a que algún número elevado al cuadrado siempre será no negativo podemos afirmar que las funciones $f$ satisfacen a todo $x$ tal que $f(x)\geq 0$\\\\

	    %----------(b)
	    \item ¿Para qué función $f$ existe una función $g$ tal que $f=1/g$?\\\\
	    Respuesta.-\; Dado a que un número divido entre cero es indeterminado se ve claramente que satisfacen a todo $x$ tal que $f(x)\neq 0$\\\\

	    %----------(c)
	    \item ¿Para qué funciones $b$ y $c$ podemos encontrar una función $x$ tal que $$(x(t))^2 + b(t)x(t)+c(t)=0$$ para todos los números $t$?\\\\
	    Respuesta.-\; Por teorema se observa que para las funciones $b$ \; y \; $c$ que satisfacen $(b(t))^2 - 4c(t) \geq 0$ para todo $t$\\\\ 

	    %----------(d)
	    \item ¿Qué condiciones deben satisfacer las funciones $a$ y $b$ si ha de existir una función $x$ tal que $$a(t)x(t)+b(t)=0$$ para todos los números $t$? ¿Cuántas funciones $x$ de éstas existirán?\\\\
	    Respuesta.-\; Es facil notar que $b(t)$ tiene que ser igual a $0$ siempre que $a(t) = 0$. Si $a(t) \neq 0$ para todo $t$, entonces existe una función única con esta condición, que es $x(t) = a(t)/b(t)$. Si $a(t)=0$ para algún $t$, entonces puede elegirse arbitrariamente $x(t)$, de modo que existen infinitas funciones que satisfacen la condición.\\\\

	\end{enumerate}

	%--------------------11.
	\item 

	\begin{enumerate}[\bfseries (a)]

	    %----------(a)
	    \item Supóngase que $H$ es una función $e$ y un número tal que $H(H(y))=y$. ¿Cuál es el valor de $$H(H(H...(H(y))))?$$\\\\
	    Respuesta.-\; Si aplicamos la hipótesis, tendremos que aplicar $78$ veces la función, luego $76$ y así, hasta llegar a $2$, donde la función sera $H(H(y))$, y una vez más por hipótesis tenemos como resultado $y$.\\\\

	    %----------(b)
	    \item La misma pregunta sustituyendo $80$ por $81$\\\\
	    Respuesta.-\; Sea $H(H(y))$ la $78ava$ vez de la función, entonces la $81ava$ vez será $H(H(H(y))$, por lo tanto queda como resultado $H(y)$.\\\\ 

	    %----------(c)
	    \item La misma pregunta si $H(H(y))=H(y)$\\\\
	    Respuesta.-\; Análogamente a la parte $a)$ si la $80ava$ vez es $y$ entonces por hipótesis nos queda $H(y)$.\\\\ 

	    %----------(d)
	    \item Encuéntrese una función $H$ tal que $H(H(x))=H(x)$ para todos los números $x$ y tal que $H(1)=36$, $H(2)=\dfrac{\pi}{3}$, $H(13)=47$, $H(36)36$, $H(\pi / 3)\dfrac{\pi}{3}$, $H(47)=47$\\\\
	    Respuesta.-\; Dar a $H(l)$, $H(2)$, $H(13)$, $H(36)$, $H(\pi /3)$, y $H(47)$ los valores especificados y hágase $H(x) = 0$ para $x \neq 1, 2, 13, 36, \pi /3, 47.$ Al ser, en particular, $H(0) = 0$, la condición $H(H(x)) = H(x)$ se cumple para todo $x$.\\\\

	    %----------(e)
	    \item Encontrar una función $H$ tal que $H(H(x))=H(x)$ para todo $x$ y tal que $H(1)=7$, $H(17)=18$\\\\
	    Respuesta.-\; Hágase $H(1) = 7$, $H(7) = 7$, $H(17) = 18$, $H(18) = 18$, y $H(x) = 0$ para $x \neq l , 7, 17, 18$.\\\\

	\end{enumerate}

	%--------------------12.
	\item Una función $f$ es par si $f(x)=f(-x),$ e impar si $f(x)=-f(-x)$. Por ejemplo, $f$ es par si $f(x)=x^2$ ó $f(x)=|x|$ ó $f(x)=\cos x$, mientras que $f$ es impar si $f(x)=x$ ó $f(x)=\sen x$.

	\begin{enumerate}[\bfseries (a)]

	    %----------(a)
	    \item Determinar si $f+g$ es par, impar o no necesariamente ninguna de las dos cosas, en los cuatro casos obtenidos al tomar $f$ par o impar y $g$ par o impar. (Las soluciones pueden ser convenientemente dispuestas en una tabla $2$ x $2$)\\\\
	    Respuesta.-\; Sea $f(x)=x^2$ y $g(x)=|x|$ entonces $f(-x)+g(-x)=(-x)^2 + |-x| = x^2 + |x| = f(x) + g(x)$ por lo tanto par y par es par.\\
	    Sea $f(x) = x$ y $g(x)=x$ entonces $-f(-x) + (-g(-x)) = -(-x) + [-x(-x)] = x + x = f(x) + g(x)$, por lo tanto impar e impar es impar.\\
	    Los otros dos últimos se prueba fácilmente y se llega a la conclusión de que ni uno ni lo otro.
	    \begin{center}
		\begin{tabular}{c|cc}
		    &\textbf{Par}&\textbf{Par}\\
		    \hline\\
		    \textbf{Par}&Par&Ninguno\\\\
		    \textbf{Par}&Ninguno&Par\\
		\end{tabular}
	    \end{center}
	    \vspace{1cm}

	    %----------(b)
	    \item Hágase lo mismo para $f\cdot g$\\\\
	    Respuesta.-\; Sea $f(x)=x^2$ \; y \; $g(x)=|x|$, entonces $f(-x) \cdot g(-x) = x^2 \cdot |x| = f(x) \cdot g(x)$, por lo tanto se cumple para par y par.\\
	    Sea $f(x)=x$ \; y \; $g(x)=x$, entonces $-f(-x) \cdot -g(-x) = -(-x) \cdot -(-x) = x \cdot x = f(x) \cdot g(x)$, por lo tanto impar impar da impar\\
	    Sea $f(x)=x^2$ \; y \; $g(x)=x$, podemos crear otra función llamada $h$ que contiene a $x^2 \cdot x$ por lo tanto $h(x)=x^3 = -(-x)^2$ y así demostramos que par e impar es impar.\\
	    De igual forma al anterior se puede probar que impar y par es impar.
	    \begin{center}
		\begin{tabular}{c|cc}
		    &\textbf{Par}&\textbf{Par}\\
		    \hline\\
		    \textbf{Par}&Par&Impar\\\\
		    \textbf{Par}&Impar&Par\\\\
		\end{tabular}
	    \end{center}
	    \vspace{1cm}

	    %----------(c)
	    \item Hágase lo mismo para $f\circ g$\\\\
	    Respuesta.-\; Sea $f(x)=x$ \; y \; $g(x)=x$, luego $h(x)=(f \circ g)(x)$ entonces $h(x) = x$ luego $-f(-x) = x$, por lo tanto impar e impar da impar.\\
	    De similar manera se puede encontrar para los demás problemas y queda:
	    \begin{center}
		\begin{tabular}{c|cc}
		    &\textbf{Par}&\textbf{Par}\\
		    \hline\\
		    \textbf{Par}&Par&Par\\\\
		    \textbf{Par}&Par&Impar\\\\
		\end{tabular}
	    \end{center}
	    \vspace{1cm}

	    %----------(d)
	    \item Demostrar que para toda función par $f$ puede escribirse $f(x)=g(|x|)$, para una infinidad de funciones $g$.\\\\
	    Demostración.-\; Sea $g(x)=f(x)$ sabemos que $f$ es par si $f(x)=f(-x)$, de donde $g(x)=f(-x)$, luego  por definición de valor absoluto se tiene $g(|x|)=f(|-x|)$, y por lo tanto $f(x)=g(|x|)$\\\\

	\end{enumerate}

	%--------------------13.
	\item 
	\begin{enumerate}[\bfseries (a)]

	    %----------(a)
	    \item Demostrar que para toda función $f$ con dominio $\mathbf{R}$ puede ser puesta en la forma $f=E+O,$ con $E$ par y $O$ impar.\\\\
	    Demostración.-\; Por la parte $(b)$ y resolviendo en $E(x)$ y $O(x)$ se tiene $$E(x)\dfrac{f(x)+f(-x)}{2}, \;\;\; O(x)\dfrac{f(x)-f(-x)}{2}$$\\\\

	    %----------(b)
	    \item Demuéstrese que esta manera de expresar $f$ es única. (Si se intenta resolver primero la parte $(b)$ despejando $E$ y $O$, se encontrará probablemente la solución a la parte $(a)$)\\\\
	    Demostración.-\; Si $f=E+O$, siendo $E$ par y $O$ impar, entonces $$f(x)=E(x)+O(x)$$ $$f(-x)=E(x)-O(x)$$\\\\

	\end{enumerate}

	%--------------------14.
	\item Si $f$ es una función cualquiera, definir una nueva función $|f|$ mediante $|f|(x)=|f(x)|$. Si $f$ y $g$ son funciones, definir dos nuevas funciones, $max(f,g)$ y $min(f,g)$ mediante $$max(f,g)(x)=max(f(x),g(x)),$$ $$min(f,g)(x)=min(f(x),g(x))$$ Encontrar una expresión para $max(f,g)$ y $min(f,g)$ en términos de $| |$.\\\\
	Respuesta.-\; Por problema $1.13$ se tiene que $$max(f,g)=\dfrac{f+g+|f-g|}{2};$$ $$min(f,g)=\dfrac{f+g-|f-g|}{2}$$\\\\

	%--------------------15.
	\item 

	\begin{enumerate}[\bfseries (a)]
	    
	    %----------(a)
	    \item Demostrar que $f=max(f,0)+min(f,0)$. Esta manera particular de escribir $f$ es bastante usada; las funciones $max(f,0)$ y $min(f,0)$ se llaman respectivamente parte positiva y parte negativa de $f$\\\\
	    Demostración.-\;

	    %----------(b)
	    \item Una función $f$ se dice que es no negativa si $f(x)\geq 0$ para todo $x$. Demostrar que para cualquier función $f$ puede ponerse $f=g-h$ de infinitas maneras con $g$ y $h$ no negativas. (La manera corriente es $g=max(f,0)$ y $h=-min(f,0)$. Cualquier número puede ciertamente expresarse de infinitas maneras como diferencia de dos números no negativos.)\\\\
	    Demostración.-\;

	\end{enumerate}

    \end{enumerate}
