\chapter*{Introducción}

\setcounter{chapter}{1}
\setcounter{section}{2}
\section{El método de exhaución para el área de un segmento de parábola}
El método consiste simplemente en lo siguiente: se divide la figura en un cierto número de bandas y se obtienen dos aproximaciones de la región, una por defecto y otra por exceso, utilizando dos conjuntos de rectángulos.\\
Se subdivide la base en $n$ partes iguales, cada una de longitud $b/n$. Los puntso de subdivisión corresponden a los siguientes valores de $x$: $$0,\dfrac{b}{n},\dfrac{2b}{n},\dfrac{3b}{n},...,\dfrac{(n-1)b}{n},\dfrac{nb}{n}=b$$
La expresión general de un punto de la subdivisión es $x=\frac{kb}{n},$ donde $k$ toma los valores sucesivos $k=0,01,2,3,...,n.$ En cada punto $\frac{kb}{n}$ se construye el rectángulo exterior de altura $(kb/n)^2$. El área de este rectángulo es el producto de la base por la altura y es igual a:
$$\left(\dfrac{b}{n}\right)\left(\dfrac{kb}{n}\right)^2=\dfrac{b^3}{3}k^3$$
Si se designa por $S_n$ la suma de las áreas de todos os rectángulos exteriores, puesto que el área del rectángulo k-simo es $(b^3/n^3)k^3$ se tiene la formula. $$S_n=\dfrac{b^3}{n^3}(1^2+2^2+3^2+...+n^2) \qquad (1)$$
De forma análoga se obtiene la fórmula para la suma $s_n$ de todos los rectángulos interiores:
$$s_n=\dfrac{b^3}{n^3}[1^2+2^2+3^2+...+(n-1)^2] \qquad (2)$$
Luego se tiene la identidad $$1^2+2^2+3^2+...+n^2=\dfrac{n^3}{3}+\dfrac{n^2}{2}+\dfrac{6}{n} \qquad (3)$$
Como también $$1^2+2^2+...+(n-1)^2=\dfrac{n^2}{3}-\dfrac{n^2}{2}+\dfrac{n}{6} \qquad (4)$$
Las expresiones exactas dadas no son necesarias para el objeto que aquí se persigue, pero sirven para deducir fácilmente las dos desigualdades que interesan $$1^2+2^2+3^2+...+(n-1)^2<\dfrac{n^3}{3}<1^2+2^2+...+n^2$$
que son válidas para todo entero $n\geq 1$. Multiplicando ambas desigualdades por $b^3/n^3$ y haciendo uso de $(1)$ y $(2)$ se tiene: $$s_n<\dfrac{b^3}{3}<S_n$$
Probemos que $b^3/3$ es el único número que goza de esta propiedad, es decir, que si $A$ es un número que verifica las desigualdades $$s_n<A<S_n \qquad (7)$$
para cada entero positivo $n$, ha de ser necesariamente $A=b^3/3$. Por esta razón dedujo Arquímedes que el área del segmento parabólico es $b^3/3$.\\
Para probar que $A=b^3/3$ se utilizan una vez más las desigualdades $(5)$. Sumando $n^2$ a los dos miembros de la desigualdad de la izquierda en $(5)$ se obtiene $$1^2+2^2+3^2+...+n^2=\dfrac{n^3}{3}+n^2$$
Multiplicando por $b^3/3$ y utilizando $(1)$ se tiene $$S_n<\dfrac{b^3}{3}+\dfrac{b^3}{n} \qquad (8)$$
Análogamente, restando $n^2$ de los dos miembros de la desigualdad de la derecha en $(5)$ y multiplicando por $b^3/n^3$ se llega a la desigualdad:
$$\dfrac{b^3}{3}-\dfrac{b^3}{n}<s_n \qquad (9)$$
Por tanto,cada número $A$ que satisfaga $(7)$ ha de satisfacer también: $$\dfrac{b^3}{3}-\dfrac{b^3}{n}<A<\dfrac{b^3}{3}+\dfrac{b^3}{n} \qquad (10)$$ para cada entero $n\geq 1$. Ahora también, hay sólo tres posibilidad: $$A>\dfrac{b^3}{3}, \quad A<\dfrac{b^3}{3} \quad A=\dfrac{b^3}{3}$$
 Si se prueba que las dos primeras conducen a una contradicción habrá de ser $A=\frac{b^3}{3}$\\
 Supongamos que la desigualdad $A>b^3/3$ fuera cierta. De la segunda desigualdad en $(10)$ se obtiene $$A-\dfrac{b^3}{3}<\dfrac{b^3}{n} \qquad (11)$$ para cada entero $n\geq 1$. Puesto que $A-b^3/3$ es positivo, se puede dividir ambos miembros de $(11)$ por $A-b^3/3$ y multiplicando después por $n$ se obtiene la desigualdad $$n<\dfrac{b^3}{A-b^3/3}$$ para cada $n$. Pero esta desigualdad es evidentemente para $n>b^3/(A-b^3/3).$ Por tanto la desigualdad es una contradicción. De forma análoga se demuestra para $A<b^3/3$ de donde concluimos que $A=b^3/3$.\\

\addcontentsline{toc}{chapter}{Introducción}
\setcounter{chapter}{3}
\setcounter{section}{1}
\section{Axiomas de cuerpo}
%axioma1
\begin{tcolorbox}[colframe=white]
\begin{axioma}[Propiedad conmutativa] $x+y=y+x, \; xy=yx$\\
\end{axioma}

%aximoa2
\begin{axioma}[Propiedad asociativa] $x+(y+z)=(x+y)+z, \; x(yz)=(xy)z$ \\
\end{axioma}

%axioma3
\begin{axioma}[Propiedad distributiva] $x(y+z)=xy+xz$ \\
\end{axioma}

%axioma4
\begin{axioma}[Existencia de elementos neutros] Existen dos números reales distintos que se indican por $0$ y $1$ tales que para cada número real $x$ se tiene:
$0+x=x+0=x \;$ y $1\cdot x = x\cdot 1 = 1$ \\
\end{axioma}

%axioma5
\begin{axioma}[Existencia de negativos] Para cada número real $x$ existe un número real y tal que $x+y=y+x=0$ \\
\end{axioma}

%axioma6
\begin{axioma}[Existena del recíproco] Para cada número real $x\neq 0$ existe un número real y tal que $xy=yx=1$ \\
\end{axioma}
\end{tcolorbox}

%teorema 1.1
\begin{teo}[Ley de simplificación para la suma]
Si $a+b=a+c$ entonces $b=c$ (En particular esto prueba que el número 0 del axioma 4 es único)\\\\
Demostración.- \;
Dado a+b=a+c. En virtud de la existencia de negativos, se puede elegir y de manera que $y+a=0$, con lo cual $y+(a+b)=y+(a+c)$ y aplicando la propiedad asociativa tenemos $(y+a)+b=(y+a)+c$ entonces, $0+b=0+c$. En virtud de la existencia de elementos neutros, se tiene $b=c$.\\
por otro lado este teorema demuestra que existe un solo número real que tiene la propiedad del 0 en el axioma 4. En efecto, si $0$ y $0^{'}$ tuvieran ambos esta propiedad, entonces $0+0^{'}=0$ y $0+0=0$; por lo tanto, $0+0^{'}=0+0$ y por la ley de simplificación para la suma $0=0^{'}$\\\\
\end{teo}

%teorema 1.2
\begin{teo}[Posibilidad de la sustracción]
Dado $a$ y $b$ existe uno y sólo un $x$ tal que $a+x=b$. Este $x$ se designa por $b-a$. En particular $0-a$ se escribe simplemente $-a$ y se denomina el negativo de $a$\\\\
Demostración.- \;
Dados $a$ y $b$ por el axioma 5 se tiene $y$ de manera que $a + y = 0$ ó $y=-a$, por hipótesis y teorema tenemos que $x=b-a$ sustituyendo $y$ tenemos $x=b+y$ y propiedad conmutativa $x=y+b$, entonces $a+x=a+(y+b)=(a+y)+b=0+b=b$ esto por sustitución, propiedad asociativa y propiedad de neutro, Por lo tanto hay por lo menos un $x$ tal que $a+x=b$. Pero en virtud del teorema 1.1, hay a lo sumo una. Luego hay una y sólo una $x$ en estas condiciones.\\\\ 
\end{teo}

%teorema 1.3
\begin{teo}
$b-a=b+(-a)$\\\\
Demostración.- \; Sea $x=b-a$ y sea $y=b+(-a)$. Se probará que $x=y$. por definición de $b-a$, $x+a=b$ y $y+a=\left[ b+(-a)\right]+a=b+\left[ (-a)+a \right]=b+0=b$, por lo tanto, $x+a=y+a$ y en virtud de teorema 1.1 $x=y$\\\\
\end{teo}

%teorema 1.4
\begin{teo}
$-(-a)=a$\\\\
Demostración.- \;
Se tiene $a+(-a)=0$ por definición de $-a$ incluido en el teorema 1.1. Pero esta igualdad dice que $a$ es el opuesto de $-a$, es decir, que si $a+(-a)=0$ entonces $a=0-(-a)=a=-(-a)$\\\\
\end{teo}

\section{Ejercicios}
\begin{enumerate}[\bfseries \Large 1.]

%----------------------------------1--------------------------
\item Demostrar los teoremas del 1.5 al 1.15, utilizando los axiomas 1 al 6 y los teoremas I.1 al I.4\\\\

%teorema 1.5
\begin{teo}
$a(b-c)=ab-ac$\\\\
Demostración.- \; Sea $a(b-c)$ por teorema tenemos que $a\left[ b+(-c)\right]$  y por la propiedad distributiva $\left[ ab + a(-c) \right]$, y en virtud de anteriores teoremas nos queda $ ab - ac $ \\\\
\end{teo}

%teorema 1.6
\begin{teo}
$0\cdot a = a\cdot 0 =0$\\\\
Demostración.- \;
Sea $0\cdot a$ por la propiedad conmutativa $a\cdot 0$, $a\cdot 0 + 0$ y $a\cdot 0 + \left[a+(-a) \right]$ y en virtud la propiedad asociativa y distributiva $a(0 + 1)+(-a)$ después $1(a)+(-a)$, luego por elemento neutro y existencia de negativos tenemos $0$, Así queda demostrado que cualquier número multiplicado por cero es cero.\\\\ 
\end{teo}

%teorema 1.7
\begin{teo}[Ley de simplificación para la multiplicación] Si $ab=ac$ y $a \neq 0$, entonces $b=c$. (En particular esto demuestra que el número 1 del axioma 4 es único)\\\\
Demostración.- \;
Sea $b$, $a\neq 0$, y por el existencia del recíproco tenemos $a\cdot a^{'}=1 $  luego,  $b=b\cdot 1=b\left[a(a^{'})\right]=(ab)(a^{'})=(ac)(a^{'})=c(a\cdot a^{'})=c\cdot 1=c$ por lo tanto queda demostrado la ley de simplificación.\\\\
\end{teo}

%teorema 1.8
\begin{teo}[Posibilidad de la división] Dados $a$ y $b$ con $a\neq 0$, existe uno y sólo un $x$ tal que $ax=b$. La $x$ se designa por $b/a$ ó $\displaystyle\frac{b}{a}$ y se denomina cociente de $b$ y $a$. En particular $1/a$ se escribe también $a^{-1}$ y se designa recíproco de $a$\\\\
Demostración .- \;
Sea $a$ y $b$ por axióma 6 se tiene un $y$ de manera que $a\cdot y = 1$ ó $y = a^{-1}$. Por hipótesis y teorema se tiene $x=b\cdot a^{-1}$, sustituyendo tenemos $x=y\cdot b$ entonces $ax=a(y\cdot b)=(a\cdot y)b=1\cdot b = b$  por lo tanto hay por lo menos un $x$ tal que $ax=b$ pero en virtud del teorema 1.7 hay por lo mucho uno, luego hay una y sólo una $x$ en estas condiciones.\\\\ 
\end{teo}

%teorema 1.9
\begin{teo}
Si $a\neq 0$, entonces $b/a=b\cdot a^{-1}$\\\\
Demostración.- \;
Sea $x =b/a$ y sea $y=b\cdot a^{-1}$ se probará que $x=y$, por definición de $b/a$, $ax=b$ y $ya=(b\cdot a^{-1})a=b(a^{-1}a)=b\cdot 1 = 1$, entonces $ya=xa$ y por la ley de simplificación para la multiplicación $y=x$ \\\\
\end{teo}

%teorema 1.10
\begin{teo}
Si $a\neq 0$, entonces $(a^{-1})^{-1}=a$\\\\
Demostración.- \;
Si $a\neq 0$ entonces  $(a^{-1})^{-1} = 1\cdot (a^{-1})^{-1} = \displaystyle\frac{1}{a^{-1}}=a$ esto por axioma de neutro, definición de $a^{-1}$ y teorema 1.9, así concluimos que $(a^{-1})^{-1}=a $ \\\\
\end{teo}

%teorema 1.11
\begin{teo}
Si $ab = 0$, entonces ó $a=0$ ó $b=0$\\\\
Demostración.- \;
Veamos dos casos, cuando $x\neq 0$ y cuando $x=0$\\
Si $x\neq 0$ y  $ab = 0$ entonces  $b=b\cdot 1 = b (a\cdot a^{-1}) = (ab)a^{-1}=0a^{-1}=0$, ahora si $a=0$ y virtud del teorema 1.6 nos queda demostrado que la multiplicación de dos números cualesquiera es igual a cero si $a=0$ ó $b=0$  \\\\
\end{teo}

%teorema 1.12
\begin{teo}
$(-a)b=-(ab) \; y \; (-a)(-b) = ab$\\\\
Demostración.- \;
Empecemos demostrando la primera proposición, Por la ley de simplificación para la suma podemos escribir como $(-a)b+ab=0$ entonces por la propiedad distributiva $b\left[ (-a)+a \right]$ por lo tanto $b\cdot 0$, luego por el teorema 1.6 queda demostrado la primera proposición.\\
Para demostrar la segunda proposición acudimos a la primera proposición, $(-a)(-b)=-\left[ a(-b)\right]$ y luego, \\ $-\left[ a(-b)+b+(-b)\right]=-\left[ (-b)(a+1)+b\right]=-\left[ (-b)(a+1)-1(-b)\right]=-\left[( -b)(a+1-1)\\\right]=-\left[ (-b)a\right]=-\left[ -(ab)\right]$ y en virtud  del el teorema 1.4 $(-a)(-b)=ab$ así queda demostrado la proposición.  \\\\
\end{teo}

%teorema 1.13
\begin{teo}
$\left(a/b \right) + \left(c/d \right) = \left( ad+bc  \right) / \left( bd \right) $ si $b\neq 0$ y $d\neq 0$.\\\\
Demostración.- \;
Si $\left(a/b \right) + \left(c/d \right)$ entonces por definición de $a/b$, $a\cdot b^{-1}+c\cdot d^{-1}=(a\cdot b^{-1})\cdot 1+(c\cdot d^{-1})\cdot 1=(a\cdot b^{-1})\cdot 1+(c\cdot d^{-1})\cdot 1=(a\cdot b^{-1})\cdot dd^{-1}+(c\cdot d^{-1})\cdot bb^{-1}$ por las propiedades asociativa  conmutativa y distributiva, $(b^{-1}d^{-1})(ad)+(b^{-1}d^{-1})(cb)=(b^{-1}d^{-1})(ad+cb)$, por lo tanto $(ad+bc)/bd$ esto por definición.\\\\
\end{teo}

%teorema 1.14
\begin{teo}
$(a/b)(c/d)=(ac)/(bd)$ si $b\neq 0$ y $d\neq 0$\\\\
Demostración.- \;
Por definición, $(ab^{-1})(cd^{-1})$, propiedades conmutativa y asociativa $(ac)(b^{-1}d^{-1})$, y por definición queda demostrado la proposición.\\\\
\end{teo}

%corolario 1.1
\begin{col.}
Si $c\neq 0$ y $d\neq 0$ entonces $(cd^{-1})=c^{-1}d$\\\\
Demostración.- \;
Por definición de $a^{-1}$ tenemos que $(cd^{-1})^{-1}=\displaystyle\frac{1}{cd^{-1}}$, por el teorema de posibilidad de la división $1=(c^{-1}d)(cd^{-1})$ y en virtud de los axiomas de conmutatividad y asociatividad $1=(c^{1}c)(dd^{-1})$, luego $1=1$. quedando demostrado el corolario.\\\\
\end{col.}

%teorema 1.15 
\begin{teo}
$(a/b)/(c/d)=(ad)/(bc)$ si $b\neq 0$,  $c\neq 0$ y $d\neq 0$\\\\
Demostración.- \;
Sea $(a/b)/(c/d)$ entonces por definición $(ab^{-1})(cd^{-1})^{-1}$, en virtud del corolario 1 se tiene que $(ab^{-1})(c^{-1}d)$, y luego por axioma conmutativa y asociativa $(ad)(c^{-1}b^{-1})$, así por definición concluimos que $(ad)/(cd)$\\\\
\end{teo}

%-------------------------------2--------------------------------------
\item $-0=0$\\\\
Demostración.- \;
Sabemos que por el axioma 5 $a+(-a)=0$, $- \left[ a+(-a) \right] = 0$ y $-a+-(-a)=0$ en virtud de teorema 1.12 y propiedad conmutativa $a+(-a)=0$, por lo tanto $0=0$.\\\\ 


%---------------------------------3-------------------------------
\item 
$1^{-1}=1$\\\\
Demostración.- \;
Por la existencia de elementos nuestros tenemos $1^{-1}\cdot 1$ y por axioma de existencia de reciproco $1=1$\\\\

%-------------------------------4-----------------------------
\item El cero no tiene reciproco\\\\
Demostración.- \;
Supongamos que el cero tiene reciproco es decir $0\cdot 0^{-1}=1$ pero por el teorema 1.6 se tiene que $0\cdot 0^{-1}=0$ y $0=1$ esto no es verdad, por lo tanto el cero no tiene reciproco.\\\\

%-----------------------------------5---------------------------------
\item $-(a+b)=-a-b$\\\\
Demostración.- \;
Por existencia de reciproco $-\left[1(a+b)\right]$ y teorema 1.12 $(-1)(a+b)$ luego por la propiedad distributiva $\left[ (-1)b \right] + \left[ (-1)b \right]$, una vez mas por el teorema 1.12 $-(1a)+ \left[-(1b)\right]$, en virtud del axioma 4 $-a+(-b)$, y teorema 1.3, $-a-b$ \\\\ 

%------------------------------------6----------------------------
\item $-(-a-b)=a+b$\\\\
Demostración\\\\
Si $-(-a-b)$ entonces por axioma $-\left[1(-a-b)\right]$, luego $(-1)(-a-b)=(-1)(-a)-\left[(-1)b\right])= (1\cdot a)-\left[-(1\cdot b)\right]$ y por axioma $a - \left[ - (b)\right]$, así por teorema $a+b$.\\\\

%------------------------------------7--------------------------
\item $(a-b)+(b-c)=a-c$\\\\
Demostración.- \;
Por definición tenemos $\left[ a+(-b) \right]+\left[ b+(-c) \right]$, y axiomas de asociatividad y conmutatividad $\left[ a+(-c) \right]+\left[ b+(-b) \right]$, luego por existencia de negativos  $\left[ a+(-c) \right] + 0$,, así $a+(-c)$ y $a-c$. \\\\

%-----------------------------------8--------------------------------
\item Si $a\neq 0$ y $b\neq 0$, entonces $(ab)^{-1} = a^{-1} b^{-1}$\\\\
Demostración.- \; Por hipótesis $\dfrac{1}{a} \dfrac{1}{b}$ luego $\dfrac{1}{ab}$ por lo tanto $(ab)^{-1}$\\\\

%------------------------------------9--------------------------------
\item $-(a/b)=(-a/b)=a/(-b)$ si $b\neq 0$\\\\
Demostración.- \;
Primero demostremos que $-(a/b)=(-a/b)$, Sea $b\neq 0$, en virtud de definición de la división y teorema 1.12 no queda que $-(a/b)=(-a)\cdot b^{-1}=-a/b$.\\
Ahora demostramos que $-(a/b)=(-a/b)=a/(-b)$, sea $b\neq0$, luego $-(b^{-1}\cdot a)=\left[-(b^-1)\right]\cdot a = a/-b$. \\\\

%------------------------------------10---------------------------------
\item $(a/b)-(c/d)=(ad-bc)/(bd)$ si $b\neq 0$ y $d\neq 0$\\\\
Demostración.- \;
Sea $b\neq 0$ y $d\neq 0$ y por definición $ab^{-1}-cd^{-1}$, luego por axiomas $(ab^{-1})(d\cdot d^{-1})-(cd^{-1})(b\cdot b^{-1})$, y en virtud del teorema 1.5 y propiedad asociativa $b^{-1}\cdot d^{-1}(ad-bc)$ y $(ad-bc)/bd$\\\\
\end{enumerate}

\section{Axiomas de orden}
\begin{tcolorbox}[colframe=white]
\begin{axioma}Si $x$ e $y$ pertenecen a $\mathbb{R}^+$, lo mismo ocurre a $x+y$ y $xy$\\
\end{axioma}
\begin{axioma}
Para todo real $x\neq 0$, ó $x \in \mathbb{R}^+$ ó $-x \in \mathbb{R}^+$, pero no ambos.\\
\end{axioma}
\begin{axioma}
$0 \not\subset \mathbb{R}^+$\\
\end{axioma}
\end{tcolorbox}

\begin{tcolorbox}[colframe=white]
\begin{def.}
$x<y $ significa que $y-x$ es positivo. \\
\end{def.}
\begin{def.}
$y>x$ significa que $x<y$\\
\end{def.}
\begin{def.}
$x \geq y$ significa que ó $x<y$ ó $x=y$\\
\end{def.}
\begin{def.}
$y \leq x$ significa que $x \leq y$ 
\end{def.}
\end{tcolorbox}

\section{Ejercicios}
\begin{enumerate}[\bfseries \Large 1.]
%-------------------------1----------------------------
\item Demostrar los teoremas 1.22 al 1.25 utilizando los teoremas anteriores y los axiomas del 1 al 9
%teorema 1.16
\begin{teo}[Propiedad de Tricotomía]
Para $a$ y $b$ números reales cualesquiera se verifica se verifica una y sólo una de las tres relaciones $a<b$, $b<a$, $a=b$\\\\
demostración.- \;
Sea $x=b-a$. Si $x=0$, entonces $x=a-b=b-a$, por axioma 9, $0\notin \mathbb{R}^+$ es decir:
$$a<b, \; \; b-a \in \mathbb{R}^+$$
$$b<a, \; \;  a-b \in \mathbb{R}^+$$
pero como $a-b=b-a=0$ entonces no ser $a<b$ ni $b<a$\\
Si $x0\neq 0,$ el axioma 8 afirma que ó $x>0$ ó $ x<0$, pero no ambos, por consiguiente, ó es $a<b$ ó es $b<a$, pero no ambos. Pro tanto se verifica una y sólo una de las tres relaciones $a=b,$ $a<b,$ $b<a$.\\\\
\end{teo}

%teorema 1.17
\begin{teo}[Propiedad Transitiva]
Si $a<b$ y $b<c$, es $a<c$\\\\
Demostración.- \;
Si $a<b$ y $b<c$, entonces por definición $b-a>0$ y $c-b>0$. En virtud de axioma  $(b-a)+(c-b)>0$, es decir, $c-a>0,$ y por lo tanto $a<c$.\\\\
\end{teo}

%teorema 1.18
\begin{teo}
Si $a<b$ es $a+c<b+c$\\\\
Demostración.- \;
Sea $x=a+c$
\end{teo}

%teorema 1.19
\begin{teo}
Si $a<b$ y $c>0$ es $ac<bc$\\\\
Demostración.- \;
Si $a<b$ por definición $b-a>0$, dado que  $c>0$ y por el axioma  $(b-a)c>0$ y $bc-ac>0$, por lo tanto $ac<bc$\\\\
\end{teo}

%teorema 1.20
\begin{teo}
Si $a\neq0$ es $a^2>0$\\\\
Demostremos por casos..- \;
Si $a>0$, entonces por axioma   \; $a\cdot a >0$ \; y \; $a^2>0$. Si $a<0$, entonces por axioma   \; $(-a)(-a)>0$ \; y  \; $a^2>0$\\\\
\end{teo}

%teorema 1.21
\begin{teo}
1>0\\\\
Demostración.- \;
Por el anterior teorema, si $1>0$ ó $1<0$ entonces $1^2>0$, y $1^2=1$, por lo tanto que $1>0$\\\\
\end{teo}

%teorema 1.22
\begin{teo}
Si $a<b$ y $c<0$, es $ac>bc$\\\\
Demostración.- \;
Si $c<0$, por definición $-c>0$, en virtud del axioma  $-c(b-a)>0$, y $ac-cb>0$, por lo tanto $ab<ac=ac>bc$\\\\
\end{teo}

%teorema 1.23
\begin{teo}
Si $a<b$, es $-a>-b$. En particular si $a<0$, es $-a>0$\\\\
Demostración.- \;
Si $1>0$, por la existencia de negativos $-1<0$ y por teorema  tenemos que $-1a>-1b$ por lo tanto $-a>-b$ \\\\
\end{teo}

%teorema 1.24
\begin{teo}
Si $ab>0$ entonces $a$ y $b$ son o ambos positivos o ambos negativos\\\\
Demostración.- \;
Sea $a>0$ y $b>0$, por axioma  \;  $ab>0$, y sea $a<0$ y $b<0$, por definición $-a>0$ y $-b>0$, por lo tanto $(-a)(-b)>0$ y por teorema 1.12 \; $ab>0$.\\\\
\end{teo}

%teorema 1.25
\begin{teo}
Si $a<c$ y $b<d,$ entonces $a+b<c+d$\\\\
Demostración.- \;
Si $a<c$ \; y \; $b<d$ por definición $c-a>0$ \; y \; $d-b>0$, en virtud del axioma 6:
$$(c-a)+(d-b)>0 \Rightarrow c-a+d-b>0 \Rightarrow (c+d)-(a+b)>0$$ 
por lo tanto $a+b<c+d$.\\\\ 
\end{teo}

%---------------------------------2---------------------------
\item No existe ningún número real tal que $x^2+1=0$\\\\
Demostración.- \; Sea $Y=x^2+1=0$ de acuerdo con la propiedad de tricotomía:
\begin{itemize}
\item Si $x>0$ entonces por teorema 2.5 \; $x^2>0$ y por axioma 7 \; $x^2+1>0$ esto es $Y > 0$ y no satisface $Y=0$ para $x>0$.
\item Si $x=0$ entonces $x^2=0$ y $x^2+1=1$ esto es $Y=1$ pero no satisface a $Y=1$ para $x=0$.
\item Si $x<0$ entonces $-x>0$ y $x^2+1>0$, esto es $Y=0$ pero tampoco satisface a $y=0$ para $x<0$.\\\\ 
\end{itemize}

%------------------------------3--------------------------
\item La suma de dos números negativos es un número negativo.\\\\
Demostración.- \;   Si $a<0$ y $b<0$ entonces $-a>0$ y $-b>0$ por axioma 7 \; $(-a)+(-b)>0$ y en virtud del teorema 1.19 \; $-(a+b)>0$ es decir $a+b<0$\\

%-------------------------------4--------------------------
\item Si $a>0$, también $1/a>0;$ Si $a<0$ entonces $1/a<0$\\\\
Demostración.- \;
\begin{itemize}
\item Si $a>0$ entonces $(2a)^{-1}\cdot a > 0 \cdot (2a)^{-1}$ por lo tanto $1/a>0$
\item Si $a<0$ entonces $-a>0$ y $(-2a)^{-1}\cdot (-a)>0\cdot (-2a)^{-1}$ por lo tanto $1/a>0$ y $-1/a<0$\\\\
\end{itemize}

%-------------------------------5--------------------------
\item 
Si $0<a<b,$ entonces, $0<b^{-1}<a^{-1}$\\\\
Demostración.- \; Si $b>0$ entonces por el teorema anterior  $b^{-1}>0$ ó $0<b^{-1}$.\\
Si $a>0$ entonces $a^{-1}>0$, dado que $a<b$ y por teorema  \; $a\cdot a^{-1}< a^{-1}b$ así $1<a^{-1}b$, luego $b{-1}<a^{-1}\cdot bb^{-1}$, por lo tanto $b^{-1}<a^{-1}$. Y por la propiedad transitiva queda demostrado que $0<b^{-1}<a^{-1}$.\\\\

%--------------------------6----------------------------
\item Si $a \leq b $ y $b \leq c$ es $a\leq c$\\\\
Demostración.- \; Si $a<b$ ó $a=b$ y $b<c$ ó $b=c$ demostremos por casos: Si $a<b$ y $b<c$ por la propiedad transitiva $a<c$, después si $a<b$ y $b=c$ entonces $a<c$, luego si $a=b$ y $b<c$ entonces $a<c$,  por último si $a=b$ y $b=c$ entonce $a=c$, por lo tanto $a\leq c$\\\\

%corolario 
\begin{col.}
Si $c\leq b$ y $b \leq c$ entonces $c=b$\\\\
Demostración .- \; Si $b-c>0$ y $c-b>0$ entonces $(b-c)+(c-b)>0$ y $0<0$ es Falso, entonces queda que $c=b$ (Usted puede comprobar para cada uno de los casos que se suscita parecido al teorema anterior.)\\\\
\end{col.}

%---------------------------7----------------------------
\item Si $a\leq b$ y $b \leq c$ y $a=c$ entonces $b=c$\\\\
Demostración.- \; Si $a\leq b$\;  y \; $a=c$ entonces $c\leq b$. Sea $c\leq b$ \; y \; $b\leq c$ y por corolario anterior \; $b=c$.\\\\

%----------------------------8-------------------------
\item Para números reales $a$ y $b$ cualquiera, se tiene $a^2+b^2\leq 0$. Si $ab\geq 0$, entonces es $a^2+b^2>0$.\\\\
Demostración.- \; Si $ab>0$ por teorema  ($a>0$ y $b>0$ ) ó ($a<0$ y $b<0$) luego por teorema  $a^2>0$ y $b^2>0$ por lo tanto por axioma 7 y ley de tricotomía $a^2+b^2>0$.\\\\ 

%-------------------------9------------------------------
\item No existe ningún número real $a$ tal que $x\leq a$ para todo real $x$\\\\
Demostración.- \; Supongamos que existe un número real $" a "$ tal que $y \leq a$. Sea $n\in \mathbb{R}$ \; y \; $x=y+n$ entonces por teorema 1.18 \; $y+n \leq a+n$ \; y \; $x\leq a+n$ esto contradice que existe un número real $a$ tal que $y\leq a$, por lo tanto no existe ningún número real tal que para todo x, $x\leq a$.\\\\

%--------------------------10-----------------------------
\item Si $x$ tiene la propiedad que $0\leq x < h$ para cada número real positivo $h$, entonces $x=0$\\\\
Demostración .- \; Por el teorema anterior ni $0<x$ ni $x<h$ satisfacen la proposición por lo tanto queda $x=0$\\\\

%lema 1.1
\begin{lema}
Para $b\geq 0$ $a^2>b$ $\Rightarrow$ $a>\sqrt{b}$ ó $a<-\sqrt{b}$\\\\
Demostración.- \;
Por hipótesis $a^2>b$ y $a^2-b>0$, luego \; $(a-\sqrt{b})(a+\sqrt{b})>0$, y por teorema  \; $a-\sqrt{b}>0$ \; y \; $a+\sqrt{b}<0$ \; ó \; $a-\sqrt{b}<0$ \; y \; $a+\sqrt{b}<0$, por lo tanto $a-\sqrt{b}<0$ \; ó \; $a<-\sqrt{b}$\\\\
\end{lema}
\end{enumerate}

\section{Números enteros y racionales}
\begin{tcolorbox}[colframe=white]
%definición de conjunto inductivo
\begin{def.}[Definción de conjunto inductivo]
Un conjunto de números reales se denomina conjunto inductivo si tiene las propiedades siguientes:
\begin{enumerate}[\bfseries a)]
\item El número $1$ pertenece al conjunto.
\item Para todo $x$ en el conjunto, el número $x+1$ pertenece también al conjunto.
\end{enumerate}
\end{def.}
\begin{def.}[Definición de enteros positivos]
Un número real se llama entero positivo si pertenece a todo conjunto inductivo.
\end{def.}
\end{tcolorbox}

\setcounter{section}{7}
\section{Cota superior de un conjunto, elemento máximo, extremo superior}
\begin{tcolorbox}[colframe=white]
\begin{def.}[definición de extremo superior]
Un número $B$ se denomina extremo superior de un conjunto no vacío $S$ si $B$ tiene las dos propiedades siguientes:
\begin{enumerate}[\bfseries a)]
\item $B$ es una cota superior de $S$.
\item Ningún número menor que $B$ es cota superior para $S$.
\end{enumerate}
\end{def.}
\end{tcolorbox}

%teorema 1.26
\begin{teo}
Dos números distintos no pueden ser extremos superiores para el mismo conjunto.\\\\
Demostración.- \; Sean $B$ y $C$ dos extremos superiores para un conjunto $S$. La propiedad $b)$ de la definición 3.1 implica que $C\geq B$ puesto que $B$ es extremo superior; análogamente, $B \geq C$ ya que $C$ es extremo superior. Luego $B = C$ \\\\
\end{teo}

\section{Axioma del extremo superior (axioma de completitud)}

\begin{tcolorbox}[colframe=white]
%axioma 10
\begin{axioma}
Todo conjunto no vacío $S$ de números reales acotado superiormente posee extremo superior; esto es, existe un número real $B$ tal que $B=sup S$
\end{axioma}
\end{tcolorbox}

\begin{tcolorbox}[colframe=white]
%definición de extremo inferior
\begin{def.}[Definición de extremo inferior (Ínfimo)]
Un número $L$ se llama extremo inferior (o ínfimo) de $S$ si:
\begin{enumerate}[\bfseries a)]
\item $L$ es una cota inferior para $S$,
$$L\leq x, \, \, \forall x \in S$$
\item Ningún número mayor que $L$ es cota inferior para $S$.
$$Si \; t\leq x, \, \; \forall x \in S, \, \, entonces \, t\leq L $$
\end{enumerate}
El extremo inferior de $S$, cuando existe, es único y se designa por $infS$. Si $S$ posee mínimo, entonces $minS=infS$\\
\end{def.}
\end{tcolorbox}

%teorema 1.27
\begin{teo}
Todo conjunto no vacío $S$ acotado inferiormente posee extremo inferior  o ínfimo; esto es, existe un número real $L$ tal que $L=infS$.\\\\
Demostración.- \; Sea $-S$ el conjunto de los números opuestos de los de $S$. Entonces $-S$ es no vacío y acotado superiormente. El axioma 10 nos dice que existe un número $B$ que es extremo superior de $-S$. Es fácil ver que $-B=infS$.\\\\
\end{teo}

\section{La propiedad Arquimediana del sistema de los números reales}
%teorema 1.28
\begin{teo}
El conjunto $P$ de los enteros positivos 1,2,3,... no está acotado superiormente.\\\\
Demostración.- \; Supóngase $P$ acotado superiormente. Demostraremos que esto nos conduce a una contradicción. Puesto que P no es vacío, el axioma 10 nos dice que $P$ tiene supremo, sea este $b$. El número $b-1$, siendo menor que $b$, no puede ser cota superior de $P$. Por b) de la definición 3.1 existe $n>b-1$ es decir $b-1 \in P, \; b-1<b \; \; \exists n \in P:\; b-1<n$. Para este $n$ tenemos $n+1>b$. Puesto que $n+1$ pertenece a $P$, esto contradice el que $b$ sea una cota superior para $P$. \\\\
\end{teo}

%teorema 1.29
\begin{teo}
Para cada real $x$ existe un entero positivo $n$ tal que $n>x$\\\\
Demostración.- \; Si no fuera así, $x$ sería una cota superior de $P$, en contradicción con el teorema 3.3.\\\\
\end{teo}

%teorema 1.30
\begin{teo}
Si $x>0$ e $y$ es un número real arbitrario, existe un entero positivo $n$ tal que $nx>y$\\\\
Demostración.- \; Aplicar teorema 3.4 cambiando $x$ por $y/x$.\\\\
\end{teo}

%teorema 1.31
\begin{teo}
Si tres números reales $a$, $x$, e $y$ satisfacen las desigualdades $a\leq x \leq a+\displaystyle\frac{y}{n}$ para todo entero $n \geq 1$, entonces $x=a$\\\\
Demostración.- \; Si $x>a$, el teorema 3.5 nos menciona que existe un entero positivo que satisface $n(x-a)>y$, en contradicción de la hipótesis, luego $x>a$ no satisface para todo número real $x$ y $a$, con lo que deberá ser $x=a$.\\\\
\end{teo} 
\setcounter{section}{11}
\section*{Propiedades fundamentales del extremo superior ó supremo}
%teorema 1.32
\begin{teo}
Sea $h$ un número positivo dado y $S$ un conjunto de números reales.
\begin{enumerate}[\bfseries a)]
\item Si $S$ tiene extremo superior o supremo, para un cierto $x$ de $S$ se tiene 
$$x>supS-h$$
Demostración.- \; Si es $x\leq supS -h$ para todo $x$ de $S$, entonces $supS-h$ sería una cota superior de $S$ menor que su supremo. Por consiguiente debe ser $x>supS-h$ por lo menos para un $x$ de $S$.
\item Si $S$ tiene extremo inferior o ínfimo, para un cierto $x$ de $S$ se tiene 
$$x<infS+h$$
Demostración.- \; Si es $x \geq supS+h$ para todo $x$ de $S$, entonces $supS+h$ sería una cota inferior de $S$ mayor que su ínfimo. Por consiguiente debe ser $x<supS+h$ por lo menos para un $x$ de $S$.
\end{enumerate}
\end{teo}

%teorema 1.33
\begin{teo}[Propiedad aditiva]
Dados dos subconjuntos no vacíos $A$ y $B$ de $\mathbb{R}$, sea $C$ el conjunto
$$C=\lbrace a+b / a\in A, \; b \in B   \rbrace$$
\begin{enumerate}[\bfseries a)]
\item Si $A$ y $B$ poseen supremo, entonces $C$ tiene supremo, y 
$$supC= supA + supB$$
Demostración.- \; Supongamos que $A$ y $B$ tengan supremo. Si $c \in C$, entonces $c=a+b$, donde $a\in A$ y $b\in B.$ Por consiguiente $c \leq supA +supB$; de modo que $supA + supB$ es una cota superior de $C$. esto demuestra por el axioma 10 que $C$ tiene supremo y que 
$$supC \leq supA + supB$$
Sea ahora $n$ un entero positivo cualquiera. Según el teorema 3.7 $\left( con \; h=1/n \right)$ existen un $a$ en $A$ y un $b$ en $B$ tales que:
$$a>supA - \displaystyle\frac{1}{n} \; y \; b>supB -\frac{1}{n}$$
Sumando estas desigualdades, se obtiene 
$$a+b>supA +supB -\displaystyle\frac{2}{n}, \; \; ó  \; \; supA + supB < a+b+\frac{2}{n} \leq supC +\frac{2}{n}$$
puesto que $a+b \leq supC.$ Por consiguiente hemos demostrado que
$$supC \leq supA + supB < supC + \displaystyle\frac{2}{n}$$
para todo entero $n \geq 1.$ En virtud del teorema 3.6, debe ser $supC = supA+supB.$ Esto demuestra $a)$\\
\item Si $A$ y $B$ tienen ínfimo, entonces $C$ tiene ínfimo, e
$$infC = infA+ infB$$ 
Demostración.- \; Supongamos que $A$ y $B$ tengan ínfimo. Si $c \in C$, entonces $c=a+b$, donde $a\in A$ y $b\in B.$ Por consiguiente $c \geq infA +infB$; de modo que $infA + infB$ es una cota inferior de $C$. esto demuestra por el axioma 10 que $C$ tiene ínfimo y que 
$$infC \geq infA + infB$$
Sea ahora $n$ un entero positivo cualquiera. Según el teorema 3.7 $\left( con \; h=1/n \right)$ existen un $a$ en $A$ y un $b$ en $B$ tales que:
$$a<infA + \displaystyle\frac{1}{n} \;\;  y \; \; b<infB + \frac{1}{n}$$
Sumando estas desigualdades, se obtiene 
$$a+b<infA +infB +\displaystyle\frac{2}{n}, \; \; ó \; \;  infA+infB \leq infC \leq a+b < infA+infB + \displaystyle\frac{2}{n} $$
puesto que $a+b \geq infC$ Por consiguiente hemos demostrado que
$$infA+ infB \leq infC <infA + infB + \displaystyle\frac{2}{n}$$
para todo entero $n \geq 1.$ En virtud del teorema 3.6, debe ser $supA+supB = infC.$ Esto demuestra $b)$\\\\
\end{enumerate}
\end{teo}

%teorema 1.34
\begin{teo}
Dados dos subconjuntos no vacíos $S$ y $T$ de $\mathbb{R}$ tales que $$s\leq t$$ para todo $s$ en $S$ y todo $t$ en $T$. Entonces $S$ tiene supremo, $T$ ínfimo, y se verifica $$supS\leq infT$$\\\\
Demostración.- \; Cada $t$ de $T$ es corta superior para $S$. Por consiguiente $S$ tiene supremo que satisface la desigualdad $supS\leq T$ para todo $t$ de $S$. Luego $supS$ es una cota inferior de $T$, con lo cual $T$ tiene ínfimo que no puede ser menor que $supS$. Dicho de otro modo, se tiene $supS\leq infT$, como se afirmó.\\\\
\end{teo}

\section{Ejercicios}
\begin{enumerate}[\bfseries \Large 1.]
%-----------------------1-----------------------------
\item Si $x$ e $y$ son números reales cualesquiera, $x<y$, demostrar que existe por lo menos un número real $z$ tal que $x<z<y$\\\\
Demostración.- \; Sea $S$ un conjunto no vacío de $\mathbb{R}$, por axioma 10 se tiene un supremo llamemosle $z$, por definición $x\leq z$ para todo $x\in S$, ahora si $y\in \mathbb{R}$ \; que cumple $x\leq y$, para todo $x\in S$, entonces $z\leq y$, por lo tanto $x\leq z \leq y$ esto  nos muestra que existe por lo menos un número real que cumple la condición $x<z<y$. \\\\

%----------------------------2------------------------------
\item Si x es un número real arbitrario, probar que existen enteros $m$ y $n$ tales que $m<x<n$\\\\
Demostración.- \;  Sea $n\in \mathbb{Z}^+$ en virtud del axioma 5 se verifica $n+m=0$, donde $m$ es el opuesto de $n$, esto nos dice que $m<n$ y por teorema anterior se tiene $m<x<n$.\\\\

%-----------------------------3---------------------
\item Si $x>0$, demuestre que existe un entero positivo $n$ tal que $1/n<x$\\\\
Demostración.- \; Sea $y=1$ entonces por teorema 3.5 \; $nx>1$, por lo tanto $1/n<x$ \\\\

%-----------------------------4------------------------------
\item Si $x$ es un número real arbitrario, demostrar que existe un entero $n$ único que verifica las desigualdades $n\leq x < n+1$. Este $n$ se denomina la parte entera de $x$, se delega por $[x]$. Por ejemplo, $[5]=5,\; \left[ \frac{5}{2}\right] =2, \; \left[ -\frac{8}{2} \right]=-3$\\\\
Demostración.-\; Primero probemos la existencia de $n$,
\begin{itemize}
\item  Sea $1\leq a $  \; y  \; $\; S=\lbrace m \in \mathbb{N}/\; m \leq a \rbrace$\\ 
Vemos que $S$ es no vacío pues contiene a 1, y \; $a$ \; es un cota superior de $S$, luego por axioma del supremo, existe un número $s=supS$, entonces por teorema 3.7 \; con $h=1$ resulta:
$$n>s-1 \; \; ó \; \; s<n+1, \; \; para \; algún n \; de \; S \; \; \; (1) $$
Como $z \in S$, se cumple $z\leq a$ y solo falta probar que $a<z+1$. En efecto, si fuese $z+1\leq a$, entonces $n+1 \in S$ y por la propiedad a), se tendría $n+1\leq s$, en contradicción con (1).   \\
Por tanto, el número entero positivo $n$ cumple con $n\leq a < n+1$
\item $0\leq a < 1$\\
En este caso, el entero $n=0$ cumple con la propiedad requerida.
\item $a<0$
Entonces $-a>0$ \; y por los dos casos anteriores, existe un entero u tal que $u\leq a< u+1$ de donde $-u-1<a\leq -u$.\\
Definiendo $n$ por  
\begin{equation}
n = \left\lbrace
\begin{array}{lcr}
-u-1 & si & a<-u\\
\textup{si } & x\leq 5 & a=-u
\end{array}        
\right.
\end{equation}
se prueba fácilmente que $z\leq a < z+1$
\end{itemize}
Luego demostramos la unicidad. Sea $w$ y $z$ dos números enteros tal que, $w\leq a < w + 1$ y $z \leq a < z + 1$ debemos probar que $w=z$. Si fuesen distintos, podemos suponer que $z<w$. Entonces $w-z\geq 1$, esto es $z+1\leq w$, y de $a<z+1\leq w \leq a$ resulta una contradicción ya que $a<a$ luego se cumple que $w=z$. \\\\ 

%---------------------------5------------------------------
\item Si $x$ es un número real arbitrario, $x<y$, probar que existe un entero único $n$ que satisface la desigualdad $n \geq x < n+1$\\\\
Demostración.- \; sabemos que para $x \in \mathbb{R}$ hay exactamente un $n \in \mathbb{Z}$ tal que $n \neq x < n+1$\\
Si $n=x$ entonces $x \neq n < x+1.$. Por otro lado si $n \neq x,$ entonces tenemos $n<x,$ así $n+1 < x+1$. Pero sabemos que $x<n+1$, por lo tanto, $$x<n+1<x+1 \rightarrow x \leq n+1 < x+1$$\\\\

%---------------------------6-------------------------------
\item Si $a$ y $b$ son números reales arbitrarios, $a<b$, probar que existe por lo menos un número racional $r$ tal que $a<r<b$ y deducir de ello que existen infinitos. Esta propiedad se expresa diciendo que el conjunto de los números racionales es denso en el sistema de los números reales.\\\\
Demostración.- \, Por la propiedad arquimediana, para el número $\displaystyle\frac{1}{b-a}$ existe un número natural $d$ tal que $\displaystyle\frac{1}{b-a}$, de donde 
$$db-da>1 \; \; ó \; \, da+1<da \; \, \, (1)$$
y también si $z$=parte entera de $da$
$$z\leq da < z+1 \, \, \, (2)$$
Sea $q=\displaystyle\frac{n}{d}$, con $n=z+1$. Entonces $q$ es un número racional y cumple $x<q<y$ pues:
$$a= d\displaystyle\frac{a}{d} < \frac{z+1}{d}<q=\frac{z+1}{d}\leq \frac{da+1}{d}<d\frac{b}{d}=b$$. y por ser $\displaystyle\frac{z+1}{d}$ deducimos que existen infinitos números racionales entre $a$ e $b$\\

%----------------------------7----------------------------------
\item Si $x$ es racional, $x\neq 0$, e $y$ es irracional, demostrar que $x+y$, $x-y$, $xy$, $x/y$, son todos irracionales.
\begin{itemize}
\item $x+y$, \, $x-y$\\
Supongamos que la suma nos da un racional, es decir $\displaystyle\frac{q}{p}+y=\frac{s}{t}\; para \; s,t\neq 0$, por lo tanto $y = \displaystyle\frac{qt+sp}{tp}$, así llegamos a una contradicción, en virtud del axioma 7 (la suma y multiplicación de dos racionales nos da otros racionales).\\
$x-y$ Se puede comprobar de similar manera a la anterior demostración.\\
\item $xy$, \, $x/y$, \; $y/x$\\
Supongamos que el producto nos da un número racional, por lo tanto $\frac{p}{q}\cdot y = \displaystyle\frac{t}{s}$ para $q,s \neq 0$ y $ \displaystyle y = \frac{sq}{pt}$ en contradicción con la hipótesis. De igual manera se comprueba que $x/y$ es irracional.\\\\
\end{itemize}

%-----------------------------8----------------------------
\item ¿La suma o el producto de dos números irracionales es siempre irracional?\\\\
Demostración.- \, No siempre se cumple la proposición, veamos dos contra ejemplos.\\
Sea $a$ un número irracional entonces por teorema anterior $1-a$ es irracional, así $a+(1-a)=1$, sabiendo que $1\in \mathbb{R}$. Por otro lado sabemos que $\displaystyle\frac{1}{a}$ es irracional, por lo tanto $1\in \mathbb{R}$.\\\\

%------------------------------9-----------------------------
\item Si $x$ e $y$ son números reales cualesquiera, $x<y$, demostrar que existe por lo menos un número irracional $z$ tal que $x<z<y$ y deducir que existen infinitos\\\\
Demostración.- \; Sea $0<x<y$ e $i$ un número irracional, por propiedad arquimediana  $y-x>\displaystyle\frac{i}{n}$ \; ó \; $\displaystyle x+ \frac{i}{n}<y$. \\\\
por teorema 3.15 \; se tiene que $\dfrac{i}{n}$ es irracional llamemosle $z$ por lo tanto  $x+z>x$, luego existe $x<z<y$. Y de $\dfrac{i}{n}$  deducimos que existen infinitos números irracionales que cumplen la condición.\\\\

%-------------------------------10----------------------------
\item Un entero $n$ se llama par si $n=2m$ para un cierto entero $m$, e impar si $n+1$ es par demostrar las afirmaciones siguientes:
\begin{enumerate}[\bfseries a)]
\item Un entero no puede ser a la vez par e impar.\\\\
Demostración.- \; Sean $2k$ y $2i+1$ dos enteros par e impar a la vez  entonces $2k=2i+1$ ó  $(k-i)=\dfrac{1}{2}$  lo cual no es cierto, ya que la resta de dos números pares siempre da par, por lo tanto es par o es impar pero no los dos al mismo tiempo.\\
\item Todo entero es par o es impar.\\\\
Demostración.- \; Por inciso a) \; $2k\neq 2k-1$ para $k\in \mathbb{Z}$, por la tricotomía ó $2k < 2k-1$ ó $2k > 2k-1$ lo cual se cumple pero no ambos a la vez.\\  
\item La suma o el producto de dos enteros pares es par. ¿ Qué se puede decir acerca de la suma o del producto de dos enteros impares ?\\\\
Demostración.- \; Sea $k\in \mathbb{R}$ entonces $2k+2k=4k=2(2k)$. Luego para el producto $2k\cdot 2k = 4k^2=2(2k^2)$\\
Por otra parte $(2k-1)+(2k-1)=4k-2=2(2k-1)$. No pasa lo mismo para el producto ya que  $(2k-1)(2k-1)=2k^2-4k+1=2(2k^2-2k)+1$\\\\
\item Si $n^2$ es par, también lo es $n$. Si $a^2=2b^2$, siendo $a$ y $b$ enteros, entonces $a$ y $b$ son ambos pares.\\\\
Demostración.- \;  Si $n$ es impar entonces $n^2$ es impar, reciprocamente hablando, entonces sea $n^2=(2k-1)^2$ para $k\in \mathbb{R}$, por lo tanto $2(2k^2+4k)-1$ es impar.\\
Por otro lado, sea $a=2k$, $b=2k-1$ '; y \; $k\in \mathbb{Z}$ entonces $(2k)^2=2(2k-1)^2$, por lo tanto $k=\dfrac{1}{2}$, esto contradice $k\in \mathbb{Z}$.\\
\item Todo número racional puede expresarse en la forma $a/b$, donde $a$ y $b$ enteros, uno de los cuales por lo menos es impar.\\\\
demostración.- \; Sea $r$ un número racional con $r=\dfrac{a}{b}$. Si $a$\; y \; $b$ son ambos pares, entonces tenemos $$a=2c \; y \; b=2d \,\; \Rightarrow \,\; \dfrac{a}{b}=\dfrac{2c}{2d}=\dfrac{c}{d},$$ con $c<a$ \; y \; $d<b$. ahora, si $c$ \; y \; d ambos son pares, repita el proceso. Esto dará una secuencia estrictamente decreciente de enteros positivos, por lo que el proceso debe terminar por el principio de buen orden. Por lo tanto debemos tener algunos enteros $r$ \; y \; $s$, no ambos con $n=\dfrac{a}{b}=\dfrac{r}{s}$.
\end{enumerate}

%-----------------------------11------------------------
\item Demostrar que no existe número racional cuyo cuadrado sea 2.\\\\
Demostración.- \; Utilizaremos el método de reducción al absurdo. Supongamos que n es impar, es decir, $n=2k+1\; k \in \mathbb{Z}$, ahora operando:
$$n^2=(2k+1)^2 \Rightarrow  n^2 = 4k^2 +4k + 1 \Rightarrow n^2=2(2k^2+2k)+1$$
Sabemos que $2k^2+2k$ es un número entero cualquiera, por lo tanto podemos realizar un cambio de variable, $2k^2+2k = k^{'}$, entonces:
$$n^2=2k^{'} +1$$
Se tiene una contradicción ya por teorema anterior se de dijo que $n^2$ es par, por lo tanto queda demostrado la proposición.  
Ahora si estamos con la facultad de demostrar que  $\sqrt{2}$ es irracional.\\
Supongamos que $\sqrt{2}$ es racional, es decir, existen números enteros tales que:
$$\displaystyle\frac{p}{q}=\sqrt{2}$$
Supongamos también que $p$ y $q$ no tienen divisor común mas que el 1. Se tiene:
$$p^2=2q^2$$
Esto nos muestra que $p^2$ es par y  por la previa demostración tenemos que $p$ es par. En otras palabras $p = 2k, \; \forall k \in \mathbb{Z}$, entonces:
$$(2k)^2 = 2q^2 \Rightarrow 4k^2 = 2q2 \Rightarrow 2k^2 = q^2 $$
Esto demuestra que $q^2$ es par y en consecuencia que $q$ es par. Así pues, son pares tanto $p$ como $q$ en contradicción con el hecho de que $p$ y $q$ no tienen divisores comunes. Esta contradicción completa la demostración.\\\\

%-----------------------------11--------------------------
\item La propiedad arquimediana del sistema de números reales se dedujo como consecuencia del axioma del supremo. Demostrar que el conjunto de los números racionales satisface la propiedad arquimediana pero no la del supremo. Esto demuestra que la propiedad arquimediana no implica el axioma del supremo.\\\\
Demostración.- \; Está claro que que el conjunto de los racionales satisface la propiedad arquimediana ya que si $x=\dfrac{p}{q}$ e $y=\dfrac{s}{t}$ para $q,\; t \neq 0$ entonces $\dfrac{p}{q}\cdot n>\dfrac{s}{t}$.\\
Por otra parte sea $S$ el conjunto de todos los racionales y supongase que esta acotado superiormente, por axioma 10 se tiene supremo, llamemosle $B$, entonces $x\leq B, \; \; x \in S$, luego existe $t\in \mathbb{R}$ tal que $B\leq t$, así por teorema 3.14 \; $B<x<t$,  esto contradice que $B$ sea supremo.\\\\ 

   
\section{Existencia de raíces cuadradas de los números reales no negativos}
\paragraph{Nota}
Los números negativos no pueden tener raíces cuadradas, pues si $x^2=a$, al ser $a$ un cuadrado ha de ser no negativo (en virtud del teorema 2.5). Además, si $a=0$, $x=0$ es la única raíz cuadrada (por el teorema 1.11). Supóngase, pues $a>0$. Si $x^2=a$ entonces $x\leq 0$ y $(-x)^2=a$, por lo tanto, $x$ y su opuesto son ambos raíces cuadradas. Pero a lo sumo tiene dos, porque si $x^2=a$ e $y^2=a$, entonces $x^2=y^2$ \; y \; $(x+y)(x-y)=0$, en virtud del teorema 1.11, \; ó $x=y$ ó $x=-y$. Por lo tanto, si $a$ tiene raíces cuadradas, tiene exactamente dos.
\begin{tcolorbox}[colframe=white]
\begin{def.}
Si $a\geq 0$, su raíz cuadrada no negativa se indicará por $a^{1/2}$ o por $\sqrt{a}$. Si $a>0$, la raíz cuadrada negativa es $-a^{1/2}$ ó $-\sqrt{a}$
\end{def.}
\end{tcolorbox}
%teorema 1.35
\begin{teo}
Cada número real no negativo $a$ tiene una raíz cuadrada no negativa única.\\\\
Demostración.- \; Si $a=0$, entonces $0$ es la única raíz cuadrada. Supóngase pues que $a>0$. Sea $S$ el conjunto de todos los números reales positivos $x$ tales que $x^2\leq a$. Puesto que $(1+a)^2>a$, el número $(a+1)$ es una cota superior de $S$. Pero, $S$ es no vacío, pues $a/(1+a)$ pertenece a $S$; en efecto $a^2\leq a(1+a)^2$ y por lo tanto $a^2/(1+a)^2\leq a$. En virtud del axioma 10, $S$ tiene un supremo que se designa por $b$. Nótese que $b\geq a/(1+a)$ y por lo tanto $b>0$. Existen sólo tres posibilidades: $b^2>a$, $b^2<a$, $b^2=a$.\\
Supóngase $b^2>a$ y sea $c=b-(b^2-a)/(2b)/(2b)=\dfrac{1}{2}(b+a/b)$. Entonces $a<c<b$ \; y \; $c^2=b^2-(b^2-a)+(b^2-a)^2/4b^2=a+(b^2-a)^2/(4b^2)>a$. Por lo tanto, $c^2>x^2$ para todo $x \in S$, es decir, $c>x$ para cada $x \in S$; luego $c$ es una cota superior de $S$, y puesto que $c<b$ se tiene una contradicción con el hecho de ser $b$ el extremo superior de $S$. Por tanto, la desigualdad $b^2>a$ es imposible.\\
Supóngase $b^2<a$. Puesto que $b>0$ se puede elegir un número positivo $c$ tal que $c<b$ y tal que $c<(a-b^2)7(3b). Se tiene entonecs$ $$(b+c)^2=b^2+c(2b+c)< b^2 +3bc < b^2 + (a-b^2)=a$$ es decir, $b+c$ pertenece a $S$. Como $b+c>b,$ esta desigualdad está en contradicción con que $b$ sea una cota superior de $S$. Por lo tanto, la desigualdad $b^2<a$ es imposible y sólo queda como posible $b^2=a$\\\\
\end{teo}
\end{enumerate}
 
\section{Raíces de orden superior. Potencias racionales}
El axioma del extremo superior se puede utilizar también para probar la existencia de raíces de orden superior. Por ejemplo, si $n$ es un entero positivo impar, para cada real $x$ existe un número real $y$, y uno sólo tal que $x^n=x$. Esta $y$ se denomina raíz n-sima de $x$ y se indica por:
\begin{tcolorbox}[colframe=white]
\begin{def.}
$$y=x^{\frac{1}{n}} \; \; ó \; \; y=\sqrt[n]{x}$$
\end{def.}
\end{tcolorbox}
Si $n$ es par, la situación es un poco distinta. En este caso, si $x$ es negativo, no existe un número real $y$ tal que $y^n = x$, puesto que $y^n\geq 0$ para cada número real $y$. Sin embargo, si $x$ es positivo, se puede probar que existe un número positivo y sólo uno tal que $y^n = x$. Este $y$ se denomina la raíz n-sima positiva de $x$ y se indica por los símbolos anteriormente mencionados. Puesto que $n$ es par, $(-y)^n = y^n$ y, por tanto, cada $x > 0$ tiene dos raíces n-simas reales, $y$ e $-y$. Sin embargo, los símbolos $x^{\frac{1}{n}}$ y $\sqrt[n]{x}$; se reservan para la raíz n-sima positiva.\\
\begin{tcolorbox}[colback=white]
\begin{def.}
Si $r$ es un número racional positivo, sea $r = m/n$, donde $m$ y $n$ son enteros positivos, se define como: $$x^r= x^{m/n }=(x^m)^{\frac{1}{n}},$$ es decir como raíz n-sima de $x^m$, siempre que ésta exista.\\ 
\end{def.}
\begin{def.}  
Si $x\neq 0$, se define $$x^{-r} = \dfrac{1}{x^r},$$ con tal que $x^r$ esté definida.
\end{def.}
\end{tcolorbox}
 Partiendo de esas definiciones, es fácil comprobar que las leyes usuales de los exponentes son válidas para exponentes racionales: 
\begin{tcolorbox}[colframe=white]
\begin{prop} Propiedades de potencia.\\
\begin{center}
\begin{enumerate}[\bfseries 1.]
\item $x^r \cdot x^s =  x^{r+s}$
\item $(x^r)^s=x^{rs}$
\item $(xy)^r=x^r \cdot y^r$
\item $\left( \dfrac{x}{y} \right)^r=\dfrac{x^r}{y^r}$
\end{enumerate}
\end{center}
\end{prop}
\end{tcolorbox}

\setcounter{chapter}{4}
\setcounter{section}{2}
\section{El principio de la inducción matemática}
\begin{tcolorbox}[colback=white]
\paragraph{Método de demostración por inducción}Sea $A(n)$ una afirmación que contiene el entero $n$. Se puede concluir que $A(n)$ es verdadero para cada $n\geq n_1$ si es posible:
\begin{enumerate}[\bfseries a)]
\item Probar que $A(n_1)$ es cierta.
\item Probar, que supuesta $A(k)$ verdadera, siendo $k$ un entero arbitrario pero fijado $\geq n_1$, que $A(k+1)$ es verdadera.\\
\end{enumerate}
En la práctica, $n_1$ es generalmente igual a $1$.
\end{tcolorbox}

%teorema 1.36
\begin{teo}[Principio de inducción matemática]
Sea $S$ un conjunto de enteros positivos que tienen las dos propiedades siguientes:
\begin{enumerate}[\bfseries a)]
\item El número 1 pertenece al conjunto $S$.
\item Si un entero $k$ pertenece al conjunto $S$, también $k+1$ pertenece a $S$.
\end{enumerate}
Entonces todo entero positivo pertenece al conjunto $S$.\\\\
Demostración.- \; Las propiedades $a)$ y $b)$ nos dicen que $S$ es un conjunto inductivo. Por consiguiente $S$ tiene cualquier entero positivo.\\\\ 
\end{teo}

%teorema 1.37
\begin{teo}[principio de buena ordenación]
Todo conjunto no vacío de enteros positivos contiene uno que es el menor \\\\
Demostración.- \; Sea $T$ una colección no vacía de enteros positivos. Queremos demostrar que $t_0$ tiene un número que es el menor, esto es, que hay en T un entero positivo t.; tal que $t_0\leq t$ para todo $t$ de $T$.\\
Supongamos que no fuera así. Demostraremos que esto nos conduce a una contradicción. El entero $1$ no puede pertenecer a $T$ (de otro modo él sería el menor número de $T$). Designemos con $S$ la colección de todos los enteros positivos $n$ tales que $n<t$ para todo $t$ de $T$. Por tanto $1$ pertenece a $S$ porque $1 < t$ para todo $t$ de $T$. Seguidamente, sea $k$ un entero positivo de $S$. Entonces $k < t$ para todo $t$ de $T$. Demostraremos que $k + 1$ también es de $S$. Si no fuera así, entonces para un cierto $t$, de $T$ tendríamos $t_1 \leq k+1$. Puesto que $T$ no posee número mínimo, hay un entero $t_2$ en $T$ tal que $t_2 < t_1$ Y por tanto $t_2 < k + 1$. Pero esto significa que $t_2 \leq k$, en contradicción con el hecho de que $k < t$ para todo $t$ de $T$. Por tanto $k + 1$ pertenece a $S$. Según el principio de inducción, $S$ contiene todos los enteros positivos. Puesto que $T$ es no vacío, existe un entero positivo $t$ en $T$. Pero este $t$ debe ser también de $S$ (ya que $S$ contiene todos los enteros positivos). De la definición de $S$ resulta que $t < t$, lo cual es absurdo. Por consiguiente, la hipótesis de que $T$ no posee un número mínimo nos lleva a una contradicción. Resulta pues que $T$ debe tener un número mínimo, y a su vez esto prueba que el principio de buena ordenación es una consecuencia del de inducción.\\\\
\end{teo}

\section{Ejercicios}
\begin{enumerate}[\bfseries \Large 1.]
%----------------------------1----------------------------
\item Demostrar por inducción las fórmulas siguientes:
\begin{enumerate}[\bfseries (a)]
%(a)
\item $1+2+3+...+n=n(b+1)/2$\\\\
Demostración.- \; Sea $n=k$ entonces $1+2+3+...+k=k(k+1)/2$.\\
Para $k=1$ se tiene $1=1(1+1)/2$. \\ 
Por ultimo si  $k=k+1$ nos queda probar que  $1+2+3+...+k+(k+1)=\dfrac{(k+1)(k+2)}{2}$, luego $\dfrac{k(k+1)}{2}+(k+1)=\dfrac{k(k+1)+2(k+1)}{2}$. Así $\dfrac{k^2+3k+2}{2}$ \; y \; $\dfrac{(x+1)(x+2)}{2}$\\\\
%(b)
\item $1+3+5+...+(2n-1)=n^2$\\\\
Demostración.- \; Sea $n=k$ entonces $1+3+5+...+(2k-1)=k^2$.\\
Para $k=1$ se tiene $[2(1)-1]=1^2$, así $1=1$\\
Luego, si $k=k+1$ entonces $1+3+5+...+[2(k+1)-1]=(k+1)^2$. Por lo tanto $k^2+2k+1=(k+1)^2$.\\\\
%(c)
\item $1^3+2^3+3^3+...+n^3=(1+2+3+...+n)^2$\\\\
Demostración.- \; Sea $n=k$ entonces, $$1^3+2^3+3^3+...+k^3=(1+2+3+...+k)^2$$
Para $k=1$ $$1=1,$$
Luego $k=k+1$, $$1^3+2^3+3^3+...+k^3+(k+1)^3=(1+2+3+...+k)^2,$$
Así,
\begin{center}
\begin{tabular}{r c l}
$\left(\dfrac{k(k+1)}{2} \right)^2+(k+1)^3=$&=&$\left( \dfrac{k(k+1)}{2}+(k+1)\right) ^2$\\\\
$\dfrac{k^2(k+1)^2}{4}+(k+1)^3$&=&$\dfrac{k^2(k+1)^2}{4}+k(k+1)^2+(k+1)^2$\\\\
$\dfrac{k^2(k+1)^2+4(k+1)^3}{4}$&=&$\dfrac{k^2(k+1)^2+4k(k+1)^2+4(k+1)^2}{4}$\\\\
$\dfrac{(k+1)^2 (k^2+4k+4)}{4}$&=&$\dfrac{(k+1)^2 (k^2 + 4k +4)}{4}$\\\\
\end{tabular}   
\end{center}
%(d)
\item $1^3+2^3+...+(n-1)^3<n^4/4<1^3+2^3+...+n^3$\\\\
Demostración.- \; Sea $n=k$ entonces $$1^3+2^3+...+(k-1)^3<k^4/4 \; \; \; (1)$$ 
Para $k=1$, \; $0<1/4$ se observa que se cumple.\\
Después, para $k=k+1$, $$1^3+2^3+...+k^3<(k+1)^4/4,$$ sumando $k^3$ a $(1)$, 
$$1^3+2^3+...+k^3<k^4/4+k^3$$ y para deducir como consecuencia de $k+1$, basta demostrar, $$k^4/4+k^3<(k+1)^4/4$$, Pero esto es consecuencia inmediata de la igualdad $$(k+1)^4/4=(k^4+4k^3+6k^2+4k+1)/4=k^4/4+k^3+(3k^2)/2+k+1$$ Por tanto se demostró que $k+1$ es consecuencia de $k$\\\\
\end{enumerate}

%-----------------------------2---------------------------------
\item Obsérvese que: 
\begin{center}
\begin{tabular}{r c l}
$1$&$=$&$1$\\
$1-4$&$=$&$-(1+2)$\\
$1-4+9$&$=$&$1+2+3$\\
$1-4+9-16$&$=$&$-(1+2+3+4)$\\
\end{tabular}
\end{center}
Indúzcase la ley general y demuéstrese por inducción\\\\
Demostración.- \: Verificando tenemos que la ley general es $1-4+9-16+...+(-1)^{n+1}\cdot n^2=(-1)^{n+1}(1+2+3+...+n)$. \\
Ahora pasemos a demostrarlo. Sea $n=k$ entonces, $$1-4+9-16+...+(-1)^{k+1}\cdot k^2=(-1)^{k+1}(1+2+3+...+k)$$ Si $k=1$, se sigue, $(-1)^2 \cdot 1^2 = (-1)^2\cdot 1$, vemos que satisface para $k=1.$
Luego $k=k+1$,  $$1-4+9-16+...+(-1)^{k+2}\cdot (k+1)^2=(-1)^{k+2}\left[\dfrac{(k+1)(k+2)}{2}\right]$$.\\
Sumando $(-1)^{k+2}\cdot (k+1)^2$  a la segunda igualdad dada, se tiene,
$$1-4+9-16+...+ (-1)^{k+2}\cdot (k+1)^2 = (-1)^{k+1}\left(\dfrac{k(k+1)}{2}\right) + (-1)^{k+2}\cdot (k+1)^2$$ Por lo tanto, basta demostrar que $(-1)^{k+1}\left(\dfrac{k(k+1)}{2}\right) + (-1)^{k+2}\cdot (k+1)^2=(-1)^{k+2}\left[\dfrac{(k+1)(k+2)}{2}\right]$\\
\begin{center}
\begin{tabular}{r c l}
$(-1)^{k+1}\left(\dfrac{k(k+1)}{2}\right) + (-1)^{k+2}\cdot (k+1)^2$&$=$&$(-1)^{k+2}\left\lbrace \dfrac{[(-1)(k+1)k]+2(k^2+2k+1)}{2} \right\rbrace$\\\\
&$=$&$(-1)^{k+2} \left( \dfrac{-k^2 -k +2k^2 +4k +2}{2} \right)$\\\\
&$=$&$(-1)^{k+2} \left( \dfrac{k^2+3k+2}{2} \right)$\\\\
&$=$&$(-1)^{k+2} \left[ \dfrac{(x+1)(x+2)}{2} \right]$\\\\
\end{tabular}
\end{center}
%-------------------------------3--------------------------------
\item Obsérvese que: 
\begin{center}
\begin{tabular}{r c l}
$1+\frac{1}{2}$&=&$2-\frac{1}{2}$\\\\
$1+\frac{1}{2}+\frac{1}{4}$&=&$2-\frac{1}{4}$\\\\
$1+\frac{1}{2}+\frac{1}{4}+\frac{1}{8}$&=&$2-\frac{1}{8}$\\\\
\end{tabular}
\end{center}
Demostración.- \; Se verifica que $1+\frac{1}{2}+\frac{1}{4}+...+\dfrac{1}{2^n}=2-\dfrac{1}{2^n}$.\\\\
Para $n=k$ $$1+\frac{1}{2}+\frac{1}{4}+...+\dfrac{1}{2^k}=2-\dfrac{1}{2^k}$$\\
$k=1$ $$1+\dfrac{1}{2^1}=2-\dfrac{1}{2^1}$$\\
Luego $k=k+1$ $$1+\frac{1}{2}+\frac{1}{4}+...+\dfrac{1}{2^{k+1}}=2-\dfrac{1}{2^{k+1}}$$\\
Así solo falta demostrar que, $$2-\dfrac{1}{2^k}+\dfrac{1}{2^{k+1}}=2-\dfrac{1}{2^{k+1}}$$
\begin{center}
\begin{tabular}{r c l}
$2-\dfrac{1}{2^k}+\dfrac{1}{2^{k+1}}$&=&$2+\dfrac{-2+1}{2^{k+1}}$\\\\
&=&$2-\dfrac{1}{2^{k+1}}$\\\\
\end{tabular}
\end{center}

%-----------------------------4-----------------------------
\item Obsérvese que:
\begin{center}
\begin{tabular}{r c l}
$1-\frac{1}{2}$&=&$\frac{1}{2}$\\\\
$(1-\frac{1}{2})(1-\frac{1}{3})$&=&$\frac{1}{3}$\\\\
$(1-\frac{1}{2})(1-\frac{1}{3})(1-\frac{1}{4})$&=&$\frac{1}{4}$\\\\
\end{tabular}
\end{center}
Indúzcase la ley general y demuéstrese por inducción.\\\\
Demostración.- \; Se induce que $\left( 1-\dfrac{1}{2} \right)\left( 1-\dfrac{1}{3} \right)...\left(1- \dfrac{1}{n} \right)=\dfrac{1}{n}$ para todo $n>1$.\\
Sea $n=k$, entonces $\left( 1-\dfrac{1}{2} \right)\left( 1-\dfrac{1}{3} \right)...\left(1- \dfrac{1}{k} \right)=\dfrac{1}{k}$. Después para $k=2$, \; $1-\dfrac{1}{2}=\dfrac{1}{2}$. Si $k=k+1$ tenemos $\left( 1-\dfrac{1}{2} \right) \left( 1-\dfrac{1}{3} \right)...\left(1- \dfrac{1}{k+1} \right)=\dfrac{1}{k+1}$. Luego es fácil comprobar que 
$\dfrac{1}{k}\left(1 - \dfrac{1}{k+1} \right)=\dfrac{1}{k+1}$. \\\\

%----------------------------5--------------------------------
\item Hallar la ley general que simplifica al producto $$\left( 1-\dfrac{1}{4} \right)\left( 1-\dfrac{1}{9} \right)\left( 1- \dfrac{1}{16} \right)..\left( 1- \dfrac{1}{n^2} \right)$$ y demuéstrese por inducción.\\\\
Demostración.- \; Inducimos que $\left( 1-\dfrac{1}{4} \right)\left( 1-\dfrac{1}{9} \right)\left( 1- \dfrac{1}{16} \right)..\left( 1- \dfrac{1}{n^2} \right)=\dfrac{n+1}{2n}$,para todo $n>1$. Después $n=k=1$, $$1-\dfrac{1}{2^2}=\dfrac{2+1}{2\dot 2} \Rightarrow \dfrac{3}{4}=\dfrac{3}{4}$$ 
Luego $k+1$, $$\left( 1-\dfrac{1}{4} \right)\left( 1-\dfrac{1}{9} \right)\left( 1- \dfrac{1}{16} \right)..\left( 1- \dfrac{1}{(k+1)^2} \right)=\dfrac{(k+1)+1}{2(k+1)}$$
Así,
\begin{center}
\begin{tabular}{r c l}
$\left( \dfrac{k+1}{2k}\right) \left( 1- \dfrac{1}{(k+1)^2} \right)$&=&$\dfrac{(k+1)+1}{2(k+1)}$\\\\
$\left( \dfrac{k+1}{2k} - \dfrac{1}{2k(k+1)} \right)$&=&$\dfrac{k+2}{2k+2}$\\\\
$\dfrac{(k+1)^2-1}{2k(k+1)}$&=&$\dfrac{k+2}{2k+2}$\\\\
$\dfrac{k+2}{2k+2}$&=&$\dfrac{k+2}{2k+2}$\\\\
\end{tabular}
\end{center}

%--------------------------6-------------------------------
\item Sea $A(n)$ la proporción: $1+2+...+n=\dfrac{1}{8} (2n+1)^2$.
\begin{enumerate}[\bfseries a)]
%(a)
\item Probar que si $A(k)$, $A(k+1)$ también es cierta.\\\\
Demostración.- \; Para $A(k+1)$, 
\begin{center}
\begin{tabular}{r c l}
$\dfrac{1}{8}(2k+1)^2+(k+1)$&=&$\dfrac{1}{8}\left[2(k+1)+1\right]^2$\\\\
$\dfrac{4k^2+12k+9}{8}$&=&$\dfrac{4k^2+12k+9}{8}$\\\\
\end{tabular}
\end{center}
%(b)
\item Critíquese la proposición $"$de la inducción se sigue que $A(n)$ es cierta para todo $n$ $"$.\\\\
Se ve que no se cumple para ningún entero $A(n)$ pero si para $A(n+1)$.\\\\ 
%(c)
\item Transfórmese $A(n)$ cambiando la igualdad por una desigualdad que es cierta para todo entero positivo $n$\\\\
Primero comprobemos para $A(1)$, \; $1<\dfrac{9}{8}$.\\
Luego para $A(k),$ $$1+2+...+(k+1)<\dfrac{1}{8}(2k+1)^2$$
Después para $A(k+1)$ $$1+2+...+(k+1)<\dfrac{1}{8}(2k+1)^2$$
Remplazando $(k+1)$ a $A(k)$ $$1+2+...+(k+1)<\dfrac{1}{8}(2k+1)^2+(k+1)$$
por último solo nos queda demostrar $$\dfrac{1}{8}(2k+1)^2<\dfrac{1}{8}(2k+1)^2+(k+1)$$
Así $\dfrac{4k^2+12k+9}{8}<\dfrac{4k^2+12k+9}{8} +(k+1)$, vemos que la inecuación se cumple para cualquier número natural.\\\\ 
\end{enumerate} 

%--------------------------7-------------------------------
\item Sea $n_1$ el menor entero positivo $n$ para el que la desigualdad $(1+x)^n>1+nx+nx^2$ es cierta para todo $x>0$. Calcular $n_1$, y demostrar que la desigualdad es cierta para todos los enteros $n\geq n_1$\\\\
Demostración.- \; vemos que la proposición es validad para $n_1=3$, $$(1+x)^3>1+3x+3x^2,$$ y no así para $n=1$ \; y \;$n=2$ entonces $A(n)=A(k)\geq 3$, $(1+x)^k>1+kx+kx^2.$ Después para un $A(k+1)$, $(1+x)^{k+1}>1+(k+1)x+(k+1)x^2,$ así \; $(1+kx+kx^2)(1+k)>1+(k+1)x+(k+1)x^2,$ luego se cumple la desigualdad $x(kx^2)+x^2+kx+x+1+x^2>kx^2+x^2+kx+x+1$.\\\\

%----------------------------8-------------------------------
\item Dados números reales positivos $a_1,a_2,a_3,...,$ tales que $a_n\leq ca_{a-1}$ para todo $n\geq 2$. Donde $c$ es un número positivo fijo, aplíquese el método de inducción para demostrar que $a_n \leq a_1 c^{n-1}$ para cada $n \geq 1$\\\\
Demostración.- \; Primero, para el caso $n=1$, tenemos $a_1c^0=a_1$, por lo tanto la desigualdad es validad. Ahora supongamos que la desigualdad es válida para algún número entero $k$: $a_k\leq a_1c^{k-1}$, luego multiplicamos por $c$, $ca_k\leq a_1c^k$, pero dado que se asume por hipótesis $a_{k+1} \leq ca_k$, entonces $a_{k+1}\leq a_1c^k$, por lo tanto, la declaración es válida para todo $n$.\\\\

%-------------------------------9----------------------------
\item Demuéstrese por inducción la proposición siguiente: Dado un segmento de longitud unidad, el segmento de longitud $\sqrt{n}$ se puede construir con regla y compás para cada entero positivo $n$.\\\\
Demostración.- \; Dada una línea de longitud 1, podemos construir una línea de longitud $\sqrt{2}$ tomando la hipotenusa del triángulo rectángulo con patas de longitud 1.\\
Ahora, supongamos que tenemos una línea de longitud 1 y una línea de longitud $\sqrt{k}$ para algún número entero $k$. Luego podemos formar un triángulo rectángulo con patas de longitud 1 y longitud $\sqrt{k}$. La hipotenusa de este triángulo es $\sqrt{k+1}$. Por lo tanto, si podemos construir una línea de longitud $\sqrt{k}$, entonces podemos construir una línea de longitud $\sqrt{k+1}$. Como podemos construir una línea de longitud $\sqrt{2}$ en el caso base, podemos construir una línea de longitud $\sqrt{n}$ para todos los enteros $n$.\\\\

%------------------------------10------------------------------
\item Sea $b$ un entero positivo. Demostrar por inducción la proposición siguiente: Para cada entero $n\geq 0$ existen enteros no negativos $q$ \; y \; $r$ tales que:
$$n=qb+r, \; \; \; 0\leq r < b$$ \\
Demostración.- \; Sea $b$ ser un entero positivo fijo. Si $n=0$ , luego $q=r=0$, la afirmación es verdadera (ya que $0=0b+0$).\\
Ahora suponga que la afirmación es cierta para algunos $k \in \mathbb{N}$. Por hipótesis de inducción sabemos que existen enteros no negativos $q$ \; y \; $r$ tales que $$k=qb+r, \; \; \; 0\leq r < b,$$ Por lo tanto, sumando 1 a ambos lados tenemos, $$k+1=qb+(r+1).$$ Pues $0\leq r < b$ entonces sabemos que $0\leq r \leq (b-1)$. Si $0\leq r < b-1,$ entonces $0\leq r+1<b,$ y la declaración aún se mantiene con la misma elección $q$ \; y \; $r+1$ en lugar de $r$.\\
Por otro lado, si $r=b-1,$ entonces $r+1=b$ y tenemos, $$k+1=qb+b=(q+1)b+0$$
Por lo tanto, la declaración se mantiene de nuevo, pero con $q+1$ en lugar de $q$ \; y con $r=0$ (que es válido ya que si $r=0$ tenemos $a\leq r < b$). Por ende, si el algoritmo de división es válido para $k$, entonces también es válido para $k+1.$ Entonces, es válido para todos $n \in \mathbb{N}$.\\\\

%-------------------------------11-------------------------------
\item Sea $n$ \; y \; $d$ enteros. Se dice que $d$ es un divisor de $n$ si $n=cd$ para algún entero $c$. Un entero $n$ se denomina primo si $n>1$ y los únicos divisores de $n$ son $1$ \; y \; $n$. Demostrar por inducción que cada entero $n>1$ es o primo o producto de primos.\\\\
Demostración.- \; LA prueba se hará por inducción. Si $n=2$ ó $n=3$ entonces $n$ es primo, entonces la proposición es verdadera.\\
Ahora supongamos que la afirmación es verdadera para todos los enteros desde $2$ hasta $k$. Se debe demostrar que esto implica $k+1$ es primo o un producto de primos. Si $k+1$ es primo, entonces no hay nada que demostrar. Por otro lado, si $k+1$ no es primo, entonces sabemos que hay enteros $c$ \; y \; $d$ tal que $1<c,$ $d<k+1$ en otras palabras decimos que $n$ es divisible por números disntintos de $1$ y de sí mismo.\\
Por hipótesis de inducción, sabemos que $2\leq c$, $d\leq k$ entonces $c$ \; y \; $d$ son primos o son producto de primos.\\
Por lo tanto, si la declaración es verdadera para todos los enteros mayores que $1$ hasta $k$ entonces, también es verdadera para $k+1$, así es cierto para todo $n\in \mathbb{Z}^+$\\\\

%--------------------------12------------------------------
\item Explíquese el error en la siguiente demostración por inducción.\\
Proposición.- Dado un conjunto de n niñas rubias, si por 10 menos una de las niñas tiene ojos azules, entonces las n niñas tienen ojos azules.\\
Demostración.-\; La proposición es evidentemente cierta si $n = 1$. El paso de $k$ a $k + 1$ se puede ilustrar pasando de $n = 3$ a $n = 4$. Supóngase para ello que la proposición es cierta para $n=3$ Y sean $G_1, G_2, G_3, G_4$ cuatro niñas rubias tales que una de ellas, por lo menos, tenga ojos azules, por ejemplo, la $G_1$, Tomando $G_1,G_2, G_3,$ conjuntamente y haciendo uso de la proposición cierta para $n =3$, resulta que también $G_2$ y $G_3$ tienen ojos azules. Repitiendo el proceso con $G_1, G_2$ y $G_4,$ se encuentra igualmente que $G_4$ tiene ojos azules. Es decir, las cuatro tienen ojos azules. Un razonamiento análogo permite el paso de $k$ a $k + 1$ en general.\\
\textbf{Corolario.} Todas las niñas rubias tienen ojos azules.\\
Demostración.- \; Puesto que efectivamente existe una niña rubia con ojos azules, se puede aplicar el resultado precedente al conjunto formado por todas las niñas rubias.\\\\
Esta prueba supone que la afirmación es cierta n = 3, es decir, supone que si hay tres chicas rubias, una de las cuales tiene ojos azules, entonces todas tienen ojos azules. Claramente, esta es una suposición falsa.\\\\
\end{enumerate}

\setcounter{section}{6}
\section{Ejercicios}
\begin{enumerate}[\bfseries \Large 1.]
%---------------------------1---------------------------------
\item Hallar los valores numéricos de las sumas siguientes:
\begin{enumerate}[\bfseries a)]
%(a)
\item $\displaystyle\sum_{k=1}^{4} k = 1 + 2 + 3 + 4 = 10$\\\\
%(b)
\item $\displaystyle\sum_{n=2}^{5} 2^{n-2} = 2^{0} + 2^{1} + 2^{2} + 2^{3} = 15$\\\\
%(c)
\item $\displaystyle\sum_{r=0}^{3} 2^{2r+1} = 2^{1} + 2^{3} +  2^{5} + 2^{7} = 2 + 8 + 32 + 128 = 170$ \\\\
%(d)
\item $\displaystyle\sum_{n=1}^{4} n^n = 1^1 + 2^2 + 3^3 + 4^4 = 1 + 4 + 27 + 256= 288$ \\\\
%(e)
\item $\displaystyle\sum_{i=0}^{5} (2i + 1) = 1 + 3 + 5 + 7 + 9 + 11 = 36$ \\\\
%(f)
\item $\displaystyle\sum_{k=1}^{5} \dfrac{1}{k(k+1)} = \dfrac{1}{2} + \dfrac{1}{6} + \dfrac{1}{12} + \dfrac{1}{20} + \dfrac{1}{30} = 0,83333.....$ \\\\
\end{enumerate}

%-------------------------------2---------------------------
\item Establecer las siguientes propiedades del símbolo sumatorio. 
\begin{enumerate}[\bfseries a)]
%(a)
\item $\displaystyle\sum_{k=1}^{n}(a_k + b_k) = \sum_{k=1}^n a_k + \sum_{k=1}^{n} b_k$ (propiedad aditiva)
\begin{center}
\begin{tabular}{r c l l}
$\displaystyle\sum_{k=1}^{n}(a_k + b_k)$&$=$&$(a_1 + b_1) + (a_2 + b_2) + ... + (a_n + b_n)$&\\\\
&$=$&$(a_1 + a_2 + ... + a_n) + (b_1 + b_2 + ... + b_n)$&Asociatividad y conmutatividad\\\\
&$=$&$\displaystyle\sum_{k=1}^{n} a_k + \sum_{k=1}^{n} b_k$&\\\\
\end{tabular}
\end{center}
%(b)
\item $\displaystyle\sum_{k=1}^{n} (ca_k) = c \sum_{k=1}^{n} a_k$ (Prpiedad homogénea)
\begin{center}
\begin{tabular}{r c l l}
$\displaystyle\sum_{k=1}^{n} (ca_k)$&$=$&$ca_1 + ca_2 + ... + ca_n$&\\\\
&$=$&$c(a_1 + a_2 +...+a_n)$&distributividad\\\\
&$=$&$ c \displaystyle\sum_{k=1}^{n} a$&\\\\
\end{tabular}
\end{center}
%(c)
\item $\displaystyle\sum_{k=1}^{n} (a_k - a_{k-1}) = a_n - a_0$ (Propiedad telescópica)
\begin{center}
\begin{tabular}{r c l l}
$\displaystyle\sum_{k=1}^{n} (a_k - a_{k-1})$&$=$&$(a_1-a_0)+(a_2-a_1)+...+(a_n-a_{n-1})$&\\
&$=$&$(a_1+a_2+...+a_{n-1})-(a_0+a_1+...+a_{n-1})+a_n$&\\\\
&$=$&$a_n + \displaystyle\sum_{k=1}^{n-1} a_k - \sum_{k=0}^{n-1} a_k$&Reindexando la 2da suma\\\\
&$=$&$a_n + \displaystyle\sum_{k=1}^{n-1} a_k - a_0 - \sum_{k=1}^{n-1} a_k$&\\\\
&$=$&$a_n - a_0$&\\\\
\end{tabular}
\end{center}
\end{enumerate}

%--------------------------3-------------------------------
\item $\displaystyle\sum_{k=1}^{n} 1 = n$ (El sentido de esta suma es $\displaystyle\sum_{k=1}^{n} a_k$, cuando $a_k=1$)\\\\
Demostración.- \; Probemos por inducción, Si $n=1$, entonces la proposición es verdadera ya que $\displaystyle\sum_{k=1}^{1} 1 = 1.$ Ahora supongamos que el enunciado es verdadero para $n=m \in \mathbb{Z}^+.$ Luego, $$\displaystyle\sum_{k=1}^m 1 = m \Rightarrow \left( \sum_{k=1}^m 1 = m \right) + 1 = m + 1 \Rightarrow \sum_{k=1}^{m+1} 1 = m+1$$\\\\

%---------------------------4---------------------------------
\item $\displaystyle\sum_{k=1}^{n} (2k - 1) = n^2$ $[Indicación, \; 2k-1 = k^2 - (k-1)^2]$\\\\
Demostración.- \; Sea $2k-1 = k^2 - (k^2 - 2k + 1) = k^2 - (k-1)^2$ entonces  por la propiedad telescópica se tiene $\sum\limits_{k=1}^n (2k-1)= \sum\limits_{k=1}^{n} (k^2 - (k-1)^2) = n^2 + 0 = n^2$\\\\

%------------------------------5--------------------------
\item $\displaystyle\sum_{k=1}^{n} k = \dfrac{n^2}{2} + \dfrac{n}{2} $ \; \; [indicación. \; Úsese el ejercicio 3 y el 4.]\\\\
Demostración.- \;Por aditividad y homogeneidad se tiene $\sum\limits_{k=1}^{n}(2k-1) = 2\sum\limits_{k=1}^{n} k - \sum\limits_{k=1}^{n} 1 = n^2$ entonces $ 2 \sum\limits_{k=1}^{n} k - n =n^2$ ya que $\sum\limits_{k=1}^{n} 1 = n$, luego $\sum\limits_{k=1}^{n} k = \dfrac{n^2}{2} + \dfrac{n}{2}$\\\\

%------------------------------6----------------------------
\item $\displaystyle\sum_{k=1}^{n} k^2 = \dfrac{n^3}{3} + \dfrac{n^2}{2} + \dfrac{n}{6}$ $[Indicación. \; k^3 - (k-1)^3 = 3k^2 - 2k + 1]$\\\\
Demostración.- \;  Por la propiedad telescópica se tiene $\sum\limits_{k=1}^n k^3 - (k-1)^3 = n^3$, luego 
\begin{center}
\begin{tabular}{r c l l}
$n^3 = \sum\limits_{k=1}^n 3k^2-3k+1$&$\Rightarrow$&$n^3 = \sum\limits_{k=1}^n k^2 - 3 \sum\limits_{k=1}^n k + \sum\limits_{k=1}^n 1$&propiedad aditiva\\\\
&$\Rightarrow$&$\sum\limits_{k=1}^n k^2 = \dfrac{n^3}{3} + \sum\limits_{k=1}^n k - \dfrac{1}{3} \sum\limits_{k=1}^n 1$&\\\\
&$\Rightarrow$&$\sum\limits_{k=1}^n k^2 = \dfrac{n^3}{3} + \dfrac{n^2}{2} + \dfrac{n}{2} - \dfrac{n}{3}$&por los anteriores ejercicios\\\\
&$\Rightarrow$&$\sum\limits_{k=1}^n k^2 = \dfrac{n^3}{3} + \dfrac{n^2}{2} + \dfrac{n}{6}$&\\\\
\end{tabular}
\end{center}

%----------------------------7--------------------------------
\item $\displaystyle\sum_{k=1}^n k^3 = \dfrac{n^4}{4} + \dfrac{n^3}{2} + \dfrac{n^2}{4}$\\\\
Demostración.- \; Sea $k^4 - (k-1)^4 = 4k^3 - 6k^2 + 4k -1$ entonces  
\begin{center}
\begin{tabular}{r c l}
$n^4 = 4 \sum\limits_{k=1}^n k^3 - 6 \sum\limits_{k=1}^n k^2 + 4 \sum\limits_{k=1}^n k - \sum\limits_{k=1}^n 1$&$\Rightarrow$&$n^4 = 4 \sum\limits_{k=1}^n k^3 - 6 \left( \dfrac{n^3}{3} + \dfrac{n^2}{2} + \dfrac{n}{6} \right) + 4 \left( \dfrac{n^2}{2} + \dfrac{n}{2} \right) - n$\\\\
&$\Rightarrow$&$4 \sum\limits_{k=1}^n k^3 = n^4 + 2n^3 + 3n^2 - 2n^2$\\\\
&$\Rightarrow$&$\sum\limits_{k=1}^n k^3 = \dfrac{n^4}{4} + \dfrac{n^3}{2} + \dfrac{n^2}{4}$\\\\
\end{tabular}
\end{center}

%----------------------------8------------------------------------
\item 
\begin{enumerate}[\bfseries a)]
%a)
\item $\displaystyle\sum_{k=0}^n x^k = \dfrac{1 - x^{n+1}}{1-x}$ si $x\neq 1$. Nota: Por definición $x^0 = 1$ [Indicación. Aplíquese el ejercicio 2 a $(1-x) \sum\limits_{k=0}^n x^k.$]\\\\
Demostración.- Por propiedades  aditiva y telescópica, 
$$(1-x) \sum\limits_{k=0}^n x^k =\sum\limits_{k=0}^n (x^k - x^{k+1}) = - \sum\limits_{k=0}^n (x^{k+1} - x^k)=-(x^{n+1}-1) = 1-x^{n+1}$$ 
pro lo tanto, $$\sum\limits_{k=0}^n x^k = \dfrac{1 - x^{n+1}}{1-x}$$\\

%b)
\item ¿Cuál es la suma cuando $x=1$?\\\\
Respuesta.- \; 	Si $x=1$ por la parte 3 y el hecho de que $1^k=1$ para $k=0,...,n$ tenemos $$\sum\limits_{k=0}^n x^k = \sum\limits_{k=0}^n 1 = n+1$$\\ 
\end{enumerate}

%----------------------------9-----------------------------
\item Demostrar por inducción, que la suma $\sum\limits_{k=1}^{2n} (-1)^k (2k+1)$ es proporcional a $n$, y hallar la constante de proporcionalidad.\\\\
Demostración.- \; Sea $n=1$ entonces $\sum\limits_{k=1}^2 (-1)^k (2k+1) = 2$. Así vemos que es valida para este caso. Observemos que  $2n=2$, lo mismo pasa para $n=2$ donde $2n=4.$
Luego supongamos que la fórmula es válida para $n=m \in \mathbb{Z}^+$, es decir $$\sum\limits_{k=0}^2m (-1)^k (2k+1) = 2m,$$ entonces
\begin{center}
\begin{tabular}{r c l}
$\sum\limits_{k=1}^{2(m+1)} (-1)^k(2k+1)$&$=$&$(-1)(2k+1)$\\\\
&$=$&$2m-[2(2m+1)+1] + [2(2m+2)+1]$\\\\
&$=$&$2m-4m-3+4m+5$\\\\
&$=$&$2(m+1)$\\\\
\end{tabular}
\end{center}

%---------------------------10-----------------------------
\item 
\begin{enumerate}[\bfseries a)]
%a)
\item Dar una definición razonable del símbolo $\sum\limits_{k=m}^{m+n} a_k$\\\\
$$\sum\limits_{k=m}^{m+n} a_k = a_m + a_{m+1} + ... + a_{m+n}$$\\

%b)
\item Demostrar por inducción que para $n\geq 1$ se tiene
$$\displaystyle\sum_{k=n+1}^{2n} \dfrac{1}{k} = \sum_{m=1}{2n}\dfrac{(-1)^{m+1}}{m}$$\\\\
Demostración.- \; Sea $n=1$ se tiene,
$$\sum\limits_{k=n+1}^{2n} \dfrac{1}{k} = \sum\limits_{k=2}^{2} \dfrac{1}{k} = \dfrac{1}{2}$$
luego, $$\sum\limits_{m=1}^{2n} \dfrac{(-1)^{m+1}}{m} = \sum\limits_{m=1}^{2} \dfrac{(-1)^{m+1}}{m} = 1 - \dfrac{1}{2} = \dfrac{1}{2}$$
Ahora supongamos que se cumple para $n=j \in \mathbb{Z}^+$ entonces,
\begin{center}
\begin{tabular}{r c l}
$\sum\limits_{k=n+1}^{2n} \dfrac{1}{k} = \sum\limits_{m=1}^{2n}\dfrac{(-1)^{m+1}}{m}$&$\Rightarrow$&$\left( \sum\limits_{k=j+1}^{2j} \dfrac{1}{k} \right) + \dfrac{1}{2j+1} + \dfrac{1}{2j+2} = \left( \sum\limits_{m=1}^{2j} \dfrac{(1)^{m+1}}{m} \right) + \dfrac{1}{2j+1} + \dfrac{1}{2j+2}$\\\\
&$\Rightarrow$&$\left( \sum\limits_{k=j+1}^{2(j+1)} \dfrac{1}{k} \right)  = \left( \sum\limits_{m=1}^{2j} \dfrac{(1)^{m+1}}{m} \right) + \dfrac{1}{2j+1} + \dfrac{1}{2j+2}$\\\\
&$\Rightarrow$&$\dfrac{1}{j+1} + \left( \sum\limits_{k=j+2}^{2(j+1)} \dfrac{1}{k} \right) = \left( \sum\limits_{m=1}^{2j} \dfrac{(1)^{m+1}}{m} \right) + \dfrac{1}{2j+1} + \dfrac{1}{2j+2}$\\\\
&$\Rightarrow$&$\sum\limits_{k=j+2}^{2(j+1)} \dfrac{1}{k} = \left( \sum\limits_{m=1}^{2j} \dfrac{(1)^{m+1}}{m} \right) + \dfrac{1}{2j+1} + \dfrac{1}{2j+2} - \dfrac{1}{j+1}$\\\\
&$\Rightarrow$&$\sum\limits_{k=j+2}^{2(j+1)} \dfrac{1}{k} = \left( \sum\limits_{m=1}^{2j} \dfrac{(1)^{m+1}}{m} \right)  - \dfrac{1}{2(j+1)}$\\\\
&$\Rightarrow$&$\sum\limits_{k=j+2}^{2(j+1)} \dfrac{1}{k} = \left( \sum\limits_{m=1}^{2(j+1)} \dfrac{(1)^{m+1}}{m} \right)$\\\\
\end{tabular}
\end{center}
\end{enumerate}

%----------------------------11---------------------------
\item Determinar si cada una de las igualdades siguientes es cierta o falsa. En cada caso razonar la decisión.
\begin{enumerate}[\bfseries (a)]
%(a)
\item $\displaystyle\sum_{n=0}^{100} n^4= \sum_{n=1}^{100} n^4$\\\\
Razonamiento.- \; Dado que se se evalúa para $n=0$ entonces se tiene que $0^4 = 0$ entonces la desigualdad es cierta.\\\\

%(b)
\item $\displaystyle\sum_{j=0}^{100} 2 = 200$\\\\
Razonamiento.- \; En vista que de si se evalúa para $n=0$ el resultado es $2$ entonces para $n=100$ será $202$, por lo tanto la igualdad es falsa\\\\

%(c)
\item $\displaystyle\sum_{k=0}^{100} (2+k) = 2 + \sum_{k=0}^{100} k$\\\\
Razonamiento.- \; La igualdad es falsa debido a que el lado derecho de la igualdad se añade el $2$ solo una vez, a diferencia de la igualdad de la izquierda que se añade a cada iteración de la suma.\\\\

%(d)
\item $\displaystyle\sum_{i=1}^{100} (i+1)^2 = \sum_{i=0}^{99} i^2$\\\\
Razonamiento.- \; La igualdad es falsa ya que al cuadrado de la serie de la izquierda se añade en $1$ a cada iteración, contemplando que $n=100$\\\\ 

%(e)
\item $\displaystyle\sum_{k=1}^{100} k^3 = \left( \sum_{k=1}^{100} k \right)\cdot \left(  \sum_{k=1}^{100} k^2 \right)$\\\\
Razonamiento.- \; La igualdad es falsa, ya que anteriormente se dijo que $$\sum\limits_{k=1}^{n} \dfrac{n^4}{4} + \dfrac{n^3}{2} + \dfrac{n^2}{4},$$ mientras que $$\left( \sum\limits_{k=1}^{n} k \right)\left( \sum\limits_{k=1}^{n} k^2 \right)$$ \\\\

%(f)
\item $\displaystyle\sum_{k=0}^{100} k^3 = \left( \sum_{k=0}^{100} k \right) ^3$\\\\
Razonamiento.- \; Similar a  la parte $(e)$ se tiene $$\sum\limits_{k=1}^{n} \dfrac{n^4}{4} + \dfrac{n^3}{2} + \dfrac{n^2}{4},$$ mientras que $$\left( \sum\limits_{k=0}^{n} k \right)^3 = \left( \dfrac{n^2}{2} + \dfrac{n}{2}\right)^3$$\\\\
\end{enumerate} 

%------------------------------12-------------------------------------
\item Inducir y demostrar una regla general que simplifique la suma $$\displaystyle\sum_{k=1}^{n} \dfrac{1}{k(k+1)}$$\\\\
Demostración.- \; Esta claro ver que $\sum\limits_{k=1}^{n} \dfrac{1}{k(k+1)} = \dfrac{k}{k+1}$ ya que si sumamos para $n=2$ el resultado es $\dfrac{2}{3}$, para $n=3$ $\dfrac{3}{4}$ y así sucesivamente.\\
Luego efectivamente se cumple para $n=1$,  $\dfrac{1}{1(1+1)} = \dfrac{1}{2} = \dfrac{1}{1+1} = \dfrac{1}{2}$. 
Así asumimos que se cumple para un entero fijo $n=m$,
$$\sum\limits_{k=1}^{m} \dfrac{1}{k(k+1)} = \dfrac{k}{k+1}$$
Ahora supongamos que se cumple para $m+1$ por lo tanto $$\sum\limits_{k=1}^{m+1} \dfrac{1}{k(k+1)} = \sum\limits_{k=1}^{m} \dfrac{1}{k(k+1)} + \dfrac{1}{(k+1)(k+2)} = \dfrac{k+1}{k+2}$$
Por último solo falta demostrar que $ \dfrac{k}{k+1} + \dfrac{1}{(k+1)(k+2)} = \dfrac{k+1}{k+2}$.\\\\
$$\dfrac{k(k+2)+1}{(k+1)(k+2)} = \dfrac{k^2 + 2k +1}{(k+1)(k+2)} = \dfrac{(k+1)(k+1)}{(k+1)(k+2)} = \dfrac{k+1}{k+2}$$\\\\

%------------------------------13----------------------------------
\item Demostrar que $2(\sqrt{n+1} - \sqrt{n}) < \dfrac{1}{\sqrt{n}}< 2(\sqrt{n} - \sqrt{n-1})$ si $n\geq 1.$ Utilizar luego este resultado para demostrar que
$$2\sqrt{m} - 2 < \sum\limits_{n=1}^m \dfrac{1}{\sqrt{n}} < 2 \sqrt{m} - 1$$
si $m \geq 2.$ En particular, cuando $m=10^6$, la suma está comprendida entre $1998$ y $1999.$\\\\
Demostración.- \; Sea  $0<1$  entonces 
\begin{center}
\begin{tabular}{r c l}
$0$&$<$&$1$\\
$4n^2 + 4n$&$<$&$4n^2 + 4n + 1$\\
$4n(n+1)$&$<$&$4n^2 + 4n + 1$\\
$2\sqrt{n} \sqrt{n+1}$&$<$&$2n+1$\\
$2\sqrt{n+1} - 2\sqrt{n}$&$<$&$\dfrac{1}{\sqrt{n}}$\\
$2(\sqrt{n+1} - \sqrt{n})$&$<$&$\dfrac{1}{\sqrt{n}}$\\\\
\end{tabular}
\end{center} 
Análogamente se puede demostrar para $\dfrac{1}{\sqrt{n}} < 2(\sqrt{n} - \sqrt{n-1})$\\\\
Ahora demostramos la segunda parte del enunciado. Consideremos la desigualdad de la izquierda. Para $m=2$ tenemos,
$$2\sqrt{2} -2 ; \; \; 1 + \dfrac{1}{\sqrt{2}}$$ 
\begin{center}
\begin{tabular}{r c l l}
$2\sqrt{2} - 2 $&$<$&$2 \sqrt{2} - \sqrt{2}$& ya que $2 > \sqrt{2}$\\
&$=$&$\sqrt{2}$&\\
&$=$&$\dfrac{1}{\sqrt{2}} + \dfrac{1}{\sqrt{2}}$&ya que $\sqrt{2} = 2\sqrt{2}$\\
&$<$&$1 + \dfrac{1}{\sqrt{2}}$&\\
\end{tabular}
\end{center} 
donde la inecuación se cumple. Luego asumimos que la inecuación se cumple para algún $k \in \mathbb{Z}$ para $k\geq 2,$ entonces sea
\begin{center}
\begin{tabular}{r c l l}
$2\sqrt{2} -2 < \sum\limits_{n=1}^k \dfrac{1}{n}$&$\Rightarrow$&$2\sqrt{2} -2 + \dfrac{1}{\sqrt{k+1}}< \sum\limits_{n=1}^{k+1} \dfrac{1}{n}$&\\\\
&$\Rightarrow$&$2\sqrt{2} - 2 + 2(\sqrt{k+2} - \sqrt{k+1}) < \sum\limits_{n=1}^{k+1} \dfrac{1}{\sqrt{n}}$& por la parte 1 y $\sum\limits_{n=1}^{k+1} \dfrac{1}{n}<\sum\limits_{n=1}^{k+1} \dfrac{1}{\sqrt{n}}$\\\\
&$\Rightarrow$&$2(\sqrt{k+1} - \sqrt{k+1} + 2\sqrt{k} -2) < \sum\limits_{n=1}^{k+1} \dfrac{1}{\sqrt{n}}$&usando la parte 1\\\\
&$\Rightarrow$&$2\sqrt{k+1} - 2 < \sum_{n=1}^{k+1} \dfrac{1}{\sqrt{n}}$&\\\\
\end{tabular}
\end{center}
\end{enumerate}
Por lo tanto la inecuación es verdadera para todo $m \in \mathbb{Z}$, $k\geq 2.$\\
Ahora veamos la inecuación de la derecha. Para el caso de $m=2$ tenemos,
$$+1\dfrac{1}{\sqrt{2}}; \; \; 2\sqrt{2} - 1$$
luego como $\sqrt{2} = \dfrac{2\sqrt{2}}{2}< \dfrac{3}{2}$ tenemos,
\begin{center}
\begin{tabular}{rcll}
$\sqrt{2}< \dfrac{3}{2}$&$\Rightarrow$&$2\sqrt{2}<3$&\\
&$\Rightarrow$&$2\sqrt{2} + 1 < 4$&\\
&$\Rightarrow$&$2 + \dfrac{1}{\sqrt{2}}< 2\sqrt{2}$&\\
&$\Rightarrow$&$1 + \dfrac{1}{\sqrt{2}}<2 \sqrt{2} -1$&\\\\
\end{tabular}
\end{center} 
por lo tanto la desigualdad es cierta. Asumamos entonces que es cierto para algunos $m = k \in \mathbb{Z}, \; k\geq 2$, luego
\begin{center}
\begin{tabular}{rcll}
$\sum\limits_{n=1}^k \dfrac{1}{\sqrt{n}} < 2\sqrt{k} - 1$&$\Rightarrow$&$\left( \sum\limits_{n=1}^k \dfrac{1}{\sqrt{n}} + \dfrac{1}{\sqrt{k+1}} < 2 \sqrt{2} -1 + \dfrac{1}{\sqrt{k+1}}\right)$&\\\\
&$\Rightarrow$&$\sum\limits_{n=1}^{k+1} \dfrac{1}{\sqrt{n}} < 2\sqrt{k} - 1 + 2(\sqrt{k+1} - \sqrt{k})$&parte 1\\\\
&$\Rightarrow$&$\sum\limits_{n=1}^{k+1} \dfrac{1}{\sqrt{n}} < 2\sqrt{k+1} - 1$&parte 1 \\\\
\end{tabular}
\end{center}
Por lo tanto la desigualdad correcta se aplica a todo $m \in \mathbb{Z} k \geq 2$.
\textbf{Cabe recalcar que el libro menciona que $m\geq 2$} pero este último solo se cumple para los extremos de la desigualdad y no así para $\sum\limits_{n=1}^m \dfrac{1}{\sqrt{2}}$\\\\

\section{Valor absoluto y desigualdad triangular}
\begin{tcolorbox}[colframe=white]
%definición valor absoluto
\begin{def.}
$$|x| = \left\lbrace
\begin{array}{crr}
\textup{si } & x, & x\geq 0\\
\textup{si }& -x, & x \leq 0
\end{array}
\right.$$
\end{def.}
\end{tcolorbox}

%teorema 1.38
\begin{teo}
Si $a\geq 0$, es \; $|x|\leq a$ \; si y sólo si \; $-a\leq x \leq a$\\\\
Demostración.- \; Debemos probar dos cuestiones: primero, que la desiguadad $|x|\leq a$ implica las dos desigualdades $-a \leq x \leq a$ y recíprocamente, que $-a \leq x \leq a$ implica $|x|\leq a$.\\ 
Ya supuesto $|x|\leq a$ se tiene también $-a \leq - |x|$. Pero ó \; $x=|x|$ \; ó \; $x = -|x|$ y, por lo tanto, $x\leq a$ \; y \; $-a \leq x$, lo cual prueba la primera parte del teorema.\\
Para probar el recíproco, supóngase $-a\leq x \leq a$. Si $x\leq 0$ se tiene $|x|=x\leq a$; si por el contrario es $x\leq 0$ , entonces $|x|=-x \leq a$. En ambos casos se tiene $|x|\leq a$, lo que demuestra el teorema.\\\\   
\end{teo}

%teorema 1.39
\begin{teo}[Desigualdad triangular]
Para $x$ e $y$ números reales cualesquiera se tiene $$|x+y| \leq |x| + |y|$$\\\\
Demostración.- \; Puesto que $x=|x|$ \; ó \; $x=-|x|$, se tiene $-|x|\leq x \leq |x|$. Análogamente $-|y| \leq y \leq |y|$. Sumando ambas desigualdades se tiene: $$-(|x|+|y|)\leq x+y \leq |x|+|y|$$ y por tanto en virtud del teorema anterior  se concluye que: $|x+y|\leq |x|+|y|$\\\\  
\end{teo}

%teorema 1.40
\begin{teo}
Si $a_1, a_2, ..., a_n$ son números reales cualesquiera $$\left|\displaystyle\sum_{k=1}^n a_k\right| \leq \sum_{k=1}^n |a_k|$$\\\\
Demostración.- \; Para $n=1$ la desigualdad es trivial y para $n=2$ es la desigualdad triangular. Supuesta cierta para $n$ números reales, para $n+1$ números reales $a_1,a_2,...,a_{n+1}$ se tiene:
$$\left| \sum\limits_{k=1}^{n+1} a_k \right|= \left| \sum\limits_{k=1}^n + a_{n+1}\right| \leq \left| \sum\limits_{k=1}^n  \right| + |a_{n+1}| \leq \sum\limits_{k=1}^n |a_k| + |a_{n+1}| = \sum\limits_{k=1}^{n+1} |a_k|$$
Por tanto, el teorema es cierto para $n+1$ números si lo es para $n;$ luego, en virtud deñ principio de inducción, es cierto para todo número positivo $n.$\\\\
\end{teo}

%teorema 1.41
\begin{teo}[Desigualdad de Cauchy-Schwarz]
Si $a_1,...,a_n$ y $b_1,...,b_n$ son números reales cualesquiera, se tiene
$$\left( \displaystyle\sum_{k=1}^n a_k b_k \right)^2 \leq \left( \sum_{k=1}^n a_k^2 \right) \left( \sum_{k=1}^n b_k^2 \right)$$
El signo de igualdad es válido si y sólo si hay un npumero real $x$ tal que $a_k x + b_k = 0$ ára cada vañpr de $k=1,2,...,n$\\\\
Demostración.- \; Para $\forall x \in \mathbb{R}$ se tiene $\sum\limits_{k=1}^n (a_kx + b_k)^2 \geq 0$. Esto se puede poner en la forma $$Ax^2 + 2Bx + C \geq 0,$$ donde $$A=\sum\limits_{k=1}^n a_k^2 , \; \; B=\sum\limits_{k=1}^n a_k b_k, \; \; C = \sum\limits_{k=1}^n b_k^2.$$ Queremos demostrar que $B^2 \leq AC.$ Si $A=0$ cada $a_k=0$, con lo que $B=0$ y el resultado es trivial. Si $A \neq 0,$ podemos completar el cuadrado y escribir $$Ax^2 + 2Bx + C = A \left( x + \dfrac{B}{A} \right)^2 + \dfrac{AC - B^2}{A}$$
El segundo miembro alcanza su valor mínimo cuando $x=-\dfrac{B}{A}$. Poniendo $x=- \dfrac{B}{A}$ en la primera ecuación, obtenemos $B^2 \leq AC.$ Esto demuestra la desigualdad dada.\\\\
\end{teo}

\section{Ejercicios}
\begin{enumerate}[\Large \bfseries 1.]
%----------------------------1.---------------------------------
\item Probar cada una de las siguientes propiedades del valor absoluto.\\
\begin{enumerate}[\bfseries (a)]
%(a)
\item $|x|=0$ si y sólo si $x=0$\\\\
Demostración.- \; Si $|x|=0$, por definición $x=0$. Luego, si $x=0$, entonces por teorema $\sqrt{x^2}=\sqrt{0^2}=0$\\\\

%(b)
\item $|-x|=|x|$\\\\
Demostración.- \; Por definición $|-x|=-(-x)$ si $x\leq 0$ y $|-x|=x$ si $x\geq 0$ por lo tanto $|-x|=x$\\\\ 

%(c)
\item $|x-y|=|y-x|$\\\\
Demostración.- \; Por teorema $|x-y|=\sqrt{(x-y)^2}=\sqrt{x^2-2xy+y^2}=\sqrt{y^2-2yx+x^2}=\sqrt{(y-x)^2}=|y-x|$\\\\ 

%(d)
\item $|x|^2=x^2$\\\\
Demostración.- \; Si $|x|^2$, por teorema $\left( \sqrt{x^2} \right)^2$, por propiedad de potencia $x^2$\\\\

%(e)
\item $|x|=\sqrt{x^2}$\\\\
Demostración.- \; Sea $x^2\geq 0$ entonces por teorema $\sqrt{x^2}$, luego $(x^2)^{\frac{1}{2}}$, por lo tanto por definición de valor absoluto $x=|x|$.\\\\ 

%(f)
\item $|xy|=|x||y|$\\\\
Demostración.- \; Por el teorema anterior $|xy|=\sqrt{(xy)^2}=\sqrt{x^2} \sqrt{y^2} = |x||y|$\\\\ 

%(g)
\item $|x/y|=|x|/|y|$ si $y \neq 0$\\\\
Demostración.- \; Similar al anterior problema se tiene $|x/y|=\sqrt{(x/y)^2}$, luego por propiedades de  potencia y raíces $\sqrt{x^2}/\sqrt{y^2}$, así nos queda $|x|/|y|$ si $y \neq 0$\\\\

%(h)
\item $|x-y|\leq |x|+|y|$\\\\
Demostración.- \; Por la desigualdad triangular y la parte $(b)$ se tiene $|x-y| = |x+(-y)| = |x|+ |-y| = |x|+|y|$\\\\

%(i)
\item $|x|-|y| \leq |x-y|$\\\\
Demostración.- \; Por la desigualdad triangular tenemos  que $|x|= |(x-y)+y| \leq |x-y|+|y|$ entonces $|x|-|y| \leq |x-y|$\\\\

%(j)
\item $\left| |x| - |y| \right| \leq |x-y|$\\\\
Demostración.- \; Parecido al anterior ejercicio se tiene $|y|=|x+(y-x)| \leq |x|+|y-x|$ entonces $|y|-|x| \leq |x-y|$ y por lo tanto $|x|-|y|\geq |x-y|$. Luego por el inciso (i) y teorema $-|x-y|\leq |x|-|y| \leq |x-y| \Rightarrow \left||x|-|y|\right| \leq |x-y|$\\\\ 

\end{enumerate}

%---------------------------------------------2----------------------------------------------
\item Cada desigualdad $(a_i)$, de las escritas a continuación, equivale exactamente a una desigualdad $(b_j)$. Por ejemplo, $|x|<3$ si y sólo si $-3<x<3$ y por tanto $(a_1)$ es equivalente a $(b_2)$. Determinar todos los pares equivalentes.\\
\begin{center}
\begin{tabular}{l c l}
$|x|<3$&$\longrightarrow$&$-3<x<3$\\\\
$|x-1|<3$&$\longrightarrow$&$-2<x<4$\\\\
$|3-2x|<1$&$\longrightarrow$&$1<x<2$\\\\
$|1+2x|\leq 1$&$\longrightarrow$&$-1\leq x \leq 0$\\\\
$|x-1|>2$&$\longrightarrow$&$x>4 \; \lor \; x<-1$\\\\
$|x+2| \geq 5$&$\longrightarrow$&$x\geq 3 \; \lor \; x\leq -7$\\\\
$|5-x^{-1}|<1 $&$\longrightarrow$&$4<x<6$\\\\
$|x-5|<|x+1|$&$\longrightarrow$&$x>2$\\\\
$|x^2-2|\leq 1$&$\longrightarrow$&$-\sqrt{3} \leq x \leq -1 \; ó \; 1\leq x \leq \sqrt{3}$\\\\
$x<x^2-12<4x$&$\longrightarrow$&$\dfrac{1}{6}<x<\dfrac{1}{4}$\\\\
\end{tabular}
\end{center}

%---------------------------------------3----------------------------------------
\item Decidir si cada una de las siguientes afirmaciones es cierta o falsa. En cada caso razonar la decisión.
\begin{enumerate}[\bfseries (a)]
\item $x<5$ implica $|x|<5$\\\\
Es falso ya que $|-6|<5$ entonces $6>5$.\\\\
\item $|x-5|<2$ implica $3<x<7$\\\\
Es verdad ya que por teorema $-2 < x-5 < 2$ entonces $3<x<7$.\\\\
\item $|1 + 3x| \leq 1$ implica $x > - \dfrac{2}{3}$ \\\\
es verdad ya que $-1\leq 1 + 3x \leq 1$ entonces $-\dfrac{2}{3} \leq x \leq 0$.\\\\
\item No existe número real $x$ para el que $|x-1|= |x-2|$\\\\
Es falso ya que se cumple para $\dfrac{3}{2}$.\\\\
\item Para todo $x>0$ existe un $y>0$ tal que $|2x + y|=5$\\\\
Es falso ya que si tomas $x=3$ será $y<0$\\\\
\end{enumerate}

%---------------------------------------4----------------------------------------------
\item Demostrar que el signo de igualdad es válido en la desigualdad de Cauchy-Schwarz si y sólo si existe un número real tal que $a_k x + b_k = 0$ para todo $k=1,2,...,n$\\\\
Demostración.- $(\Rightarrow)$ \; Si $a_k=0$ para todo $k$ entonces la igualdad es verdadera así asumimos que $a_k\neq 0$ para al menos un $k$, $$\left( \displaystyle\sum_{k=1}^n a_k b_k \right)^2 = \left( \sum_{k=1}^n a_k^2 \right) \left( \sum_{k=1}^n b_k^2 \right)$$, Luego, sea $\sum\limits_{k=1}^n (a_k x + b_k)^2 \geq 0$, entonces $$A=\sum\limits_{k=1}^n a_k^2, \; \; B=\sum\limits_{k=1}^n b_k a_k, \; \; C= \sum\limits_{k=1}^n b_k^2$$, así se tiene $$Ax^2+2b^x + C \geq 0 \Rightarrow x= \dfrac{-2B \pm \sqrt{4B^2 - 4AC}}{2A} = \dfrac{-B \pm \sqrt{B^2 - AC}}{A}$$ pero sabemos por suposición que $B^2=AC$, donde $x=- \dfrac{B}{A}$ el cual esta en $\mathbb{R}$ ya que $A\neq 0$ y por lo tanto $a_k \neq 0$ para algún $k$ y $a_k^2$ es no negativo. asís que la suma es estrictamente positivo como se vio en la desigualdad de Cauchy-Schwarz.\\\\
$(\Leftarrow)$ Supongamos que existe $x\in \mathbb{R}$ tal que $a_k x + b_k=0$ para cada $k=1,...,n.$ Luego $a_k x + b_k = 0 \Rightarrow b_k =(-x)a_k.$ Entonces,
\begin{center}
\begin{tabular}{r c l}
$\left( \sum\limits_{k=1}^n a_k b_k \right)$&$=$&$\left[ -x \left( \sum\limits_{k=1}^n a_k^2 \right)\right]^2$\\\\
&$=$&$x^2 \left( \sum\limits_{k=1}^n a_k^2 \right)\left( \sum\limits_{k=1}^n a_k^2 \right)$\\\\
&$=$&$\left( \sum\limits_{k=1}^n a_k^2 \right) \left( \sum\limits_{k=1}^n x^2 a_k^2 \right)$\\\\
&$=$&$\left( \sum\limits_{k=1}^n a_k^2\right) \left( \sum\limits_{k=1}^n  (-xa_k)^2\right)$\\\\
&$=$&$\left( \sum\limits_{k=1}^n a_k^2 \right)\left( \sum\limits_{k=1}^n b_k^2\right)$\\\\
\end{tabular}
\end{center}
\end{enumerate}

\section{Ejercicios varios referentes al método de inducción}
%definición coeficiente factorial y binomial
\begin{tcolorbox}[colframe=white]
\begin{def.}[Coeficiente factorial y binomial]
El símbolo $n!$ (que se lee $n$ factorial) se puede definir por inducción como sigue: $0!=1, \; n!=(n-1!n$ si $n\geq 1$\\
Obsérvese que $n!=1\cdot 2 \cdot 3 \cdot \cdot \cdot n.$\\
Si $0\leq k \leq n$ el coeficiente binomial ${n \choose k}$ se define por 
$${n \choose k} = \dfrac{n!}{k!(n-k)!}$$
\end{def.}
\end{tcolorbox}
\begin{enumerate}[\Large \bfseries 1.]

%------------------------------------------1--------------------------------------
\item Calcúlese los valores de los siguientes coeficientes binomiales:
\begin{enumerate}[\bfseries (a)]
%------------------a---------------------
\item ${5 \choose 3} = \dfrac{5!}{3!(5-3)!} = \dfrac{4\cdot 5}{2\cdot 1} = 10$\\\\

%------------------b---------------------
\item ${ 7 \choose 0 } = \dfrac{7!}{0!(7-0)!} = \dfrac{7!}{7!} = 1$\\\\

%------------------c---------------------
\item ${ 7 \choose 1 } = \dfrac{7!}{1!(7-1)!} = \dfrac{7!}{6!} = 7$\\\\

%------------------d---------------------
\item ${ 7 \choose 2 } = \dfrac{7!}{2!(7-2)!} = \dfrac{6\cdot 7}{2} = 21$\\\\

%------------------e---------------------
\item ${ 17 \choose 14 } = \dfrac{17!}{14!(17-14)!} = \dfrac{15 \cdot 16 \cdot 17}{3\cdot 2 \cdot 1} = 680$\\\\

%------------------f---------------------
\item ${ 0 \choose 0 } = \dfrac{0!}{0!(0-0)!} = 1$\\\\
\end{enumerate}

%--------------------------------2---------------------------------------------
\item 
\begin{enumerate}[\bfseries (a)]
%-------------------(a)-----------------------------
\item Demostrar que: ${n \choose k} = {n \choose n-k}$\\\\
Demostración.- \; Sea $\dfrac{n!}{(n-k)!\left[ n - (n-k) \right]!}$ entonces $\dfrac{n!}{(n-k)!k}$ por lo tanto ${n \choose k}$\\\\

%-------------------(b)-----------------------------
\item Sabiendo que ${n \choose 10} = {n \choose 7}$ calcular $n$\\\\
Respuesta.- \;  Usando la parte $(a)$ sabemos que $k=10$ y $n-k=7$ por lo tanto $n=17$\\\\

%-------------------(c)-----------------------------
\item Sabiendo que ${14 \choose k} = {14 \choose k - 4}$ calcular $k$\\\\ 
Respuesta.- \; Similar a la parte $(b)$ $k=14 - (k-4) \Rightarrow 2k=18 \Rightarrow k =9$\\\\

%-------------------(d)-----------------------------
\item ¿Existe un $k$ tal que ${12 \choose k} = {12 \choose k-3}$\\\\
Respuesta.- \; No existe ya que $k=12 - (k-3) \Rightarrow 2k=15$ no es un entero.\\\\
\end{enumerate}

%-------------------------------------3------------------------------------------
\item Demostrar que ${n+1 \choose k} = {n \choose k-1} + {n \choose k}.$ Esta propiedad se denomina fórmula aditiva de los coeficientes combinatorios o ley del triángulo de Pascal y proporciona un método rápido para calcular sucesivamente los coeficientes binomiales. A continuación se da el triángulo de Pascal para $n \leq 6.$
\begin{center}
\begin{tabular}{>{$n=}l<{$\hspace{12pt}}*{13}{c}}
0 &&&&&&&1&&&&&&\\
1 &&&&&&1&&1&&&&&\\
2 &&&&&1&&2&&1&&&&\\
3 &&&&1&&3&&3&&1&&&\\
4 &&&1&&4&&6&&4&&1&&\\
5 &&1&&5&&10&&10&&5&&1&\\
6 &1&&6&&15&&20&&15&&6&&1\\\\
\end{tabular}
\end{center}
Demostración.- \; 
\begin{center}
\begin{tabular}{r c l}
${n \choose k-1}$&$=$&$\dfrac{n!}{(k-1)![n-(k-1)]!} + \dfrac{n!}{k!(n-k)!}$\\\\
&$=$&$\dfrac{(n!)k + (n!)(n-k+1)}{k!(n-k+1)!}$\\\\
&$=$&$\dfrac{(n!)(k+n-k+1)}{k!(n+1-k)!}$\\\\
&$=$&$\dfrac{(n!(n+1))}{k!(n+1-k)!}$\\\\
&$=$&$\dfrac{(n+1)!}{k![(n+1)-k]!}$\\\\
&$=$&${n+1 \choose k}$\\\\
\end{tabular}
\end{center}

%-------------------------------------4-----------------------------------------
\item Demuéstrese por inducción la fórmula de la potencia del binomio:
$$(a+b)^n = \displaystyle\sum_{k=0}^n {n \choose k} a^k b^{n-k}$$
Y utilícese el teorema para deducir las fórmulas:
$$\displaystyle\sum_{k=0}^n {n \choose k} = 2^n \; \; y \; \; \sum_{k=0}^n (-1)^k {n \choose k} = 0 \; \; si \; \; n>0$$\\\\
Demostración.- \; Sea $n=1$ entonces $$(a+n)^1 = a + b = \sum\limits_{k=0}^1 {n \choose k} a^k b^{n-k} = a^0 b + a b^0 = b + a = a + b$$
Por lo tanto la formula es cierta para $n=1$\\ 
Luego supongamos que la formula es cierta para algún $n=m \in \mathbb{Z}^+$ entonces,
$$(a+b)^m = \sum\limits_{k=0}^m {m \choose k} a^k b^{m-k}$$ Luego suponemos que se cumple para $m+1$
$$(a+b)^{m+1} = \sum\limits_{k=0}^{m+1} {m+1 \choose k} a^k b^{m+1-k}$$
\begin{center}
\begin{tabular}{r c l}
$(a+b)^{m+1}$&$=$&$\left[ \sum\limits_{k=0}^m {m \choose k} a^k b^{m-k} \right] (a+b)$\\\\
&$=$&$\sum\limits_{k=0}^m {m \choose k} a^{k+1} b^{m-k} + \sum\limits_{k=0}^m {m \choose k}a^k b^{m+1-k}$\\\\
&$=$&$a^{m+1} + \sum\limits_{k=0}^{m-1} {m \choose k} a^{k+1} b^{m-k} + \sum\limits_{k=0}^m {m \choose k}a^k b^{m+1-k}$\\\\
&$=$&$a^{m+1} + \sum\limits_{k=1}^{m} {m \choose k-1} a^k b^{m+1-k} + \sum\limits_{k=0}^{m} a^k b^{m+1-k}$\\\\
&$=$&$a^{m+1} + b^{m+1} + \sum\limits_{k=1}^{m} {m \choose k-1} a^k b^{m+1-k} + \sum\limits_{k=0}^m {m \choose k} a^k b^{m+1-k}$\\\\
&$=$&$a^{m+1} + b^{m+1} + \sum\limits_{k=1}^m \left\lbrace \left[ {m \choose k-1} + {m \choose k} \right] \left( a^k b^{m+1-k} \right)\right\rbrace$\\\\
&$=$&$a^{m+1} + b^{m+1} + \sum_{k=1}^m {m+1 \choose k} a^k b^{m+1-k}$\\\\
&$=$&$\sum\limits_{k=0}^{m+1} {m+1 \choose k} a^k b^{m+1-k}$\\\\
\end{tabular}
\end{center}
Por lo tanto, si la fórmula es verdadera para el caso $m$ entonces es verdadera para el caso $m+1$.\\\\
Luego aplicando el teorema del binomio con $a=1, \; b=1$, entonces,
$$(a+b)^n = \sum\limits_{k=0}^n a^k b^{n-k} \Rightarrow (1+1)^n = 2^n = \sum\limits_{k=0}^n {n \choose k}$$
Para la segunda fórmula aplicamos una vez mas pero con $a=-1$ y $b=1$, entonces,
$$(a+b)^n = (-1+1)^n = 0^n = 0 = \sum\limits_{k=0}^n {n \choose k} (-1)^k$$\\\\

%definición 4.15
\begin{tcolorbox}[colframe=white]
\begin{def.}[Símbolo producto.] El producto de $n$ números reales $a_i, a_2,...,a_n$ se indica por el símbolo $\prod\limits_{k=1}^n a_k,$ que se puede definir por inducción. El símbolo $a_1 a_2 \cdot \cdot \cdot a_n$ es otra forma de escribir este producto. Obsérvese que:
$$n! = \displaystyle\prod_{k=1}^n k$$\\
\end{def.}
\end{tcolorbox}

%-------------------------------------------5---------------------------------------
\item Dar una definición por inducción del producto $\displaystyle\prod_{k=1}^n a_k$\\\\
Definición.- \; $$\prod\limits_{k=1}^0 a_k = 1; \; \; \prod\limits_{k=1}^{n+1} a_k = a_{n+1} \cdot  \prod\limits_{k=1}^n a_k$$\\\\

Demostrar por inducción las siguientes propiedades de los productos:
%-------------------------------------------6---------------------------------------
\item $\displaystyle \prod_{k=1}^n (a_k b_k) = \left( \prod_{k=1}^n a_k \right) \left( \prod_{k=1}^n b_k \right)$ (Propiedad multiplicativa)\\
Un caso importante es la relación: $\displaystyle\prod_{k=1}^n (ca_k) = c_n \prod_{k=1}^n a_k$\\\\
Demostración.- \; Sea $n=1$ entonces $a_1 b_1 = a_1 b_1$. Supongamos que se cumple para $n = m \in \mathbb{Z}^+$, $$\prod\limits_{k=1}^m (a_k b_k) = \left( \prod\limits_{k=1}^m a_k \right) \left( \prod\limits_{k=1}^m b_k \right)$$
Luego sea $m+1$, por lo tanto $$\prod\limits_{k=1}^{m+1} (a_k b_k) = \left( \prod\limits_{k=1}^{m+1} a_k \right) \left( \prod\limits_{k=1}^{m+1} b_k \right)$$
Así, \\
\begin{center}
\begin{tabular}{r c l}
$\prod\limits_{k=1}^{m+1} (a_k b_k)$&$=$&$\left( \prod\limits_{k=1}^m a_k \right) \cdot a_{m+1} \left( \prod\limits_{k=1}^m b_k \right) \cdot b_{m+1}$\\\\
&$=$&$\left( \prod\limits_{k=1}^{m+1} a_k \right) \left( \prod\limits_{k=1}^{m+1} b_k \right)$\\\\
\end{tabular}
\end{center}
El caso $m+1$ es cierto por lo tanto la propiedad es válida para todo $n\in \mathbb{Z}^+$\\\\

%------------------------------------------7-----------------------------------------
\item $\displaystyle\prod_{k=1}^n \dfrac{a_k}{a_{k-1}} = \dfrac{a_n}{a_0}$ si cada $a_k\neq 0$ (propiedad telescópica)\\\\
Demostración.- \; Sea $n=1$ entonces $\prod\limits_{k=1}^1 \dfrac{a_k}{a_{k-1}} = \dfrac{a_n}{a_0} \Rightarrow \dfrac{a_1}{a_0} = \dfrac{a_1}{a_0}$ si $a_k \neq 0$. De ésta manera se cumple para $n=1$.\\
Luego supongamos que se cumple para $n = m \in \mathbb{Z}^+ $ así nos queda, $$\prod\limits_{k=1}^m \dfrac{a_k}{a_{k-1}} = \dfrac{a_m}{a_0}$$
Ahora supongamos que es cierto para $m+1$, luego $$\prod\limits_{k=1}^{m+1} \dfrac{a_k}{a_{k-1}} = \dfrac{a_{m+1}}{a_0}$$ de donde,\\
\begin{center}
\begin{tabular}{rcl}
$\prod\limits_{k=1}^{m+1} \dfrac{a_k}{a_{k-1}}$&$=$&$\dfrac{a_{m+1}}{a_m} \cdot \prod\limits_{k=1}^{m} \dfrac{a_k}{a_{k-1}}$\\\\
&$=$&$\dfrac{a_{m+1}}{a_m} \cdot \dfrac{a_m}{a_0}$\\\\
&$=$&$\dfrac{a_{m+1}}{a_0}$\\\\
\end{tabular}
\end{center}
Vimos que el caso $m+1$ es cierto, por lo tanto la propiedad es válida para $\forall n \in \mathbb{Z}^+$\\\\

%----------------------------------------8---------------------------------------------
\item Si $x\neq 1,$ demostrar que: $\displaystyle\prod_{k=1}^n (1 + x^{2^{k-1}}) = \dfrac{1 - x^{2^n}}{1-x}$\\
¿Cuál es el valor del producto cuando $x=1$?\\\\
Demostración.- \; Se cumple la condición para $n=1$ ya que $\prod\limits_{k=1}^1 (1 + x^{2^{k-1}}) = \dfrac{1 - x^{2^{n}}}{1-x} \Rightarrow 1 + x = \dfrac{1 - x^2}{1-x} = \dfrac{(1-x)(1+x)}{1-x} = 1+x$.\\
Supongamos que se cumple para $n= m \in \mathbb{Z}^+$ entonces $$\prod\limits_{k=1}^m (1 + x^{2^{k-1}}) = \dfrac{1 - x^{2^m}}{1-x}$$, luego se cumple para $m+1$, $$\prod\limits_{k=1}^{m+1} (1 + x^{2^{k-1}}) = \dfrac{1 - x^{2^{m+1}}}{1-x}$$, por lo tanto nos queda,\\
\begin{center}
\begin{tabular}{rcl}
$\prod\limits_{k=1}^{m+1} (1 + x^{2^{k-1}})$&$=$&$(1 + x^{2^{m}})\cdot \prod\limits_{k=1}^{m} (1 + x^{2^{k-1}})$\\\\
&$=$&$(1 + x^{2^{m}})\cdot \dfrac{1 - x^{2^m}}{1-x}$\\\\
&$=$&$\dfrac{(1+x^{2^m})(1-x^{2^m})}{1-x}$\\\\
&$=$&$\dfrac{1 - x^{2^{m+1}}}{1-x}$\\\\
\end{tabular}
\end{center}
Por lo tanto se cumple para $\forall n \in \mathbb{Z}^+$\\\\

%-----------------------------------------9----------------------------------------------
\item Si $a_k < b_k$ para cada valor de $k=1,2,...,n$, es fácil demostrar por inducción $\displaystyle\sum_{k=1}^n < \sum_{k=1}^n b_k.$\\
Discutir la desigualdad correspondiente para productos:
$$\displaystyle\prod_{k=1}^n a_k < \prod_{k=1}^n b_k$$\\\\
Demostración.- \; Es fácil ver que es la afirmación es cierta para $n=1$. Supongamos que es cierto para $n=m \in \mathbb{Z}^+$, luego por hipótesis sabemos que $b_{m+1}>a_{m+1} \geq 0$ de donde, $$\prod\limits_{k=1}^{m+1} a_k = a_{m+1} \cdot \prod\limits_{k=1}^m a_k < a_{m+1} \cdot \prod\limits_{k=1}^m b_k<b_{m+1} \cdot \prod\limits_{k=1}^m b_k = \prod\limits_{k=1}^{m+1} b_k$$\\\\

Algunas desigualdades notables\\\\
%---------------------------------------10----------------------------------------------
\item Si $x>1,$ demostrar por inducción que $x^n >x$ para cada $n\geq 2.$ Si $0<x<1$, demostrar que $x^n<x$ para cada $x \geq 2.$\\\\
Demostración.- \; Es fácil probar que la afirmación se cumple para $n=2$. Supongamos que se cumple para alguna $n=m \geq 2$, así $$x^m \geq x,$$ de donde $m+1$ se cumple para $$x^{m+1}\geq x,$$ así solo nos queda probar que $$x \cdot x \geq x,$$ el cuál se cumple a simple vista.\\\\
Por otra parte sea $n=2$ por lo tanto $x^2<x$, como $0<x<1$ vemos se cumple la desigualdad. Supongamos que se cumple para $n=m \geq 2$, entonces,
\begin{center}
\begin{tabular}{rcl}
$x^m<x$&$\Rightarrow$&$x^m < x\cot x$\\
&$\Rightarrow$&$x^{m+1}<x^2$\\
&$\Rightarrow$&$x^{m+1}<x^2<x$\\
&$\Rightarrow$&$x^{m+1}<x$\\
\end{tabular}
\end{center}
Así la desigualdad es válida para $m+1 \in \mathbb{Z}_{\geq 2}$\\\\


%---------------------------------------11-------------------------------------------------
\item Determínense tofos los enteros positivos $n$ para los cuales $2^n < n!$\\\\
Demostración.- \; Podemos observar que no se cumple para $n=1,2,3$. Luego observemos que la afirmación es válida para $n=4$,   $$2^4 < 4! \Rightarrow 16<24.$$
Ahora demostremos que la afirmación es válida para $n=m+1$ suponiendo que se cumple para algún $n=m \in \mathbb{Z}_{\geq 4}$, entonces,
$$2^m < m! \Longrightarrow 2^{m} (m+1) < m! (m+1)! \Longrightarrow 2^{m+1} < (m+1)! $$
Ya que $m\geq 4 > 2$ entonces $2^m(m+1) > 2^m \cdot 2 = 2^{m+1}$. Así la inecuación es verdadera para $m+1 \; \forall n \in \mathbb{Z}_{\geq 4}$\\\\

%--------------------------------------12---------------------------------------------------
\item 
\begin{enumerate}[\bfseries (a)]
%---------------------------------(a)--------------------------------
\item Con el teorema del binomio demostrar que para $n$ entero positivo se tiene 
$$\left( 1 + \dfrac{1}{n}  \right)^n = 1 + \displaystyle\sum_{k=1}^n \left[ \dfrac{1}{k!} \prod_{r=0}^{k-1} \left( 1 - \dfrac{r}{n} \right) \right]$$\\\\
Demostración.- \; Sea $a=\dfrac{1}{n}$ y $b=1$ tenemos,
\begin{center}
\begin{tabular}{rcll}
$\left( \dfrac{1}{n} + 1 \right)^n$&$=$&$\sum\limits_{k=0}^n {n \choose k} \left( \dfrac{1}{n} \right)^k$&\\\\
&$=$&$1 + \sum\limits_{k=1}^n \dfrac{n!}{k!(n-k)!} \left( \dfrac{1}{n} \right)^k$&\\\\
&$=$&$1+ \sum\limits_{k=1}^n \dfrac{1}{k!} \left[ \dfrac{\prod\limits_{r=1}^n r}{\prod\limits_{r=1}^{n-k} r}\cdot \left( \dfrac{1}{n} \right)^k \right]$&\\\\
&$=$&$1 + \sum\limits_{k=1}^n \dfrac{1}{k!} \left( \prod\limits_{r=n-k+1}^n r \right) \left( \dfrac{1}{n} \right) ^k$&\\\\
&$=$&$1 + \sum\limits_{k=1}^n \dfrac{1}{k!} \left[ \prod\limits_{r=0}^{k-1} (n-r) \right] \left( \prod\limits_{r=0}^{k-1} \dfrac{1}{n} \right)$&\\\\
&$=$&$1 + \sum\limits_{k=1}^n \dfrac{1}{k!} \left[ \prod\limits_{r=0}^{k-1} \left( 1 - \dfrac{r}{n} \right) \right] $&\\\\
\end{tabular}
\end{center}

%---------------------------------(b)-----------------------------
\item Si $n>1,$ aplíquese la parte $(a)$ y el Ejercicio 11 para deducir las desigualdades 
$$2 < \left( 1 + \dfrac{1}{n} \right)^n < 1 + \displaystyle\sum_{k=1}^n \dfrac{1}{k!} < 3 $$ \\\\
Demostración.- \; Si $n>1$ y $n\geq 2$ por el teorema bonomial tenemos:
$$\left( 1 + \dfrac{1}{n} \right)^n = \sum\limits_{k=0}^n {n \choose k} \left( \dfrac{1}{n} \right) ^k = 1 + n\left( \dfrac{1}{n} \right) + \sum\limits_{k=2}^n {n \choose k} \left( \dfrac{1}{n} \right)^k > 2$$
Donde la desigualdad es estricta ya que $\sum\limits_{k=2}^n {n \choose k} \left( \dfrac{1}{n} \right)^k $ para $n \geq 2.$\\
Luego por la parte (a) sabemos que $$\left( 1 + \dfrac{1}{n} \right) ^n = 1 + \sum\limits_{k=1}^n \dfrac{1}{k!} \left[ \prod\limits_{r=0}^{k-1} \left( 1 - \dfrac{r}{n} \right) \right] < 1 + \sum\limits_{k=1}^n \dfrac{1}{k!}$$
ya que si $n>1$ entonce $\left( 1 - \dfrac{r}{n} \right) < 1,$ para todo $ r = 0,...,n-1$ por lo tanto $\prod\limits_{r=0}^{k-1} \left( 1 - \dfrac{r}{n} \right) < \prod\limits{r=0}^{k-1} 1 = 1$ y en consecuencia $\dfrac{1}{k!} \left[ \prod\limits[{r=0}^{k-1} \right) \left( 1 - \dfrac{r}{n} \right) < \dfrac{1}{k!}$\\
Por último demostraremos que $1+ \sum\limits_{k=1}^n \dfrac{1}{k!} < 3$, 
\begin{center}
\begin{tabular}{rcll}
$1 + \sum\limits{k=1}^n \dfrac{1}{k!}$&$=$&$1+ 1 +\dfrac{1}{2} + \dfrac{1}{6} + \sum\limits_{k=4}^n \dfrac{1}{k!}$&\\\\
&$<$&$\dfrac{8}{3} + \sum\limits_{k=4}^n \dfrac{1}{2^k}$&por el problema 11\\\\
&$=$&$\dfrac{8}{3} + \dfrac{1}{16} \left( \sum\limits_{k=0}^{n-4} \dfrac{1}{2^k} \right)$&\\\\
&$=$&$\dfrac{8}{3} + \dfrac{1}{16} \left( 2 - \dfrac{1}{2^{n-4}} \right)$&por el problema 8\\\\
&$=$&$\dfrac{8}{3} + \dfrac{1}{8} - \dfrac{1}{2^n} $&\\\\
&$<$&$3$&\\\\
\end{tabular}
\end{center}
\end{enumerate}

%-----------------------------------13--------------------------------------
\item 
\begin{enumerate}[\bfseries (a)]
%--------------------------(a)----------------------------
\item Sea $p$ un entero positivo. Demostrar que:
$$b^p - a^p = (b-a)(b^{p-1} + b^{p-2}a + b^{p-3}a^2 + ... + ba^{p-2} + a^{p-1})$$\\\
Demostración.- \; Sea $(b-a)(b^{p-1} + b^{p-2}a + b^{p-3}a^2 + ... + ba^{p-2} + a^{p-1})$ entonces,
\begin{center}
\begin{tabular}{rcl}
&$=$&$(b-a)\left( \sum\limits_{k=0}^{p-1} b^{p-k-1} a^k \right)$\\\\
&$=$&$b \left( \sum\limits_{k=0}^{p-1} b^{p-k-1} a^k \right) - a\left( \sum\limits_{k=0}^{p-1} b^{p-k-1} a^k \right)$\\\\
&$=$&$ \sum\limits_{k=0}^{p-1} b^{p-k} a^k - \sum\limits_{k=0}^{p-1} b^{p-k} a^{k-1}  $\\\\
&$=$&$b^p + \sum\limits_{k=1}^{p-1} b^{p-k}a^k - a^p - \sum\limits_{k=1}^{p-1} b^{p-k}a^k$\\\\
&$=$&$b^p - a^p$\\\\
\end{tabular}
\end{center}

%----------------------(b)---------------------------
\item Si $p$ y $n$ son enteros positivos, demostrar que $$n^p < \dfrac{(n+1)^{p+1} - n^{p+1}}{p+1} < (n+1)^p$$\\\\
Demostración.- \; Por el teorema binomial se tiene:
\begin{center}
\begin{tabular}{rcl}
$\dfrac{(n+1)^{p+1} - n^{p+1}}{p+1}$&$=$&$\left( \dfrac{1}{p+1} \right) \left[ \left( \sum\limits_{k=0}^{p+1} {p+1 \choose k} n^k \right) - n^{p+1} \right]$\\\\
&$=$&$\left( \dfrac{1}{p+1} \right) \left[ n^{p+1} + (p+1)n^p + \left( \sum\limits_{k=0}^{p-1} {p+1 \choose k} n^k \right) - n^{p+1} \right]$\\\\
&$=$&$n^p + \left( \dfrac{1}{p+1} \right) \left( {p+1 \choose k} n^k \right)$\\\\
&$>$&$n^p$\\\\
\end{tabular}
\end{center}
Luego para la desigualdad de la derecha usaremos la parte $(a)$, de la siguiente manera:
\begin{center}
\begin{tabular}{rcl}
$\dfrac{(n+1)^{p+1} - n^{p+1}}{p+1}$&$=$&$\left( \dfrac{1}{p+1} \right)\left[ (n+1)^p + n(n+1)^{p-1} + ... + n^{p-1}(n+1) + n^p \right]$\\\\
&$<$&$\left( \dfrac{1}{p+1} \right)\left[ (n+1)^p + (p+1)^p + ... + (n+1)^p + (n+1)^p \right]$\\\\
&$=$&$\left( \dfrac{1}{p+1} \right) \left[ (p+1)(n+1)^p \right]$\\\\
&$=$&$(n+1)^p$\\\\
\end{tabular}
\end{center}

%------------------------(c)--------------------------------
\item Demuéstrese por inducción que:
$$\displaystyle\sum_{k=1}^{n-1} k^p < \dfrac{n^{p+1}}{p+1} < \sum_{k=1}^n k^p$$\\\\
Demostración.- \; Sea $n=1$ entonces $\sum\limits_{k=1}^{0} k^p = 0 < \dfrac{1^{p+1}}{p+1}< \sum\limits_{k=1}^1 k^p =1$ por lo tanto la inecuación se cumple. Luego asumimos que es verdad para algún $n=m \in \mathbb{Z}^{>0}$. Así para la inecuación de la izquierda se tiene:
$$\sum\limits_{k=1}^{m+1} k^p < \dfrac{m^{p+1}}{p+1}$$ Luego,
\begin{center}
\begin{tabular}{rcl}
$\sum\limits_{k=1}^m k^p$&$<$&$\dfrac{m^{p+1}}{p+1} + m^p$\\\\
&$<$&$\dfrac{m^{p+1}}{p+1} + \dfrac{(m+1)^{p+1} - m^{p+1}}{p+1}$\\\\
&$=$&$\dfrac{(m+1)^{p+1}}{p+1}$\\\\
\end{tabular}
\end{center}
Ahora veamos para la inecuación de la derecha. Asumiendo que es verdad para algún $n= m \in \mathbb{Z}_{>0},$ tenemos:
$$\dfrac{m^{p+1}}{p+1} < \sum\limits_{k=1}^m k^p$$, entonces
\begin{center}
\begin{tabular}{rcl}
$\dfrac{m^{p+1}}{p+1} + (m+1)^p$&$<$&$\sum\limits_{k=1}^{m+1} k^p$\\\\
$\dfrac{m^{p+1} + (m+1)^{p+1} - m^{p+1}}{p+1}$&$<$&$\sum\limits_{k=1}^{m+1} k^p$\\\\
$\dfrac{(m+1)^{p+1}}{p+1}$&$<$&$\sum\limits_{k=1}^{m+1} k^p$\\\\
\end{tabular}
\end{center}
Así vemos que se cumple para todo $n \in \mathbb{Z}^{>0}$\\\\
\end{enumerate}

%---------------------------------------------14----------------------------------------------------
\item Sean $a_1,...,a_n$ $n$ números reales, todos del mismo signo y todos mayores que $-1$. Aplicar el método de inducción para demostrar que:
$$(1+a_1)(1+a_2)\cdot \cdot \cdot (1+ a_n) \geq 1 + a_1 + a_2 + ... + a_n$$
en particular, cuando $a_1=a_2=...=a_n=x$, donde $x>-1$, se transforma en:
$$(1+x)^n \geq 1+nx \; \; \mbox{(desigualdad de Bernoulli)}$$
Probar que si $n>1$ el signo de igualdad se presenta en (1.25) sólo para $x=0$\\\\
Demostración.- \; Sea $n=1$ entonces $1+a_1 \geq 1 + a_1$ por lo que la desigualdad es válida. Ahora supongamos que la desigualdad es válida para algún $n=k \in \mathbb{Z}_{>0}$. así,
$$(1+a_1)...(1+ a_k) \geq 1 + a_1 + ... + a_k$$ de donde,
\begin{center}
\begin{tabular}{rcl}
$(1+a_1)...(1+a_k)(1+a_{k+1})$&$\geq$&$(1+a_1 + .. + a_k)(1+a_{k+1})$\\
&$\geq$&$(1+a_1 + ... + a_{k+1})+ a_{k+1} (a_1+...+a_k)$\\
\end{tabular}
\end{center}
dado que $a_i$ deben ser del mismo signo por lo tanto $a_{k+1}$ y $(a_1+...+a_k)$ debe ser positivo. Así, $$(1+a_1)\cdot \cdot \cdot (1+a_{k+1}) \geq 1 + a_1 + ...+ a_{k+1}$$ que cumple la desigualdad para $n \in \mathbb{Z}_{>0}$\\
Vemos ahora la desigualdad de bernoulli. Si $x=0$ entonces $(1+0)^n = 1 = 1 + n\cdot 0$, por lo tanto se cumple la igualdad si sólo si $x=0$\\
Para el caso de $n=2$ tenemos $(1+x)^2 0 1+2x +x^2 > 1 + 2x$ entonces la desigualdad se cumple para $n=2$. Supongamos ahora que la desigualdad es estricta para algún $n=k \in \mathbb{Z}_{>1}$ por lo tanto$$(1+x)^k > 1 + kx$$ así,
\begin{center}
\begin{tabular}{rcl}
$(1+x)^k (1+x)$&$>$&$(1+kx)(1+x)$\\
$(1+x)^{k+1}$&$>$&$(k+1)x + kx^2$\\
$(1+x)^{k+1}$&$>$&$1 + (k+1)x$\\
\end{tabular}
\end{center}
Ya que $k>0$ y $x>0$ implica que $kx^2>0$ por lo tanto la desigualdad es estricta para todo $n>1$ si $x\neq 0$. Por lo tanto, la igualdad es válida si y sólo si $x=0$\\\\

%------------------------------------------------15-------------------------------------------------
\item Si $n \geq 2,$ demostrar que $n!/n^n \leq \left( \dfrac{1}{2} \right)^k,$ siendo $k$ la parte entera de $n/2.$\\\\
Demostración.- \; Demostremos por inducción. Sea $n=2$ entonces $$\dfrac{2!}{2^2} = \dfrac{1}{2} = \left( \dfrac{1}{2} \right)^2 = \dfrac{1}{2}$$ ya que $n/2 = 2/2 = 1 = k$ siendo $k$ la parte entera de $n/2$. Supongamos que la desigualdad se cumple para algún $n=m \in \mathbb{Z} \geq 2$. Entonces,
\begin{center}
\begin{tabular}{rcl}
$\dfrac{m!}{m^m} \leq \left( \dfrac{1}{2} \right)^k$ & $\Rightarrow$ & $\left( \dfrac{m!}{m^m} \right)\left[ \dfrac{(m+1)m^m}{(m+1)^{m+1}} \right] \leq \left( \dfrac{1}{2} \right)^k  \left[ \dfrac{(m+1)m^m}{(m+1)^{m+1}} \right]$\\\\
&$\Rightarrow$&$\dfrac{(m+1)!}{(m+1)^{m+1}} \leq \left( \dfrac{1}{2} \right)^k \left( \dfrac{m}{m+1} \right)^m $\\\\
\end{tabular}
\end{center} 
Luego por el problema $13$ se tiene $$m^p < \dfrac{(m+1)^{p+1} - m^{p+1}}{p+1} \;\; \forall \; p \in \mathbb{Z}^{+}$$
Supongamos que $p=m-1$ entonces,
\begin{center}
\begin{tabular}{rcll}
$(p+1)m^p$&$\Rightarrow$&$(m+1)^{p+1} - m^{p+1}$&\\
&$\Rightarrow$&$m^{p+1} + (p+1)m^p < (m+1)^{p+1}$&\\
&$\Rightarrow$&$m^m + m^m < (m+1)^m$& ya que $p=m-1$\\
&$\Rightarrow$&$2m^m < (m+1)^m$\\
&$\Rightarrow$&$\left( \dfrac{m}{m+1}\right) ^m < \dfrac{1}{2}$\\
\end{tabular}
\end{center}
Así nos queda que $$\dfrac{(m+1)!}{(m+1)^{m+1}} \leq \left( \dfrac{1}{2} \right)^k \left( \dfrac{m}{m+1}\right)^m \leq \left( \dfrac{1}{2} \right)^{k+1}$$
Por último recordemos que $k$ es la parte entera de $m/2$, para completar la demostración debemos demostrar que si la desigualdad se  cumple para $m$ entonces e cumplirá para $m+1$, en efecto si $m$ es par, entonces $k=m/2$ y $k+1 = (m+1)/2$, por otro lado si $m$ es impar entonces $k=(m+1)/2$. En cualquier caso $k+1  \geq (m+1)/2$ entonces $$\dfrac{(m+1)!}{(m+1)^{m+1}} \leq \left(\dfrac{1}{2} \right)^2 \leq \left( \dfrac{1}{2} \right) ^{\dfrac{m+1}{2}}$$
Por lo tanto la desigualdad es válida para el caso $m+1$ y en consecuencia es cierta para $n \in \mathbb{Z}_{\geq 2}$\\\\

%--------------------------------------------16-----------------------------------------------
\item Los números $1,2,3,5,8,13,21,...$ tales que cada uno después del segundo es la suma de los dos anteriores, se denomina números de Fibonacci. Se pueden definir por inducción como sigue:
$$a_1=1, \; \; a_2=2, \; \; a_{n+1} = a_n + a_{n-1}, \; \; si \; n \geq 2.$$
Demostrar que $$a_n < \left( \dfrac{1 + \sqrt{5}}{2} \right)^n$$
para cada $n \geq 1.$\\\\
Demostración.- \; Para el caso de $n=1$ tenemos $$1 < \dfrac{1+\sqrt{5}}{2}$$ ya que $\sqrt{5}>2$ se cumple la desigualdad para $n=1$. Ahora supongamos que es verdad para algún $n= k \in \mathbb{Z}_{\geq 1}$, luego 
\begin{center}
\begin{tabular}{crcll}
&$a_k + a_{k-1}$&$<$&$ \left( \dfrac{1+\sqrt{5}}{2} \right)^k +  \left( \dfrac{1+\sqrt{5}}{2} \right)^{k-1}$&\\\\
$\Rightarrow$&$a_{k+1}$&$<$&$\left( \dfrac{1 + \sqrt{5}}{2} \right)^{k-1} \cdot \left( \dfrac{1 + \sqrt{5}}{2} + 1 \right)$&\\\\ 
&&$=$&$\left( \dfrac{1 + \sqrt{5}}{2} \right)^{k-1} \cdot \left( \dfrac{3 + \sqrt{5}}{2} \right)$&\\\\
&&$=$&$\left( \dfrac{1 + \sqrt{5}}{2} \right)^{k-1} \cdot \left( \dfrac{1 + \sqrt{5}}{2} \right)^2$&ya que $\left( \dfrac{3 + \sqrt{5}}{2} \right) = \left( \dfrac{1+\sqrt{5}}{2}\right)^2$\\\\
&&$=$&$\left( \dfrac{1 + \sqrt{5}}{2} \right)^{k-1}$&\\\\
\end{tabular}
 
\end{center}
Por lo tanto la desigualdad es válida para $n \in \mathbb{Z}_{\geq 1}$\\\\

\begin{tcolorbox}[colframe=white]
\begin{def.}[Desigualdades que relacionan distintos tipos de promedios]
Sean $x_1,x_2,...,x_n$ $n$ números reales positivos. Si $p$ es un entero no nulo, la media de potencias p-énesimas $M_p$ se define como sigue.
$$M_p = \left( \dfrac{x_1^{p} + ... + x_{n}^{p}}{n} \right)^{1/p}$$
El número $M_1$ se denomina media aritmética, $M_2$ media cuadrática y $M_{-1}$ media armónica.\\
\end{def.}
\end{tcolorbox}
\vspace{1cm}
%---------------------------------17------------------------------------
\item Si $p>0$ demostrar que $M_p< M_{2p}$ cuando $x_1,x_2,...,x_n$ no son todos iguales.\\\\
Demostración.- \; Sea $a_k=x^p_k$ y $b_k=1$ entonces por la desigualdad de Cauchy-Schwarz, $$\left( \sum\limits_{k=1}^n x_k^p \cdot 1 \right)^2 \leq \left( \sum\limits_{k=1}^n a_k^{2p} \right) \left( \sum\limits_{k=1}^n 1^2 \right) $$ luego, vemos que la desigualdad es estricta ya que si se sostuviera la igualdad existiría alguna $y\in \mathbb{R}$ tal que $x_k^p \cdot y + 1 = 0, \; \forall k$, pero esto implicaría que $x_k = \left( -\dfrac{1}{y} \right)^{1/p}, \; \forall k,$ contradiciendo nuestro supuesto de que $x_k$ no todos son iguales. Luego sabemos que $\sum\limits_{k=1}^n 1 = n$ entonces
\begin{center}
\begin{tabular}{rcll}
$\sum\limits_{k=1}^n x_k^p$&$<$&$\left( \sum\limits_{k=1}^n a_k^{2p} \right)^{1/2} \cdot n^{1/2}$&\\\\
$\left( \sum\limits_{k=1}^n x_k^p \right)^{1/p}$&$<$&$ \left( \sum\limits_{k=1}^n a_k^{2p} \right)^{1/2p} \left( \dfrac{1}{n}\right)^{1/2p}$&elevado por $1^{1/p}$\\\\
$\left( \dfrac{1}{p} \right)^{1/p} \left( \sum\limits_{k=1}^n x_k^p \right)^{1/p}$&$<$&$\left( \sum\limits_{k=1}^n a_k^{2p}\right)^{1/2p}  \left( \dfrac{1}{n}\right)^{1/2p}$&por $\left( \dfrac{1}{n}\right)^{1/2}$\\\\
$\left( \dfrac{\sum\limits_{k=1}^n x_k^p}{n} \right)^{1/p}$&$<$&$\left( \dfrac{\sum\limits_{k=1}^{2p} x_k^{2p}}{n} \right)^{1/2p}$&\\\\
\end{tabular}
\end{center}

%-------------------------------------18------------------------------------
\item Aplíquese	el resultado del Ejercicio 17 para demostrar que
$$a^4 + b^4 + c^4 \geq \dfrac{64}{3}$$ si $a^2 + b^2 +c^2 = 8$ y $a, \; b, \;c>0.$\\\\
Demostración.- \; Sea $a + b + c \in \mathbb{R}$ distintos entre si, aplicamos el problema 17 con $p=2$, 
\begin{center}
\begin{tabular}{rcl}
$\left( \dfrac{a^2 + b^2 + c^2}{3} \right)^{1/2}< \left( \dfrac{a^2 + b^2 + c^2}{3} \right)^{1/4}$&$\Rightarrow$&$\left(\dfrac{8}{3}\right)^2 < \dfrac{a^4 + b^4 + c^4}{3}$\\\\
&$\Rightarrow$&$a^4+b^4+c^4 > \dfrac{64}{3}$\\\\
\end{tabular}
\end{center} 
Luego si $a=b=c,$ entonces 
\begin{center}
\begin{tabular}{rcl}
$a^2+b^2+c^2$&$\Rightarrow$&$3a^2=8$\\\\
&$\Rightarrow$&$a^2=b^2=c^2=\dfrac{8}{3}$\\\\
&$\Rightarrow$&$a^4=b^4=c^4 = \dfrac{64}{9}$\\\\
&$\Rightarrow$&$a^4+b^4+c^4 = \dfrac{64}{3}$\\\\
\end{tabular}
\end{center}
Por lo tanto la desigualdad se cumple para cualquier $a,b,c \in \mathbb{R}$\\\\

%---------------------------------19-----------------------------------
\item Sean $a_1,...,a_n$  $n$ números reales positivos cuyo producto es igual a $1$. Demostrar que $a_1+...+a_n \geq n$ y que el signo de igualdad se presenta sólo cuando cada $a_k =1$\\\\
Demostración.- \; Consideremos dos casos:
\begin{enumerate}[C1]
\item Si $a_1=...=a_n=1,$ entonces 
$$\sum\limits_{k=1}^n = \sum\limits_{k=1}^n 1 = n \geq \prod\limits_{k=1}^n a_n = \prod\limits_{k=1}^n 1 = 1 $$
Por lo tanto la desigualdad se cumple.
\item Sea $a_k \neq 1$ y $n=2$ por inducción vemos que no se cumple la desigualdad $$a_1 \cdot a_2 \neq 1$$. Si uno de ellos no es uno tampoco lo es el siguiente es decir,
\begin{center}
\begin{tabular}{rcl}
$a_1 \cdot a_2 = 1$&$\Rightarrow$&$a_2 = \dfrac{1}{a_1}$\\\\
&$\Rightarrow$&$a_1 + a_2 = a_1 + \dfrac{1}{a_1}$\\\\
&$\Rightarrow$&$a_1 + a_2=\dfrac{a_1^2 + 1}{a_1}$\\\\
&$\Rightarrow$&$a_1 + a_2=\dfrac{a_1^2 - 2a_1 + 1 +2a_1}{a_1}$\\\\
&$\Rightarrow$&$a_1 + a_2=\dfrac{(a_1 - 1)^2}{a_1} + 2$\\\\
&$\Rightarrow$&$a_1 + a_2 > 2$\\\\
\end{tabular}
\end{center}
Donde la desigualdad final sigue desde $\dfrac{(a_1-1)^2}{a_1}$ para $a_1>0,$ por lo tanto la desigualdad es válida para el caso $n=2$\\
Supongamos que la desigualdad se cumple para $n=k \in \mathbb{Z}^+$. Luego $a_1,...,a_{k+1} \in \mathbb{R}^+$ con $a_i \neq 1$ para al menos un $i=1,...,k+1.$ Si $a_i < 1$, entonces debe haber algún $j\neq i$ tal que $a_j > 0$ porque de lo contrario si $a_j \leq 1$ para  todo $j\neq i$, entonces $a_1 \cdot \cdot \cdot a_{k+1}<1.$ De manera similar, si $a_i>1,$ entonces hay algún $j\neq i$ tal que $a_j <1.$ Por lo tanto tenemos un par $a_i, \; a_j$ con un miembro del par mayor que $1$ y el otro menor que $1$. Sea este par $a_1$ y $a_{k+1}$ entonces definamos $a_1 \cdot a_{k+1}$, entonces,
$$b\cdot a_2 \cdot \cdot \cdot a_k = 1 \Rightarrow b_1 + a_2 + ... - a_k \geq k$$
Además, dado que $(1-a_i)(1-a_{k+1})<0$ (dado que uno de $a_1, a_{k+1}$ es mayor que $1$ y el otro es menor que $1$, uno de $(1-a_1), (1-a_{k+1})$ es positivo y el otro es negativo, luego $$1-a_1-a_{k+1}+a_1a_{k+1}<0 \Rightarrow b<a_1 + a_{k+1}-1,$$
Así
$$b+a_2+...+a_k\geq k \Rightarrow a_1 + a_2 +...+a_k + a_{k+1} \geq k+1$$
Por lo tanto, la desigualdad es válida para $k+1$ y en consecuencia es verdadera para $n\in \mathbb{Z}^+$\\\\
\end{enumerate}

\begin{tcolorbox}[colframe = white]
\begin{def.}
La media geométrica $G$ de $n$ números reales positivos $x_1,...,x_n$ está definida por la fórmula $G=(x_1 x_2 \cdot \cdot \cdot x_n)^{1/n}$
\end{def.}
\end{tcolorbox}
%----------------------------------20-----------------------------------
\item 
\begin{enumerate}[\bfseries (a)]
\item Desígnese con $M_p$ la media de potencias p-ésimas. Demostrar que $G \leq M_1$ y que $G=M_1$ solo cuando $x_1 = x_2 = \cdot \cdot \cdot = x_n.$\\\
Demostración.- \; Si $x_1, ... ,x_n$ no todos iguales, entonces $$G_n = \left[ (x_1 \cdot \cdot \cdot x_n)^{1/n} \right]^n = x_1 \cdot \cdot \cdot x_n,$$ así, 
\begin{center}
\begin{tabular}{rcll}
$\left( \dfrac{1}{G_n} \right)(x_1\cdot \cdot \cdot x_n)$&$=$&$1$&\\\\
$\left( \dfrac{x_1}{G} \right) \left( \dfrac{x_2}{G} \right) \cdot \cdot \cdot \left( \dfrac{x_n}{G} \right)$&$=$&$1$&\\\\
$\dfrac{x_1}{G} + \dfrac{x_2}{G} + ... + \dfrac{x_n}{G}$&$>$&$n$& por problema anterior\\\\
$x_1+x_2+...+x_n$&$>$&$n \cdot G$\\\\
$M_1$&$>$&$G$\\\\
\end{tabular}
\end{center}
luego si $x_1,...x_n$ son todos iguales , entonces, $$G=(x_1\cdot \cdot \cdot x_n)^{1/n} = (x_1^n)^{1/n}=x_1 =  \dfrac{n x_1}{n} = \dfrac{x_1 + ... + x_n}{n} = M_1$$\\
\item Sean $p$ y $q$ enteros, $q<0<p$. A partir de $(a)$ deducir que $M_q< G < M_p$ si $x_1,x_2,...,x_n$ no son todos iguales.\\\\
Demostración.- \; Probemos primero que $G < M_p$. Para ello primero veamos que si $x_1,...,x_n$ son números reales positivos, no todos iguales entonces $x_P,...,x_n^P$ también son números reales positivos también no todos iguales. A partir de la definición de $M_p$ y dejando $M_p(x_1^p,...,x_n^p)$ denotar la p-enésima potencia media de los números $x_i^p,...,x_n^p$ tenemos,
\begin{center}
\begin{tabular}{rcl}
$M_1(x_1^p,...,x_n^p)$&$=$&$\dfrac{x_1^p + ... + x_n^p}{n}$\\\\
&$=$&$\left[ \left( \dfrac{x_1^p + ... + x_n^p}{n} \right)^{1/p} \right]^p$\\\\
&$=$&$[M_p(x_1,...,x_n)]^p$\\\\
\end{tabular}
\end{center}
así se observa que $[M_p(x_1,...,x_n)]^p = M_1(x_i^p,...,x_n^p)$ luego por la parte $(a)$,
\begin{center}
\begin{tabular}{rcl}
$G(x_1,...,x_n)$&$=$&$(x_1\cdot \cdot \cdot x_n)^{p/n}$\\
&$=$&$(x_i^p \cdot \cdot \cdot x_n^p)^{1/n}$\\
&$=$&$G(x_1^p,...,x_n^p)$\\
&$<$&$M_1(x_1^p,...,x-n^p)$\\
&$=$&$[M_p(x_1,...,x_n)]^p$\\
\end{tabular}
\end{center}
Por lo tanto implica que  $G<M_p$\\
ahora debemos  $M_q<G$ para $q<0$, visto de otra forma $-q>0$ entonces $G<M_{-q}$ y por la desigualadad que demostramos se tiene, $$G^{-q}<(M_{-q})^{-q} \Rightarrow G^q > M_q^q \Rightarrow G > M_q$$\\\\
\end{enumerate}

%---------------------------------21-----------------------------------
\item Aplíquese los resultados del Ejercicio 20 para probar la siguiente proposición: Si $a,b$ y $c$ son números reales y positivos tales que $abc=8$, entonces $a+b+c \geq 6$ y $ab +ac +bc \geq 12.$\\\\
Demostración.- \; Sea $a,b \in \mathbb{R}^+$ tenemos 
\begin{center}
\begin{tabular}{rcl}
$(a\cdot b \cdot c)^{1/3}$&$\leq$&$ \dfrac{a+b+c}{3} $\\\\
$8$&$\leq$&$\dfrac{(a+b+c)^3}{3^3}$\\\\
$2^3 \cdot 3^3$&$\leq$&$(a+b+c)^3$\\\\
$a+b+c$&$\geq$&$6$\\\\
\end{tabular}
\end{center}
Luego para la segunda desigualdad utilizamos la parte $(b)$ del anterior problema,
\begin{center}
\begin{tabular}{rcl}
$\left( \dfrac{a^{-1} + b^{-1} + c^{-1}}{3} \right)^{-1}$&$\leq$&$(a \cdot b \cdot c)^{1/3}$\\\\
$\left[ \dfrac{3^3}{\left( \dfrac{1}{a} + \dfrac{1}{b} + \dfrac{1}{c} \right)^3} \right] ^3$&$\leq$&$2^3$\\\\
$\left( \dfrac{1}{a} + \dfrac{1}{b} + \dfrac{1}{c} \right)^3$&$\geq $&$\dfrac{3^3}{2^3}$\\\\
$\dfrac{1}{a} + \dfrac{1}{b} + \dfrac{1}{c}$&$\geq$&$\dfrac{3}{2}$\\\\
$\dfrac{bc + ac + ab}{abc}$&$\geq$&$\dfrac{3}{2}$\\\\
$\dfrac{bc + ac + ab}{8}$&$\geq$&$\dfrac{3}{2}$\\\\
$ab+ac+bc$&$\geq$&$12$\\\\
\end{tabular}
\end{center}
%---------------------------------22------------------------------------
\item Si $x_1,...,x_n$ son números positivos y si $y_k = 1/x_k,$ demostrar que $$\displaystyle \left( \sum_{k=1}^n x_k \right) \left( \sum_{k=1}^n y_k \right) \geq n^2.$$\\\\
Demostración.- \; Sea $\sqrt{x_k} \sqrt{y_k} = 1$ ya que $y_k = \dfrac{1}{x_k}$ entonces por la desigualdad de Cauchy-Schwarz tenemos,
\begin{center}
\begin{tabular}{rcl}
$\left( \sum\limits_{k=1}^n a_k b_k \right)^2$&$\leq$&$\left( \sum\limits_{k=1}^n a_k^2 \right) \left( \sum\limits_{k=1}^n  b_k^2 \right)$\\\\
$\left( \sum\limits_{k=1}^n \sqrt{x_k} \sqrt{y_k} \right)^2$&$\leq$&$\left( \sum\limits_{k=1}^n \sqrt{x_k}^2 \right)\left( \sum\limits_{k=1}^n \sqrt{y_k}^2 \right)$\\\\
$\left( \sum\limits_{k=1}^n 1 \right)^2$&$\leq$&$\left(\sum\limits_{k=1}^n x_k \right)\left( \sum\limits_{k=1}^n  y_k\right)$\\\\
$\left(\sum\limits_{k=1}^n x_k \right)\left( \sum\limits_{k=1}^n  y_k\right)$&$\geq$&$n^2$\\\\
\end{tabular}
\end{center}

%--------------------------------23------------------------------------
\item Si $a,b$ y $c$ son números positivos y si $a+b+c =1$, demostrar que $(1-a)(1-b)(1-c)\geq 8abc$\\\\
Demostración.- \; Sea $M_{-1}(a,b,c) \leq M_1(a,b,c)$ entonces $$\left( \dfrac{a_{-1} + b_{-1} + c_{-1}}{3}\right)^{-1} \leq \dfrac{a+b+c}{3}$$ por lo tanto,
\begin{center}
\begin{tabular}{rcl}
$\dfrac{3}{\dfrac{1}{a}+ \dfrac{1}{b} + \dfrac{1}{c}}$&$\leq$&$\dfrac{1}{3}$\\\\
$9$&$\leq$&$\dfrac{1}{a}+ \dfrac{1}{b} + \dfrac{1}{c}$\\\\
$9abc$&$\leq$&$bc+ac+ab$\\\\
$8abc$&$\leq$&$bc+ac+ab-abc$\\\\
$8abc$&$\leq$&$1-(a+b+c)+ab+ac+bc-abc$\\\\
\end{tabular}
\end{center}
ya que $1=a+b+c$, luego $1-(a+b+c)+ab+ac+bc-abc = (1-a)(1-b)(1-c)$ así nos queda $$8abc\leq (1-a)(1-b)(1-c)$$\\\\
\end{enumerate}



