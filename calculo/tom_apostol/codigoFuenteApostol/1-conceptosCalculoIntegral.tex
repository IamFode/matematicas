\chapter{Los conceptos del Cálculo Integral}
\setcounter{chapter}{1}
\setcounter{section}{2}

\section{Funciones. Definición formal como conjunto de pares ordenados}
En cálculo elemental tiene interés considerar en primer lugar, aquellas funciones en las que el dominio y el recorrido son conjuntos de números reales. Estas funciones se llaman 
\textbf{Funciones de variable real} o funciones reales.\\

    \begin{tcolorbox}[colframe=white]
        %-----------------------------1.1. definición par ordenado-----------------------------
        \begin{def.}[Par ordenado]
            Dos pares ordenados $(a,b)$ y $(c,d)$ son iguales si y sólo si sus primeros elementos son iguales y sus segundos elementos son iguales.
            $$(a,b) = (c,d) \; \; \mbox{si y sólo si} \; \; a=c \; y \; b=d$$
        \end{def.}
    \end{tcolorbox}

    \begin{tcolorbox}[colframe=white]
        %----------------------------1.2 definición de función---------------------------------
        \begin{def.}[Definición de función]
            Una función $f$ es un conjunto de pares ordenados $(x,y)$ ninguno de los cuales tiene el mismo primero elemento.\\\\
            Debe cumplir las siguientes condiciones de existencia y unicidad:
            \begin{enumerate}[\bfseries i)]
                \item $\forall x \in D_f, \exists y / (x,y) \in f(x) \; ó \; y=f(x)$
                \item $(x,y) \in  f \land (x,z) \in  f \Rightarrow y = z$
            \end{enumerate}
        \end{def.}
    \end{tcolorbox}

    \begin{tcolorbox}[colframe=white]
        %----------------------------1.3 definición dominio recorrido---------------------------
        \begin{def.}[Dominio y recorrido]
            Si $f$ es una función, el conjunto de todos los elementos $x$ que aparecen como primeros elementos de pares $(x,y)$ de $f$ se llama el \textbf{dominio} de $f$.  El conjunto de los segundos elementos y se denomina \textbf{recorrido} de $f$, o conjunto de valores de $f$.
        \end{def.}
    \end{tcolorbox}

        %----------------------teorema 1.1------------------------
        \begin{teo}
            Dos funciones $f$ y $g$ son iguales si y sólo si 
            \begin{enumerate}[\bfseries (a)]
                \item $f$ y $g$ tienen el mismo dominio, y
                \item $f(x) = g(x)$ para todo $x$ del dominio de $f.$\\
            \end{enumerate}
            Demostración.- \; Sea $f$ función tal que $x \in  D_f,\exists y \; / \; y=f(x)$ es decir $(x,f(x))$, $g$ una función talque $\forall  z \in  D_g , \exists  y \; / \; y=g(z)$ es decir $(z,g(z))$, entonces por definición de par ordenado tenemos que $(x,f(x)) = (z,g(z)) $ si y sólo si $x=z$ y $f(x)=g(z)$\\\\
        \end{teo}

    \begin{tcolorbox}[colframe=white]
        %--------------1.4 definición de sumas productos y cocientes de una función----------------------
        \begin{def.}[Sumas, productos y cocientes de funciones]
            Sean $f$ y $g$ dos funciones reales que tienen el mismo dominio $D$. Se puede construir nuevas funciones a partir de $f$ y $g$ por adición, multiplicación o división de sus valores. La función $u$ definida por,
            $$u(x) = f(x) + g(x) \; \; si \; x \in D$$
            se denomina suma de $f$ y $g$, se representa por $f+g.$ Del mismo modo, el producto $v=f cdot g$ y el cociente $w=f/g$ están definidos por las fórmulas
            $$v(x) = f(x) g(x) \; \; si \; x \in D, \; \; \, \, w(x) = f(x)/g(x) \; \; si \, x \, \in D \; y \; g(x) \neq 0$$
        \end{def.}
    \end{tcolorbox}
    
\setcounter{section}{4}
\section{Ejercicios}
    \begin{enumerate}[\Large \bfseries 1.]
        %--------------------------------1.--------------------------------------
        \item 
    \end{enumerate}