\chapter{Algunas aplicaciones de la integración}

\setcounter{section}{1}
\section{El área de una región comprendida entre dos gráficas expresada como una integral}

%--------------------2.1
\begin{teo} Supongamos que $f$ y $g$ son integrables y que satisfacen $f\leq q$ en $[a,b]$. La región $S$ entre sus gráficas es medible y su área $a(S)$ viene dada por la integral $$a(S) = \int_a^b [g(x)-f(x)] \; dx$$
    Demostración.- \; 
	Demostración.- \; Supongamos primero que $f$ y $g$ son no negativas,. Sean $F$ y $G$ los siguientes conjuntos:
	$$F = {(x,y)|a\leq x \leq b, 0\leq y \leq f(x)}, \quad G = {(x,y) | a\leq x \leq b, 0\leq y \leq g(x)}.$$
	Esto es, $G$ es el conjunto de ordenadas de $g$, y $F$ el de $f$, menos la gráfica de $f$. La región $S$ es la diferencia $G-F$. Según los teoremas 1.10 y 1.11, $F$ y $G$ son ambos medibles. Puesto que $F \subseteq G$ la diferencia $S = G-F$ es también medible, y se tiene 
	$$a(S) = a(G) - a(F) = \int_a^b g(x) \; dx = \int_a^b [g(x)-f(x)] \; dx$$
	Consideremos ahora el caso general cuando $f\leq q$ en $[a,b]$, pero no son necesariamente no negativas. Este caso lo podemos reducir al anterior trasladando la región hacia arriba hasta que quede situada encima del eje $x$. Esto es, elegimos un número positivo $c$ suficientemente grande que asegure que $0 \leq f(x) + c \leq g(x) + c$ para todo $x$ en $[a,b]$. Por lo ya demostrado la nueva región $T$ entre las gráficas de $f+c$ y $g+c$ es medible, y su parea viene dad por la integral
	$$a(T) = \int_a^b [(g(x)+c) - (f(x)+c)] = \int_[g(x)-f(x)] \; dx$$
	Pero siendo $T$ congruente a $S$, ésta es también medible y tenemos $$a(S) = a(T) = \int_a^b [g(x) - f(x)]\; dx$$
	Esto completa la demostración.\\\\
\end{teo}

%----------nota 2.1
\begin{tcolorbox}[colframe = white]
    \begin{nota} En los intervalos $[a,b]$ puede descomponerse en un número de subintervalos en cada uno de los cuales $f\leq g$ o $g\leq f$ la fórmula (2.1) del teorema 2.1 adopta la forma 
    $$a(S) = \int_a^b |g(x) -f(x)| \; dx$$
    \end{nota}
\end{tcolorbox}

    %--------------------lema 2.1
    \begin{lema}[Área de un disco circular] Demostrar que $A(r) = r^2 A(1)$. Esto es, el área de un disco de radio $r$ es igual al producto del área de un disco unidad (disco de radio $1$) por $r^2$.\\\\
	Demostración.-\; Ya que $g(x) - f(x) = 2g(x),$ el teorema 2.1 nos da 
	    $$A(r) = \int_{-r}^r g(x) \; dx = 2 \int_{-r}^r \sqrt{r^2 - x^2} \; dx$$
	    En particular, cuando $r = 1$, se tiene la fórmula $$A(1) = 2\int_{-1}^1 \sqrt{1 - x^2} \; dx$$
	    Cambiando la escala en el eje $x$, y utilizando el teorema 1.19 con $k=1/r$, se obtiene
	    $$A(r) = 2\int_{-r}^r g(x) \; dx = 2r \int_{-1}^1 g(rx) \; dx = 2r\int_{-1}^1 \sqrt{r^2 - (rx)^2} \; dx = 2r^2 \int_{-1}^1 \sqrt{1-x^2} \; dx = r^2 A(1)$$
	    Esto demuestra que $A(r) = r^2 A(1)$, como se afirmó.\\\\
    \end{lema}

%---------------------definición 2.1
\begin{tcolorbox}[colframe = white]
    \begin{def.} Se define el número $\pi$ como el área de un disco unidad.
	$$\pi = 2 \int_{-1}^1 \sqrt{1-x^2}\; dx$$
    \end{def.}
\end{tcolorbox}
\begin{center}
    La formula que se acaba de demostrar establece que $A(r) = \pi r^2$\\
\end{center}

\begin{tcolorbox}[colframe = white]
Generalizando el anterior lema se tiene 
    $$a(kS) = \int_{ka}^{kb} g(x)\; dx = k \int_{ka}^{kb} f(x/k) \; dx = k^2 \int_a^b f(x) \; dx$$
\end{tcolorbox}

%--------------------teorema 2.1
\begin{teo} Para $a>0$, $b>0$ y $n$ entero positivo, se tiene $$\int_a^b x^{\frac{1}{n}} \; dx = \dfrac{b^{1-1/n} - a^{1-\frac{1}{n}}}{1+\frac{1}{n}}$$\\
    Demostración.-\; Sea $\int_0^a x^{\frac{1}{n}}$. El rectángulo de base $a$ y altura $a^{\frac{1}{n}}$ consta de dos componentes: el conjuntos de ordenadas de $f(x) = x^{\frac{1}{n}}$ a $a$ y el conjuntos de ordenadas $g(y) = y^n$ a $a^{\frac{1}{n}}$. Por lo tanto,
    $$a\cdot a^{\frac{1}{n}} = a^{1+\frac{1}{n}} = \int_0^a x^{\frac{1}{n}} \; dx + \int_0^{a^{\frac{1}{n}}} y^n \; dy \; \Longrightarrow \; \int_0^a x^{\frac{1}{n}} \; dx = a^{1+\frac{1}{n}} - \dfrac{y^{n+1}}{n+1}\bigg|_0^{a^{\frac{1}{n}}} = a^{1+\frac{1}{n}} -  \dfrac{a^{1+\frac{1}{n}}}{n+1} = \dfrac{a^{1+\frac{1}{n}}}{1 + 1/n}$$
    Análogamente se tiene $$\int_0^b x^{\frac{1}{n}}\; dx = \dfrac{b^{1 + \frac{1}{n}}}{1 + 1/n}$$
    Luego notemos que $$\int_a^b x^{\frac{1}{n}} \; dx = \int_0^b x^{\frac{1}{n}}\; dx - \int_0^a x^{\frac{1}{n}} \;dx$$
    por lo tanto $$\int_a^b x^{\frac{1}{n}}\; dx = \dfrac{b^{1+\frac{1}{n}} - b^{1 + \frac{1}{n}}}{1 + 1/n}$$\\\\
\end{teo}



\setcounter{section}{3}
\section{Ejercicios}

En los ejercicios del 1 al 14, calcular el área de la región $S$ entre las gráficas de $f$ y $g$ para el intervalo $[a,b]$ que en cada caso se especifica. Hacer un dibujo de las dos gráficas y sombrear $S$.\\

\begin{enumerate}[\Large\bfseries 1.]

%--------------------1.
\item $f(x) = 4 - x^2, \quad g(x)=0, \quad a = -2, \quad b = 2$\\\\
    Respuesta.-\; $$\int_{-2}^2 [4-x^2 - 0] \; dx = 4x \bigg|_{-2}^2 -\dfrac{x^3}{3}\bigg|_{-2}^2 = 4(2-(-2)) - \left(\dfrac{2^3 - (-2)^3}{3}\right)  = \dfrac{32}{3}$$\\ 

%--------------------2.
\item $f(x) = 4 - x^2, \quad g(x) = 8 - 2x^2,\quad a = -2, \quad b = 2.$\\\\
    Respuesta.-\; $$\int_{-2}^2 [8 - 2x^2 - (4 - x^2)] \; dx = \int_{-2}^2 4-x^2 \; dx = \dfrac{32}{3} \; (por \, ejercicio \; 1)$$\\

%--------------------3.
\item $f(x)=x^3+x^2,\quad g(x)=x^3 + 1, \quad a=-1, \quad b=1$.\\\\
    Respuesta.-\; $$\int_{-1}^1 x^3 + 1 - (x^3 + x^2) \;dx = \int_{-1}^1 1-x^2\; dx = x\bigg|_{-1}^1 - \dfrac{x^3}{3}\bigg|_{-1}^1   = 2 - \dfrac{1-(-1)}{3} = \dfrac{4}{3}$$\\

%--------------------4.
\item $f(x)=x-x^2,\quad g(x)=-x,\quad a=0,\quad b=2$\\\\
    Respuesta.-\; $$\int_0^2 x-x^2 - (-x) \; dx = \int_0^2 2x - x^2 = 2\dfrac{x^2}{2}\bigg|_0^2 - \dfrac{x^3}{3}\bigg|_0^2 = 2\dfrac{2^2}{2} - \dfrac{2^3}{3} = \dfrac{4}{2}$$\\

%--------------------5.
\item $f(x) = x^{1/3}, \quad g(x) = x^{1/2}, \quad a=0, \quad b=1$\\\\ 
    Respuesta.-\; $$\int_0^1 x^{1/3} - x^{1/2} \; dx = \dfrac{x^{1+1/3}}{1+1/3}\bigg|_0^1 - \dfrac{x^{1+1/2}}{1+1/2}\bigg|_0^1 = \dfrac{3}{4} - \dfrac{2}{3} = \dfrac{1}{12}$$\\

%--------------------6.
\item $f(x) = x^{1/3}, \quad g(x) = x^{1/2},\quad a=1,\quad b=2.$\\\\
    Respuesta.-\; $$\int_1^2 x^{1/2}-x^{1/3}\; dx = \dfrac{x^{1/2 + 1}}{1 + 1/2}\bigg|_1^2 - \dfrac{x^{1/3}+1}{1+1/3}\bigg|_1^2 = \dfrac{2^{1/2+1}-1}{1+1/2}-\dfrac{2^{1/3+1}}{1+1/3} = \dfrac{4\sqrt{2}}{3}-\dfrac{3\sqrt[3]{2}}{2}+\dfrac{1}{12}$$\\

%--------------------7.
\item $f(x)=x^{1/3},\quad g(x) = x^{1/2}, \quad a = 0,\quad b=2$\\\\
    Respuesta.-\; Sea $$\int_0^1 |x^{1/3}-x^{1/2}|\; dx + \int_1^2 |x^{1/3}-x^{1/2}|\; dx$$
    por los problemas 5 y 6 se tiene $$\dfrac{1}{12} + \dfrac{4\sqrt{2}}{3}-\dfrac{3\sqrt[3]{2}}{2}+\dfrac{1}{12} = \dfrac{4\sqrt{2}}{3}-\dfrac{3\sqrt[3]{2}}{2}+\dfrac{1}{6}$$\\

%--------------------8.
\item $f(x) = x^{1/2}, \quad g(x) = x^2, \quad a=0, \quad b=2$\\\\
    Respuesta.-\;
    \begin{center}
	\begin{tabular}{rcl}
	    $\displaystyle\int_0^1 x^{1/2} - x^2 \; dx + \int_1^2 x^2 - x^{1/2}\; dx$ & $=$ & $\left(\dfrac{x^{1+1/2}}{1+1/2}\bigg|_0^1 - \dfrac{x^3}{3}\bigg|_0^1\right) + \left( \dfrac{x^3}{3}\bigg|_1^2 - \dfrac{x^{1+1/2}}{1+1/2}\bigg|_1^2 \right)$\\\\
	    & $=$ & $\left(\dfrac{1}{1+1/2} - \dfrac{1}{3}\right) + \left(\dfrac{2^3-1}{3} - \dfrac{2^{1+1/2} - 1}{1+1/2}\right)$\\\\
	    & $=$ & $\dfrac{2}{3} - \dfrac{1}{3} + \dfrac{7}{3} - \dfrac{4\sqrt{2}-2}{3}$\\\\
	    &$=$&$\dfrac{10}{3}-\dfrac{4\sqrt{2}}{3}$\\\\
	\end{tabular}
    \end{center}

%--------------------9.
\item $f(x) = x^2, \quad g(x) = x+1, \quad a=-1, \quad b = (1+\sqrt{5})/2$\\\\
    Respuesta.$$\int_{-1}^{(1-\sqrt{5})/2} x^2 - (x+1)\; dx +  \int_{(1-\sqrt{5})/2}^{(1+\sqrt{5})/2} (x+1) - x^2\; dx-\; = $$
    \begin{center}
	\begin{tabular}{rcl}
	     & $=$ & $\displaystyle\int_{-1}^{(1-\sqrt{5})/2} x^2 - x - 1\; dx +  \int_{(1-\sqrt{5})/2}^{(1+\sqrt{5})/2} x + 1 - x^2 \; dx$ \\\\
	    & $=$ & $\left(\dfrac{x^3}{3} - \dfrac{x^2}{2} - x\right)\bigg|_{-1}^{(1-\sqrt{5})/2} + \left(\dfrac{x^2}{2} + x - \dfrac{x^3}{3}\right)\bigg|_{(1-\sqrt{5})/2}^{(1+\sqrt{5})/2}$ \\\\
	    & $=$ & $-\dfrac{3}{4} + \dfrac{5 \sqrt{5}}{12}+\dfrac{5\sqrt{5}}{6}$ \\\\
	    & $=$ & $\dfrac{5\sqrt{5}-3}{4}$\\\\
	\end{tabular}
    \end{center}

%--------------------10.
\item $f(x)=x(x^2-1),\quad g(x)=x,\quad a=-1, \quad b=\sqrt{2}$\\\\
    Respuesta.-\; $$\int_{-1}^{0} x(x^2-1) - x \; dx + \int_{0}^{\sqrt{2}} x - [x(x^2-1)] \; dx = \int_{-1}^{0} x^3-2x \; dx + \int_{0}^{\sqrt{2}} - x^3 + 2x \; dx = $$
	$$=\left(\dfrac{x^4}{4} - x^2 \right)\bigg|_{-1}^0 + \left( -\dfrac{x^4}{4} + x^2 \right)\bigg|_{0}^{\sqrt{2}} = -\dfrac{1}{4} + 1 + (-1+2) = \dfrac{7}{4}$$\\

%--------------------11.
    \item $f(x)=|x|,\quad g(x) = x^2-1, \quad a=-1,\quad b=1$\\\\
	Respuesta.-\; Definimos $f$ de la siguiente manera: $$f(x) = |x| = \left\{\begin{array}{rcl}
	    -x&si&x\in [-1,0)\\
	    x&si&x\in [0,1]\\
	    \end{array}\right.$$
	    Luego,
	    \begin{center}
	    \begin{tabular}{rcl}
		$\displaystyle\int_{-1}^1 f(x)-g(x) \; dx$ & $=$ & $\displaystyle\int_{-1}^0 -x-x^2+1 \; dx + \int_0^1 x-x^2+1 \; dx$\\\\
		& $=$ & $ \left(-\dfrac{x^2}{2}-\dfrac{x^3}{3} + x \right) \bigg|_{-1}^0 + \left(\dfrac{x^2}{2} - \dfrac{x^3}{3} + x\right)\bigg|_0^1 $\\\\
		& $=$ & $\left(\dfrac{1}{2} - \dfrac{1}{3}+1\right)+\left(\dfrac{1}{2}-\dfrac{1}{3} + 1\right)$\\\\
		& $=$ & $\dfrac{7}{3}$ \\\\
	    \end{tabular}
	    \end{center}

%--------------------12.
\item $f(x) = |x+1|,\quad g(x)=x^2-2x, \quad a=0, \quad b=2$\\\\
    Respuesta.- \; Definamos $f$ de la siguiente manera:
	$$f(x)= |x-1| = \left\{
	    \begin{array}{rcl}
		-(x+1)&si&x\in[0,1)\\
		x+1&si&x\in [1,2]\\
	    \end{array}
	    \right.$$
	    Entonces, 
	    \begin{center} 
		\begin{tabular}{rcl}
		    $\displaystyle\int_0^2 f(x)-g(x) \; dx$&$=$&$\displaystyle\int_0^1 -(x-1)-x^2+2x\; dx + \int_1^2 x-1-x^2+2x \; dx$\\\\
		    & $=$ &$\displaystyle\int_0^1 -x^2+x+1\; dx + \int_1^2 -x^2+3x - 1 \; dx$\\\\
		    & $=$ & $\left(-\dfrac{x^3}{3} + \dfrac{x^2}{2} + x\right)\bigg|_0^1 + \left(-\dfrac{x^3}{3} + \dfrac{3x^2}{2} -x\right)\bigg|_1^2$\\\\
		    & $=$ & $\left(-\dfrac{1}{3} + \dfrac{1}{2} + 1\right)+\left(-\dfrac{8}{3} + 6-2+\dfrac{1}{3} -\dfrac{3}{2} + 1\right)$\\\\
		    & $=$ & $\dfrac{7}{3}$\\\\
		\end{tabular}
	    \end{center}

%--------------------13.
\item $f(x) = 2|x|, \quad g(x) = 1-3x^3, \quad a=-\sqrt{3}/3, \quad b=\dfrac{1}{3}$\\\\ 
    Respuesta.-\; Definimos $f$ de la siguiente manera: $$f(x) = |x| = \left\{\begin{array}{rcl}
	-x&si&x\in [-\sqrt{3}/3,0)\\
	    x&si&x\in [0,1/3]\\
	    \end{array}\right.$$
	    de donde se tiene,
	    \begin{center}
		\begin{tabular}{rcl}
		    $\displaystyle\int_{-\frac{\sqrt{3}}{3}}^{\frac{1}{3}} g(x)-f(x)\; dx$ & $=$ & $\displaystyle\int_{-\frac{-\sqrt{3}}{3}}^{\frac{1}{3}}g(x) \; dx - \int_{-\frac{\sqrt{3}}{3}}^{\frac{1}{3}} f(x) \; dx$\\\\\
		    & $=$ & $\displaystyle\int_{-\frac{\sqrt{3}}{3}}^{\frac{1}{3}} 1-3x^3 \; dx - \int_{-\frac{\sqrt{3}}{3}}^{0} -2x \; dx - \int_{0}^{\frac{1}{3}} 2x \; dx$\\\\
		    & $=$ & $\left(x-\dfrac{3}{4} x^4\right)\bigg|_{-\frac{\sqrt{3}}{3}}^{\frac{1}{3}} + x^2 \bigg|_{-\frac{\sqrt{3}}{3}}^{0} - x^2 \bigg|_0^{\frac{1}{3}}$\\\\
		    & $=$ & $\left(\dfrac{1}{3} - \dfrac{1}{108} + \dfrac{\sqrt{3}}{3} + \dfrac{1}{12}\right)+\left(-\dfrac{1}{3}\right)-\dfrac{1}{9}$\\\\
		    & $=$ & $\dfrac{9\sqrt{3}-1}{27}$\\\\
		\end{tabular}
	    \end{center}


%--------------------14.
\item $f(x) = |x|+|x-1|, \quad g(x)=0, \quad a=-1,\quad b=2$\\\\
    Respuesta.-\; En este problema $f(x)\geq g(x)$ en el intervalo $[-1,2]$, por lo tanto 
    $$\int_{-1}^2 f(x)-g(x)\; dx = \int_{-1}^2 |x| + |x+1|\; dx = \int_{-1}^2 |x|\; dx + \int_{-1}^2 |x-1| \; dx$$
    Definimos cada función por separado,
    $$|x| = \left\{ \begin{array}{rcl} -x & si & x\in [-1,0)\\ x & si & x\in [0,2] \end{array}\right.$$
    $$|x-1| = \left\{ \begin{array}{rcl} -(x-1) & si & x\in [-1,1)\\ x-1 & si & x\in [1,2] \end{array}\right.$$
    por lo tanto
    \begin{center}
	\begin{tabular}{rcl}
	    $\displaystyle\int_{-1}^2 |x|\; dx + \int_{-1}^2 |x-1|\; dx$&$=$&$\displaystyle\int_{-1}^0 -x \; dx + \int_0^2 x\; dx + \int_{-1}^1 -(x-1)\; dx + \int_1^2 x-1\; dx$\\\\
	    &$=$&$\left(-\dfrac{x^2}{2}\right)\bigg|_{-1}^0 + \left(\dfrac{x^2}{2}\right)\bigg|_{0}^2 + \left(-\dfrac{x^2}{2}\right)\bigg|_{-1}^1 + \left(x\right)\bigg|_{-1}^1 + \left(\dfrac{x^2}{2}\right)\bigg|_{1}^2 + \left(-x\right)\bigg|_{1}^2$\\\\\
	    &$=$&$\dfrac{1}{2}+2-\dfrac{1}{2}+\dfrac{1}{2}+1+1+2-\dfrac{1}{2}-2+1$\\\\\
	    &$=$&$5$\\\\\
	\end{tabular}
    \end{center}

%--------------------15.
\item Las gráficas de $f(x) = x^2$ y $g(x)=cx^3$, siendo $c>0$, se cortan en los puntos $(0,0)$ y $(1/c,1/c^2)$. Determinar $c$ de modo que la región limitada entre esas gráficas y sobre el intervalo $[0,1/c]$ tengan área $\frac{2}{3}$.\\\\
    Respuesta.-\; Tenemos que $f\geq g$ en el intervalo $[0,1/c]$ de donde, 
    $$\int_0^{1/c} x^2 -cx^3 \; dx = \int_0^{1/c} x^2\; dx - c\int_0^{1/c} x^3 \; dx = \dfrac{1}{12c^3}$$
    luego $\dfrac{1}{12c^3}=\dfrac{2}{3}$ por lo tanto $c=\dfrac{1}{2}$.\\\\

%--------------------16.
\item Sea $f(x)=x-x^2$, $g(x) = ax$. Determinar $a$ para que la región situada por encima de la gráfica de $g$ y por debajo de $f$ tenga área $frac{9}{2}$.\\\\
    Respuesta.-\; Tomaremos los casos cuando $a=0, a>0$ y $a<0$. \\
    Veamos primero que si $g(x)\leq f(x)$ entonces $$f(x)-g(x)\geq 0 \Longrightarrow x-x^2-ax\geq 0 \Longrightarrow (1-a)x \geq x^2$$
    de donde si $x=0$ se tiene una igualdad. Luego si $x\neq 0$ entonces $x\leq (1-a)$. Ahora sea $a<0$ por suposición se tendrá $1-a>0$, que nos muestra que el intervalo estará dado por $[0,1-a]$. Análogamente se tiene el intervalo $[1-a,0]$ para $a>0$.\\
    \begin{enumerate}[\bfseries C 1.]
	\item Si $a=0$, esto no es posible ya que si $a = 0$ entonces $g(x) = ax = 0$ y entonces el área arriba del gráfico de $g$ y debajo del gráfico de $f$ es igual a
	    $$\int_0^1 x-x^2\; dx = \left(\dfrac{x^2}{2} - \dfrac{x^3}{3}\right)\bigg|_0^1 = \dfrac{1}{6} \neq \dfrac{9}{2}$$\\
	
	\item Si $a<0$, $f(x)\geq g(x)$ para $[0,1-a]$, por lo que tenemos la zona, $a(S)$ de la región entre las dos gráficas dadas por 
	    \begin{center}
		\begin{tabular}{rcl}
		    $\displaystyle\int_0^{1-a} x-x^2 - ax \; dx$&$=$&$\displaystyle(1-a)\int_0^{1-a} x\; dx - \int_0^{1-a} x^2 \; dx$\\\\
		    &$=$&$(1-a)\left(\dfrac{(1-a)^2}{2}\right) - \dfrac{(1-a)^3}{3}$\\\\
		    &$=$&$-\dfrac{(1-a)^3}{6}$\\\\
		\end{tabular}
	    \end{center}
	    así nos queda que $$-\dfrac{(1-a)^3}{6} = \dfrac{9}{2} \Longrightarrow (1-a)^3 = -27 \Longrightarrow a = a$$\\

	\item Sea $a>0$ y  $f(x)\geq g(x)$ entonces $[1-a,0]$ lo que 
	    \begin{center}
		\begin{tabular}{rcl}
		    $\displaystyle\int_{1-a}^0 x-x^2-ax \; dx$&$=$&$\displaystyle(1-a)\int_{1-a}^0 x\; dx  - \int_{1-a}^0 x^2 \; dx$\\\\ 
		    &$=$&$(1-a)\left(-\dfrac{(1-a)^2}{2} - \dfrac{(1-a)^3}{2}\right)$\\\\
		    &$=$&$-\dfrac{(1-a)^3}{6}$\\\\
		\end{tabular}
	    \end{center}
	    Así igualando por $\frac{9}{2}$ tenemos 
	    $$-\dfrac{(1-a)^3}{6} = \dfrac{9}{2} \Longrightarrow (1-a)^3 = -27 \Longrightarrow a=4$$\\
    \end{enumerate}
    Por lo tanto los valores posibles para $a$ son $-2$ y $4$.\\\\

%--------------------17.
\item Hemos definido $\pi$ como el área de un disco circular unidad. En el ejemplo 3 de la Sección 2.3, se ha demostrado que $\pi=2 \int_{-1}^1 \sqrt{1-x^2}\; dx$. Hacer uso de las propiedades de la integral para calcular la siguiente en función de $\pi$.
\begin{enumerate}[\bfseries (a)]

    %----------(a)
    \item $\displaystyle\int_{-3}^3 \sqrt{9-x^2}\; dx$.\\\\
	Respuesta.-\; Por el teorema 19 de dilatación, $\dfrac{1}{\frac{1}{3}}\displaystyle\int_{-3\frac{1}{3}}^{3\frac{1}{3}} \sqrt{9 - \left(\dfrac{x}{\frac{1}{3}}\right)^2} \; dx$, de donde nos queda $9 \displaystyle\int_{-1}^1 \sqrt{1-x^2}\; dx$, por lo tanto y en función de $\pi$ se tiene $\dfrac{9}{2} \pi$.\\\\

    %----------(b)
    \item $\displaystyle\int_0^2 \sqrt{1-\frac{1}{4}x^2}\; dx$.\\\\
	Respuesta.-\; Similar al anterior ejercicio se tiene 
	$$\int_0^2 \sqrt{1-\dfrac{1}{4}x^2}\; dx = 2\int_0^1 \sqrt{1-x^2}\; dx = \int_{-1}^1 \sqrt{1-x^2}\; dx = \dfrac{\pi}{2}$$\\

    %----------(c)
    \item $\displaystyle\int_{-2}^2 (x-3)\sqrt{4-x^2}\; dx$.\\\\
	Respuesta.-\; Comencemos usando la linealidad respecto al integrando de donde tenemos $\displaystyle\int_{-2}^2 x\sqrt{4-x^2}\; dx - 3\int_{-2}^2 \sqrt{4-x^2}\; dx$. Luego por el problema 25 de la sección 1.26, $\displaystyle\int_{-2}^2 x\sqrt{4-x^2}\; dx = 0$, de donde $$ -6\displaystyle\int_{-1}^1 \sqrt{4-4x^2}\; dx = -12 \int_{-1}^1 \sqrt{1-x^2}\; dx = -6\pi$$\\

\end{enumerate}

%--------------------18.
\item Calcular las áreas de los dodecágonos regulares inscrito y circunscrito en un disco circular unidad y deducir del resultado las desigualdades $3<\pi<12(2-\sqrt{3}).$\\\\
    Respuesta.-\; Como estos son dodecágonos, el ángulo en el origen del círculo de cada sector triangular es $2 \pi / 12 = \pi / 6$, y el ángulo de los triángulos rectángulos formado al dividir cada uno de estos sectores por la mitad es entonces $\pi/12$. Luego usamos el hecho de que,
    $$\tan\left(\dfrac{\pi}{12}\right)=2-\sqrt{2}, \qquad \sen\left(\dfrac{\pi}{12}\right) = \dfrac{\sqrt{3}-1}{2\sqrt{2}}, \qquad \cos\left(\dfrac{\pi}{12}\right) = \dfrac{\sqrt{3}+1}{2\sqrt{2}}$$
    Ahora, para el dodecágono circunscrito tenemos el área del triángulo rectángulo $T$ con base $1$ dado por, $$a(T) = \dfrac{1}{2}bh = \dfrac{1}{2}\cdot 1 \cdot (2-\sqrt{3}) = 1 - \dfrac{\sqrt{3}}{2}.$$
    Como hay 24 triángulos de este tipo en el dodecaedro, tenemos el área del dodecaedro circunscrito $D_c$ dada por $$a(D_c) = 24\left(\dfrac{1-\sqrt{3}}{2}\right) = 12(2-\sqrt{3})$$
    Por otro lado para el dodecágono inscrito, consideramos el triángulo rectángulo $T$ con hipotenusa $1$ en el diagrama. La longitud de uno de los catetos viene dada por $\sen \left(\frac{\pi}{12} \right) = \frac{\sqrt{3} - 1} {2}$ y la otra por $\cos \left(\frac{\pi}{12}\right).$ Entonces el área del triángulo es, $$a(T) = \dfrac{1}{2} bh = \dfrac{1}{2}\cdot \dfrac{\sqrt{3}-1}{2\sqrt{2}}\cdot \dfrac{\sqrt{3}+1}{2\sqrt{2}} = \dfrac{2}{16} = \dfrac{1}{8}.$$
    Dado que hay $24$ triángulos de este tipo en el dodecaedro inscrito, $D_{i}$ tenemos,
    $$a(D_i) = 24\cdot \dfrac{1}{8} = 3$$
    Por lo tanto, en vista de que el área del círculo unitario es, por definición $\pi$ y se encuentra entre estos dos dodecaedros, tenemos, $$3<\pi<12(2-\sqrt{3})$$\\

%--------------------19.
\item Sea $C$ la circunferencia unidad, cuya ecuación cartesiana es $x^2+y^2 = 1$. Sea $E$ el conjunto de puntos obtenido multiplicando la coordenada $x$ de cada punto $(x,y)$ de $C$ por un factor constante $a>0$ y la coordenada $y$ por un factor constante $b>0$. El conjunto $E$ se denomina elipse. (Cuando $a=b$, la elipse es otra circunferencia.).

\begin{enumerate}[\bfseries a)]

    %----------a)
    \item Demostrar que cada punto $(x,y)$ de $E$ satisface la ecuación cartesiana $(x/a)^2+(y/b)^2 = 1$.\\\\
	Demostración.-\; Sea $E=\lbrace (ax,by) / (x,y) \in C, a>0,b>0 \rbrace$. Si $(x,y)$ es un punto en $E$ entonces $\left(\frac{x}{a},\frac{y}{b}\right)$ es un punto es $C$, ya que todos los puntos de $E$ se obtienen tomando un punto de $C$ y multiplicando la coordenada $x$ por $a$ y la coordenada $y$ por $b$. Por definición de $C$, se tiene $$\left(\dfrac{x}{a}\right)^2+\left(\dfrac{y}{b}\right)^2 = 1$$

    %----------b)
    \item Utilizar las propiedades de la integral para demostrar que la región limitada por esa elipse es medible y que su área es $\pi ab$.\\\\
	Demostración.-\; 

\end{enumerate}

%--------------------20.
\item El ejercicio 19 es una generalización del ejemplo 3 de la sección 2.3. Establecer y demostrar una generalización correspondiente al ejemplo 4 de la sección 2.3.\\\\
    Demostración.-\; 

%--------------------21.
\item Con un razonamiento parecido al del ejemplo 5 de la sección 2.3 demostrar el teorema 2.2.\\\\
    Demostración.-\; Esta demostración ya fue dada junto a la definición del teorema 2.2.\\\\
\end{enumerate}


