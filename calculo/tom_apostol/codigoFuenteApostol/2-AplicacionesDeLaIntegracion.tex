\chapter{Algunas aplicaciones de la integración}

\setcounter{section}{1}
\section{El área de una región comprendida entre dos gráficas expresada como una integral}

%--------------------2.1
\begin{teo} Supongamos que $f$ y $g$ son integrables y que satisfacen $f\leq q$ en $[a,b]$. La región $S$ entre sus gráficas es medible y su área $a(S)$ viene dada por la integral $$a(S) = \int_a^b [g(x)-f(x)] \; dx$$
    Demostración.- \; 
	Demostración.- \; Supongamos primero que $f$ y $g$ son no negativas,. Sean $F$ y $G$ los siguientes conjuntos:
	$$F = {(x,y)|a\leq x \leq b, 0\leq y \leq f(x)}, \quad G = {(x,y) | a\leq x \leq b, 0\leq y \leq g(x)}.$$
	Esto es, $G$ es el conjunto de ordenadas de $g$, y $F$ el de $f$, menos la gráfica de $f$. La región $S$ es la diferencia $G-F$. Según los teoremas 1.10 y 1.11, $F$ y $G$ son ambos medibles. Puesto que $F \subseteq G$ la diferencia $S = G-F$ es también medible, y se tiene 
	$$a(S) = a(G) - a(F) = \int_a^b g(x) \; dx = \int_a^b [g(x)-f(x)] \; dx$$
	Consideremos ahora el caso general cuando $f\leq q$ en $[a,b]$, pero no son necesariamente no negativas. Este caso lo podemos reducir al anterior trasladando la región hacia arriba hasta que quede situada encima del eje $x$. Esto es, elegimos un número positivo $c$ suficientemente grande que asegure que $0 \leq f(x) + c \leq g(x) + c$ para todo $x$ en $[a,b]$. Por lo ya demostrado la nueva región $T$ entre las gráficas de $f+c$ y $g+c$ es medible, y su parea viene dad por la integral
	$$a(T) = \int_a^b [(g(x)+c) - (f(x)+c)] = \int_[g(x)-f(x)] \; dx$$
	Pero siendo $T$ congruente a $S$, ésta es también medible y tenemos $$a(S) = a(T) = \int_a^b [g(x) - f(x)]\; dx$$
	Esto completa la demostración.\\\\
\end{teo}

%----------nota 2.1
\begin{tcolorbox}[colframe = white]
    \begin{nota} En los intervalos $[a,b]$ puede descomponerse en un número de subintervalos en cada uno de los cuales $f\leq g$ o $g\leq f$ la fórmula (2.1) del teorema 2.1 adopta la forma 
    $$a(S) = \int_a^b |g(x) -f(x)| \; dx$$
    \end{nota}
\end{tcolorbox}

    %--------------------lema 2.1
    \begin{lema}[Área de un disco circular] Demostrar que $A(r) = r^2 A(1)$. Esto es, el área de un disco de radio $r$ es igual al producto del área de un disco unidad (disco de radio $1$) por $r^2$.\\\\
	Demostración.-\; Ya que $g(x) - f(x) = 2g(x),$ el teorema 2.1 nos da 
	    $$A(r) = \int_{-r}^r g(x) \; dx = 2 \int_{-r}^r \sqrt{r^2 - x^2} \; dx$$
	    En particular, cuando $r = 1$, se tiene la fórmula $$A(1) = 2\int_{-1}^1 \sqrt{1 - x^2} \; dx$$
	    Cambiando la escala en el eje $x$, y utilizando el teorema 1.19 con $k=1/r$, se obtiene
	    $$A(r) = 2\int_{-r}^r g(x) \; dx = 2r \int_{-1}^1 g(rx) \; dx = 2r\int_{-1}^1 \sqrt{r^2 - (rx)^2} \; dx = 2r^2 \int_{-1}^1 \sqrt{1-x^2} \; dx = r^2 A(1)$$
	    Esto demuestra que $A(r) = r^2 A(1)$, como se afirmó.\\\\
    \end{lema}

%---------------------definición 2.1
\begin{tcolorbox}[colframe = white]
    \begin{def.} Se define el número $\pi$ como el área de un disco unidad.\\\\
    \end{def.}
\end{tcolorbox}
\begin{center}
    La formula que se acaba de demostrar establece que $A(r) = \pi r^2$\\
\end{center}

\begin{tcolorbox}[colframe = white]
Generalizando el anterior lema se tiene 
\begin{center}
    $$a(kS) = \int_{ka}^{kb} g(x)\; dx = k \int_{ka}^{kb} f(x/k) \; dx = k^2 \int_a^b f(x) \; dx$$
\end{center}
\end{tcolorbox}

%--------------------teorema 2.1
\begin{teo} Para $a>0$, $b>0$ y $n$ entero positivo, se tiene $$\int_a^b x^{\frac{1}{n}} \; dx = \dfrac{b^{1-1/n} - a^{1-\frac{1}{n}}}{1+\frac{1}{n}}$$\\
    Demostración.-\; Sea $\int_0^a x^{\frac{1}{n}}$. El rectángulo de base $a$ y altura $a^{\frac{1}{n}}$ consta de dos componentes: el conjuntos de ordenadas de $f(x) = x^{\frac{1}{n}}$ a $a$ y el conjuntos de ordenadas $g(y) = y^n$ a $a^{\frac{1}{n}}$. Por lo tanto,
    $$a\cdot a^{\frac{1}{n}} = a^{1+\frac{1}{n}} = \int_0^a x^{\frac{1}{n}} \; dx + \int_0^{a^{\frac{1}{n}}} y^n \; dy \; \Longrightarrow \; \int_0^a x^{\frac{1}{n}} \; dx = a^{1+\frac{1}{n}} - \dfrac{y^{n+1}}{n+1}\bigg|_0^{a^{\frac{1}{n}}} = a^{1+\frac{1}{n}} -  \dfrac{a^{1+\frac{1}{n}}}{n+1} = \dfrac{a^{1+\frac{1}{n}}}{1 + 1/n}$$
    Análogamente se tiene $$\int_0^b x^{\frac{1}{n}}\; dx = \dfrac{b^{1 + \frac{1}{n}}}{1 + 1/n}$$
    Luego notemos que $$\int_a^b x^{\frac{1}{n}} \; dx = \int_0^b x^{\frac{1}{n}}\; dx - \int_0^a x^{\frac{1}{n}} \;dx$$
    por lo tanto $$\int_a^b x^{\frac{1}{n}}\; dx = \dfrac{b^{1+\frac{1}{n}} - b^{1 + \frac{1}{n}}}{1 + 1/n}$$\\\\
\end{teo}



\setcounter{section}{3}
\section{Ejercicios}

En los ejercicios del 1 al 14, calcular el área de la región $S$ entre las gráficas de $f$ y $g$ para el intervalo $[a,b]$ que en cada caso se especifica. Hacer un dibujo de las dos gráficas y sombrear $S$.\\

\begin{enumerate}[\Large\bfseries 1.]

%--------------------1.
\item 

\end{enumerate}


