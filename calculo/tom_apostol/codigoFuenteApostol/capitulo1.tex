\section*{Axiomas de cuerpo}
%axioma1
\begin{tcolorbox}[colback=white]
\begin{axioma}[Propiedad conmutativa] $x+y=y+x, \; xy=yx$\\
\end{axioma}

%aximoa2
\begin{axioma}[Propiedad asociativa] $x+(y+z)=(x+y)+z, \; x(yz)=(xy)z$ \\
\end{axioma}

%axioma3
\begin{axioma}[Propiedad distributiva] $x(y+z)=xy+xz$ \\
\end{axioma}

%axioma4
\begin{axioma}[Existencia de elementos neutros] Existen dos números reales distintos que se indican por $0$ y $1$ tales que para cada número real $x$ se tiene:
$0+x=x+0=x \;$ y $1\cdot x = x\cdot 1 = 1$ \\
\end{axioma}

%axioma5
\begin{axioma}[Existencia de negativos] Para cada número real $x$ existe un número real y tal que $x+y=y+x=0$ \\
\end{axioma}

%axioma6
\begin{axioma}[Existena del recíproco] Para cada número real $x\neq 0$ existe un número real y tal que $xy=yx=1$ \\
\end{axioma}
\end{tcolorbox}

%teorema 1.1
\begin{teo}[Ley de simplificación para la suma]
Si $a+b=a+c$ entonces $b=c$ (En particular esto prueba que el número 0 del axioma 4 es único)\\\\
Demostración.- \;
Dado a+b=a+c. En virtud de la existencia de negativos, se puede elegir y de manera que $y+a=0$, con lo cual $y+(a+b)=y+(a+c)$ y aplicando la propiedad asociativa tenemos $(y+a)+b=(y+a)+c$ entonces, $0+b=0+c$. En virtud de la existencia de elementos neutros, se tiene $b=c$.\\
por otro lado este teorema demuestra que existe un solo número real que tiene la propiedad del 0 en el axioma 4. En efecto, si $0$ y $0^{'}$ tuvieran ambos esta propiedad, entonces $0+0^{'}=0$ y $0+0=0$; por lo tanto, $0+0^{'}=0+0$ y por la ley de simplificación para la suma $0=0^{'}$\\\\
\end{teo}

%teorema 1.2
\begin{teo}[Posibilidad de la sustracción]
Dado $a$ y $b$ existe uno y sólo un $x$ tal que $a+x=b$. Este $x$ se designa por $b-a$. En particular $0-a$ se escribe simplemente $-a$ y se denomina el negativo de $a$\\\\
Demostración.- \;
Dados $a$ y $b$ por el axioma 5 se tiene $y$ de manera que $a + y = 0$ ó $y=-a$, por hipótesis y teorema tenemos que $x=b-a$ sustituyendo $y$ tenemos $x=b+y$ y propiedad conmutativa $x=y+b$, entonces $a+x=a+(y+b)=(a+y)+b=0+b=b$ esto por sustitución, propiedad asociativa y propiedad de neutro, Por lo tanto hay por lo menos un $x$ tal que $a+x=b$. Pero en virtud del teorema 1.1, hay a lo sumo una. Luego hay una y sólo una $x$ en estas condiciones.\\\\ 
\end{teo}

%teorema 1.3
\begin{teo}
$b-a=b+(-a)$\\\\
Demostración.- \; Sea $x=b-a$ y sea $y=b+(-a)$. Se probará que $x=y$. por definición de $b-a$, $x+a=b$ y $y+a=\left[ b+(-a)\right]+a=b+\left[ (-a)+a \right]=b+0=b$, por lo tanto, $x+a=y+a$ y en virtud de teorema 1.1 $x=y$\\\\
\end{teo}

%teorema 1.4
\begin{teo}
$-(-a)=a$\\\\
Demostración.- \;
Se tiene $a+(-a)=0$ por definición de $-a$ incluido en el teorema 1.1. Pero esta igualdad dice que $a$ es el opuesto de $-a$, es decir, que si $a+(-a)=0$ entonces $a=0-(-a)=a=-(-a)$\\\\
\end{teo}

\section*{I 3.3 Ejercicios}
\begin{enumerate}[\bfseries \Large 1.]

%----------------------------------1--------------------------
\item Demostrar los teoremas del 1.5 al 1.15, utilizando los axiomas 1 al 6 y los teoremas I.1 al I.4\\\\

%teorema 1.5
\begin{teo}
$a(b-c)=ab-ac$\\\\
Demostración.- \;
Sea $a(b-c)$ por teorema 1.3 tenemos que $a\left[ b+(-c)\right]$  y por la propiedad distributiva $\left[ ab + a(-c) \right]$, y en virtud de los teorema 1.12 y 1.3 nos queda $ ab - ac $ \\\\
\end{teo}

%teorema 1.6
\begin{teo}
$0\cdot a = a\cdot 0 =0$\\\\
Demostración.- \;
Sea $0\cdot a$ por la propiedad conmutativa $a\cdot 0$, $a\cdot 0 + 0$ y $a\cdot 0 + \left[a+(-a) \right]$ y en virtud la propiedad asociativa y distributiva $a(0 + 1)+(-a)$ después $1(a)+(-a)$, luego por elemento neutro y existencia de negativos tenemos $0$, Así queda demostrado que cualquier número multiplicado por cero es cero.\\\\ 
\end{teo}

%teorema 1.7
\begin{teo}[Ley de simplificación para la multiplicación] Si $ab=ac$ y $a \neq 0$, entonces $b=c$. (En particular esto demuestra que el número 1 del axioma 4 es único)\\\\
Demostración.- \;
Sea $b$, $a\neq 0$, y por el existencia del recíproco tenemos $a\cdot a^{'}=1 $  luego,  $b=b\cdot 1=b\left[a(a^{'})\right]=(ab)(a^{'})=(ac)(a^{'})=c(a\cdot a^{'})=c\cdot 1=c$ por lo tanto queda demostrado la ley de simplificación.\\\\
\end{teo}

%teorema 1.8
\begin{teo}[Posibilidad de la división] Dados $a$ y $b$ con $a\neq 0$, existe uno y sólo un $x$ tal que $ax=b$. La $x$ se designa por $b/a$ ó $\displaystyle\frac{b}{a}$ y se denomina cociente de $b$ y $a$. En particular $1/a$ se escribe también $a^{-1}$ y se designa recíproco de $a$\\\\
Demostración .- \;
Sea $a$ y $b$ por axióma 6 se tiene un $y$ de manera que $a\cdot y = 1$ ó $y = a^{-1}$. Por hipótesis y teorema se tiene $x=b\cdot a^{-1}$, sustituyendo tenemos $x=y\cdot b$ entonces $ax=a(y\cdot b)=(a\cdot y)b=1\cdot b = b$  por lo tanto hay por lo menos un $x$ tal que $ax=b$ pero en virtud del teorema 1.7 hay por lo mucho uno, luego hay una y sólo una $x$ en estas condiciones.\\\\ 
\end{teo}

%teorema 1.9
\begin{teo}
Si $a\neq 0$, entonces $b/a=b\cdot a^{-1}$\\\\
Demostración.- \;
Sea $x =b/a$ y sea $y=b\cdot a^{-1}$ se probará que $x=y$, por definición de $b/a$, $ax=b$ y $ya=(b\cdot a^{-1})a=b(a^{-1}a)=b\cdot 1 = 1$, entonces $ya=xa$ y por la ley de simplificación para la multiplicación $y=x$ \\\\
\end{teo}

%teorema 1.10
\begin{teo}
Si $a\neq 0$, entonces $(a^{-1})^{-1}=a$\\\\
Demostración.- \;
Si $a\neq 0$ entonces  $(a^{-1})^{-1} = 1\cdot (a^{-1})^{-1} = \displaystyle\frac{1}{a^{-1}}=a$ esto por axioma de neutro, definición de $a^{-1}$ y teorema 1.9, así concluimos que $(a^{-1})^{-1}=a $ \\\\
\end{teo}

%teorema 1.11
\begin{teo}
Si $ab = 0$, entonces ó $a=0$ ó $b=0$\\\\
Demostración.- \;
Veamos dos casos, cuando $x\neq 0$ y cuando $x=0$\\
Si $x\neq 0$ y  $ab = 0$ entonces  $b=b\cdot 1 = b (a\cdot a^{-1}) = (ab)a^{-1}=0a^{-1}=0$, ahora si $a=0$ y virtud del teorema 1.6 nos queda demostrado que la multiplicación de dos números cualesquiera es igual a cero si $a=0$ ó $b=0$  \\\\
\end{teo}

%teorema 1.12
\begin{teo}
$(-a)b=-(ab) \; y \; (-a)(-b) = ab$\\\\
Demostración.- \;
Empecemos demostrando la primera proposición, Por la ley de simplificación para la suma podemos escribir como $(-a)b+ab=0$ entonces por la propiedad distributiva $b\left[ (-a)+a \right]$ por lo tanto $b\cdot 0$, luego por el teorema 1.6 queda demostrado la primera proposición.\\
Para demostrar la segunda proposición acudimos a la primera proposición, $(-a)(-b)=-\left[ a(-b)\right]$ y luego, \\ $-\left[ a(-b)+b+(-b)\right]=-\left[ (-b)(a+1)+b\right]=-\left[ (-b)(a+1)-1(-b)\right]=-\left[( -b)(a+1-1)\\\right]=-\left[ (-b)a\right]=-\left[ -(ab)\right]$ y en virtud  del el teorema 1.4 $(-a)(-b)=ab$ así queda demostrado la proposición.  \\\\
\end{teo}

%teorema 1.13
\begin{teo}
$\left(a/b \right) + \left(c/d \right) = \left( ad+bc  \right) / \left( bd \right) $ si $b\neq 0$ y $d\neq 0$.\\\\
Demostración.- \;
Si $\left(a/b \right) + \left(c/d \right)$ entonces por definición de $a/b$, $a\cdot b^{-1}+c\cdot d^{-1}=(a\cdot b^{-1})\cdot 1+(c\cdot d^{-1})\cdot 1=(a\cdot b^{-1})\cdot 1+(c\cdot d^{-1})\cdot 1=(a\cdot b^{-1})\cdot dd^{-1}+(c\cdot d^{-1})\cdot bb^{-1}$ por las propiedades asociativa  conmutativa y distributiva, $(b^{-1}d^{-1})(ad)+(b^{-1}d^{-1})(cb)=(b^{-1}d^{-1})(ad+cb)$, por lo tanto $(ad+bc)/bd$ esto por definición.\\\\
\end{teo}

%teorema 1.14
\begin{teo}
$(a/b)(c/d)=(ac)/(bd)$ si $b\neq 0$ y $d\neq 0$\\\\
Demostración.- \;
Por definición, $(ab^{-1})(cd^{-1})$, propiedades conmutativa y asociativa $(ac)(b^{-1}d^{-1})$, y por definición queda demostrado la proposición.\\\\
\end{teo}

%corolario 1.1
\begin{col.}
Si $c\neq 0$ y $d\neq 0$ entonces $(cd^{-1})=c^{-1}d$\\\\
Demostración.- \;
Por definición de $a^{-1}$ tenemos que $(cd^{-1})^{-1}=\displaystyle\frac{1}{cd^{-1}}$, por el teorema de posibilidad de la división $1=(c^{-1}d)(cd^{-1})$ y en virtud de los axiomas de conmutatividad y asociatividad $1=(c^{1}c)(dd^{-1})$, luego $1=1$. quedando demostrado el corolario.\\\\
\end{col.}

%teorema 1.15 
\begin{teo}
$(a/b)/(c/d)=(ad)/(bc)$ si $b\neq 0$,  $c\neq 0$ y $d\neq 0$\\\\
Demostración.- \;
Sea $(a/b)/(c/d)$ entonces por definición $(ab^{-1})(cd^{-1})^{-1}$, en virtud del corolario 1 se tiene que $(ab^{-1})(c^{-1}d)$, y luego por axioma conmutativa y asociativa $(ad)(c^{-1}b^{-1})$, así por definición concluimos que $(ad)/(cd)$\\\\
\end{teo}

%-------------------------------2--------------------------------------
\item $-0=0$\\\\
Demostración.- \;
Sabemos que por el axioma 5 $a+(-a)=0$, $- \left[ a+(-a) \right] = 0$ y $-a+-(-a)=0$ en virtud de teorema 1.12 y propiedad conmutativa $a+(-a)=0$, por lo tanto $0=0$.\\\\ 


%---------------------------------3-------------------------------
\item 
$1^{-1}=1$\\\\
Demostración.- \;
Por la existencia de elementos nuestros tenemos $1^{-1}\cdot 1$ y por axioma de existencia de reciproco $1=1$\\\\

%-------------------------------4-----------------------------
\item El cero no tiene reciproco\\\\
Demostración.- \;
Supongamos que el cero tiene reciproco es decir $0\cdot 0^{-1}=1$ pero por el teorema 1.6 se tiene que $0\cdot 0^{-1}=0$ y $0=1$ esto no es verdad, por lo tanto el cero no tiene reciproco.\\\\

%-----------------------------------5---------------------------------
\item $-(a+b)=-a-b$\\\\
Demostración.- \;
Por existencia de reciproco $-\left[1(a+b)\right]$ y teorema 1.12 $(-1)(a+b)$ luego por la propiedad distributiva $\left[ (-1)b \right] + \left[ (-1)b \right]$, una vez mas por el teorema 1.12 $-(1a)+ \left[-(1b)\right]$, en virtud del axioma 4 $-a+(-b)$, y teorema 1.3, $-a-b$ \\\\ 

%------------------------------------6----------------------------
\item $-(-a-b)=a+b$\\\\
Demostración\\\\
Si $-(-a-b)$ entonces por axioma $-\left[1(-a-b)\right]$, luego $(-1)(-a-b)=(-1)(-a)-\left[(-1)b\right])= (1\cdot a)-\left[-(1\cdot b)\right]$ y por axioma $a - \left[ - (b)\right]$, así por teorema $a+b$.\\\\

%------------------------------------7--------------------------
\item $(a-b)+(b-c)=a-c$\\\\
Demostración.- \;
Por definición tenemos $\left[ a+(-b) \right]+\left[ b+(-c) \right]$, y axiomas de asociatividad y conmutatividad $\left[ a+(-c) \right]+\left[ b+(-b) \right]$, luego por existencia de negativos  $\left[ a+(-c) \right] + 0$,, así $a+(-c)$ y $a-c$. \\\\

%-----------------------------------8--------------------------------
\item Si $a\neq 0$ y $b\neq 0$, entonces $(ab)^{-1} = a^{-1} b^{-1}$\\\\
Demostración.- \; Por hipótesis $\dfrac{1}{a} \dfrac{1}{b}$ luego $\dfrac{1}{ab}$ por lo tanto $(ab)^{-1}$\\\\

%------------------------------------9--------------------------------
\item $-(a/b)=(-a/b)=a/(-b)$ si $b\neq 0$\\\\
Demostración.- \;
Primero demostremos que $-(a/b)=(-a/b)$, Sea $b\neq 0$, en virtud de definición de la división y teorema 1.12 no queda que $-(a/b)=(-a)\cdot b^{-1}=-a/b$.\\
Ahora demostramos que $-(a/b)=(-a/b)=a/(-b)$, sea $b\neq0$, luego $-(b^{-1}\cdot a)=\left[-(b^-1)\right]\cdot a = a/-b$. \\\\

%------------------------------------10---------------------------------
\item $(a/b)-(c/d)=(ad-bc)/(bd)$ si $b\neq 0$ y $d\neq 0$\\\\
Demostración.- \;
Sea $b\neq 0$ y $d\neq 0$ y por definición $ab^{-1}-cd^{-1}$, luego por axiomas $(ab^{-1})(d\cdot d^{-1})-(cd^{-1})(b\cdot b^{-1})$, y en virtud del teorema 1.5 y propiedad asociativa $b^{-1}\cdot d^{-1}(ad-bc)$ y $(ad-bc)/bd$\\\\
\end{enumerate}

\section*{Axiomas de orden}
