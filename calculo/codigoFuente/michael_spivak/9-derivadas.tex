\chapter{Derivadas}

\begin{tcolorbox}
    \begin{def.}
    La función $f$ es \textbf{\boldmath diferenciable en $a$} si existe el  
    $$\lim_{h\to 0}\dfrac{f(a+h)-f(a)}{h}$$
    En este caso, dicho límite se presenta mediante $f'(a)$ y se denomina la derivada de $f$ en $a$. (Diremos también que $f$ es diferenciable si $f$ es diferenciable en $a$ para todo $a$ del dominio de $f.$) 
    \end{def.}
\end{tcolorbox}
