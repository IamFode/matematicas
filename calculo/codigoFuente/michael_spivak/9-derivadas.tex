\chapter{Derivadas}

\begin{tcolorbox}
    \begin{def.}
    La función $f$ es \textbf{\boldmath diferenciable en $a$} si existe el  
    $$\lim_{h\to 0}\dfrac{f(a+h)-f(a)}{h}$$
    En este caso, dicho límite se presenta mediante $f'(a)$ y se denomina la derivada de $f$ en $a$. (Diremos también que $f$ es diferenciable si $f$ es diferenciable en $a$ para todo $a$ del dominio de $f.$)\\\\
    La derivada $f'$ son representadps a menudo mediante
    $$\dfrac{dg(x)}{dx}, \qquad \dfrac{df(x)}{dx}\bigg|_{x=a}$$
    \end{def.}
\end{tcolorbox}

Recordemos que la diferenciabilidad se supone que es una mejora respecto a la simple continuidad. Esto lo demuestran los numerosos ejemplos de funciones que son continuas, pero no diferenciables; sin embargo, hay que destacar un punto importante:\\

% -------------------- teorema 1
\begin{teo}
    Si $f$ es diferenciable en el punto $a$, entonces $f$ es continua en $a$.\\\\
	Demostración.-\;
	$$\lim_{h\to 0} \left[f(a+h)-f(a)\right] = \lim_{h\to 0} \left[\dfrac{f(a+h)-f(a)}{h}\cdot h\right] = \lim_{h\to 0} \dfrac{f(a+h)-f(a)}{h}\cdot \lim_{h\to 0} h = f'(a)\cdot 0 = 0.$$\\
	La ecuación $\lim\limits_{h\to 0}f(a+h)-f(a)=0$ es equivalente a $\lim\limits_{x\to a} f(x)=f(a);$ así, $f$ es continua en $a$.\\\\
\end{teo}

Es muy importante recordar el Teorema 1, e igualmente importante recordar que el recíproco no es cierto. Una función diferenciable es continua, pero una función continua no necesariamente es diferenciable.\\

Las distintas funciones $f^{((k)}$, para $k\leq 2$ se denominan generalmente derivadas de orden superior de $f$. Y en la notación se tiene $$\dfrac{d\left(\frac{df(x)}{dx}\right)}{dx}\dfrac{d^2f(x)}{dx^2}.$$

\section{Problemas}

\begin{enumerate}[\bfseries 1]

    %-------------------- 1.
    \item 
	\begin{enumerate}[(a)]

	    %---------- (a)
	    \item Demuestre, aplicando directamente la definición, que si $f(x)=1/x$, entonces $f'(a)=-1/a^2$ para $a\neq 0$.\\\\
		Demostración.-\; Por definición se tiene $$f'(a)=\lim_{h\to 0}\dfrac{\frac{1}{a+h}-\frac{1}{a}}{h} = \lim_{h\to 0} \dfrac{\frac{a-(a+h)}{a(a+h)}}{h} = \lim_{h\to a}\dfrac{-h}{a(a+h)h} = -\dfrac{1}{a^2}, \quad \mbox{siempre que}\; x\neq 0.$$\\

	    %---------- (b)
	    \item Demuestre que la recta tangente a la gráfica de $f$ en el punto $(a,1/a)$ sólo corta a la gráfica de $f$ en este punto.\\\\
		Demostración.-\; La pendiente de la tangente cuando $x=a$ es $-\dfrac{1}{a^2}$, de donde la ecuación de la recta tangente es,
		$$y-\dfrac{1}{a} = -\dfrac{1}{a^2}(x-a)\quad \Rightarrow \quad a^2y-a=-x+a \qquad \Rightarrow \quad a^2y+x=2a.$$
		Luego $y=\dfrac{1}{x}$ ya que necesitamos encontrar el punto de intersección, y en consecuencia,
		$$a^2\dfrac{1}{x}+x=2a\quad \Rightarrow \quad x^2-2ax+a^2=0 \quad \Rightarrow \quad (x-a)^2=0 \quad \Rightarrow \quad x=a$$
		Por lo tanto, el único punto de intersección será $\left(a,\dfrac{1}{a}\right)$.\\\\

	\end{enumerate}

    %-------------------- 2.
    \item
	\begin{enumerate}[(a)]

	    %---------- (a)
	    \item Demuestre que si $f(x)=1/x^2$, entonces $f'(a)=-2/a_3$ para $a\neq 0$.\\\\
		Demostración.-\; Por definición tenemos,
		$$f'(a)=\lim_{h\to 0}\dfrac{\frac{1}{(a+h)^2}-\frac{1}{a^2}}{h} = \lim_{h\to 0} \dfrac{\frac{a^2-(a+h)^2}{a^2(a+h)^2}}{h} = \lim_{h\to a}\dfrac{-h(2a+h)}{ha^2(a+h)^2} = -\dfrac{2}{a^3}, \quad \mbox{siempre que}\; x\neq 0.$$\\

	    %---------- (b)
	    \item Demuestre que la recta tangente a $f$ en el punto $(a,1/a^2)$ corta a $f$ en otro punto, que se encuentra en el lado opuesto del eje vertical.\\\\
		Demostración.-\; Ya que $f'(a)=-\dfrac{2}{a^3}$ que representa la pendiente de la tangente entonces la ecuación de la tangente estará dada para $(a,1/a^2)$ por,
		$$y-y_1=m(x-x_1)\quad \Rightarrow \quad y-\dfrac{1}{a^2}=-\dfrac{2}{a^3}(x-a)\quad \Rightarrow \quad y=-\dfrac{2x}{a^3}+\dfrac{3}{a^2}$$
		luego resolviendo para $x$ e $y$ sabiendo que $y=\dfrac{1}{x^2}$ tenemos,
		$$x_1 = a,\quad x_2 = -\dfrac{a}{2} \qquad \mbox{e}\qquad y_1 = \dfrac{1}{a^2}, \quad y_2=\dfrac{4}{a^2}.$$
		Por lo que la tangente intersecta a la $f$ en $(a,1/a^2)$ y en $(-a/2,4/a^2)$. Así estos puntos son apuestos al eje vertical.\\\\



	\end{enumerate}

    %-------------------- 3.
    \item Demuestre que si $f(x) = \sqrt{x}$, entonces $f'(a) = 1/\left(2\sqrt{a}\right)$, para $a > 0$.\\\\
	Demostración.-\; Por definición,
	$$f'(a) = \lim_{h\to 0}\dfrac{\sqrt{a+h}-\sqrt{a}}{h} \cdot \dfrac{\sqrt{a+h}-\sqrt{a}}{\sqrt{a+h}-\sqrt{a}} = \lim_{a\to 0}\dfrac{a+h-a}{h\sqrt{a+h}\sqrt{a}} = \dfrac{1}{2\sqrt{a}}$$
	para $a>0$.\\\\

    %-------------------- 4.
    \item Para cada número natural $n$, sea $S_n(x)=x^n$. Recordando que $S_1'(x)=1, S_2'(x)=2x$, y que $S_3'(x)=3x^2$, encuentre una fórmula para $X_n'(x)$. Demuestre que la fórmula es correcta.\\\\
	Demostración.-\; Usando el teorema del binomio se tiene,
	$$\begin{array}{rcl}
	    S_n'(x)&=& \displaystyle\lim_{h\to 0}\dfrac{(x+h)^n-x^n}{h}\\\\
		   &=&\displaystyle\lim_{h\to 0} \dfrac{\displaystyle\sum_{j=0}^n {n\choose j} \left(x^{n-j} h^{j}\right)-x^n}{h}\\\\
		   &=&\displaystyle \lim_{h\to 0}\dfrac{h\left[\displaystyle\sum_{h\to 0}{n\choose j} x^{n-j}h^{j-1}\right]}{h}\\\\
		   &=&\displaystyle \lim_{h\to 0}\left[{n\choose 1}x^{n-1}h^{1-1}+{n\choose 2}x^{n-2}h^{2-1}+\ldots + {n\choose n}x^{0}h^{n-1}\right]\\\\
		   &=&\displaystyle \lim_{h\to 0}\left[{n\choose 1}x^{n-1}\cdot 1+{n\choose 2}x^{n-2}h^{2-1}+\ldots + {n\choose n}x^{0}h^{n-1}\right]\\\\
		   &=&\displaystyle{n\choose 1}x^{n-1} = nx^{n-1}\\\\
       \end{array}$$

    %-------------------- 5.
   \item Halle $f'$ si $f(x)=[x]$.\\\\
       Respuesta.-\; Sea $x$ un número no entero, entonces
       $$f'(x)=\lim_{h\to 0} \dfrac{[x+h]-[x]}{h}=\lim_{h\to 0}\dfrac{0}{h}=0.$$
       Luego si $x$ es un número entero, entonces por el límite por la izquierda se tiene,
       $$\lim_{h\to 0^-}\dfrac{[x+h]-[h]}{h}=\lim_{h\to 0^-}\dfrac{-1}{h}=\infty$$
       por último si $x$ es un número entero, entonces por el límite por la derecha se tiene,
       $$\lim_{h\to 0^+}\dfrac{[x+h]-[x]}{h}=\lim_{h\to 0^+}\dfrac{0}{h}=0.$$
       Por lo tanto, $f'(x)=0, \; \forall x \notin \mathbb{Z}$ y $f'(x)=\mbox{ no existe, } \forall x\in \mathbb{Z}$.\\\\

    %-------------------- 6.
   \item Demuestre, aplicando la definición:\\

   \begin{enumerate}[(a)]

       %---------- (a)
       \item si $g(x)=f(x)+c$, entonces $g'(x)=f'(x)$.\\\\
	   Demostración.-\; Por definición se tiene,
	   $$g'(x) = \lim_{h\to 0} \dfrac{f(x+h)+c-[f(x)-c]}{h} = \lim_{h\to 0}\dfrac{f(x+h)-f(x)}{h}=f'(x).$$\\\\

       %---------- (b)
       \item si $g(x)=cf(x)$, entonces $g'(x)=cf'(x)$.\\\\
	   Demostración.-\; Por definición,
	   $$g'(x) = \lim_{h\to 0} \dfrac{cf(x+h)-cf(x)}{h} = \lim_{h\to 0}c\left(\dfrac{f(x+h)-f(x)}{h}\right)=cf'(x).$$\\\\

    \end{enumerate}

    %-------------------- 7.
    \item Suponga que $f(x)=x^3$.\\

	\begin{enumerate}[(a)]

	    %---------- (a)
	    \item ¿Cuál es el valor de $f'(9), f'(25), f'(36)$?.\\\\
		Respuesta.-\; Se sabe que $f'(x)=x^n = nx^{n-1}$ por lo que 
		$$f'(x)=3x^{2}$$
		Así,
		$$\begin{array}{rclcl}
		    f'(9) &=& 3\cdot 9^{2} &=& 243.\\
		    f'(25) &=& 3\cdot 25^{2} &=& 1875.\\
		    f'(36) &=& 3\cdot 36^{2} &=& 3888.\\
		\end{array}$$
		\vspace{.7cm}

	    %---------- (b)
	    \item ¿Y el valor de $f'\left(3^2\right),f'\left(5^2\right),f'\left(6^2\right)$?.\\\\
		Respuesta.-\; Sea $f'(x)=3x^{2}$, entonces
		$$\begin{array}{rclcl}
		    f'\left(3^2\right) &=& 3\cdot \left(3^2\right)^{2} &=& 243.\\
		    f'\left(5^2\right) &=& 3\cdot \left(5^2\right)^{2} &=& 1875.\\
		    f'\left(6^2\right) &=& 3\cdot \left(6^2\right)^{2} &=& 3888.\\
		\end{array}$$
		\vspace{.7cm}


	    %---------- (c)
	    \item Calcule $f'\left(a^2\right),f'\left(x^2\right)$. Si no encuentra este problema trivial es que no tiene en cuenta una cuestión muy importante: $f'(x^2)$ significa la derivada de $f$ en el punto que denominamos $x^2$; no es la derivada en el punto $x$ de la función $g(x)=f\left(x^2\right)$. \\\\
		Respuesta.- Ya que $f'(x)=3x^{2}$, entonces
		$$\begin{array}{rclcl}
		    f'\left(a^2\right)&=&3\left(a^2\right)^2&=&3a^4\\
		    f'\left(x^2\right)&=&3\left(x^2\right)^2&=&3x^4\\
		\end{array}$$

	    %---------- (d)
	    \item Para aclarar la cuestión anterior. Si $f(x)=x^3$, compare $f'\left(x^2\right)$ y $g'(x)$ donde $g(x)=f\left(x^2\right)$.\\\\
		Respuesta.-\; Es $f'\left(x^2\right) = 3\left(x^2\right)^2 =3x^4$, pero 
		$$\begin{array}{rcl}
		    g'(x)&=&\lim\limits_{h\to 0}\dfrac{f(x+h)^2-f\left(x^2\right)}{h}\\\\
			 &=&\lim\limits_{h\to 0}\dfrac{\left[(x-h)^2\right]^3-\left(x^2\right)^3}{h}\\\\
			 &=&\lim\limits_{h\to 0}\dfrac{h\left(6x^5+15x^5h+20x^3h^2+15x^2h^3+6xh^4+h^5\right)}{h}\\\\
			 &=&6x^5.\\\\
	    \end{array}$$

	\end{enumerate}

    %-------------------- 8.
    \item 
	\begin{enumerate}[(a)]

	    %---------- (a)
	    \item Suponga que $g(x)=f(x+c).$ Demuestre (partiendo de la definición) que $g'(x)=f'(x+c)$. Dibuje un esquema para ilustrarlo. Para resolver el problema debe escribir las definiciones de $g'(x)$ y $f'(x+c)$ correctamente. El objetivo del Problema 7 era convencerle de que aunque este problema es fácil no se trata de una trivialidad, y que hay algo que debe demostrarse: no se puede simplemente poner primas en la ecuación $g(x) = f ( x + c)$.\\\\
		Demostración.-\; Por el hecho de que $g(x)=f(x+c)$ y por definición de diferencia tenemos, 
		$$g'(x) = \lim_{h\to 0}\dfrac{g(x+h)-g(x)}{h}=\lim_{h\to 0}\dfrac{f\left[(x+c)+h\right]-f(x-c)}{h} = f'(x+c).$$\\

	    %---------- (b)
	    \item Con el objeto de enfatizar el anterior punto. Demuestre que si $g(x)=f(cx)$, entonces $g(x)=c\cdot f'(cx)$. Intente también visualizar gráficamente por qué esta igualdad es cierta.\\\\
		Demostración.-\; Por el hecho de que $g(x)=f(cx)$ y por definición de diferencia se tiene,
		$$\begin{array}{rcl}
		    g'(x)&=&\lim\limits_{h\to 0}\dfrac{g(x+h)-g(x)}{h}\\\\
			 &=&\lim\limits_{h\to 0}\dfrac{f(cx+ch)-f(cx)}{h}\\\\
			 &=&\lim\limits_{h\to 0}\dfrac{c\left[f(cx+ch)-f(cx)\right]}{ch}\\\\
		\end{array}$$
		Sea $ch=k$ de donde
		$$c\lim_{k\to 0}\dfrac{f(cx+k)-f(cx)}{k}\quad \Rightarrow \quad cf'(cx).$$\\

	    %---------- (c)
	    \item Suponga que $f$ es diferenciable y periódica, con periodo $a$ (por ejemplo, $f(x+a)=f(x)$) para todo $x$). Demuestre que $f'$ es también periódica.\\\\
		Demostración.-\; Por hipótesis tenemos,
		$$f'(x)=\lim_{h\to 0}\dfrac{f(x+h)-f(x)}{h} = \lim_{h\to 0}\dfrac{f\left[(x+a)+h\right]-f(x-a)}{h} = f'(x+a).$$\\

	\end{enumerate}

    %-------------------- 9.
    \item Halle $f'(x)$ y también $f'(x+3)$ en los siguientes casos. Hay que ser muy metódico para no cometer un error en algún paso.\\

	\begin{enumerate}[(a)]

	    %---------- (a)
	    \item $f(x)=(x+3)^5$.\\\\
	    Respuesta.-\; si $g(x)=x^5$, entonces $g'(x)=5x^4$. Ahora, $f(x)=g(x+3)$ por tanto $$f'(x)=g'(x+3)=5(x+3)^4\quad  \mbox{y} \quad f'(x+3) = 5[(x+3)+3]^4 =5(x+6)^4.$$\\

	    %---------- (b)
	    \item $f(x+3)=x^5$.\\\\
		Respuesta.-\; Sea $t=x+3\; \Rightarrow \; x=t-3$, entonces
		$$f(t)=(t-3)^5.$$
		De donde, $f(x)=(x-3)^5$.\\

		Si $g(x)=x^5$, entonces $g'(x)=5x^4$. Ahora,  $f(x)=g(x-3)$, por tanto  
		$$f'(x) = 5(x-3)^4 \qquad \mbox{y}\qquad f'(x+3) = 5[(x+3)-3]^4 = 5x^4.$$\\

	    %---------- (c)
	    \item $f(x+3)=(x+5)^7$.\\\\
		Respuesta.-\; Sea $t=x+3$, de donde $x=t-3$, entonces
		$$f(t) = \left[(t-3)+5\right]^7 = (t+2)^7.$$
		Podemos reescribir esta última función como, $f(x)=(x+2)^7$.\\

		Sea $g(x)=x^7$ que implica $g'(x)=7x^6$, por lo que,
		$$f'(x)=g'(x+2)=7(x+2)^6\qquad \mbox{y}\qquad f'(x+3)=g'(x+3+2)=7(x+5)^6.$$\\

	\end{enumerate}

    %-------------------- 10.
    \item Halle $f'(x)$ si $f(x)=g(t+x),$ y si $f(t)=g(t+x)$. Las respuestas no son idénticas.\\\\
	Respuesta.-\; Por definición e hipótesis,
	$$f'(x)=\lim_{h\to 0} \dfrac{f(x+h)-f(x)}{h}=\lim_{h\to 0}\dfrac{g(t+x+h)-g(t+x)}{h}=\lim_{h\to 0}\dfrac{g\left[(t+x)+h\right]-g(t+x)}{h}=g'(t+x).$$
	Por otro lado,
	$$f'(x)=\lim_{h\to 0}\dfrac{f(x+h)-f(x)}{h}=\lim_{h\to 0}\dfrac{g(x+h+x)-f(x+x)}{h}=\lim_{h\to 0}\dfrac{g(2x+h)-f(2x)}{h} = g'(2x).$$\\

    %-------------------- 11.
    \item 
	\begin{enumerate}[(a)]

	    %---------- (a)
	    \item Demuestre que Galileo se equivocó: si un cuerpo cae una distancia $s(t)$ en $t$ segundos, y $s'$ es proporcional a $s$, entonces $s$ no puede ser una función de la forma $s(t)=ct^2.$\\\\
		Demostración.-\; Si $x$ es una función de la forma $s(t)=ct^2$ entonces $s'(t)=2ct$. Luego sustituyendo el valor $t=\sqrt{\dfrac{s}{c}}$, se tiene
		$$s'(t)=2c \sqrt{\dfrac{s}{c}}=2\sqrt{cs}.$$
		Dado que $s'(t)$ es directamente proporcional a $\sqrt{s(t)}$ entonces contradice la afirmación obtenida.\\\\ 

	    %---------- (b)
	    \item Demuestre que las siguientes afirmaciones sobre $s$ son ciertas, si $s(t)=(a/2)t^2$ (la primera afirmación demostrará por qué hemos hecho el cambio de $c$ a $a/2$):

		\begin{enumerate}[(i)]

		    %----- (i)
		    \item $s''(t)=a$ (la aceleración es constante).\\\\
			Demostración.-\; Sea $s(t)=\dfrac{a}{2}t^2$, entonces
			$$s'(t)=\dfrac{a}{2}\lim_{g\to 0}\dfrac{(t+h)^2-t^2}{h}=\dfrac{a}{2}(2t)=at.$$
			Así,
			$$s''(t)=\lim_{h\to 0}\dfrac{s'(t+h)-s'(t)}{h}=a\lim_{h\to 0}\dfrac{(t+h)-t}{h}=a.$$
			Donde vemos que la aceleración es constante.\\\\

		    %----- (ii)
		    \item $[s'(t)]^2=2as(t)$.\\\\
			Demostración.-\; Sea $t^2=\dfrac{2s(t)}{a}$, entonces
			$$\left[s'(t)\right]^2 = a^2t^2 = a^2\left[\dfrac{2s(t)}{a}\right] = 2as(t).$$\\

		\end{enumerate}

	    %---------- (c)
	    \item Si $s$ mide en pies, el valor de $a$ es $32$. ¿Cuántos segundos tendrá que permanecer fuera de la trayectoria de una lámpara que cae del techo, desde una altura de $400$ pues?. Si no se aparta, ¿cuál será la velocidad de la lámpara cuando le golpee? ¿A qué altura encontraba la lámpara cuando se desplazaba a la mitad de dicha velocidad?.\\\\
		Respuesta.-\; Ya que $a=32$, entonces
		$$s(t)=\dfrac{a}{2}t^2 \;\Rightarrow \; 400=\dfrac{32}{2}t^2 \; \Rightarrow \; t=5\; \mbox{s}.$$
		Luego,
		$$s'(t)=(2\cdot 32 \cdot 400)^{1/2} = 160\; \mbox{m/s}.$$
		Por último para $s'(t)=80$, tenemos
		$$(80)^2 = 2\cdot 32 \cdot s(t)\; \Rightarrow \; s(t)=100\; \mbox{pies desde arriba}.$$\\

	\end{enumerate}

    %-------------------- 12.
    \item Suponga que en una carretera el límite de velocidad se especifica en cada punto. En otras palabras, existe una cierta función $L$ tal que la velocidad límite a $x$ millas desde el inicio de la carretera es $L(x)$. Dos automóviles, $A$ y $B$, se desplazan por dicha carretera; la posición del automóvil $A$ en el tiempo $t$ es $a(t)$, y la del automóvil $B$ es $b(t)$.

    \begin{enumerate}[(a)]

	%---------- (a)
	\item ¿Qué ecuación expresa el hecho de que el automóvil $A$ siempre se desplaza a la velocidad límite? (La respuesta no es $a'(t) = L(t).)$\\\\
	    Respuesta.-\; La ecuación estará dada por,
	    $$a'(t)=L(x)=L\left[a(t)\right].$$

	%---------- (b)
	\item Suponga que $A$ siempre se desplaza a la velocidad límite, y que la posición de $B$ en el tiempo $t$ es la posición de $A$ en el tiempo $t - 1$. Demuestre que $B$ también se desplaza en todo momento a la velocidad límite.\\\\
	    Demostración.-\; Sea $b(t)=a(t-1)$ y $L(x)=L\left[a(t)\right]$, entonces para $B$ se tiene
	    $$b'(t)=L\left[b(t)\right]\quad \Rightarrow \quad a'(t-1)=L\left[a(t-1)\right].$$
	    Así, si $t-1$ es reemplazado por $t$, tenemos $a'(t)=L(t)$ el cual es cierto. Por lo que $B$ se desplaza a la velocidad límite.\\\\

	%---------- (c)
	\item Suponga, por el contrario, que $B$ siempre se mantiene a una distancia constante por detrás de $A$. ¿En qué condiciones $B$ se desplazará todavía en todo momento a la velocidad límite?.\\\\
	    Respuesta.-\; Sea $b(t)=a(t)-d$ para alguna constante $d>0$, entonces dado $a'(t)=L\left[a(t)\right]$, se tiene
	    $$b'(t)=L\left[b(t)\right] \; \Rightarrow \; b'(t)=L\left[a(t)-d\right] = L\left[a(t)\right] = L\left[b(t)+d\right]$$
	    Esto es, $B$ se desplaza a la velocidad límite, si $L$ es una función periódica con periodo $d$.\\\\
		
    \end{enumerate}

    %-------------------- 13.
    \item Suponga que $f(a)=g(a)$ y que la derivada por la izquierda de $f$ en $a$ es igual a la derivada por la derecha de $g$ en $a$. Defina $h(x)=f(x)$ para $x\leq a,$ y $h(x)=g(x)$ para $x\geq a$. Demuestre que $h$ es diferenciable en $a$.\\\\
	Demostración.-\; Sea 
	$$\lim_{t\to 0+}\dfrac{h(a+t)-h(a)}{t}=\lim_{t\to 0+}\dfrac{g(a+t)-g(a)}{t}$$
	$$\mbox{y}$$
	$$\lim_{t\to 0-}\dfrac{h(a+t)-h(a)}{t}=\lim_{t\to 0-}\dfrac{f(a+t)-f(a)}{t}.$$

	Y por el hecho de la existencia del límite por la derecha y por la izquierda entonces existe el límite:
	$$\lim_{t\to 0}\dfrac{h(a+t)-h(a)}{t}.$$\\

    %-------------------- 14.
    \item Sea $f(x)=x^2$ si $x$ es racional, y $f(x)=0$ si $x$ es irracional. Demuestre que $f$ es diferencial en $0$.\\\\
	Demostración.-\; Por definición, 
	$$f'(0)=\lim_{h\to 0}\dfrac{f(0+h)-f(0)}{h}=\lim_{h\to 0}\dfrac{h^2-0}{h}=0.$$
	Luego,
	$$\lim_{h\to 0^+}\lim_{h\to 0}\dfrac{f(0+h)-f(0)}{h}=0= \lim_{h\to 0^-}\dfrac{f(0+h)-f(0)}{h}$$

	Entonces, $f$ es diferencial en $0$.\\\\

    %-------------------- 15.
    \item 
	\begin{enumerate}[(a)]

	    %---------- (a)
	    \item Sea $f$ una función tal que $|f(x)|\leq x^2$ para todo $x$. Demuestre que $f$ es diferenciable en el punto $0$.\\\\
		Demostración.-\; Tenemos que
		$$|f(x)|\leq x^2 \quad \Rightarrow \quad |f(0)|=0\quad \Rightarrow \quad f(0)=0.$$
		Ahora, vemos que
		$$\bigg|\dfrac{f(h)}{h}\bigg|\leq \dfrac{h^2}{|h|}\quad \Rightarrow \quad \bigg|\dfrac{f(h)}{h}\bigg|\leq |h|.$$
		Por lo tanto,
		$$\begin{array}{rcl}
		    \lim\limits_{h\to 0}\dfrac{f(h)}{h}&\leq&\lim\limits_{h\to 0}|h|\\\\
		    \lim\limits_{h\to 0} \dfrac{f(h)}{h}&=&0\\\\
		    \lim\limits_{h\to 0} \dfrac{f(h)-0}{h}&=&0\\\\
		    \lim\limits_{h\to 0} \dfrac{f(0+h)-f(0)}{0}&=&0\\\\
					      f'(0)&=&0\\\\
		\end{array}$$
		Así, ya que $f'(0)$ existe entonces $f$ es diferencial en $0$.\\\\

	    %---------- (b)
	    \item Este resultado se puede generalizar si $x^2$ se sustituye por $|g(x)|$, en el caso de que $g$ cumpla una determina propiedad. ¿Cuál?.\\\\
		Respuesta.-\; Reemplacemos $x^2=|g(x)|$ por lo que nos queda
		$$|f(x)|\leq |g(x)|.$$
		Al ser $f$ diferenciable en $0$ entonces $f'(0)$ debe existir, por lo tanto
		$$|g(x)|\geq |f(x)|\quad \Rightarrow \quad |g(0)|\geq |f(0)|\quad \Rightarrow \quad |g(0)|\geq 0.$$
		Luego para $g$ tan pequeño como se quiera,
		$$|g(0)|=0\quad \Rightarrow \quad g(0)=0.$$
		Además podemos observar,
		$$\begin{array}{rcl}
		    \bigg|\dfrac{f(h)}{h}\bigg|&\leq&\bigg|\dfrac{g(h)}{h}\bigg|\\\\
		    \lim\limits_{h\to 0}\bigg|\dfrac{f(h)}{h}\bigg|&\leq&\lim\limits_{h\to 0}\bigg|\dfrac{g(h)}{h}\bigg|\\\\
		    \lim\limits_{h\to 0} \bigg|\dfrac{f(h)-f(0)}{h}\bigg| & \leq &\lim\limits_{h\to 0} \bigg|\dfrac{g(h)-f(0)}{h}\bigg|\\\\
					       |f'(0)|&\leq&|g'(0)|\\\\
					       0&\leq&|g'(0)|\\\\
					       |g'(0)|&\geq&0\\
		\end{array}$$	
		Luego ya que $g'(0)$ podría ser lo más pequeño que se quiera, entonces
		$$|g'(0)|=0\quad \Rightarrow \quad g'(0)=0.$$
		Por lo tanto $f$ es diferenciable en $0$ si se tiene,
		$$g(0)=0\qquad \mbox{y} \qquad g'(0).$$\\

	\end{enumerate}

    %-------------------- 16.
    \item Sea $\alpha>1.$ Si $f$ satisface $|f(x)|\leq |x|^{\alpha}$, demuestre que $f$ es diferenciable en $0$.\\\\
	Demostración.-\; Sea $|f(x)|\leq |x|^\alpha\;\Rightarrow \; -x^\alpha\leq f(x)\leq x^\alpha$ y $f(0)=0$ entonces,
	$$-h^{\alpha-1}\leq \dfrac{f(h)-f(0)}{h}\leq h^{\alpha-1}\quad \Rightarrow \quad -h^\alpha \leq f(h)\leq h^\alpha.$$
	Por lo que concluimos que $f$ es diferencial en $0$ y $f'(0)=0.$\\\\


    %-------------------- 17.
    \item Sea $0<\beta<1.$ Demuestre que si $f$ satisface $|f(x)|\geq |x|^{\beta}$ y $f(0)=a,$ entonces $f$ no es diferenciable en $0$.\\\\
	Demostración.-\; Sea $|f(x)|\geq |x|^\beta\;\Rightarrow \; -x^\beta\leq f(x)\leq x^\beta$ y $f(0)=0$ entonces,

	$$-h^{\beta-1}\leq \dfrac{f(h)-f(0)}{h}\leq h^{\beta-1}.$$
	Ya que $0<\beta < 1,$, $h^\beta$ será $\dfrac{1}{h^{1-\beta}}$, el cual tiende a $\infty$. Por lo tanto $f$ no es diferenciable en $0$.\\\\

    %-------------------- 18.
    \item Sea $f(x)=0$ si $x$ es irracional, y $1/q$ si $x=p/q$, fracción irreducible. Demuestre que $f$ no es diferenciable en $a$ para cualquier $a$.\\\\
	Demostración.-\; Si $a$ es racional, al no ser $f$ continua en $a$ será $f$ derivable en $a$. Si $a=0 a_1a_2a_3\ldots$ es irracional y $h$ es racional, entonces $a+h$ es irracional, co lo que $f(a+h)-f(a)=0$. Pero si $h = -0.00\ldots 0 a_{n+1}a_{n+2}\ldots,$ entonces $a+h=0a_1a_2\ldots a_n 000 \ldots$, con lo que $f(a+h)\geq 10^{-n}$, mientras que $|h|<10^{-n}$, la cual hace que $|[f(a+h)-f(a)]/h|\geq 1$. Así pues, $[f(a+h)-f(a)/h]$ es $0$ para valores de $h$ tan  pequeños como se quiera y tiene valores absolutos mayores que $1$ también con $h$ tan pequeño como se quiera, lo cual dice que $\lim\limits_{h\to 0}\dfrac{f(a+h)-f(a)}{h}$ no existe.\\\\

    %-------------------- 19.
    \item 
	\begin{enumerate}[(a)]

	    %---------- (a)
	    \item Suponga que $f(a)=g(a)=h(a)$, que $f(x)\leq g(x)\leq h(x)$ para todo $x$, y que $f'(a)=h'(a)$. Demuestre que $g$ es diferenciable en $a$, y que $f'(a)=g'(a)=h'(a)$.\\\\
		Demostración.-\; Ya que $f(x)\leq g(x)\leq h(x)$ y $f(a)=g(a)=h(a)$, entonces

		$$\begin{array}{rcccl}
		    f(x)&\leq&g(x)&\leq&h(x)\\\\
		    f(a+t)&\leq&g(a+t)&\leq&h(a+t)\\\\
		    \dfrac{f(a+t)-f(a)}{t}&\leq&\dfrac{g(a+t)-g(a)}{t}&\leq&\dfrac{h(a+t)-h(a)}{t}\\\\
		    \lim\limits_{h\to 0}\dfrac{f(a+t)f(a)}{t}&\leq&\lim\limits_{h\to 0}\dfrac{g(a+t)-g(a)}{t}&\leq&\lim\limits_{h\to 0}\dfrac{h(a+t)-h(a)}{t}\\\\
		    f'(a)&\leq&g'(a)&\leq&h'(a)\\\\
		\end{array}$$
		Luego sabemos que $f'(a)=h'(a)$ lo que implica $g'(a)$ existe y $f'(a)=g'(a)=h'(a)$. Por lo tanto $g$ es diferenciable en $a$.\\\\


	    %---------- (b)
	    \item Demuestre que la conclusión no es cierta si se omite la hipótesis $f(a)=g(a)=h(a)$.\\\\
		Demostración.-\; Se dará un contraejemplo sin la condición $f(a)=g(a)=h(a)$.\\
		Sea $f(x)=-1$, $g(x)=\dfrac{1}{1+e^{-x}}$ y $h(x)=2.$ Entonces
		    $$f(a) = -1,\quad g(a)=\dfrac{1}{1+e^{-a}},\quad h(a)=2.$$
		    Por lo que la conclusión no es cierta.\\\\

	\end{enumerate}

    %-------------------- 20.
    \item Sea $f$ una función polinómica; veremos en el próximo capítulo que $f$ es diferenciable. La recta tangente a $f$ en $\left(a,f(a)\right)$ es la gráfica $g(x)=f'(a)(x-a)+f(a)$. Por tanto, $f(x)-g(x)$ es la función polinómica $d(x)=f(x)-f'(a)(x-a)-f(a)$. Ya hemos visto que si $f(x)=x^2$, entonces $d(x)=(x-a)^2,$ y si $f(x)=x^3,$ entonces $d(x)=(x-a)^2(x-2a)$.\\

	\begin{enumerate}[(a)]

	    %---------- (a)
	    \item Halle $d(x)$ cuando $f(x)=x^4$, y demuestre que es divisible por $(x-a)^2$.\\\\
		Demostración.-\; Se tiene,
		$$\begin{array}{rcl}
		    d(x)&=&f(x)-f(a)(x-a)-f(a)=x^4-4a^3(x-a)-a^4\\\\
			&=&x^4-4a^3x+3a^4\\\\
			&=&(x-a)(x^3+ax^2+a^2x-3a^3)\\
		\end{array}$$
		Por lo que $d(x)$ es divisible por $(x-a)^2$.\\\\

	    %---------- (b)
	    \item Parece, ciertamente que $d(x)$ siempre sea divisible por $(x-a)^2$. En general, las rectas paralelas a la tangente cortan la gráfica de la función en dos puntos; la recta tangente corta a la gráfica sólo una vez cerca del punto, de manera que la intersección debería ser una doble intersección. Para dar una demostración riguroso, observe en primer lugar que 
	    $$\dfrac{d(x)}{x-a}=\dfrac{f(x)-f(a)}{x-a}-f'(a).$$
	    Ahora responda las siguientes cuestiones. ¿Por qué $f(x)-f(a)$ es divisible por $(x-a)$? ¿Por qué existe una función polinómica $h$ tal que $h(x)=d(x)/(x-a)$ para $x\neq a$? ¿Por qué el $\lim\limits_{x\to a} h(x) = 0$? ¿Por qué $h(a)=0$? ¿Por qué esto resuelve el problema?.\\\\
		Repuesta.-\; Tenemos $d(x)=f(x)-f'(a)(x-a)-f(a)$, entonces
		$$d(x)=f(x)-f'(a)(x-a)-f(a) \quad \Rightarrow \quad \dfrac{d(x)}{x-a}=\dfrac{f(x)-f(a)}{x-a}-f'(a)$$
		También sabemos que
		$$d(x)=[f(x)-f(a)]-f'(a)(x-a)$$
		de donde $d(x)$ es divisible por $(x-a)^2$ lo que implica que $f(x)-f(a)$ es divisible por $(x-a)$. Así,
		$$\dfrac{f(x)-f(a)}{x-a}$$
		es una función polinómica. Y por lo tanto la función $h$ es un polinomio, como sigue
		$$h(x)=\dfrac{d(x)}{x-a}\quad \Rightarrow \quad h(x)=\dfrac{f(x)-f(a)}{x-a}-f'(a)$$
		para $x\neq a$. Ahora tenemos 
		$$\begin{array}{rcl}
		    \lim\limits_{x\to a} h(x) &=&\lim\limits_{x\to a} \left[\dfrac{f(x)-f(a)}{x-a}-f'(a)\right]\\\\
		    \lim\limits_{x\to a} h(x) &=&\lim\limits_{x\to a} \left[\dfrac{f(x)-f(a)}{x-a}\right]-f'(a)\\\\
		    \lim\limits_{x\to a} h(x) &=&f'(a)-f'(a)\\\\
		    \lim\limits_{x\to a} h(x) &=&0\\\\
					      h(a)&=&0\\\\
		\end{array}$$

		    Significa que $h$ tiene a $a$ como una raíz. Esto indica que $\dfrac{d(x)}{x-a}$ es divisible por $x-a$. Y así, $d(x)$ es divisible por $(x-a)^2$.\\\\

	\end{enumerate}

    %-------------------- 21.
    \item 
	\begin{enumerate}[(a)]

	    %---------- (a)
	    \item Demuestre que $f'(a)=\lim\limits_{x\to a} \dfrac{f(x)-f(a)}{x-a}$.\\\\
		Demostración.-\; Sean $h=x-a$,  el hecho de que $\lim\limits_{x\to a}f(x)=\lim\limits_{h\to 0}f(a+h)$ y por un cambio infinitesimal $a\to a+h$, entonces 
		$$\begin{array}{rcl}
		     \lim\limits_{x\to a}\dfrac{f(x)-f(a)}{x-a}&=& \lim\limits_{x\to a+h}\dfrac{f(x)-f(a)}{x-a}\\\\
							       &=& \lim\limits_{h\to 0} \dfrac{f(a+h)-f(a)}{(a+h)-a}\\\\
							       &=& \lim\limits_{h\to 0}\dfrac{f(a+h)-f(a)}{h}\\\\
					&=& f'(a)\\\\
		\end{array}$$
		\vspace{.5cm}

	    %---------- (b)
	    \item Demuestre que las derivadas son una propiedad local: si $f(x)=g(x)$ para todo $x$ en algún intervalo abierto que contiene $a$, entonces $f'(a)=g'(a)$. (Esto significa que al calcular $f'(a)$, puede ignorarse a $f(x)$ para un determinado $x\neq a$. Evidentemente, ¡no! se puede ignorar a $f(x)$ para todos estos $x$ simultáneamente.)\\\\
		Demostración.-\; Se da $f$ y $g$ son iguales en un intervalo abierto que contiene $a$, entonces en el intervalo tenemos una pequeña cantidad $h\to 0$ como,
		$$\begin{array}{rcl}
		    \lim\limits_{h\to 0} f(a+h) &=& \lim\limits_{h\to 0} g(a+h)\\\\
		    \lim\limits_{h\to 0} \left[f(a+h)-f(a)\right] &=& \lim\limits_{h\to 0} \left[g(a+h)-g(a)\right]\\\\
		    \lim\limits_{h\to 0} \left[\dfrac{f(a+h)-f(a)}{h}\right] &=& \lim\limits_{h\to 0} \left[\dfrac{g(a+h)-g(a)}{h}\right]\\\\
									     f'(a)&=&g'(a)\\\\
		\end{array}$$

	\end{enumerate}

    %-------------------- 22.
    \item
	
\end{enumerate}
