\chapter{Diferenciación}

% -------------------- teorema 1
\begin{teo}
    Si $f$ es una función constante, $f(x)=c,$ entonces
    \begin{center}
	$f'(a)=0$ para todo número $a$.
    \end{center}
    \vspace{.5cm}
	Demostración.-\; $$f'(a)=\lim\limits_{h\to 0}\dfrac{f(a+h)-f(a)}{h}=\lim\limits_{h\to 0}\dfrac{(c+h)-c}{h}=0.$$\\\\
\end{teo}

% -------------------- teorema 2
\begin{teo}
    Si $f$ es la función identidad, $f(x)=x,$ entonces
    \begin{center}
	$f'(a)=1$ para todo número $a$.
    \end{center}
    \vspace{.5cm}
	Demostración.-\;
	$$f'(a)=\lim_{h\to 0}\dfrac{f(a+h)-f(a)}{h}=\lim\limits_{h\to 0}\dfrac{a+h-a}{h}=\lim_{h\to 0}\dfrac{h}{h}=1.$$\\
\end{teo}

% -------------------- teorema 3
\begin{teo}
    Si $f$ y $g$ son diferencibles en $a$, entonces $f+g$ es también diferenciable en $a$, y
    $$(f+g)'(a)=f'(a)+g'(a).$$\\
	Demostración.-\; 
	$$\begin{array}{rcl}
	    (f+g)'(a) &=& \lim\limits_{h\to 0} \dfrac{(f+g)(a+h)-(f+g)(a)}{h}\\\\
		      &=& \lim\limits_{h\to 0} \dfrac{f(a+h)+g(a+h)-\left[f(a)+g(a)\right]}{h}\\\\
		      &=& \lim\limits_{h\to 0} \left[\dfrac{f(a+h)-f(a)}{h}+\dfrac{g(a+h)-g(a)}{h}\right]\\\\
		      &=& \lim\limits_{h\to 0} \dfrac{f(a+h)-f(a)}{h} \lim\limits_{h\to 0}\dfrac{g(a+h)-g(a)}{h}\\\\
		      &=& f'(a)+g'(a).\\\\
	\end{array}$$
\end{teo}

% -------------------- teorema 4
\begin{teo}
    Si $f$ y $g$ son diferenciables en $a$, entonces $f\cdot g$ es también diferenciable en $a$, y
    $$(f\cdot g)'(a)=f'(a)\cdot g(a)+f(a)\cdot g'(a).$$\\
	Demostración.-\; 
	$$\begin{array}{rcl}
	    (f\cdot g)'(a) &=& \lim\limits_{h\to 0} \dfrac{(f\cdot g)(a+h)-(f\cdot g)(a)}{h}\\\\
			   &=& \lim\limits_{h\to 0} \dfrac{f(a+h)g(a+h)-f(a)g(a)}{h}\\\\
			   &=& \lim\limits_{h\to 0} \left[\dfrac{f(a+h)\left(g(a+h)-g(a)\right)}{h}+\dfrac{\left(f(a+h)-f(a)\right)g(a)}{h}\right]\\\\
			   &=& \lim\limits_{h\to 0} f(a+h)\cdot \lim\limits_{h\to 0} \dfrac{g(a+h)-g(a)}{h}+\lim\limits_{h\to 0} \dfrac{f(a+h)-f(a)}{h} \cdot \lim\limits_{h\to 0} g(a)\\\\
			   &=&f(a)\cdot g'(a)+f'(a)\cdot g(a).\\\\
	\end{array}$$
	Observemos que hemos utilizado el hecho de que $\lim\limits_{h\to 0}f(a+h)-f(a)=0\; \Rightarrow \; \lim\limits_{h\to 0}f(a+h)=f(a).$\\\\
\end{teo}

% -------------------- teorema 5
\begin{teo}
    Si $f(x)=cf(x)$ y $f$ es diferenciable en $a$, entonces $g$ es diferenciable en $a$, y
    $$g'(a)=c\cdot f'(a).$$\\
	Demostración.-\; Si $h(x)=c$, de manera que $g=h\cdot f$, entonces,
	$$\begin{array}{rcl}
	    g'(a)&=&(h\cdot f)'(a)\\
		 &=&h(a)\cdot f'(a)+h'(a)\cdot f(a)\\
		 &=&c\cdot f'(a)+0\cdot f(a)\\
		 &=&c\cdot f'(a).\\\\
	\end{array}$$
	En particular, $(-f)'(a)=-f'(a)$, por tanto $(f-g)'(a)=\left(f+[-g]\right)'(a)=f'(a)-g'(a).$\\\\
\end{teo}

% -------------------- teorema 6
\begin{teo}
    Si $f(x)=x^n$ para algún número natural $n$, entonces
    \begin{center}
	$f'(x)=nx^{n-1}$ para todo número $a$.
    \end{center}
    \vspace{.5cm}
    	Demostración.-\; La demostración la haremos por inducción sobre $n$. Para $n=1$ se aplica simplemente el teorema 2. Supongamos ahora que el teorema es cierto para $n$, de manera que si $f(x)=x^n$, entonces 
	$$f'(a)=na^{n-1}\; \mbox{para todo}\; a.$$
	Sea $g(x)=x^{n+1}$. Si $I(x)=x$, la ecuación $x^{n+1}=x^n\cdot x$ se puede escribir como
	\begin{center}
	    $g(x)=f(x)\cdot I(x)$ para todo $x$.
	\end{center}
	así, $g=f\cdot I$. A partir del teorema 4 deducimos que 
	$$\begin{array}{rcl}
	    g'(a)&=&\left(f\cdot I\right)'(a)\\
		 &=&f'(a)\cdot I(a)+f(a)\cdot I'(a)\\
		 &=&na^{n-1}\cdot a + a^n \cdot 1\\
		 &=&na^n+a^n\\
		 &=&(n+1)a^n,\; \mbox{para todo}\; a.\\\\
	\end{array}$$
	Este es precisamente el caso $n+1$ que queríamos demostrar.\\
	Si $f(x)=x^{-n}=\dfrac{1}{x^n}$ para algún número natural $n$, entonces
	$$f'(x)=\dfrac{-nx^{n-1}}{x^{2n}}=(-n)x^{-n-1};$$
	así es válido tanto para enteros positivos como negativos. Si interpretamos $f(x)=x^0$ como $f(x)=1$ y $0\cdot x^{-1}$ como $f'(x)=0$, entonces se verifica también para $n=0$.\\\\
\end{teo}

% -------------------- teorema 7
\begin{teo}
    Si $g$ es diferenciable en $a$ y $g(a)\neq 0,$ entonces $1/g$ es diferenciable en $a$, y 
    $$\left(\dfrac{1}{g}\right)'(a)=\dfrac{-g'(a)}{\left[g(a)\right]^2}.$$\\
	Demostración.-\; Incluso antes de escribir
	$$\dfrac{\left(\dfrac{1}{g}\right)(a+h)-\left(\dfrac{1}{g}\right)(a)}{h}$$
	debemos asegurarnos que esta expresión tiene sentido; es necesario comprobar que $(1/g)(a+h)$ está definido para valores suficientemente pequeños de $h$. Para ello son necesarias solamente dos observaciones. Como $g$ es, por hipótesis, diferenciable en $a$, se deduce del teorema 9-1 que $g$ es continua en $a$. Como $g(a)\neq 0$, deducimos también, a partir del teorema 6-3, que existe un $\delta>0$ tal que $g(a+h)\neq 0$ para $|h|<\delta$. Por tanto, $(1/g)=(a+h)$ tiene sentido para valores de $h$ suficientemente pequeños, y así podemos escribir
	$$\begin{array}{rcl}
	    \lim\limits_{h\to 0} \dfrac{\left(\dfrac{1}{g}\right)(a+h)-\left(\dfrac{1}{g}\right)(a)}{h}&=& \lim\limits_{h\to 0} \dfrac{\dfrac{1}{g(a+h)}-\dfrac{1}{g(a)}}{h}\\\\			
				 &=& \lim\limits_{h\to 0} \dfrac{g(a)-g(a+h)}{h\left[g(a)\cdot g(a+h)\right]}\\\\
				 &=& \lim\limits_{h\to 0} \dfrac{-\left[g(a+h)-g(a)\right]}{h}\cdot \dfrac{1}{g(a)g(a+h)}\\\\
				 &=& \lim\limits_{h\to 0} \dfrac{-\left[g(a+h)-g(a)\right]}{h}\cdot \lim\limits_{h\to 0}\dfrac{1}{g(a)\cdot g(a+h)}\\\\
				 &=&-g'(a)\cdot \dfrac{1}{\left[g(a)\right]^2}.\\\\
	\end{array}$$

\end{teo}

% -------------------- teorema 8
\begin{teo}
    Si $f$ y $g$ son diferenciables en $a$ y $g(a)\neq 0$, entonces $f/g$ es diferenciable en $a$, y
    $$\left(\dfrac{f}{g}\right)'(a)=\dfrac{g(a)\cdot f'(a)-f(a)\cdot g'(a)}{\left[g(a)\right]^2}$$
	Demostración.-\; Como $f/g=f\cdot (1/g)$ obtenemos
	$$\begin{array}{rcl}
	    \left(\dfrac{f}{g}\right)'(a) &=& \left(f\cdot \dfrac{1}{g}\right)'(a)\\\\
					  &=& f'(a)\cdot \left(\dfrac{1}{g}\right)(a)+f(a)\cdot \left(\dfrac{1}{g}\right)'(a)\\\\
					  &=& \dfrac{f'(a)}{g(a)}+\dfrac{f'(a)\cdot g(a)-f(a)\cdot g'(a)}{\left[g(a)\right]^2}\\\\
					  &=& \dfrac{f'(a)\cdot g(a)-f(a)\cdot g'(a)}{\left[g(a)\right]^2}.\\\\
	\end{array}$$
\end{teo}
\vspace{.7cm}

% -------------------- teorema 9
\begin{teo}[Regla de la cadena]
    Si $g$ es diferenciable en $a$ y $f$ es diferenciable en $g(a)$, entonces $f\circ g$ es diferenciable en $a$ y 
    $$\left(f\circ g\right)'(a)=f'\left[g(a)\right]\cdot g'(a).$$\\
	Demostración.-\; Definamos una función $\phi$ de la manera siguiente:
	$$\phi(h)=\left\{\begin{array}{ll}
		\dfrac{f\left[g(a+h)\right]-f\left[g(a)\right]}{g(a+h)-g(a)}, & \mbox{si}\; g(a+h)-g(a)\neq 0\\\\
		f'\left[g(a)\right], & \mbox{si}\; g(a+h)-g(a)=0.\\
	\end{array}\right.$$
	Se intuye fácilmente que $\phi$ es continua en $0$: cuando $h$ es pequeño, $g(a+h)-g(a)$ también es pequeño, de manera que si $g(a+h)-g(a)$ no es $0$, entonces $\phi(h)$ se aproximará a $f'\left[g(a)\right]$; y si es $0$ entonces $\phi(h)$ es igual a $f'\left[g(a)\right]$, lo que es mejor todavía. Ya que la continuidad de $\phi$ es el punto crucial de toda la demostración, vamos a desarrollar rigurosamente este argumento intuitivo.\\
	Sabemos que $f$ es diferenciable en $g(a)$. Esto significa que 
	$$\lim_{k\to 0}\dfrac{f(g(a)+k)-f(g(a))}{k}=f'(g(a)).$$
	Así, si $\epsilon>0$ existe algún número $\delta'>0$ tal que, para todo $k$,
	\begin{center}
	    si $0<|k|<\delta'$, entonces $\bigg|\dfrac{f(g(a)+k)-f(g(a))}{k}-f'(g(a))\bigg|<\epsilon.$
	\end{center}
	Pero $g$ es diferenciable en $a$ y  por lo tanto continua en $a$, de manera que existe un $\delta>0$ tal que para todo $h$,
	\begin{center}
	    si $|h|<\delta$, entonces $|g(a+h)-g(a)|<\delta'$
	\end{center}
	Consideremos ahora cualquier $h$ con $|h|<\delta$. Si $k=g(a+h)-g(a)\neq 0,$ entonces
	$$\phi(h)=\dfrac{f(g(a+h))-f(g(a))}{g(a+h)-g(a)}=\dfrac{f(g(a)+k)-f(g(a))}{k};$$
	se deduce que $|k|<\delta'$, y por tanto deducimos que 
	$$|\phi(h)-f'(h(a))|<\epsilon.$$
	Por otro lado, si $g(a+h)-g(a)=0$, entonces $\phi(g)=f'(g(a))$, de manera que se verifica ciertamente que 
	$$|\phi(h)-f'(g(a))|<\epsilon.$$
	Por tanto hemos demostrado que 
	$$\lim_{h\to 0}\phi(h)=f'(g(a)),$$
	o sea que $\phi$ es continua en $0$. El resto de la demostración es fácil. Si $h\neq 0$, entonces tenemos 
	$$\dfrac{f(g(a+h))-f(g(a))}{h}=\phi(h)\cdot \dfrac{g(a+h)-g(a)}{h}$$
	incluso aunque $g(a+h)-g(a)=0$ (ya que en este caso ambos miembros de la igualdad son iguales a $0$). Por tanto
	$$(f\circ g)'(a)=\lim_{h\to 0}\dfrac{f(g(a+h))-f(g(a))}{h}\lim_{h\to 0}\phi(h)\cdot \lim_{h\to 0}\dfrac{g(a+h)-g(a)}{h}=f'(g(a))\cdot g'(a).$$
\end{teo}

\section{Problemas}
\begin{enumerate}[\bfseries 1.]

    %-------------------- 1.
    \item Como ejercicio de precalentamiento, halle $f'(x)$ para cada una de las siguientes $f$. (No se preocupe por el dominio de $f$ o de $f'$; obtenga tan sólo una fórmula para $f'(x)$ que dé la respuesta correcta cuando tenga sentido.)\\

	\begin{enumerate}[(i)]

	    %-------------------- (i) 
	    \item $f(x)=\sen(x+x^2).$\\\\ 
		Respuesta.-\; $f'(x)=\cos(x+x^2)\cdot (1+2x).$\\\\

	    %-------------------- (ii)
	    \item $f(x)=\sen x + \sen x^2$.\\\\
		Respuesta.-\; $f'(x)=\cos x + \cos(x^2)\cdot 2x.$\\\\

	    %-------------------- (iii)
	    \item $f(x)=\sen(\cos x)$.\\\\
		Respuesta.-\; $f'(x)=\cos(\cos x)\cdot (-\sen x)=-\sen x \cos(\cos x).$\\\\

	    %-------------------- (iv)
	    \item $f(x)=\sen(\sen x)$.\\\\
		Respuesta.-\; $f'(x)=\cos(\sen x)\cdot \cos x = \cos x \cos(\sen x).$\\\\

	    %-------------------- (v)
	    \item $f(x)=\sen\left(\dfrac{\cos x}{x}\right)$.\\\\
		Respuesta.-\; $f'(x)=\cos\left(\dfrac{\cos x}{x}\right)\cdot \dfrac{-x\sen x - \cos x}{x^2}$.\\\\

	    %-------------------- (vi)
	    \item $f(x)=\dfrac{\sen(\cos x)}{x}$.\\\\
		Respuesta.-\; $f'(x)=\dfrac{\cos(\cos x)\cdot (-\sen x)}{x^2}=-\dfrac{\sen x \cos(\cos x)}{x^2}$.\\\\

	    %-------------------- (vii)
	    \item $f(x)=\sen(x+\sen x)$.\\\\
		Respuesta.-\; $f'(x)=\cos(x+\sen x)\cdot (1+\cos x)$.\\\\

	    %-------------------- (viii)
	    \item $f(x)=\sen\left[\cos(\sen x)\right]$.\\\\
		Respuesta.-\; $f'(x)=\cos\left[\cos(\sen x)\right]\left[-\sen(\sen x)\cos x\right]$.\\\\

	\end{enumerate}

    %-------------------- 2.
    \item Halle $f(x)$ para cada una de las siguientes funciones $f.$ (El autor tardó 20 minutos en calcular las derivadas para la sección de soluciones, y al lector no debería costarle mucho más tiempo calcularlas. Aunque la rapidez en los cálculos no es un objetivo de las matemáticas, si se desea tratar con aplomo las aplicaciones teóricas de la Regla de la Cadena, estas aplicaciones concretas deberían ser un juego de niños; a los matemáticos les gusta hacer ver que ni siquiera saben sumar, pero la mayoría pueden hacerlo cuando lo necesitan.)\\

	\begin{enumerate}[(i)]

	    %-------------------- (i)
	    \item $f(x)=\sen \left[(x+1)^2(x+2)\right]$.\\\\
		Respuesta.-\; 
		$$\begin{array}{rcl}
		f'(x)&=&\cos\left[(x+1)^2(x+2)\right]\cdot \left[2(x+1)(x+2)+(x+1)^2\right]\\
		     &=&(x+1)(3x+5)\cos\left[(x+1)^2(x+2)\right].\\
		\end{array}$$
		\vspace{0.7cm}

	    %-------------------- (ii)
	    \item $f(x)=\sen^3\left(x^2+\sen x\right)$.\\\\
		Respuesta.-\; $f'(x)=3\sen^2\left(x^2+\sen x\right)\cdot \cos\left(x^2+\sen x\right)\cdot (2x+\cos x)$.\\\\


	    %-------------------- (iii)
	    \item $f(x)=\sen^2\left[(x+\sen x)^2\right]$.\\\\
		Respuesta.-\; 
		$$\begin{array}{rcl}
		    f'(x)&=&2\sen \left[(x+\sen x)^2 \right] \cos \left[(x+\sen x)^2\right] \cdot 2(x+\sen x)(1+\cos x)\\
			 &=&4(1+\cos x)(x+\sen x)\sen\left[(x+\sen x)^2\right]\cos \left[(x+\sen x)^2\right]\\
		\end{array}$$
		\vspace{0.7cm}

	    %-------------------- (iv)
	    \item $f(x)=\sen\left(\dfrac{x^3}{\cos x^3}\right)$.\\\\
		Respuesta.-\; 
		$$\begin{array}{rcl}
		    f'(x)&=&\cos\left(\dfrac{x^3}{\cos x^3}\right)\cdot \dfrac{3x^2\cos\left(x^3\right)-x^3\left[-\sen\left(x^3\right)\right]3x^2}{\cos^2\left(x^3\right)}\\\\
			 &=&\cos\left(\dfrac{x^3}{\cos x^3}\right)\cdot \dfrac{3x^2\cos\left(x^3\right)+x^3\sen\left(x^3\right)3x^2}{\cos^2\left(x^3\right)}
		\end{array}$$
		\vspace{0.7cm}


	    %-------------------- (v)
	    \item $f(x)=\sen(x\sen x)+\sen\left(\sen x^2\right)$.\\\\
		Respuesta.-\; Aplicando las reglas de derivada se tiene,
		$$f'(x)=\cos(x\sen x)\sen x + x\cos x + \cos\left(\sen x^2\right)\cos\left(x^2\right)2x.$$\\

	    %-------------------- (vi)
	    \item $f(x)=f(x)=(\cos x)^{31^2}$.\\\\
		Respuesta.-\; Aplicando las reglas de derivada se tiene,
		$$f'(x)=-\left(31^2-1\right)(\cos x)^{31^2-1}\sen x.$$\\

	    %-------------------- (vii)
	    \item $f(x)=\sen^2 x \sen x^2\sen^2 x^2$.\\\\
		Respuesta.-\; Aplicando las reglas de derivada se tiene,
		$$\begin{array}{rcl}
		    f'(x)&=&2\sen x \cos x \sen\left(x^2\right)\sen^2\left(x^2\right)+\sen^2 x \left[\cos\left(x^2\right)2x\sen^2\left(x^2\right)\right.\\
			 &+&\left.\sen\left(x^2\right)2\sen\left(x^2\right)\cos\left(x^2\right)2x\right]\\
			 &=&2\sen x \cos x \sen\left(x^2\right)\sen^2\left(x^2\right)+2x\sen^2 x\cos\left(x^2\right)\sen^2\left(x^2\right)\\
			 &+&4x\sen^2 x \sen^2\left(x^2\right)\cos\left(x^2\right).\\
		\end{array}$$
		\vspace{0.7cm}

	    %-------------------- (viii)
	    \item $f(x)=\sen^3\left[\sen^2(\sen x)\right]$.\\\\
		Respuesta.-\; Aplicando las reglas de derivada se tiene,
		    $$f'(x)=3\sen^2\left[\sen^2(\sen x)\right]\cos\left[\sen^2(\sen x)\right]2\sen(\sen x)\cos(\sen x)\cos x.$$\\

	    %-------------------- (ix)
	    \item $f(x)=\left(x+\sen^5 x\right)^6$.\\\\
		Respuesta.-\; Aplicando las reglas de derivada se tiene,
		$$f'(x)=6\left(x+\sen^5 x\right)\left[1+5\sen^4 x \cos x\right].$$\\

	    %-------------------- (x)
	    \item $f(x)=\sen\left[\sen(\sen(\sen(\sen x)))\right]$.\\\\
		Respuesta.-\; Aplicando las reglas de derivada se tiene,
		$$f'(x)=\cos\left(\sen(\sen (\sen x))\right)\cdot \cos (\sen (\sen (\sen x)))\cdot \cos (\sen (\sen x))\cdot \cos (\sen x)\cdot \cos x.$$\\

	    %-------------------- (xi)
	    \item $f(x)=\sen\left[(\sen^7 x^7 + 1)^7\right]$.\\\\
		Respuesta.-\; Aplicando las reglas de derivada se tiene,
		$$f'(x)=\cos\left[\left(\sen^7 x^7 + 1\right)^7\right]\cdot 7\left(\sen^7 x^7 + 1\right)^6\cdot 7\sen^6 x^7 \cdot \cos x^7 \cdot 7x^6.$$\\

	    %-------------------- (xii)
	    \item $f(x)=\left\{\left[\left(x^2+x\right)^3+x\right]^4+x\right\}^5$.\\\\
		Respuesta.-\; Aplicando las reglas de derivada se tiene,
		$$f'(x)=5\left[\left(x^2+x\right)^3+x\right]^4\cdot \left\{1+4\left[\left(x^2+x\right)^3+x\right]^3\left[1+3\left(x^2+x\right)^2\left(1+2x\right)\right]\right\}.$$\\

	    %-------------------- (xiii)
	    \item $f(x)=\sen\left[x^2+\sen\left(x^2+\sen^2 x\right)\right]$.\\\\
		Respuesta.-\; Aplicando las reglas de derivada se tiene,
		$$f'(x)=\cos\left[x^2+\sen\left(x^2+\sen x^2\right)\right]\cdot\left[2x+\cos\left(x^2+\sen x^2\right)\cdot \left(2x+2x\cos x^2\right)\right].$$\\

	    %-------------------- (xiv)
	    \item $f(x)=\sen\left\{6\cos\left[6\sen\left(6\cos 6x\right)\right]\right\}$.\\\\
		Respuesta.-\; Aplicando las reglas de derivada se tiene,
		$$\begin{array}{rcl}
		    f'(x)&=&\cos\left\{6\cos\left[6\sen\left(6\cos 6x\right)\right]\right\}\cdot\left\{ -6\sen\left[6\sen(6\sen 6x)\right]\right\}\cdot6\cos(6\sen 6x)\cdot 6[-\sen(6x)]\cdot 6.\\
		    &=&6^4\cos\left\{6\cos\left[6\sen\left(6\cos 6x\right)\right]\right\}\cdot\sen\left[6\sen(6\sen 6x)\right]\cdot\cos(6\sen 6x)\cdot [-\sen(6x)]\cdot.\\
		\end{array}$$
		\vspace{.7cm}

	    %-------------------- (xv)
	    \item $f(x)=\dfrac{\sen x^2 \sen^2 x}{1+\sen x}$.\\\\
		Respuesta.-\; Aplicando las reglas de derivada se tiene,
		$$\begin{array}{rcl}
		    f'(x)&=&\dfrac{\left[\sen\left(x^2\right)2x\sen^2 x +\sen \left(x^2\right)2\sen x\cos x \right]\cdot (1+\sen x)-\sen \left(x^2\right)\sen^2 x\cos x}{(1+\sen x)^2}\\\\
		    &=&\dfrac{\left[2x\sen\left(x^2\right)\sen^2 x +2\sen \left(x^2\right)\sen x\cos x \right]\cdot (1+\sen x)-\sen \left(x^2\right)\sen^2 x\cos x}{(1+\sen x)^2}\\\\
		\end{array}$$
		\vspace{.7cm}

	    %-------------------- (xvi)
	    \item $f(x)=\dfrac{1}{x-\dfrac{2}{x+\sen x}}$.\\\\

		Respuesta.-\; Aplicando las reglas de derivada se tiene,

		$$f'(x)=\dfrac{-\left[1-\dfrac{-2(1+\cos x)}{(x+\sen x)^2}\right]}{\left(x-\dfrac{2}{x+\sen x}\right)^2}$$\\

	    %-------------------- (xvii)
	    \item $f(x)=\sen\left[\dfrac{x^3}{\sen\left(\dfrac{x^3}{\sen x}\right)}\right]$.\\\\

		Respuesta.-\; Aplicando las reglas de derivada se tiene,

		$$f'(x)=\cos\left[\dfrac{x^3}{\sen\left(\dfrac{x^3}{\sen x}\right)}\right]\cdot \left[  \dfrac{3x^2\cdot \sen\left(\dfrac{x^3}{\sen x}\right)-x^3\cdot \cos \left(\dfrac{x^3}{\sen x}\right) \cdot \dfrac{3x^2\cdot \sen x - x^3\cdot \cos x}{\sen^2 x}}{\sen^2\left( \dfrac{x^3}{\sen x} \right)} \right].$$\\

	    %-------------------- (xviii)
	    \item $f(x)=\sen\left[\dfrac{x}{x-\sen\left(\dfrac{x}{x-\sen x}\right)}\right]$.\\\\

		Respuesta.-\; Aplicando las reglas de derivada se tiene,

	    $$\begin{array}{rcl}
		f'(x)&=&\cos\left[\dfrac{x}{x-\sen\left(\dfrac{x}{x-\sen x}\right)}\right]\cdot \\\\
		     && \dfrac{\left[x-\sen\left(\dfrac{x}{x-\sen x}\right)\right]-x\left[1-\cos\left(\dfrac{x}{x-\sen x}\right)\cdot \dfrac{(x-\sen x)-x(1-\cos x)}{(x-\sen x)^2}\right]}{\left[x-\sen\left(\dfrac{x}{x-\sen x}\right)\right]^2}
	    \end{array}$$
	    \vspace{.7cm}

	\end{enumerate}

    %-------------------- 3.
    \item Halle las derivadas de las funciones $\tan$, $\cotan$, $\sec$, $\cosec$. (No es necesario memorizar estas fórmulas, aunque se necesitarán de vez en cuando; si se expresan las soluciones de manera correcta, resultan sencillas y algo simétricas.)\\\\
	Respuesta.-\; Sea $f(x)=\tan x = \dfrac{\sen x}{\cos x}$, entonces
	$$f'(x)=\dfrac{\cos x \cos x  - \sen (-\sen x)}{\cos^2 x}=\dfrac{\cos^2 x + \sen^2 x}{\cos^2 x}=\dfrac{1}{\cos^2x}=\sec^2 x.$$

	Sea $f(x)=\cotan x = \dfrac{1}{\tan x}=\dfrac{\cos x}{\sen x}$, entonces
	$$f'(x)=\dfrac{-\sen x\sen x - \cos x \cos x}{\sen^2 x}=-\dfrac{1}{\sen^2 x}=\cosec^2 x.$$

	Sea $f(x)=\sec x = \dfrac{1}{\cos x} = \cos^{-1}x$, entonces
	$$f'(x)=-\cos^{-2}x\cdot (-\sen x)=\dfrac{\sen x}{\cos^2 x}=\tan x \sec x.$$

	Sea $f(x)=\cosec = \dfrac{1}{\sen x} = \sen^{-1}x$, entonces
	$$f'(x)=-\sen^{-2}x\cdot \cos x = -\dfrac{\cos x}{\sen^2 x} = -\cosec x \cotan x.$$\\

    %-------------------- 4.
    \item Para cada una de las siguientes funciones $f$, halle $f'(f(x))$ no $(f\circ f)(x)$.\\

	\begin{enumerate}[(i)]

	    %---------- (i)
	    \item $f(x)=\dfrac{1}{1+x}.$\\\\
		Respuesta.-\;  Sea $f'(x)=-\dfrac{1}{(1+x)^2}$, entonces
		$$f'\left(\dfrac{1}{1+x}\right)=-\dfrac{1}{\left(1+\dfrac{1}{1+x}\right)^2}=-\left(\dfrac{1+x}{2+x}\right)^2.$$\\

	    %---------- (ii)
	    \item $f(x)=\sen x.$\\\\
		Respuesta.-\; Se tiene $f'(\sen x) = \cos(x) =\cos(\sen x).$\\\\

	    %---------- (iii)
	    \item $f(x)=x^2$.\\\\
		Respuesta.-\; Se tiene $f'\left(x^2\right)=2x=2x^2.$\\\\

	    %---------- (iv)
	    \item $f(x)=17.$\\\\
		Respuesta.-\; Se tiene $f'(17)=0.$\\\\

	\end{enumerate}

    %-------------------- 5.
    \item Para cada una de las siguientes funciones $f$, halle $f\left[f'(x)\right]$.\\

	\begin{enumerate}[(i)]

	    %---------- (i)
	    \item $f(x)=\dfrac{1}{x}$.\\\\
		Respuesta.-\; Sea $f'(x)=-\dfrac{1}{x^2}$, entonces
		$f\left[f'(x)\right]=f\left(-\dfrac{1}{x^2}\right)=\dfrac{1}{-\dfrac{1}{x^2}}=-x^2.$\\\\

	    %---------- (ii)
	    \item $f(x)=x^2$.\\\\
		Respuesta.-\; Sea $f'(x)=2x$, entonces
		$f'\left(2x\right)=(2x)^2=4x^2.$\\\\

	    %---------- (iii)
	    \item $f(x)=17$.\\\\
		Respuesta.-\; Sea $f'(17)=0$, entonces
		$f'\left(17\right)=17$\\\\

	    %---------- (iv)
	    \item $f(x)=17x$.\\\\
		Respuesta.-\; Sea $f'(17x)=17$, entonces $f'\left(17\right)=17\cdot 17=289$.\\\\

	\end{enumerate}

    %-------------------- 6.
    \item Halle $f'$ en función de $g'$ si

	\begin{enumerate}[(i)]

	    %---------- (i)
	    \item $f(x)=g(x+g(a))$.\\\\
		Respuesta.-\; Por las reglas de derivación tenemos,
		$$f'(x)=g'\left[x+g(a)\right]\cdot \left[x+g(a)\right]'=g'\left[x+g(a)\right].$$\\

	    %---------- (ii)
	    \item $f(x)=g(x\cdot g(a))$.\\\\
		Respuesta.-\; Por las reglas de derivación tenemos,
		$$f'(x)=g'\left[x\cdot g(a)\right]\cdot \left[x\cdot g(a)\right]'=g'\left[x\cdot g(a)\right]\cdot g(a).$$\\

	    %---------- (iii)
	    \item $f(x)=g(x+g(x))$.\\\\
		Respuesta.-\; Por las reglas de derivación tenemos,
		$$f'(x)=g'\left[x+g(x)\right]\left[x+g(x)\right]'=g'\left[x+g(x)\right]\left[1-g'(x)\right].$$\\

	    %---------- (iv)
	    \item $f(x)=g(x)(x-a)$.\\\\
		Respuesta.-\; Por las reglas de derivación tenemos,
		$$f'(x)=g'(x)(x-a)+g'(x).$$\\

	    %---------- (v)
	    \item $f(x)=g(a)(x-a)$.\\\\
		Respuesta.-\; Por las reglas de derivación tenemos,
		$$f'(x)=g'(a)(x-a)+g(a)(x-a)' = g(a).$$\\

	    %---------- (vi)
	    \item $f(x+3)=g\left(x^2\right)$.\\\\
		Respuesta.-\; Sea $z=x+3\; \Rightarrow \; x=z-3$, entonces
		$$f'(z)=g'\left[(z-3)^2\right]\cdot \left[(z-3)^2\right]'=g'\left[(z-3)^2\right](2(z-3)=2g'\left[(z-3)^2\right](z-3).$$\\

	\end{enumerate}

    %-------------------- 7.
    \item 
	\begin{enumerate}[(a)]

	    %---------- (a)
	    \item Un objeto circular va aumentando de tamaño de manera no especificada, pero se sabe que cuando el radio es $6$, la tasa de variación del mismo es $4$. Halle la tasa de variación del área cuando el radio es $6$. (Si $r(t)$ y $A(t)$ representan el radio y el área en el tiempo $t$, entonces las funciones $r$ y $A$ satisfacen $A = \pi r^2$; tan sólo es necesario aplicar directamente la Regla de la Cadena.)\\\\
		Respuesta.-\; Encontrando la primera derivada con respecto de $t$ para se tiene,
		$$A'(t)=2\pi r\cdot r'(t)$$
		Dado que $r'(t)=4$ cuando $r=6$, entonces
		$$A'(6)=2\pi\cdot 6 \cdot 4 = 48\pi.$$\\

	    %---------- (b)
	    \item Suponga que el objeto circular que hemos estado observando es la sección transversal de un objeto esférico. Halle la tasa de variación del volumen cuando el radio es $6$. (Es necesario conocer la fórmula del volumen de una esfera; en caso de que el lector la haya olvidado, el volumen es $\frac{4}{3}\pi$ veces el cubo del radio.)\\\\
		Respuesta.-\; Encontrando la primera derivada con respecto de $t$ para se tiene,
		$$V'(t)=\frac{4}{3}\pi\cdot r^3 \cdot r'(t)$$
		Dado que $r'(t)=4$ cuando $r=6$, entonces
		$$V'(6)=4\pi\cdot 6^2 \cdot 4 = 576\pi.$$\\

	    %---------- (c)
	    \item Suponga ahora que la tasa de variación del área de la sección transversal circular es $5$ cuando el radio es $3$. Halle la tasa de variación del volumen cuando el radio es $3$. Este problema se puede resolver de dos maneras: primero, utilizando las fórmulas del área y el volumen en función del radio; y después expresando el volumen en función del área (para utilizar este método se necesita el Problema 9-3).\\\\
		Respuesta.-\; Sabemos que 
		$$A'(t)=2\pi r r'(t)$$
		y $A'=5$ cuando $r=3$, entonces
		$$5=2\pi 3r'$$
		Luego dividimos ambos lados por $6\pi$, de donde
		$$r'=\dfrac{5}{6\pi}.$$
		Sustituyendo el valor de $r$ y $r'$ nos queda
		$$V'8t)=4\pi r^2 r'(t)\quad \Rightarrow \quad V'=4\pi\cdot 3^2 \cdot \dfrac{5}{6\pi}=30.$$\\

	\end{enumerate}

    %-------------------- 8.
    \item El área entre dos círculos concéntricos variables vale siempre $9\pi\; cm^2$. La tasa de cambio del área del círculo mayor es de $10\pi\; cm^2/seg$. ¿A qué velocidad varía la circunferencia del círculo pequeño cuando su área es de $16\pi cm^2$?.\\\\
	Respuesta.-\; Sea $r_1$ y $r_2$ que representa el radio de de los círculos más pequeños y más grandes respectivamente. El área entre los dos círculos está dada por:
	$$A=\pi\left[(r_2)^2-(r_1)^2\right]$$
	Reemplacemos $A$ con $9\pi$, de donde 
	$$9\pi = \pi\left[(r_2)^2-(r_1)^2\right] \quad \Rightarrow \quad (r_2)^2-(r_1)^2=9$$
	Derivando tenemos,
	$$2r_2r'^2 - 2r_1r'_1 = 0\quad \Rightarrow \quad r_2'=\dfrac{r_1}{r_2}r_1'\qquad (1)$$

	Por otro lado el área del circulo mayor es dado por,
	$$A_2=\pi(r_2)^2$$
	Derivando se tiene,
	$$A_2'=2\pi r_2 r_2'$$

	Dada que la tasa de cambio del área del circulo más grande es $10\pi$, entonces
	$$10\pi = 2\pi r_2 r_2' \quad \Rightarrow \quad r_2'=\dfrac{5}{r_2}\qquad (2)$$
	Igualando (1) y (2),
	$$\dfrac{5}{r_2}=\dfrac{r_1}{r_2}r_1'\quad \Rightarrow \quad r_1'=\dfrac{5}{r_1}.$$\\
	El área del circulo pequeño es dada por,
	$$A_1=\pi(r_1)^2$$
	reemplazando $A_1$ con $16\pi$,
	$$16\pi=\pi(r_1)^2 \quad \Rightarrow \quad r_1=4$$
	Por lo tanto 
	$$r_1'=\dfrac{5}{4}.$$
	Así la circunferencia del circulo pequeño es,
	$$C'=2\pi r_1' \quad \Rightarrow \quad C'=2\pi \dfrac{5}{4} = \dfrac{5}{2}\pi.$$\\

    %-------------------- 9.
    \item Una partícula $A$ se desplaza a lo largo del eje horizontal positivo, y una partícula $B$ a lo largo de la gráfica de $f(x)=-\sqrt{3}x,\; x\leq 0$. En un momento dado, $A$ se encuentra en el punto $(5,0)$ y se desplaza a una velocidad de $3$ unidades del origen y se desplaza a una velocidad de $4$ unidades/seg. ¿Cuál es la tasa de variación de la distancia entre $A$ y $B$?.\\\\
	Respuesta.-\; Sean $x_1$ la coordenada el eje $x$ de $A$ y $x_2$ representa el eje $x$ de $B$ en el momento $t$ y el eje $y$ de $B$ en el momento $t$ es $-\sqrt{3}x_2.$\\
	La distancia $d$ entre $A$ y $B$ puede ser calculado usando el teorema de Pitágoras como,
	$$d^2=(x_1-x_2)^2+(-\sqrt{3}x_2)^2$$
	Derivando con respecto $t$ es,
	$$2d\cdot d' = 2(x_1-x_2)(x_1'-x_2')+2(-\sqrt{3}x_2)x_2' \quad \Rightarrow \quad d\cdot d'=(x_1-x_2)(x_1'-x_2')+(-\sqrt{3}x_2)x_2'\qquad (1)$$
	La distancia entre $B$ y el origen es 
	$$s=\sqrt{x_2^2+(-\sqrt{3}x_2)^2}=2x_2$$
	Derivando con respecto $t$ es,
	$$s'=2x_2'\quad \Rightarrow \quad x_2'=\dfrac{1}{2}s'.$$
	En el momento dado cuando $B$ es $3$ unidades desde el origen se tiene $3=2x_2$. Ya que $x_2\leq 0,$ entonces
	$$x_2=-\dfrac{3}{2}.$$
	Reemplazando $x_1=5$, $x_2=-\dfrac{3}{2},$ $x_1'=3$, $x_2'=\dfrac{1}{2}$, $s'=-2$ y $d=\sqrt{(5+\frac{3}{2})^2+(-\sqrt{3}\frac{3}{2})^2}=7$ en (1) se tiene,
	$$7d'=\left(5+\dfrac{3}{2}\right)(3+2)+\left(-\sqrt{3}\dfrac{-3}{2}\right)(-2)=27.3 \quad \Rightarrow \quad d'=3.9.$$\\


    %-------------------- 10.
    \item Sea $f(x)=x^2\sen 1/x$ para $x\neq 0$ y $f(0)=0$. Supongamos también que $h$ y $k$ son dos funciones tales que 
	$$h'(x)=\sen^2\left[\sen(x+1)\right]\qquad k'(x)=f(x+1)$$
	$$h(0)=3\qquad k(0)=0$$
	Halle

	\begin{enumerate}[(i)]

	    %---------- (i)
	    \item $(f\circ h)'(0)$.\\\\
		Respuesta.-\; Se tiene,
		$$\begin{array}{rcl}
		    (f\circ h)'(0)&=&f'\left[h(0)\right]h'(0) = f'(3)h'(0)\\\\
				  &=& \left[2\cdot 3\sen \dfrac{1}{3}+3^2\cos \dfrac{1}{3}\left(-\dfrac{1}{3^2}\right)\right]\cdot \sen^2\left[\sen(0+1)\right]\\\\
				  &=&\left(6\sen \dfrac{1}{3}-\cos \dfrac{1}{3}\right)\cdot \sen^2\left[\sen(1)\right]\\\\
		\end{array}$$
		\vspace{.7cm}

	    %---------- (ii)
	    \item $(k\circ f)'(0)$.\\\\
		Respuesta.-\; Sea $f(0)=0$, entonces
		$$\begin{array}{rcl}
		    (k\circ f)'(0)&=&k'\left[f(0)\right]\cdot f'(0)\\\\
				  &=&f(0+1)\cdot 0\\\\
				  &=&0\\\\
		\end{array}$$
		\vspace{.7cm}


	    %---------- (iii)
	    \item $\alpha'\left(x^2\right)$, donde $\alpha(x)=h\left(x^2\right)$. Ir con mucho cuidado en la resolución de este apartado.\\\\
		Respuesta.-\; Sea 
		$$\alpha'(x)=\left[h\left(x^2\right)\right]' = h'\left(x^2\right)\cdot \left(x^2\right)' = \sen^2\left[\sen\left(x^2+1\right)\right]\cdot 2x$$
		Así, para $\alpha'\left(x^2\right)$ se tiene,
		$$\alpha'\left(x^2\right)=2x^2\sen^2\left\{\sen\left[\left(x^2\right)^2+1\right]\right\}=2x^2 \sen^2\left[\sen\left(x^4+1\right)\right].$$\\

	\end{enumerate}

    %-------------------- 11.
    \item Halle $f'(0)$ si 
	$$f(x)=\left\{\begin{array}{ll}
		g(x)\sen \dfrac{1}{x},&x\neq 0\\\\
		0,&x=0\\
	\end{array}\right.$$
	 y $$g(0)=g'(0)=0.$$\\
	 Respuesta.-\; Por definición se tiene,
	 $$f'(0)=\lim\limits_{h\to 0}\dfrac{f(h)-f(0)}{h}=\lim\limits_{h\to 0}\dfrac{g(h)\sen \frac{1}{h}-0}{h}=\lim\limits_{h\to 0}\dfrac{g(h)}{h}\sen \dfrac{1}{h}.$$
	 Ya que $\lim\limits_{x \to 0} g(x)=0 \; \Rightarrow \; \lim\limits_{x\to 0}g(x)\sen \dfrac{1}{x}=0$, entonces
	 $$\lim\limits_{h\to 0}\dfrac{g(h)}{h}\sen\dfrac{1}{h}=0=f'(0).$$\\

    %-------------------- 12.
     \item Utilizando la derivada de $f(x)=1/x$, tal como se ha hallado en el problema 9-1, calcule $(1/g)'(x)$ mediante la regla de la cadena.\\\\
	 Respuesta.-\; Sea $\dfrac{1}{g}=f\left[g\right]=f\circ g$,  por la regla de la cadena se tiene,
	 $$\left(f\circ g\right)'(x)=f'\left[g(x)\right]\cdot g'(x)=\dfrac{-1}{\left[g(x)\right]^2}\cdot g'(x)=\dfrac{-g'(x)}{\left[g(x)\right]^2}.$$\\

    %-------------------- 13.
     \item 
	 \begin{enumerate}[(a)]

	     %---------- (a)
	     \item Aplicando el problema 9-3, halle $f'(x)$ para $-1<x<1,$ si $f(x)=\sqrt{1-x^2}$.\\\\
		 Respuesta.-\; Sabiendo que $f(x)=\sqrt{x}\;\Rightarrow \; f'(a)=\dfrac{1}{2\sqrt{a}}$ y la regla de la cadena se tiene,
		 $$f'(x)=\dfrac{1}{2}(1-x^2)^{-1/2}(-2x)=\dfrac{-x}{\sqrt{1-x^2}}.$$\\

	     %---------- (b)
	     \item Demuestre que la tangente a la gráfica de $f$ en $\left(a,\sqrt{1-a^2}\right)$ corta a la gráfica solamente en este punto (y así demuestre que la definición geométrica de tangente coincide con la nuestra).\\\\
		 Demostración.-\; La pendiente de la tangente a $\left(a,\sqrt{1-a^2}\right)$ es,
		 $$f'(a)=\dfrac{-a}{\sqrt{1-a^2}}$$
		 Después, la ecuación de la tangente viene dado por,
		 $$y=mx+c=\dfrac{-a}{\sqrt{1-a^2}}x+c$$
		 Luego ya que la tangente pasa a través de los puntos $\left(a,\sqrt{1-a^2}\right)$, entonces
		 $$\sqrt{1-a^2}=\dfrac{-a}{\sqrt{1-a^2}}a+c\quad \Rightarrow \quad c=\sqrt{1-a^2}+\dfrac{a^2}{\sqrt{1-a^2}}=\dfrac{1}{\sqrt{1-a^2}},$$
		 de donde la tangente se convertirá en,
		 $$y=\dfrac{-a}{\sqrt{1-a^2}}x+\dfrac{1}{\sqrt{1-a^2}}$$
		 Por último sea $f(x)=\sqrt{1-a^2}=y$, entonces
		 $$\begin{array}{rcl}
		     \sqrt{1-x^2}=\dfrac{-a}{\sqrt{1-a^2}}x+\dfrac{1}{\sqrt{1-a^2}}& \Rightarrow & \sqrt{\left(1-a^2\right)\left(1-x^2\right)}=-ax+1\\\\
										   & \Rightarrow & 1-a^2-x^2+a^2x^2=a^2x^2-2ax+1\\\\
										   &=&x^2-2ax+a^2=0\\\\
										   &=&x=a.\\
		 \end{array}$$
		 Por lo tanto la curva y la tangente se cortan en un sólo punto $\left(a,\sqrt{1-a^2}\right)$\\\\

	 \end{enumerate}

    %-------------------- 14.
     \item Demuestre análogamente que las tangentes a una elipse o a una hipérbola cortan a las gráficas correspondientes solamente una vez.\\\\
	 Demostración.-\; La ecuación general de la elipse es,
	 $$\dfrac{x^2}{a^2}+\dfrac{y^2}{b^2}=1 \quad \Rightarrow \quad y=\pm b \sqrt{1-\dfrac{x^2}{a^2}} \quad \Rightarrow \quad y= b \sqrt{1-\dfrac{x^2}{a^2}}.\qquad (1)$$
	 Tomando la derivada de la elipse se tiene,
	 $$y'=\dfrac{-\dfrac{b2x}{a^2}}{2\sqrt{1-\dfrac{x^2}{a^2}}}=\dfrac{-bx}{a^2\sqrt{1-\dfrac{x^2}{a^2}}}$$
	 por lo que la pendiente de la tangente será con respecto de $k$ estará dada por,
	 $$\dfrac{-bk}{a^2\sqrt{1-\dfrac{k^2}{a^2}}}$$
	 Por otro lado, la ecuación de la tangente es,
	 $$y=\dfrac{-bk}{a^2\sqrt{1-\dfrac{k^2}{a^2}}}x+c$$
	 Reemplazando $x$ por $k$ e $y$ con $b\sqrt{1-\dfrac{k^2}{a^2}}$,
	 $$b\sqrt{1-\dfrac{k^2}{a^2}}=\dfrac{-bk}{a^2\sqrt{1-\dfrac{k^2}{a^2}}}k+c,$$
	 de donde 
	 $$c=b\sqrt{1-\dfrac{k^2}{a^2}}+\dfrac{bk^2}{a^2\sqrt{1-\dfrac{k^2}{a^2}}}=\dfrac{a^2 b\left(\sqrt{ 1-\dfrac{k^2}{a^2}}\right)^2 - bk^2}{a^2 \sqrt{ 1- \dfrac{k^2}{a^2}} }=\dfrac{b}{\sqrt{1-\dfrac{k^2}{a^2}}}$$
	 Por lo tanto la ecuación de la tangente viene dada por,
	 $$y=\dfrac{-bk}{a^2\sqrt{1-\dfrac{k^2}{a^2}}}x+\dfrac{b}{\sqrt{1-\dfrac{k^2}{a^2}}}\qquad (2)$$
	 De (1) y (2), se tiene,
	 $$b\sqrt{1-\dfrac{x^2}{a^2}}=\dfrac{-bk}{a^2\sqrt{1-\dfrac{k^2}{a^2}}} x + \dfrac{b}{\sqrt{1-\dfrac{k^2}{a^2}}}\quad \Rightarrow \quad x^2-2kx+k^2=0\; \Rightarrow \; (x-k)^2=0\; \Rightarrow \; x=k.$$\\

    %-------------------- 15.
     \item Si $f+g$ es diferenciable en $a$, ¿son $f$ y $g$ necesariamente diferenciables en $a$? Si $f\cdot g$ y $f$ son diferenciables en $a$, ¿qué condiciones debe cumplir $f$ para que $g$ sea diferenciable en $a$?.\\\\
	 Respuesta.-\; Supongamos que $f$ no es diferenciable en ninguna parte. Sea $g=-f$, entonces  
	 $$(f+g)(x)=f(x)-f(x)=0.$$
	 Esta función cero, es diferenciable en cualquier parte.\\\\
	 Por otro lado, supongamos que $f\cdot g$ y $f$ son diferenciables en $a$. Por el teorema 8 (Si $f$ y $g$ son diferenciables en $a$ y $f(a)\neq 0$, entonces $f/g$ es diferenciable en $a$), la condición que debe cumplir $f$ para que $g$ sea diferenciable en $a$ será, 
	 $$g=\dfrac{f\cdot g}{f}.$$\\

    %-------------------- 16.
     \item
	 \begin{enumerate}[(a)]

	     %---------- (a)
	     \item Demuestre que si $f$ es diferenciable en $a$, entonces $|f|$ también es diferenciable en $a$, si $f(a)\neq 0.$\\\\
		 Demostración.-\; Al ser $f$ derivable en $a$ es continua en $a$. Luego al ser $f(a)\neq 0$, se sigue que $f(x)\neq 0$ para todos los $x$ de un intervalo entorno de $a$. Así pues, $f=|f|$ o $-f=|f|$ en este intervalo, con lo que $|f|'(a)=f'(a)$ o $|f|'(a)=-f'(a)$. Se puede hacer uso también de la regla de la cadena y del hecho que si $f(x)=\sqrt{x}\; \Rightarrow \; f'(a)=\dfrac{1}{2\sqrt{a}}$. Sea $|f|=\sqrt{f^2}$ con lo que 
		 $$|f|'(x)=\dfrac{1}{2\sqrt{f(x)^2}} \cdot 2f(x)f'(x)=f'(x)\cdot \dfrac{f(x)}{|f(x)|}.$$\\

	     %---------- (b)
	     \item Dé un contraejemplo si $f(a)=0$.\\\\
		 Respuesta.-\; Sea $f(x)=x$, entonces $f(0)=0$ y $|f|(x)=|x|.$ De donde sabemos que $|f|$ no es diferenciable en $0$.\\\\ 

	     %---------- (c)
	     \item Demuestre que si $f$ y $g$ son diferenciables en $a$, entonces las funciones $\max(f,g)$ y $\min(f,g)$ son diferenciables en $a$, si $f(a)\neq g(a)$.\\\\
		 Demostración.-\; Sea $r>0$ para cualquier $x\in (a-r,a+r)$ y sea $f(x)>g(x)$, dados por $f(x)=\max{f,g}(x)=f(x)$ y $\min{f,g}(x)=g(x).$ entonces por definición de difernciabilidad existen $f'(a)$ y $g'(a)$ tal que
		 $$f'(a)=\lim_{h\to 0}\dfrac{f(a+h)-f(a)}{h}\quad \mbox{y}\quad g'(a)=\lim_{h\to 0}\dfrac{g(a+h)-g(a)}{h}.$$
		 Luego por definición de límites se tiene que para todo $\epsilon>0$ existe algún $\delta_1>0$ de donde
		 $$\bigg|\dfrac{f(a+h)-f(a)}{h}-f'(a)\bigg|<\epsilon\; \mbox{siempre que}|h|<\delta_1$$
		 y sea $\epsilon>0$ con $\delta_2>0$,
		 $$\bigg|\dfrac{g(a+h)-g(a)}{h}-g'(a)\bigg|<\epsilon\; \mbox{siempre que}|h|<\delta_2$$
		 Pongamos a $\delta'=\dfrac{\min{\delta_1,\delta_2,r}}{2}$, entonces para cualquier $x\in\left(a-\delta',a+\delta'\right)$, tenemos $\max{f,g}(x)=f(x)$ y $\min{f-g}(x)=g(x)$ de la siguiente forma,
		 $$\bigg|\dfrac{\max{f,g}(a+h)-\max{f,g}(a)}{h}-f'(a)\bigg|<\epsilon\; \mbox{siempre que }|h|<\delta'$$
		 y
		 $$\bigg|\dfrac{\min{f,g}(a+h)-\min{f,g}(a)}{h}-g'(a)\bigg|<\epsilon\; \mbox{siempre que }|h|<\delta'.$$
		 Pero ya que $\max{f,g}(a+h)=f(a+h)$ y $\min{f,g}(a+h)=g(a+h)$ para $|h|<\delta'$, entonces $\max{f,g}$ y $\min{f,g}$ son diferenciables en $a$.\\\\  

	     %---------- (d)
	     \item Dé un contraejemplo si $f(a)=g(a)$.\\\\
		 Respuesta.-\; Sean $f(x)=x$ y $g(x)=0$ de donde $f(0)=g(0)$, entonces
		 $$\max{f,g}(x)=\left\{\begin{array}{rcl}
			 x & \mbox{si} & x>0\\
			 0 & \mbox{si} & x\leq 0,\\
		 \end{array}\right. \quad \mbox{y}\quad 
		    \min{f,g}(x)=\left\{\begin{array}{rcl}
			 0 & \mbox{si} & x\geq 0\\
			 x & \mbox{si} & x< 0.\\
		    \end{array}\right.
		 $$
		 Derivando la parte derecha $\max{f,g}(x)=x$ para $0$ se tiene $1$, la derivada $\max{f,g}(x)=0$ para $0$ es $0$. Con respecto a la parte izquierda se tiene las derivada de $\min{f,g}(x)=0$ y $\min{f,g}(x)=x$ cuando tiende a $0$ como $0$ y $1$ respectivamente. Por lo que se demuestra que $\max{f,g}$ y $\min{f,g}$ no son diferenciales en $a=0$.\\\\

	 \end{enumerate}

    %------------------- 17
     \item De un ejemplo de funciones $f$ y $g$ tales que $g$ toma todos los valores, y $f\circ g$ y $g$ son diferenciables, pero $f$ no es diferenciable. ((El problema es trivial si no se exige que $g$ tome todos los valores; en este caso $g$ podría ser una función constante, o una función que sólo tomara valores de un intervalo $(a,b)$, en cuyo caso el comportamiento de $f$ fuera de $(a,b)$ sería irrelevante.).\\\\
	 Respuesta.-\; Sea $f(x)=\sqrt[3]{x^2}$ no diferenciable en $x=0$. Y sea $g(x)=x^3$ diferenciable en $x=0$, entonces
	 $$(f\circ g)(x)=f(g)(x)=f\left(x^3\right)= \sqrt[3]{\left(x^3\right)^2} = \sqrt[3]{x^6}=x^2.$$
	     donde $(f\circ g)(x)$ es diferenciable en $x=0.$\\\\

    %-------------------- 18
     \item 
	 \begin{enumerate}[(a)]

	     %---------- (a)
	     \item Si $g=f^2$ halle una fórmula para $g'$ (que incluya a $f'$).\\\\
		 Respuesta.- Aplicando la regla de la cadena, tenemos
		 $$g'=2f\cdot f'.$$\\

	     %---------- (b)
	     \item Si $g=\left(f'\right)^2$, halle una fórmula para $g'$ (que incluya a $f''$).\\\\
		 Respuesta.-  Aplicando la regla de la cadena, tenemos
		 $$g'=2f'\cdot f''.$$\\

	     %---------- (c)
	     \item Suponga que la función $f>0$ verifica que 
	     $$\left(f'\right)^2=f+\dfrac{1}{f^3}.$$
	     Halle una fórmula para $f''$ en función de $f$. (En este apartado, además de cálculos sencillos, es necesario tener cuidado.)\\\\
		 Respuesta.- Aplicando la regla de la cadena a los dos lados, tenemos
		 $$2f'\cdot f''=f'+\dfrac{-3}{f^4}\dot f' \quad \Rightarrow \quad f''=\dfrac{1}{2}-\dfrac{3}{2f^4}.$$\\
		 
	 \end{enumerate}

    %-------------------- 19
     \item Si $f$ es tres veces diferenciable y $f'(x)\neq 0$, la \textbf{derivada de Schwarz} de $f$ en $x$ se define mediante 
	 $$\mathscr{D}f(x)=\dfrac{f'''(x)}{f'(x)}-\dfrac{3}{2}\left(\dfrac{f''(x)}{f'(x)}\right)^2.$$

	 \begin{enumerate}[(a)]

	     %---------- (a)
	     \item Demuestre que $\mathscr{D}(f\circ g)=\left[\mathscr{D}f\circ g\right]\cdot g^{'^2}\mathscr{D}g.$\\\\
		 Demostración.-\; Primeramente calculemos $(f\cdot g)'$, $(f\cdot g)''$ y $(f\cdot g)'''$.
		 $$\begin{array}{rcl}
		     (f\circ g)'(x)&=&f'\left[g(x)\right]\cdot g'(x).\\\\
		     (f\circ g)''(x)&=&\left\{f'\left[g(x)\right]\cdot g'(x)\right\}'\\
				    &=&f''\left[g(x)\right]g'(x)^2+f'\left[g(x)\right]g''(x)\\\\
		     (f\circ g)'''(x)&=&\left\{f''\left[g(x)\right]g'(x)^2+f'\left[g(x)\right]g''(x)\right\}\\
				     &=&f'''\left[g(x)\right]g'(x)^3 + 2f''\left[g(x)\right]g''(x)g'(x)+f''\left[g(x)\right]g''(x)g'(x)+f'\left[g(x)\right]g'''(x)\\
				     &=&f'''\left[g(x)\right]g'(x)^3+3f''\left[g(x)\right]g''(x)g'(x)+f'\left[g(x)\right]g'''(x).\\
		 \end{array}$$

		 Por último calculamos la derivada de Schwarz para $f\circ g$,
		 $$\begin{array}{rcl}
		     \mathscr{D}(f\circ g)(x)&=&\dfrac{(f\circ g)'''(x)}{(f\circ g)'(x)}-\dfrac{3}{2}\left[\dfrac{(f\circ g)''(x)}{(f\circ g)'(x)}\right]^2\\\\
					     &=&\dfrac{f'''\left[g(x)\right]g'(x)^3}{f'\left[g(x)\right]\cdot g'(x)}+\dfrac{2f''\left[g(x)\right]g''(x)g'(x)}{f'\left[g(x)\right]\cdot g'(x)}+\dfrac{f'\left[g(x)\right]g'''(x)}{f'\left[g(x)\right]\cdot g'(x)}\\\\
					     &-&\dfrac{3}{2}\left\{\dfrac{f''\left[g(x)\right]g'(x)^2}{f'\left[g(x)\right]\cdot g'(x)}+\dfrac{f'\left[g(x)\right]g''(x)}{f'\left[g(x)\right]\cdot g'(x)}\right\}^2\\\\
					     &=&\dfrac{f'''\left[g(x)\right]g'(x)^2}{f'\left[g(x)\right]}+\dfrac{2f''\left[g(x)\right]g''(x)}{f'\left[g(x)\right]}+\dfrac{g'''(x)}{g'(x)}-\dfrac{3}{2}\left[\dfrac{f''\left[g(x)\right]g'(x)}{f'\left[g(x)\right]}+\dfrac{g''(x)}{g'(x)}\right]^2\\\\
					     &=&\dfrac{f'''\left[g(x)\right]g'(x)^2}{f'\left[g(x)\right]}+\dfrac{2f''\left[g(x)\right]g''(x)}{f'\left[g(x)\right]}+\dfrac{g'''(x)}{g'(x)}\\\\
					     &-&\dfrac{2}{3}\left\{\dfrac{f''\left[g(x)\right]g'(x)}{f'\left[g(x)\right]}\right\}^2-3\dfrac{f''\left[g(x)\right]g''(x)}{f'\left[g(x)\right]}-\dfrac{3}{2}\left[\dfrac{g''(x)}{g'(x)}\right]^2\\\\
					     &=&\left[\dfrac{f'''}{f'}\circ g(x)-\dfrac{3}{2}\dfrac{(f''\circ g)(x)}{(f'\circ g)(x)}\right]\cdot g'(x)^2+\dfrac{g'''(x)}{g'(x)}-\dfrac{3}{2}\dfrac{g''(x)}{g'(x)}\\\\
					     &=&\left[\mathscr{D}f\circ g(x)\right]\cdot g'(x)^2+\mathscr{D}g(x).\\\\
		 \end{array}$$
		 \vspace{.5cm}

	     %---------- (b)
	     \item Demuestre que si $f(x)=\dfrac{ax+b}{cx+d},$ con $ad-bc\neq 0$, entonces $\mathscr{D}f=0.$ Por consiguiente, $\mathscr{D}(f\circ g)=\mathscr{D}g.$\\\\
		 Demostración.-\; Usando la regla de la cadena de Leibniz se tiene,
		 $$\begin{array}{rcl}
		     f'(x)&=&\dfrac{a(cx+d)-(ax+b)c}{(cx+d)^2} = \dfrac{ad-bc}{(cx+d)^2}\\\\
		     f''(x)&=&-\dfrac{2c(ad-bc)}{(cx+d)^3}\\\\
		     f'''(x)&=&\dfrac{6c^2(ad-bc)}{(cx+d)^4}\\\\
		 \end{array}$$
		 Luego utilizamos la definición de la derivada de Schwarz, de la siguiente manera:
		 $$\begin{array}{rcl}
		     \mathscr{D}f(x)&=&\dfrac{f'''(x)}{f'(x)}-\dfrac{3}{2}\left[\dfrac{f''(x)}{f'(x)}\right]^2=\dfrac{\dfrac{6c^2(ad-bc)}{(cx+d)^4}}{\dfrac{ad-bc}{(cx+d)^2}}-\dfrac{3}{2}\left[-\dfrac{\dfrac{2c(ad-bc)}{(cx-d)^3}}{\dfrac{ad-bc}{(cx+d)^2}}\right]^2\\\\
				    &=&\dfrac{6c^2}{(cx+d)^2}-\dfrac{3}{2}\left(-\dfrac{2c}{cx+d}\right)^2=\dfrac{6c^2}{(cx+d)^2}-\dfrac{6c^2}{(cx+d)^2}=0.\\\\
		 \end{array}$$
		 Sea $\left[\mathscr{D}f\circ g(x)\right]\cdot g'(x)^2+\mathscr{D}g(x)$, entonces $\mathscr{D}(f\circ g)=\mathscr{D}g.$\\\\

	 \end{enumerate}

    %-------------------- 20.
     \item Suponga que existen $f^{(n)}(a)$ y $g^{(n)}(a)$. Demuestre la \textbf{fórmula de Leibniz}:
	 $$(f\cdot g)^{(n)}(a)=\sum_{k=0}^n {n\choose k}f^{(k)}(a)\cdot g^{(n-k)}(a).$$\\
	 Demostración.-\; Demostraremos por inducción matemática. Sea $n=1$, entonces 
	 $$(f\cdot g)'(a)=\sum_{k=0}^1 {1\choose k}f^{(k)}(a)\cdot g^{(1-k)}(a)= {1\choose 0} f(a)\cdot g'(a) + {1\choose 1}f'(a)\cdot g(a)=f'(a)g(a)+f(a)g'(a).$$
	 El cual se cumple para $n=1$. Luego la hipótesis de inducción estará dada por,
	 $$(f\cdot g)^{(n)}(a)=\sum_{k=0}^n {n\choose k}f^{(k)}(a)\cdot g^{(n-k)}(a),$$
	 que es cierta para $n$ y por no tanto $f^{(n+1)}(a)$ y $g^{(n+1)}(a)$ existen. Así,
	 $$\begin{array}{rcl}
	     (f\cdot g)^{(n+1)}(a)&=&(f\cdot g)^{(n)}(a)(f\cdot g)'(a)\\\\
				  &=&\displaystyle\left[\sum_{k=0}^n {n\choose k}f^{(k)}(a)\cdot g^{(n-k)}(a)\right]\left[f'(a)g(a)+f(a)g'(a)\right]\\\\
				  &=&\displaystyle\sum_{k=0}^n{n \choose k}\left[f^{(k+1)}(a)g^{(n-k)}(a)+f^{(k)}g^{(n+1-k)}(a)\right]\\\\
				  &=&\displaystyle\sum_{k=1}^{n+1}{n\choose k-1}f^{(k)}(a) g^{(n+1-k)}(a)+\sum_{k=0}^n {n\choose k}f^{(k)}(a)g^{(n+1-k)}(a)\\\\
	 \end{array}$$	
	 Ya que $\displaystyle{n+1\choose k}={n\choose k-1}+{n\choose k}$. Entonces,
	 $$(f\cdot g)^{(n+1)}(a)=\sum_{k=0}^{n+1}{n+1\choose k}f^{(k)}(a)g^{[(n+1)-k]}(a).$$\\

    %-------------------- 21.
     \item Demuestre que si $f^{(n)}\left[g(a)\right]$ y $g^{(n)}(a)$ existen ambas, entonces también existe $(f\circ g)^{(n)}(a)$. Con un poco de práctica el lector debería convencerse que no es sensato tratar de encontrar una fórmula para $(f \circ g)^{(n)}(a)$. Para demostrar que $(f \circ g)^{(n)}(a)$ existe, es necesario, por tanto, encontrar una proposición razonable acerca de $(f \circ g)^{(n)}(a)$ que pueda ser demostrada por inducción. Se puede intentar algo como: existe $(f \circ g)^{(n)}(a)$ y es una suma de términos, cada uno de los cuales es un producto de términos de la forma $\ldots$.\\\\ 
	 Demostración.-\; La fórmulas,
	 $$\begin{array}{rcl}
	     (f\circ g)'(x) &=& f'\left[g(x)\right]\cdot g'(x)\\
	     (f\circ g)''(x) &=& f''\left[g(x)\right] \cdot g'(x) + f'\left[g(x)\right]\cdot g''(x)\\
	     (f\circ g)'''(x) &=& f'''\left[g(x)\right]\cdot g'(x)^3 + 3f''\left[g(x)\right]\cdot g'(x)g''(x)+f'\left[g(x)\right]g'''(x),\\
	 \end{array}$$
	 Llevan a la siguiente conjetura: Si $f^{(n)}\left[g(a)\right]$ y $g^{(a)}(a)$ existen, entonces también existe $(f\circ g)^{(n)}(a)$ y es una suma de términos de la forma
	 $$c\cdot \left[g'(a)\right]^{m_1}\cdots \left[g^{(n)}(a)\right]^{m_n}\cdot f^{(k)}\left[g(a)\right],$$
	 para algún número $c$, enteros no negativos $m_1,\ldots , m_n$ y un número natural $k\leq n$. Para probar esta proposición utilizaremos el método de inducción, de donde notamos que es verdadero para $n=1$ con $a=m_1=k=1$. Ahora supóngase que para un cierto $n$, es cierto para todo número $a$ tal que $f^{(n)}\left[g(a)\right]$ y $g^{(n)}(a)$ existan. Supóngase también que $f^{(n+1)}\left[g(a)\right]$ y $g^{(n+1)}(a)$ existen. Entonces $g^{(k)}(x)$ podría existe para todo $k\leq n$ y todo $x$ en algún intervalo alrededor de $a$, y $f^{(k)}(y)$  debe existe para todo $k\leq n$ y todo $y$ en algún intervalo alrededor de $g(a)$. Ya que $g$ es continua en $a$, esto implica que $f^{(k)}\left[g(x)\right]$ exista para todo $x$ en algún intervalo alrededor de $a$. Así la proposición es verdadera para todo $x$, esto es, $(f\}circ g)^{(n)}$ es una suma de términos de la forma:
	 $$c\left[g'(x)\right]^{m_1}\cdots \left[g^{(n)}(x)\right]^{m_n}\cdot f^{(k)}\left[g(a)\right],\quad m_1,\ldots, m_n \geq 0,\quad 1\leq k \leq n.$$
	 Como consecuencia, $(f\circ g)^{(n+1)}(a)$ es una suma de términos de la forma
	 $$c\cdot m_\alpha \left[g'(x)\right]^{m_1}\cdot \ldots \cdot\left[g^{(n)}(x)\right]^{m_\alpha - 1}\cdot \ldots \cdot\left[g^{(n)}(a)\right]^{m_n}\cdot f^{(k)}\left[g(a)\right]\qquad m_\alpha>0$$
	 o de la forma
	 $$c\left[g'(x)\right]^{m_1+1} \cdot \ldots \cdot\left[g^{(n)}(x)\right]^{m_n}\cdot f^{(k+1)}\left[g(a)\right].$$
	 Para un número $c$.\\\\

    %-------------------- 22.
     \item 
	 \begin{enumerate}[(a)]

	     %---------- (a)
	     \item Si $f(x)=a_n x^n + a_{n-1} x^{n-1}+\ldots + a_0,$ halle una función $g$ tal que $g'=f$. Encuentre otra.\\\\
		 Respuesta.-\; Sea 
		 $$g(x)=\dfrac{a_n}{n+1}x^{n+1}+\dfrac{a_{n-1}}{n}x^n+\ldots + \dfrac{a_1}{2}x^2+a_0 x.$$
		 Derivando $g$ tenemos,
		 $$\begin{array}{rcl}
		     g'(x)&=&\dfrac{a_n}{n+1}(n+1)x^n + \dfrac{a_{n-1}}{n}\cdot n x^{n-1}+\ldots + \dfrac{a_1}{2}2x+a_0\\\\
			  &=& a_nx^n+a_{n-1}x^{n-1}+\ldots + a_0\\\\
			  &=&f(x).\\\\
		\end{array}$$
		Otro $g_1$ tal que $g_1'=f$ sería,
		$$g_1(x)=g(x)+c.$$\\


	     %---------- (b)
	     \item Si $$f(x)=\dfrac{b_2}{x^2}+\dfrac{b_3}{x^3}+\ldots + \dfrac{b_m}{x^m}$$
	     halle una función $g$ que verifique $g'=f.$\\\\
		 Respuesta.-\; Sea
		 $$g(x)=-\left[\dfrac{b_2}{x}+\dfrac{b_3}{2x^2}+\ldots + \dfrac{b_m}{(m-1)x^{m-1}}\right].$$
		 Derivando $g$ tenemos,
		 $$\begin{array}{rcl}
		     g'(x)&=&-\left[-\dfrac{b_2}{x^2}-\dfrac{2b_3 x}{2x^4}+\ldots + \dfrac{b_m(m-1)x^{m-2}}{(m-1)x^{2(m-1)}}\right]\\\\
			  &=&\dfrac{b_2}{x^2}+\dfrac{b_3}{x^3}+\ldots + \dfrac{b_m}{x^m}\\\\
			  &=&f(x).\\\\
		 \end{array}$$
		 \vspace{.5cm}


	     %---------- (c)
	     \item ¿Existe una función 
	     $$f(x)=a_nx^n + \ldots + a_0 + \dfrac{b_1}{x}+\ldots + \dfrac{b_m}{x^m}$$
	     tal que $f'(x)=1/x$?.\\\\
		 Respuesta.-\; No, ya que la derivada de $f$ es
		 $$f'(x)=na_n x^{n-1}+\ldots + a_1 - \dfrac{b_1}{x^2}-\dfrac{2b_2}{x_3}-\ldots - \dfrac{mb_m}{x^{m+1}}.$$\\

	 \end{enumerate}

    %-------------------- 23.
     \item Demuestre que existe una función polinómica $f$ de grado $n$ tal que\\
	 \begin{enumerate}[(a)]

	     %---------- (a)
	     \item $f'(x)=0$ para exactamente $n-1$ números $x$.\\\\
		 Demostración.-\; Ya que para todo $n$ existe una función polinómica de grado $n$. Es decir, existe $g$ una función polinómica de grado $n-1$ con $n-1$ raíces. Entonces para $f(x)=a_n x^n + a_{n-1} x^{n-1}+\ldots + a_0,$ de grado $n$ existe una función $g$ tal que $g=f'$. Esto por el problema 7(b) capítulo 3 Spivak y por el problema 22 (a) capítulo 10, Spivak.\\\\


	     %---------- (b)
	     \item $f'(x)=0$ para ningún $x$, si $n$ es impar.\\\\
		 Demostración.-\; Sea $n$ impar que implica $n-1$ es par. Si $g$ es una función polinómica de grado $n-1$ sin raíces (Capítulo 3, problema 7, Spivak). Entonces, existe un polinomio $f$ de grado $n$ tal que $f'=g$ (Capítulo 10, problema 22(a), Spivak). Por lo tanto $g$ no tiene raíces, así $f'(0)=0$ no tiene raíces.\\\\ 

	    %---------- (c)
	     \item $f'(x)=0$ para exactamente un $x$, si $n$ es par.\\\\
		 Demostración.-\; Sea $n$ par que implica $n-1$ par. Por el capítulo 3, problema 7, Spivak, existe un polinomio $g$ de grado $n-1$ con exactamente una raíz. Luego por el capítulo 10, problema 22(a), spivak. Existe una función polinómica $f$ de grado $n$ tal que $f'=g$. Por lo tanto $g$ tiene una sola raíz, así $f'(x)=0$ tiene exactamente una raíz.\\\\

	    %---------- (d)
	     \item $f'(x)=0$ para exactamente $k$ números $x$, si $n-k$ es impar.\\\\
		 Demostración.-\; Sea $n-k$ impar que implica $n-k-1$ es par. Por el capítulo 3, problema 7, Spivak, existe un polinomio $g$ de grado $n-k-1$ con exactamente $k$ raíces. Luego por el capítulo 10, problema 22(a), spivak, existe una función polinómica $f$ de grado $n$ tal que $f'=g$. Por lo tanto $g$ tiene $k$ raíces, así $f'(0)=0$ para exactamente $k$ números $x$.\\\\

	 \end{enumerate}

    %-------------------- 24.
     \item 
	 \begin{enumerate}[(a)]

	     %---------- (a)
	     \item El número $a$ se denomina una raíz doble de la función polinómica $f$ si $f(x)=(x-a)^2g(x)$ para alguna función polinómica $g$. Demuestre que $a$ es una raíz doble de $f$ si y sólo si $a$ es una raíz de $f$ y de $f'$.\\\\
		 Demostración.-\; 

	    %---------- (b)
	     \item ¿Cuándo tiene $f(x)=ax^2+bx+c\; (a\neq 0)$ una raíz doble? ¿Cuál es la interpretación geométrica de esta condición?.\\\\
		 Respuesta.-\;

	 \end{enumerate}
    

\end{enumerate}
