\chapter{Diferenciación}

% -------------------- teorema 1
\begin{teo}
    Si $f$ es una función constante, $f(x)=c,$ entonces
    \begin{center}
	$f'(a)=0$ para todo número $a$.
    \end{center}
    \vspace{.5cm}
	Demostración.-\; $$f'(a)=\lim\limits_{h\to 0}\dfrac{f(a+h)-f(a)}{h}=\lim\limits_{h\to 0}\dfrac{(c+h)-c}{h}=0.$$\\\\
\end{teo}

% -------------------- teorema 2
\begin{teo}
    Si $f$ es la función identidad, $f(x)=x,$ entonces
    \begin{center}
	$f'(a)=1$ para todo número $a$.
    \end{center}
    \vspace{.5cm}
	Demostración.-\;
	$$f'(a)=\lim_{h\to 0}\dfrac{f(a+h)-f(a)}{h}=\lim\limits_{h\to 0}\dfrac{a+h-a}{h}=\lim_{h\to 0}\dfrac{h}{h}=1.$$\\
\end{teo}

% -------------------- teorema 3
\begin{teo}
    Si $f$ y $g$ son diferencibles en $a$, entonces $f+g$ es también diferenciable en $a$, y
    $$(f+g)'(a)=f'(a)+g'(a).$$\\
	Demostración.-\; 
	$$\begin{array}{rcl}
	    (f+g)'(a) &=& \lim\limits_{h\to 0} \dfrac{(f+g)(a+h)-(f+g)(a)}{h}\\\\
		      &=& \lim\limits_{h\to 0} \dfrac{f(a+h)+g(a+h)-\left[f(a)+g(a)\right]}{h}\\\\
		      &=& \lim\limits_{h\to 0} \left[\dfrac{f(a+h)-f(a)}{h}+\dfrac{g(a+h)-g(a)}{h}\right]\\\\
		      &=& \lim\limits_{h\to 0} \dfrac{f(a+h)-f(a)}{h} \lim\limits_{h\to 0}\dfrac{g(a+h)-g(a)}{h}\\\\
		      &=& f'(a)+g'(a).\\\\
	\end{array}$$
\end{teo}

% -------------------- teorema 4
\begin{teo}
    Si $f$ y $g$ son diferenciables en $a$, entonces $f\cdot g$ es también diferenciable en $a$, y
    $$(f\cdot g)'(a)=f'(a)\cdot g(a)+f(a)\cdot g'(a).$$\\
	Demostración.-\; 
	$$\begin{array}{rcl}
	    (f\cdot g)'(a) &=& \lim\limits_{h\to 0} \dfrac{(f\cdot g)(a+h)-(f\cdot g)(a)}{h}\\\\
			   &=& \lim\limits_{h\to 0} \dfrac{f(a+h)g(a+h)-f(a)g(a)}{h}\\\\
			   &=& \lim\limits_{h\to 0} \left[\dfrac{f(a+h)\left(g(a+h)-g(a)\right)}{h}+\dfrac{\left(f(a+h)-f(a)\right)g(a)}{h}\right]\\\\
			   &=& \lim\limits_{h\to 0} f(a+h)\cdot \lim\limits_{h\to 0} \dfrac{g(a+h)-g(a)}{h}+\lim\limits_{h\to 0} \dfrac{f(a+h)-f(a)}{h} \cdot \lim\limits_{h\to 0} g(a)\\\\
			   &=&f(a)\cdot g'(a)+f'(a)\cdot g(a).\\\\
	\end{array}$$
	Observemos que hemos utilizado el hecho de que $\lim\limits_{h\to 0}f(a+h)-f(a)=0\; \Rightarrow \; \lim\limits_{h\to 0}f(a+h)=f(a).$\\\\
\end{teo}

% -------------------- teorema 5
\begin{teo}
    Si $f(x)=cf(x)$ y $f$ es diferenciable en $a$, entonces $g$ es diferenciable en $a$, y
    $$g'(a)=c\cdot f'(a).$$\\
	Demostración.-\; Si $h(x)=c$, de manera que $g=h\cdot f$, entonces,
	$$\begin{array}{rcl}
	    g'(a)&=&(h\cdot f)'(a)\\
		 &=&h(a)\cdot f'(a)+h'(a)\cdot f(a)\\
		 &=&c\cdot f'(a)+0\cdot f(a)\\
		 &=&c\cdot f'(a).\\\\
	\end{array}$$
	En particular, $(-f)'(a)=-f'(a)$, por tanto $(f-g)'(a)=\left(f+[-g]\right)'(a)=f'(a)-g'(a).$\\\\
\end{teo}

% -------------------- teorema 6
\begin{teo}
    Si $f(x)=x^n$ para algún número natural $n$, entonces
    \begin{center}
	$f'(x)=nx^{n-1}$ para todo número $a$.
    \end{center}
    \vspace{.5cm}
    	Demostración.-\; La demostración la haremos por inducción sobre $n$. Para $n=1$ se aplica simplemente el teorema 2. Supongamos ahora que el teorema es cierto para $n$, de manera que si $f(x)=x^n$, entonces 
	$$f'(a)=na^{n-1}\; \mbox{para todo}\; a.$$
	Sea $g(x)=x^{n+1}$. Si $I(x)=x$, la ecuación $x^{n+1}=x^n\cdot x$ se puede escribir como
	\begin{center}
	    $g(x)=f(x)\cdot I(x)$ para todo $x$.
	\end{center}
	así, $g=f\cdot I$. A partir del teorema 4 deducimos que 
	$$\begin{array}{rcl}
	    g'(a)&=&\left(f\cdot I\right)'(a)\\
		 &=&f'(a)\cdot I(a)+f(a)\cdot I'(a)\\
		 &=&na^{n-1}\cdot a + a^n \cdot 1\\
		 &=&na^n+a^n\\
		 &=&(n+1)a^n,\; \mbox{para todo}\; a.\\\\
	\end{array}$$
	Este es precisamente el caso $n+1$ que queríamos demostrar.\\
	Si $f(x)=x^{-n}=\dfrac{1}{x^n}$ para algún número natural $n$, entonces
	$$f'(x)=\dfrac{-nx^{n-1}}{x^{2n}}=(-n)x^{-n-1};$$
	así es válido tanto para enteros positivos como negativos. Si interpretamos $f(x)=x^0$ como $f(x)=1$ y $0\cdot x^{-1}$ como $f'(x)=0$, entonces se verifica también para $n=0$.\\\\
\end{teo}

% -------------------- teorema 7
\begin{teo}
    Si $g$ es diferenciable en $a$ y $g(a)\neq 0,$ entonces $1/g$ es diferenciable en $a$, y 
    $$\left(\dfrac{1}{g}\right)'(a)=\dfrac{-g'(a)}{\left[g(a)\right]^2}.$$\\
	Demostración.-\; Incluso antes de escribir
	$$\dfrac{\left(\dfrac{1}{g}\right)(a+h)-\left(\dfrac{1}{g}\right)(a)}{h}$$
	debemos asegurarnos que esta expresión tiene sentido; es necesario comprobar que $(1/g)(a+h)$ está definido para valores suficientemente pequeños de $h$. Para ello son necesarias solamente dos observaciones. Como $g$ es, por hipótesis, diferenciable en $a$, se deduce del teorema 9-1 que $g$ es continua en $a$. Como $g(a)\neq 0$, deducimos también, a partir del teorema 6-3, que existe un $\delta>0$ tal que $g(a+h)\neq 0$ para $|h|<\delta$. Por tanto, $(1/g)=(a+h)$ tiene sentido para valores de $h$ suficientemente pequeños, y así podemos escribir
	$$\begin{array}{rcl}
	    \lim\limits_{h\to 0} \dfrac{\left(\dfrac{1}{g}\right)(a+h)-\left(\dfrac{1}{g}\right)(a)}{h}&=& \lim\limits_{h\to 0} \dfrac{\dfrac{1}{g(a+h)}-\dfrac{1}{g(a)}}{h}\\\\			
				 &=& \lim\limits_{h\to 0} \dfrac{g(a)-g(a+h)}{h\left[g(a)\cdot g(a+h)\right]}\\\\
				 &=& \lim\limits_{h\to 0} \dfrac{-\left[g(a+h)-g(a)\right]}{h}\cdot \dfrac{1}{g(a)g(a+h)}\\\\
				 &=& \lim\limits_{h\to 0} \dfrac{-\left[g(a+h)-g(a)\right]}{h}\cdot \lim\limits_{h\to 0}\dfrac{1}{g(a)\cdot g(a+h)}\\\\
				 &=&-g'(a)\cdot \dfrac{1}{\left[g(a)\right]^2}.\\\\
	\end{array}$$

\end{teo}

% -------------------- teorema 8
\begin{teo}
    Si $f$ y $g$ son diferenciables en $a$ y $g(a)\neq 0$, entonces $f/g$ es diferenciable en $a$, y
    $$\left(\dfrac{f}{g}\right)'(a)=\dfrac{g(a)\cdot f'(a)-f(a)\cdot g'(a)}{\left[g(a)\right]^2}$$
	Demostración.-\; Como $f/g=f\cdot (1/g)$ obtenemos
	$$\begin{array}{rcl}
	    \left(\dfrac{f}{g}\right)'(a) &=& \left(f\cdot \dfrac{1}{g}\right)'(a)\\\\
					  &=& f'(a)\cdot \left(\dfrac{1}{g}\right)(a)+f(a)\cdot \left(\dfrac{1}{g}\right)'(a)\\\\
					  &=& \dfrac{f'(a)}{g(a)}+\dfrac{f'(a)\cdot g(a)-f(a)\cdot g'(a)}{\left[g(a)\right]^2}\\\\
					  &=& \dfrac{f'(a)\cdot g(a)-f(a)\cdot g'(a)}{\left[g(a)\right]^2}.\\\\
	\end{array}$$
\end{teo}
\vspace{.7cm}

% -------------------- teorema 9
\begin{teo}[Regla de la cadena]
    Si $g$ es diferenciable en $a$ y $f$ es diferenciable en $g(a)$, entonces $f\circ g$ es diferenciable en $a$ y 
    $$\left(f\circ g\right)'(a)=f'\left[g(a)\right]\cdot g'(a).$$\\
	Demostración.-\; Definamos una función $\phi$ de la manera siguiente:
	$$\phi(h)=\left\{\begin{array}{ll}
		\dfrac{f\left[g(a+h)\right]-f\left[g(a)\right]}{g(a+h)-g(a)}, & \mbox{si}\; g(a+h)-g(a)\neq 0\\\\
		f'\left[g(a)\right], & \mbox{si}\; g(a+h)-g(a)=0.\\
	\end{array}\right.$$
	Se intuye fácilmente que $\phi$ es continua en $0$: cuando $h$ es pequeño, $g(a+h)-g(a)$ también es pequeño, de manera que si $g(a+h)-g(a)$ no es $0$, entonces $\phi(h)$ se aproximará a $f'\left[g(a)\right]$; y si es $0$ entonces $\phi(h)$ es igual a $f'\left[g(a)\right]$, lo que es mejor todavía. Ya que la continuidad de $\phi$ es el punto crucial de toda la demostración, vamos a desarrollar rigurosamente este argumento intuitivo.\\
	Sabemos que $f$ es diferenciable en $g(a)$. Esto significa que 
	$$\lim_{k\to 0}\dfrac{f(g(a)+k)-f(g(a))}{k}=f'(g(a)).$$
	Así, si $\epsilon>0$ existe algún número $\delta'>0$ tal que, para todo $k$,
	\begin{center}
	    si $0<|k|<\delta'$, entonces $\bigg|\dfrac{f(g(a)+k)-f(g(a))}{k}-f'(g(a))\bigg|<\epsilon.$
	\end{center}
	Pero $g$ es diferenciable en $a$ y  por lo tanto continua en $a$, de manera que existe un $\delta>0$ tal que para todo $h$,
	\begin{center}
	    si $|h|<\delta$, entonces $|g(a+h)-g(a)|<\delta'$
	\end{center}
	Consideremos ahora cualquier $h$ con $|h|<\delta$. Si $k=g(a+h)-g(a)\neq 0,$ entonces
	$$\phi(h)=\dfrac{f(g(a+h))-f(g(a))}{g(a+h)-g(a)}=\dfrac{f(g(a)+k)-f(g(a))}{k};$$
	se deduce que $|k|<\delta'$, y por tanto deducimos que 
	$$|\phi(h)-f'(h(a))|<\epsilon.$$
	Por otro lado, si $g(a+h)-g(a)=0$, entonces $\phi(g)=f'(g(a))$, de manera que se verifica ciertamente que 
	$$|\phi(h)-f'(g(a))|<\epsilon.$$
	Por tanto hemos demostrado que 
	$$\lim_{h\to 0}\phi(h)=f'(g(a)),$$
	o sea que $\phi$ es continua en $0$. El resto de la demostración es fácil. Si $h\neq 0$, entonces tenemos 
	$$\dfrac{f(g(a+h))-f(g(a))}{h}=\phi(h)\cdot \dfrac{g(a+h)-g(a)}{h}$$
	incluso aunque $g(a+h)-g(a)=0$ (ya que en este caso ambos miembros de la igualdad son iguales a $0$). Por tanto
	$$(f\circ g)'(a)=\lim_{h\to 0}\dfrac{f(g(a+h))-f(g(a))}{h}\lim_{h\to 0}\phi(h)\cdot \lim_{h\to 0}\dfrac{g(a+h)-g(a)}{h}=f'(g(a))\cdot g'(a).$$
\end{teo}

\section{Problemas}
\begin{enumerate}[\bfseries 1.]

    %-------------------- 1.
    \item Como ejercicio de precalentamiento, halle $f'(x)$ para cada una de las siguientes $f$. (No se preocupe por el dominio de $f$ o de $f'$; obtenga tan sólo una fórmula para $f'(x)$ que dé la respuesta correcta cuando tenga sentido.)\\

	\begin{enumerate}[(i)]

	    %-------------------- (i) 
	    \item $f(x)=\sen(x+x^2).$\\\\ 
		Respuesta.-\; $f'(x)=\cos(x+x^2)\cdot (1+2x).$\\\\

	    %-------------------- (ii)
	    \item $f(x)=\sen x + \sen x^2$.\\\\
		Respuesta.-\; $f'(x)=\cos x + \cos(x^2)\cdot 2x.$\\\\

	    %-------------------- (iii)
	    \item $f(x)=\sen(\cos x)$.\\\\
		Respuesta.-\; $f'(x)=\cos(\cos x)\cdot (-\sen x)=-\sen x \cos(\cos x).$\\\\

	    %-------------------- (iv)
	    \item $f(x)=\sen(\sen x)$.\\\\
		Respuesta.-\; $f'(x)=\cos(\sen x)\cdot \cos x = \cos x \cos(\sen x).$\\\\

	    %-------------------- (v)
	    \item $f(x)=\sen\left(\dfrac{\cos x}{x}\right)$.\\\\
		Respuesta.-\; $f'(x)=\cos\left(\dfrac{\cos x}{x}\right)\cdot \dfrac{-x\sen x - \cos x}{x^2}$.\\\\

	    %-------------------- (vi)
	    \item $f(x)=\dfrac{\sen(\cos x)}{x}$.\\\\
		Respuesta.-\; $f'(x)=\dfrac{\cos(\cos x)\cdot (-\sen x)}{x^2}=-\dfrac{\sen x \cos(\cos x)}{x^2}$.\\\\

	    %-------------------- (vii)
	    \item $f(x)=\sen(x+\sen x)$.\\\\
		Respuesta.-\; $f'(x)=\cos(x+\sen x)\cdot (1+\cos x)$.\\\\

	    %-------------------- (viii)
	    \item $f(x)=\sen\left[\cos(\sen x)\right]$.\\\\
		Respuesta.-\; $f'(x)=\cos\left[\cos(\sen x)\right]\left[-\sen(\sen x)\cos x\right]$.\\\\

	\end{enumerate}

    %-------------------- 2.
    \item Halle $f(x)$ para cada una de las siguientes funciones $f.$ (El autor tardó 20 minutos en calcular las derivadas para la sección de soluciones, y al lector no debería costarle mucho más tiempo calcularlas. Aunque la rapidez en los cálculos no es un objetivo de las matemáticas, si se desea tratar con aplomo las aplicaciones teóricas de la Regla de la Cadena, estas aplicaciones concretas deberían ser un juego de niños; a los matemáticos les gusta hacer ver que ni siquiera saben sumar, pero la mayoría pueden hacerlo cuando lo necesitan.)\\

	\begin{enumerate}[(i)]

	    %-------------------- (i)
	    \item $f(x)=\sen \left[(x+1)^2(x+2)\right]$.\\\\
		Respuesta.-\; 
		$$\begin{array}{rcl}
		f'(x)&=&\cos\left[(x+1)^2(x+2)\right]\cdot \left[2(x+1)(x+2)+(x+1)^2\right]\\
		     &=&(x+1)(3x+5)\cos\left[(x+1)^2(x+2)\right].\\
		\end{array}$$
		\vspace{0.7cm}

	    %-------------------- (ii)
	    \item $f(x)=\sen^3\left(x^2+\sen x\right)$.\\\\
		Respuesta.-\; $f'(x)=3\sen^2\left(x^2+\sen x\right)\cdot \cos\left(x^2+\sen x\right)\cdot (2x+\cos x)$.\\\\


	    %-------------------- (iii)
	    \item $f(x)=\sen^2\left[(x+\sen x)^2\right]$.\\\\
		Respuesta.-\; 
		$$\begin{array}{rcl}
		    f'(x)&=&2\sen \left[(x+\sen x)^2 \right] \cos \left[(x+\sen x)^2\right] \cdot 2(x+\sen x)(1+\cos x)\\
			 &=&4(1+\cos x)(x+\sen x)\sen\left[(x+\sen x)^2\right]\cos \left[(x+\sen x)^2\right]\\
		\end{array}$$
		\vspace{0.7cm}

	    %-------------------- (iv)
	    \item $f(x)=\sen\left(\dfrac{x^3}{\cos x^3}\right)$.\\\\
		Respuesta.-\; 
		$$\begin{array}{rcl}
		    f'(x)&=&\cos\left(\dfrac{x^3}{\cos x^3}\right)\cdot \dfrac{3x^2\cos\left(x^3\right)-x^3\left[-\sen\left(x^3\right)\right]3x^2}{\cos^2\left(x^3\right)}\\\\
			 &=&\cos\left(\dfrac{x^3}{\cos x^3}\right)\cdot \dfrac{3x^2\cos\left(x^3\right)+x^3\sen\left(x^3\right)3x^2}{\cos^2\left(x^3\right)}
		\end{array}$$
		\vspace{0.7cm}


	    %-------------------- (v)
	    \item $f(x)=\sen(x\sen x)+\sen\left(\sen x^2\right)$.\\\\
		Respuesta.-\; Aplicando las reglas de derivada se tiene,
		$$f'(x)=\cos(x\sen x)\sen x + x\cos x + \cos\left(\sen x^2\right)\cos\left(x^2\right)2x.$$\\

	    %-------------------- (vi)
	    \item $f(x)=f(x)=(\cos x)^{31^2}$.\\\\
		Respuesta.-\; Aplicando las reglas de derivada se tiene,
		$$f'(x)=-\left(31^2-1\right)(\cos x)^{31^2-1}\sen x.$$\\

	    %-------------------- (vii)
	    \item $f(x)=\sen^2 x \sen x^2\sen^2 x^2$.\\\\
		Respuesta.-\; Aplicando las reglas de derivada se tiene,
		$$\begin{array}{rcl}
		    f'(x)&=&2\sen x \cos x \sen\left(x^2\right)\sen^2\left(x^2\right)+\sen^2 x \left[\cos\left(x^2\right)2x\sen^2\left(x^2\right)\right.\\
			 &+&\left.\sen\left(x^2\right)2\sen\left(x^2\right)\cos\left(x^2\right)2x\right]\\
			 &=&2\sen x \cos x \sen\left(x^2\right)\sen^2\left(x^2\right)+2x\sen^2 x\cos\left(x^2\right)\sen^2\left(x^2\right)\\
			 &+&4x\sen^2 x \sen^2\left(x^2\right)\cos\left(x^2\right).\\
		\end{array}$$
		\vspace{0.7cm}

	    %-------------------- (viii)
	    \item $f(x)=\sen^3\left[\sen^2(\sen x)\right]$.\\\\
		Respuesta.-\; Aplicando las reglas de derivada se tiene,
		    $$f'(x)=3\sen^2\left[\sen^2(\sen x)\right]\cos\left[\sen^2(\sen x)\right]2\sen(\sen x)\cos(\sen x)\cos x.$$\\

	    %-------------------- (ix)
	    \item $f(x)=\left(x+\sen^5 x\right)^6$.\\\\
		Respuesta.-\; Aplicando las reglas de derivada se tiene,
		$$f'(x)=6\left(x+\sen^5 x\right)\left[1+5\sen^4 x \cos x\right].$$\\

	    %-------------------- (x)
	    \item $f(x)=\sen\left[\sen(\sen(\sen(\sen x)))\right]$.\\\\
		Respuesta.-\; Aplicando las reglas de derivada se tiene,
		$$f'(x)=\cos\left(\sen(\sen (\sen x))\right)\cdot \cos (\sen (\sen (\sen x)))\cdot \cos (\sen (\sen x))\cdot \cos (\sen x)\cdot \cos x.$$\\

	    %-------------------- (xi)
	    \item $f(x)=\sen\left[(\sen^7 x^7 + 1)^7\right]$.\\\\
		Respuesta.-\; Aplicando las reglas de derivada se tiene,
		$$f'(x)=\cos\left[\left(\sen^7 x^7 + 1\right)^7\right]\cdot 7\left(\sen^7 x^7 + 1\right)^6\cdot 7\sen^6 x^7 \cdot \cos x^7 \cdot 7x^6.$$\\

	    %-------------------- (xii)
	    \item $f(x)=\left\{\left[\left(x^2+x\right)^3+x\right]^4+x\right\}^5$.\\\\
		Respuesta.-\; Aplicando las reglas de derivada se tiene,
		$$f'(x)=5\left[\left(x^2+x\right)^3+x\right]^4\cdot \left\{1+4\left[\left(x^2+x\right)^3+x\right]^3\left[1+3\left(x^2+x\right)^2\left(1+2x\right)\right]\right\}.$$\\

	    %-------------------- (xiii)
	    \item $f(x)=\sen\left[x^2+\sen\left(x^2+\sen^2 x\right)\right]$.\\\\
		Respuesta.-\; Aplicando las reglas de derivada se tiene,
		$$f'(x)=\cos\left[x^2+\sen\left(x^2+\sen x^2\right)\right]\cdot\left[2x+\cos\left(x^2+\sen x^2\right)\cdot \left(2x+2x\cos x^2\right)\right].$$\\

	    %-------------------- (xiv)
	    \item 

	\end{enumerate}

\end{enumerate}
