\chapter{Funciones Continuas}

% ---------- definición 
\begin{tcolorbox}[colframe = white]
    \begin{def.} La función $f$ es continua  en $a$ si $$\lim\limits_{x\to a} f(x) = f(a)$$
	Para todo $\epsilon > 0$ existe un $\delta > 0$ tal que, para todo $x$, si $0<|x-a|<\delta$.\\
	Pero en este caso, en que el límite es $f(a)$, la frase $$0<|x-a|<\delta$$
	puede cambiarse por la condición más sencilla $$|x-a|<\delta$$
	puesto que si $x=a$ se cumple ciertamente que $|f(x)-f(a)|<\epsilon$.
    \end{def.}
\end{tcolorbox}

%---------- teorema 1
\begin{teo} Si $f$ y $g$ son continuas en $a$, entonces 
    \begin{center}
	\begin{tabular}{rl}
	    $(1)$ & $f+g$ es continua en $a$.\\
	    $(2)$ & $f\cdot g$ es continua en $a$.\\
    Además, si $g(a)\neq 0$, entonces 
	    $(3)$&$1/g$ es continua en $a$\\
	\end{tabular}
    \end{center}	
    Demostración.-\; Puesto que $f$ y $g$ son continuas en $a$,
    $$\lim_{x\to a}f(x) = f(a) \qquad y \qquad \lim_{x\to a} g(x) = g(a).$$
    Por el teorema 2(1) del capítulo 5 esto implica que 
    $$\lim_{x\to a} (f+g)(x) = f(a)+g(a) = (f+g)(a),$$
    lo cual es precisamente la afirmación de que $f+g$ es continua en $a$.\\
    Para $f\cdot g$ se tiene que $$\lim_{x\to a}(f\cdot g)(x) = f(a)\cdot fg(a) = (f\cdot g)(a)$$
    Por último para $1/g$ tenemos que $$\lim_{x\to a} 1/g = 1/g(a) , \qquad para \; g(a)\neq 0$$\\
\end{teo}

%---------- teorema 2
\begin{teo} Si $g$ es continua en $a$, y $f$ es continua en $g(a)$, entonces $f\circ g$ es continua en $a$.\\\\
    Demostración.-\; Sea $\epsilon > 0$. Queremos hallar un $\delta > 0$ tal que para todo $x$, 

    \begin{center}
	Si $|x-a|<\delta$ entonces $|(f\circ g)(x)-(f\circ g)(a)|<\epsilon$, es decir, $|f(g(x))-f(g(a))|<\epsilon$
    \end{center}

	Tendremos que aplicar primero la continuidad de $f$ para estimar cómo de cerca tiene que estar $g(x)$ de $g(a)$ para que se cumpla esta desigualdad. Puesto que $f$ es continua en $g(a)$, existe un $\delta^{'} > 0$ tal que para todo $y$,

    \begin{center}
	Si $|y-g(a)|<\delta^{'}$, entonces $|f(y)-f(g(a))|<\epsilon. \qquad (1)$ 
    \end{center}

    En particular, esto significa que

    \begin{center}
	Si $|g(x)-g(a)|<\delta^{'}$, entonces $|f(g(x))-f(g(a))|<\epsilon. \qquad (2)$
    \end{center}
    
    Aplicamos ahora la continuidad de $g$ para estimar cómo de cerca tiene que estar $x$ de $a$ para que se cumpla la desigualdad $|g(x)-g(a)|<\delta^{'}$. El número $\delta^{'}$ es un número positivo como cualquier otro número positivo; podemos, por lo tanto, tomar $\delta^{'}$ como el $epsilon$ de la definición de continuidad de $g$ en $a$. Deducimos que existe un $\delta > 0$ tal que, para todo $x$,

    \begin{center}
	Si $|x-a|<\delta,$ entonces $|g(x)-g(a)|<\delta^{'}, \qquad (3)$
     \end{center}

     combinando (2) y (3) vemos que para todo $x$,

    \begin{center}
	Si $|x-a|<\delta,$ entonces $|f(g(x))-f(g(a))|<\epsilon.$
     \end{center}
     \vspace{1cm}

\end{teo}

% definición
\begin{tcolorbox}[colframe = white]
    \begin{def.} Si $f$ es continua en $x$ para todo $x$ en $(a,b)$, entonces se dice que $f$ es continua en $(a,b)$ si 
	\begin{center}
	    $f$ es continua en $x$ para todo $x$ de $(a,b), \qquad (1)$\\
	    \vspace{.5cm}
	    $\lim\limits_{x\to a^+} f(x) = f(a)$ y $\lim\limits_{x\to b^-}f(x) = f(b). \qquad (2)$
	\end{center}
    \end{def.}
\end{tcolorbox}

%---------- teorema 3
\begin{teo} Supóngase que $f$ es continua en $a$, y $f(a)>0$. Entonces existe un número $\delta>0$ tal que $f(x)>0$ para todo $x$ que satisface $|x-a|<\delta$. Análogamente, si $f(a)>0$, entonces existe un número $\delta>0$ tal que $f(x)<0$ para todo $x$ que satisface $|x-a|<\delta.$\\\\
    Demostración.-\; Considérese el caso $f(a)>0$ puesto que $f$ que es continua en $a$, si $\epsilon>0$ existe un $\delta>0$ tal que, para todo $x$,
    \begin{center}
	Si $|x-a|<\delta$, entonces $|f(x)-f(a)|<\epsilon.$
    \end{center}
    Puesto que $f(a)>0$ podemos tomar a $f(a)$ como el $epsilon$. Así, pues, existe $\delta>0$ tal que para todo $x$,
    \begin{center}
	Si $|x-a|<\delta$, entonces $|f(x)-f(a)|<f(a)$
    \end{center}
    Y esta última igualdad implica $f(x)>0$.\\
    Puede darse una demostración análoga en el caso $f(a)<0$; tómese $\epsilon=-f(a)$. O también se puede aplicar el primer caso a la función $-f$.\\\\

\end{teo}

\section{Problemas}
\begin{enumerate}[\Large\bfseries 1.]

%--------------------1.
\item ¿para cuáles de las siguientes funciones $f$ existe una función $F$ de dominio $R$ tal que $F(x)=f(x)$ para todo $x$ del dominio de $f$?

    \begin{enumerate}[\Large\bfseries (i)]

	%---------- (i)
	\item $f(x) = \dfrac{x^2 - 4}{x - 2}$\\\\
	    Respuesta.-\; Sabiendo que el límite cuando $x$ tiende a $2$ existe, entonces existe una función $F$ de dominio $R$ tal que $F(x)=f(x)$ para todo $x$ del dominio de $f$.\\\\

	%---------- (ii)
	\item $f(x) = \dfrac{|x|}{x}$\\\\
	    Respuesta.-\; No existe $F$, ya que $\lim\limits_{x\to 0} \dfrac{|x|}{x}$ no existe.\\\\

	%---------- (iii)
	\item $f(x) = 0,\; x$ irracional.\\\\
	    Respuesta.-\; Existe $F$ de dominio $R$ tal que $F(x)=f(x)$ para todo $x$ del dominio de $f$.\\\\

	%---------- (iv)
	\item $f(x) = 1/q, \; x = p/q$ racional en fracción irreducible.
	    Respuesta.-\; No existe $F$, ya que $F(a)$ tendría que ser $0$ para los $a$ irracionales, y entonces $F$ no podría ser continua en $a$ si $a$ es racional.\\\\

    \end{enumerate}

%--------------------2.
\item ¿En qué puntos son continuas las funciones de los problemas 4-17 y 4-19?.\\\\
    Respuesta.-\; Problema 4-17.\\
    Para (i), (ii) y (iii) son continuas para todos los puntos menos para los enteros. Para (iv) es continua en todos los puntos.Para (v) es entera para todos los puntos excepto para $0$ y $1/n$ para $n$ en los enteros.\\\\
    Problema 4-19.\\
    (i) todos los puntos que no sean de la forma $n+k/10$ para todos los enteros $k$ y $n$. El (ii) para todo los puntos que no sea de la forma $n+k/100$ para todos los enteros $k$ y $n$. (iii) y (iv) para ningún punto. (v) para todos los puntos que el decimal no termine en $7999 \ldots$. Y (vi) para todos los puntos que el decimal contenga al menos un $1$.\\\\

%--------------------3.
\item 
    \begin{enumerate}[\Large\bfseries (a)]

	%---------- (a)
	\item Supóngase que $f$ es una función que satisface $|f(x)|\leq |x|$ para todo $x$. Demostrar que $f$ es continua en $0$.[Observe que $f(0)$ debe ser igual a $0$.]\\\\
	    Demostración.-\; Supongamos que $|f(x)|\leq |x|$. afirmamos que, $\lim\limits_{x\to 0} f(x)=0$. De hecho dado $\epsilon>0$, tomamos $\delta=\epsilon$. Si $|x|<\delta$ entonces $|f(x)|\leq |x|<\delta = \epsilon$. Esto prueba que $\lim\limits_{x\to 0} f(x) = 0$. Para concluir que $f$ es constante en $0$, tenga en cuenta que, como se señala en la pregunta, aplicar $|f(x)|\leq |x|$ para todo $x$, en $x=0$ se da $|f(0)|\leq 0$ y en consecuencia $f(0)=0$.  Por lo tanto $\lim\limits_{x \to 0} f(x) = 0$ implica que $\lim\limits_{x\to 0} f(x) = f(0)$, así $f$ es constante en $0$.\\\\

	%---------- (b)
	\item Dar un ejemplo de una función $f$ que no sea continua en ningún $a\neq 0$.\\\\
	    Respuesta.-\; Sea $f(x)=0$ para $x$ irracional, y $f(x)=x$ para $x$ racional.\\\\

	%---------- (c)
	\item Supóngase que $g$ es continua en $0$, $g(0)=0,$ y $|f(x)|\leq |g(x)|.$ Demostrar que $f$ es continua en $0$.\\\\
	    Demostración.-\; La condición $|f(x)|\leq |g(x)|$ para todo $x$ y $g(0)=0$ implica que $ff(0)=0$, así que sólo tenemos que demostrar que $\lim\limits_{x\to 0} f(x) = 0$.\\
	    Sea $\epsilon>0$, luego ya que $g$ es continua en $0$, existe un $\delta >0$ tal que $|x|<\delta$ entonces $|g(x)-g(0)|=|g(x)|<\epsilon$. Usando $|f(x)|\leq |g(x)|$ para todo $x$, vemos que $|x|<\delta$ implica $|f(x)|\leq |g(x)|<\epsilon$. Por lo tanto esto demuestra que $\lim\limits_{x\to 0} f(x) = 0$.\\\\

    \end{enumerate}

%--------------------4.
\item Dar un ejemplo de una función $f$ que no sea continua en ningún punto, pero tal que $|f|$ sea continua en todos lo puntos.\\\\
    Respuesta.-\; Sea $f(x)=1$ para $x$ racional, y $f(x)=-1$ para $x$ irracional.\\\\

%--------------------5.
\item Para todo número $a$, hallar la función que sea continua en $a$, pero no lo sea en ningún otro punto.\\\\
    Respuesta.-\; Sea $f(x)=a$ para $x$ irracional, y $f(x)=x$ para $x$ racional.\\\\

%--------------------6.
\item 
\begin{enumerate}[\bfseries (a)]

    %---------- (a)
    \item Hallar una función $f$ que sea descontinua en $1,\frac{1}{2},\frac{1}{3},\ldots,$ pero continua en todos los demás puntos.\\\\
	Respuesta.-\; Define $f$ como sigue,
	$$f(x) = \left\{\begin{array}{rl}
	0,&x\leq 0\\\\
	\dfrac{1}{\left[\dfrac{1}{x}\right]},&0<x\leq 1\\\\
	  2,&x>1\\\\
	\end{array}\right.$$

    %---------- (b)
    \item Hallar una función $f$ que sea descontinua en $1,\frac{1}{2},\frac{1}{3},\ldots,$ y en $0$, pero sea continua en ningún en todos los demás puntos.\\\\
	Respuesta.-\; Sea 
	$$f(x) = \left\{\begin{array}{rl}
	-1,&x\leq 0\\\\
	\dfrac{1}{\left[\dfrac{1}{x}\right]},&0<x\leq 1\\\\
	  2,&x>1\\\\
	\end{array}\right.$$

\end{enumerate}

%--------------------7.
\item Supóngase que  $f$ satisface $(x+y) = f(x)+f(y)$, y que $f$ es continua en $0$. Demostrar que $f$ es continua en $a$ para todo $a$.\\\\
    Demostración.-\; Sea $f(x+0) = f(x)+f(0)$, por lo tanto $f(0)=0$, entonces
    $$\begin{array}{rcl}
	\lim\limits_{h\to 0} f(a+h) - f(a)&=&\lim\limits_{h\to 0}f(a)+f(h)-f(a)\\\\
	&=&\lim\limits_{h\to 0}f(h)\\\\
	&=&\lim\limits_{h\to 0} f(h)-f(0) = 0\\\\
    \end{array}$$

%--------------------8.
\item Supóngase que $f$ es continua en $a$ y $f(a)=0$. Demostrar que si $\alpha \neq 0$, entonces $f+\alpha$ es distinta de $0$ en algún intervalo abierto que contiene $a$.\\\\
    Demostración.-\; Sabiendo que $(f+\alpha)(a)\neq 0$, entonces por el teorema 3, $f+\alpha$ es distinto de cero en algún intervalo que contiene a $a$.\\\\

%--------------------9.
\item 
\begin{enumerate}[\bfseries (a)]

    %---------- (a)
    \item Supóngase que $f$ no es continua en $a$. Demostrar que para algún $\epsilon>0$ existen números $x$ tan próximos como se quiere de $a$ con $|f(x)-f(a)|>\epsilon$.\\\\
	Demostración.-\; Lógicamente equivalente a la definición de continuidad se tiene 
	$$\exists\,\epsilon>0, \forall \delta>0, \exists\, x |x-a|<\delta \; y \; |f(x)-f(a)|\geq \epsilon$$
	Existe $\epsilon > 0$ tal que $|f(x)-f(a)|>\epsilon$. Luego sea $\epsilon^{'}=\dfrac{1}{2}\epsilon$, entonces tenemos $|f(x)-f(a)|\geq \epsilon > \epsilon^{'}.$\\\\

    %---------- (b)
    \item Dedúzcase que para algún $\epsilon>0$, o bien existen números $x$ tan próximos como se quiera de $a$ con $f(x)<f(a)-\epsilon$ o bien existen números $x$ tan próximos como se quiera de $a$ con $f(x)>f(a)+\epsilon$.\\\\
	Demostración.-\; La demostración es directa aplicando la reciproca de la definicion de continuidad. Como se vio en el inciso $a$.\\\\

\end{enumerate}

%--------------------10.
\item 
\begin{enumerate}[\bfseries (a)]

    %---------- (a)
    \item Demostrar que si $f$ es continua en $a$, entonces también lo es $|f|$.\\\\
	Demostración.-\; Ya que $\lim\limits_{x\to a} f(x) = l \; \Longrightarrow \; \lim\limits_{x\to a} |f|(x) = |l|$ como se vio en el problema 5-16, entonces 
	$$\lim_{x\to a} |f|(x) = \left|\lim_{x\to a} f(x)\right| = |f(a)| = |f|(a).$$\\

    %---------- (b)
    \item Demostrar que toda función continua $f$ puede escribirse en la forma $f=E+O$, donde $E$ es par y continua y $O$ es impar y continua.\\\\
	Demostración.-\; Por el problema 13 del capítulo 3 (funciones) mostramos que $E$ y $O$ son continuas si $f$ lo es.\\\\

    %---------- (c)
    \item Demostrar que si $f$ y $g$ son continuas, también lo son $\max(f,g)$ y $\min(f,g)$.\\\\
	Demostración.-\; Por la parte a) y sabiendo que 
	$$\begin{array}{rcl}
	    \max(f,g)&=&\dfrac{f+g+|f-g|}{2}\\\\
	    \min(f,g)&=&\dfrac{f+g-|f-g|}{2}\\\\
	    \end{array}$$

    %---------- (d)
    \item Demostrar que toda función continua $f$ puede escribirse en la forma $f=g-h$, donde $g$ y $h$ son no negativas y continuas.\\\\
	Demostración.-\; Por el problema 15 del capítulo 3 (funciones) podemos comprobar que $f=g-h$ siempre que $f$ sea continua.\\\\ 

\end{enumerate}

%--------------------11.
\item Demostrar el teorema 1(3) aplicando el teorema 2 y la continuidad de la función $f(x)=1/x$.\\\\
    Demostración.-\; Sea $f\circ g = \dfrac{1}{g}$ y $f$ es continua en $g(a)$ para $g(a)\neq 0$, entonces por el teorema 2, se tiene que $\dfrac{1}{g}$ es continua en $a$ para $g(a)\neq 0$.\\\\

%--------------------12.
\item 
    \begin{enumerate}[\bfseries (a)]

	%---------- (a)
	\item Demostrar que si $f$ es continua en $l$, y $\lim\limits_{x\to a} g(x) = l$ entonces $\lim\limits_{x\to a}f(g(x)) = f(l)$.\\\\
	    Demostración.-\; Sea 
	    $$G(x) = \left\{\begin{array}{rcl}
		    g(x)&si&x\neq a\\\\
			l&si&x=a\\\\
	    \end{array}\right.$$
	    entonces $G$ es continua en $a$, ya que $G(a)=l=\lim\limits_{x\to a} g(x) = \lim_{x\to a} G(x)$. Así $f\circ G$ es continua en $a$ esto por el teorema 2. Luego 
	    $$f(l)=f(G(a))=(f\circ G)(a) = \lim\limits_{x\to a} = \lim\limits_{x\to a}(f\circ G)(x) = \lim\limits_{x\to 0} f(g(x)).$$\\

	%---------- (b)
	\item Demostrar que si no se supone la continuidad de $f$ en $l$, entonces no se cumple, por lo general, que $\lim\limits_{x\to a} f(g(x)) = f\left(\lim\limits_{x\to a} g(x)\right)$.\\\\
	    Demostración.-\; Sea $g(x)=l+x-a$ y 
	    $$f(x) = \left\{ \begin{array}{rl}
		    0,&x\neq 1\\
		    1, & x=1\\
		\end{array}\right.$$

		Luego $\lim\limits_{x\to a} g(x) = l$, así $f\left(\lim\limits_{x\to a} g(x)\right) = f(l) = l$, pero $g(x)\neq l$ para $x\neq a$, por lo tanto $\lim\limits_{x\to a} f(g(x)) = \lim\limits_{x\to a} = 0$.\\\\

    \end{enumerate}

%--------------------13.
\item
\begin{enumerate}[\bfseries (a)]

    %---------- (a)
    \item Demostrar que si $f$ es continua en $[a,b]$, entonces existe una función $g$ el cual es contuna en $\mathbb{R}$ y que satisface a $g(x)=f(x)$ para todo $x$ en $[a,b]$.\\\\
    Demostración.-\; 

    %---------- (b)
    \item Hágase ver con un ejemplo que ésta afirmación es falsa si se sustituye $[a,b]$ por $(a,b)$.\\\\
	Respuesta.-\; Definimos $f(x)=1/(x^2-1)$ en el intervalo $(-1,1)$. Es continuo, pero no existe $\lim\limits_{x\to 1^+} g(x)$ ni $\lim\limits_{x\to 1^-} f(x)$. Ahora, para que $f$ se extienda a una función $g$ que sea constante en toda la línea real, es necesario que existan tanto $\lim\limits_{x \to -1} g (x)$ como $\lim\limits_{x \to 1} g (x)$, lo que requiere $\lim\limits_{x \to -1^+}  f (x)$ y $\lim\limits_{x \to 1^-} f (x)$ para existir. Entonces $f$ no se puede extender a una función que sea constante en toda la línea real.\\\\

\end{enumerate}

%--------------------14.
\item
\begin{enumerate}[\bfseries (a)]

    %---------- (a)
    \item Supóngase que $g$ y $h$ son continuos en $a$, y que $g(a)=h(a)$, Defínase $f(x)$  como $g(x)$ si $x\geq a$ y $h(x)$ si $x\leq a$. Demuestre que $f$ es continua en $a$.\\\\
	Demostración.-\; Sea 
	$$f(x) = \left\{\begin{array}{rcl}
	    g(x)&si&x\geq a\\
	    h(x)&si&x\leq a\\
	\end{array}\right.$$
	Luego se tiene $$\lim\limits_{x\to a} g(x) = g(a)\quad  y \quad \lim\limits_{x\to a} h(x) = h(a)$$ de donde $$g(a) = \lim\limits_{x\to a^+} g(x) = \lim\limits_{x\to a^+} f(x) = f(a) \quad y \quad h(a) = \lim\limits_{x\to a^-} h(x) = \lim\limits_{x\to a^-} f(x) = f(a)$$
	y por lo tanto $$\lim\limits_{x\to a} f(x) = f(a)$$\\

    %---------- (b)
    \item Supóngase que $g$ es continuo en $[a,b]$ y $h$ es continuo en $[b,c]$ y $g(b)=h(b)$. Sea $f(x)$ igual $g(x)$ para $x$ en $[a,b]$ y $h(x)$ para $x$ en $[b,c]$. Demuestre que $f$ es continuo en $[a,c]$. (Así pues, las funciones continuas pueden ser $"$pegados juntos$"$).\\\\
	Demostración.-\; 

\end{enumerate}



\end{enumerate}
