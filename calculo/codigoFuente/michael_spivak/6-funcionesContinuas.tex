\chapter{Funciones Continuas}

% ---------- definición 
\begin{tcolorbox}[colframe = white]
    \begin{def.} La función $f$ es continua  en $a$ si $$\lim\limits_{x\to a} f(x) = f(a)$$
	Para todo $\epsilon > 0$ existe un $\delta > 0$ tal que, para todo $x$, si $0<|x-a|<\delta$.\\
	Pero en este caso, en que el límite es $f(a)$, la frase $$0<|x-a|<\delta$$
	puede cambiarse por la condición más sencilla $$|x-a|<\delta$$
	puesto que si $x=a$ se cumple ciertamente que $|f(x)-f(a)|<\epsilon$.
    \end{def.}
\end{tcolorbox}

%---------- teorema 1
\begin{teo} Si $f$ y $g$ son continuas en $a$, entonces 
    \begin{center}
	\begin{tabular}{rl}
	    $(1)$ & $f+g$ es continua en $a$.\\
	    $(2)$ & $f\cdot g$ es continua en $a$.\\
    Además, si $g(a)\neq 0$, entonces 
	    $(3)$&$1/g$ es continua en $a$\\
	\end{tabular}
    \end{center}	
    Demostración.-\; Puesto que $f$ y $g$ son continuas en $a$,
    $$\lim_{x\to a}f(x) = f(a) \qquad y \qquad \lim_{x\to a} g(x) = g(a).$$
    Por el teorema 2(1) del capítulo 5 esto implica que 
    $$\lim_{x\to a} (f+g)(x) = f(a)+g(a) = (f+g)(a),$$
    lo cual es precisamente la afirmación de que $f+g$ es continua en $a$.\\
    Para $f\cdot g$ se tiene que $$\lim_{x\to a}(f\cdot g)(x) = f(a)\cdot fg(a) = (f\cdot g)(a)$$
    Por último para $1/g$ tenemos que $$\lim_{x\to a} 1/g = 1/g(a) , \qquad para \; g(a)\neq 0$$\\
\end{teo}

%---------- teorema 2
\begin{teo} Si $g$ es continua en $a$, y $f$ es continua en $g(a)$, entonces $f\circ g$ es continua en $a$.\\\\
    Demostración.-\; Sea $\epsilon > 0$. Queremos hallar un $\delta > 0$ tal que para todo $x$, 

    \begin{center}
	Si $|x-a|<\delta$ entonces $|(f\circ g)(x)-(f\circ g)(a)|<\epsilon$, es decir, $|f(g(x))-f(g(a))|<\epsilon$
    \end{center}

	Tendremos que aplicar primero la continuidad de $f$ para estimar cómo de cerca tiene que estar $g(x)$ de $g(a)$ para que se cumpla esta desigualdad. Puesto que $f$ es continua en $g(a)$, existe un $\delta^{'} > 0$ tal que para todo $y$,

    \begin{center}
	Si $|y-g(a)|<\delta^{'}$, entonces $|f(y)-f(g(a))|<\epsilon. \qquad (1)$ 
    \end{center}

    En particular, esto significa que

    \begin{center}
	Si $|g(x)-g(a)|<\delta^{'}$, entonces $|f(g(x))-f(g(a))|<\epsilon. \qquad (2)$
    \end{center}
    
    Aplicamos ahora la continuidad de $g$ para estimar cómo de cerca tiene que estar $x$ de $a$ para que se cumpla la desigualdad $|g(x)-g(a)|<\delta^{'}$. El número $\delta^{'}$ es un número positivo como cualquier otro número positivo; podemos, por lo tanto, tomar $\delta^{'}$ como el $epsilon$ de la definición de continuidad de $g$ en $a$. Deducimos que existe un $\delta > 0$ tal que, para todo $x$,

    \begin{center}
	Si $|x-a|<\delta,$ entonces $|g(x)-g(a)|<\delta^{'}, \qquad (3)$
     \end{center}

     combinando (2) y (3) vemos que para todo $x$,

    \begin{center}
	Si $|x-a|<\delta,$ entonces $|f(g(x))-f(g(a))|<\epsilon.$
     \end{center}
     \vspace{1cm}

\end{teo}

% definición
\begin{tcolorbox}[colframe = white]
    \begin{def.} Si $f$ es continua en $x$ para todo $x$ en $(a,b)$, entonces se dice que $f$ es continua en $(a,b)$ si 
	\begin{center}
	    $f$ es continua en $x$ para todo $x$ de $(a,b), \qquad (1)$\\
	    \vspace{.5cm}
	    $\lim\limits_{x\to a^+} f(x) = f(a)$ y $\lim\limits_{x\to b^-}f(x) = f(b). \qquad (2)$
	\end{center}
    \end{def.}
\end{tcolorbox}

%---------- teorema 3
\begin{teo} Supóngase que $f$ es continua en $a$, y $f(a)>0$. Entonces existe un número $\delta>0$ tal que $f(x)>0$ para todo $x$ que satisface $|x-a|<\delta$. Análogamente, si $f(a)>0$, entonces existe un número $\delta>0$ tal que $f(x)<0$ para todo $x$ que satisface $|x-a|<\delta.$\\\\
    Demostración.-\; Considérese el caso $f(a)>0$ puesto que $f$ que es continua en $a$, si $\epsilon>0$ existe un $\delta>0$ tal que, para todo $x$,
    \begin{center}
	Si $|x-a|<\delta$, entonces $|f(x)-f(a)|<\epsilon.$
    \end{center}
    Puesto que $f(a)>0$ podemos tomar a $f(a)$ como el $epsilon$. Así, pues, existe $\delta>0$ tal que para todo $x$,
    \begin{center}
	Si $|x-a|<\delta$, entonces $|f(x)-f(a)|<f(a)$
    \end{center}
    Y esta última igualdad implica $f(x)>0$.\\
    Puede darse una demostración análoga en el caso $f(a)<0$; tómese $\epsilon=-f(a)$. O también se puede aplicar el primer caso a la función $-f$.\\\\

\end{teo}

\section{Problemas}
\begin{enumerate}[\Large\bfseries 1.]

%--------------------1.
\item ¿para cuáles de las siguientes funciones $f$ existe una función $F$ de dominio $R$ tal que $F(x)=f(x)$ para todo $x$ del dominio de $f$?

    \begin{enumerate}[\Large\bfseries (i)]

	%---------- (i)
	\item $f(x) = \dfrac{x^2 - 4}{x - 2}$\\\\
	    Respuesta.-\; Sabiendo que el límite cuando $x$ tiende a $2$ existe, entonces existe una función $F$ de dominio $R$ tal que $F(x)=f(x)$ para todo $x$ del dominio de $f$.\\\\

	%---------- (ii)
	\item $f(x) = \dfrac{|x|}{x}$\\\\
	    Respuesta.-\; No existe $F$, ya que $\lim\limits_{x\to 0} \dfrac{|x|}{x}$ no existe.\\\\

	%---------- (iii)
	\item $f(x) = 0,\; x$ irracional.\\\\
	    Respuesta.-\; Existe $F$ de dominio $R$ tal que $F(x)=f(x)$ para todo $x$ del dominio de $f$.\\\\

	%---------- (iv)
	\item $f(x) = 1/q, \; x = p/q$ racional en fracción irreducible.
	    Respuesta.-\; No existe $F$, ya que $F(a)$ tendría que ser $0$ para los $a$ irracionales, y entonces $F$ no podría ser continua en $a$ si $a$ es racional.\\\\

    \end{enumerate}

%--------------------2.
\item ¿En qué puntos son continuas las funciones de los problemas 4-17 y 4-19?.\\\\
    Respuesta.-\; Problema 4-17.\\
    Para (i), (ii) y (iii) son continuas para todos los puntos menos para los enteros. Para (iv) es continua en todos los puntos.Para (v) es entera para todos los puntos excepto para $0$ y $1/n$ para $n$ en los enteros.\\\\
    Problema 4-19.\\
    (i) todos los puntos que no sean de la forma $n+k/10$ para todos los enteros $k$ y $n$. El (ii) para todo los puntos que no sea de la forma $n+k/100$ para todos los enteros $k$ y $n$. (iii) y (iv) para ningún punto. (v) para todos los puntos que el decimal no termine en $7999 \ldots$. Y (vi) para todos los puntos que el decimal contenga al menos un $1$.\\\\

%--------------------3.
\item 




\end{enumerate}
