\chapter{Funciones inversas}

%-------------------- Definición 1.1
\begin{def.}
    Una función $f$ es \textbf{uno-uno} (que se lee "uno a uno") si $f(a)\neq f(b)$ cuando $a\neq b$.
\end{def.}
\vspace{0.3cm}
La condición $f(a)\neq f(b)$ para $a\neq b$ significa que ninguna línea \textit{horizontal} corta a la gráfica de $f$ en más de un punto.

%-------------------- Definición 1.2
\begin{def.}
    Para cualquier función $f$, la \textbf{inversa} de $f$, denotada por $f^{-1}$, es el conjunto de todos los pares $(a,b)$ para los que el par $(b,a)$ pertenece a $f$.
\end{def.}

%-------------------- Teorema 1.1
\begin{teo}
    $f^{-1}$ es una función si y sólo si $f$ es uno-uno.\\\\
	Demostración.-\; Supongamos primero que $f$ es uno-uno. Sean $(a,b)$ y $(a,c)$ dos pares de $f^{-1}$. Entonces $(b,a)$ y $(c,a)$ pertenecen a $f$, de manera que $a=f(b)$ y $a=f(c)$; como $f$ es uno-uno, $b=c$. Por lo tanto, $f^{-1}$ es una función.\\
	Recíprocamente, Supongamos que $f^{-1}$ es una función. Si $f(b)=f(c)$, entonces $f$ contiene a los pares $\left(b,f(b)\right)$ y $\left(c,f(c)\right)=\left(c,f(b)\right)$, y por lo tanto $\left(f(b),b\right)$ y $\left(f(b),c\right)$ pertenecen a $f^{-1}$. Como por hipótesis, $f^{-1}$ es una función, $b=c$; es decir, $f$ es uno-uno.
\end{teo}
\vspace{0.3cm}
Es evidente por definición que 
$$\left(f^{-1}\right)^{-1}=f.$$

Como $(a,b)$ pertenece a $f$ si y sólo si $(b,a)$ pertenece a $f^{-1}$, deducimos que

