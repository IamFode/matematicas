\chapter{Funciones inversas}

%-------------------- Definición 1.1
\begin{def.}
    Una función $f$ es \textbf{uno-uno} (que se lee "uno a uno") si $f(a)\neq f(b)$ cuando $a\neq b$.
\end{def.}
\vspace{0.3cm}
La condición $f(a)\neq f(b)$ para $a\neq b$ significa que ninguna línea \textit{horizontal} corta a la gráfica de $f$ en más de un punto.

%-------------------- Definición 1.2
\begin{def.}
    Para cualquier función $f$, la \textbf{inversa} de $f$, denotada por $f^{-1}$, es el conjunto de todos los pares $(a,b)$ para los que el par $(b,a)$ pertenece a $f$.
\end{def.}

%-------------------- Teorema 1.1
\begin{teo}
    $f^{-1}$ es una función si y sólo si $f$ es uno-uno.\\\\
	Demostración.-\; Supongamos primero que $f$ es uno-uno. Sean $(a,b)$ y $(a,c)$ dos pares de $f^{-1}$. Entonces $(b,a)$ y $(c,a)$ pertenecen a $f$, de manera que $a=f(b)$ y $a=f(c)$; como $f$ es uno-uno, $b=c$. Por lo tanto, $f^{-1}$ es una función.\\
	Recíprocamente, Supongamos que $f^{-1}$ es una función. Si $f(b)=f(c)$, entonces $f$ contiene a los pares $\left(b,f(b)\right)$ y $\left(c,f(c)\right)=\left(c,f(b)\right)$, y por lo tanto $\left(f(b),b\right)$ y $\left(f(b),c\right)$ pertenecen a $f^{-1}$. Como por hipótesis, $f^{-1}$ es una función, $b=c$; es decir, $f$ es uno-uno.
\end{teo}
\vspace{0.3cm}
Es evidente por definición que 
$$\left(f^{-1}\right)^{-1}=f.$$

Como $(a,b)$ pertenece a $f$ si y sólo si $(b,a)$ pertenece a $f^{-1}$, deducimos que

\begin{center}
    $b=f(a)$ significa lo mismo que $a=f^{-1}(b).$
\end{center}

Por lo tanto, $f^{-1}(b)$ es el \textit{único} número $a$ tal que $f(a)=b$.\\\\

El hecho de que $f^{-1}(x)$ sea el número y tal que $f(y)=x$, puede ser expresado de una forma mucho más compacta:
\begin{center}
    $f\left[f^{-1}(x)\right]=x$ para todo $x$ del dominio de $f^{-1}$.
\end{center}

Además,
\begin{center}
    $f^{-1}\left[f(x)\right]=x$ para todo $x$ del dominio de $f$.
\end{center}

Estas dos importantes ecuaciones pueden escribirse como
$$f\circ f^{-1}=I,$$
$$f^{-1}\circ f=I.$$

% ------------------- Lema 12.1
\begin{lema}
    Si $f$ es creciente, $f^{-1}$ también es creciente, y si $f$ es decreciente, $f^{-1}$ también es decreciente.\\\\
	Demostración.-\; 
\end{lema}

%-------------------- Lema 12.2
\begin{lema}
    $f$ es creciente si y sólo si $-f$ es decreciente.\\\\
	Demostración.-\; 
\end{lema}

%-------------------- Teorema 2
\begin{teo}
    Si $f$ es continua y uno-uno en un intervalo, entonces $f$ es creciente o decreciente en dicho intervalo.\\\\
	Demostración.-\; La demostración se da en tres etapas sencillas:
	\begin{enumerate}[(1)]
	    \item Si $a<b<c$ son tres puntos del intervalo, entonces
		\begin{enumerate}[(i)]
		    \item o bien $f(a)<f(b)<f(c)$
		    \item o $f(a)>f(b)>f(c)$.
		\end{enumerate}
		Supongamos, por ejemplo, que $f(a)<f(c)$. Si $f(b)<f(a)$, entonces aplicando el Teorema del valor intermedio al intervalo $[b,c]$ se obtendrá un $x$ con $b<x<c$ y $f(x)=f(a)$. Lo que contradice el hecho de que $f$ sea uno-uno en $[a,c]$. Análogamente, $f(b)>f(c)$ conduciría también a una contradicción, de manera que $f(a)<f(b)<f(c)$. Naturalmente, el mismo argumento sirve para el caso $f(a)>f(c).$
	    \item Si $a<b<c<d$ son cuatro puntos del intervalo, entonces
		\begin{enumerate}[(i)]
		    \item o bien $f(a)<f(b)<f(c)<f(d),$
		    \item o $f(a)>f(b)>f(c)>f(d)$.
		\end{enumerate}
		Ya que puede aplicarse (1) primero a los puntos $a<b<c$ y luego a los puntos $b<c<d.$
	    \item Tomemos cualquier $a<b$ del intervalo y supongamos que $f(a)<f(b)$. Entonces $f$ es creciente ya que si $c$ y $d$ son dos puntos cualesquiera del intervalo, podemos aplicar (2) al conjunto $\left\{a,b,c,d\right\}$ (después de colocar a los puntos en orden creciente).
	\end{enumerate}
\end{teo}
