
\chapter{Tres teoremas fuertes}


% -------------------- teorema 1 --------------------
\begin{tcolorbox}[colback = white]
    \begin{teo}
	Si $f$ es continua en $[a,b]$ y $f(a)<0<f(b)$ entonces existe algún $x$ en $[a,b]$ tal que $f(x)=0$.\\\\
	Geométricamente, esto significa que la gráfica de una función continua que empieza por debajo del eje horizontal y termina por encima del mismo debe cruzar a este eje en algún punto.
    \end{teo}
\end{tcolorbox}

% -------------------- teorema 2 --------------------
\begin{tcolorbox}[colback = white]
    \begin{teo}
	Si $f$ es continua en $[a,b]$, entonces $f$ está acotada superiormente en $[a,b]$, es decir, existe algún número $N$ tal que $f(x)\leq N$ para todo $x$ en $[a,b]$.\\\\
	Geométricamente, este teorema significa que la gráfica $f$ queda por debajo de alguna línea paralela al eje horizontal. 
    \end{teo}
\end{tcolorbox}

% -------------------- teorema 3 --------------------
\begin{tcolorbox}[colback = white]
    \begin{teo}
	Si $f$ es continua en $[a,b]$ entonces existe algún número $y$ en $[a,b]$ tal que $f(y)\leq f(x)$ para todo $x$ en $[a,b]$.\\\\
	Se dice que una función continua en un intervalo cerrado alcanza su valor máximo en dicho intervalo.
    \end{teo}
\end{tcolorbox}

% -------------------- teorema 4 --------------------
\begin{teo}
    Si $f$ es continua en $[a,b]$ y $f(a)<c<f(b)$, entonces existe algún $x$ en $[a,b]$ tal que $f(x)=c$.\\\\
    Demostración.-\; Sea $g=f-c$. Entonces $g$ es continua, y $\; g(a)+c < c < g(b) + c \; \Longrightarrow \; g(a)<0<g(b)$. Por el teorema 1, existe algún $x$ en $[a,b]$ tal que $g(x)=0$. Pero esto significa que $f(x)=c.$\\\\
\end{teo}

% -------------------- teorema 5 --------------------
\begin{teo}
    Si $f$ es continua en $[a,b]$ y $f(a)>c>f(b)$, entonces existe algún $x$ en $[a,b]$ tal que $f(x)=c$.\\\\
    Demostración.-\; La función $-f$ es continua en $[a,b]$ y $-f(a)<-c<-f(b)$. Por el teorema 4 existe algún $x$ en $[a,b]$ tal que $-f(x)=-c$, lo que significa que $f(x)=c$.\\\\
\end{teo}

Si una función continua en un intervalo toma dos valores, entonces toma todos los valores comprendidos entre ellos; esta ligera generalización del teorema 1 recibe a menudo el nombre de \textbf{teorema de los valores intermedios}.\\\\ 

%---------------------- teorema 6 ----------------------
\begin{teo}
    Si $f$ es continua en $[a,b]$, entonces $f$ es acotada inferiormente en $[a,b]$, es decir, existe algún número $N$ tal que $f(x)\geq N$ para todo $x$ en $[a,b]$.\\\\
    Demostración.-\; La función $-f$ es continua en $[a,b]$, así por el teorema 2 existe un número $M$ tal que $-f(x)\leq M$ para todo $x$ en $[a,b]$. Pero esto significa que $f(x)\geq -M$ para todo $x$ en $[a,b]$, así podemos poner $N=-M$.\\\\
\end{teo}

Los teoremas 2 y 6 juntos muestran  que una función continua $f$ en $[a,b]$ son acotados en $[a,b]$, es decir, existe un número $N$ tal que $|f(x)|\leq N$ para todo $x$ en $[a,b]$. En efecto, puesto que el teorema 2 asegura la existencia de un número $N_1$ tal que $f(x)\leq N,$ para todo $x$ de $[a,b]$ y el teorema 6 asegura la existencia de un número $N_2$ tal que $f(x)\geq N$, para todo $x$ en $[a,b]$, podemos tomar $N=\max(|N_1|,|N_2|)$.\\\\

%---------------------- teorema 7 ----------------------
\begin{teo}
    Si $f$ es continua en $[a,b]$, entonces existe algún $y$ en $[a,b]$ tal que $f(y)\leq f(x)$ para todo $x$ en $[a,b]$.\\\\
    Demostración.-\; La función $-f$ es continua en $[a,b]$; por el teorema 3 existe algún $y$ en $[a,b]$ tal que $-f(y)\geq -f(x)$ para todo $x$ en $[a,b]$, lo que significa que $f(y)\leq f(x)$ para todo $x$ en $[a,b]$.\\\\
\end{teo}

%---------------------- teorema 8 ----------------------
\begin{teo}
    Todo número positivo posee una raíz cuadrada. En otras palabras, si $\alpha > 0$, entonces existe algún número $x$ tal que $x^2=\alpha$.\\\\
    Demostración.-\; Consideremos la función $f(x)=x^2$, el cual es ciertamente continuo. Notemos que la afirmación del teorema puede ser expresado en términos de $f$: $"$el número $\alpha$ posee una raíz cuadrada$"$ significa que $f$ toma el valor $alpha$. La demostración de este hecho acerca de $f$ será una consecuencia fácil del teorema 4.\\
    Existe, evidentemente, un número $b>0$ tal que $f(b)>\alpha$; en efecto, si $\alpha>1$ podemos tomar $b=\alpha$, mientras que si $\alpha<1$ podemos tomar $b=1$. Puesto que $f(0)<\alpha <f(b)$, el teorema 4 aplicado a $[0,b]$ implica que para algún $x$ de $[0,b]$, tenemos $f(x)=\alpha$, es decir, $x^2=\alpha$.\\\\
    Precisamente el mismo raciocinio puede aplicarse para demostrar que todo número positivo tiene una raíz n-ésima, cualquiera que sea el número $n$. Si $n$ es impar, se puede decir mas: todo número tiene una raíz n-ésima. Para demostrarlo basta observar que si el número positivo $\alpha$ tiene la raíz n-ésima $x$, es decir, si $x^n=\alpha$, entonces $(-x)^n=-\alpha$ (puesto que $n$ es impar), de modo que $\alpha$ tiene una raíz n-ésima $-\alpha$. Afirmar que, para un $n$ impar, cualquier número $\alpha$ tiene una raíz n-ésima equivale a afirmar que la ecuación
    $$x^n - \alpha = 0$$
    tiene una raíz si $n$ es impar. El resultado expresado de este modo es susceptible de gran generalización.\\\\
\end{teo}

%---------------------- teorema 9 ----------------------
\begin{teo}
    Si $n$ es impar, entonces cualquier ecuación 
    $$x^n + a_{n-1}x^{n-1}+\ldots + a_0 = 0$$
    posee raíz.\\\\
	Demostración.-\; Tendremos que considerar, evidentemente, la función 
	$$f(x) = x^n + a_{n-1}x^{n-1} + \ldots + a_0$$
	habría que demostrar que $f$ es unas veces positiva y otras veces negativa. La idea intuitiva es que para un $|x|$ grande, la función se parece mucho más a $g(x)=x^n$ y puesto que $n$ es impar, ésta función es positiva para $x$ grandes positivos y negativos para $x$ grandes negativos. Un poco de cálculo algebraico es todo lo que hace falta para dar formar a esta idea intuitiva.\\
	Para analizar debidamente la función $f$ conviene escribir 
	$$f(x) = x^n + a_{n-1}x^{n-1}+\ldots + a_0 = x^n\left(1+\dfrac{a_{n-1}}{x} + \ldots + \dfrac{a_0}{x^n}\right)$$
	obsérvese que
	$$\bigg| \dfrac{a_{n-1}}{x} + \dfrac{a_{n-2}}{x^2} + \ldots + \dfrac{a_0}{x^n} \bigg|\leq \dfrac{|a_{n-1}|}{|x|} + \ldots + \dfrac{|a_0|}{|x^n|}$$
	En consecuencia, si elegimos un $x$ que satisfaga
	$$|x|>1,2n|a_{n-1}|,\ldots,2n|a_0|\qquad (*)$$
	entonces $|x^k| > |x|$ y 
	$$\dfrac{|a_{n-k}}{x^k}<\dfrac{a_{n-k}}{|x|}<\dfrac{|a_{n-k}|}{2n|a_{n-k}}=\dfrac{1}{2n}$$
	de modo que 
	$$\bigg| \dfrac{a_{n-1}}{x} + \dfrac{a_{n-2}}{x^2} + \ldots + \dfrac{a_0}{x^n} \bigg|\leq \dfrac{1}{2n}+\ldots + \dfrac{1}{2n}=\dfrac{1}{2}$$
	expresado de otra forma,
	$$-\dfrac{1}{2}\leq \dfrac{a_{n-1}}{x}+\ldots + \dfrac{a_0}{x_n}\leq \dfrac{1}{2},$$
	lo cual implica que 
	$$\dfrac{1}{2}\leq 1 + \dfrac{a_{n-1}}{x}+\ldots + \dfrac{a_0}{x_n}..$$
	Por lo tanto, si elegimos un $x_1>0$ que satisfaga $(*)$, entonces 
	$$\dfrac{x_1^n}{2} \leq x_1^n \left(1+\dfrac{a_{n-1}}{x_1}+\ldots + \dfrac{a_0}{x^n}\right) = f(x_1)$$
	así que $f(x_1)>0$. Por otro lado, si $x_2<0$ satisface $(*)$, entonces $x_2^n < 0$ y 
	$$\dfrac{x_2^n}{2}\geq x_2^n \left(1+\dfrac{a_{n-1}}{x_2}+\ldots + \dfrac{a_0}{x_2^n}\right)=f(x_2),$$
	así $f(x_2)<0$. \\
	Ahora aplicando el teorema 1 para el intervalo $[x_2,x_1]$ llegamos a la conclusión de que existe un $x$ en $[x_2,x_1]$ tal que $f(x)=0$.\\\\

\end{teo}

%---------------------- teorema 10 ----------------------
\begin{teo}
    Si $n$ es par y $f(x)=x^n + a_{n-1} x^{n-1} + \ldots + a_0$, entonces existe un número $y$ tal que $f(y)\leq f(x)$ para todo $x$.\\\\
	Demostración.-\; Lo mismo que en el teorema 9, si 
	$$M = \max(1,2n|a_{n-1}|,\ldots , 2n|a_0|),$$
	entonces para todo $x$ con $|x|\geq M,$ tenemos
	$$\dfrac{1}{2}\leq 1 + \dfrac{a_{n-1}}{x}+\ldots + \dfrac{a_0}{x^n}$$
	Al ser $n$ par, $x^n \geq 0$ para todo $x$, de modo que 
	$$\dfrac{x^n}{2}\leq x^n \left(1+\dfrac{a_{n-1}}{x} + \ldots + \dfrac{a_0}{x^n}\right) = f(x),$$
	siempre que $|x|\geq M$. Consideremos ahora el número $f(0)$. Sea $b>0$ un número tal que $b^n \geq 2f(0)$ y también $b>M$. Entonces si $x\geq b,$ tenemos 
	$$f(x)\geq \dfrac{x^n}{2}\geq \dfrac{b^n}{2}\geq f(0).$$
	Análogamente, si $x\leq -b$, entonces
	$$f(x)\geq \dfrac{}{}\geq \dfrac{}{} = \dfrac{}{}\geq f(0).$$
	Resumiendo ahora el teorema 7 a la función $f$ en el intervalo $[-b,b]$. Se deduce que existe un número $y$ tal que
	$$(1)\qquad \mbox{si}\; -b\leq x\leq b, \; \mbox{entonce}\; f(y)\leq f(x).$$
	En particular, $f(y)\leq f(0).$ De este modo
	$$(2)\qquad \mbox{si}\; x\leq -b \; \mbox{o}\; x\geq b,\; \mbox{entonces}\; f(x)\geq f(0)\geq f(y).$$
	Cambiando (1) y (2) vemos que $f(y)\leq f(x)$ para todo $x$.\\\\
\end{teo}

%---------------------- teorema 11 ----------------------
\begin{teo}
    Consideremos la ecuación
    $$(*)\qquad x^n + a_{n-1}x^{n-1}+ \ldots + a_0 = c,$$
    y supongamos que $n$ es par. Entonces existe un número $m$ tal que $(*)$ posee una solución para $c\geq m$ y no posee ninguna para $c<m.$\\\\
    Demostración.-\; Sea $f(x) = x^n + a_{n-1}x^{n-1}+\ldots + a_0$
    Según el teorema 10, existe un número $y$ tal que $f(y)\leq f(x)$ para todo $x$.\\
    Sea $m=f(y)$. Si $c<m$ entonces la ecuación $(*)$ no tiene, evidentemente, ninguna solución, puesto que el primer miembro tiene un valor $\geq m$. Si $c=m$ entonces $(*)$ tiene $y$ como solución. Finalmente, supongamos $c>m$. Sea $b$ un número tal que $b>y$, $f(b)>c$. Entonces $f(y)=m<c<f(b)$. En consecuencia, según el teorema 4, existe algún número $x$ en $[y,b]$ tal que $f(x)=c$, con lo que $x$ es una solución de $(*).$\\\\
\end{teo}





\section{Problemas}

\begin{enumerate}[\Large\bfseries 1.]

%-------------------- 1.
\item Para cada una de las siguientes funciones, decidir cuáles está acotadas superiormente o inferiormente en el intervalo indicado, y cuáles de ellas alcanzan sus valores máximo y mínimo.

    \begin{enumerate}[\bfseries (i)]

	%----------(i)
	\item $f(x) = x^2$ en $(-1,1)$.\\\\
	    Respuesta.-\; Se encuentra acotada superior como inferiormente. El mínimo es $0$ e no tiene máximo.\\\\

	%----------(ii)
	\item $f(x) = x^3$ en  $(-1,1)$.\\\\
	    Respuesta.-\; Se encuentra acotada superior como inferiormente. No tiene máximo ni mínimo\\\\

	%----------(iii)
	\item $f(x) = x^2$ en $\mathbf{R}$.\\\\
	    Respuesta.-\; No está acotado superior pero si inferiormente. Su mínimo es $0$ y no tiene  máximo.\\\\ 

	%----------(iv)
	\item $f(x)=x^2$ en $[0,\infty)$.\\\\
	    Respuesta.-\; Está acotada inferiormente pero no así superiormente. Su mínimo es $0$ y no tiene máximo.\\\\ 

	%----------(v)
	\item $f(x) = \left\{\begin{array}{ll} x^2, & x\leq a \\ a+2, & x\geq a \end{array}\right.$ en $(-a-1,a+1)$\\\\
	    Respuesta.-\; Es acotado superior e inferiormente. Se entiende que $a>-1$ (de modo que $-a-1<a+1$). Si $-1<a\leq 1/2$, entonces $a<-a-1$, así $f(x)=a+2$ para todo $x$ en $(-a-1,a+1)$, por lo tanto $a+2$ es el máximo y mínimo valor. Si $-1/2<a\leq 0$, entonces $f$ tiene el mínimo valor en $a^2$, y si $a\geq 0$, entonces $f$ tiene un mínimo valor en $0$. Ya que $a+2>(a+1)^2$ solo para $[-1-\sqrt{5}]/2 < a < [1+\sqrt{5}]/2$, cuando $a\geq -1/2$ésta función $f$ tiene un máximo valor solo para $a\leq [1+\sqrt{5}]/2$ (el máximo valor será $a+2$).\\\\

	%----------(vi)
	\item $f(x) = \left\{\begin{array}{ll}x^2, & x<a \\ a+2, & x\geq a \end{array}\right.$ en $[-a-1,a+1].$\\\\
	    Respuesta.-\; Está acotado superior e inferiormente. Como en la parte (v), se asume que $a>-1$. Si $a\leq -1/2$ entonces $f$ tiene el valor mínimo y un máximo $3/2$. Si $a\geq 0$, entonces $f$ tiene un valor mínimo en $0$, y un valor máximo $max(a^2,a+2)$. Si $-1/2<a<0$ , entonces $f$ tiene un máximo valor $3/2$ y no así con un valor mínimo.\\\\

	%----------(vii)
	\item $f(x) = \left\{\begin{array}{ll} 0,& x\; \mbox{irracional}  \\  1/q,& x=p/q \; \mbox{fracción irreducible} \end{array}\right.$ en $[0,1].$\\\\
	    Respuesta.-\; Acotada superior e inferiormente. El mínimo es $0$ y el máximo es $1$.\\\\

	%----------(viii)
	\item $f(x) = \left\{\begin{array}{rcl}1, & x \; \mbox{irracional} \\ 1/q, & x = p/q \; \mbox{fracción irreducible}\end{array}\right.$ en $[0,1]$.\\\\
	    Respuesta.-\; Acotada superior e inferiormente.  El máximo es $1$ y no existe un mínimo.\\\\

	%----------(ix)
	\item $f(x) = \left\{\begin{array}{ll}1, & x \; \mbox{irracional} \\ 0, & x=p/q\; \mbox{fracción irreducible}\end{array}\right.$ en $[0,1]$.\\\\
	    Respuesta.-\; Acotada superior e inferiormente. El mínimo es $-1$ y el máximo es $1$.\\\\

	%----------(x)
	\item $f(x) = \left\{ \begin{array}{ll} x, & x\; \mbox{racional} \\ 0, & x\; \mbox{irracional}\end{array} \right.$ en $[0,a]$.\\\\
		Respuesta.-\; Acotada superior e inferiormente. El mínimo es $0$ y el máximo es $a$.\\\\

	%----------(xi)
	\item $f(x) = \sen^2(\cos x + \sqrt{1-a^2})$ en $[0,a^3]$.\\\\
	    Respuesta.-\; Ya que es continua $f$ tiene máximo como también mínimo.\\\\

	%----------(xii)
	\item $f(x)=[x]$ en $[0,a]$.\\\\
	    Respuesta.-\; Acotada superior e inferiormente. El mínimo es $0$ y el máximo es $a$.\\\\

    \end{enumerate}

%-------------------- 2.
\item Para cada una de las siguientes funcione polinómicas $f$, hallar un entero $n$ tal que $f(x)=0$ para algún $x$ entre $n$ y $n+1$.\\
    \begin{enumerate}[\bfseries (i)]

	%----------(i)
	\item $f(x)=x^3-x+3$.\\\\
	    Respuesta.-\; $n=-2$, ya que $f(-2) = (-2)^3+2+3 = -3 < 0 <  3 =  (-1)^3 - (-1) + 3$\\\\

	%----------(ii)
	\item $f(x) = x^5+5x^4 + 2x + 1$.\\\\
	    Respuesta.-\; $n=-5$ ya que $f(-5) = -11<0<f(-4)$.\\\\

	%----------(iii)
	\item $f(x) = x^5 + x + 1$.\\\\
	    Respuesta.-\; $n=-1$ ya que, $f(-1) = -1<0f(0)$.\\\\ 

	%----------(iv)
	\item $4x^2-4x+1$\\\\
	    Respuesta.-\; No existe un entero $n$ tal que $f(x)=0$.\\\\

    \end{enumerate}

%-------------------- 3.
\item Demostrar que existe algún número $x$ tal que

    \begin{enumerate}[\bfseries (i)]

	%----------(i)
	\item $x^{179} + \dfrac{163}{1+x^2+\sen^2 x} = 119$.\\\\
	    Respuesta.-\; Si $x^{179}$  y  $\dfrac{163}{1+x^2+\sen^2 x}$, son continuas en $\mathbb{R}$ entonces $f(x) = x^{179}+\dfrac{163}{1+x^2+\sen^2 x}$ es continua en $\mathbb{R}$ y $f(1)>0$, mientras que $f(-2)<0$, de modo que $f(x)=0$ para algún $x$ en $(-2,1)$.\\\\

	%----------(ii)
	\item $\sen x =x-1.$\\\\
	    Respuesta.-\; Sea $f(x) = \sen x - x + 1$ entonces $f$ es continua en $\mathbb{R}$ y $f(0)>0$, mientras que $f(2)<0$, así por el teoremas 4 se tiene que $f(x)=c$ para algún $x$ en $(0,2)$.\\\\

    \end{enumerate}

%-------------------- 4.
\item Este problema es una continuación del problema 3-7
    \begin{enumerate}[\bfseries (a)]

	%----------(a)
	\item Si $n-k$ es par, y $\geq 0$, hallar una función polinómica de grado $n$ que tenga exactamente $k$ raíces.\\\\
	    Respuesta.-\; Sea $l = (n-k)/2$ de donde 
	    $$f(x) = (x^{2(n-k)/2} + 1)(x-1)(x-2)\cdot \cdot \cdot (x-k).$$\\\\
	
	%----------(b)
	\item Una raíz $a$ de una función polinómica $f$ se dice que tiene multiplicidad $m$ si $f(x)=(x-a)^m g(x),$ donde $g$ es una función polinómica que no tiene la raíz $a$. Sea $f$ una función polinómica de grado $n$. Supóngase que $f$ tiene $k$ raíces, contando multiplicidades, es decir supóngase que $k$ es la suma de las multiplicidades de todas las raíces. Demostrar que $n-k$ es par.\\\\
	    Demostración.-\; Por la condición dada, $f$ es una función polinómica real de grado $n$ tal que $f$ tiene exactamente $k$ raíces en $\mathbb{R}$ contando multiplicidades. Probaremos que $n-k$ es par. Para ello consideraremos los siguientes casos.\\\\
	    \textbf{Caso 1}.- Si $n=k$ es trivial decir que $n-k=0$ de donde se sabe que es par.\\\\
	    \textbf{Caso 2}.- Si $n>k$, sea $x_1,x_2,\ldots , _m$ raíces reales de $f$ con multiplicidades $k_1,k_2,\ldots, k_m$ respectivamente y por lo tanto,
	    $$k_1 + k_2 + \ldots + k_m = k.$$
	    Entonces $f$ puede ser escrito como,
	    $$f(x) = (x-x_1)^{k_1}(x-x_2)^{k_2}\cdots (x-x_m)^{k_m}p_1(x)p_2(x)\cdots p_i(x)$$
	    donde $p_i(x)$ son polinomios irreducibles en $\mathbb{R}$ tal que el grado de $p_i$  suma $n-k$. Ahora recordemos que todo polinomio irreducible en $\mathbb{R}$ debe tener de grado un entero par. Esto se debe a que cada polinomio de orden impar tiene al menos una raíz real, esto por el teorema 9, por lo tanto $p_i(x)$ no puede ser irreducible en $\mathbb{R}$. Ahora observe que sin pérdida de generalidad hemos asumido que hay $l$ polinomios irreducibles tales que la suma de sus grados $n-k$. Dado que cada uno de los $l$ polinomios tienen grado par, entonces la suma de sus grados debe ser un entero par. Se sigue que $n-k$ es un entero par.\\\\ 

    \end{enumerate}

%-------------------- 5.
\item Supóngase que $f$ es continua en $[a,b]$ y que $f(x)$ es siempre racional. ¿Qué puede decirse acerca de $f$?.\\\\
    Respuesta.-\; $f$ es constante, ya que si $f$ tomara dos valores distintos, entonces $f$ tomaría todos los valores intermedios, incluyendo valores irracionales, es decir, si no fuera constante, entonces existe dos números racionales $r_1$ y $r_2$ tal que para algún $c,d$ se tiene $a\leq c<d\leq$, $f(c)=r_1$ y $f(d) = r_2$. Por el teorema 7.4 en el intervalo $[c,d]$, $f$ toma todos los valores entre $r_1$ y $r_2$, donde se concluye que existe algún número irracional, contradiciendo el hecho de que $f$ solo toma valores racionales.\\\\

%-------------------- 6.
\item Supóngase que $f$ es una función continua en $[-1,1]$ tal que $x^2+f^2(x) = 1$ para todo $x$. (Esto significa que $(x,f(x))$ siempre está sobre el circulo unidad.) Demostrar que o bien es $f(x)=\sqrt{1-x^2}$ para todo $x$, o bien $f(x)=-\sqrt{1-x^2}$ para todo $x$.\\\\
    Demostración.-\; De lo contrario, $f$ toma valores tanto positivos como negativos, por lo que $f$ tendría el valor $0$ en  $(-1, 1)$, lo cual es imposible, ya que $\sqrt{1-x^2} \neq 0$ para $x$ en $(-1,1)$.\\\\ 
 

%-------------------- 7.
\item ¿Cuántas funciones continuas $f$ existen satisfaciendo $f^2(x)=x^2$ para todo $x$?.\\\\
    Respuesta.-\; Existen 4 funciones continuas que satisfacen la condición dada, es decir,
    $$\begin{array}{rcl}
	f(x)&=&x\\
	f(x)&=&-x\\
	f(x)&=&|x|\\
	f(x)&=&-|x|\\\\
    \end{array}$$

%-------------------- 8.
\item Supóngase que $f$ y $g$ son continuas, que $f^2=g^2$, y que $f(x)\neq 0$ para todo $x$. Demostrar que o bien $f(x)=g(x)$ para todo $x$, o bien $f(x)=-g(x)$ para todo $x$.\\\\
    Demostración.-\; Si no fuera así, entonces $f(x)=g(x)$ para algún $x$ y $f(y)=-g(y)$ para algún $y$. Pero ya que $f(x)\neq 0 \; \forall \;x,$ entonces será o bien siempre positiva o bien siempre negativa. Así pues, $g(x)$ y $g(y)$ tendría distinto signo. Esto implicaría que $g(z)=0$ para algún $z$, lo cual es imposible, ya que $0\neq f(z) = \pm g(z).$\\\\

%-------------------- 9.
\item 
    \begin{enumerate}[\bfseries (a)]

	%----------(a)
	\item Supóngase que $f$ es continuo, que $f(x)=0$ solo para $x=a$, y que $f(x)>0$ tanto para algún $x>a$, así como para algún $x<a$. ¿Que puede decirse acerca de $f(x)$ para todo $x\neq a$?.\\\\ 
	    Respuesta.-\; Por hipótesis, existe algún $x_1\in (a,\infty)$ tal que $f(x_1)>0$. Ahora si existe algún $y_i\in (a,\infty)$ con $f(y_1)<0$, entonces debe existir $z_1\in (a,\infty)$ entre $x_1$ y $y_1$ tal que $f(z_1)=0$. Pero esto contradice  que $f$ es cero solo en $x=a$. Por lo tanto, no existe algún $y_1\in (a,\infty)$ con $f(y_1)<0$.\\
	    Esto es, $f(x)>0$ para todo  $x\in(a,\infty).$ Similarmente, $f(x)>0$ para todo  $x\in(-\infty,a).$ Por lo tanto podemos decir que $f(x)>0$ para todo $x\neq a$.\\\\

	%----------(b)
	\item Supongamos ahora que $f$ es continuo y que $f(x)=0$ solo para $x=a$, pero supongamos, en cambio, que $f(x)>0$ para algún $x>a$ y $f(x)<0$ para algún $x<a$. Ahora que puede decir de $f(x)$ para $x\neq a$?.\\\\
	    Respuesta.-\; Por hipótesis, existe algún $x_1\in (a,\infty)$ tal que $f(x_1)>0.$. Ahora si existe $y_1\in (a,\infty)$ con $f(y_1)<0$, entonces existe algún $z_1\in (a,\infty)$ entre $x_1$ y $y_1$ tal que $f(z_1)=0.$ Esto contradice que $f$ es cero sólo en $x=a$. Por lo tanto, no existe algún $y_1\in (a,\infty)$ con $f(y_1)<0.$\\
	    Esto es, $f(x)>0$ para todo $x\in (a,\infty).$ Luego por similar argumento, $f(x)<0$ para todo $x\in (-\infty,a)$. Así, $f(x)>0$ para todo $x>a$ y $f(x)<0$ para todo $x<a$.\\\\

	%----------(c)
	\item Discutir el signo de $x^3+x^2+xy^2+y^3$ cuando $x$ e $y$ no son ambos $0$.\\\\
	    Respuesta.-\; Para $y\neq 0$, sea $f(x)=x^3+x^2y+xy^2+y^3$. Luego
	    $$f(x)=\dfrac{x^4-y^4}{x-y}$$

    \end{enumerate}

%-------------------- 10.
\item Supóngase $f$ y $g$ son continuas en $[a,b]$ y que $f(a)<g(a)$, pero $f(b)>g(b)$. Demostrar que $f(x)=g(x)$ para algún $x$ en $[a,b]$.\\\\
    Demostración.-\; Sea $$h=f-g$$
    entonces por el teorema 1 se tiene $$h(x)=0$$
    por lo que $$f(x)=g(x)\; \mbox{para algún }x \in [a,b].$$\\

%-------------------- 11.
\item Supóngase que $f$ es una función continua en $[0,1]$ y que $f(x)$ es en $[0,1]$ para cada $x$. Demostrar que $f(x)=x$ para algún número $x$.\\\\
    Demostración.-\; Para $f(0)=0$ o $f(1)=1$ entonces se puede elegir $x=0$ o $x=1$. Ya que $x$ es continuo entonces $$g(x)=x-f(x)$$ también es continuo. Luego, por el teorema 1 se tiene,
    $$f(x) - x = 0 \; \Longrightarrow \; x=f(x) \; \mbox{para algún}\; x \in [0,1].$$\\

%-------------------- 12.
\item  
    \begin{enumerate}[\bfseries (a)]

	%----------(a)
	\item El problema 11 muestra que $f$ intersecta la diagonal del cuadrado. Demostrar que $f$ debe cortar a la otra diagonal.\\\\
	    Demostración.-\; Vemos que la linea representa una función $f$ en $[0,1]$, dado por,
	    $$f(x)=x.$$
	    es continuo sobre $[0,1]$. Ahora supongamos una nueva función $g$ en $[0,1]$ tal que
	    $$g(x)=x-f(x)$$
	    de donde,
	    $$\begin{array}{l}
		g(0)=0-f(0)=-f(0)\leq 0\\\\
		g(1)=1-f(1)\\
	    \end{array}$$

	    De las dos funciones anteriores se tiene,
	    $$\begin{array}{l}
		f(0)<g(0)\\\\
		f(1)>g(1)\\
	    \end{array}$$

	    Por último definamos una nueva función continua $h$ de forma que,
	    $$h=f-g$$
	    entonces,
	    $$\begin{array}{l}
		h(1)=f(1)-g(1)>0\\\\
		h(0)=f(0)-g(0)<0\\
	    \end{array}$$
	    
	    Luego, existe algún punto $c$ en $[0,1]$ por lo que ambas curvas es,
	    $$h(c)=0$$
	    y 
	    $$\begin{array}{rcl}
		f(c)-g(c)&=&0\\
		f(c)&=&g(c)\\
	    \end{array}$$

	    Por lo tanto, hay algún $c$ en $[0,1]$ donde $f$ intersecta a la otra linea diagonal.\\\\

	%----------(b)
	\item Demostrar el siguiente hecho más general: Si $g$ es continuo en $[0,1]$ y $g(0)=0$, $g(1)=1$ o $g(0)=1$, $g(1)=0$, entonces $f(x)=g(x)$ para algún $x$.\\\\
	    Demostración.-\;  Sea $f$ en $[0,1]$ entonces
	    $$\begin{array}{l}
		f(0)=0\\\\
		f(1)=1\\
	    \end{array}$$

	    La otra linea punteada representa una función $g$ en $[0,1]$ dada por,
	    $$\begin{array}{l}
		g(0)=0\\\\
		g(1)=1\\
	    \end{array}$$

	    De donde 

	    $$\begin{array}{l}
		f(0)<g(0)\\\\
		f(1)>g(1)\\
	    \end{array}$$

	    Ahora definimos una nueva función continua $h$ de forma que,
	    $$h=f-g$$
	    entonces,
	    $$\begin{array}{l}
		h(1)=f(1)-g(1)>0\\\\
		h(0)=f(0)-g(0)<0\\
	    \end{array}$$
	    
	    Luego, existe algún punto $c$ en $[0,1]$ por lo que ambas curvas es,
	    $$h(c)=0$$
	    y 
	    $$\begin{array}{rcl}
		f(c)-g(c)&=&0\\
		f(c)&=&g(c)\\
	    \end{array}$$

	    Por lo tanto, un continuo $f(g)$ y $g$, existe $f(x)=g(x)$ para algún $x$.\\\\

    \end{enumerate}

%-------------------- 13.
\item 
    \begin{enumerate}[\bfseries (a)]

	%----------(a)
	\item Sea $f(x)=\sen 1/X$ para $x\neq 0$ y sea $f(0)=0,$ ¿Es $f$ continuo en $[-1,1]$?. Demostrar que $f$ satisface la conclusión del teorema de valor intermedio en $[-1,1]$; en otras palabras, si $f$ toma dos valores comprendidos en $[-1,1]$, toma también todos los valores intermedios.\\\\
	    Demostración.-\; Sea la secuencia ${x_n}_n$ en $[-1,1]$ definida por,
	    $$x_ = \dfrac{2}{\pi(4n-3)},\quad n\geq 1$$
	    luego notamos que 
	    $$\lim_{x\to \infty} x_n = 0$$
	    Ahora aplicando la función $f$ se tiene,
	    $$\lim_{n\to \infty}f(x_n) = \lim_{n\to \infty}\sin\left(\dfrac{1}{\frac{2}{\pi(4n-3)}}\right) = \lim_{n\to \infty}\sin\left(\dfrac{\pi(4n-3)}{2}\right) = 1, \quad \forall n\geq 1.$$
	    por lo tanto la función $f$, como tal, no es continuo en $[-1,1]$.\\\\
	    Ahora demostraremos que $f$ satisface el teorema de valor intermedio en $[-1,1]$. Ya que $f(0)=0$ según la hipótesis, podemos decir que $f$ es continuo en $[0,1]$. Vamos a considerar los siguientes casos:\\

	    \begin{enumerate}[\bfseries C1]

		%----------C1
		\item Si $a<b$ son dos puntos de $[-1,1]$ con $a,b>0$ o $a,b<0$, entonces $f$ toma cada valor entre $f(a)$ y $f(b)$ en el intervalo $[a,b]$ ya que $f$ es continuo en $[a,b]$.\\

		%----------C2
		\item Si $a<0<b$, entonces $f$ toma todos los valores entre $-1$ y $1$ en $[a,b]$. \\ 

	    \end{enumerate}

	    Así $f$ tomas todos los valores entre $f(a)$ y $f(b)$. Lo mismo ocurre para $a=0$ o $b=0$.\\\\


	%----------(b)
	\item Supóngase que $f$ satisface la conclusión del teorema del valor intermedio y que $f$ toma cada valor solo una vez. Demostrar que $f$ es continua.\\\\
	Demostración.-\; Si $f$ no fuese continua en $a$, entonces por el problema 6-9(b) para algún $\epsilon>0$ existirían $x$ tan cerca como se quiera de $a$ con $f(x)>f(a)+\epsilon$
	$f(x)<f(a)-\epsilon$. Supongamos que ocurre lo primero. Podemos incluso suponer que existen $x$ tan cerca como se quiera de $a$ y $>a$, o bien tan cerca como se quiera de $a$ y $<a$. Supongamos también aquí lo primero. Tomemos un $x>a$ con $f(x)>f(a)+\epsilon$. Según el teorema de los valores intermedios, existe un $x^{'}$ entre $a$ y $x$ con $f(x^{'})<f(a)+\epsilon$. Pero existe también $y$ entre $a$ y $x^{'}$ con $f(y)<f(a)+\epsilon$. Pero existe también $y$ entre $a$ y $x^{'}$ con $f(y)>f(a)+\epsilon$. Según el teorema de los valores intermedios, $f$ forma el valor $f(a)+\epsilon$ entre $x$ y $x^{'}$ y también entre $x^{'}$ e $y$, contrariamente a la hipótesis.\\\\

	%----------(c)
	\item Generalizar para el caso donde $f$ toma cada valor solo un número finito de veces.\\\\
	    Respuesta.-\; Lo mismo que en $(b)$ elíjase $x_1>a$ con $f(x_1)>f(a)+\epsilon$. Después elijase $x_1^{'}$ entre $a$ y $x_1$ con $f(x_1^{'})<f(a)+\epsilon.$ Luego elíjase $x_2$ entre $a$ y $x_1^{'}$ con $f(x_2)>f(a)+\epsilon$ y $x_2^{'}$ entre $a$ y $x_2$ con $f(x_2^{'})<f(a)+\epsilon$, etc. Entonces $f$ toma el valor $f(a)+\epsilon$ en cada uno de los intervalos $[x_n^{'},x_n]$ en contradicción con la hipótesis.\\\\

    \end{enumerate}

% -------------------- 14.
\item Si $f$ es una función continua en $[0,1]$, sea $\|f\|$ el máximo valor de $|f|$ en $[0,1]$.

    \begin{enumerate}[\bfseries (a)]

	%----------(a)
	\item Demostrar que para algún número $c$ tenemos $\|cf\| = |c|\cdot \|f\|$.\\\\
	    Demostración.-\; Ya que $|cf|=|c|\cdot |f(x)|$ para todo $x$ de $[0,1].$ entonces podemos elegir un $x_0$ tal que $|f|(x_0)=\|f\|$, y por lo tanto $\|cf\| = |c|\cdot \|f\|$.\\\\

	%----------(b)
	\item Demostrar que $\|f+g\|\leq \|f\|+\|g\|.$ Dar un ejemplo donde $\|f+g\|\neq \|f\|+\|g\|.$\\\\
	    Demostración.-\; Para las todos funciones dadas, tenemos
	    $$\begin{array}{rcl}
		|f+g|(x)&=&|f(x)+g(x)|\\
		|f+g|(x)&\leq&|f(x)|+|g(x)|\\
		|f+g|(x)&\leq&|f|(x)+|g|(x)\\
	    \end{array}$$
	    También sabemos que si $f$ o $g$ tienen el máximo valor en $x_0$ entonces,
	    $$\begin{array}{rcl}
		|f|(x_0)&=&\|f\|\\
		|g|(x_0)&=&\|g\|\\
	    \end{array}$$

	    Luego si la función $|f+g|$ tiene el máximo valor en $x_0$ entonces,
	    $$\begin{array}{rcl}
		\|f+g\|&=&|f+g|(x_0)\\
		\|f+g\|&\leq&|f|(x_0) + |g|(x_0)\\
		       \|f+g\|&\leq&\|f\| + \|g\|\\
	    \end{array}$$
	    Por último, sea $f(x)=x+3$ y $g(x)=x-4$ entonces se cumple que $\|f+g\|\neq \|f\|+\|g\|$.\\\\

	%----------(c)
	\item Demostrar que $\|h-f\|\leq \|h-g\|+\|g-f\|$.\\\\
	    Demostración.-\; Para las todos funciones dadas, tenemos
	    $$\begin{array}{rcl}
		|h-f|(x)&=&|(h-g)-(g-f)|(x)\\
		|h-f|(x)&\leq&|(h-g)(x)|+|(g-f)(x)|\\
		|h-f|(x)&\leq&|h-g|(x)+|g-f|(x)\\
	    \end{array}$$
	    También sabemos que si $f$ o $g$ tienen el máximo valor en $x_0$ entonces,
	    $$\begin{array}{rcl}
		|f|(x_0)&=&\|f\|\\
		|g|(x_0)&=&\|g\|\\
		|h|(x_0)&=&\|h\|\\
	    \end{array}$$

	    Luego si la función $|f+g|$ tiene el máximo valor en $x_0$ entonces,
	    $$\begin{array}{rcl}
		\|h-f\|&=&|h-f|(x_0)\\
		\|h-f\|&\leq&|h-g|(x_0) + |g-f|(x_0)\\
		       \|h-f\|&\leq&\|h-g\| + \|g-f\|\\\\
	    \end{array}$$

    \end{enumerate}
   
% -------------------- 15.
\item Supóngase que $\phi$ es continua y $\lim\limits_{x\to\infty} \phi(x)/x^n = 0 = \lim\limits_{x\to -\infty} \phi (x)/x^n$.

    \begin{enumerate}[\bfseries (a)]

	%----------(a)
	\item Demostrar que si $n$ es impar, entonces existe un número $x$ tal que $x^n+\phi(x)=0$.\\\\
	    Demostración.-\; Sea $b>0$ tal que $$\bigg|\dfrac{\phi(b)}{b^n}\bigg|<\dfrac{1}{2}$$
	    entonces,
	    $$b^n+\phi(b)=b^n\left(1+\dfrac{\phi(b)}{b^n}\right)>\dfrac{1}{2}>0$$
	    De la misma manera, sea $a<0$ tal que
	    $$\bigg|\dfrac{\phi(a)}{a^n}\bigg|<\dfrac{1}{2}$$
	    entonces, ya que $n$ es impar,
	    $$a^n + \phi(a)=a^n\left(1+\dfrac{\phi(a)}{a^n}\right)<\dfrac{a^n}{2}<0$$
	    Por lo tanto, existe un $x$ tal que 
	    $$x^n + \phi(x)=0.$$\\

	%----------(b)
	\item Demostrar que si $n$ es par, entonces existe un número $y$ tal que $y^n+\phi(y)\leq x^n + \phi(x)$ para todo $x$.\\\\
	    Demostración.-\; Sea $b>0$ tal que 
	    $$b^n>2\phi(0)$$
	    Y $|x|>b$ 
	    $$\bigg|\dfrac{\phi(x)}{x^n}<\dfrac{1}{2}\bigg|$$
	    de donde tenemos,
	    $$\begin{array}{rcl}
		x^n+\phi(x)&>&x^n\left(1+\dfrac{\phi(x)}{x^n}\right)\\\\
		x^n+\phi(x)&>&\dfrac{x^n}{2}\\\\
		x^n+\phi(x)&>&\dfrac{b^n}{2}\\\\
		x^n+\phi(x)&>&\phi(0).\\\\
	    \end{array}$$

	    Así, el mínimo de $x^n+\phi(x)$ para $x$ in $[-b,b]$ es el mínimo del intevalo.
	    Y por lo tanto, existe un número $y$, para todo $x$, tal que,
	    $$y^n+\phi(y)\leq x^n +\phi(x).$$\\

    \end{enumerate}

% -------------------- 16.
\item 
    \begin{enumerate}[\bfseries (a)]

	%----------(a)
	\item Supóngase que $f$ es continua en $(a,b)$ y $\lim\limits_{x\to a^+} f(x)=\lim\limits_{x\to b^-} f(x)=\infty$. Demostrar que $f$ tiene un mínimo en todo el intervalo $(a,b)$.\\\\
	    Demostración.-\;

    \end{enumerate}

\end{enumerate}
