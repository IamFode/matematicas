\chapter{Tres teoremas fuertes}

% -------------------- teorema 1 --------------------
\begin{tcolorbox}[colframe = white]
    \begin{teo}
	Si $f$ es continua en $[a,b]$ y $f(a)<0<f(b)$ entonces existe algún $x$ en $[a,b]$ tal que $f(x)=0$.\\\\
	Geométricamente, esto significa que la gráfica de una función continua que empieza por debajo del eje horizontal y termina por encima del mismo debe cruzar a este eje en algún punto.
    \end{teo}
\end{tcolorbox}

% -------------------- teorema 2 --------------------
\begin{tcolorbox}[colframe = white]
    \begin{teo}
	Si $f$ es continua en $[a,b]$, entonces $f$ está acotada superiormente en $[a,b]$, es decir, existe algún número $N$ tal que $f(x)\leq N$ para todo $x$ en $[a,b]$.\\\\
	Geométricamente, este teorema significa que la gráfica $f$ queda por debajo de alguna línea paralela al eje horizontal. 
    \end{teo}
\end{tcolorbox}

% -------------------- teorema 3 --------------------
\begin{tcolorbox}[colframe = white]
    \begin{teo}
	Si $f$ es continua en $[a,b]$ entonces existe algún número $y$ en $[a,b]$ tal que $f(y)\leq f(x)$ para todo $x$ en $[a,b]$.\\\\
	Se dice que una función continua en un intervalo cerrado alcanza su valor máximo en dicho intervalo.
    \end{teo}
\end{tcolorbox}

% -------------------- teorema 4 --------------------
\begin{teo}
    Si $f$ es continua en $[a,b]$ y $f(a)<c<f(b)$, entonces existe algún $x$ en $[a,b]$ tal que $f(x)=c$.\\\\
    Demostración.-\; 
\end{teo}








\section{Problemas}

\begin{enumerate}[\Large\bfseries 1.]

%-------------------- 1.
\item Para cada una de las siguientes funciones, decidir cuáles está acotadas superiormente o inferiormente en el intervalo indicado, y cuáles de ellas alcanzan sus valores máximo y mínimo.

    \begin{enumerate}[\bfseries (i)]

	\item $f(x) = x^2$ en $(-1,1)$.

    \end{enumerate}

\end{enumerate}
