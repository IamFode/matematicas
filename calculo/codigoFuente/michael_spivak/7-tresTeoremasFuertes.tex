\chapter{Tres teoremas fuertes}

% -------------------- teorema 1 --------------------
\begin{tcolorbox}[colframe = white]
    \begin{teo}
	Si $f$ es continua en $[a,b]$ y $f(a)<0<f(b)$ entonces existe algún $x$ en $[a,b]$ tal que $f(x)=0$.\\\\
	Geométricamente, esto significa que la gráfica de una función continua que empieza por debajo del eje horizontal y termina por encima del mismo debe cruzar a este eje en algún punto.
    \end{teo}
\end{tcolorbox}

% -------------------- teorema 2 --------------------
\begin{tcolorbox}[colframe = white]
    \begin{teo}
	Si $f$ es continua en $[a,b]$, entonces $f$ está acotada superiormente en $[a,b]$, es decir, existe algún número $N$ tal que $f(x)\leq N$ para todo $x$ en $[a,b]$.\\\\
	Geométricamente, este teorema significa que la gráfica $f$ queda por debajo de alguna línea paralela al eje horizontal. 
    \end{teo}
\end{tcolorbox}

% -------------------- teorema 3 --------------------
\begin{tcolorbox}[colframe = white]
    \begin{teo}
	Si $f$ es continua en $[a,b]$ entonces existe algún número $y$ en $[a,b]$ tal que $f(y)\leq f(x)$ para todo $x$ en $[a,b]$.\\\\
	Se dice que una función continua en un intervalo cerrado alcanza su valor máximo en dicho intervalo.
    \end{teo}
\end{tcolorbox}

% -------------------- teorema 4 --------------------
\begin{teo}
    Si $f$ es continua en $[a,b]$ y $f(a)<c<f(b)$, entonces existe algún $x$ en $[a,b]$ tal que $f(x)=c$.\\\\
    Demostración.-\; Sea $g=f-c$. Entonces $g$ es continua, y $\; g(a)+c < c < g(b) + c \; \Longrightarrow \; g(a)<0<g(b)$. Por el teorema 1, existe algún $x$ en $[a,b]$ tal que $g(x)=0$. Pero esto significa que $f(x)=c.$\\\\
\end{teo}

% -------------------- teorema 5 --------------------
\begin{teo}
    Si $f$ es continua en $[a,b]$ y $f(a)>c>f(b)$, entonces existe algún $x$ en $[a,b]$ tal que $f(x)=c$.\\\\
    Demostración.-\; La función $-f$ es continua en $[a,b]$ y $-f(a)<-c<-f(b)$. Por el teorema 4 existe algún $x$ en $[a,b]$ tal que $-f(x)=-c$, lo que significa que $f(x)=c$.\\\\
\end{teo}

Si una función continua en un intervalo toma dos valores, entonces toma todos los valores comprendidos entre ellos; esta ligera generalización del teorema 1 recibe a menudo el nombre de \textbf{teorema de los valores intermedios}.\\\\ 

%---------------------- teorema 6 ----------------------
\begin{teo}
    Si $f$ es continua en $[a,b]$, entonces $f$ es acotada inferiormente en $[a,b]$, es decir, existe algún número $N$ tal que $f(x)\geq N$ para todo $x$ en $[a,b]$.\\\\
    Demostración.-\; La función $-f$ es continua en $[a,b]$, así por el teorema 2 existe un número $M$ tal que $-f(x)\leq M$ para todo $x$ en $[a,b]$. Pero esto significa que $f(x)\geq -M$ para todo $x$ en $[a,b]$, así podemos poner $N=-M$.\\\\
\end{teo}

Los teoremas 2 y 6 juntos muestran  que una función continua $f$ en $[a,b]$ son acotados en $[a,b]$, es decir, existe un número $N$ tal que $|f(x)|\leq N$ para todo $x$ en $[a,b]$. En efecto, puesto que el teorema 2 asegura la existencia de un número $N_1$ tal que $f(x)\leq N,$ para todo $x$ de $[a,b]$ y el teorema 6 asegura la existencia de un número $N_2$ tal que $f(x)\geq N$, para todo $x$ en $[a,b]$, podemos tomar $N=\max(|N_1|,|N_2|)$.\\\\

%---------------------- teorema 7 ----------------------
\begin{teo}
    Si $f$ es continua en $[a,b]$, entonces existe algún $y$ en $[a,b]$ tal que $f(y)\leq f(x)$ para todo $x$ en $[a,b]$.\\\\
    Demostración.-\; La función $-f$ es continua en $[a,b]$; por el teorema 3 existe algún $y$ en $[a,b]$ tal que $-f(y)\geq -f(x)$ para todo $x$ en $[a,b]$, lo que significa que $f(y)\leq f(x)$ para todo $x$ en $[a,b]$.\\\\
\end{teo}

%---------------------- teorema 8 ----------------------
\begin{teo}
    Todo número positivo posee una raíz cuadrada. En otras palabras, si $\alpha > 0$, entonces existe algún número $x$ tal que $x^2=\alpha$.\\\\
    Demostración.-\; Consideremos la función $f(x)=x^2$, el cual es ciertamente continuo. Notemos que la afirmación del teorema puede ser expresado en términos de $f$: $"$el número $\alpha$ posee una raíz cuadrada$"$ significa que $f$ toma el valor $alpha$. La demostración de este hecho acerca de $f$ será una consecuencia fácil del teorema 4.\\
    Existe, evidentemente, un número $b>0$ tal que $f(b)>\alpha$; en efecto, si $\alpha>1$ podemos tomar $b=\alpha$, mientras que si $\alpha<1$ podemos tomar $b=1$. Puesto que $f(0)<\alpha <f(b)$, el teorema 4 aplicado a $[0,b]$ implica que para algún $x$ de $[0,b]$, tenemos $f(x)=\alpha$, es decir, $x^2=\alpha$.\\\\
    Precisamente el mismo raciocinio puede aplicarse para demostrar que todo número positivo tiene una raíz n-ésima, cualquiera que sea el número $n$. Si $n$ es impar, se puede decir mas: todo número tiene una raíz n-ésima. Para demostrarlo basta observar que si el número positivo $\alpha$ tiene la raíz n-ésima $x$, es decir, si $x^n=\alpha$, entonces $(-x)^n=-\alpha$ (puesto que $n$ es impar), de modo que $\alpha$ tiene una raíz n-ésima $-\alpha$. Afirmar que, para un $n$ impar, cualquier número $\alpha$ tiene una raíz n-ésima equivale a afirmar que la ecuación
    $$x^n - \alpha = 0$$
    tiene una raíz si $n$ es impar. El resultado expresado de este modo es susceptible de gran generalización.\\\\
\end{teo}

%---------------------- teorema 9 ----------------------
\begin{teo}
    Si $n$ es impar, entonces cualquier ecuación 
    $$x^n + a_{n-1}x^{n-1}+\ldots + a_0 = 0$$
    posee raíz.\\\\
	Demostración.-\; Tendremos que considerar, evidentemente, la función 
	$$f(x) = x^n + a_{n-1}x^{n-1} + \ldots + a_0$$
	habría que demostrar que $f$ es unas veces positiva y otras veces negativa. La idea intuitiva es que para un $|x|$ grande, la función se parece mucho más a $g(x)=x^n$ y puesto que $n$ es impar, ésta función es positiva para $x$ grandes positivos y negativos para $x$ grandes negativos. Un poco de cálculo algebraico es todo lo que hace falta para dar formar a esta idea intuitiva.\\
	Para analizar debidamente la función $f$ conviene escribir 
	$$f(x) = x^n + a_{n-1}x^{n-1}+\ldots + a_0 = x^n\left(1+\dfrac{a_{n-1}}{x} + \ldots + \dfrac{a_0}{x^n}\right)$$
	obsérvese que
	$$\bigg| \dfrac{a_{n-1}}{x} + \dfrac{a_{n-2}}{x^2} + \ldots + \dfrac{a_0}{x^n} \bigg|\leq \dfrac{|a_{n-1}|}{|x|} + \ldots + \dfrac{|a_0|}{|x^n|}$$
	En consecuencia, si elegimos un $x$ que satisfaga
	$$|x|>1,2n|a_{n-1}|,\ldots,2n|a_0|\qquad (*)$$
	entonces $|x^k| > |x|$ y 
	$$\dfrac{|a_{n-k}}{x^k}<\dfrac{a_{n-k}}{|x|}<\dfrac{|a_{n-k}|}{2n|a_{n-k}}=\dfrac{1}{2n}$$
	de modo que 
	$$\bigg| \dfrac{a_{n-1}}{x} + \dfrac{a_{n-2}}{x^2} + \ldots + \dfrac{a_0}{x^n} \bigg|\leq \dfrac{1}{2n}+\ldots + \dfrac{1}{2n}=\dfrac{1}{2}$$
	expresado de otra forma,
	$$-\dfrac{1}{2}\leq \dfrac{a_{n-1}}{x}+\ldots + \dfrac{a_0}{x_n}\leq \dfrac{1}{2},$$
	lo cual implica que 
	$$\dfrac{1}{2}\leq 1 + \dfrac{a_{n-1}}{x}+\ldots + \dfrac{a_0}{x_n}..$$
	Por lo tanto, si elegimos un $x_1>0$ que satisfaga $(*)$, entonces 
	$$\dfrac{x_1^n}{2} \leq x_1^n \left(1+\dfrac{a_{n-1}}{x_1}+\ldots + \dfrac{a_0}{x^n}\right) = f(x_1)$$
	así que $f(x_1)>0$. Por otro lado, si $x_2<0$ satisface $(*)$, entonces $x_2^n < 0$ y 
	$$\dfrac{x_2^n}{2}\geq x_2^n \left(1+\dfrac{a_{n-1}}{x_2}+\ldots + \dfrac{a_0}{x_2^n}\right)=f(x_2),$$
	así $f(x_2)<0$. \\
	Ahora aplicando el teorema 1 para el intervalo $[x_2,x_1]$ llegamos a la conclusión de que existe un $x$ en $[x_2,x_1]$ tal que $f(x)=0$.\\\\

\end{teo}

%---------------------- teorema 10 ----------------------
\begin{teo}
    Si $n$ es par y $f(x)=x^n + a_{n-1} x^{n-1} + \ldots + a_0$, entonces existe un número $y$ tal que $f(y)=f(x)$ para todo $x$.\\\\
	Demostración.-\; Lo mismo que en el teorema 9, si 
	$$M = \max(1,2n|a_{n-1}|,\ldots , 2n|a_0|),$$
	entonces para todo $x$ con $|x|\geq M,$ tenemos
	$$\dfrac{1}{2}\leq 1 + \dfrac{a_{n-1}}{x}+\ldots + \dfrac{a_0}{x^n}$$
	Al ser $n$ par, $x^n \geq 0$ para todo $x$, de modo que 
	$$\dfrac{x^n}{2}\leq x^n \left(1+\dfrac{a_{n-1}}{x} + \ldots + \dfrac{a_0}{x^n}\right) = f(x),$$
	siempre que $|x|\geq M$. Consideremos ahora el número $f(0)$. Sea $b>0$ un número tal que $b^n \geq 2f(0)$ y también $b>M$. Entonces si $x\geq b,$ tenemos 
	$$f(x)\geq \dfrac{x^n}{2}\geq \dfrac{b^n}{2}\geq f(0).$$
	Análogamente, si $x\leq -b$, entonces
	$$f(x)\geq \dfrac{}{}\geq \dfrac{}{} = \dfrac{}{}\geq f(0).$$
	Resumiendo ahora el teorema 7 a la función $f$ en el intervalo $[-b,b]$. Se deduce que existe un número $y$ tal que
	$$(1)\qquad \mbox{si}\; -b\leq x\leq b, \; \mbox{entonce}\; f(y)\leq f(x).$$
	En particular, $f(y)\leq f(0).$ De este modo
	$$(2)\qquad \mbox{si}\; x\leq -b \; \mbox{o}\; x\geq b,\; \mbox{entonces}\; f(x)\geq f(0)\geq f(y).$$
	Cambiando (1) y (2) vemos que $f(y)\leq f(x)$ para todo $x$.\\\\
\end{teo}

%---------------------- teorema 11 ----------------------
\begin{teo}
    Consideremos la ecuación
    $$(*)\qquad x^n + a_{n-1}x^{n-1}+ \ldots + a_0 = c,$$
    y supongamos que $n$ es par. Entonces existe un número $m$ tal que $(*)$ posee una solución para $c\geq m$ y no posee ninguna para $c<m.$\\\\
    Demostración.-\; Sea $f(x) = x^n + a_{n-1}x^{n-1}+\ldots + a_0$
    Según el teorema 10, existe un número $y$ tal que $f(y)\leq f(x)$ para todo $x$.\\
    Sea $m=f(y)$. Si $c<m$ entonces la ecuación $(*)$ no tiene, evidentemente, ninguna solución, puesto que el primer miembro tiene un valor $\geq m$. Si $c=m$ entonces $(*)$ tiene $y$ como solución. Finalmente, supongamos $c>m$. Sea $b$ un número tal que $b>y$, $f(b)>c$. Entonces $f(y)=m<c<f(b)$. En consecuencia, según el teorema 4, existe algún número $x$ en $[y,b]$ tal que $f(x)=c$, con lo que $x$ es una solución de $(*).$\\\\
\end{teo}





\section{Problemas}

\begin{enumerate}[\Large\bfseries 1.]

%-------------------- 1.
\item Para cada una de las siguientes funciones, decidir cuáles está acotadas superiormente o inferiormente en el intervalo indicado, y cuáles de ellas alcanzan sus valores máximo y mínimo.

    \begin{enumerate}[\bfseries (i)]

	%----------(i)
	\item $f(x) = x^2$ en $(-1,1)$.\\\\
	    Respuesta.-\; Se encuentra acotada superior como inferiormente. El mínimo es $0$ e no tiene máximo.\\\\

	%----------(ii)
	\item $f(x) = x^3$ en  $(-1,1)$.\\\\
	    Respuesta.-\; Se encuentra acotada superior como inferiormente. No tiene máximo ni mínimo\\\\

	%----------(iii)
	\item $f(x) = x^2$ en $\mathbf{R}$.\\\\
	    Respuesta.-\; No está acotado superior pero si inferiormente. Su mínimo es $0$ y no tiene  máximo.\\\\ 

	%----------(iv)
	\item $f(x)=x^2$ en $[0,\infty)$.\\\\
	    Respuesta.-\; Está acotada inferiormente pero no así superiormente. Su mínimo es $0$ y no tiene máximo.\\\\ 

	%----------(v)
	\item $f(x) = \left\{\begin{array}{ll} x^2, & x\leq a \\ a+2, & x\geq a \end{array}\right.$ en $(-a-1,a+1)$\\\\
	    Respuesta.-\; Es acotado superior e inferiormente. Se entiende que $a>-1$ (de modo que $-a-1<a+1$). Si $-1<a\leq 1/2$, entonces $a<-a-1$, así $f(x)=a+2$ para todo $x$ en $(-a-1,a+1)$, por lo tanto $a+2$ es el máximo y mínimo valor. Si $-1/2<a\leq 0$, entonces $f$ tiene el mínimo valor en $a^2$, y si $a\geq 0$, entonces $f$ tiene un mínimo valor en $0$. Ya que $a+2>(a+1)^2$ solo para $[-1-\sqrt{5}]/2 < a < [1+\sqrt{5}]/2$, cuando $a\geq -1/2$ésta función $f$ tiene un máximo valor solo para $a\leq [1+\sqrt{5}]/2$ (el máximo valor será $a+2$).\\\\

	%----------(vi)
	\item $f(x) = \left\{\begin{array}{ll}x^2, & x<a \\ a+2, & x\geq a \end{array}\right.$ en $[-a-1,a+1].$\\\\
	    Respuesta.-\; Está acotado superior e inferiormente. Como en la parte (v), se asume que $a>-1$. Si $a\leq -1/2$ entonces $f$ tiene el valor mínimo y un máximo $3/2$. Si $a\geq 0$, entonces $f$ tiene un valor mínimo en $0$, y un valor máximo $max(a^2,a+2)$. Si $-1/2<a<0$ , entonces $f$ tiene un máximo valor $3/2$ y no así con un valor mínimo.\\\\

	%----------(vii)
	\item $f(x) = \left\{\begin{array}{ll} 0,& x\; \mbox{irracional}  \\  1/q,& x=p/q \; \mbox{fracción irreducible} \end{array}\right.$ en $[0,1].$\\\\
	    Respuesta.-\; Acotada superior e inferiormente. El mínimo es $0$ y el máximo es $1$.\\\\

	%----------(viii)
	\item $f(x) = \left\{\begin{array}{rcl}1, & x \; \mbox{irracional} // 1/q, & x = p/q \; \mbox{fracción irreducible}\end{array}\right.$ en $[0,1]$.\\\\
	    Respuesta.-\; Acotada superior e inferiormente.  El máximo es $1$ y no existe un mínimo.\\\\

	%----------(ix)
	\item 

    \end{enumerate}

\end{enumerate}
