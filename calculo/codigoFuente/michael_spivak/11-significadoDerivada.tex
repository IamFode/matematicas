\chapter{Significado de la derivada}

\begin{tcolorbox}
    \begin{def.}
	Sea $f$ una función y $A$ un conjunto de números contenido en el dominio de $f$. Un punto $x$ de $A$ es un punto máximo de $f$ en $A$ si
	$$f(x)\geq f(y)\qquad \mbox{para todo} \; y \; \mbox{de}\; A.$$
	El número $f(x)$ se denomina el \textbf{valor máximo} de $f$ en $A$ (y también diremos que $f$ alcanza su valor máximo en el punto $x$ de $A$).\\\\
	$f$ tiene un \textbf{mínimo} en el punto $x$ de $A$ si $-f$ tiene un máximo en el punto $x$ de $A$.
    \end{def.}
\end{tcolorbox}

En general, nos interesará el caso en que $A$ es un intervalo cerrado $[a,b];$ si $f$ es continua, entonces el Teorema 7-3 garantiza que $f$ alcanza realmente dicho valor máximo en $[a,b].$\\

Ahora ya estamos en condiciones para enunciar un teorema que ni siquiera depende de la existencia de cotas superiores mínimas.\\

\begin{teo}
    Sea $f$ cualquier función definida en $(a,b)$. Si $x$ es un punto máximo (o mínimo) de $f$ en $(a,b)$ y $f$ es diferenciable en $x$, entonces $f'(x)=0.$ (Observemos que no hemos supuesto la diferenciabilidad, ni siquiera la continuidad, de $f$ en otros puntos.)\\\\
	Demostración.-\; Consideremos el caso en que $f$ tiene un máximo en $x$. Si $h$ es cualquier número tal que $x+h$ pertenece a $(a,b)$, entonces
	$$f(x)\geq f(x+h),$$
	ya que $f$ tiene un máximo en el punto $x$ de $(a,b)$. Esto significa que 
	$$f(x+h)-f(x)\leq 0.$$
	De manera que, si $h>0$ tenemos
	$$\dfrac{f(x+h)-f(x)}{h}\leq 0,$$
	y por tanto
	$$\lim_{h\to 0^+}\dfrac{f(x+h)-f(x)}{h}\leq 0.$$
	Por otra parte, si $h<0$, tenemos
	$$\dfrac{f(x+h)-f(x)}{h}\geq 0,$$
	o sea
	$$\lim_{h\to 0^-}\dfrac{f(x+h)-f(x)}{h}\geq 0.$$
	Por hipótesis, $f$ es diferenciable en $x$, de manera que ambos límites deben ser iguales (de hecho son iguales a $f'(x)$). Esto significa que
	$$f'(x)\leq 0\quad \mbox{y}\quad f'(x)\geq 0,$$
	de lo cual se deduce que $f'(x)=0.$\\
\end{teo}

\begin{tcolorbox}
    \begin{def.}
	Sea $f$ una función, y $A$ un conjunto de números contenido en el dominio de $f$. Un punto $x$ de $A$ es un \textbf{punto máximo [mínimo] local} de $f$ en $A$ si existe algún $\delta>0$ tal que $x$ es un punto máximo [mínimo] de $f$ en $A\cap (x-\delta,x+\delta.)$
    \end{def.}
\end{tcolorbox}

\begin{teo}
    Si $x$ es un máximo o mínimo local de $f$ en $(a,b)$ y $f$ es diferenciable en $x$, entonces $f'(x)=0$.\\\\
	Demostración.-\; Se trata de una aplicación del teorema 1 (capítulo 11, Spivak).
\end{teo}

El recíproco del teorema 2 no es cierto; la condición $f'(0)$ no implica que $x$ sea un punto máximo o mínimo local en $f$. Precisamente por esta razón, se ha adoptado una terminología especial para describir a aquellos números $x$ que satisfacen la condición $f'(0)$.

\begin{tcolorbox}
    \begin{def.}
	Un \textbf{punto critico} de una función $f$ es un número $x$ tal que 
	$$f'(x)=0.$$
	Al número $f(x)$ se le denomina \textbf{valor critico} de $f$.
    \end{def.}
\end{tcolorbox}

Consideremos en primer lugar el problema de hallar el máximo o el mínimo de $f$ en un intervalo cerrado $[a,b]$. (En este caso, si $f$ es continua, sabemos que dicho valor máximo y mínimo debe existir.) Para localizarlos, deben considerarse tres clases de puntos:

\begin{enumerate}[(1)]
    \item Los puntos críticos de $f$ en $[a,b]$.
    \item Los puntos extremos $a$ y $b$.
    \item Aquellos puntos $x$ de $[a,b]$ tales que $f$ no es diferenciable en $x$.
\end{enumerate}

Si $x$ no pertenece al segundo no al tercer grupo entonces forzosamente debe pertenecer al primero.\\


\begin{obs}
    En el capítulo 7 ya resolvimos el problema de este tipo cuando demostramos que si $n$ es par, la función
    $$f(x)=x^n+a_{n-1}x^{n-1}+\ldots + a_0$$
    tiene un valor mínimo en toda la recta real. Dicho valor mínimo se puede encontrar resolviendo la ecuación, si es posible, y comparando los valores de $f(x)$ en dichos $x$.\\
\end{obs}

\begin{teo}[Teorema de Rolle]
    Si $f$ es continua en $[a,b]$ y diferenciable en $(a,b)$, y $f(a)=f(b)$, entonces existe un número $x$ en $(a,b)$ tal que $f'(x)=0$.\\\\
	Demostración.-\; A partir de la continuidad en $f$ en $[a,b]$ deducimos que $f$ tiene valor máximo y mínimo en $[a,b]$. Supongamos primero que el valor máximo se presenta en un punto $x$ de $(a,b)$. Entonces $f'(x)=0$ según el teorema 1, y la demostración queda completa. Supongamos ahora que el valor mínimo de $f$ se presenta en algún punto $x$ de $(a,b)$. Entonces, de nuevo $f'(x)=0$ según el teorema 1. Finalmente, supongamos que los valores máximo y mínimo se presentan ambos en los extremos del intervalo. Como $f(a)=f(b),$ dichos valores coinciden, de manera que $f$ es una función constante, y en este caso se puede elegir cualquier valor $x$ de $(a,b)$.
\end{teo}

\begin{teo}[Teorema del valor medio]
    Si $f$ es continua en $[a,b]$ y diferenciable en $(a,b)$, existe un número $x$ en $(a,b)$ tal que 
    $$f'(x)=\dfrac{f(b)-f(a)}{b-a}.$$\\
	Demostración.-\; Sea 
	$$h(x)=f(x)-\left[\dfrac{f(b)-f(a)}{b-a}\right](x-a).$$
	Evidentemente, $h$ es continua en $[a,b]$ y diferenciable en $(a,b)$, y 
	$$h(a)=f(a),\qquad h(b)=f(b)-\left[\dfrac{f(b)-f(a)}{b-a}\right](b-a)=f(a).$$
	Por tanto, se puede aplicar el teorema de Rolle a la función $h$ y deducir que existe algún $x$ en $(a,b)$ tal que
	$$0=h'(x)=f'(x)-\dfrac{f(b)-f(a)}{b-a},$$
	de modo que 
	$$f'(x)=\dfrac{f(b)-f(a)}{b-a}.$$\\
\end{teo}

\begin{cor}
    Si $f$ está definida en un intervalo y $f'(x)=0$ en todo $x$ del intervalo, entonces $f$ es constante en dicho intervalo.\\\\
	Demostración.-\; Sean $a$ y $b$ dos puntos del intervalo con $a\neq b$. Entonces existe algún $x$ de $(a,b)$ tal que 
	$$f'(x)=\dfrac{f(b)-f(a)}{b-a}.$$
	Pero $f'(x)=0$ para todo $x$ del intervalo, por tanto
	$$0=\dfrac{f(b)-f(a)}{b-a},$$
	y por consiguiente $f(a)=f(b)$. Así pues, el valor de $f$ en dos puntos cualesquiera del intervalo es el mismo, lo cual significa que $f$ es constante en el intervalo.\\\\
\end{cor}

\begin{cor}
    Si $f$ y $g$ están definidas en el mismo intervalo y $f'(x)=g'(x)$ para todo $x$ del intervalo, entonces existe algún número $c$ tal que $f=g+c.$\\\\
	Demostración.-\; Para todo $x$ del intervalo se verifica que $(f-g)'(x)=f'(x)-g'(x)=0,$ de manera que, según el corolario 1, existe un número $c$ tal que $f-g=c.$
\end{cor}

\begin{tcolorbox}
    \begin{def.}
	Una función es \textbf{creciente} en un intervalo si $f(a)<f(b)$ siendo $a$ y $b$ dos números del intervalo con $a<b$. La función $f$ es \textbf{decreciente} en un intervalo si $f(a)>f(b)$ para todo $a$ y $b$ del intervalo con $a<b$. (A menudo se dice simplemente que $f$ es creciente o decreciente, en cuyo caso se deduce que el intervalo es el dominio de $f$.)
    \end{def.}
\end{tcolorbox}

\begin{cor}
    Si $f'(x)>0$ para todo $x$ de un intervalo, entonces $f$ es creciente en dicho intervalo; si $f'(x)<0$ para todo $x$ del intervalo, entonces $f$ es decreciente en dicho intervalo.\\\\
	Demostración.-\; Consideremos el caso en que $f'(x)>0$. Sean $a$ y $b$ dos puntos del intervalo con $a<b$. Entonces existe algún punto $x$ en $(a,b)$ que verifica
	$$f'(x)=\dfrac{f(b)-f(a)}{b-a}.$$
	Pero $f'(x)>0$ para todo $x$ en $(a,b)$, por tanto 
	$$\dfrac{f(b)-f(a)}{b-a}>0.$$
	Como $b-a>0$ se deduce que $f(b)>f(a).$\\\\
	Consideremos ahora el caso en que $f'(x)<0$. Sean $a$ y $b$ dos puntos del intervalo con $a<b$. Entonces existe algún punto $x$ en $(a,b)$ que verifica
	$$f'(x)=\dfrac{f(b)-f(a)}{b-a}.$$
	Pero $f'(x)<0$ para todo $x$ en $(a,b)$, por tanto 
	$$\dfrac{f(b)-f(a)}{b-a}<0.$$
	De donde se deduce que $f(b)<f(a).$\\
\end{cor}

Podemos dar un esquema general para decidir si un punto crítico es un máximo local, un mínimo local o ninguna de las dos cosas:

\begin{enumerate}[(1)]
    \item Si $f'>0$ en algún intervalo a la izquierda de $x$ y $f'<0$ en algún intervalo a la derecha de $x$, entonces $x$ es un punto máximo local.
    \item  Si $f'<0$ en algún intervalo a la izquierda de $x$ y $f'>0$ en algún intervalo a la derecha de $x$, entonces $x$ es un punto mínimo local.
    \item Si $f'$ tiene el mismo signo en algún intervalo a la izquierda de $x$ que en algún intervalo a la derecha, entonces $x$ no es ningún punto máximo ni mínimo local.
\end{enumerate}

En varios problemas de este capítulo y de capítulos sucesivos se pide hacer una representación gráfica de funciones. En cada caso debe determinar

\begin{enumerate}[(1)]
    \item los puntos críticos de $f$,
    \item el valor de $f$ en los puntos críticos,
    \item el signo de $f'$ en las regiones entre los puntos críticos (si esto no está claro ya),
    \item los números $x$ tales que $f(x)=0$ (si es posible),
    \item el comportamiento de $f(x)$ cuando $x$ se hace grande o grande negativo (si es posible).
\end{enumerate}

Existe un criterio popular para hallar los máximos y mínimos locales, que depende del comportamiento de la función sólo en los puntos críticos.\\

\begin{teo}
    Supongamos que $f'(a)=0.$ Si $f''(a)>0,$ entonces $f$ tiene un mínimo local en $a$; si $f''(a)<0,$ entonces $f$ tiene un máximo local en $a.$\\\\
	Demostración.-\;
\end{teo}




\section{Ejercicios}

\begin{enumerate}[\bfseries 1.]

    %-------------------- 1.
    \item Para cada una de las siguientes funciones, halle los valores máximos y mínimos en los intervalos indicados, determinando aquellos puntos del intervalo en los que la derivada es igual a $0$ y comparando los valores de la función en estos puntos con sus valores en los extremos del intervalo.
	\begin{enumerate}[(i)]

	    %---------- (i)
	    \item $f(x)=x^3-x^2-8x+1$ en $[-2,2]$.\\\\
		Respuesta.-\; Primeramente derivemos la función $f$.
		$$f'(x)=3x^2-2x-8.$$
		Luego igualemos a cero para hallar el grupo de candidatos para localizar el o los puntos máximos y mínimos.
		$$3x^2-2x-8=0 \quad \Rightarrow \quad x_1=2,\quad x_2=-\dfrac{4}{3}$$
		Ambos número $x_1=2$ y $x_2=-\dfrac{4}{3}$ pertenecen al intervalo $[-2,2]$, de manera que el primer grupo de candidatos para localizar el máximo y el mínimo es
		$$x_1=2,\qquad x_2=-\dfrac{4}{3}$$
		El segundo grupo incluye a los extremos del intervalo. Es decir,
		$$-2,2$$
		El tercer grupo es vacío, ya que $f$ es diferenciable en todas partes. Por último calculamos 
		$$\begin{array}{ccccl}
		    f\left(2\right) &=& 2^3-2^2-8\cdot 2+1&=&-11.\\\\
		    f\left(-\dfrac{4}{3}\right) &=& \left(-\dfrac{4}{3}\right)^3-\left(-\dfrac{4}{3}\right)^2-8\cdot \left(-\dfrac{4}{3}\right)+1&=&\dfrac{203}{27}.\\\\
		    f\left(-2\right) &=& -2^3-2^2-8\cdot (-2)+1&=&5.\\\\
		\end{array}$$
		Por lo tanto el mínimo viene dado por $-11$ y el máximo viene dado por $\dfrac{203}{27}.$\\\\

	    %---------- (ii)
	    \item $f(x)=x^5+x+1$ en $[-1,1]$.\\\\
		Respuesta.-\; 
		Respuesta.-\; Primeramente derivemos la función $f$.
		$$f'(x)=5x^4+1.$$
		Luego igualemos a cero para hallar el grupo de candidatos para localizar el o los puntos máximos y mínimos.
		$$5x^4+1=0 \quad \Rightarrow \quad x^4=-\dfrac{1}{5}.$$
		El cual no es posible para ningún $x$ real.\\\\
		El segundo grupo incluye a los extremos del intervalo. Es decir,
		$$-1,1$$
		El tercer grupo es vacío, ya que $f$ es diferenciable en todas partes. Por último calculamos 
		$$\begin{array}{ccccl}
		    f\left(1\right) &=& 1^5+1+1&=&3.\\\\
		    f\left(-1\right) &=& (-1)^5-1+1 &=&-1.\\\\
		\end{array}$$
		Por lo tanto el mínimo viene dado por $-1$ y el máximo viene dado por $3.$\\\\

	    %---------- (iii)
	    \item $f(x)=3x^4-8x^3+6x^2$ en $\left[-\frac{1}{2},\frac{1}{2}.\right]$\\\\\
		Respuesta.-\; Primeramente derivemos la función $f$.
		$$f'(x)=4x^3-24x^2+12x.$$
		Luego igualemos a cero para hallar el grupo de candidatos para localizar el o los puntos máximos y mínimos.
		$$12x^3-24x^2+12x=0 \quad \Rightarrow \quad x(x^2-2x+1)=0\quad \Rightarrow \quad \left\{\begin{array}{rcl}x_1&=&0\\ x_2 &=& 1. \end{array}\right.$$
		Sólo el número $x_1=0$  pertenece al intervalo $[-\frac{1}{2},\frac{1}{2}]$, de manera que el primer grupo de candidatos para localizar el máximo y el mínimo es sólo el número:
		$$x_1=0.$$
		El segundo grupo incluye a los extremos del intervalo. Es decir,
		$$-\dfrac{1}{2},\;\dfrac{1}{2}.$$
		El tercer grupo es vacío, ya que $f$ es diferenciable en todas partes. Por último calculamos 
		$$\begin{array}{ccccl}
		    f\left(-\dfrac{1}{2}\right) &=& 3\left(-\dfrac{1}{2}\right)^4-8\left(-\dfrac{1}{2}\right)^3+ 6\left(-\dfrac{1}{2}\right)^2 &=&\dfrac{43}{16}.\\\\
		    f\left(0\right) &=& 3\cdot 0^4-8\cdot 0^3 + 6\cdot 0^2&=&0.\\\\
		    f\left(\dfrac{1}{2}\right) &=& 3\left(\dfrac{1}{2}\right)^4-8\left(\dfrac{1}{2}\right)^3+ 6\left(\dfrac{1}{2}\right)^2 &=&\dfrac{11}{16}.\\\\
		\end{array}$$
		Por lo tanto el mínimo viene dado por $0$ y el máximo viene dado por $\dfrac{43}{16}.$\\\\

	    %---------- (iv)
	    \item $f(x)=\dfrac{1}{x^5+x+1}$ en $\left[-\dfrac{1}{2},1\right]$.\\\\
		Respuesta.-\; Primeramente derivemos la función $f$.
		$$f'(x)=5x^4+1.$$
		Luego igualemos a cero para hallar el grupo de candidatos para localizar el o los puntos máximos y mínimos.
		$$35x^4+1=0 \quad \Rightarrow \quad x_4=-\dfrac{1}{5}.$$
		El cual no es posible para ningún $x$ real.\\\\
		El segundo grupo incluye a los extremos del intervalo. Es decir,
		$$-\dfrac{1}{2},\;1$$
		El tercer grupo es vacío, ya que $f$ es diferenciable en todas partes. Por último calculamos 
		$$\begin{array}{ccccl}
		    f\left(-\dfrac{1}{2}\right) &=& \dfrac{1}{\left(-\dfrac{1}{2}\right)^5+\left(-\dfrac{1}{2}\right)+1} &=&\dfrac{32}{15}.\\\\
		    f\left(1\right) &=& \dfrac{1}{1^5+1+1} &=&\dfrac{1}{3}.\\\\
		\end{array}$$
		Por lo tanto el mínimo viene dado por $\dfrac{32}{15}$ y el máximo viene dado por $\dfrac{1}{3}.$\\\\

	    %---------- (v)
	    \item $f(x)=\dfrac{x+1}{x^2+1}$ en $\left[-1,\dfrac{1}{2}\right]$.\\\\
		Respuesta.-\; Primeramente derivemos la función $f$.
		$$f'(x)=\dfrac{x^2+1-2(x+1)}{\left(x^2+1\right)^2}.$$
		Luego igualemos a cero para hallar el grupo de candidatos para localizar el o los puntos máximos y mínimos.
		$$\dfrac{x^2+1-2(x+1)}{\left(x^2+1\right)^2}=0 \quad \Rightarrow \quad x^2-2x-1=0\quad \Rightarrow \quad \left\{\begin{array}{rcl}x_1&=&-1+\sqrt{2}\\ x_2 &=& -1-\sqrt{2}. \end{array}\right.$$
		Sólo el número $x_1=-1+\sqrt{2}$ pertenece al intervalo $[-1,\frac{1}{2}]$, de manera que el primer grupo de candidatos para localizar el máximo y el mínimo es sólo el número:
		$$x_1=-1+\sqrt{2}.$$
		El segundo grupo incluye a los extremos del intervalo. Es decir,
		$$-1,\;\dfrac{1}{2}.$$
		El tercer grupo es vacío, ya que $f$ es diferenciable en todas partes. Por último calculamos 
		$$\begin{array}{ccccl}
		    f\left(-1\right) &=& \dfrac{-1+1}{\left(-1\right)^2+1}&=&0.\\\\
		    f\left(1+\sqrt{2}\right) &=& \dfrac{\left(-1+\sqrt{2}\right)+1}{\left(-1+\sqrt{2}\right)^2+1} &=&\dfrac{\sqrt{2}}{2\left(\sqrt{2}+2\right)}.\\\\
		    f\left(\dfrac{1}{2}\right) &=& \dfrac{\left(\dfrac{1}{2}\right)+1}{\left(\dfrac{1}{2}\right)^2+1} &=&\dfrac{6}{5}.\\\\
		\end{array}$$
		Por lo tanto el mínimo viene dado por $0$ y el máximo viene dado por $\dfrac{6}{5}.$\\\\

	    %---------- (vi)
	    \item $f(x)=\dfrac{x}{x^2-1}$ en $[0,5]$.\\\\
		Respuesta.-\; Primeramente derivemos la función $f$.
		$$f'(x)=-\dfrac{x^2+1}{\left(x^2-1\right)^2}$$
		Luego igualemos a cero para hallar el grupo de candidatos para localizar el o los puntos máximos y mínimos.
		$$-\dfrac{x^2+1}{\left(x^2-1\right)^2}=0 \quad \Rightarrow \quad x^2+1=0\quad  \Rightarrow \quad  x^2=-1.$$
		El cual no es posible para ningún $x$ real.\\\\
		El segundo grupo incluye a los extremos del intervalo. Es decir,
		$$0,\;5$$
		El tercer grupo es vacío, ya que $f$ es diferenciable en todas partes. Por último calculamos 
		$$\begin{array}{ccccl}
		    f\left(0\right) &=& \dfrac{0}{0^2-1} &=&0.\\\\
		    f\left(5\right) &=& \dfrac{5}{5^2-1} &=&\dfrac{5}{24}.\\\\
		\end{array}$$
		Por lo tanto el mínimo viene dado por $0$ y el máximo viene dado por $\dfrac{5}{24}.$\\\\

	\end{enumerate}

    %------------------ 2.
    \item Trace ahora la gráfica de cada una de las funciones del Problema 1 (spivak, capítulo 11.) y halle todos los puntos máximos y mínimos locales.\\\\
	Respuesta.-\;

\end{enumerate}
