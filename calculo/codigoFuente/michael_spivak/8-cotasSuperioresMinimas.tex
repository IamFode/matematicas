\chapter{Cotas superiores mínimas}

%------------------- Definición 8.1
\begin{tcolorbox}
    \begin{def.}
	Un conjunto $A$ de números reales está acotado superiormente si existe un número $x$ tal que 
	\begin{center}
	    $x\geq a$ para todo $a$ de $A$.
	\end{center}
	este número $x$ se denomina una cota superior de $A$.\\\\
    $A$ está acotado superiormente si y sólo si existe un número $x$ que es una cota superior de $A$.(y en este caso existirán muchas cotas superiores de $A$);\\\\
    \end{def.}
\end{tcolorbox}

%------------------- Definición 8.2
\begin{tcolorbox}
    \begin{def.}
	Un número $x$ es una cota superior mínima de $A$ si
	\begin{enumerate}[\bfseries (1)]
	    \item $x$ es una cota superior de $A$,
	    \item si $y$ es una cota superior de $A$, entonces $x\leq y$.
	\end{enumerate}
	El término \textbf{supremo} de $A$ es sinónimo al de cota superior mínima y tiene una ventaja: se puede abreviar mediante un símbolo muy adecuado
	$$\sup A$$
    \end{def.}
\end{tcolorbox}
\vspace{0.2cm}

    Si $x$ e $y$ son ambos cotas superiores mínimas de $A$, entonces $x=y$.\\\\
	Demostración.-\; En efecto, en este caso
	\begin{center}
	\begin{tabular}{ll}
	    $x\leq y,$ & ya que $y$ es una cota superior, y $x$ es una cota superior mínima,\\
	    $y\leq x$ & ya que $x$ es un cota superior, e $y$ es una cota superior mínima.
	\end{tabular}
	\end{center}
	por lo tanto, $x=y$. \\\\

%------------------- Definición 8.3
\begin{tcolorbox}
    \begin{def.}
	Un conjunto $A$ de números reales está acotado inferiormente si existe un número $x$ tal que
	\begin{center}
	    $x\leq a$ para todo $a$ de $A$.
	\end{center}
	Dicho número $x$ se denomina una cota inferior de $A$.
    \end{def.}
\end{tcolorbox}

\begin{tcolorbox}
    \begin{def.}
	Un número $x$ es la cota inferior máxima de $A$ si
	\begin{enumerate}[\bfseries (1)]
	    \item $x$ es una cota inferior de $A$, y
	    \item si $y$ es una cota inferior de $A$, entonces $x\geq y$.
	\end{enumerate}
	La cota inferior máxima de $A$ se denomina también el \textbf{ínfimo} de $A$, abreviadamente
	$$\inf A$$
    \end{def.}
\end{tcolorbox}

%------------------- P13
\begin{tcolorbox}
\begin{prop}[Propiedad de la cota superior mínima]
    Si $A$ es un conjunto de números reales, $A\neq \emptyset$, y $A$ está acotado superiormente, entonces $A$ posee una cota superior mínima. 
\end{prop}
\end{tcolorbox}

El enorme significado de P13 se hará patente sólo de manera gradual, aunque ya estamos en condiciones de comprobar su importancia dando las demostraciones que omitimos en el Capítulo 7.\\


\setcounter{chapter}{7}
\setcounter{teo}{0}
\begin{teo}
    Si $f$ es continua en $[a,b]$ y $f(a)<0<f(b)$, entonces existe algún número $x$ de $[a,b]$ tal que $f(x)=0.$\\\\
	Demostración.-\; 
\end{teo}

\setcounter{chapter}{8}
\setcounter{teo}{0}

\begin{teo}
\end{teo}
