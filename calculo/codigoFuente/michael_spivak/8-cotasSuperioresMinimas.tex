\chapter{Cotas superiores mínimas}

%------------------- Definición 8.1
\begin{tcolorbox}
    \begin{def.}
	Un conjunto $A$ de números reales está acotado superiormente si existe un número $x$ tal que 
	\begin{center}
	    $x\geq a$ para todo $a$ de $A$.
	\end{center}
	este número $x$ se denomina una cota superior de $A$.\\\\
    $A$ está acotado superiormente si y sólo si existe un número $x$ que es una cota superior de $A$.(y en este caso existirán muchas cotas superiores de $A$);\\\\
    \end{def.}
\end{tcolorbox}

%------------------- Definición 8.2
\begin{tcolorbox}
    \begin{def.}
	Un número $x$ es una cota superior mínima de $A$ si
	\begin{enumerate}[\bfseries (1)]
	    \item $x$ es una cota superior de $A$,
	    \item si $y$ es una cota superior de $A$, entonces $x\leq y$.
	\end{enumerate}
	El término \textbf{supremo} de $A$ es sinónimo al de cota superior mínima y tiene una ventaja: se puede abreviar mediante un símbolo muy adecuado
	$$\sup A$$
    \end{def.}
\end{tcolorbox}
\vspace{0.2cm}

    Si $x$ e $y$ son ambos cotas superiores mínimas de $A$, entonces $x=y$.\\\\
	Demostración.-\; En efecto, en este caso
	\begin{center}
	\begin{tabular}{ll}
	    $x\leq y,$ & ya que $y$ es una cota superior, y $x$ es una cota superior mínima,\\
	    $y\leq x$ & ya que $x$ es un cota superior, e $y$ es una cota superior mínima.
	\end{tabular}
	\end{center}
	por lo tanto, $x=y$. \\\\

%------------------- Definición 8.3
\begin{tcolorbox}
    \begin{def.}
	Un conjunto $A$ de números reales está acotado inferiormente si existe un número $x$ tal que
	\begin{center}
	    $x\leq a$ para todo $a$ de $A$.
	\end{center}
	Dicho número $x$ se denomina una cota inferior de $A$.
    \end{def.}
\end{tcolorbox}

\begin{tcolorbox}
    \begin{def.}
	Un número $x$ es la cota inferior máxima de $A$ si
	\begin{enumerate}[\bfseries (1)]
	    \item $x$ es una cota inferior de $A$, y
	    \item si $y$ es una cota inferior de $A$, entonces $x\geq y$.
	\end{enumerate}
	La cota inferior máxima de $A$ se denomina también el \textbf{ínfimo} de $A$, abreviadamente
	$$\inf A$$
    \end{def.}
\end{tcolorbox}

%------------------- P13
\begin{tcolorbox}
\begin{prop}[Propiedad de la cota superior mínima]
    Si $A$ es un conjunto de números reales, $A\neq \emptyset$, y $A$ está acotado superiormente, entonces $A$ posee una cota superior mínima. 
\end{prop}
\end{tcolorbox}

El enorme significado de P13 se hará patente sólo de manera gradual, aunque ya estamos en condiciones de comprobar su importancia dando las demostraciones que omitimos en el Capítulo 7.\\


\setcounter{chapter}{7}
\setcounter{teo}{0}
\begin{teo}
    Si $f$ es continua en $[a,b]$ y $f(a)<0<f(b)$, entonces existe algún número $x$ de $[a,b]$ tal que $f(x)=0.$\\\\
	Demostración.-\; La demostración es tan sólo una versión rigurosa del método esbozado al final del capítulo 7: localizaremos el menor número $x$ de $[a,b]$ tal que $f(x)=0$.\\
	Definamos el conjunto $A$, de la manera siguiente:

	$$A=\lbrace x:a\leq x \leq b, \; \mbox{y}\; f \; \mbox{negativa en en el intervalo}\; [a,x]\rbrace.$$

	Obviamente $A\neq \emptyset$ ya que $a$ pertenece a $A$; de hecho, existe un $\delta >0$  tal que $A$ contiene a todos los puntos $x$ que satisfacen $a\leq x < a+\delta$ ya que $f$ es continua en $[a,b]$ y $f(a)<0$. Análogamente, $b$ es una cota superior de $A$ y, de hecho, existe un $\delta>0$ tal que todos los puntos $x$ que satisfacen $b-\delta<x\leq b$ son cotas superiores de $A$; esto también se deduce del Problema 6-16, ya que $f(b)>0$.\\
	A partir de estas observaciones se deduce que $A$ posee cota superior mínima $\alpha$ y que $a<\alpha<b$. Ahora demostraremos que $f(\alpha)=0$, excluyendo las posibilidades $f(\alpha)<0$ y $f(\alpha)>0$.\\
	Supongamos primero que $f(\alpha)<0$. Según el teorema 6-3, existe un $\delta>0$ tal que $f(x)<0$ si $\alpha- \delta<x<\alpha+\delta$. Ha de existir un número $x_0$ de $A$ que satisface $\alpha-\delta<x_0<\alpha$ (ya que sino $\alpha$ no sería la mínima cota superior de $A$). Esto significa que $f$ es negativa en todo el intervalo $[a,x_0]$. Pero si $x_1$ es un número situado entre $\alpha$ y $\alpha+\delta$, entonces $f$ también es negativa en todo el intervalo a $A$. Pero esto contradice el hecho de que $\alpha$ sea una cota superior de $A$; concluimos pues que la suposición que hemos hecho anteriormente, de que $f(\alpha)<0$ sebe ser falsa.\\
	Supongamos ahora que $f(\alpha)>0$. Entonces existe un número $\delta>0$ tal que $f(x)>0$ si $\alpha-\delta<x<\alpha+\delta$. Una vez más, sabemos que existe un $x_0$ de $A$ que satisface $\alpha-\delta<x_0<\alpha$; pero esto significa que $f$ es negativa en $[a,x_0]$, lo cual es imposible ya que $f(x_0)>0$. Así pues, la suposición de que $f(\alpha)>0$ conduce a una contradicción, quedando sólo la posibilidad de que $f(\alpha)=0$.\\
\end{teo}

Las demostraciones de los teoremas 2 y 3 del capítulo 7 requieren un sencillo resultado preliminar, que va a desempeñar una función muy similar a la del teorema 6-3 en la demostración anterior.\\\\

\setcounter{chapter}{8}
\setcounter{teo}{0}

\begin{teo}
    Si $f$ es continua en $a$, existe un número $\delta>0$ tal que $f$ está acotada superiormente en el intervalo $(a-\delta,a+\delta)$.\\\\
    Demostración.-\; Como el $\lim\limits_{x\to a} f(x) =f(a)$, para cada $\epsilon>0$, existe un $\delta>0$ tal que, para todo $x$,
    \begin{center}
	si $|x-a|<\delta$, entonces $|f(x)-f(a)|<\epsilon,$
    \end{center}
    Tan sólo es necesario aplicar esta propiedad a algún $\epsilon$ en particular, por ejemplo $\epsilon=1$. Deducimos pues que existe un $\delta>0$ tal que, para todo $x$,

    \begin{center}
	si $|x-a|<\delta$, entonces $|f(x)-f(a)|<1,$
    \end{center}

    y, en particular, si $|x-a|<\delta$ entonces $f(x)-f(a)<1$. Esto completa la demostración: en el intervalo $(a-\delta,a+\delta)$ la función $f$ está acotada superiormente por $f(a)+1.$\\\\
\end{teo}

Por supuesto, ahora podríamos demostrar también que $f$ está acotada inferiormente en algún intervalo $(a-\delta,a+\delta)$, concluyendo, por tanto, que $f$ está acotada en algún intervalo abierto que contiene a $a$.\\
En este sentido, cabe destacar en particular la observación de que si $\lim\limits_{x\to a^{+}},$ entonces existe un $\delta>0$ tal que $f$ está acotada en el conjunto $\lbrace x:a\leq x < a+\delta\rbrace$, pudiendo hacerse una observación análoga si $\lim\limits_{x\to b^-}=f(b)$.\\\\

\setcounter{chapter}{7}
\setcounter{teo}{1}
\begin{teo}
    Si $f$ es continua en $[a,b]$, entonces $f$ está acotada superiormente en $[a,b]$.\\\\
	Demostración.-\; Sea,
	$$A=\lbrace x:a\leq x \leq b \;\mbox{y}\; f \; \mbox{está acotada superiormente en}\; [a,x]\rbrace$$
	Obviamente $A\neq \emptyset$ (ya que $a$ pertenece a $A$), y está acotada superiormente por $b$, de manera que $A$ posee una cota superior mínima $\alpha$. Observemos que estamos aplicando el término acotado superiormente tanto al conjunto $A$, localizado en el eje horizontal, como a la función $f$, es decir, a conjuntos del tipo $\lbrace f(y):a\leq y \leq x\rbrace$, localizados en el eje vertical.\\
	La primera etapa de la demostración consiste en probar que $\alpha=b$. Supongamos, por el contrario, que $a<b$. Según el teorema 1 existe un $\delta >0$ tal que $f$ está acotada en $(a-\delta,a+\delta)$. Como $\alpha$ es la cota superior mínima de $A$ existe algún $x_0$ de $A$ que satisface $\alpha-\delta<x_0<\alpha$. Esto significa que $f$ está acotada en $[a,x_0]$. Pero si $x_1$ es cualquier número tal que $\alpha<x_1<\alpha+\delta$, entonces $f$ también está acotada en $[x_0,x_1]$. Por lo tanto $f$ está acotada en $[a,x_1]$, de manera que $x_1$ pertenece a $A$, lo que contradice el hecho de que $\alpha$ sea una cota superior a $A$. Esta contradicción demuestra que $\alpha=b$. Debemos mencionar un detalle: en la demostración hemos puesto implícitamente que $a<\alpha$ de manera que $f$ está definida en algún intervalo $(\alpha-\delta,\alpha+\delta)$; la posibilidad $a=\alpha$ puede excluirse de manera similar, utilizando el hecho de que existe un $\delta>0$ tal que $f$ está acotada en $\lbrace x:a\leq x < a + \delta\rbrace$.\\
	La demostración todavía no es completa; únicamente sabemos que $f$ está acotada en $[a,x]$ para todo $x<b$, no necesariamente que $f$ está acotada en $[a,b]$. Sin embargo, sólo es necesario añadir una pequeña observación.\\
	Existe un $\delta>0$ tal que $f$ está acotada en $\lbrace x:b-\delta < x \leq b \rbrace$. Existe también un $x_0$ de $A$ tal que $b-\delta<x_0<b$. De manera que $f$ está acotada en $[a,x_0]$ y también en $[x_0,b],$ por tanto $f$ está acotada en $[a,b]$.\\\\
\end{teo}

%-------------------- teorema 7.3
\begin{teo}
    Si $f$ es continua en $[a,b]$, entonces existe un número $y$ de $[a,b]$ tal que $f(y)\geq f(x)$ para todo $x$ de $[a,b]$.\\\\
	Demostración.-\; Sabemos que $f$ está acotada en $[a,b]$, lo que significa que le conjunto
	$$\lbrace f(x):x \mbox{ pertenezca a }[a,b]\rbrace$$
	está acotado. Además, dicho conjunto es, obviamente, distinto del $\emptyset$, de manera que admite una cota superior mínima $\alpha$. Como $\alpha\geq f(x)$ para $x$ de $[a,b]$, basta demostrar que $\alpha=f(y)$ para algún $y$ de $[a,b]$.\\
	Supongamos, por el contrario, que $\alpha\neq f(y)$ para todo $y$ de $[a,b]$. Entonces la función $g$ definida mediante
	$$g(x)=\dfrac{1}{\alpha-f(x)},\quad x\in [a,b]$$
	es continua en $[a,b]$, ya que el denominador de la expresión del lado derecho de la igualdad nunca vale $0$. Por otra parte, $\alpha$ es la mínima cota superior de $\lbrace f(x):x \mbox{ pertenezca a }[a,b] \rbrace$, esto significa que 
	\begin{center}
	    para cada $\epsilon > 0$ existe un $x$ de $[a,b]$ con $\alpha - f(x)<\epsilon$.
	\end{center}
	Esto, a su vez, significa que 
	\begin{center}
	    para cada $\epsilon > 0$ existe un $x$ de $[a,b]$ con $g(x)>1/\epsilon$.
	\end{center}
	Pero esto quiere decir que $g$ no está acotada en $[a,b]$, lo que contradice el resultado del teorema anterior.\\\\
\end{teo}

\setcounter{chapter}{8}
\setcounter{teo}{1}
%-------------------- teorema 8.2
\begin{teo}
    $\mathbb{N}$ no está acotado superiormente.\\\\
	Demostración.-\; Supongamos que $\mathbb{N}$ estuviera acotado superiormente. Como $\mathbb{N}=\emptyset$, existiría una cota superior mínima $\alpha$ de $\mathbb{N}$. Entonces
	\begin{center}
	    $\alpha\geq n$ para todo $n$ de $\mathbb{N}.$
	\end{center}
	Por consiguiente,
	\begin{center}
	    $\alpha\geq n+1$ para todo $n$ de $\mathbb{N},$
	\end{center}
	ya que si $n$ pertenece a $\mathbb{N}$, $n+1$ también. Pero esto significa que 
	\begin{center}
	    $\alpha-1\geq n$ para todo $n$ de $\mathbb{N},$
	\end{center}
	lo cual quiere decir que $\alpha-1$ también es una cota superior de $\mathbb{N}$, lo que contradice el hecho de que $\alpha$ sea la cota superior mínima de $\mathbb{N}$.\\\\
\end{teo}

% ------------------- teorema 8.3
\begin{teo}
    Para cualquier $\epsilon>0$ existe un número natural $n$ con $1/n<\epsilon$.\\\\
    Demostración.-\; Supongamos que no fuese así; entonces $1/n\geq \epsilon$ para todo $n$ de $\mathbb{N}$. Por tanto, $n\leq 1/\epsilon$ para todo $n$ de $\mathbb{N}.$ Pero esto significa que $1/\epsilon$ es una cota superior de $\mathbb{N},$ lo cual contradice el resultado del teorema 8.2.\\\\
\end{teo}


\section{Problemas}

\begin{enumerate}[\bfseries 1.]

    %-------------------- 1.
    \item Hallar la cota superior mínima y la cota inferior máxima (si existe) de los siguientes conjuntos. Decida también cuáles de ellos poseen un elemento máximo y un elemento mínimo (es decir, en qué casos la cota superior mínima y cota inferior máxima pertenecen al conjunto).

	\begin{enumerate}[\bfseries (i)]

	    %---------- (i)
	    \item $\left\{ \dfrac{1}{n}: n\mbox{ en }\mathbb{N}\right\}$.\\\\
		Respuesta.-\; Sea $A$ el conjunto dado. Vemos que el $\sup A = \max A = 1$, $\inf A = 0$ y $\min A$ no existe.\\\\

	    %---------- (ii)
	    \item $\left\{ \dfrac{1}{n}: n \mbox{ en } \mathbb{Z} \mbox{ y } n\neq 0\right\}$.\\\\
		Respuesta.-\; Sea $A$ el conjunto dado. Podemos ver que $\sup A = \max A = 1$ y $\inf A  = \min A = -1$.\\\\

	    %---------- (iii)
	    \item $\left\{ x:x=0 \mbox{ o } x=1/n \mbox{ para algún } n \mbox{ en } \mathbb{N}\right\}$.\\\\
		Respuesta.-\;Llamemos $A$ al conjunto dado. Todos los números están en el intervalo $[0,1]$. De donde el $0$ es el mayor límite inferior y el $1$ es el menor límite superior, es decir,
		$\sup A = \max A = 1$ y $\inf A = \min A = 0$.\\\\

	    %---------- (iv)
	    \item $\left\{ x:0\leq x \leq \sqrt{2} \mbox{ y } x \mbox{ racional }\right\}$.\\\\
		Respuesta.-\; Sea $A$ el conjunto dado. En este caso $0$ es el mayor límite inferior y está contenido en el conjunto y $\sqrt{2}$ es el límite superior mínimo pero no está en el conjunto, por lo tanto, $\inf A = \min A = 0$ y $\sup A = \sqrt{2}.$\\\\

	    %---------- (v)
	    \item $\left\{ x:x^2+x+1\geq 0\right\}$.\\\\
		Respuesta.-\; No existe ni máximo ni mínimo, tampoco supremo o ínfimo.\\\\

	    %---------- (vi)
	    \item $\left\{ x:x^2+x-1<0\right\}$.\\\\
		Respuesta.-\; Sea $A=\left\{ x:x^2+x-1<0\right\}$, entonces se tiene $$x^2+x-1<0 \; \Leftrightarrow \; \dfrac{-1-\sqrt{5}}{2}<x<\dfrac{-1+\sqrt{5}}{2}$$. Por lo tanto $\sup A = \dfrac{-1-\sqrt{5}}{2}$ y $A=\dfrac{-1+\sqrt{5}}{2}$ los dos no contenidos en $A$.\\\\ 

	    %---------- (vii)
	    \item $\left\{ x:x<0\mbox{ y } x^2+x-1<0\right\}$.\\\\
		Respuesta.-\; Designemos al conjunto dado con $A$. Luego ya que 
		$$x<0 \mbox{ y } x^2+x-1<0\; \Leftrightarrow \; \dfrac{-1-\sqrt{5}}{2}<x<0$$
		Entonces $\sup A = 0 \notin A,\; \inf A = \dfrac{-1-\sqrt{5}}{2}\notin A.$\\\\

	    %---------- (viii)
	    \item $\left\{ \dfrac{1}{n}+(-1)^n:n \mbox{ en } \mathbb{N}\right\}$.\\\\
		Respuesta.-\; Sea $A$ el conjunto designado y sea $a_n = \dfrac{1}{n}+(-1)^n$. Entonces $a_1=0$, para indices pares $a_n=\dfrac{1}{n}+1.$ La sucesión es decreciente, converge en $1$ y el mayor elemento se obtiene en $n=2$, $a_2=\dfrac{3}{2}$. Para indices impares $a_n=\dfrac{1}{n}-1$, la secuencia es decreciente, converge en $-1$ pero este número no se consigue, por lo que 
		$$-1<a_n\leq \dfrac{3}{2}.$$
		De donde concluimos que $\inf A = -1$ pero no existe el mínimo y $\sup A \max A = \dfrac{3}{2}.$\\\\

	\end{enumerate}

    %------------------- 2.
    \item 

\end{enumerate}

