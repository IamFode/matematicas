\chapter{Distintas clases de números}
  \section{Problemas}
    \begin{enumerate}[\bfseries 1.]
      %---------------------------------------1-------------------------------------
      \item Demostrar por inducción las siguientes fórmulas: 
        %------------------------(a)-----------------------------
        \begin{enumerate}[\bfseries (i)]
          \item $1^2+...+n^2=\dfrac{n(n+1)(2n+1)}{6}$\\\\
          Demostración.- \; Sea $n=k$: $$1^2+...+k^2=\dfrac{k(k+1)(2k+1)}{6},$$ Para $k=1$, $$1^2=\dfrac{1(1+2)(2+1)}{6}$$ por lo tanto se cumple para $k=1$, Luego para $k=k+1$, $$1^2+...+(k+1)^2=\dfrac{(k+1)(k+2)(2k+3)}{6},$$ así cabe demostrar que:
            \begin{center}
              \begin{tabular}{r c l}
              $\dfrac{k(k+1)(2k+1)}{6}+(k+1)^2$&=&$\dfrac{(k+1)(k+2)(2k+3)}{6}$\\\\
              $\dfrac{2k^3+k^2+2k^2+k+6k^2+12k+6}{6}$&=&$\dfrac{2k^3+3k^2+6k^2+9k+4k+6}{6}$\\\\
              $\dfrac{2k^3+9k^2+13k+6}{6}$&=&$\dfrac{2k^3+9k^2+13k+6}{6}$\\\\
              \end{tabular}
            \end{center}

          %-----------------------(b)---------------------------------
          \item $1^3 + ... + n^3 = (1+...+n)^2$\\\\
          Demostración.- \; Sea $n=1$ entonces la igualdad es verdadera ya que $1^3 = 1^2$.Supongamos que se cumple para algún número $k \in \mathbb{Z}^+$,
          $$1^3 + ... + k^3 = (1+...+k)^2,$$ Luego suponemos que se cumple para $k+1$, $$1^3 + ... + k^3 + (k+1)^3 = (1+...+(k+1))^2$$
          Así solo falta demostrar que 
            \begin{center}
              \begin{tabular}{rcl}
              $(1+...+(k+1))^2$&$=$&$(1+...+k)^2 + 2(1+...+k)(k+1) + (k+1)^2$\\\\
              &$=$&$(1^2 + ... + k)^2 + 2\dfrac{k(k+1)}{2} (k+1) + (k+1)^2$\\\\
              &$=$&$1^3 + ... + k^3 + (k^3 + 2k^2 + k) + (k^2 + 2k +1)$\\\\
              &$=$&$1^3 + ... + k^3 + (k+1)^3$\\\\
              \end{tabular}
            \end{center}
          Por lo tanto es válido para cualquier $n \in \mathbb{Z}^+$\\\\
        \end{enumerate}

      %-----------------------------------------2-----------------------------------------
     \item Encontrar una fórmula para 
	 \begin{enumerate}[\bfseries (i)]
	%-----------------------(i)---------------------------
	 \item  $\displaystyle\sum_{i=1}^{n} (2i-1) = 1 +3 +5 + ... + (2n-1)$\\
            \begin{center}
              \begin{tabular}{r c c c l}
              1&=&1&=&$1^2$\\
              1+3&=&4&=&$2^2$\\
              1+3+5&=&9&=&$3^2$\\
              1+3+5+7&=&16&=&$4^2$\\
              1+3+5+7+9&=&25&=&$5^2$\\
              \end{tabular}
            \end{center}
          Por lo tanto $ \displaystyle\sum_{i=1}^{n} (2i-1) = 1 +3 +5 + ... + (2n-1) = n^2$\\\\

          %------------------------(ii)---------------------------
          \item $\displaystyle\sum_{i=1}^{n} (2i-1)^2 = 1^2 + 3^2 + 5^2 + ... + (2n-1)^2$
            \begin{center}
              \begin{tabular}{r c l l}
              $1^2 + 3^2 + 5^2 + ... + (2n-1)^2$&$=$&$\left[ 1^2 +2^2 +...+(2n)^2 \right] - \left[ 2^2 + 4^2 +6^2 +...+ (2n)^2\right]$&\\\\
              &$=$&$\left[ 1^2 + 2^2 + ...+ (2n)^2 \right] - 4\left[ 1^2 + 2^2 +3^2 + ... + n^2 \right]$&\\\\
              &$=$&$\dfrac{2n(2n+1)(4n+1)}{6} -\dfrac{4n(n+1)(2n+1)}{6}$&\\\\
              &$=$&$\dfrac{2n(2n+1)\left[ 4n+1 -2 (n+1) \right]}{6}$&\\\\
              &$=$&$\dfrac{2n(2n+1)(2n-1)}{6}$&\\\\ 
              &$=$&$\dfrac{n(2n+1)(2n-1)}{3}$&\\\\
              \end{tabular}
            \end{center}
        \end{enumerate}

      %---------------------------------------------3------------------------------------------
      \begin{tcolorbox}
        \begin{def.}[Coeficiente Binomial]
          Si $0 \leq k \leq n,$ se define el coeficiente binomial $ {n \choose k} $ por $${n \choose k} = \dfrac{n!}{k!(n-k)!}=\dfrac{n(n-1)...(n - k + 1)}{k!}, \; si \; k \neq 0, \; n$$ $${n \choose 0} = {n \choose n} = 1.$$ Esto se convierte en un caso particular de la primera fórmula si se define $0! = 1.$
        \end{def.}
      \end{tcolorbox}

      \item 
        \begin{enumerate}[\bfseries (a)]
          %------------------------(a)---------------------------
          \item Demostrar que $${n +1 \choose k} = {n \choose k - 1} + {n \choose k}$$ Esta relación de lugar a la siguiente configuración, conocida por triángulo de Pascal: Todo número que no esté sobre uno de los lados es la suma de los dos números que tiene encima: El coeficiente binomial ${n \choose k}$ es el número k-ésimo de la fila $(n+1)$.
            \begin{center}
              \begin{tabular}{ccccccccccc}
                &    &    &    &    &  1 &    &    &    &    &   \\
                &    &    &    &  1 &    &  1 &    &    &    &   \\
                &    &    &  1 &    &  2 &    &  1 &    &    &   \\
                &    &  1 &    &  3 &    &  3 &    &  1 &    &   \\
                &  1 &    &  4 &    &  6 &    &  4 &    &  1 &   \\
              1 &    &  5 &    & 10 &    & 10 &    &  5 &    & 1 \\\\
              \end{tabular}
            \end{center}
          Demostración.- \; \\
            \begin{center}
              \begin{tabular}{r c l}
                $ {n \choose k-1}  +  {n \choose k} $&$=$&$\dfrac{n!}{(k-1)(n-k+1)!}+ \dfrac{n!}{k!(n-k)!}$\\\\
                &$=$&$\dfrac{kn!}{k!(n+1-k)!} + \dfrac{(n+1-k)n!}{k!(n+1-k)!}$\\\\
                &$=$&$\dfrac{(n+1)n!}{k!(n+1-k)!}$\\\\
                &$=$&$ {n+1 \choose k} $\\\\
              \end{tabular}
            \end{center}

          %------------------------(b)---------------------------
          \item Obsérvese que todos los números del triángulo de Pascal son números naturales. Utilícese la parte $(a)$ para demostrar por inducción que $ {n \choose k}$ es siempre un número natural.\\\\
          Demostración.- \; Se ve claramente que ${1 \choose 1}$ es un número natural. Supóngase que ${n \choose p}$ es un número natural para todo $p \leq n$. Al ser: $${ n+1 \choose p } = {n \choose p-1} + {n \choose p} \; para \; p \leq n,$$ se sigue que ${n+1 \choose p}$ es un número natural para todo $p \leq n,$ mientras que ${n+1 \choose n+1}$ es también un número natural. Así pues, ${n+1 \choose p}$ es un número natural para todo $p \leq n+1$\\\\

          %------------------------(c)---------------------------
          \item Dése otra demostración de que ${n \choose k}$ es un número natural, demostrando que ${n \choose k}$ es el número de conjuntos de exactamente $k$ enteros elegidos cada uno entre $1,...,n$.\\\\
          Demostración.- \; Existen $n(n -1) \cdot ... \cdot (n - k + 1)$ k-tuplas de enteros distintos elegidos entre $1, ..., n$, ya que el primero puede ser elegido de n maneras, el segundo de $n - 1$ maneras, etc. Ahora bien, cada conjunto formado exactamente por $k$ enteros distintos, da lugar a $k!$ k-tuplas, de
          modo que el número de conjuntos será $n(n — 1) \cdot ... \cdot (n - k + l)/k! = {n\choose k}$\\\\ 

          %------------------------(d)---------------------------
          \item Demostrar el \textbf{TEOREMA DEL BINOMIO}: Si $a$ \; y \; $b$ son números cualesquiera, entonces
          $$(a+b)^n = a^n + {n \choose 1} a^{n-1} b + {n \choose 2} a^{n-2} b^2 + ... + {n \choose n-1} a b^{n-1} + b^n = \displaystyle \sum_{j=0}^n {n \choose j} a^{n-j} b^j.$$\\\\
          El teorema del binomio resulta claro para $n=1.$ Sea algún número $k \in \mathbb{Z}^+$  
          $$(a+b)^k =  \sum\limits_{j=0}^n {n \choose j} a^{n-j} b^j.$$
          de donde  suponemos que se cumple  para $k+1$ por lo tanto $$(a+b)^k =  \sum\limits_{j=0}^{n+1} {n+1 \choose j} a^{n+1-j} b^j.$$
          entonces,
            \begin{center}
              \begin{tabular}{r c l l}
                $(a+b)^{n+1}$&=&$(a+b)(a+b)^n$&\\\\
                &=&$(a+b) \displaystyle \sum_{j=0}^n {n \choose j} a^{n-j} b^j$&\\\\
                &=&$\displaystyle \sum_{j=0}^n {n \choose j} a^{n+1-j} b^j + \sum_{j=0}^{n} {n \choose j} a^{n-j} b^{j+1}$&\\\\
                &=&$\displaystyle \sum_{j=0}^n {n \choose j} a^{n+1-j} b^j + \sum_{j=0}^{n+1} {n \choose j-1} a^{n+1-j} b^{j}$&\\\\
                &=&$\displaystyle \sum_{j=0}^{n+1} {n+1 \choose j} a^{n+1-j} b^j$& por la parte $(a)$\\\\
              \end{tabular}
            \end{center}
          con lo que el teorema del binomio es válido para $n+1.$ y por lo tanto para $n \in  \mathbb{Z}^+$\\\\

          %------------------------(e)---------------------------
          \item Demostrar que 
          \begin{enumerate}[\bfseries (i)]
            %---------------(i)-----------------
            \item $\displaystyle\sum_{j=0}^{n} {n \choose j} = {n \choose 0} + ... + {n \choose n} = 2^n$\\\\
            Demostración.- \; Por el teorema del binomio $2^n=(1+1)^n = \sum\limits_{j=0}^n {n \choose j}(1^j)(1^{n-j})=\sum_{j=0}^n {n \choose j}$\\\\

            %---------------(ii)-----------------
            \item $\displaystyle\sum_{j=0}^n (-1)^j {n \choose j} = {n \choose 0}- {n \choose 1}+...\pm {n \choose n} =0$\\\\
            Demostración.- \;  De igual manera por el teorema del binomio $0=(1+(-1))^n = \sum\limits_{j=0}^n(-1)^j {n \choose j}$\\\\ 

            %---------------(iii)-----------------
            \item $\displaystyle\sum_{l \; impar} {n \choose l} = {n \choose 1} + {n \choose 3}+ ... = 2^{n-1}$\\\\
            Demostración.- \;  Restando $(ii)$ de $(i)$ se tiene que,
              \begin{center}
                \begin{tabular}{rcl}
                  $2 {n \choose 0} + 2 {n \choose 2} + ... + 2{n \choose n}$&$=$&$2^n - 0$\\\\
                  $2\sum\limits_{j \; impar}^n {n \choose j} $&$=$&$2^n$\\\\ 
                  $\sum\limits_{j \; impar}^n {n \choose j}$&$=$&$2^n \cdot 2^{-1}$\\\\ 
                  $\sum\limits_{j \; impar}^n {n \choose j}$&$=$&$2^{n-1}$\\\\
                \end{tabular}
              \end{center}

            %---------------(iv)-----------------
            \item $\displaystyle\sum_{l \; par} {n \choose l} = {n \choose 0} + {n \choose 2} + ...  = 2^{n-1}$\\\\
            Demostración.- \; La demostración es similar al problema anterior pero esta vez sumamos $(i)$ en $(ii)$ \\\\ 
            \end{enumerate}
          \end{enumerate}

      %---------------------------------------------4------------------------------------------
      \item 
        \begin{enumerate}[\bfseries (a)]
        %------------------------(a)---------------------------
        \item Demostrar que $$\displaystyle\sum_{k=0}^{l} {n \choose k} {m \choose l-k} = {n+m \choose l}$$\\\\
        Demostración.- \; La multiplicación de series formales de potencias se realiza recolectando los términos con las mismas potencias de $x$: 
        $$\left( \sum\limits_{a_K}^{\infty} a_k x^k \right) \left( \sum\limits_{a_K}^{\infty} b_k x^k \right) = \sum\limits_{k=0}^{\infty} \left(\sum\limits_{j=0}^{k} a_j x^j b_{k-j} x^{k-j}  \right) = \sum\limits_{k=0}^{\infty} \left(\sum\limits_{j=0}^{k} a_j  b_{k-j}   \right) x^k$$
        Tenga en cuenta que los subíndices en la suma interna suman $k$, la potencia de $x$ en la suma externa.
        Luego aplicando la multiplicación de series de potencias a $(1+x)^m = \sum\limits_{k=0}^{\infty} {m \choose k}x^k$ y $(1+x)^n = \sum\limits_{k=0}^{\infty} {n \choose k}x^k$ de donde $$(1+x)^{m+n} = \sum\limits_{k=0}^{\infty} {m+n \choose k }x^k$$
        (Los índices en las sumas a $\infty$ ya que para $k>n$, ${n  \choose k} =0$), Se tiene:
        $$(1+x)^m (1+x)^n = \sum\limits_{k=0}^{\infty} \left[ \sum\limits_{j=0}^{k} {m \choose j} {n \choose k-j} \right]x^k$$
        ya que $(1+x)^m (1+x)^n = (1+x)^{m+n}$ entonces $${m+n \choose k} = \sum\limits_{j=0}^{k} {m \choose j} {n \choose k-j}$$\\ 
        

        %------------------------(b)---------------------------
        \item demostrar que $$\displaystyle\sum_{k=0}^{n} {n \choose k}^2 = {2n \choose n}$$\\\\
        Demostración.- \; Sea $m=n$, $l=n$ en la parte $(a)$ y notar que ${n \choose k} = {n \choose n-k}.$ de donde se tiene que $$ \sum\limits_{k=0}^{n} {n \choose k} \cdot  {n \choose k} = {n+n \choose n}$$ y por lo tanto $$\sum\limits_{k=0}^{n} {n \choose k}^2 = {2n \choose n}$$\\\\  
        
        \end{enumerate}

      %---------------------------------------------5------------------------------------------
     \item \begin{enumerate}[\bfseries (a)] 

     ------------------------(a)---------------------------
      \item Demostrar por inducción sobre $n$ que $$1 + r +r^2 + ... + r^n = \dfrac{1 - r^{n+1}}{1-r}$$ si $r\neq 1$ (Si es $r=1$, el cálculo de la suma no presenta problema alguno).\\\\
      Demostración.- \; Sea $n=1$ entonces $$1+r = \dfrac{1- r^2}{1-r}$$ el cual vemos que se cumple.\\
      Luego
      \begin{center}
      \begin{tabular}{r c l}
      $1+r+r^2 + ... + r^n + r^{n+1}$&$=$&$\dfrac{1- r^{r+1}}{1-r} + r^{r+1}$\\\\
      &$=$&$\dfrac{1 - r^{n+1} + r^{n+1} (1-r)}{1-r}$\\\\
      &$=$&$\dfrac{1 - r^{n+1}}{1-r}$\\\\
      \end{tabular}
      \end{center} 

      %------------------------(b)----------------------------
      \item Deducir este resultado poniendo $S=1+r+...+r^n$, multiplicando esta ecuación por $r$ y despejando $S$ entre las dos ecuaciones.\\\\
      Tenemos $r\cdot S = r + .... + r^n + r^{n+1}$, luego sea $S - rS$ entonces $S(1-r) = 1-r^{n+1}$ por lo tanto $S = \dfrac{1-r^{n+1}}{1-r}$\\\\
      \end{enumerate}

      %---------------------------------------------6------------------------------------------
      \item La fórmula para $1^2 + 2^2 + ... + n^2$ se puede obtener como sigue: Empezamos con la fórmula $$(k+1)^3 - k^3 = 3k^2 + 3k +1$$
      particularmente esta fórmula para $k=1,...,n$ \; y sumando, obtenemos
	 \begin{center}
	    \begin{tabular}{r c l}
	       $2^3 - 1^3$&=&$3\cdot 1^2 + 3 \cdot 1 +1$\\
	       $3^3 - 2^3$&=&$2^2 + 3\cdot 2 +1 $\\
	       $.$&=&\\
	       $.$&=&\\
	       $.$&=&\\
	       $(n+1)^3 - n^3$&=&$n^2 + 3 \cdot n + 1$\\
	       \hline
	       $(n+1)^3 - 1$&=&$3 \left[ 1^2 + ... + n^2 \right] + 3 \left[ 1 + .... + n \right] + n$\\
	    \end{tabular}
	 \end{center}
      De este modo podemos obtener $\sum\limits_{k=1}^n k^2$ una vez conocido $\sum\limits_{k=1}^n k $ (lo cual puede obtenerse mediante un procedimiento análogo). Aplíquese este método para obtener.
	 \begin{enumerate}[\bfseries (i)]
	 %-------------------------(i)--------------------------
	 \item $1^3 + ... + n^3$\\\\
	    Sea $(k+1)^4 = k^4 + 4k^3 + 6k^2 + 4k + 1^4$ entonces $$(k+1)^4 - k^4 = 4k^3 + 6k^2 +4k + 1, \; \; para \; k=1,...,n $$ por hipótesis tenemos $(n+1)^4 - 1 = 4 \sum\limits_{k=1}^n k^2 + 6 \sum\limits_{k=1}^n k^2 + 4 \sum\limits_{k=1}^n k + n,$ de modo que $$\sum\limits_{k=1}^n k^3 = \dfrac{ (n+1)^4 -1 - 6 \dfrac{n(n+1)(2n+1)}{6} - 4 \dfrac{n(n+1)}{2} - n}{4} = \dfrac{n^4}{4} + \dfrac{n^3}{2} + \dfrac{n^2}{4}$$\\\\

	 %------------------------(ii)---------------------------
	 \item $1^4 + ... + n^4$\\\\
	 Similar al anterior ejercicio partimos de $(k+1)^5 - k^5 = 5k^4 + 10k^3 + 10k^2 + 5k + 1 \; \; \; k=1,...,n$ para obtener $(k+1)^5 - k^5 = 5 \left( \sum\limits_{k=1}^n k^4 \right) + 10 \left( \sum\limits_{k=1}^n k^3 \right) + 10 \left( \sum\limits_{k=1}^n k^2 \right) + 5 \left( \sum\limits_{k=1}^n k \right) + n,$ así $$\sum\limits_{k=1}^n k^4 = \dfrac{(n+1)^5 - 1 - 10\left( \dfrac{n^4}{4} + \dfrac{n^3}{2} + \dfrac{n^2}{4} - 10 \dfrac{n(n+1)(n+2)}{6} - 5 \dfrac{n(n+1)(n+2)}{2} -n \right)}{5} = $$ $$\dfrac{n^5}{5} + \dfrac{n^4}{2} + \dfrac{n^3}{3} - \dfrac{n}{30}$$\\\\

	 %------------------------(iii)---------------------------
	 \item $\dfrac{1}{1 \cdot 2} + \dfrac{1}{2 \cdot 3} + ... + \dfrac{1}{n(n+1)}$\\\\ 
	 A partir de $$\dfrac{1}{k} - \dfrac{1}{k+1} = \dfrac{1}{k(k+1)}, \; \; \; k=1,...,n$$ obtenemos $$1-\dfrac{1}{n+1} = \displaystyle\sum_{k=1}^n \dfrac{1}{k(k+1)}$$ \\\\

	 %------------------------(iv)---------------------------
	 \item $\dfrac{3}{1^2 \cdot 2^2} + \dfrac{5}{2^2 \cdot 3^2} + ... + \dfrac{2n +1}{n^2 (n+1)^2}$\\\\
	 De $$\dfrac{1}{k^2} - \dfrac{1}{(k+1)^2} = \dfrac{2k+1}{k^2(k+1)^2}, \; \; \; k=1,...,n$$ obtenemos $$1-\dfrac{1}{(n+1)^2} = \displaystyle\sum_{k=1}^n \dfrac{2k+1}{k^2(k+1)^2}$$  \\\\
	 \end{enumerate}

      %---------------------------------------------7-------------------------------------------
      \item Utilizar el método del problema 6 para demostrar que $\displaystyle\sum_{k=1}^n k^p$ puede escribirse siempre en la forma $$\dfrac{n^{p+1}}{p+1} + An^p + Bn^{p-1} + Cn^{p-2} + ...$$
      Las diez primeras de estas expresiones son
      \begin{center}
      \begin{tabular}{r c l}
      $\displaystyle\sum_{k=1}^n k$&=&$\dfrac{1}{2}n^2 + \dfrac{1}{2}n$\\\\
      $\displaystyle\sum_{k=1}^n k^2$&=&$\dfrac{1}{3} n^3 + \dfrac{1}{2}n^2 + \dfrac{1}{6}n$\\\\
      $\displaystyle\sum_{k=1}^n k^3$&=&$\dfrac{1}{4}n^4 + \dfrac{1}{2} n^3 + \dfrac{1}{4} n^2$\\\\
      $\displaystyle\sum_{k=1}^n k^4$&=&$\dfrac{1}{5} n^6 + \dfrac{1}{2} n^4 + \dfrac{1}{3} n^3 - \dfrac{1}{30}n$\\\\
      $\displaystyle\sum_{k=1}^n k^5$&=&$\dfrac{1}{6}n^6 + \dfrac{1}{2} n^5 + \dfrac{5}{12}n^4 - \dfrac{1}{12}n^2$\\\\
      $\displaystyle\sum_{k=1}^n k^6$&=&$\dfrac{1}{7}n^7 + \dfrac{1}{2}n^6 + \dfrac{1}{2}n^5 - \dfrac{1}{6}n^3 + \dfrac{1}{42}n$\\\\
      $\displaystyle\sum_{k=1}^n k^7$&=&$\dfrac{1}{8}n^8 + \dfrac{1}{2}n^7 + \dfrac{7}{12}n^6 - \dfrac{7}{24}n^4 + \dfrac{1}{12}n^2$\\\\
      $\displaystyle\sum_{k=1}^n k^8$&=&$\dfrac{1}{9}n^9 + \dfrac{1}{2}n^8 + \dfrac{2}{3}n^7 - \dfrac{7}{15}n^5 + \dfrac{2}{9}n^3 - \dfrac{1}{30}n$\\\\
      $\displaystyle\sum_{k=1}^n k^9$&=&$\dfrac{1}{10}n^{10} + \dfrac{1}{2} n^9 + \dfrac{3}{4}n^8 - \dfrac{7}{10}n^6 + \dfrac{1}{2}n^4 - \dfrac{3}{20} n^2$\\\\
      $\displaystyle\sum_{k=1}^n k^10$&=&$\dfrac{1}{11}n^{11} + \dfrac{1}{2}n^{10} + \dfrac{5}{6}n^9 - 1n^7 + 1n5 - \dfrac{1}{2}n^3 + \dfrac{5}{66}n$\\\\
      \end{tabular}
      \end{center}
      Obsérvese que los coeficientes de la segunda columna son siempre $\dfrac{1}{2}$ y que después de la tercera columna las potencias de $n$ de coeficiente no nulo van decreciendo de dos en dos hasta llegar a $n^2$ o a $n$. Los coeficientes de todas las columnas, salvo las dos primeras, parecen bastante fortuitos, pero en realidad obedecen a cierta regla; encontrarla puede considerarse como una prueba de superperspicacia. Para descifrar todo el asunto, véase el problema 26-17)\\\\
      Demostración.- Sea  $(k+1)^{p+1}$ entonces por el teorema del binomio:
      \begin{center}
	 \begin{tabular}{crl}
	    $(k+1)^{p+1}$&$=$&${p+1 \choose p+1} k^{p+1}+{p+1 \choose p} k^p + ... + {p+1 \choose 1} k^p + {p+1 \choose 0}k^0$\\\\
	    $(k+1)^{p+1}$&$=$&$1 \cdot k^{p+1} + {p+1 \choose p} k^p + ... + {p+1 \choose 1}k + {p+1 \choose 0} k^0$\\\\
	    $(k+1)^{p+1}-k^{p+1}$&$=$&$(p+1)k^p + ... + (p+1)k+1 \cdot k^0$\\\\
	 \end{tabular}
      \end{center}
      Luego sumando para cada $k=1,...,n$ se tiene:
      \begin{center}
	 \begin{tabular}{rcl}
	    $2^{p+1} - 1^{p+1}$&$=$&$(o+1)1^p + ... + (p+1)1 + 1 \cdot 1^0$ \\
	    $3^{p+1} -2{p+1}$&$=$&$(p+1)2^p + ... + (p+1)2+1\cdot 2^0$\\
	    &.&\\
	    &.&\\
	    &.&\\
	    $(n+1)^{p+1}$&$=$&$(p+1)n^{p}+...+(p+1)n + 1 \cdot ^0$\\
	 \end{tabular} 
      \end{center}
      Luego por el anterior problema 
      \begin{center}
	 \begin{tabular}{rcl}
	    $(n+1)^{p+1}$ & $=$ & $(p+1)\sum\limits_{k=1}^n k^p + ... + (p+1)\sum\limits_{k=1}^n k^1 + \sum\limits_{k=1}^n k^0 + k^0$\\\\
	    $\dfrac{(n+1)^{p+1}}{(p+1)}$&$=$&$\sum\limits_{k=1}^n k^p + \dfrac{{p+1 \choose p-1}}{(p+1)} \sum\limits_{k=1}^n k^{p-1} + ... + \dfrac{(p+1)}{(p+1)} \sum\limits_{k=1}^n k^1 + \dfrac{1}{(p+1)} \left( \sum\limits_{k=1}^n k^0+ k^0 \right)$\\\\
	 \end{tabular}
      \end{center}
      Luego asumimos que la proposición es verdad para $p-1$ donde podríamos escribir como,
      \begin{center}
	 \begin{tabular}{rcl}
	    $\dfrac{(n+1)^{p+1}}{(p+1)}$&$=$&$\sum\limits_{k=1}^{n} k^p +$ términos que involucran las potencias de $n \leq p$\\\\
	    $\sum\limits_{k=1}^n k^p$&$=$&$\dfrac{(n+1)^{p+1}}{(p+1)} + $ términos que involucran las potencias de $n \leq p$\\\\
	 \end{tabular}
      \end{center}

      %---------------------------------------------8-------------------------------------------
      \item Demostrar que todo número natural es o par o impar.\\\\
      Demostración.- \; Asumimos que $n$ es impar o par, entonces debemos probar que $n+1$ también, es o bien impar o bien par.\\ 
      Sea $n$ par entonces $n=2k$ para algún $k$. Así $n+1=2k+1$  y por definición vemos que es impar.\\
      Luego sea $n$ impar, entonces $n=2k+1$ para algún $k$.  Por lo tanto $n+1=2k+1+1 = 2k +2 = 2(k+1)$. Así en cualquiera de los dos casos, $n+1$ es o bien par o impar. \\\\   

      %---------------------------------------------9-------------------------------------------
      \item Demostrar que si un conjunto $A$ de números naturales contiene $n_0$ y contiene $k+1$ siempre que contenga $k$, entonces $A$ contiene todos los números naturales $\geq n_0$.\\\\
	 Demostración.- \; Sea $B$ el conjunto de todos los números naturales $l$ tales que $n_0 -1 +1$ está en $A$. Entonces $1$ está en $B$, y $l+1$ está en  $B$ si $l$ está en $B$, es decir $k=n_0 -1 + l $, por lo tanto $k+1=(n_0 -1 )+(l+1)$ está en $A$, lo que implica que $l+1$ está en $B$ de modo que $B$ contiene todos los números naturales, los cuales significa que $A$  contiene todos los números naturales $\geq n_0$\\\\

       %---------------------------------------------10-------------------------------------------
      \item Demostrar el principio de inducción completa a partir del principio de buena ordenación.\\\\
      Demostración.-\; Supongamos que $A$ contiene a $1$ y que $A$ contiene a $n+1$, si contiene a $n$. Si $A$ no contiene todos los números naturales, entonces el conjunto $B$ de números naturales que no están en $A$ es distinto de $\emptyset$. Por lo tanto, $B$ tiene un número natural $n_0$. Ahora $n_0 ,\neq 0$ ya que $A$ contiene a $1$ entonces podemos escribir $n_0 = (n_0 -1) +1$, donde $n_0-1$ es un número natural. Luego $n_0-1$ no está en $B$, entonces $n_0 -1$ está en $A$. Por hipótesis, $n_0$ debe estar en $A$, entonces $n_o$ no está en $B$, el cual es una contradicción.             	 
      
      %---------------------------------------------11-------------------------------------------
      \item Demostrar el principio de inducción completa a partir del principio de inducción ordinario.\\\\
      Demostración.- \; Se sabe que $1$ está en $B$. Luego si $k$ está en $B$, entonces $1,...,k$ están todos en $A$, de modo que $k+1$ está en $A$ y así $1,...,k+1$ están en $A$, con lo que $k+1$ está en $B$. Por inducción, $B=N$, así que también $A=N$.\\\\

      %---------------------------------------------12--------------------------------------------
      \item 
	 \begin{enumerate}[\bfseries (a)]

	 %------------------------(a)---------------------------
	 \item Si $a$ es racional y \; $b$ es irracional ¿es $a+b$ necesariamente irracional? ¿Y si $a$ \; y \; $b$ es irracional? \\\\
	 Respuesta.- \; Si, puesto que si $a+b$ fuese racional, entonces $b=(a+b) -a$ seria racional. Luego si $a$ y $b$ son irracionales, entonces $a+b$ podría ser racional, ya que $b$ podría ser $r-a$ para algún número racional $a$.\\\\          

	 %------------------------(b)---------------------------
	 \item Si $a$ es racional y \; $b$ es irracional, ¿es $ab$ necesariamente irracional?\\\\
	 Respuesta.- \; Si $a=0$, entonces $ab$ es racional. Pero si $a\neq 0$ entonces $ab$ no podría ser racional, ya que entonces $b=(ab) \cdot a^{-1}$ sería racional.\\\\     

	 %------------------------(c)---------------------------
	 \item ¿Existe algún número $a$ tal que $a^2$ es irracional pero $a^4$ racional?\\\\
	 Si existe por ejemplo $\sqrt[4]{2}$\\\\

	 %------------------------(d)---------------------------
	 \item ¿Existen dos números irracionales tales que sean racionales tanto su suma como su producto?\\\\
	 Si existen por ejemplo $\sqrt{2}$ y $- \sqrt{2}$\\\\
      \end{enumerate}

      %---------------------------------------------13--------------------------------------------
      \item 
      \begin{enumerate}[\bfseries a)]
      %------------------------(a)---------------------------
      \item Demostrar que $\sqrt{3}$, $\sqrt{5}$ y $\sqrt{6}$ son irracionales. Indicación: Para tratar $\sqrt{3},$ por ejemplo, aplíquese el hecho de que todo entero es de la forma $3n$ ó $3n+1$ ó $3n+2$ ¿Por qué no es aplicable esta demostración para $\sqrt{4}$?\\\\
      Demostración.- \; Puesto que:
      \begin{center}
      \begin{tabular}{r c l c l}
      $(3n+1)^2$&=&$9n^2 + 6n + 1$&=&$3(3n^2+2n) + 1$\\
      $(3n+2)^2$&=&$9n^2+12n + 4$&=&$3(3n^2 + 4n + 1) + 1$\\
      \end{tabular}
      \end{center}
      queda demostrado que un número no es múltiplo de $3$ si es de la forma $3n+1$ ó $3n+2$.\\
      se sigue que $k^2$ es divisible por $3$, entones $k$ debe ser también divisible por $3$. Supóngase ahora que $\sqrt{3}$ fuese racional, y sea $\sqrt{3} = p/q$, donde $p$ \; y \; $q$ no tienen factores comunes. Entonces $p^2=3q^2,$ de modo que $p^2$ es divisible por $3$, así que también lo debe ser $p$. De este modo, $p=3p^{'}$ para algún número natural $p^{'}$, y en consecuencia $(3p^{'})^2 = 3q^2$ ó $(3p^{'})^2 = q^2.$ Así pues, $q$ es también divisible por $3$, lo cual es una contradicción.\\
      Las mismas demostraciones valen para $\sqrt{5}$ y $\sqrt{6}$, ya que las ecuaciones,
      \begin{center}
      \begin{tabular}{rclcl}
      $(5n+1)^2$&=&$25n^2 + 10n + 1$&=&$5(5n^2 + 2n)+1$\\
      $(5n+2)^2$&=&$25n^2 + 20n + 4$&=&$5(5n^2 + 4n)+4$\\
      $(5n+3)^2$&=&$25n^2 + 30n + 9$&=&$5(5n^2 + 6n + 1)+4$\\
      $(5n+4)^2$&=&$25n^2 + 40n + 16$&=&$5(5n^2+8n+3)+1$\\
      \end{tabular}
      \end{center}
      la ecuación correspondiente para los números de la forma $6n+m$ demuestran que si $k^2$ es divisible por $5$ ó $6$, entones también lo debe ser $k$. La demostración falla para $\sqrt{4}$, porque $(4n+2)^2$ es divisible por $4$.\\\\

      %------------------------(b)---------------------------
      \item Demostrar que $\sqrt[3]{2}$ y $\sqrt[3]{3}$ son irracionales.\\\\
      Demostración.- \; Puesto que,
      $$(2n+1)^3 = 8n^3 + 12n^2 + 6n + 1 = 2(4n^3 + 6n^2 + 3n) + 1,$$
      se sigue que si $k^3$ es par, entonces $k$ es par. Si $\sqrt[3]{2} = p/q,$ donde $p$ \; y \; $q$ no tienen factores comunes, entonces $p^3 = 2q^3,$ de modo que $p^3$ es divisible por $2,$ por lo que también lo debe ser $p.$ Así pues, $p=2p^{'}$ para algún número natural $p^{'}$ y en consecuencia $(2p^{'})^3 = 2q^3,$ ó $4(p^{'})^3 = q^3.$ Por lo tanto, $q$ es también par, lo cual es una contradicción.\\
      La demostración para $\sqrt[3]{3}$ es análogo, utilizando las ecuaciones.
      $$(3n+1)^3 = 27n^3 + 27n^3 + 27n^2 + 9n +1 = 3(9n^3 + 9n^2 + 3n) + 1,$$
      $$(3n+2)^3 = 27n^3 + 54n^2 + 36n + 8 = 3(9n^2 + 18n^2 + 12n + 2) + 2.$$\\\\
      \end{enumerate}

      %---------------------------------------------14--------------------------------------------
      \item Demostrar que:
      \begin{enumerate}[\bfseries (a)]
      %------------------------(a)---------------------------
      \item $\sqrt{2} + \sqrt{3}$ es irracional.\\\\
      Demostración.- \; Sea $\sqrt{2} + \sqrt{3}$ racional, entonces $\left( \sqrt{2} + \sqrt{3} \right)^2$ sería racional, luego $$5 + 2 \sqrt{6}$$ y en consecuencia $\sqrt{6}$ sería racional lo cual es falso.\\\\

      %------------------------(b)---------------------------
      \item $\sqrt{6} - \sqrt{2} - \sqrt{3}$ es irracional.\\\\
      Demostración.- \;  Sea $\sqrt{6} - \sqrt{2} - \sqrt{3}$ racional, entonces 
      \begin{center}
      \begin{tabular}{rcl}
      $\left[ \sqrt{6} + \left( \sqrt{2} + \sqrt{3} \right) \right]^2$&=&$6 + \left( \sqrt{2} + \sqrt{3} \right)^2 - 2 \sqrt{6} \left( \sqrt{2} + \sqrt{3} \right)$\\
      &=&$11 + 2\sqrt{6} \left[ 2 - \left( \sqrt{2} + \sqrt{3} \right) \right]$\\
      \end{tabular}
      \end{center}
      Así, $\sqrt{6} \left[ 2 - \left( \sqrt{2} + \sqrt{3} \right) \right]$ sería racional, con lo que de igual manera sería,
      \begin{center}
      \begin{tabular}{r c l}
      $\lbrace \sqrt{6} \left[ 1 - \left( \sqrt{2} + \sqrt{3} \right) \right] \rbrace ^2$&=&$6 \left[ 1 - \left( \sqrt{2} + \sqrt{3} \right) \right]^2$\\
      &=&$11 + 2\sqrt{6} \left[1 - \left( \sqrt{2} + \sqrt{3} \right) \right]$\\
      \end{tabular}
      \end{center}
      De este modo $\sqrt{6} - (\sqrt{2} + \sqrt{3})$ y $\sqrt{6} - 2 \left( \sqrt{2} + \sqrt{3} \right)$ serían racionales, lo que implicaría que $\sqrt{2} + \sqrt{3}$ fuese racional, en contradicción de la parte $a)$.\\\\
      \end{enumerate}

      %---------------------------------------------15--------------------------------------------
      \item 
      \begin{enumerate}[\bfseries (a)]
      %------------------------(a)---------------------------
      \item Demostrar que si $x=p+ \sqrt{q}$, donde $p$ \; y \; $q$ son racionales, y \; $m$ es un número natural, entonces $x^m = a + b \sqrt{q}$ siendo $a$ \; y \; $b$ números racionales.\\\\
      Demostración.- \; Sea $m=1$ entonces $(p + \sqrt{q})^1 = a + b\sqrt{q}$. Supongamos que se cumple para $m$, entonces $$(p+\sqrt{q})^{m+1} = (a + b\sqrt{q})(p + \sqrt{q}) = (ap+bq)+(a+pb)\sqrt{q}$$ donde $ap+bq$ y $a+bp$ son racionales.\\\\

      %------------------------(b)---------------------------
      \item Demostrar también que $(p - \sqrt{q})^m = a - b\sqrt{q}$\\\\
      Demostración.- \; Similar a la parte $a)$, se cumple para $m=1$. Si es verdad para $m$, entonces $$(p-\sqrt{q})^{m+1} = (a - b\sqrt{q})(p - \sqrt{q}) = (ap+bq)-(a+pb)\sqrt{q}$$.\\\\
      \end{enumerate}

      %---------------------------------------------16--------------------------------------------
      \item 
      \begin{enumerate}[\bfseries (a)]
      %------------------------(a)---------------------------
      \item Demostrar que si $m$ \; y \; $n$ son números naturales y $m^2/n^2 < 2$, entonces $\left( m+2n \right)^2 / \left( m + 2 \right)^2 > 2;$ demostrar, además que $$\dfrac{\left( m + 2n \right)^2}{\left( m + n \right)^2} - 2 < 2 - \dfrac{m^2}{n^2}$$\\\\
      Demostración.- \; Si $m^2/n^2 < 2$ entonces $m^2 < 2 n^2$, sumando $m^2$, $4mn$ y $2n^2$ tenemos $2m^2 + 4mn + 2n^2 < 4n^2 + m^2 + 4mn$, luego $2(m+n)^2 < (m+2n)^2$, así nos queda $(m + 2n)^2 / (m+n)^2 > 2$\\
      Para la segunda parte podemos partir de $m^2 - 2n^2<0$, entonces:
      \begin{center}
      \begin{tabular}{r c l l}
      $m^2 - 2n^2$&$<$&$0$&\\\\
      $m^3 - 2mn^2$&$<$&$0$&multiplicando por $m$\\\\
      $mn^2 + m^3 + mn^2 - 4mn^2$&$<$&$0$&escribiendo $mn^2$ de otra manera\\\\
      $mn^2 + 2n^3 + m^3 + m^2 n + mn^2 -2m^2 n - 4mn^2 -2n^3$&$<$&$0$&sumando $2n^3$ y $2m^2 n$\\\\
      $n^2 (m+2n) + \left[ (m^2 + 2mn + n^2)(m-2n) \right]$&$<$&$0$&\\\\
      $ n^2(m+2n)^2  +  \left[ (m+n)^2 (m+2n)(m-2n) \right]$&$<$&$0$&multiplicando por $m+2n$\\\\
      $n^2(m+2n)^2  +  \left[ (m+n)^2 (m^2 - 4n^2) \right]$&$<$&$0$&\\\\
      $\dfrac{n^2(m+2n)^2 - 4n^2(m+n)^2 + m^2(m+n)^2}{n^2(m+n)^2}$&$<$&$0$&dividimos por $n^2(m+n)^2$\\\\
      $\dfrac{(m+2n)^2 - 2(m+2)^2 - 2n^2(m+2)^2}{n^2(m+n)^2}$&$<$&$- \dfrac{m^2}{n^2}$&\\\\
      $\dfrac{(m+2n)^2}{(m+n)^2} - 2$&$<$&$2 - \dfrac{m^2}{n^2}$&\\\\
      \end{tabular}
      \end{center}

      %------------------------(b)---------------------------
      \item Demostrar los mismos resultados con todos los signos de desigualdad invertidos. \\\\
      Demostración.- \; Quedará de la siguiente forma, Si $m^2/n^2>2$, entonces $\left( m+2n \right)^2 / \left( m + 2 \right)^2 < 2$, luego demostrar que $$\dfrac{\left( m + 2n \right)^2}{\left( m + n \right)^2} - 2 > 2 - \dfrac{m^2}{n^2}$$
      Similar a la parte $a)$ tendremos $m^2 > 2n^2$, luego $2m^2 + 4mn + 2n^2 > 4n^2 + m^2 + 4mn$, así $ (m+2n)^2 > 2(m + n)^2 $\\
      Después se puede demostrar la segunda parte con facilidad siguiendo el ejemplo $a)$ pero invirtiendo la desigualdad ya que $n$ \; y \; $m$ son números natural.\\\\

      %------------------------(c)---------------------------
      \item Demostrar que si $m/n < \sqrt{2}$, entonces existe otro número racional $m^2 / n^2$ con $m/n < m^{'} / n^{'} < \sqrt{2}$\\\\
      Demostración.- \; Sea $m_1=m+2n$ y $n_1=m+n$, luego elegimos y remplazamos en,  
      \begin{center}
      \begin{tabular}{rclcr}
      $m^{'}$ & $=$ & $m_1+2n_1$ & $=$ & $3m+4n$\\
      $n^{'}$ & $=$ & $m_1+n_1$  & $=$ & $2m+3n$\\
      \end{tabular}
      \end{center}
      \end{enumerate}
      De donde $\dfrac{m}{n}<h$ si y sólo si $0<m^{'} - mn^{'}=(3m+4n)n - (2m + 3n)m=2(2n^2 - m^2)$. Es claro para $\dfrac{m}{n} < \sqrt{2}$ que $2n^2 - m^2 >0$\\
      Por otro lado tenemos $\dfrac{m^{'}}{n^{'}}$ si y sólo si $0<2n^{'^{2}} - m^{'^{2}}=2(2m+3n)^2-(3m+4n)^2=2n^2 - m^2.$ Como antes, $2n^2-m^2>0$ se sigue de $\dfrac{m}{n}<\sqrt{2}$\\\\
 
      %---------------------------------------------17----------------------------------------------
      \item Parece normal que $\sqrt{n}$ tenga que ser irracional siempre que el número natural $n$ no sea el cuadrado de otro número natural. Aunque puede usarse en realidad el método del problema 13 del capitulo 2 de Michael Spivak para tratar cualquier caso particular, no está claro, sin más, que este método tenga que dar necesariamente resultados, y para una demostración del caso general se necesita más información. Un número natural $p$ se dice que es un número primo si es imposible escribir $p=ab$; por conveniencia se considera que $1$ no es un número primo. Los primeros números primos son $2,3,5,7,11,13,17,19.$ Si $n>1$ no es primo, entonces $n=ab,$ con $a$ \; y \; $b$ ambos $<n;$ si uno de los dos $a$ \; o \; $b$ no es primo, puede ser factorizado de manera parecido; continuando de esta manera se demuestra que se puede escribir $n$ como producto de números primos. Por ejemplo, $28=2\cdot 2\cdot 7.$\\
      \begin{enumerate}[\bfseries a)]
         %------------------------(a)---------------------------
         \item Conviértase este argumento en una demostración riguroso por inducción completa. (En realidad, cualquier matemático razonable aceptaría este argumento informal, pero ello se debería en parte a que para él estaría claro cómo formularla rigurosamente.)\\
         Un teorema fundamental acerca de enteros, que no demostraremos aquí, afirma que esta factorización es única, salvo en lo que respeta al orden de los factores. Así, por ejemplo, $28$ no puede escribirse nunca como producto de números primos uno de los cuales sea $3$, ni puede ser escrito de manera que $2$ aparezca una sola vez (ahora debería verse clara la razón de no admitir a $1$ como número primo.)\\\\
         demostración.- \; Supóngase que para todo número $<n$ puede ser escrito  como un producto de primos. Si $n>1$ no es primo, entonces $n=ab$, para $a,b<n$. Pero $a$ \; y \; $b$ son ambos producto de primos, así que $n=ab$ lo es también.\\\\

         %------------------------(b)---------------------------
         \item Utilizando este hecho, demostrar que $\sqrt{n}$ es irracional a no ser que $n=m^2$ para algún número natural $m.$\\\\
         Demostración.- \; Sea $\sqrt{n} = \dfrac{p}{q}$, entonces $nb^2 = a^2,$  luego si descomponemos en producto de factores primos, $nb^2$ y $a^2$ deberían coincidir. Ahora según lo explicado anteriormente, cada número primo debe aparecer un número par de veces en $a^2$ y $b^2$, y por lo tanto deberá ocurrir lo mismo con $n.$ Esto implica que $n$ es un cuadrado perfecto.\\\\

         %------------------------(c)---------------------------
         \item Demostrar que $\sqrt[k]{n}$ es irracional a no ser que $n=m^k$\\\\
         Demostración.- \; La Demostración es parecida a la parte $b$ pero haciendo uso del hecho de que cada número primo entra en $a^k$ y en $b^k$ un número de veces que es múltiplo de $k$\\\\   

         %------------------------(d)---------------------------
         \item Al tratar de números primos no se puede omitir la hermosa demostración de Euclides de que existe un número infinito de ellos. Demuestre que no puede haber sólo un número finito de números primos $p_1, p_2, p_3,...,p_n$ considerando $p_1\cdot p_2 \cdot ... \cdot p_k + 1$\\\\
         Demostración.- \; Si $p_1,...,p_n$ fuesen los únicos números primos, entonces $p_1 \cdot p_2 \cdot ... \cdot p_n + 1$ no podría ser primo, ya que es mayor que cada uno de ellos, de modo que tiene que ser divisible por un número primo. Pero es claro que este número primo no es ninguno de los $p_1,...,p_n$, lo cual constituye una contradicción. Para poder explicarlo mejor si $p_1,...,p_n$ son los $n$ primeros números primos, entonces el primo que ocupa el lugar $n+1$ es $\leq p_1 \cdot p_2 \cdot ... \cdot p_n+1$. Sin embargo, $p_1\cdot p_2 \cdot ...\cdot p_n + 1$ no tiene que ser necesariamente primo.\\\\       
         \end{enumerate} 

      %----------------------------------18.---------------------------------------
      \item
	 \begin{enumerate}[\bfseries (a)]
	    %--------------------(a)-------------------------
	    \item Demostrar que si $x$ satisface
	       $$x_n + a_{n-1}x^{n-1} + ... + a_0 = 0$$
	       para algunos enteros $a_{n-1},...,a_n$ entonces $x$ es irracional si no es entero. (¿Por qué es esto una generalización del problema 17?)\\\\
	       Demostración.- \; Supóngase que es $x=p/q$ donde $p$ y $q$ son números naturales primos entre si. Entonces $$\dfrac{p^n}{q^n} + a_{n-1} \dfrac{p^{n-1}}{q^{n-1}} + ... + a_0 = 0,$$
	       con lo que 
	       $$p^n + a_{n-1} p ^{n-1}q + ... + a_0 q^n = 0$$
	       Ahora bien, si $q \neq \pm 1,$ entonces $q$ tiene por lo menos un divisor primo. Este divisor primo divide a cada uno de los términos que siguen a $p^n$, con lo que también deberá dividir a $p^n$. Dividirá por lo tanto a $p$, lo cual es una contradicción. Así pues, $q= \pm 1,$ lo que significa que $x$ es entero.\\\\

	    %--------------------(b)--------------------------
	    \item Demostrar que $\sqrt{2} + \sqrt[3]{2}$ es irracional.\\\\ 
	       Demostración.- \; Sea $a=\sqrt{2} + 2^{1/3}$. Se demostrará por contradicción. Supongamos que $a$ es racional, entones,
	       \begin{center}
		  \begin{tabular}{crl}
		     $2$ & $=$ & $(2^{1/3})^3$ \\
			 & $=$ & $(a- \sqrt{2})^3$ \\
			 & $=$ & $a^3 - 3a^2 \sqrt{2} + 3a \cdot 2 - 2^{3/2}$ \\
			 & $=$ & $a^3 + 6a - \sqrt{2}(3a^2+2)$ \\
		  \end{tabular}
	       \end{center}
	       por lo tanto, $$\sqrt{2}=\dfrac{a^3 + 6a -2}{3a^2 + 2} \in \mathbb{Q}$$
	       es bien sabido que $\sqrt{2}$ es irracional. De ahí llegamos a una contradicción.\\\\
      \end{enumerate}

      %------------------------------------------19.----------------------------------------------
   \item Demostrar la desigualdad de Bernoulli: Si $h>-1$, entonces $$(1+h)^n \geq 1 + nh$$
      ¿Por qué es esto trivial si $h>0$?\\\\
      Demostración.-\; Si $n=1$, entonces $(1+h)^n=1+nh$. Supóngase que $(1+h)^n \geq 1+nh.$ Entonces $(1+h)^n+1=(1+h)(1+h)^n \geq (1+h)(1+nh)$, ya que $1+h>0$ luego $1+(n+1)h+nh^2\geq 1+(n+1)h$\\
      Para $h>0$, la igualdad se sigue directamente del teorema del binomio, ya que todos los demás términos que aparecer en el desarrollo de $(1+h)^n$ son positivos.\\\\
      
      %------------------------------------------20.--------------------------------------------------
   \item La sucesión de Fibonacci $a_1,a_2,a_3,...$ se define como sigue:
      \begin{center}
	 \begin{tabular}{rcll}
	    $a_1$ & $=$ & $1,$ & \\
	    $a_2$ & $=$ & $1,$ & \\
	    $a_n$ & $=$ & $a_{n-1} + a_{n-2}$ & para $n \geq 3$\\
	 \end{tabular}
      \end{center}
      Esta sucesión, cuyos primeros términos son $1,1,2,3,5,8,...,$ fue descubierta por Fibonacci (1175-1250, aprox.) en relación con un problema de conejos. Fibonacci supuso que una pareja de conejos criaba una nueva pareja cada mes y que después de dos meses cada nueva pareja se comportaba del mismo modo. El número $a_n$ de parejas nacidas en el n-ésimo mes es $n_{n-1} + a_{n-2}$, puesto que nace una pareja por cada pareja nacida en el mes anterior, y además, cad pareja nacida hace dos meses produce ahora una nueva pareja. Es verdaderamente asombroso el número de resultados interesantes relacionados con esta sucesión, hasta el punto de existir una Asociación de Fibonacci que publica una revista, the Fibonacci Quarterly.\\
      Demostrar que $$a_n = \dfrac{\left( \dfrac{1+\sqrt{5}}{2} \right)^n - \left( \dfrac{1-\sqrt{5}}{2} \right)^n}{\sqrt{5}}$$\\
      Demostración.- \; Al ser $$\dfrac{\left( \dfrac{1+\sqrt{5}}{2} \right)^1 - \left( \dfrac{1-\sqrt{5}}{2} \right)^1}{\sqrt{5}}=\dfrac{\sqrt{5}}{\sqrt{5}}=1$$
      La fórmula es válida para $n=1$ y también se cumple para $n=1$. Supóngase que es válida para todo $k<n,$ donde $n\geq 3.$ En tal caso es válida en particular para $n-1$ y para $n-2$, luego por hipótesis
      \begin{center}
	 \begin{tabular}{rcl}
	    $a_n$ & $=$ & $a_{n-1} + a_{n-2}$\\\\
		  & $=$ & $\dfrac{\left( \dfrac{1+\sqrt{5}}{2} \right)^{n-2} - \left( \dfrac{1-\sqrt{5}}{2} \right)^{n-2} + \left( \dfrac{1+\sqrt{5}}{2} \right)^{n-1} - \left( \dfrac{1-\sqrt{5}}{2} \right)^{n-1}}{\sqrt{5}}$\\\\
		  & $=$ & $\dfrac{\left( \dfrac{1+\sqrt{5}}{2}\right)^{n-2} \left( \dfrac{1+\sqrt{5}}{2} \right) - \left( 1 - \dfrac{1-\sqrt{5}}{2} \right)^{n-2} \left( 1 - \dfrac{1- \sqrt{5}}{2} \right)}{\sqrt{5}}$\\\\
		  & $=$ & $\dfrac{\left( \dfrac{1+\sqrt{5}}{2} \right)^{n-2} \left( \dfrac{1+\sqrt{5}}{2}\right)^2 - \left( 1 - \dfrac{1-\sqrt{5}}{2} \right)^{n-2} \left( 1 - \dfrac{1-\sqrt{5}}{2} \right)^2}{\sqrt{5}}$\\\\
	       & $=$ & $\dfrac{\left( \dfrac{1+\sqrt{5}}{2} \right)^n - \left( \dfrac{1-\sqrt{5}}{2} \right)^n}{\sqrt{5}}$\\\\
	 \end{tabular}
      \end{center}

      %-------------------------------------------------21.------------------------------------------------
   \item La desigualdad de Schwarz (problema 1-19) tiene en realidad una forma más general: $$\displaystyle\sum_{i=1}^n x_1 y_1 \leq \sqrt{\sum_{i=1}^n x_i^2} \sqrt{\sum_{i=1}^n y_i^2}$$
      Dar de esto tres demostraciones, análogas a las tres demostraciones del problema 1-19\\\\
      Demostración.-\; 
      \begin{enumerate}[\bfseries i)]
	 \item Como antes, la demostración es trivial si para todo $y_i=0$ o si hay algún numero $\lambda$ con $x_i=\lambda y_i$ para todo $i$. Es decir, 
	    \begin{center} 
	       \begin{tabular}{rcl}
		  $0$ & $<$ & $\sum\limits_{i=1}n (\lambda y_i - x_i)^2$\\\\
		   & $=$ & $\lambda \left( \sum\limits_{i=1}^n y_i^2 \right) -2 \lambda \left( \sum\limits_{i=1}^n x_i y_i\right) + \sum\limits_{i=1}^n x_i^2$\\\\
	       \end{tabular}
	       así por el problema 1-18 queda demostrado. 
	    \end{center}
	 \item Usando $2xy \leq x^2 + y^2$ con 
            $$x=\dfrac{x_i}{\sqrt{\sum\limits_{i=1}^n x_i^2}}, \,\,\,\, y=\dfrac{y_i}{\sqrt{\sum\limits_{i=1}^n y_i^2}}$$
            obtenemos, 
            $$\dfrac{2x_iy_1}{\sqrt{\sum\limits_{i=1}^n x_i^2} \sqrt{\sum\limits_{i=1}^n y_i^2}} \leq \dfrac{x_i^2}{\sum\limits_{i=1}^n x_i^2} + \dfrac{y_i^2}{\sum\limits_{i=1}^n y_i^2} \,\,\,\, (1)$$
            luego
            $$\dfrac{\sum\limits_{i=1}^n 2x_iy_i}{\sqrt{\sum\limits_{i=1}^n x_i^2} \sqrt{\sum\limits_{i=1}^n y_i^2}} \leq \dfrac{\sum\limits_{i=1}^n x_i^2}{\sum\limits_{i=1}^n x_i^2} + \dfrac{\sum\limits_{i=1}^n y_i^2}{\sum\limits_{i=1}^n y_i^2}=2$$
            Nuevamente, la igualdad se cumple solo si se cumple en $(1)$ para todo $i$, lo que significa que, 
            $$\dfrac{x_i}{\sqrt{\sum\limits_{i=1}^n x_i^2}} = \dfrac{y_i}{\sqrt{\sum\limits_{i=1}^n y_i^2}}$$
            para todo $i$. Si todo $y_i$ es distinto de $0$. Esto significa que $x_i=\lambda y_i$ para 
            $$\lambda = \dfrac{\sqrt{\sum\limits_{i=1}^n x_i^2}}{\sqrt{\sum\limits_{i=1}^n y_i^2}}$$
         \item La demostración depende de la siguiente igualdad:
            $$\sum\limits_{i=1}^n x_i^2 \cdot \sum\limits_{i=1}^n y_i^2 = \left( \sum\limits_{i=1}^n x_iy_i \right)^n + \sum\limits_{i<j} (x_i y_j -x_j y_i)^2$$
            al verificar esta igualdad notamos que,
            $$\sum\limits_{i=1}x_i^2 \cdot \sum\limits_{i=1}^n y_i^2 = \sum\limits_{i=1}^n x_i^2 y_i^2 + \sum\limits_{i\neq j} x_i2 y_j^2$$
             y por lo tanto,
             $$\left( \sum\limits_{i=1}^n x_i y_i \right)^2 = \sum\limits_{i=1}^n (x_i y_i)^2 + \sum\limits_{i\neq j} x_i y_i x_j y_j$$
             La diferencia es
             \begin{center} 
                \begin{tabular}{rlc}
                   $\sum\limits_{i \neq j} (x_i^2 y_j^2 - x_i y_i x_j y_j)$ & $=$ & $2 \sum\limits_{i<j} (2_i^2 y_j^2 + x_j^2 y_i^2 - x_i y_i x_j y_j)$\\\\
                      & $=$ & $2 \sum\limits_{i<j} (x_i y_j - x_j y_i)^2$\\\\
                \end{tabular}
             \end{center}
             Si la igualdad se cumple en la desigualdad de Schwarz, para todo $x_iy_j = x_j y_i$. Si algún $y_i \neq 0$ y $y_i \neq 0$, luego $x_i=\dfrac{x_1}{y_1}y_i$ para todo $i$, así tenemos que $\lambda = \dfrac{x_1}{y_1}$\\\\
      \end{enumerate}

      %--------------------------------------------------22.-----------------------------------
       \item El resultado del problema 1-7 tiene una generalización importante:\\
          Si $a_i,...,a_n \geq 0,$ entonces la \textbf{media aritmética} 
          $$A_n=\dfrac{a_1+...+a_n}{n}$$
          y la \textbf{media geométrica}
          $$G_n=\sqrt[n]{a_1\cdot \cdot \cdot a_n}$$
          satisfacen
          $$G_n \leq A_n$$
          \begin{enumerate}[\bfseries (a)]
             %--------------------(a)-----------------------------
             \item Supóngase que $a_1<A_n.$ Entonces algún $a_i$ tiene que satisfacer $a_i>A_n$, pongamos que sea $a_2>A_n$. Sea $\bar{a}_1 = A_n$ y sea $\bar{a}_2=a_1+a_2 - \bar{a}_1$. Demostrar que $$\bar{a}_1 \bar{a}_2 \geq a_1 a_2$$ 
                ¿Por qué la repetición de este proceso un suficiente número de veces demuestra que $G_n \leq A_n$? (He aquí otra ocasión en que resulta ser un buen ejercicio establecer una demostración formal por inducción, al tiempo que se da una explicación informal.)¿Cuándo se cumple la igualdad en la fórmula $G_n A_n$?\\\\
		  Demostración.-\; Tenemos que probar que $$A_n (a_1+a_2-A_n)\geq a_1 a_2$$ 
		  Vemos que es lo mismo demostrar que $A_n^2 - (a_1 + a_2)A_n + a_1 a_2 \leq 0$ de donde $(A_n - a_1)(A_n-a_2)\leq 0$, el cual sabemos que es verdad porque $(A_n -a_1)>0$ y $(A_n - a_2)<0$\\\\

               %------------------(b)------------------------
             \item Haciendo uso del hecho de ser $G_n \leq A_n$ cuando $n=2$, demostrar por inducción sobre $k$, que $G_n \leq A_n$ para $n=2_k$\\\\
		 Demostración.- \; Sabemos que $G_n \leq A_n$ cuando $n=2^1$. Supóngase que $G_n \leq A_n$ para $n=2^k$ para luego $m=2^{k+1}=2n$, entonces,
		  \begin{center}
		      \begin{tabular}{rcll}
			  $G_m$&$=$&$\sqrt[m]{a_1 \cdot \cdot \cdot a_m}$&\\\\
			  &$=$&$\sqrt{\sqrt[n]{a_1 \cdot \cdot \cdot a_n} \sqrt[n]{a_{n+1}\cdot \cdot \cdot a_m}}$&\\\\
			  &$\leq$&$\dfrac{\sqrt[n]{a_1 \cdot \cdot \cdot a_n} + \sqrt[n]{a_{n+1}\cdot \cdot \cdot a_m}}{2}$& ya que $G_2 \leq A_2$\\\\
			  &$\leq$&$\dfrac{\dfrac{a_1 + ... + a_n}{n} + \dfrac{a_{n+1} + ... + a_m}{n}}{2}$&\\\\
			  &$=$&$\dfrac{a_1 + ... + a_m}{2n}$&\\\\
			  &$=$&$A_m$&\\\\
		      \end{tabular}
		  \end{center}

                %------------------(c)---------------------------
             \item Para un $n$ general, sea $2_m>n.$ Aplíquese la parte $(b)$ a los $2_m$ números 
                $$a_1,\cdot \cdot \cdot , a_n,\,\,\,\, \underbrace{A_n,\cdot \cdot \cdot A_n}_{{2^{m}-n \;veces}}$$
                para demostrar que $G_n \leq A_n$\\\\
		  Demostrar.- \; Aplicando la parte $(b)$  a los $2^m$ números, para $k=2^m-n$
		  \begin{center}
		      \begin{tabular}{rcl}
			  $(a_1 \cdot \cdot \cdot a_n)(A_n)^k$&$\leq$&$\left[\dfrac{a_1 + ... + a_n + k A_n}{2^m}\right]^{2^m}$\\\\
			  &$=$&$\left[\dfrac{n A_n + k A_n}{2^m}\right]^{2^m}$\\\\
			  &$=$&$(A_n)^{2^m}$\\\\
		      \end{tabular}
		  \end{center}
		  así,
		  $$a_1\cdot \cdot \cdot a_n \leq (A_n)^{2^m - k} = (A_n)^n$$\\\\
          \end{enumerate}

      %----------------------------------------------23.----------------------------------------
       \item Lo que sigue es una definición recursiva de $a_n:$\\
          \begin{center}
             \begin{tabular}{rcl}
                $a_1$&$=$&$a$\\
                $a_{n+1}$&$=$&$a_n \cdot a$\\
             \end{tabular}
          \end{center}
          Demostrar por inducción que
          \begin{center}
             \begin{tabular}{rcl}
                $a_{n+m}$&$=$&$a_n \cdot a_m$\\
                $(a^n)^m$&$=$&$a^{nm}$\\
             \end{tabular}
          \end{center}
	  Demostración.-\; La primera ecuación es verdad para $m=1$ ya que $a^{n+1}=a^n \cdot a^1$. Supongamos que $a^{n+m}=a^n \cdot a^m$, entonces 
	  \begin{center}
	      \begin{tabular}{rcl}
		  $a^{n+(n+1)}$&$=$&$a^{(n+1)+1} \cdot a$\\
		  &$=$&$(a^n \cdot a^m)\cdot a$\\
		  &$=$&$a^n \cdot (a^m \cdot a)$\\
		  &$=$&$a^n \cdot a^{m+1}$\\
	      \end{tabular}
	  \end{center}
	  por lo tanto la primera ecuación es verdad para $m+1$\\\\
	  La segunda ecuación es verdad para $m=1$ ya que $(a^n)^1=a^{n\cdot 1}$. Supóngase que $(a^n)^m=a^{nm}$, entonces, 
	  \begin{center}
	      \begin{tabular}{rcl}
		  $$&$=$&$$\\
		  &$=$&$(a^n \cdot a^m)\cdot a$\\
		  &$=$&$a^n \cdot (a^m \cdot a)$\\
		  &$=$&$a^n \cdot a^{m+1}$\\\\
	      \end{tabular}
	  \end{center}

      %--------------------------------------------24.--------------------------------------
       \item Supóngase que conocemos las propiedades P1 y P4 de los números naturales, pero que no se ha hablado de multiplicación. Entonces se puede dar la siguiente definición recursiva de multiplicación: $$1\cdot b=b\,\,\,\,\,\,\,\,\, (a+1)\cdot b =a\cdot b +b$$
          Demostrar lo siguiente (¡en el orden indicado!)
          \begin{center}
             \begin{tabular}{rcll}
                $a\cdot (a+c)$&$=$&$a\cdot b + a\cdot c$&Utilizar inducción sobre $a$\\
                $a \cdot 1$&$=$&$a$&\\
                $a \cdot b$&$=$&$b\cdot a$&lo anterior era el caso $b=1$\\\\
             \end{tabular}
          \end{center}
          Demostración.-\; Al ser 
	  \begin{center}
	      \begin{tabular}{rcll}
		  $1\cdot (b+c)$&$=$&$b+c$&\\
		  &$=$&$1\cdot b + 1 \cdot c$&por definición,\\
	      \end{tabular}
	  \end{center}
	  el primer resultado es válido para $a=1$. Supóngase que $a\cdot(b+c)=a\cdot b + a\cdot c$ para todo $b$ y $c$. Entonces,
	  \begin{center}
	      \begin{tabular}{rcl}
		  $(a+1)\cdot (b+c)$&$=$&$a\cdot (b+c)+(b+c)$\\
		  &$=$&$(a\cdot b + a\cdot c)+(b+c)$\\
		  &$=$&$(a\cdot b + b) + (a\cdot c + c)$\\
		  &$=$&$(a+1)\cdot b + (a+1)\cdot c$\\
	      \end{tabular}
	  \end{center}
	  La ecuación $a\cdot 1 = a$ vale para $a=1$ por definición. Supóngase que $a\cdot 1 = a$. Entonces
	  \begin{center}
	      \begin{tabular}{rcl}
		  $(a+1)\cdot 1$&$=$&$a\cdot 1 + 1 \cdot 1$\\
		  &$=$&$a+1$\\
	      \end{tabular}
	  \end{center}
	  Para $b=1$, la ecuación $a\cdot b=b\cdot a$ es consecuencia de $a\cdot 1=a$, que acaba de ser demostrada, y de $1\cdot a = a$, que vale por definición. Supóngase que $a\cdot b=b\cdot a$, entonces,
	  \begin{center}
	      \begin{tabular}{rcl}
		  $a\cdot (b+1)$&$=$&$a\cdot b + a\cdot 1$\\
		  &$=$&$a\cdot b + a$\\
		  &$=$&$b\cdot a + a$\\
		  &$=$&$(b+1)\cdot a$\\\\
	      \end{tabular}
	  \end{center}

      %------------------------------------------25.------------------------------------------------
       \item En este capítulo hemos empezado con los números naturales y gradualmente hemos ido ampliando hasta los reales. Un estudio completamente riguroso de este proceso requiere de por sí un pequeño libro. Nadie ha encontrado la manera de llegar a los números reales como dados, entonces los números naturales pueden ser definidos como los números naturales de la forma $1,1+1,1+1+1$, etc. Todo el objeto de este problema consiste en hacer ver que existe una manera matemática riguroso de decir \textbf{etc.} 
          \begin{enumerate}[\bfseries (a)]
             %--------------------(a)---------------------
             \item Se dice que un conjunto $A$ de números reales es \textbf{inductivo} si
                \begin{enumerate}[\bfseries (i)]
		    %----------(i)----------
                   \item $\mathbb{R}$ es inductivo.\\\\
                      Demostración.-\;  Está claro según la definición.\\\\
		    %----------(ii)----------
                   \item El conjunto de los números reales positivos es inductivo.\\\\
                      Demostración.-\; Esto está claro, ya que $1$ es positivo, y si $k$ es positivo, entonces por definición $k+1$ es positivo.\\\\
		    %---------(iii)---------
                   \item El conjunto de los números reales positivos distintos de $\dfrac{1}{2}$ es inductivo.\\\\
                      Demostración.-\; Está claro que $1$ pertenece a este conjunto. Si para el mismo no se cumpliera la condición $2$, existiría entonces en el conjunto algún $k$ con $k+1=1/2$. Pero esto es falso, ya que $k=-1/2$ no es positivo.\\\\
		    %----------(iv)----------
                   \item El conjunto de los números reales positivos distintos de $5$ no es inductivo.\\\\
                      Demostración.-\; Este conjunto contiene $4$, pero no $4+1$.\\\\
		    %----------(v)----------
                   \item Si $A$ y $B$ son inductivos, entonces el conjunto $C$ de los números reales que están a la vez en $A$ y en $B$ es también inductivo.\\\\
                      Demostración.-\; Al estar $1$ en $A$ y en $B$, también está en $C$. Si $k$ está en $C$, entonces $k$ está a la vez en $A$ y en $B$, con lo que $k+1$ está en $A$ y en $B$, de modo que $k+1$ está en $C$.\\\\
                \end{enumerate}

            %--------------------(b)------------------------
             \item Un número real $n$ será llamado \textbf{número natural} si $n$ está en todo conjunto inductivo.
                \begin{enumerate}[\bfseries (i)]
		    %----------(i)----------
                   \item Demostrar que $1$ es un número natural.\\\\
                      Demostración.-\; $1$ es un número natural, puesto que $1$ está en todo conjunto inductivo, por la misma definición de conjunto inductivo.\\\\

	 	    %----------(ii)---------
                   \item Demostrar que $k+1$ es un  número natural si $k$ es un número natural.\\\\
                      Demostración.-\; Si $k$ es un número natural, entonces $k$ está en todo conjunto inductivo. Así pues, $k+1$ está en todo conjunto inductivo. Por lo tanto, $k+1$ es un número natural.\\\\
                \end{enumerate}
          \end{enumerate}

          %----------------------------------26.------------------------------------------
       \item Un rompecabezas consiste en disponer de tres vástagos cilíndricos, el primero de los cuales lleva engastados $n$ anillos concéntricos de diámetro decreciente. Se puede quitar el anillo superior de un vástago para engastarlo sobre otro vástago siempre que al hacer esto último el anillo desplazado no venga a caer sobre otro de diámetro inferior. Por ejemplo, si el anillo más pequeño se pasa al vástago 2 y el que le sigue pasar también al vástago 3 encima del que le sigue en tamaño. Demostrar que la pila completa se puede pasar al vástago 3 en $2_n+1$ pasos  y no en menos.\\\\
          Demostración.-\; Si hay solo $n=1$ anillos, claramente se puede mover al eje $3$ en $1=2^1 - 1$ movimientos. Suponiendo el resultado para $k$ anillos, luego dados $k+1$ anillos,
	  \begin{enumerate}[\bfseries (a)]
	  	\item Mueve los anillos elevados a la $k$ al eje 2 en $2k-1$ movimientos,
		\item mueva el anillo inferior al eje 3,
		\item mueva los $k$ anillos superiores de nuevo al eje $3$ en movimientos $2k-1$.
	  \end{enumerate}
	Esto toma $2(2k - 1) + 1 = 2k + 1 - 1$ se mueve. Si $2k - 1$ movimientos es el mínimo posible para $k$ anillos, luego $2k + 1- 1$ es el mínimo para $k + 1$ anillos, ya que la parte inferior El anillo no se puede mover en absoluto hasta que los primeros $k$ anillos se muevan a algún lugar, tomando al menos $2k - 1$ se mueve, el anillo inferior debe moverse al eje $3$, tomando al menos $1$ movimiento, y luego los otros anillos deben colocarse encima, tomando al menos otros movimientos $2^k  - 1$.\\\\

          %.-------------------------------27.------------------------------------------
       \item Hubo un tiempo en que la universidad $B$ se preciaba de tener $17$ profesores numerarios de matemáticas. La tradición obligaba a que el almuerzo comunitario semanal, al que concurrían fielmente los $17$, todo miembro que hubiese descubierto un error en una de sus publicaciones tenia que hacer público este hecho y a continuación dimitir. Una declaración de este tipo no se había producido nunca porque ninguno de los profesores era consciente de la existencia de un error en su propio trabajo. Lo cual, sin embargo, mo quiere decir que no existieran errores. De hecho, en el transcurso de los años, por lo menos un error había sido descubierto en el trabajo de cada uno de los miembros por otro de entro ellos.La existencia de este error había sido comunicada a todos los demás miembros del departamento salvo al responsable, con objeto de evitar dimisiones.\\
          Llegó un fatídico año en que el departamento aumentó el número de sus miembros con un visitante de otra universidad, un Profesor $X$ que venía con la esperanza de que se le ofreciera un puesto permanente al final del año académico. Una vez que vio frustrada su esperanza, el Profesor $X$ tomó su venganza en el último almuerzo comunitario del año diciendo: Me ha sido muy grata mi estancia entre ustedes, pero hay una cosa que creo que es mi deber comunicarles. Por lo menos uno de entre ustedes tiene publicado un resultado incorrecto, lo cual ha sido descubierto por otro del departamento. ¿Qué ocurrió al año siguiente?\\\\
          Respuesta.-\; Primero Suponga que solo hay $2$ profesores $A$ y $B$, cada uno consciente del error en el trabajo del otro, pero sin darse cuenta de cualquier error en el suyo. Entonces ninguno se sorprende por la declaración del profesor $X$, pero cada uno espera que el otro sea sorprendido, y dimitir en el primer almuerzo del próximo año. Cuando esto no suceda, cada uno se da cuenta que esto solo puede ser porque él también ha cometido un error. Entonces, el la próxima reunión, ambos renunciaran.\\
	  A continuación, considere el caso de 3 profesores, $A$, $B$ y $C$. El profesor $C$ sabe que el profesor $A$ es consciente de un error en el trabajo del profesor $B$, ya sea porque el profesor $A$ encontró el error e informó, o porque encontró el error e informó al profesor $A$. Del mismo modo, él sabe que el profesor $B$ sabe que hay un error en el trabajo del profesor $A$. Pero el profesor $C$ piensa que no a cometido errores, por lo que a el respecta, la situación frente a los profesores $A$ y $B$ es precisamente el analizado en el párrafo anterior. El profesor $C$ está asumiendo, de que nadie cree que exista un error cuando uno no lo hace. Entonces el profesor $C$ espera tanto al profesor $A$ como al profesor $B$ renunciar en la segunda reunión. Por supuesto de manera similar los profesores $A$ y $B$ esperan que los otros dos renuncien en la segunda reunión. Cuando nadie renuncia todos se dan cuenta de que ha cometido un error, por lo que todos renuncian en la tercera reunión. Podría ser demostrado por inducción.\\\\ 

          %-------------------------------------28--------------------------------------------
       \item Después de imaginarse, o de consultar, la solución del problema 27, considere lo siguiente: Cada uno de los miembros del departamento era ya sabedor de lo que el Profesor $X$ afirmaba. ¿Cómo pudo pues su afirmación cambiar las cosas?\\\\
          Respuesta.-\; Ganar es una buena idea comenzar con el caso en el que el departamento consta solo de Profesores $A$ y $B$. Ahora, por supuesto, ambos profesores saben que alguien ha publicado un resultado incorrecto, pero el Profesor $A$ piensa que el Profesor $B$ no lo sabe, y viceversa. Una vez que el Profesor $X$ hace su anuncio, el Profesor $A$ sabe que el Profesor $B$ lo sabe. Y por eso espera que el Profesor $B$ renuncie en la próxima reunión. En el caso de tres profesores, la situación es más complicada. Cada uno sabe que alguien ha cometido un error, y además cada uno sabe que los demás saben Por ejemplo, el Profesor $C$ sabe que el Profesor $A$ lo sabe, ya que él y el Profesor $A$ han discutido el error en el trabajo del Profesor $B$, y él sabe de manera similar que el Profesor $B$ lo sabe. Pero el Profesor $C$ no cree que el Profesor $A$ sepa que el Profesor $B$ lo sabe. Así que el anuncio del Profesor $X$ cambia las cosas: ahora el Profesor $C$ sabe que el Profesor $A$ sabe que el Profesor $B$ sabe. Bueno, puedes ver lo que pasa en general. Esto parece probar que las declaraciones como $A$ sabía que $B$ sabía que $C$ sabía que realmente tiene sentido.\\\\


    \end{enumerate}
    

