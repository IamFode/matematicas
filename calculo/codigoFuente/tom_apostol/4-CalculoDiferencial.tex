\chapter{Cálculo diferencial}

\setcounter{section}{3}
\section{Derivada de una función}

\begin{tcolorbox}
    \begin{def.}[Definición de derivada]
	La derivada $f'(x)$ está definida por la igualdad 
	$$f'(x)=\lim_{h\to 0}\dfrac{f(x+h)-f(x)}{h},$$
	siempre que exista el límite. El número $f'(x)$ también se denomina coeficiente de variación de $f$ en $x$.
    \end{def.}
\end{tcolorbox}
\vspace{.7cm}

\begin{ejem}[Derivada de una función potencial de exponente entero positivo]
    Consideremos el caso $f(x)=x^n$, siendo $n$ un entero positivo. El cociente de diferencias es ahora\\
    $$\dfrac{f(x+h)-f(x)}{h}=\dfrac{(x+h)^n-x^n}{h}$$\\
    Para estudiar este cociente al tender $h$ a cero, podemos proceder de dos maneras, o por la descomposición factorial del numerador considerado como diferencia de dos potencias n-simas o aplicando el teorema del binomio para el desarrollo de $(x + h)^n$. Seguiremos con el primer método.\\

    En álgebra elemental se tiene la identidad
    $$a^n - b^n = (b-a)\sum_{k=0}^{n-1} a^k b^{n-1-k}$$
    Si se toma $a=x+h$ y $b=x$ y dividimos ambos miembros por $h$, esa identidad se transforma en \\
    $$\dfrac{(x+h)^n-x^n}{h} = \sum_{k=0}^{n-1}(x+h)^k x^{n-1-k}$$\\
    En la suma hay $n$ términos. Cuando $h$ tiende a $0$, $(x+h)^k$ tiende a $x^k$, el k-ésimo término tiende a $x^k x^{n-1-k}=x^{n-1},$ y por tanto la suma de los $n$ términos tiende a $nx^{n-1}$. De esto resulta que \\
    $$f'(x)=nx^{n-1}\quad \forall \; x.$$\\
\end{ejem}

\begin{ejem}[Derivada de la función seno]
    Sea $s(x)=\sen x$. El cociente de diferencias es
    $$\dfrac{s(x+h)-s(x)}{h}=\dfrac{\sen(x+h)-\sen x}{h}$$
    Para transformarlo de modo que haga posible calcular el límite cuando $h\to 0$, utilizamos la identidad trigonométrica
    $$\sen y -\sen x = 2\sen \dfrac{y-x}{2}\cos \dfrac{y+x}{2}$$
    poniendo $y=x+h$. Esto conduce a la fórmula
    $$\dfrac{\sen(x+h)-\sen x}{h}=\dfrac{\sen\frac{h}{2}}{\frac{h}{2}} \cos \left(x+\dfrac{h}{2}\right)$$
    Como $h\to 0$, el factor $\cos(x+\frac{1}{2}h)\to \cos x$ por la continuidad del coseno. Así mismo, la fórmula 
    $$\lim_{x\to 0}\dfrac{\sen x}{x}=1,$$
    demuestra que
    $$\dfrac{\sen \frac{h}{2}}{\frac{h}{2}}\to 1\;\mbox{ para todo } \; h\to 0$$
    Por lo tanto el cociente de diferencias tiene como límite $\cos x$ cuando $h\to 0$. Dicho de otro modo, $s'(x)=\cos x;$ para todo $x$; la derivada de la función seno es la función coseno.\\\\ 
\end{ejem}

\begin{ejem}[Derivada de la función coseno]
    Sea $c(x)=\cos x$. Demostraremos que $c'(x)=-\sen x$; esto es, la derivada de la función coseno es menos la función seno. Partamos de la identidad
    $$\cos y - \cos x = -2\sen \dfrac{y-x}{2}\sen \dfrac{y+x}{2}$$
    y pongamos $y=x+h$. Esto nos conduce a la fórmula
    $$\dfrac{\cos(x+h)-\cos x}{h}=-\dfrac{\sen \frac{h}{2}}{\frac{h}{2}}\sen \left(x+\dfrac{h}{2}\right).$$
    La continuidad del seno demuestra que $\left(x+\frac{1}{2}h\right)\to x$ cuando $h\to 0$; luego ya que 
    $$\dfrac{\sen \frac{h}{2}}{\frac{h}{2}}\to 1\;\mbox{ para todo } \; h\to 0$$
    obtenemos $c'(x)=-\sen x$.\\
\end{ejem}

\begin{ejem}[Derivada de la función raíz n-esima]
    Si $n$ es un entero positivo, sea $f(x)=x^{1/n}$ para $x>0$. El cociente de diferencias para $f$ es
    $$\dfrac{f(x+h)-f(x)}{h}=\dfrac{(x+h)^{1/n}-x^{1/n}}{h}.$$
    Pongamos $u=(x+h)^{1/n}$ y $v=x^{1/n}$. Tenemos entonces $u^n = x+h$ y $v^n=x,$ con lo que $h=u^n-v^n$, y el cociente de diferencias toma la forma
    $$\dfrac{f(x+h)-f(x)}{h}=\dfrac{u-v}{u^n -v^n}=\dfrac{1}{u^{n-1}+u^{n-2}v + \ldots + uv^{n-2}v^{n-1}}$$
    La continuidad de la función raíz n-sima prueba que $u \to v$ cuando $h \to O$. Por consiguiente cada término del denominador del miembro de la derecha tiene límite $v^{n-1}$ cuando $h \to O$. En total hay $n$ términos, con lo que el cociente de diferencias tiene como límite $v^{1-n}/n$. Puesto que  $v = x^{1/n}$, esto demuestra que
    $$f'(x)=\dfrac{1}{n}x^{1/n-1}.$$\\
\end{ejem}

\begin{tcolorbox}
    \begin{ejem}[Continuidad de las funciones que admiten derivada]
	Si una función $f$ tiene derivada en un punto $x$, es también continua en $x$. Para demostrar, empleamos la identidad
	$$f(x+h)=f(x)+h\left(\dfrac{f(x+h)-f(x)}{h}\right)$$
	que es válida para $h\neq 0$. Si hacemos que $h\to 0$, el cociente de diferencias del segundo miembro tiende a $f'(x)$, puesto que este cociente está multiplicando por un factor que tiende a $0$, el segundo término del segundo miembro tiende a $0\cdot f'(x)$. Esto demuestra que $f(x+h)\to f(x)$ cuando $h\to 0$ y por tanto que $f$ es continua en $x$.
    \end{ejem}
\end{tcolorbox}

\section{Álgebra de las derivadas}

\begin{teo}
    Sean $f$ y $g$ dos funciones definidas en un intervalo común. En cada punto en que $f$ y $g$ tienen derivadas, también las tienen la suma $f+g$
\end{teo}
