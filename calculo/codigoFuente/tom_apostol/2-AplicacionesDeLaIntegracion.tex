\chapter{Algunas aplicaciones de la integración}

\setcounter{section}{1}
\section{El área de una región comprendida entre dos gráficas expresada como una integral}

%--------------------2.1
\begin{teo} Supongamos que $f$ y $g$ son integrables y que satisfacen $f\leq q$ en $[a,b]$. La región $S$ entre sus gráficas es medible y su área $a(S)$ viene dada por la integral $$a(S) = \int_a^b [g(x)-f(x)] \; dx$$
    Demostración.- \; 
	Demostración.- \; Supongamos primero que $f$ y $g$ son no negativas,. Sean $F$ y $G$ los siguientes conjuntos:
	$$F = {(x,y)|a\leq x \leq b, 0\leq y \leq f(x)}, \quad G = {(x,y) | a\leq x \leq b, 0\leq y \leq g(x)}.$$
	Esto es, $G$ es el conjunto de ordenadas de $g$, y $F$ el de $f$, menos la gráfica de $f$. La región $S$ es la diferencia $G-F$. Según los teoremas 1.10 y 1.11, $F$ y $G$ son ambos medibles. Puesto que $F \subseteq G$ la diferencia $S = G-F$ es también medible, y se tiene 
	$$a(S) = a(G) - a(F) = \int_a^b g(x) \; dx = \int_a^b [g(x)-f(x)] \; dx$$
	Consideremos ahora el caso general cuando $f\leq q$ en $[a,b]$, pero no son necesariamente no negativas. Este caso lo podemos reducir al anterior trasladando la región hacia arriba hasta que quede situada encima del eje $x$. Esto es, elegimos un número positivo $c$ suficientemente grande que asegure que $0 \leq f(x) + c \leq g(x) + c$ para todo $x$ en $[a,b]$. Por lo ya demostrado la nueva región $T$ entre las gráficas de $f+c$ y $g+c$ es medible, y su parea viene dad por la integral
	$$a(T) = \int_a^b [(g(x)+c) - (f(x)+c)] = \int_[g(x)-f(x)] \; dx$$
	Pero siendo $T$ congruente a $S$, ésta es también medible y tenemos $$a(S) = a(T) = \int_a^b [g(x) - f(x)]\; dx$$
	Esto completa la demostración.\\\\
\end{teo}

%----------nota 2.1
\begin{tcolorbox}[colframe = white]
    \begin{nota} En los intervalos $[a,b]$ puede descomponerse en un número de subintervalos en cada uno de los cuales $f\leq g$ o $g\leq f$ la fórmula (2.1) del teorema 2.1 adopta la forma 
    $$a(S) = \int_a^b |g(x) -f(x)| \; dx$$
    \end{nota}
\end{tcolorbox}

    %--------------------lema 2.1
    \begin{lema}[Área de un disco circular] Demostrar que $A(r) = r^2 A(1)$. Esto es, el área de un disco de radio $r$ es igual al producto del área de un disco unidad (disco de radio $1$) por $r^2$.\\\\
	Demostración.-\; Ya que $g(x) - f(x) = 2g(x),$ el teorema 2.1 nos da 
	    $$A(r) = \int_{-r}^r g(x) \; dx = 2 \int_{-r}^r \sqrt{r^2 - x^2} \; dx$$
	    En particular, cuando $r = 1$, se tiene la fórmula $$A(1) = 2\int_{-1}^1 \sqrt{1 - x^2} \; dx$$
	    Cambiando la escala en el eje $x$, y utilizando el teorema 1.19 con $k=1/r$, se obtiene
	    $$A(r) = 2\int_{-r}^r g(x) \; dx = 2r \int_{-1}^1 g(rx) \; dx = 2r\int_{-1}^1 \sqrt{r^2 - (rx)^2} \; dx = 2r^2 \int_{-1}^1 \sqrt{1-x^2} \; dx = r^2 A(1)$$
	    Esto demuestra que $A(r) = r^2 A(1)$, como se afirmó.\\\\
    \end{lema}

%---------------------definición 2.1
\begin{tcolorbox}[colframe = white]
    \begin{def.} Se define el número $\pi$ como el área de un disco unidad.
	$$\pi = 2 \int_{-1}^1 \sqrt{1-x^2}\; dx$$
    \end{def.}
\end{tcolorbox}
\begin{center}
    La formula que se acaba de demostrar establece que $A(r) = \pi r^2$\\
\end{center}

\begin{tcolorbox}[colframe = white]
Generalizando el anterior lema se tiene 
    $$a(kS) = \int_{ka}^{kb} g(x)\; dx = k \int_{ka}^{kb} f(x/k) \; dx = k^2 \int_a^b f(x) \; dx$$
\end{tcolorbox}

%--------------------teorema 2.1
\begin{teo} Para $a>0$, $b>0$ y $n$ entero positivo, se tiene $$\int_a^b x^{\frac{1}{n}} \; dx = \dfrac{b^{1-1/n} - a^{1-\frac{1}{n}}}{1+\frac{1}{n}}$$\\
    Demostración.-\; Sea $\int_0^a x^{\frac{1}{n}}$. El rectángulo de base $a$ y altura $a^{\frac{1}{n}}$ consta de dos componentes: el conjuntos de ordenadas de $f(x) = x^{\frac{1}{n}}$ a $a$ y el conjuntos de ordenadas $g(y) = y^n$ a $a^{\frac{1}{n}}$. Por lo tanto,
    $$a\cdot a^{\frac{1}{n}} = a^{1+\frac{1}{n}} = \int_0^a x^{\frac{1}{n}} \; dx + \int_0^{a^{\frac{1}{n}}} y^n \; dy \; \Longrightarrow \; \int_0^a x^{\frac{1}{n}} \; dx = a^{1+\frac{1}{n}} - \dfrac{y^{n+1}}{n+1}\bigg|_0^{a^{\frac{1}{n}}} = a^{1+\frac{1}{n}} -  \dfrac{a^{1+\frac{1}{n}}}{n+1} = \dfrac{a^{1+\frac{1}{n}}}{1 + 1/n}$$
    Análogamente se tiene $$\int_0^b x^{\frac{1}{n}}\; dx = \dfrac{b^{1 + \frac{1}{n}}}{1 + 1/n}$$
    Luego notemos que $$\int_a^b x^{\frac{1}{n}} \; dx = \int_0^b x^{\frac{1}{n}}\; dx - \int_0^a x^{\frac{1}{n}} \;dx$$
    por lo tanto $$\int_a^b x^{\frac{1}{n}}\; dx = \dfrac{b^{1+\frac{1}{n}} - b^{1 + \frac{1}{n}}}{1 + 1/n}$$\\\\
\end{teo}



\setcounter{section}{3}
\section{Ejercicios}

En los ejercicios del 1 al 14, calcular el área de la región $S$ entre las gráficas de $f$ y $g$ para el intervalo $[a,b]$ que en cada caso se especifica. Hacer un dibujo de las dos gráficas y sombrear $S$.\\

\begin{enumerate}[\Large\bfseries 1.]

%--------------------1.
\item $f(x) = 4 - x^2, \quad g(x)=0, \quad a = -2, \quad b = 2$\\\\
    Respuesta.-\; $$\int_{-2}^2 [4-x^2 - 0] \; dx = 4x \bigg|_{-2}^2 -\dfrac{x^3}{3}\bigg|_{-2}^2 = 4(2-(-2)) - \left(\dfrac{2^3 - (-2)^3}{3}\right)  = \dfrac{32}{3}$$\\ 

%--------------------2.
\item $f(x) = 4 - x^2, \quad g(x) = 8 - 2x^2,\quad a = -2, \quad b = 2.$\\\\
    Respuesta.-\; $$\int_{-2}^2 [8 - 2x^2 - (4 - x^2)] \; dx = \int_{-2}^2 4-x^2 \; dx = \dfrac{32}{3} \; (por \, ejercicio \; 1)$$\\

%--------------------3.
\item $f(x)=x^3+x^2,\quad g(x)=x^3 + 1, \quad a=-1, \quad b=1$.\\\\
    Respuesta.-\; $$\int_{-1}^1 x^3 + 1 - (x^3 + x^2) \;dx = \int_{-1}^1 1-x^2\; dx = x\bigg|_{-1}^1 - \dfrac{x^3}{3}\bigg|_{-1}^1   = 2 - \dfrac{1-(-1)}{3} = \dfrac{4}{3}$$\\

%--------------------4.
\item $f(x)=x-x^2,\quad g(x)=-x,\quad a=0,\quad b=2$\\\\
    Respuesta.-\; $$\int_0^2 x-x^2 - (-x) \; dx = \int_0^2 2x - x^2 = 2\dfrac{x^2}{2}\bigg|_0^2 - \dfrac{x^3}{3}\bigg|_0^2 = 2\dfrac{2^2}{2} - \dfrac{2^3}{3} = \dfrac{4}{2}$$\\

%--------------------5.
\item $f(x) = x^{1/3}, \quad g(x) = x^{1/2}, \quad a=0, \quad b=1$\\\\ 
    Respuesta.-\; $$\int_0^1 x^{1/3} - x^{1/2} \; dx = \dfrac{x^{1+1/3}}{1+1/3}\bigg|_0^1 - \dfrac{x^{1+1/2}}{1+1/2}\bigg|_0^1 = \dfrac{3}{4} - \dfrac{2}{3} = \dfrac{1}{12}$$\\

%--------------------6.
\item $f(x) = x^{1/3}, \quad g(x) = x^{1/2},\quad a=1,\quad b=2.$\\\\
    Respuesta.-\; $$\int_1^2 x^{1/2}-x^{1/3}\; dx = \dfrac{x^{1/2 + 1}}{1 + 1/2}\bigg|_1^2 - \dfrac{x^{1/3}+1}{1+1/3}\bigg|_1^2 = \dfrac{2^{1/2+1}-1}{1+1/2}-\dfrac{2^{1/3+1}}{1+1/3} = \dfrac{4\sqrt{2}}{3}-\dfrac{3\sqrt[3]{2}}{2}+\dfrac{1}{12}$$\\

%--------------------7.
\item $f(x)=x^{1/3},\quad g(x) = x^{1/2}, \quad a = 0,\quad b=2$\\\\
    Respuesta.-\; Sea $$\int_0^1 |x^{1/3}-x^{1/2}|\; dx + \int_1^2 |x^{1/3}-x^{1/2}|\; dx$$
    por los problemas 5 y 6 se tiene $$\dfrac{1}{12} + \dfrac{4\sqrt{2}}{3}-\dfrac{3\sqrt[3]{2}}{2}+\dfrac{1}{12} = \dfrac{4\sqrt{2}}{3}-\dfrac{3\sqrt[3]{2}}{2}+\dfrac{1}{6}$$\\

%--------------------8.
\item $f(x) = x^{1/2}, \quad g(x) = x^2, \quad a=0, \quad b=2$\\\\
    Respuesta.-\;
    \begin{center}
	\begin{tabular}{rcl}
	    $\displaystyle\int_0^1 x^{1/2} - x^2 \; dx + \int_1^2 x^2 - x^{1/2}\; dx$ & $=$ & $\left(\dfrac{x^{1+1/2}}{1+1/2}\bigg|_0^1 - \dfrac{x^3}{3}\bigg|_0^1\right) + \left( \dfrac{x^3}{3}\bigg|_1^2 - \dfrac{x^{1+1/2}}{1+1/2}\bigg|_1^2 \right)$\\\\
	    & $=$ & $\left(\dfrac{1}{1+1/2} - \dfrac{1}{3}\right) + \left(\dfrac{2^3-1}{3} - \dfrac{2^{1+1/2} - 1}{1+1/2}\right)$\\\\
	    & $=$ & $\dfrac{2}{3} - \dfrac{1}{3} + \dfrac{7}{3} - \dfrac{4\sqrt{2}-2}{3}$\\\\
	    &$=$&$\dfrac{10}{3}-\dfrac{4\sqrt{2}}{3}$\\\\
	\end{tabular}
    \end{center}

%--------------------9.
\item $f(x) = x^2, \quad g(x) = x+1, \quad a=-1, \quad b = (1+\sqrt{5})/2$\\\\
    Respuesta.$$\int_{-1}^{(1-\sqrt{5})/2} x^2 - (x+1)\; dx +  \int_{(1-\sqrt{5})/2}^{(1+\sqrt{5})/2} (x+1) - x^2\; dx-\; = $$
    \begin{center}
	\begin{tabular}{rcl}
	     & $=$ & $\displaystyle\int_{-1}^{(1-\sqrt{5})/2} x^2 - x - 1\; dx +  \int_{(1-\sqrt{5})/2}^{(1+\sqrt{5})/2} x + 1 - x^2 \; dx$ \\\\
	    & $=$ & $\left(\dfrac{x^3}{3} - \dfrac{x^2}{2} - x\right)\bigg|_{-1}^{(1-\sqrt{5})/2} + \left(\dfrac{x^2}{2} + x - \dfrac{x^3}{3}\right)\bigg|_{(1-\sqrt{5})/2}^{(1+\sqrt{5})/2}$ \\\\
	    & $=$ & $-\dfrac{3}{4} + \dfrac{5 \sqrt{5}}{12}+\dfrac{5\sqrt{5}}{6}$ \\\\
	    & $=$ & $\dfrac{5\sqrt{5}-3}{4}$\\\\
	\end{tabular}
    \end{center}

%--------------------10.
\item $f(x)=x(x^2-1),\quad g(x)=x,\quad a=-1, \quad b=\sqrt{2}$\\\\
    Respuesta.-\; $$\int_{-1}^{0} x(x^2-1) - x \; dx + \int_{0}^{\sqrt{2}} x - [x(x^2-1)] \; dx = \int_{-1}^{0} x^3-2x \; dx + \int_{0}^{\sqrt{2}} - x^3 + 2x \; dx = $$
	$$=\left(\dfrac{x^4}{4} - x^2 \right)\bigg|_{-1}^0 + \left( -\dfrac{x^4}{4} + x^2 \right)\bigg|_{0}^{\sqrt{2}} = -\dfrac{1}{4} + 1 + (-1+2) = \dfrac{7}{4}$$\\

%--------------------11.
    \item $f(x)=|x|,\quad g(x) = x^2-1, \quad a=-1,\quad b=1$\\\\
	Respuesta.-\; Definimos $f$ de la siguiente manera: $$f(x) = |x| = \left\{\begin{array}{rcl}
	    -x&si&x\in [-1,0)\\
	    x&si&x\in [0,1]\\
	    \end{array}\right.$$
	    Luego,
	    \begin{center}
	    \begin{tabular}{rcl}
		$\displaystyle\int_{-1}^1 f(x)-g(x) \; dx$ & $=$ & $\displaystyle\int_{-1}^0 -x-x^2+1 \; dx + \int_0^1 x-x^2+1 \; dx$\\\\
		& $=$ & $ \left(-\dfrac{x^2}{2}-\dfrac{x^3}{3} + x \right) \bigg|_{-1}^0 + \left(\dfrac{x^2}{2} - \dfrac{x^3}{3} + x\right)\bigg|_0^1 $\\\\
		& $=$ & $\left(\dfrac{1}{2} - \dfrac{1}{3}+1\right)+\left(\dfrac{1}{2}-\dfrac{1}{3} + 1\right)$\\\\
		& $=$ & $\dfrac{7}{3}$ \\\\
	    \end{tabular}
	    \end{center}

%--------------------12.
\item $f(x) = |x+1|,\quad g(x)=x^2-2x, \quad a=0, \quad b=2$\\\\
    Respuesta.- \; Definamos $f$ de la siguiente manera:
	$$f(x)= |x-1| = \left\{
	    \begin{array}{rcl}
		-(x+1)&si&x\in[0,1)\\
		x+1&si&x\in [1,2]\\
	    \end{array}
	    \right.$$
	    Entonces, 
	    \begin{center} 
		\begin{tabular}{rcl}
		    $\displaystyle\int_0^2 f(x)-g(x) \; dx$&$=$&$\displaystyle\int_0^1 -(x-1)-x^2+2x\; dx + \int_1^2 x-1-x^2+2x \; dx$\\\\
		    & $=$ &$\displaystyle\int_0^1 -x^2+x+1\; dx + \int_1^2 -x^2+3x - 1 \; dx$\\\\
		    & $=$ & $\left(-\dfrac{x^3}{3} + \dfrac{x^2}{2} + x\right)\bigg|_0^1 + \left(-\dfrac{x^3}{3} + \dfrac{3x^2}{2} -x\right)\bigg|_1^2$\\\\
		    & $=$ & $\left(-\dfrac{1}{3} + \dfrac{1}{2} + 1\right)+\left(-\dfrac{8}{3} + 6-2+\dfrac{1}{3} -\dfrac{3}{2} + 1\right)$\\\\
		    & $=$ & $\dfrac{7}{3}$\\\\
		\end{tabular}
	    \end{center}

%--------------------13.
\item $f(x) = 2|x|, \quad g(x) = 1-3x^3, \quad a=-\sqrt{3}/3, \quad b=\dfrac{1}{3}$\\\\ 
    Respuesta.-\; Definimos $f$ de la siguiente manera: $$f(x) = |x| = \left\{\begin{array}{rcl}
	-x&si&x\in [-\sqrt{3}/3,0)\\
	    x&si&x\in [0,1/3]\\
	    \end{array}\right.$$
	    de donde se tiene,
	    \begin{center}
		\begin{tabular}{rcl}
		    $\displaystyle\int_{-\frac{\sqrt{3}}{3}}^{\frac{1}{3}} g(x)-f(x)\; dx$ & $=$ & $\displaystyle\int_{-\frac{-\sqrt{3}}{3}}^{\frac{1}{3}}g(x) \; dx - \int_{-\frac{\sqrt{3}}{3}}^{\frac{1}{3}} f(x) \; dx$\\\\\
		    & $=$ & $\displaystyle\int_{-\frac{\sqrt{3}}{3}}^{\frac{1}{3}} 1-3x^3 \; dx - \int_{-\frac{\sqrt{3}}{3}}^{0} -2x \; dx - \int_{0}^{\frac{1}{3}} 2x \; dx$\\\\
		    & $=$ & $\left(x-\dfrac{3}{4} x^4\right)\bigg|_{-\frac{\sqrt{3}}{3}}^{\frac{1}{3}} + x^2 \bigg|_{-\frac{\sqrt{3}}{3}}^{0} - x^2 \bigg|_0^{\frac{1}{3}}$\\\\
		    & $=$ & $\left(\dfrac{1}{3} - \dfrac{1}{108} + \dfrac{\sqrt{3}}{3} + \dfrac{1}{12}\right)+\left(-\dfrac{1}{3}\right)-\dfrac{1}{9}$\\\\
		    & $=$ & $\dfrac{9\sqrt{3}-1}{27}$\\\\
		\end{tabular}
	    \end{center}


%--------------------14.
\item $f(x) = |x|+|x-1|, \quad g(x)=0, \quad a=-1,\quad b=2$\\\\
    Respuesta.-\; En este problema $f(x)\geq g(x)$ en el intervalo $[-1,2]$, por lo tanto 
    $$\int_{-1}^2 f(x)-g(x)\; dx = \int_{-1}^2 |x| + |x+1|\; dx = \int_{-1}^2 |x|\; dx + \int_{-1}^2 |x-1| \; dx$$
    Definimos cada función por separado,
    $$|x| = \left\{ \begin{array}{rcl} -x & si & x\in [-1,0)\\ x & si & x\in [0,2] \end{array}\right.$$
    $$|x-1| = \left\{ \begin{array}{rcl} -(x-1) & si & x\in [-1,1)\\ x-1 & si & x\in [1,2] \end{array}\right.$$
    por lo tanto
    \begin{center}
	\begin{tabular}{rcl}
	    $\displaystyle\int_{-1}^2 |x|\; dx + \int_{-1}^2 |x-1|\; dx$&$=$&$\displaystyle\int_{-1}^0 -x \; dx + \int_0^2 x\; dx + \int_{-1}^1 -(x-1)\; dx + \int_1^2 x-1\; dx$\\\\
	    &$=$&$\left(-\dfrac{x^2}{2}\right)\bigg|_{-1}^0 + \left(\dfrac{x^2}{2}\right)\bigg|_{0}^2 + \left(-\dfrac{x^2}{2}\right)\bigg|_{-1}^1 + \left(x\right)\bigg|_{-1}^1 + \left(\dfrac{x^2}{2}\right)\bigg|_{1}^2 + \left(-x\right)\bigg|_{1}^2$\\\\\
	    &$=$&$\dfrac{1}{2}+2-\dfrac{1}{2}+\dfrac{1}{2}+1+1+2-\dfrac{1}{2}-2+1$\\\\\
	    &$=$&$5$\\\\\
	\end{tabular}
    \end{center}

%--------------------15.
\item Las gráficas de $f(x) = x^2$ y $g(x)=cx^3$, siendo $c>0$, se cortan en los puntos $(0,0)$ y $(1/c,1/c^2)$. Determinar $c$ de modo que la región limitada entre esas gráficas y sobre el intervalo $[0,1/c]$ tengan área $\frac{2}{3}$.\\\\
    Respuesta.-\; Tenemos que $f\geq g$ en el intervalo $[0,1/c]$ de donde, 
    $$\int_0^{1/c} x^2 -cx^3 \; dx = \int_0^{1/c} x^2\; dx - c\int_0^{1/c} x^3 \; dx = \dfrac{1}{12c^3}$$
    luego $\dfrac{1}{12c^3}=\dfrac{2}{3}$ por lo tanto $c=\dfrac{1}{2}$.\\\\

%--------------------16.
\item Sea $f(x)=x-x^2$, $g(x) = ax$. Determinar $a$ para que la región situada por encima de la gráfica de $g$ y por debajo de $f$ tenga área $frac{9}{2}$.\\\\
    Respuesta.-\; Tomaremos los casos cuando $a=0, a>0$ y $a<0$. \\
    Veamos primero que si $g(x)\leq f(x)$ entonces $$f(x)-g(x)\geq 0 \Longrightarrow x-x^2-ax\geq 0 \Longrightarrow (1-a)x \geq x^2$$
    de donde si $x=0$ se tiene una igualdad. Luego si $x\neq 0$ entonces $x\leq (1-a)$. Ahora sea $a<0$ por suposición se tendrá $1-a>0$, que nos muestra que el intervalo estará dado por $[0,1-a]$. Análogamente se tiene el intervalo $[1-a,0]$ para $a>0$.\\
    \begin{enumerate}[\bfseries C 1.]
	\item Si $a=0$, esto no es posible ya que si $a = 0$ entonces $g(x) = ax = 0$ y entonces el área arriba del gráfico de $g$ y debajo del gráfico de $f$ es igual a
	    $$\int_0^1 x-x^2\; dx = \left(\dfrac{x^2}{2} - \dfrac{x^3}{3}\right)\bigg|_0^1 = \dfrac{1}{6} \neq \dfrac{9}{2}$$\\
	
	\item Si $a<0$, $f(x)\geq g(x)$ para $[0,1-a]$, por lo que tenemos la zona, $a(S)$ de la región entre las dos gráficas dadas por 
	    \begin{center}
		\begin{tabular}{rcl}
		    $\displaystyle\int_0^{1-a} x-x^2 - ax \; dx$&$=$&$\displaystyle(1-a)\int_0^{1-a} x\; dx - \int_0^{1-a} x^2 \; dx$\\\\
		    &$=$&$(1-a)\left(\dfrac{(1-a)^2}{2}\right) - \dfrac{(1-a)^3}{3}$\\\\
		    &$=$&$-\dfrac{(1-a)^3}{6}$\\\\
		\end{tabular}
	    \end{center}
	    así nos queda que $$-\dfrac{(1-a)^3}{6} = \dfrac{9}{2} \Longrightarrow (1-a)^3 = -27 \Longrightarrow a = a$$\\

	\item Sea $a>0$ y  $f(x)\geq g(x)$ entonces $[1-a,0]$ lo que 
	    \begin{center}
		\begin{tabular}{rcl}
		    $\displaystyle\int_{1-a}^0 x-x^2-ax \; dx$&$=$&$\displaystyle(1-a)\int_{1-a}^0 x\; dx  - \int_{1-a}^0 x^2 \; dx$\\\\ 
		    &$=$&$(1-a)\left(-\dfrac{(1-a)^2}{2} - \dfrac{(1-a)^3}{2}\right)$\\\\
		    &$=$&$-\dfrac{(1-a)^3}{6}$\\\\
		\end{tabular}
	    \end{center}
	    Así igualando por $\frac{9}{2}$ tenemos 
	    $$-\dfrac{(1-a)^3}{6} = \dfrac{9}{2} \Longrightarrow (1-a)^3 = -27 \Longrightarrow a=4$$\\
    \end{enumerate}
    Por lo tanto los valores posibles para $a$ son $-2$ y $4$.\\\\

%--------------------17.
\item Hemos definido $\pi$ como el área de un disco circular unidad. En el ejemplo 3 de la Sección 2.3, se ha demostrado que $\pi=2 \int_{-1}^1 \sqrt{1-x^2}\; dx$. Hacer uso de las propiedades de la integral para calcular la siguiente en función de $\pi$.
\begin{enumerate}[\bfseries (a)]

    %----------(a)
    \item $\displaystyle\int_{-3}^3 \sqrt{9-x^2}\; dx$.\\\\
	Respuesta.-\; Por el teorema 19 de dilatación, $\dfrac{1}{\frac{1}{3}}\displaystyle\int_{-3\frac{1}{3}}^{3\frac{1}{3}} \sqrt{9 - \left(\dfrac{x}{\frac{1}{3}}\right)^2} \; dx$, de donde nos queda $9 \displaystyle\int_{-1}^1 \sqrt{1-x^2}\; dx$, por lo tanto y en función de $\pi$ se tiene $\dfrac{9}{2} \pi$.\\\\

    %----------(b)
    \item $\displaystyle\int_0^2 \sqrt{1-\frac{1}{4}x^2}\; dx$.\\\\
	Respuesta.-\; Similar al anterior ejercicio se tiene 
	$$\int_0^2 \sqrt{1-\dfrac{1}{4}x^2}\; dx = 2\int_0^1 \sqrt{1-x^2}\; dx = \int_{-1}^1 \sqrt{1-x^2}\; dx = \dfrac{\pi}{2}$$\\

    %----------(c)
    \item $\displaystyle\int_{-2}^2 (x-3)\sqrt{4-x^2}\; dx$.\\\\
	Respuesta.-\; Comencemos usando la linealidad respecto al integrando de donde tenemos $\displaystyle\int_{-2}^2 x\sqrt{4-x^2}\; dx - 3\int_{-2}^2 \sqrt{4-x^2}\; dx$. Luego por el problema 25 de la sección 1.26, $\displaystyle\int_{-2}^2 x\sqrt{4-x^2}\; dx = 0$, de donde $$ -6\displaystyle\int_{-1}^1 \sqrt{4-4x^2}\; dx = -12 \int_{-1}^1 \sqrt{1-x^2}\; dx = -6\pi$$\\

\end{enumerate}

%--------------------18.
\item Calcular las áreas de los dodecágonos regulares inscrito y circunscrito en un disco circular unidad y deducir del resultado las desigualdades $3<\pi<12(2-\sqrt{3}).$\\\\
    Respuesta.-\; Como estos son dodecágonos, el ángulo en el origen del círculo de cada sector triangular es $2 \pi / 12 = \pi / 6$, y el ángulo de los triángulos rectángulos formado al dividir cada uno de estos sectores por la mitad es entonces $\pi/12$. Luego usamos el hecho de que,
    $$\tan\left(\dfrac{\pi}{12}\right)=2-\sqrt{2}, \qquad \sen\left(\dfrac{\pi}{12}\right) = \dfrac{\sqrt{3}-1}{2\sqrt{2}}, \qquad \cos\left(\dfrac{\pi}{12}\right) = \dfrac{\sqrt{3}+1}{2\sqrt{2}}$$
    Ahora, para el dodecágono circunscrito tenemos el área del triángulo rectángulo $T$ con base $1$ dado por, $$a(T) = \dfrac{1}{2}bh = \dfrac{1}{2}\cdot 1 \cdot (2-\sqrt{3}) = 1 - \dfrac{\sqrt{3}}{2}.$$
    Como hay 24 triángulos de este tipo en el dodecaedro, tenemos el área del dodecaedro circunscrito $D_c$ dada por $$a(D_c) = 24\left(\dfrac{1-\sqrt{3}}{2}\right) = 12(2-\sqrt{3})$$
    Por otro lado para el dodecágono inscrito, consideramos el triángulo rectángulo $T$ con hipotenusa $1$ en el diagrama. La longitud de uno de los catetos viene dada por $\sen \left(\frac{\pi}{12} \right) = \frac{\sqrt{3} - 1} {2}$ y la otra por $\cos \left(\frac{\pi}{12}\right).$ Entonces el área del triángulo es, $$a(T) = \dfrac{1}{2} bh = \dfrac{1}{2}\cdot \dfrac{\sqrt{3}-1}{2\sqrt{2}}\cdot \dfrac{\sqrt{3}+1}{2\sqrt{2}} = \dfrac{2}{16} = \dfrac{1}{8}.$$
    Dado que hay $24$ triángulos de este tipo en el dodecaedro inscrito, $D_{i}$ tenemos,
    $$a(D_i) = 24\cdot \dfrac{1}{8} = 3$$
    Por lo tanto, en vista de que el área del círculo unitario es, por definición $\pi$ y se encuentra entre estos dos dodecaedros, tenemos, $$3<\pi<12(2-\sqrt{3})$$\\

%--------------------19.
\item Sea $C$ la circunferencia unidad, cuya ecuación cartesiana es $x^2+y^2 = 1$. Sea $E$ el conjunto de puntos obtenido multiplicando la coordenada $x$ de cada punto $(x,y)$ de $C$ por un factor constante $a>0$ y la coordenada $y$ por un factor constante $b>0$. El conjunto $E$ se denomina elipse. (Cuando $a=b$, la elipse es otra circunferencia.).


\begin{enumerate}[\bfseries a)]

    %----------a)
    \item Demostrar que cada punto $(x,y)$ de $E$ satisface la ecuación cartesiana $(x/a)^2+(y/b)^2 = 1$.\\\\
	Demostración.-\; Sea $E=\lbrace (ax,by) / (x,y) \in C, a>0,b>0 \rbrace$. Si $(x,y)$ es un punto en $E$ entonces $\left(\frac{x}{a},\frac{y}{b}\right)$ es un punto es $C$, ya que todos los puntos de $E$ se obtienen tomando un punto de $C$ y multiplicando la coordenada $x$ por $a$ y la coordenada $y$ por $b$. Por definición de $C$, se tiene $$\left(\dfrac{x}{a}\right)^2+\left(\dfrac{y}{b}\right)^2 = 1$$

    %----------b)
    \item Utilizar las propiedades de la integral para demostrar que la región limitada por esa elipse es medible y que su área es $\pi ab$.\\\\
	Demostración.-\; De la parte $(a)$ sabemos que $E$ es el conjunto de puntos $(x,y)$ tales que $\left(\dfrac{x}{a}\right)^2 + \left(\dfrac{y}{b}\right)^2 = 1$. Esto implica, $$g(x)=b\sqrt{1-\left(\right)^2}, \quad o \quad f(x)=-b\sqrt{1-\left(\dfrac{x}{a}\right)^2}^2$$
	Por lo tanto, el área de $E$ es el área cerrada de $-a$, $a$.\\
	Para demostrar que esta región es medible y tiene área $\pi ab$, comenzamos por mencionar $$\pi = 2\int_{-1}^1 \sqrt{1-x^2}$$ 
	y por lo tanto 
	\begin{center}
	    \begin{tabular}{rcl}
		$\pi b$ & $=$ & $2\displaystyle\int_{-1}^1 b\sqrt{1-x^2}\; dx$\\\\
		$\pi ab$ & $=$ & $2a\displaystyle\int_{-1}^1 b\sqrt{1-x^2}\; dx$\\\\
		$\pi ab$ & $=$ & $2\displaystyle\int_{-a}^a \sqrt{1 - \left(\dfrac{1}{a}\right)^2} \; dx$\\\\
		$\pi ab$ & $=$ & $\displaystyle\int_{-a}^a b\sqrt{1 - \dfrac{x}{a}} - \left(-b\sqrt{1-\left(\dfrac{x}{a}\right)^2}\right) \; dx$\\\\
	    \end{tabular}
	\end{center}
	Por lo tanto, sabemos que la integral de $-a$, $a$ de $g(x)-f(x)$ existe y tiene valor $\pi ab$, concluyendo que $E$ es medible y $a(E)=\pi ab$.\\\\

\end{enumerate}

%--------------------20.
\item El ejercicio 19 es una generalización del ejemplo 3 de la sección 2.3. Establecer y demostrar una generalización correspondiente al ejemplo 4 de la sección 2.3.\\\\
    Demostración.-\; Para generalizar esto, procedemos de la siguiente manera. Sea $f$ una función integrable no negativa en $[a, b]$, y $S$ sea el conjunto de ordenadas de $f$. Si aplicamos una transformación bajo la cual multiplicamos la coordinada $x$ de cada punto $(x, y)$ en la gráfica de $f$ por una constante $k> 0$ y cada coordinada $y$ por una constante $j>0$, entonces obtenemos una nueva función $g$ donde un punto $(x, y)$ está en $g$ si  y sólo si $ \left(\dfrac{x}{k}, \dfrac{y}{j}\right)$ está en $f$. Luego,
    $$\dfrac{y}{j} = f\left(\dfrac{x}{k}\right) \Longrightarrow y = j\cdot f\left(\dfrac{x}{k}\right) \Longrightarrow g(x)=j\cdot j\left(\dfrac{x}{k}\right)$$
    Sea $jkS$ y denotamos el conjunto ordenado de $g$.
    $$ a(S) = \int_a^b f(x)\; dx$$
    entonces 
    \begin{center}
	\begin{tabular}{rcl}
	    $a(jsS)$&$=$&$\displaystyle\int_{ka}^{kb} g(x) \; dx$\\\\
	    &$=$&$j\cdot \displaystyle\int_{ka}^{kb} f\left(\dfrac{x}{k}\right) \; dx$\\\\
	    &$=$&$jk\cdot \displaystyle\int_{a}^{b} f(x) \; dx$\\\\
	    &$=$&$\displaystyle\int_{ka}^{kb} jk\cdot a(S) \; dx$\\\\

	\end{tabular}
    \end{center}

%--------------------21.
\item Con un razonamiento parecido al del ejemplo 5 de la sección 2.3 demostrar el teorema 2.2.\\\\
    Demostración.-\; Esta demostración ya fue dada junto a la definición del teorema 2.2.\\\\

\end{enumerate}


\section{Las funciones trigonométricas}

%--------------------Propiedades fundamentales del seno y del coseno
\begin{tcolorbox}[colframe = white]
    \begin{prop}\; \\
	\begin{enumerate}[\bfseries 1.]
	    \item Dominio de definición. Las funciones seno y coseno están definidas en toda la recta real.
	    \item Valores especiales. Tenemos $\cos 0 = \sen \frac{1}{2}\pi = 1, \cos \pi = -1$.
	    \item Coseno de una diferencia. Para $x$ e $y$ cualesquiera, tenemos
		$$\cos(y-x) = \cos y \cos x + \sen y \sen x.$$
	    \item Desigualdades fundamentales. Para $0<x<\frac{1}{2}\pi$, tenemos
		$$0<\cos x < \dfrac{\sen x}{x} < \dfrac{1}{\cos x}$$
	\end{enumerate}
    \end{prop}
\end{tcolorbox}

%--------------------teorema 2.3
\begin{teo} Si dos funciones $\sen$ y $\cos$ satisfacen las propiedades 1 a 4, satisfacen también las siguientes:
    \begin{enumerate}[\bfseries (a)]
	
	%----------(a)
	\item La identidad pitagórica, $\sen^2x + \cos^2 x = 1$, para todo $x$.\\\\
	    Demostración.-\; La parte (a) se deduce inmediatamente si tomamos $x=y$ en $$\cos(y-x) = \cos y \cos x + \sen y \sen x$$
	    y usamos la relación $\cos 0 = 1$.\\\\

	%----------(b)
	\item Valores especiales, $\sen 0 = \cos \frac{1}{2} \pi = \sen \pi = 0$.\\\\
	    Demostración.-\; Resulta de (a) tomando $x=0$, $x=\frac{1}{2}\pi$, $x=\pi$ y utilizando la relación $\sen \frac{1}{2}\pi = 1$.\\\\

	%----------(c)
	\item El coseno es función par y el seno es función impar. Esto es, para todo $x$ tenemos
	    $$\cos(-x)=\cos x,\qquad \sen(-x) = -\sen x$$\\
	    Demostración.-\; Que el coseno es par resulta también de $$\cos(y-x) = \cos y \cos x + \sen y \sen x$$ 
	    haciendo $y=0$. A continuación deducimos la fórmula $$\cos \left(\frac{1}{2}\pi - x \right) = \sen x,$$
	    haciendo $y=\frac{1}{2}\pi$ en $\cos(y-x) = \cos y \cos x + \sen y \sen x$. Partiendo de esto, encontramos que el seno es impar, puesto que 
	    $$\sen(-x)=\cos\left(\dfrac{\pi}{2}+x\right) = \cos \left[\pi - \left(\dfrac{\pi}{2} - x\right)\right] = \cos \pi \cos \left(\dfrac{\pi}{2} - x\right) + \sen \pi \sen\left(\dfrac{\pi}{2}-x\right) = -\sen x$$\\
	
	%----------(d)
	\item Co-relaciones. Para todo $x$, se tiene
	    $$\sen\left(\frac{1}{2} \pi\right) = \cos x, \qquad \sen\left(\frac{1}{2}\pi + x\right) = -\sen x$$\\
	    Demostración.-\; Para demostrarlo utilizaremos $\cos\left(\frac{1}{2}\pi\right)=\sen x$ reemplazando primero $x$ por $\frac{1}{2} \pi + x$ y luego $x$ por $-x$.\\\\

	%----------(e)
	\item Periodicidad. Para todo $x$ se tiene $\sen\left(x+2\pi \right) = \sen x, \; \cos(x+2\pi) = \cos x$.\\\\
	    Demostración.-\; El uso reiterado de (d) nos da entonces las relaciones de periodicidad (e).\\\\
	
	%----------(f)
	\item Fórmulas de adición. Para $x$ e $y$ cualesquiera, se tiene 
	    $$\cos (x+y) = \cos x \cos y - \sen x \sen y$$
	    $$\sen (x+y) = \sen x \cos x + \cos x \sen y$$\\
	    Demostración.-\; Para demostrar basta reemplazar $x$ por $-x$ en  $\cos(y-x) = \cos y \cos x + \sen y \sen x$ y tener en cuenta la paridad o imparidad. Luego utilizando la parte (d) y la fórmula de adición para el coseno se obtiene $$\sen(x+y) = -\cos \left(x+y+\frac{\pi}{2}\right) = =-\cos x \cos \left(y+\dfrac{\pi}{2}\right) + \sen x \sen \left(y + \dfrac{\pi}{2}\right) = \cos x\sen y + \sen x \cos y$$ \\

	%----------(g)
	\item  Fórmulas de diferencias. Para todo los valores $a$ y $b$, se tiene 
	    $$\sen a - \sen b = 2\sen \dfrac{a-b}{2} = \cos \dfrac{a+b}{2},$$
	    $$\cos a - \cos b = -2 \sen \dfrac{a+b}{2} \sen \dfrac{a+b}{2}.$$\\
	    Demostración.-\; Reemplazaremos primero $y$ por $-y$ en la fórmula de adición para $\sen(x+y)$ obteniendo $$\sen(x-y)=\sen x \cos y - \cos x \sen y$$
	    Restando ésta de la fórmula para $\sen(x+y)$ y haciendo lo mismo para función coseno, llegamos a
	    $$\sen(x+y) -\sen(x-y) =  2\sen y \cos x,$$
	    $$\cos(x+y) - \cos(x-y) = -2\sen y \sen x.$$
	    Haciendo $x=(a+b)/2$,  $y=(a-b)/2$ encontramos que esas se convierten en las fórmulas de diferencia (g).\\\\

	%----------(h)
	\item Monotonía. En el intervalo $[0,\frac{1}{2}\pi]$, el seno es estrictamente creciente y el coseno estrictamente decreciente.\\\\
	    Demostración.-\; La propiedad 4 se usa para demostrar (h). Las desigualdades $0<\cos x < \frac{\sen x}{x}<\frac{1}{\cos x}$ prueban que $cos x$ y $\sen x$ son positivas si $0<x<\frac{1}{2}\pi$. Después de esto, si $0<b<a<\frac{1}{2}\pi$, los números $(a+b)/2$ y $(a-b)/2$ están en el intrevalo $(0,\frac{1}{2}\pi)$, y las fórmulas de diferencias (g) prueban que $\sen a > \sen b$ y $\cos a < \cos b$. Esto completa la demostración.\\\\

    \end{enumerate}
\end{teo}


\section{Fórmulas de integración para el seno y el coseno}

%--------------------teorema 2.4
\begin{teo} Si $0<a\leq \frac{1}{2}\pi,$ y $n\geq 1,$ tenemos $$\dfrac{a}{n}\sum_{k=1}^n \cos \dfrac{ka}{n}<\sen a < \dfrac{a}{n}\sum_{k=0}^{n-1} \cos \dfrac{ka}{n} \qquad \mbox{(2.6)}.$$
    Demostración.-\; Las desigualdades anterior serán deducidas de la identidad $$2\sen \dfrac{1}{2} x \sum_{k=1}^n \cos kx = \sen\left(n+\dfrac{1}{2}\right)x-\sen \dfrac{1}{2}x, \qquad \mbox{(2.7)}$$
    válida para $n\geq 1$ y todo real $x$. Para demostrar, utilizaremos las fórmulas de diferencias (g) del teorema 2.3 para poner $$2\sen\dfrac{1}{2}x\cos kx = \sen \left(k+\dfrac{1}{2}\right)x - \sen\left(k-\dfrac{1}{2}\right)x$$
    Haciendo $k=1,2,...,n$ y sumando esas igualdades, encontramos que en la suma del segundo miembro se reduce unos términos con otros obteniéndose (2.7).\\
    Si $\dfrac{1}{2}x$ no es un múltiplo entero de $\pi$ podemos dividir ambos miembros de (2.7) por $2\sen \dfrac{1}{2}x$ resultando $$\sum_{k=1}^n \cos kx = \dfrac{\sen(n+\frac{1}{2}x-\sen \frac{1}{2} x)}{2\sen \frac{1}{2}x}$$ 
    Reemplazando $n$ por $n-1$ y sumando $1$ a ambos miembros también obtenemos. (Ya que $\cos 0 = 1$).
    $$\sum_{k=0}^{n-1} \cos kx = \dfrac{\sen\left(n-\frac{1}{2}\right)x + \sen\frac{1}{2}x}{2\sen\frac{1}{2}x}$$
    Esas dos fórmulas son válidas si $x\neq2m\pi$, siendo $m$ entero. Tomando $x=a/n$, donde $0<a<\frac{1}{2}\pi$ encontramos que el par de desigualdades (2.6) es equivalente al siguiente 
    $$\dfrac{a}{n}\dfrac{\sen\left(n+\dfrac{1}{2}\right)\dfrac{a}{n} - \sen \left(\dfrac{a}{2n}\right)}{2\sen\left(\dfrac{a}{2n}\right)} < \sen a < \dfrac{a}{n}\dfrac{\sen\left(n-\dfrac{1}{2}\right)\dfrac{a}{b}+\sen\left(\dfrac{a}{2n}\right)}{2\sen\left(\dfrac{a}{2n}\right)}$$
    Este par, a su vez es equivalente al par
    $$\sen\left(n+\dfrac{1}{2}\right)\dfrac{a}{n}-\sen\left(\dfrac{a}{2n}\right) < \dfrac{\sen\left(\dfrac{a}{2n}\right)}{\left(\dfrac{a}{2n}\right)} \sen a < \sen\left(n-\dfrac{1}{2}\right)\dfrac{a}{n}+\sen\left(\dfrac{a}{2n}\right) \qquad \mbox{(2.8)}$$
    Por consiguiente demostrar (2.6) equivale a demostrar (2.8). Demostraremos que se tiene 
    $$\sen\left(2n+1\right)\theta -\sen \theta < \dfrac{\sen \theta}{\theta} < \sen(2n-1)\theta +\sen \theta \qquad \mbox{(2.9)}$$
    para $0<2n\theta\leq \frac{1}{2}\pi$. Cuando $\theta = a/(2n)$ (2.9) se reduce a (2.8).\\
    Para demostrar la desigualdad de la parte izquierda de (2.9), usamos la fórmula de adición para el seno poniendo $$\sen(2n+1)\theta = \sen 2n\theta \cos \theta + \cos 2n\theta \sen \theta < \sen 2n\theta \dfrac{\sen \theta}{\theta} + \sen \theta,$$
    habiendo usado también las desigualdades $$\cos \theta < \dfrac{\sen \theta}{\theta}, \quad 0z<\cos 2n\theta \leq 1, \quad \sen \theta > 0, \qquad \mbox{(2.10)}$$
    siendo todas válidas ya que $0<2n\theta \leq \frac{1}{2}\pi$. La desigualdad (2.10) equivale a la parte izquierda de (2.9).\\
    Para demostrar la parte derecha de (2.9), utilizamos nuevamente la fórmula de adición para el seno poniendo $$\sen(2n-1)\theta = \sen 2n\theta \cos \theta - \cos 2n\theta \sen \theta$$
    Sumando $\sen \theta$ ambos miembros, obtenemos $$\sen(2n-1)\theta + \sen \theta = \sen 2n\theta \left(\cos \theta +\sen \theta \dfrac{1-\cos 2n\theta}{\sen 2n\theta}\right) \qquad \mbox{(2.11)}$$
    Pero ya que tenemos $$\dfrac{1-\cos 2n\theta}{\sen2n\theta}=\dfrac{2\sen^2n\theta}{2\sen n\theta \cos n \theta}=\dfrac{\sen n\theta}{\cos n \theta}$$
    el segundo miembro de (2.11) es igual a $$\sen 2n\theta \left(\cos \theta + \sen \theta \dfrac{\sen n \theta}{\cos n\theta}\right) = \sen 2 n\theta \dfrac{\cos \theta \cos n\theta + \sen \theta \sen n \theta}{\cos n \theta} = \sen 2n\theta \dfrac{\cos (n-1) \theta}{\cos n \theta}$$
    Por consiguiente, para completar la demostración de (2.9), necesitamos tan sólo demostrar que 
    $$\dfrac{\cos(n-1)\theta}{\cos n\theta}>\dfrac{\sen \theta}{\theta} \qquad \mbox{(2.12)}$$
    Pero tenemos $$\cos n\theta = \cos(n-1)\cos theta - \sen(n-1)\theta \sen \theta < \cos(n-1)\theta \cos \theta < \cos(n-1)\theta \dfrac{\theta}{\sen \theta},$$
    en donde otra vez hemos utilizado la desigualdad fundamental $\cos \theta < \theta/sen \theta.$ ya que $\left(\cos x < \dfrac{x}{\sen x}\right)$, esta último relación implica (2.12), con lo que se completa la demostración del teorema 2.4.\\\\

\end{teo}

%--------------------teorema 2.5
\begin{teo} Si dos funciones $\sen$ y $\cos$ satisfacen las propiedades fundamentales de la 1 a la 4, para todo $a$ real se tiene $$\int_0^a \cos x\; dx = \sen a, \qquad \mbox{(2.13)}$$ $$\int_0^a x\; dx = 1-\cos a. \qquad \mbox{(2.14)}$$\\
    Demostración.-\; Primero se demuestra (2.13), y luego usamos (2.13) para deducir (2.14). Supongamos que $0<a\leq \frac{1}{2}\pi$. Ya que el coseno es decreciente en $[0,a]$ podemos aplicar el teorema 1.14 y las desigualdades del teorema 2.4 obteniendo (2.13). La fórmula es valida también para $a=0$, ya que ambos miembros son cero. Pueden ahora  utilizarse las propiedades de la integral para ampliar su validez todos los valores reales $a$.\\
    Por ejemplo, si $-\frac{1}{2}\pi \leq a \leq 0$, entonces $0\leq -a\leq \frac{1}{2}\pi$, y la propiedad de reflexión nos da $$\int_0^a \cos x\; dx = -\int_0^{-a} \cos (-x) \; dx = - \int_0^{-a} \cos x \; dx = -\sen (-a) = \sen a.$$
    Así, pues, (2.13) es válida en el intervalo $\left[-\frac{1}{2}\pi,\frac{1}{2}\pi\right]$. Supongamos ahora que $\frac{1}{2}\pi \leq a \leq \frac{3}{2}\pi$. Entonces $-\frac{1}{2}\pi\leq a-\pi\leq \frac{1}{2}\pi$, de modo que $$\int_0^a \cos x \; dx = \int_{0}^{\pi/2} \cos x \; dx + \int_{\pi/2}^a \cos x \; dx = \sen\frac{1}{2}\pi + \int_{-\pi/2}^{a-\pi} \cos(x+\pi)\; dx = 1-\int_{-\pi/2}^{a-\pi} \cos x \; dx = $$ $$=1 - \sen(a-\pi) + \sen\left(-\frac{1}{2}\pi\right) = \sen a$$
    Con ellos resulta que (2.13) es válida para todo $a$ en el intervalo $\left[-\frac{1}{2}\pi,\frac{3}{2}\pi\right]$. Pero este intervalo tiene longitud $2\pi$, con lo que la fórmula (2.13) es válida para todo $a$ puesto que ambos miembros son periódicos respecto a $a$ con período $2\pi$\\
    Seguidamente usamos (2.13) para deducir (2.14). Ante todo demostramos que (2.14) es válida cuando $a=\pi/2$. Aplicando sucesivamente, la propiedad de traslación, la co-relación $\sen\left(x+\frac{1}{2}\pi\right)$, y la propiedad de reflexión, encontramos 
    $$\int_0^{\pi/2}\sen x\; dx = \int_{-\pi/2}^0 \sen \left(x+\dfrac{\pi}{2}\right)\; dx = \int_{-\pi/2}^0 \cos x \; dx = \int_0^{\pi/2}\cos (-x) \; dx$$
    Haciendo uso de la relación $\cos(-x)=\cos x$ y la igualdad (2.13), se obtiene $$\int_0^{\pi/2}\sen x\; dx = 1$$
    Por consiguiente, para cualquier $a$ real, podemos escribir $$\int_0^a \sen x\;dx = \int_0^{\pi/2} \sen x\; dx + \int_{\pi/2}^a \sen x \; dx = 1 + \int_0^{a-\pi/2}\sen \left(x+\dfrac{\pi}{2}\right)\; dx =$$ $$= 1+\int_0^{a-\pi/2} \cos x \; dx = 1+\sen\left(a-\dfrac{\pi}{2}\right) = 1-\cos a$$
    Esto demuestra que la igualdad (2.13) implica (2.14).\\\\
\end{teo}

\begin{tcolorbox}[colframe = white,colback=black!2!]
    \begin{nota} Usando (2.13) y (2.14) junto con la propiedad aditiva $$\int_a^b f(x) \; dx = \int_0^b f(x)\; dx - \int_0^a f(x)\; dx $$
    llegamos a las fórmulas de integración más generales
    $$\int_a^b \cos x \; dx = \sen b - \sen a$$
	y $$\int_a^b \sen x \; dx = (1-\cos b) - (1-\cos a) = -(\cos b - \cos a).$$
	Si nuevamente utilizamos el símbolo especial $f(x)\left|_a^b\right.$ para indicar la diferencia $f(b)-f(a)$, podemos escribir esas fórmulas de integración en la forma 
	$$\int_a^b \cos x \;dx = \sen x \bigg|_a^b \qquad y \qquad \int_a^b \sen x\; dx = -\cos x \bigg|_a^b$$
    \end{nota}
\end{tcolorbox}

\begin{tcolorbox}[colframe = white,colback=black!2!]
    \begin{nota} 
	Con los resultados del ejemplo 1 y la propiedad de dilatación $$\int_a^b f(x)\; dx = \dfrac{1}{c}\int_{ca}^{cb} f(x/c)\; dx,$$
	obtenemos las fórmulas siguientes, válidas para $c\neq 0$;
	$$\int_a^b \cos cx \; dx = \dfrac{1}{c}\int_{ca}^{cb} \cos x \; dx = \dfrac{1}{c}(\sen cb - \sen ca)$$
	y $$\int_a^b \sen cx \; dx = \dfrac{1}{c} \int_{ca}^{cb} \sen x \; dx = -\dfrac{1}{c}(\cos cb -\cos ca).$$
    \end{nota}
\end{tcolorbox}

\begin{tcolorbox}[colframe = white,colback=black!2!]
    \begin{nota} 
	La identidad $\cos 2x = 1-2\sen^2 x$ implica $\sen^2 x = \frac{1}{2}(1-\cos 2x)$ con lo que, a partir del ejemplo 2, obtenemos $$\int_a^b \sen^2 x \; dx = \dfrac{1}{2}\int_0^a (1-\cos 2x)\; dx = \dfrac{a}{2}-\dfrac{1}{4}\sen 2a$$
	Puesto que $\sen^2 x + \cos^2 x = 1,$ encontramos también $$\int_0^a \cos^2 x \; dx = \int_0^a(1-\sen^2 x)\; dx = a - \int_0^a \sen^2 x\; dx = \dfrac{a}{2}+\dfrac{1}{4}\sen 2a$$
    \end{nota}
\end{tcolorbox}

\setcounter{section}{7}
\section{Ejercicios}

En este conjunto de ejercicios, se pueden emplear las propiedades del seno y del coseno citadas en las Secciones de la 2.5 a la 2.7.\\\\

\begin{enumerate}[\Large\bfseries 1.]

%--------------------1.
\item 
\begin{enumerate}{\bfseries (a)}

    %----------(a)
    \item Demostrar que $\sen n\pi =0$ para todo entero $n$ y que esos son los únicos valores de $x$ para los que $\sen x = 0$.\\\\
	Demostración.-\; Primero, ya que $\sen x = -\sen x$ implica que si $\sen x = 0$ entonces $\sen(-x)=0,$, lo que es suficiente mostrar que la declaración es válida para todo $n \in \mathbb{Z}^+$. Sabemos que $\sen 0 = \sen \pi = 0$. Por lo tanto, es válida para el caso de $n=0$ y $n=1$. Ahora utilizaremos la inducción dos veces, primero para los enteros pares y luego para los impares.\\
	Supongamos que la declaración es válida para algunos pares $m\in \mathbb{Z}^+$. Es decir, $\sen(m\pi)=0$. Luego usando la periocidad de la función seno, $$0=\sen(m\pi)=\sen(m\pi + 2\pi)=\sen\left[(m+2)\pi\right].$$  
	Por lo tanto se cumple para todo par $n\in \mathbb{Z}^+$.\\
	Luego supongamos que es verdad para algunos impares $m\in \mathbb{Z}^+$, entonces $$0=\sen m\pi = \sen(m\pi+2\pi)=\sen\left[(m+2)\pi\right]$$
	de donde es verdad para todo impar $n\in \mathbb{Z}^+$. Se sigue que es verdad para todo entero no negativo $n$, y por lo tanto para todo $n\in \mathbb{Z}$.\\
	Por otro lado debemos demostrar que estos son únicos valores reales el cual el seno es $0$. Por la periocidad del seno, es suficiente mostrar que se cumple para cualquier intervalo $2\pi$. Escogeremos el intervalo $(.\pi,\pi)$ y demostraremos que $\sen x=0 \Longleftrightarrow x=0$ para todo $x\in (-\pi,\pi)$. Por la primera parte conocemos que $\sen 0 = 0$. Entonces por la propiedad fundamental del seno y el coseno, tenemos las inecuaciones $$0<\cos x <\dfrac{\sen x}{x}<\dfrac{1}{\cos x} \qquad x\in \left(0,\dfrac{\pi}{2}\right)$$
	de donde ambos $\sen x$ y $\cos x$ son positivos en $\left(0,\frac{\pi}{2}\right)$. Pero, de la identidad de correlación, tenemos 
	$$\sen\left(\dfrac{\pi}{2}+x\right)=\cos x$$
	así, para $x\in \in (\frac{\pi}{2},\pi)$ $\sen x \neq 0$. Pero también sabemos que $\sen\frac{\pi}{2}=1$ y porque $x\in (0,\pi)$ tenemos $\sen x\neq 0$. Dado que seno es una función impar, $$\sen(-x)=x\sen x \; \Longrightarrow \; \sen(-x)\neq 0 \quad \mbox{para}\; x \in (0,\pi)$$
	en consecuencia, $\sen x\neq 0$ para $x\in (-\pi,0).$ así, $$\sen x=0 \; \Longrightarrow \; x=0 \quad \mbox{para}\; x\in (-\pi,\pi)$$\\

    %----------(b)
    \item Hallar todos los valores reales $x$ tales que $\cos x =0$.\\\\\
	Respuesta.-\; Se tiene que  $\cos x=0$ si y sólo si $x=\frac{pi}{2}+n\pi$. Probando esta proposición se tiene que $x=0\;\Longrightarrow \sen(x+\frac{\pi}{2})=0$, aplicando la parte (a) concluimos que $$x+\dfrac{\pi}{2}=n\pi \quad \Longrightarrow \quad x=\dfrac{\pi}{2}+n\pi.$$.\\

\end{enumerate}

%--------------------2.
\item Hallar todos los reales $x$ tales que 
\begin{enumerate}[\bfseries a)]
    
    %----------(a)
    \item $\sen x = 1$\\\\
	Respuesta.-\; $x$ está dado por $\dfrac{\pi}{2}+2n\pi$. Para todo $n \in \mathbb{N}$.\\\\

    %----------(b)
    \item $\cos x = 1$\\\\
	Respuesta.-\; $x$ es igual a $2n\pi.$ Para todo $n \in \mathbb{N}$.\\\\

    %----------(c)
    \item $\sen x = -1$\\\\
	Respuesta.-\; $x=\dfrac{3\pi}{2}+2n\pi$. Para todo $n \in \mathbb{N}$.\\\\

    %----------(d)
    \item $\cos x = -1$\\\\
	Respuesta.-\; Está dado por $x=(2n+1)\pi$. Para todo $n \in \mathbb{N}$.\\\\

\end{enumerate}

%--------------------3.
\item Demostrar que $\sen(x+\pi) = -\sen x$ y $\cos(x+\pi) = -\cos x$ para todo $x$.\\\\
    Demostración.-\; Por las fórmulas de adición se tiene $$\sen\left(x+\pi\right) = \sen x \cos \pi +\cos x \sen \pi = -\sen x$$ 
	Por otro lado se tiene $$\cos (x+\pi) = \cos x \cos \pi - \sen x \sen \pi = -\cos x$$\\

%-------------------4. 
\item Demostrar que $\sen 3x = 3\sen x-4\sen^3 x $ y $\cos 3x = \cos x -4\sen^2 x \cos x$ para todo real $x$. Demostrar también que $\cos 3x=4\cos^3 x -3\cos x$.\\\\
	Demostración.-\; Por la fórmula de adición y la identidad Pitagórica se tiene,
	\begin{center}
	    \begin{tabular}{rcl}
		$\sen3x$&$=$&$\sen(2x+x)$\\
		&$=$&$\sen2x \cos x + \cos x + \cos 2x \sen x$\\
		&$=$&$(\sen x \cos x + \cos x \sen x)\cos x + \sen x(1-\sen^2x)$\\
		&$=$&$2\sen x \cos^2 x +\sen x - \sen^3 x$\\
		&$=$&$2\sen x(1-\sen^2 x)+\sen x -\sen^3 x$\\
		&$=$&$2\sen x - 2\sen^3 x + \sen x - \sen^3 x$\\
		&$=$&$3\sen x - 3 \sen^3 x$\\\\
	    \end{tabular}
	\end{center}
	 Luego,
	\begin{center}
	    \begin{tabular}{rcll}
		$\cos 3x$&$=$&$\cos 2x \cos x - \sen 2x \sen x$&\\
		&$=$&$(1-2\sen^2 x)\cos x - 2\sen^2 x \cos x$&\\
		&$=$&$\cos x - 2\sen^2 x \cos x - 2\sen^2 x \cos x$&\\
		&$=$&$\cos x - 4\sen^2 x \cos x$&Se demuestra la segunda proposición\\
		&$=$&$\cos x - 4(1-\cos^2x)\cos x$&\\
		&$=$&$\cos x - 4\cos x +4\cos^3 x$&\\
		&$=$&$4\cos x - 3 \cos x$&Se demuestra la tercera proposición\\\\
	    \end{tabular}
	\end{center}

%-------------------5.
\item 
\begin{enumerate}[\bfseries (a)]

    %----------(a)
    \item Demostrar que $\sen \frac{1}{6}\pi=\frac{1}{2},\; \cos \frac{1}{6}\pi = \frac{1}{2}\sqrt{3}$.\\\\
	Demostración.-\; Por el anterior problema se tiene $$\sen\dfrac{\pi}{2}=\sen\left(3\cdot \dfrac{\pi}{6}\right)  =3 \sen \dfrac{\pi}{6}-4\sen^3 \dfrac{\pi}{6}$$
	luego ya que $\sen\dfrac{\pi}{2}=1$ se tiene, 
	$$3\sen\dfrac{\pi}{6}-4\sen^3\dfrac{\pi}{6}=1 \; \Longrightarrow \; \sen\dfrac{\pi}{6}=\dfrac{1}{2}.$$
	se sigue, $$\cos\dfrac{\pi}{2}=\cos\left(3\cdot \dfrac{\pi}{6} = 4\cos^3\dfrac{\pi}{6}-3\cos\dfrac{\pi}{6}\right)$$
	por lo tanto $$4\cos^3\dfrac{\pi}{6}-3\cos\dfrac{\pi}{6}=0 \; \Longrightarrow \; \cos\dfrac{\pi}{6}=\dfrac{\sqrt{3}}{2}.$$\\

    %----------(b)
    \item Demostrar que $\frac{1}{3}\pi = \dfrac{1}{2}\sqrt{3}, \cos\frac{1}{3}\pi = \frac{1}{2}$.\\\\
	Demostración.-\; Se usa la parte (a) y la correlación del teorema 2.3, parte d,
	$$\cos\left(\dfrac{\pi}{2}+x\right) = -\sen x \; \Longrightarrow \; \cos\dfrac{\pi}{6}=-\sen\left(-\dfrac{\pi}{3}\right) \; \Longrightarrow \; \sen\dfrac{\pi}{3}=\dfrac{\sqrt{3}}{2}.$$
	similarmente,
	$$\sen\left(\dfrac{\pi}{2}+x\right) = \cos x \; \Longrightarrow \; \sen\dfrac{\pi}{6} = \cos\left(-\dfrac{\pi}{3}\right)=\cos\dfrac{\pi}{3} \; \Longrightarrow \; \cos{\pi}{3}=\dfrac{1}{2}.$$\\

    %----------(c)
    \item Demostrar que $\sen\frac{1}{4} \pi = \cos\frac{1}{4}\pi = \frac{1}{2}\sqrt{2}.$\\\\
	Demostración.-\; Primeramente se tiene $$1=\sen\dfrac{\pi}{2} = \sen\left(2\cdot \dfrac{\pi}{4}\right) = 2\sen\dfrac{\pi}{4}\cos\dfrac{\pi}{3}.$$
	Luego $$0=\cos\dfrac{\pi}{2}=\cos\left(2\cdot \dfrac{\pi}{4}\right)=1-2\sen^2 \dfrac{\pi}{4}.$$
	se sigue $$1-2\sen^2\dfrac{\pi}{4}=0 \; \Longrightarrow \; \sen^2 \dfrac{\pi}{4} = \dfrac{1}{2} \; \Longrightarrow \; \sen \dfrac{\pi}{4}=\dfrac{\sqrt{2}}{2}.$$
	    entonces, $$2\sen\dfrac{\pi}{4}\cos\dfrac{\pi}{4}=1 \; \Longrightarrow\; \sqrt{2}\cos \dfrac{\pi}{4}=1 \; \Longrightarrow\; \cos \dfrac{\pi}{4}=\dfrac{\sqrt{2}}{2}.$$
	    por lo tanto $$\sen\dfrac{\pi}{4}=\dfrac{\sqrt{2}}{2}.$$\\

\end{enumerate}

%--------------------6.
\item Demostrar que $\tan(x-y)=(\tan x-\tan y)/(1+\tan x\tan y)$ para todo par de valores $x$, $y$ tales que $\tan x \tan y \neq -1$. Obtener las correspondientes fórmulas para $\tan(x+y)$ y $\cot(x+y)$.\\\\
    Demostración.-\; Por definición de tangente primeramente se tiene que, 
    \begin{center}
	\begin{tabular}{rcl}
	    $\tan(x-y)=\dfrac{\sen(x-y)}{\cos(x-y)}$&$=$&$\dfrac{\sen x \cos y - \sen x \cos y}{\cos x \cos y + \sen x \sen y}$\\\\
	    &$=$&$\dfrac{\sen x \cos y - \sen y \cos x}{\cos x \cos y(1+\tan x \tan y)}$\\\\
	    &$=$&$\dfrac{\tan x- \tan y}{1 + \tan x}$\\\\
	\end{tabular}
    \end{center}
    Luego probemos $\tan (x+y)$ para encontrar una fórmula de la siguiente manera,
    \begin{center}
	\begin{tabular}{rcl}
	    $\tan(x+y)=\dfrac{\sen(x+y)}{\cos(x+y)}$&$=$&$\dfrac{\sen x \cos y + \sen x \cos y}{\cos x \cos y - \sen x \sen y}$\\\\
	    &$=$&$\dfrac{\sen x \cos y + \sen y \cos x}{\cos x \cos y(1-\tan x \tan y)}$\\\\
	    &$=$&$\dfrac{\tan x + \tan y}{1 - \tan x}$\\\\
	\end{tabular}
    \end{center}
    Por último probemos para la fórmula cotagente,
    \begin{center}
	\begin{tabular}{rcl}
	    $\cot(x+y)=\dfrac{\cos(x+y)}{\sen(x+y)}$&$=$&$\dfrac{\cos x \cos y + \sen x \sen y}{\sen x \cos y - \sen x \cos y}$\\\\
	    &$=$&$\dfrac{\sen x \sen y (\cot x \cot - 1)}{\sen x \cos y + \sen y \cos x)}$\\\\
	    &$=$&$\dfrac{\cot x + \cot y - 1}{\cot x + \cot y}$\\\\
	\end{tabular}
    \end{center}

%--------------------7.
\item Hallar dos números $A$ y $B$ tales que $3\sen (x+\frac{1}{3}\pi)=A\sen x + B\cos x$ para todo $x$.\\\\
    Respuesta.-\; 
    \begin{center}
	\begin{tabular}{rcl}
	    $3\sen\left(x+\dfrac{\pi}{3}\right)$&$=$&3$\left(\sen x \cos \dfrac{\pi}{3} + \sen \dfrac{\pi}{3} \cos x\right)$\\\\
	    &$=$&$3\left(\dfrac{1}{2}\sen x \dfrac{\sqrt{3}}{2} \cos x\right)$\\\\
	    &$=$&$\dfrac{3}{2} \sen x + \dfrac{3\sqrt{3}}{2}\cos x.$\\\\
	\end{tabular}
    \end{center}
    Por lo tanto $$A=\dfrac{3}{2}, \qquad B=\dfrac{3\sqrt{3}}{2}$$\\

%--------------------8.
\item Demostrar que si $C$ y $\alpha$ son números reales dados, existen dos números reales $A$ y $B$ tal es que $C$ $\sen(x-\alpha)=A \sen x + b \cos x $ para todo $x$.\\\\
    Respuesta.-\;  Para $C$ y $\alpha \in \mathbb{R}$ entonces, 
    \begin{center}
	\begin{tabular}{rcl}
	    $C\sen(x+\alpha)$&$=$&$C(\sen x \cos \alpha + \sen \alpha \cos x)$\\
			     &$=$&$(C \cos \alpha)\sen x + (C \sen \alpha)\cos x$\\
	\end{tabular}
    \end{center}
    Ya que sen y cos está definida para todo $\alpha \in \mathbb{R}$, tenemos $C \cos \alpha$ $C\sen \alpha \in \mathbb{R}$. Por lo tanto, los números que requerimos serán: 
    $$A = C \cos \alpha, \qquad B=C\sen \alpha$$\\

%--------------------9.
\item Demostrar que si $A$ y $B$ son números reales dados, existen dos números $C$ y $\alpha$ siendo $C\geq 0$ tales que la fórmula del ejercicio 8 es válida.\\\\
    Respuesta.-\; Sea $$C = \sqrt{A^2+B^2}$$
    de donde es verdad que $$|A|\leq \left|\sqrt{A^2+B^2}\right|$$
    y por lo tanto $$\left|\dfrac{A}{C}\leq 1\right|$$
    así sabemos que existe un $\alpha \in \mathbb{R}$ de manera que $$\cos \alpha = \dfrac{A}{C}.$$
    Pero si $\cos \alpha = \dfrac{A}{C}$ entonces,
    \begin{center}
	\begin{tabular}{rcrcl}
	    $\cos^2 \alpha = \dfrac{A^2}{C^2}$&$\Longrightarrow$&$1-\sen^2 \alpha$& $=$ & $\dfrac{A^2}{A^2+B^2}$\\\\
	  &$\Longrightarrow$& $\sen^2 \alpha $ & $=$ & $1-\dfrac{A^2}{A^2 + B^2} = \dfrac{B^2}{A^2+B^2}$\\\\
	  &$\Longrightarrow$& $\sen \alpha$ & $=$ & $\dfrac{B}{C}$\\\\
	\end{tabular}
    \end{center}
    en consecuencia,
    $$C\sen(x+\alpha) = C \sen x \cos \alpha + C \sen \alpha \cos x = (A^2+B^2)^{1/2} \dfrac{A}{(A^2+B^2)^{1/2}} \sen x + (A^2 + B^2)^{1/2} \dfrac{B}{(A^2+B^2)^{1/2}} \cos x  $$
    $$=A\sen x + B\cos x$$\\

%--------------------10.
\item Determinar $C$ y $ \alpha$, siendo $C>0$, tales que $C \sen (x+\alpha) = -2\sen x-2\cos x$ para todo $x$.\\\\
    Respuesta.-\; Por el anterior ejercicio tenemos que $$A\sen x + B\cos x = C \sen(x+\alpha)$$ para 
    \begin{center}
	$C=\sqrt{A^2+B^2}$, y $\sen \alpha = \dfrac{A}{(A^2+B^2)^{1/2}}, \quad \cos \alpha = \dfrac{B}{(A^2 + B^2)^{1/2}}$
    \end{center}
    Luego ya que tenemos $A=B=-2$, entonces
    $$C=\left[(-2)^2+(-2)^2\right]^{1/2} = 2\sqrt{2}$$
    y,
    $$\sen \alpha = \cos \alpha = -\dfrac{\sqrt{2}}{2} \; \qquad \; \alpha = \dfrac{5\pi}{4}.$$\\

%--------------------11.
\item Demostrar que si $A$ y $B$ son números reales dados, existen dos números $C$  y $\alpha$ siendo $C\geq 0$ tales que $C \cos(x+\alpha) = A \sen x + B\cos x.$ Determinar $C$ y $\alpha$ si $A=B=1$.\\\\
    Demostración.-\; Por el problema 9, sabemos que para $A,B \in \mathbb{R}$, existe $D, \beta \in \mathbb{R}$ tal que 
    \begin{center}
	\begin{tabular}{rcl}
	    $D\sen (x+\beta)$&$=$&$A\sen x + B \cos x$\\\\
	    $D \sen \left[\dfrac{\pi}{2} + x + \left(\beta - \dfrac{\pi}{2}\right)\right]$&$=$&$A \sen x +B \cos x$\\\\
	    $D\cos \left[x + \left(\beta - \dfrac{\pi}{2}\right)\right]$&$=$&$A\sen x + B \cos x$\\\\
	\end{tabular}
    \end{center}
    entonces, $$C=D, \quad \alpha = \beta -\dfrac{\pi}{2}.$$
    Por lo tanto, $C$ y $\alpha$ existe si $C\cos(x+\alpha) = A \sen x + B\cos x.$
    Luego, si $A=B=1$, entonces $C=\sqrt{2}$ y $\alpha = \beta - \dfrac{\pi}{2}$ donde $\sen \beta = \cos \beta = \dfrac{\sqrt{2}}{2} \Longrightarrow \beta = \dfrac{\pi}{4}$. Así,
    $$C = \sqrt{2}, \quad \alpha = - \dfrac{\pi}{4}.$$\\

%--------------------12.
\item Hallar todos los números reales $x$ tales que $\sen x = \cos x$.\\\\
    Respuesta.-\; Sea $\sen x = \cos x$ entonces 
    \begin{center}
	\begin{tabular}{rcl}
	    $\cos x - \sen x$& $=$ & $0$\\\\
			    $\cos^2 x - 2\sen x\cos x$&$=$&$0$\\\\
			    $\sen 2 x$&$=$&$\sen^2x + \cos^2 x$\\\\
			    $\sen 2x$&$=$&$1$\\\\
			    $2x$&$=$&$\dfrac{\pi}{2}+2n\pi$\\\\
			    $x$&$=$&$\dfrac{\pi}{4} + n\pi \quad para\; n \in \mathbb{Z}$\\\\
	\end{tabular}
    \end{center}

%--------------------13.
\item Hallar todos los números reales tales que $\sen x - \cos x = 1.$\\\\
    Respuesta.-\; Por el ejercicio 9 sabemos que existe $C,\alpha$ tal que $$A\sen x +B\cos x = C\sen(x+\alpha),$$
    para cualquier $A,B \in \mathbb{R}$. También sabemos que,
    $$C= (A^2 + B^2)^{1/2}, \quad \sen \alpha = \dfrac{A}{(A^2+B^2)^{1/2}} \qquad \dfrac{B}{(A^2+B^2)^{1/2}}$$
    Así, en este caso tenemos $A=1$, $B=1$ y calculando,
    $$C=\sqrt{2}, \quad \cos \alpha = - \dfrac{\sqrt{2}}{2}, \sen \alpha = -\dfrac{\sqrt{2}}{2} \Longrightarrow \alpha = \dfrac{\pi}{4}$$
    luego sabiendo que $\sen x - \cos x = 1$ entonces,
    \begin{center}
	\begin{tabular}{rcl}
	    $\sqrt{2} \sen \left(x-\dfrac{\pi}{4}\right)$&$=$&$1$\\\\
	    $\sen\left(x-\dfrac{\pi}{4}\right)$&$=$&$\dfrac{\sqrt{2}}{2}$\\\\
	    $x-\dfrac{\pi}{4}$&$=$&$x-\dfrac{\pi}{4}+2n\pi \; o \; x-\dfrac{\pi}{4} = \dfrac{4\pi}{4}+2n\pi$\\\\
	    $x$&$=$&$\dfrac{\pi}{2}+2n\pi \; o \; x = \pi + 2n\pi$\\\\
	\end{tabular}
    \end{center}

%--------------------14.
\item Demostrar que las identidades siguientes son válidas para todos los pares $x$ e $y$:

    \begin{enumerate}[\bfseries (a)]

	%----------(a)
	\item $2\sen x \cos y = \cos(x-y) + \cos(x+y).$\\\\
	    Demostración.-\; Usando la fórmula para el coseno de una suma y de una diferencia, se sigue,\\
	    \begin{center}
		\begin{tabular}{rcl}
		    $2\cos x \cos y$&$=$&$\cos x \cos y + \cos x \cos y$\\
		    &$=$&$\cos x \cos y + \sen x \sen y + \cos x \cos y -\sen x \sen y$\\
		    &$=$&$\cos(x-y)+\cos(x+y)$\\\\
		\end{tabular}
	    \end{center}


	%----------(b)
	\item $2\sen x \sen y = \cos(x-y)-\cos(x+y).$\\\\
	    Demostración.-\; Similar al anterior ejercicios tenemos,\\
	    \begin{center}
		\begin{tabular}{rcl}
		    $2\sen x \sen y$&$=$&$\sen x \sen y + \sen x \sen y$\\
				    &$=$&$\cos x \cos y + \sen x \sen y - (\cos x \cos y - \sen x \sen y)$\\
				    &$=$&$\cos(x-y)-\cos(x+y)$\\\\
		\end{tabular}
	    \end{center}


	%----------(c)
	\item $2\sen x \cos y = \sen(x-y)+\sen(x+y).$\\\\
	    Demostración.-\; Usando la fórmula del seno de una suma y diferencia:\\
	    \begin{center}
		\begin{tabular}{rcl}
		    $2\sen x \cos y$&$=$&$\sen x \cos y + \sen x \cos y$\\
		      &$=$&$\sen x \cos y - \sen y \cos x + \sen x \cos y + \sen y \cos x$\\
		      &$=$&$\sen(x-y) + \sen(x+y)$\\\\
		\end{tabular}
	    \end{center}

    \end{enumerate}

%--------------------15.
\item Si $h\neq 0$, demostrar que las identidades siguientes son válidas para todo $x$:
    $$\dfrac{\sen(x+h)-\sen x}{h} = \dfrac{\sen(h/2)}{h/2}\cos\left(x+\dfrac{h}{2}\right)$$
    $$\dfrac{\cos(x+h)-\cos x}{h} = -\dfrac{\sen(h/2)}{h/2}\cos\left(x+\dfrac{h}{2}\right)$$
    Estas fórmulas se utilizan en cálculo diferencial.\\\\
    Demostración.-\; Usando el seno y coseno para la suma y diferencia, se sigue,
    \begin{center}
	\begin{tabular}{rcl}
	    $\dfrac{\sen(x+h)-\sen x}{h}$&$=$&$\dfrac{\sen(x+\frac{h}{2}+\frac{h}{2})-\sen(x+\frac{h}{2}-\frac{h}{2})}{h}$\\\\
					 &$=$&$\dfrac{\sen(x+\frac{h}{2})\cos(\frac{h}{2})+\sen(\frac{h}{2})\cos(x+\frac{h}{2})-\sen(x+\frac{h}{2})\cos (\frac{h}{2})+\sen(\frac{h}{2})\cos(x+\frac{h}{2})}{h}$\\\\
					 &$=$&$\dfrac{2\sen \frac{h}{2}}{h}\cos\left(x+\dfrac{h}{2}\right)$\\\\
					 &$=$&$\dfrac{\sen \frac{h}{2}}{h/2}\cos\left(x+\dfrac{h}{2}\right)$\\\\
	\end{tabular}
    \end{center}
    y
    \begin{center}
	\begin{tabular}{rcl}
	    $\dfrac{\cos(x+h)-\cos x}{h}$&$=$&$\dfrac{\cos(x+\frac{h}{2}+\frac{h}{2})-\cos(x+\frac{h}{2}-\frac{h}{2})}{h}$\\\\
					 &$=$&$\dfrac{\cos(x+\frac{h}{2})\cos(\frac{h}{2})-\sen(x+\frac{h}{2})\sen(\frac{h}{2})-\cos(x+\frac{h}{2})\cos(\frac{h}{2})-\sen(x+\frac{h}{2})\sen(\frac{h}{2})}{h}$\\\\
					 &$=$&$\dfrac{2\sen(x+\frac{h}{2})\sen(\frac{h}{2})}{h}$\\\\
					 &$=$&$-\dfrac{\sen(\frac{h}{2})}{h/2}\sen\left(x+\dfrac{h}{2}\right)$\\\\
	\end{tabular}
    \end{center}

%--------------------16.
\item Demostrar si son o no ciertas las siguientes afirmaciones.

    \begin{enumerate}[\bfseries (a)]

	%----------(a)
	\item Para todo $x\neq 0$ se tiene $\sen2x \neq 2\sen x$.\\\\
	    Demostración.-\; Sea $x=\pi$, entonces $\sen 2 x = \sen2\pi = 0$, y $2\sen x = 2\sen \pi = 0$. Luego por hipótesis $x\neq 0$,  pero $\sen2x = 2\sen x$, por lo tanto este proposición es falsa.\\\\

	%----------(b)
	\item Para cualquier $x$, existe un y tal que $\cos(x+y)=\cos x+\cos y$.\\\\
	    Demostración.-\; Sea $x=0$, entonces $\cos(x+y) = \cos y,$ pero $\cos x +\cos y = 1+\cos y.$ Por lo tanto la proposición es falsa.\\\\

	%----------(c)
	\item Existe un $x$ tal que $\sen(x+y)=\sen x + \sen y $ para todo $y$.\\\\
	    Demostración.-\; Sea $x=0$, entonces $\sen(x+y) = \sen y,$ para todo $y$, y $x+\sen y =\sen y$ para todo $y.$ Y por lo tanto la proposición es verdadera.\\\\

	%----------(d)
	\item Existe un $y\neq 0$ tal que $\int_0^y \sen x\; dx = \sen y$.\\\\
	    Demostración.-\; Sea $y=\dfrac{\pi}{2}$, entonces $\displaystyle\int_0^pi 2 \sen x \; dx = 1-\cos\dfrac{\pi}{2}=1$ y $\sen \dfrac{\pi}{2}=1.$ Por lo tanto la proposición es verdadera.\\\\

    \end{enumerate}

%--------------------17.
\item Calcular la integral $\int_a^b \sen x \; dx$ para cada uno de los siguientes valores de $a$ y $b$. En cada caso interpretar el resultado geométricamente en función del área.

    \begin{enumerate}[\bfseries (a)]

	%----------(a)
	\item $a=0,\; b=\pi/6.$\\\\
	    Respuesta.-\; $\displaystyle\int_0^{\pi/6}\sen x \; dx = 1-\cos(\pi/6) = 1-\sqrt{3}/2$.\\\\

	%----------(b)
	\item $a=0,b=\pi/4.$\\\\
	    Respuesta.-\; $\displaystyle\int_0^{\pi/4}\sen x \; dx = 1-\cos(\pi/4) = 1-\sqrt{2}/2$.\\\\

	%----------(c)
	\item $a=0,b=\pi/3.$\\\\
	    Respuesta.-\; $\displaystyle\int_0^{\pi/3}\sen x \; dx = 1-\cos(\pi/3) = 1-\dfrac{1}{2}=\dfrac{1}{2}$.\\\\

	%----------(d)
	\item $a=0,b=\pi/2.$\\\\
	    Respuesta.-\; $\displaystyle\int_0^{\pi/2}\sen x \; dx = 1-\cos(\pi/2) = 1$.\\\\

	%----------(e)
	\item $a=0,b=\pi.$\\\\
	    Respuesta.-\; $\displaystyle\int_0^{\pi}\sen x \; dx = 1-\cos(\pi) = 1-(-1) = 2$.\\\\

	%----------(f)
	\item $a=0,b=2\pi.$\\\\
	    Respuesta.-\; $\displaystyle\int_0^{2\pi}\sen x \; dx = 1-\cos(2\pi) = 1-1 = 0$.\\\\

	%----------(g)
	\item $a=-1,b=1$\\\\
	    Respuesta.-\; $\displaystyle\int_{-1}^{1}\sen x \; dx = -[\cos(1) - \cos( -1)] = 0$.\\\\

	%----------(h)
	\item $a=-\pi/6,b=\pi/4$.\\\\
	    Respuesta.-\; $\displaystyle\int_{\pi/4}^{-\pi/6}\sen x \; dx = -[\cos(\pi/4) - \cos(-\pi/6)] = -\left(\dfrac{\sqrt{2}}{2}-\dfrac{\sqrt{3}}{2}\right) = \dfrac{\sqrt{3}-\sqrt{2}}{2}$.\\\\

    \end{enumerate}

%--------------------18.
\item $\displaystyle\int_{0}^\pi (x+\sen x)\; dx = \dfrac{x^2}{2}\bigg|_0^\pi + 1 - (\cos \pi) = \dfrac{\pi^2}{2}+1-(-1) = \dfrac{\pi}{2} + 2$.\\\\

%--------------------19.
\item $\displaystyle\int_{0}^{\pi/2} (x^2+\cos x)\; dx = \dfrac{x^3}{3}\bigg|_0^{\pi/2}+\sen\left(\dfrac{\pi}{2}\right) = \dfrac{(\frac{\pi}{2})^3}{3} + 1 = \dfrac{\pi^3}{24}+1.$\\\\

%--------------------20.
\item $\displaystyle\int_0^{\pi/2} (\sen x - \cos x)\; dx = 1-\cos \left(\dfrac{\pi}{2}\right) - \sen\left(\dfrac{\pi}{2}\right) = 1-0-1 = 0$.\\\\

%--------------------21.
\item $\displaystyle\int_0^{\pi/2} |\sen x - \cos x|\; dx$\\\\
    Respuesta.-\; Ya que $\cos x - \sen x \geq 0$ esta definida por $\left(0,\frac{\pi}{4}\right)$ y $\cos x - \sen x < 0$ esta dado por $\left(\frac{\pi}{4},\frac{\pi}{2}\right]$ tenemos,
    \begin{center}
	\begin{tabular}{rcl}
	    $\displaystyle\int_0^{\pi/2} |\sen x - \cos x|\; dx$&$=$&$\displaystyle\int_0^{\pi/4}(\cos x - \sen x)\; dx + \int_{\pi/4}^{\pi/2}(\sen x - \cos x)\; dx$\\\\
								&$=$&$\sen\left(\dfrac{\pi}{4}\right)-\left[1-\cos\left(\dfrac{\pi}{4}\right)\right]  -\left[\cos\left(\dfrac{\pi}{2}\right)-\cos\left(\dfrac{\pi}{4}\right)\right]+\left[\sen\left(\dfrac{\pi}{2}\right)-\sen \left(\dfrac{\pi}{4}\right)\right]$\\\\
								&$=$&$\dfrac{\sqrt{2}}{2}+\dfrac{\sqrt{2}}{2}-1+\dfrac{\sqrt{2}}{2}-0-1+\dfrac{\sqrt{2}}{2}$\\\\
	      &$=$&$2\sqrt{2}-2$\\\\
	  \end{tabular}
    \end{center}

%--------------------22.
\item $\displaystyle\int_0^\pi \left(\dfrac{1}{2}+\cos t\right) \; dt = \dfrac{t}{2}\bigg|_0^\pi+ \sen \pi = \dfrac{\pi}{2}.$\\\\

%--------------------23.
\item $\displaystyle\int_0^\pi  \bigg|\dfrac{1}{2} + \cos t\bigg|\; dx$.\\\\
    \begin{center}
	\begin{tabular}{rcl}
	    $\displaystyle\int_{0}^{\pi} \bigg| \dfrac{1}{2}+\cos t\bigg| \; dt$&$=$&$\displaystyle\int_0^{\frac{2\pi}{3}}\left(\dfrac{1}{2} + \cos t\right)\; dt + \int_{\frac{2\pi}{3}}^{\pi}\left(\dfrac{1}{2} + \cos t\right)\; dt$\\\\
										&$=$&$\displaystyle\int_0^{\frac{2\pi}{3}} \dfrac{1}{2}\; dt + \int_{0}^{\frac{2\pi}{3}} \cos(t) \; dt - \int_{\frac{2\pi}{3}}\pi \dfrac{1}{2}\; dt + \int_{\frac{2\pi}{3}}^\pi \cos(t) \; dt$\\\\
										&$=$&$\dfrac{t}{2}\bigg|_0^{\frac{2\pi}{3}} + \sen(t)\bigg|_{0}^{\frac{2\pi}{3}} -  \dfrac{t}{2} \bigg|_{\frac{2\pi}{3}}^\pi \dfrac{1}{2}\; dt - \sen(t)\bigg|_{\frac{2\pi}{3}}^\pi$\\\\
										&$=$&$\dfrac{\pi}{3}+\sen\left(\dfrac{2\pi}{3}\right) - \left(\dfrac{\pi}{2}-\dfrac{2\pi}{6}\right)-\left[\sen \pi - \sen\left( \dfrac{2\pi}{3}\right)\right]$\\\\
										&$=$&$\dfrac{\pi}{6}+\sqrt{3}$\\\\
	\end{tabular}
    \end{center}

%--------------------24.
\item 

\end{enumerate}



