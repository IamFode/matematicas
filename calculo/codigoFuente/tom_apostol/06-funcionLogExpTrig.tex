\chapter{Función Logaritmo, función exponencial y funciones trigonométricas inversas}

\setcounter{section}{1}
\section{definición de logaritmo natural como integral}

Una ecuación tal como 
$$f(xy)=f(x)+f(y)$$ 
que expresa una relación entre los valores de una función en dos o más puntos se denomina \textbf{ecuación funcional}. Una solución de $f(xy)=f(x)+f(y)$ es la función que es cero en todo el eje real; y demás es la única solución que está definida para todos los números reales. En efecto: sea $f$ una función que satisfaga la ecuación funcional, si $0$ pertenece al dominio de $f$ se puede poner $y=0$ obteniéndose $f(0)=f(x)+f(0)$ lo que implica $f(x)=0$ para cada $x$ en el dominio de $f$. Dicho de otra forma, si $0$ pertenece al dominio de $f$, $f$ ha de ser idénticamente nula. Por tanto, una solución no idénticamente nula no puede estar definida en $0$.\\

Si $f$ es una solución de $f(xy)=f(x)+f(y)$ y el dominio de $f$ contiene el punto $1$, se puede poner $x=y=1$ y se obtiene $f(1)=2f(1)$ de donde
$$f(1)=0.$$
Si ambos $1$ y $-1$ pertenecen al dominio de $f$ se puede tomar $x=-1$ e $y=-1$ de donde se deduce $f(1)=2f(-1)$ es decir $f(-1)=0$. Si ahora, $x$, $-x$, $1$, $-1$ pertenecen al dominio de $f$ se puede poner $y=-1$ obteniéndose $f(-x)=f(-1)+f(x)$, y puesto que $f(-1)=0$ se tiene
$$f(-x)=f(x).$$
Es decir, toda solución de $f(xy)=f(x)+f(y)$ es necesariamente una función par.\\

Supóngase ahora, que $f$ tiene una derivada $f'(x)$ en cada $x\neq 0$. Dejando fijo en $f(xy)=f(x)+f(y)$ y derivando con respecto a $x$ (aplicando en el primer miembro la regla de la cadena) se tiene:
$$yf'(xy)=f'(x).$$
Si $x=1$, de esta ecuación se deduce $yf'(y)=f'(1)$. Por tanto:
$$f'(y)=\dfrac{f(1)}{y}\mbox{ para cada }y\neq 0.$$

La derivada es monótona e integrable en cada intervalo cerrado que no contenga el origen. Además, $f'$ es continua en cada uno de estos intervalo y se puede aplicar el segundo teorema fundamental del cálculo:
$$f(x)-f(c)=\int_c^x f'(t)\; dt = f'(1)\int_c^x \dfrac{1}{t}\; dt.$$
Si $x>0$, esta ecuación es válida para cada positivo $c>0$, y si es $x<0$ es válida para cada $c$ negativo. Puesto que $f(1)=0$, eligiendo $c=1$ se tiene
$$f(x)=f'(1)\int_1^x \dfrac{1}{t}\; dt\quad \mbox{ si } \quad x>0.$$
Si $x$ es negativa, $-x$ es positiva y puesto que $f(x)=f(-x)$ se tiene:
$$f(x)=f'(1)\int_1^{-x}\dfrac{1}{t}\; dt\quad \mbox{si}\quad x<0.$$
Estas dos formulas se puede unir en una sola:
$$f(x)=f'(1)\int_1^{|x|}\dfrac{1}{t}\; dt\quad \mbox{si}\quad x\neq 0.$$
En consecuencia, si existe una solución de $f(xy)=f(x)+f(y)$ que tiene una derivada en cada punto $x\neq 0$ esta solución ha de venir dada necesariamente por la fórmula integral de $f(x)=f'(1)\int_1^{|x|}\frac{1}{t}\; dt$. Si $f'(1)=0$ entonces $f(x)=0$ para cada $x\neq 0$ y esta solución coincide con la idénticamente nula. Por tanto, si $f$ no es idénticamente nula ha de ser $f'(1)\neq 0$, en cuyo caso se pueden dividir ambos miembros por $f'(1)$ obteniéndose
$$g(x)=\int_1^{|x|}\dfrac{1}{t}\; dt\quad \mbox{si}\quad x\neq 0.$$
donde $g(x)=f(x)/f'(1)$. La función $g$ es también una solución de $f(xy)=f(x)-f(y)$, puesto que si $f$ es solución también lo es $cf$. Esto demuestra que si tiene una solución que no es la idénticamente nula, y si esta función es derivable en todos los puntos, excepto en el origen, entonces la función $f$ es una solución, y todas las soluciones pueden obtenerse de esta multiplicando $g$ por una constante conveniente.\\

Dos números distintos tendrían un mismo logaritmo, puesto que la función $g$ tendría la propiedad: $g(x) = g( - x)$. En atención a consideraciones que posteriormente se harán, es preferible definir el logaritmo de manera que dos números distintos no tengan el  mismo logaritmo, lo cual se logra definiendo el logaritmo sólo para los números positivos. Por tanto, se tomará la siguiente definición.


\section{Definición de logaritmo. Propiedades fundamentales}

%-------------------- Definición de logaritmo  6.1.
\begin{def.}
    Si $x$ es un número real positivo, definamos el logaritmo natural de $x$, designando provisionalmente por $L(x)$, como la integral
    $$L(x)=\int_1^{x}\dfrac{1}{t}\; dt.$$
    Cuando $x>1$, $L(x)$ puede interpretarse geométricamente como el área de la región sombreada.
\end{def.}

%-------------------- Teorema 6.1.
\begin{teo}
    La función logaritmo tiene las propiedades siguientes:
    \begin{enumerate}[a)]
	\item $L(1)=0.$
	\item $L'(x)=\dfrac{1}{x}$ para todo $x>0.$
	\item $L(ab)=L(a)+L(b)$ para todo $a>0$, $b>0.$\\
    \end{enumerate}
	Demostración.-\; La parte a) se deduce inmediatamente de la definición. Para demostrar b), observamos simplemente que $L$ es una integral indefinida de una función continua y apliquemos el primer teorema fundamental del Cálculo. La propiedad c) es consecuencia de la propiedad aditiva de la integral. Escribamos
	$$L(ab)=\int_1^{ab}\dfrac{dt}{t}=\int_1^a \dfrac{dt}{t}+\int_a^{ab}\dfrac{dt}{t}=L(a)+\int_a^{ab}\dfrac{dt}{t}.$$
	En la última integral hemos hecho la sustitución $u=t/a$, $du=dt/a$, y encontramos que la integral se deduce a $L(b)$, lo que demuestra c).
\end{teo}


\section{Gráfica del logaritmo natural}
La derivada segunda es $L''(x) = - 1/x^2$ que es negativa para todo $x$, por lo que $L$ es una función cóncava.

\section{Consecuencias de la ecuación funcional \boldmath$L(ab)=L(a)+L(b)$}

La función no está acotada superiormente; esto es, para todo número positivo $M$ (por grande que sea) existen valores de $x$ tales que
$$L(x)>M.$$
Podemos deducirlo de la ecuación funcional. Cuando $a=b$, tenemos $L\left(a^2\right)=2L(a)$. Utilizando la ecuación funcional una vez más poniendo $b=a^2$, obtenemos $L\left(a^3\right)=3L(a)$. Por inducción encontramos la fórmula general
$$L\left(a^n\right)=nL(a)$$
para cualquier entero $n\geq 1$. Cuando $a=2$, se obtiene $L\left(2^n\right)=nL(2)$, y por tanto resulta
$$L\left(2^n\right) \quad \mbox{cuando}\quad n>\dfrac{M}{L(2)}.$$
Esto demuestra la afirmación 
$$L(x)>M.$$
Tomando $b=1/a$ en la ecuación funcional, encontramos $L(1/a)=-L(a)$. En particular, cuando $a=2^n$, habiendo elegido $n$ como en $L(x)>M$, se tiene
$$L\left(\dfrac{1}{2^n}\right)=-L(2^n)<-M,$$
lo que indica que tampoco existe cota inferior para los valores de la función.

%-------------------- Teorema 6.2.
\begin{teo}
    Para cada número real $b$ existe exactamente un número real positivo $x$ cuyo logaritmo, $L(a)$, es igual a $b$.\\\\
    Demostración.-\; Si $b>0$, elegimos un entero cualquiera $n>b/L(2)$. Entonces, en virtud de $L(x)>M$, $L\left(2^n\right)>b.$ Seguidamente examinamos la función $L$ en el intervalo cerrado $[1,2^n]$. Su valor en el extremo izquierdo es $L(1)=0$, y en el extremo derecho es $L(2^n)$. Puesto que $0<b<L\left(2^n\right)$, el teorema del valor intermedio para funciones continuas (teorema 3.8, sección 3.10) asegura la existencia por lo menos de un $a$ tal que $L(a)=b$. No puede existir otro valor $a'$ tal que $L\left(a'\right)=b$ porque esto significaría $L(a)=L\left(a'\right)$ para $a\neq a'$, y esto contradice la propiedad de crecimiento del logaritmo. Por consiguiente la proposición $L(a)=b$ ha sido demostrada para $b>0$. La demostración para $b$ negativo es consecuencia de esa si utilizamos la igualdad $L(1/a)=-L(a)$.
\end{teo}

En particular, existe un único número cuyo logaritmo natural es igual a $1$. Este número, al igual que $\pi$, se encuentra tan repetidamente en fórmulas matemáticas que es inevitable el adoptar para él un símbolo especial. Este número es $e$.

%-------------------- Definición 6.2.
\begin{def.}
    Designamos por $e$ el número para el que
    $$L(e)=1.$$
\end{def.}

Los logaritmos naturales se denomina también \textbf{logaritmos neperianos}. En honor al inventor, Juan Neper (1550-1617). Es frecuente en la práctica utilizar los símbolos $\ln x$ o $\log x$ en vez de $L(x)$ para designar el logaritmo de $x$.


\section{Logaritmos referidos a un base positiva \boldmath $b\neq 1$}
