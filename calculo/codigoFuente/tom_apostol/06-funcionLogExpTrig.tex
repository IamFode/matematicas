\chapter{Función Logaritmo, función exponencial y funciones trigonométricas inversas}

\setcounter{section}{1}
\section{definición de logaritmo natural como integral}

Una ecuación tal como 
$$f(xy)=f(x)+f(y)$$ 
que expresa una relación entre los valores de una función en dos o más puntos se denomina \textbf{ecuación funcional}. Una solución de $f(xy)=f(x)+f(y)$ es la función que es cero en todo el eje real; y además es la única solución que está definida para todos los números reales. En efecto: sea $f$ una función que satisfaga la ecuación funcional, si $0$ pertenece al dominio de $f$ se puede poner $y=0$ obteniéndose $f(0)=f(x)+f(0)$ lo que implica $f(x)=0$ para cada $x$ en el dominio de $f$. Dicho de otra forma, si $0$ pertenece al dominio de $f$, $f$ ha de ser idénticamente nula. Por tanto, una solución no idénticamente nula no puede estar definida en $0$.\\

Si $f$ es una solución de $f(xy)=f(x)+f(y)$ y el dominio de $f$ contiene el punto $1$, se puede poner $x=y=1$ y se obtiene $f(1)=2f(1)$ de donde
$$f(1)=0.$$
Si ambos $1$ y $-1$ pertenecen al dominio de $f$ se puede tomar $x=-1$ e $y=-1$ de donde se deduce $f(1)=2f(-1)$ es decir $f(-1)=0$. Si ahora, $x$, $-x$, $1$, $-1$ pertenecen al dominio de $f$ se puede poner $y=-1$ obteniéndose $f(-x)=f(-1)+f(x)$, y puesto que $f(-1)=0$ se tiene
$$f(-x)=f(x).$$
Es decir, toda solución de $f(xy)=f(x)+f(y)$ es necesariamente una función par.\\

Supóngase ahora, que $f$ tiene una derivada $f'(x)$ en cada $x\neq 0$. Dejando fijo en $f(xy)=f(x)+f(y)$ y derivando con respecto a $x$ (aplicando en el primer miembro la regla de la cadena) se tiene:
$$yf'(xy)=f'(x).$$
Si $x=1$, de esta ecuación se deduce $yf'(y)=f'(1)$. Por tanto:
$$f'(y)=\dfrac{f(1)}{y}\mbox{ para cada }y\neq 0.$$

La derivada es monótona e integrable en cada intervalo cerrado que no contenga el origen. Además, $f'$ es continua en cada uno de estos intervalo y se puede aplicar el segundo teorema fundamental del cálculo:
$$f(x)-f(c)=\int_c^x f'(t)\; dt = f'(1)\int_c^x \dfrac{1}{t}\; dt.$$
Si $x>0$, esta ecuación es válida para cada positivo $c>0$, y si es $x<0$ es válida para cada $c$ negativo. Puesto que $f(1)=0$, eligiendo $c=1$ se tiene
$$f(x)=f'(1)\int_1^x \dfrac{1}{t}\; dt\quad \mbox{ si } \quad x>0.$$
Si $x$ es negativa, $-x$ es positiva y puesto que $f(x)=f(-x)$ se tiene:
$$f(x)=f'(1)\int_1^{-x}\dfrac{1}{t}\; dt\quad \mbox{si}\quad x<0.$$
Estas dos formulas se puede unir en una sola:
$$f(x)=f'(1)\int_1^{|x|}\dfrac{1}{t}\; dt\quad \mbox{si}\quad x\neq 0.$$
En consecuencia, si existe una solución de $f(xy)=f(x)+f(y)$ que tiene una derivada en cada punto $x\neq 0$ esta solución ha de venir dada necesariamente por la fórmula integral de $f(x)=f'(1)\int_1^{|x|}\frac{1}{t}\; dt$. Si $f'(1)=0$ entonces $f(x)=0$ para cada $x\neq 0$ y esta solución coincide con la idénticamente nula. Por tanto, si $f$ no es idénticamente nula ha de ser $f'(1)\neq 0$, en cuyo caso se pueden dividir ambos miembros por $f'(1)$ obteniéndose
$$g(x)=\int_1^{|x|}\dfrac{1}{t}\; dt\quad \mbox{si}\quad x\neq 0.$$
donde $g(x)=f(x)/f'(1)$. La función $g$ es también una solución de $f(xy)=f(x)-f(y)$, puesto que si $f$ es solución también lo es $cf$. Esto demuestra que si tiene una solución que no es la idénticamente nula, y si esta función es derivable en todos los puntos, excepto en el origen, entonces la función $f$ es una solución, y todas las soluciones pueden obtenerse de estar multiplicando $g$ por una constante conveniente.\\

Dos números distintos tendrían un mismo logaritmo, puesto que la función $g$ tendría la propiedad: $g(x) = g( - x)$. En atención a consideraciones que posteriormente se harán, es preferible definir el logaritmo de manera que dos números distintos no tengan el  mismo logaritmo, lo cual se logra definiendo el logaritmo sólo para los números positivos. Por tanto, se tomará la siguiente definición.


\section{Definición de logaritmo. Propiedades fundamentales}

%-------------------- Definición de logaritmo  6.1.
\begin{def.}
    Si $x$ es un número real positivo, definamos el logaritmo natural de $x$, designando provisionalmente por $L(x)$, como la integral
    $$L(x)=\int_1^{x}\dfrac{1}{t}\; dt.$$
    Cuando $x>1$, $L(x)$ puede interpretarse geométricamente como el área de la región sombreada.
\end{def.}

%-------------------- Teorema 6.1.
\begin{teo}
    La función logaritmo tiene las propiedades siguientes:
    \begin{enumerate}[a)]
	\item $L(1)=0.$
	\item $L'(x)=\dfrac{1}{x}$ para todo $x>0.$
	\item $L(ab)=L(a)+L(b)$ para todo $a>0$, $b>0.$\\
    \end{enumerate}
	Demostración.-\; La parte a) se deduce inmediatamente de la definición. Para demostrar b), observamos simplemente que $L$ es una integral indefinida de una función continua y apliquemos el primer teorema fundamental del Cálculo. La propiedad c) es consecuencia de la propiedad aditiva de la integral. Escribamos
	$$L(ab)=\int_1^{ab}\dfrac{dt}{t}=\int_1^a \dfrac{dt}{t}+\int_a^{ab}\dfrac{dt}{t}=L(a)+\int_a^{ab}\dfrac{dt}{t}.$$
	En la última integral hemos hecho la sustitución $u=t/a$, $du=dt/a$, y encontramos que la integral se deduce a $L(b)$, lo que demuestra c).
\end{teo}


\section{Gráfica del logaritmo natural}
La derivada segunda es $L''(x) = - 1/x^2$ que es negativa para todo $x$, por lo que $L$ es una función cóncava.

\section{Consecuencias de la ecuación funcional \boldmath$L(ab)=L(a)+L(b)$}

La función no está acotada superiormente; esto es, para todo número positivo $M$ (por grande que sea) existen valores de $x$ tales que
$$L(x)>M.$$
Podemos deducirlo de la ecuación funcional. Cuando $a=b$, tenemos $L\left(a^2\right)=2L(a)$. Utilizando la ecuación funcional una vez más poniendo $b=a^2$, obtenemos $L\left(a^3\right)=3L(a)$. Por inducción encontramos la fórmula general
$$L\left(a^n\right)=nL(a)$$
para cualquier entero $n\geq 1$. Cuando $a=2$, se obtiene $L\left(2^n\right)=nL(2)$, y por tanto resulta
$$L\left(2^n\right) \quad \mbox{cuando}\quad n>\dfrac{M}{L(2)}.$$
Esto demuestra la afirmación 
$$L(x)>M.$$
Tomando $b=1/a$ en la ecuación funcional, encontramos $L(1/a)=-L(a)$. En particular, cuando $a=2^n$, habiendo elegido $n$ como en $L(x)>M$, se tiene
$$L\left(\dfrac{1}{2^n}\right)=-L(2^n)<-M,$$
lo que indica que tampoco existe cota inferior para los valores de la función.

%-------------------- Teorema 6.2.
\begin{teo}
    Para cada número real $b$ existe exactamente un número real positivo $x$ cuyo logaritmo, $L(a)$, es igual a $b$.\\\\
    Demostración.-\; Si $b>0$, elegimos un entero cualquiera $n>b/L(2)$. Entonces, en virtud de $L(x)>M$, $L\left(2^n\right)>b.$ Seguidamente examinamos la función $L$ en el intervalo cerrado $[1,2^n]$. Su valor en el extremo izquierdo es $L(1)=0$, y en el extremo derecho es $L(2^n)$. Puesto que $0<b<L\left(2^n\right)$, el teorema del valor intermedio para funciones continuas (teorema 3.8, sección 3.10) asegura la existencia por lo menos de un $a$ tal que $L(a)=b$. No puede existir otro valor $a'$ tal que $L\left(a'\right)=b$ porque esto significaría $L(a)=L\left(a'\right)$ para $a\neq a'$, y esto contradice la propiedad de crecimiento del logaritmo. Por consiguiente la proposición $L(a)=b$ ha sido demostrada para $b>0$. La demostración para $b$ negativo es consecuencia de esa si utilizamos la igualdad $L(1/a)=-L(a)$.
\end{teo}

En particular, existe un único número cuyo logaritmo natural es igual a $1$. Este número, al igual que $\pi$, se encuentra tan repetidamente en fórmulas matemáticas que es inevitable el adoptar para él un símbolo especial. Este número es $e$.

%-------------------- Definición 6.2.
\begin{def.}
    Designamos por $e$ el número para el que
    $$L(e)=1.$$
\end{def.}

Los logaritmos naturales se denomina también \textbf{logaritmos neperianos}. En honor al inventor, Juan Neper (1550-1617). Es frecuente en la práctica utilizar los símbolos $\ln x$ o $\log x$ en vez de $L(x)$ para designar el logaritmo de $x$.


\section{Logaritmos referidos a un base positiva \boldmath $b\neq 1$}

En la sección 6.2 se vio que la función $f$ más general derivable en el eje real, que satisface la ecuación funcional $f(xy)=f(x)+f(y)$ está por la fórmula
$$f(x)=c\log x,$$
donde $c$ es una constante. Para cada $c$ esta $f(x)$ se denominará el logaritmo de $x$ asociado a $c$, y como es evidente, su valor no será necesariamente el mismo que el logaritmo natural de $x$. Si $c=0$, $f$ es idénticamente nulo y este caso carece de interés. Si $c\neq 0$ se indicará de otra forma la dependencia de $f$ y $c$ introduciendo el concepto de \textbf{base} de logaritmos.\\
De $f(x)=c\log x$ se deduce que cuando $c\neq 0$ existe un número real único $b>0$ tal que $f(b)=1$. Esta $b$ está relacionada con $c$ por medio de la igualdad $c\log b=1$; como $b\neq 1$ es $c=1/\log b,$ lo que expresamos como
$$f(x)=\dfrac{\log x}{\log b}.$$

Para esta elección de $c$ se dice que $f(x)$ es el logaritmo de $x$ en base $b$ y se escribe $\log_b x$ en vez de $f(x)$.

%-------------------- Definición 6.3.
\begin{def.}
    Si $b>0$, $b\neq 1$, y si $x>0$, el logaritmo de $x$ en base $b$ es el número
    $$\log_b x = \dfrac{\log x}{\log b},$$
    donde los logaritmos del segundo miembro son logaritmos naturales.
\end{def.}

Obsérvese que $\log_b b = 1$. Si $b=e$ se tiene $\log_e x= \log x$, es decir, los logaritmos naturales son los que tienen de base $e$. Puesto que los logaritmos de base $e$ son los más frecuentemente usando en matemáticas, la palabra logaritmo indica casi siempre el logaritmo natural.


\section{Fórmulas de derivación e integración en las que intervienen logaritmos}
Puesto que la derivada del logaritmo viene dada por la fórmula $D\log x = 1/x$ para $x>0$, se tiene la fórmula de integración 
$$\int \dfrac{1}{x}\; dx = \log(x)+C.$$
Aún más general, si $u=f(x)$, siendo $f$ una función con derivada continua, se tiene
$$\int \dfrac{du}{u} = \log u + C \quad \mbox{o}\quad \int \dfrac{f'(x)}{f(x)}\; dx = \log f(x) + C.$$
para $u$ o $f(x)$ positiva.\\

Afortunadamente, es fácil extender el campo de validez de etas fórmulas de manera que pueden aplicarse para funciones que sean positivas o negativas, pero no cero. Se introduce simplemente una nueva función $L_0$ definida para todos los números reales $x\neq 0$ por la ecuación:
$$L_0(x)=\log|x|=\int_1^{|x|}\dfrac{1}{t}\; dt.$$
Puesto que $\log|xy|=\log(|x||y|)=\log|x|+\log|y|$, la función $L_0$ satisface también la ecuación funcional básica; es decir, se tiene:
$$L_0(xy)=L_0(x)+L_0(y)$$
para $x$ e $y$ reales cualesquiera distintos de cero. Para $x>0$ se tiene $L_0'(x)=1/x$ ya que $L_0(x)$ para $x$ positivo es lo mismo que $\log(x)$. La fórmula de la derivada vale también para $x<0$ puesto que en este caso $L_0(x)=L(-x)$ y por tanto $L_0(x)=-L'(-x)=-1/(-x)=1/x$. De aquí resulta
$$L_0'(x)=\dfrac{1}{x}\quad \mbox{para todo valor real }\; x\neq 0.$$
Por tanto, si en las fórmulas de integración precedentes se pone $L_0$ en vez de $L$, se puede extender su alcance a funciones que toman valores tanto negativos como positivos. Por ejemplo 
$$\int \dfrac{du}{u} = \log u + C \quad \mbox{o}\quad \int \dfrac{f'(x)}{f(x)}\; dx = \log f(x) + C.$$
se puede generalizar como sigue:
$$\int \dfrac{du}{u} = \log |u| + C, \quad \int \dfrac{f'(x)}{f(x)}\; dx = \log |f(x)| + C.$$
Evidentemente, cuando se aplique junto con el segundo teorema funda- mental del Cálculo para calcular una integral indefinida no se pueden tomar intervalos que incluyan puntos en los que $u$ o $f(x)$ sean cero.


Una de las propiedades más notables de la función exponencial es la fórmula
$$E'(x)=E(x),$$
que nos dice que la función es su propia derivada. \\

Supongamos $f(x)=a^x$ para $x>0$. Según la definición de $a^x$, podemos escribir
$$f(x)=e^{x\log a}=E(x\log a);$$
luego en virtud de la regla de cadena, encontramos
$$f'(x)=E'(x\log a )\cdot \log a=E(x\log a)\cdot \log a=a^x\log a.$$
Dicho de otro modo, la derivación de $a^x$ multiplica simplemente $a^x$ por el factor constante $\log a$, siendo este factor $1$ cuando $a=e$. Por ejemplo $E'(x)=E(x)$ da como resultado 
$$\int e^x \; dx = e^x+C,$$
en tanto que $f'(x)=a^x\log a$ conduce a la fórmula más general
$$\int a^x \; dx = \dfrac{a^x}{\log a}+C \quad (a>0,\; a\neq 1).$$
Sustituimos $x$ en $\int e^x \; dx = e^x+C,$ y la ecuación anterior por $u$ obteniendo
$$\int e^u \; du = e^u+C, \quad \int a^u \; du = \dfrac{a^u}{\log a}+C (a>0,\; a\neq 1),$$
donde $u$ representa cualquier función con derivada continua. Si escribimos $u=f(x),$ y $fu=f'(x)\; dx$, las fórmulas anteriores se convierten En
$$\int e^{f(x)} f'(x) \; dx = e^{f(x)}+C, \quad \int a^{f(x)} f'(x) \; dx = \dfrac{a^{f(x)}}{\log a}+C (a>0,$$
siendo la segunda integral válida para $a>0$, $a\neq 1$.

%-------------------- Ejercicio 6.1
\begin{ejem}
    Integrar $\displaystyle\int x^2e^{x^3}\;dx.$\\\\
	Respuesta.-\; Sea $u=x^3$. Entonces, $fu=3x^2\; dx$  y se obtiene
	$$\int x^2e^{x^3}\; dx = \dfrac{1}{3}\int e^{x^3}(3x^2\; dx)=\dfrac{1}{3}\int e^u\; dx = \dfrac{1}{3}e^{x^3}+C.$$
\end{ejem}

%-------------------- Ejercicio 6.2
\begin{ejem}
    Integrar $\displaystyle\int \log(x)\; dx.$\\\\
	Respuestas.-\; Sea $u=\log(x)$, $dv=dx.$ Entonces, $du=dx/x$, $v=x$, y obtenemos
	$$\int \log(x)\; dx = \int u\; dv = uv-\int v\; du = x\log(x)-\int x\dfrac{1}{x}\; dx = x\log(x)-x+C.$$
\end{ejem}


\section{Derivación logarítmica}
El método fue desarrollado en 1697 por Johann Bernoulli (1667-1748) y su fundamento es una hábil aplicación de la regla de la cadena.\\

Supóngase que se forma la función compuesta de $L_0$ con una función derivable cualquier $f(x)$; es decir,
$$g(x)=L_0\left[f(x)\right] = \log|f(x)|$$
para todo $x$ tal que $f(x)\neq 0$. La regla de la cadena aplica junto con $L'_0(x)=\dfrac{1}{x},\; x\neq 0$ conduce a la fórmula
$$g'(x)=L'_0\left[f(x)\right]\cdot f'(x)=\dfrac{f'(x)}{f(x)}.$$
Si la derivada $f'(x)$ se puede calcular de otra forma, entonces se puede obtener $f'(x)=g¡(x)\cdot f(x)$.

%-------------------- Ejercicio 6.2.
\begin{ejem}
    Calcular $f'(x)$ si $f(x)=x^2\cos x\left(1+x^4\right)^{-7}.$\\\\
	Respuesta.-\; Se toma el logaritmo del valor absoluto de $f(x)$ y luego se deriva.\\
	Sea pues
	$$
	\begin{array}{rcl}
	    g(x)&=&\log|f(x)|\\
		&=&\log x^2 + \log|\cos x| + \log\left(1+x^4\right)^{-7}\\
		&=& 2\log x + \log|\cos x| - 7\log\left(1+x^4\right)
	\end{array}
	$$
	Derivando $g(x)$ con respecto a $x$ se obtiene
	$$
	\begin{array}{rcl}
	    g'(x) &=& \dfrac{f'(x)}{f(x)}\\\\
		  &=& \dfrac{2}{x}-\dfrac{\sen x}{\cos x}-\dfrac{28x^3}{(1+x^4)}.
	\end{array}
	$$
	Multiplicando ambos miembros por $f(x)$ tenemos
	$$
	\begin{array}{rcl}
	    f'(x) &=& \dfrac{2x\cos x}{\left(1+x^4\right)^7}-\dfrac{x^2\sen x}{\left(1+x^4\right)^7}-\dfrac{28x^5}{\left(1+x^4\right)^7}
	\end{array}
	$$

\end{ejem}


\section{Ejercicios}

\begin{enumerate}[\bfseries 1.]

    %-------------------- 1.
    \item 
	\begin{enumerate}[a)]

	    %---------- a)
	    \item Hallar todos los valores de $c$ tales que $\log x = c + \displaystyle\int_e^x t^{-1}\; dt$ para todo $x>0$.\\\\
		Respuesta.- Resolvamos de la siguiente manera:
		$$
		\begin{array}{rcl}
		    \log(x) = c+\displaystyle\int_c^x \dfrac{1}{t}\; dt &\Rightarrow& \log(x) = c+\log(x)-\log(c)\\\\
									&\Rightarrow& \log(x)=c+\log(x)-1\\\\ 
									&\Rightarrow& c=1.
		\end{array}
		$$

	    %---------- b)
	    \item Sea $f(x)=\log\left[(1+x)/(1-x)\right]$ si $x>0$. Si $a$ y $b$ son números dados, siendo $ab\neq -1$, hallar todos los $x$ tales que $f(x)=f(a)+f(b)$.\\\\
		Respuesta.- Resolvamos de la siguiente manera:
		$$
		\begin{array}{rcl}
		    f(x)=f(a)+f(b) &\Rightarrow& \log\left(\dfrac{1+x}{1-x}\right) = \log\left(\dfrac{1+a}{1-a}\right)+\log\left(\dfrac{1+b}{1-b}\right)\\\\
				   &\Rightarrow& \log\left(\dfrac{1+x}{1-x}\right) = \log\left[\dfrac{(1+a)(1+b)}{(1-a)(1-b)}\right]\\\\
				   &\Rightarrow& \dfrac{1+x}{1-x} = \dfrac{(1+a)(1+b)}{(1-a)(1-b)}\\\\
				   &\Rightarrow& x = \dfrac{a+b}{1+ab}.
		\end{array}
		$$
		\vspace{.5cm}

	\end{enumerate}

    %-------------------- 2.
    \item En cada caso, hallar un $x$ real que satisfaga la igualdad dada,

	\begin{enumerate}[(a)]

	    %---------- a)
	    \item $\log(1+x)=\log(1-x)$.\\\\
		Respuesta.-\; Calculemos de la siguiente manera:
		$$
		\begin{array}{rcl}
		    \log(1+x)=\log(1-x) &\Rightarrow& 1+x=1-x\\\\
					&\Rightarrow& x=0.
		\end{array}
		$$
		\vspace{.5cm}

	    %---------- b)
	    \item $\log(1+x)=1+\log(1-x)$.\\\\
		Respuesta.-\; Calculemos de la siguiente manera:
		$$
		\begin{array}{rcl}
		    \log(1+x)=1+\log(1-x) &\Rightarrow& \log\left(\dfrac{1+x}{1-x}\right)=1\\\\
					  &\Rightarrow& \dfrac{1+x}{1-x}=e\\\\
					  &\Rightarrow& x=\dfrac{e-1}{e+1}.
		\end{array}
		$$
		\vspace{.5cm}

	    %---------- c)
	    \item $2\log(x) = x\log(2), \; x\neq 2$.\\\\
		Respuesta.-\; Calculemos de la siguiente manera:
		$$
		\begin{array}{rcl}
		    2\log(x) = x\log(2), \; x\neq 2 &\Rightarrow& \log(x^2)=\log(2^x)\\\\
						     &\Rightarrow& x^2=2^x\\\\
						     &\Rightarrow& x=2,\; x=4.
		\end{array}
		$$
		\vspace{.5cm}

	    %---------- d)
	    \item $\log\left(\sqrt{x}+\sqrt{x+1}\right)=1$.\\\\
		Respuesta.-\; Calculemos de la siguiente manera:
		$$
		\begin{array}{rcl}
		    \log\left(\sqrt{x}+\sqrt{x+1}\right)=1 &\Rightarrow& \sqrt{x}+\sqrt{x+1}=e\\\\
							   &\Rightarrow& x-(x+1)=e\left(\sqrt{x}-\sqrt{x+1}\right)\\\\
							   &\Rightarrow& \sqrt{x+1}-\sqrt{x}=\dfrac{1}{e}\\\\
							   &\Rightarrow& \left(\sqrt{x+1}-\sqrt{x}\right)+\left(\sqrt{x+1}-\sqrt{x}\right)=\dfrac{1}{e}+e\\\\
							   &\Rightarrow& 2\sqrt{x+1}=e+\dfrac{1}{e}\\\\
							   &\Rightarrow& \sqrt{x+1}=\dfrac{e}{2}+\dfrac{1}{2e}\\\\
							   &\Rightarrow& x+1=\dfrac{e^2}{4}+\dfrac{1}{4e^2}+\dfrac{1}{2}\\\\
							   &\Rightarrow& x=\dfrac{\left(e^2-1\right)^2}{4x^2}.
		\end{array}
		$$
		\vspace{.5cm}

	\end{enumerate}

    %-------------------- 3.
    \item Sea $f(x)=\frac{\log(x)}{x}$. Describir los intervalos en los que $f$ es creciente, decreciente, convexa y cóncava. Esbozar la gráfica de $f$.\\\\
	Respuesta.-\; Tomemos primero la segunda derivada de $f$:
	$$
	\begin{array}{rcl}
	    f(x) &=& \dfrac{\log(x)}{x}\\\\
	    f'(x) &=& \dfrac{1-\log(x)}{x^2}\\\\
	    f''(x) &=& \dfrac{2\log(x)-3}{x^3}.
	\end{array}
	$$
	Se sigue que,
	$$
	\left\{
	\begin{array}{rcl}
	    f'(x)>0 &\mbox{para} & 0<x<e\\\\
	    f'(x)<0 &\mbox{para} & x>e.
	\end{array}
	\right.
	$$

	Esto significa que $f$ es es creciente en $0<x<e$ y decreciente en $x>e$.\\
	
	Luego se tiene, 
	$$
	\left\{
	\begin{array}{rcl}
	    f''(x)<0 &\mbox{para} & 0<x<\sqrt{e^3}\\\\
	    f''(x)>0 &\mbox{para} & x>\sqrt{e^3}.
	\end{array}
	\right.
	$$
	Lo que significa que $f$ es cóncava en $0<x<\sqrt{e^3}$ y convexa en $x>\sqrt{e^3}$.
	\begin{center}
	    \begin{tikzpicture}
		\begin{axis}[scale=.5,draw opacity =.5,samples=100,smooth, 
		  axis x line=center, 
		  axis y line=center,
		  ylabel = {$f(x)$},
		  xlabel = {$x$},
		  xlabel style={below right},
		  ylabel style={above left},
		  label style={font=\tiny},
		  tick label style={font=\tiny},
		  enlargelimits=upper] 
		  \addplot[black,opacity=1,domain=0.7:5]{ln(x)/x};
		\end{axis}
	    \end{tikzpicture}
	\end{center}
	\vspace{.5cm}

    \end{enumerate}

    En los ejercicios 4 al 14, hallar la derivada $f'(x)$. En cada caso, la función $f$ se supone definida para todo $x$ real para los que la fórmula dada para $f(x)$ tiene sentido.\\

\begin{enumerate}[\bfseries 1.]
\setcounter{enumi}{3}

    %-------------------- 4.
    \item $f(x)=\log\left(1+x^2\right)$.\\\\
	Respuesta.-\; Usando la regla de la cadena tenemos que,
	$$f'(x)= \dfrac{1}{1+x^2}\cdot 2x =\dfrac{2x}{1+x^2}.$$\\

    %-------------------- 5.
    \item $f(x)=\log\left(\sqrt{1+x^2}\right)$.\\\\
	Respuesta.-\; Usando la regla de la cadena tenemos que,
	$$f'(x)= \left(\dfrac{1}{\sqrt{1+x^2}}\right)\left( \dfrac{1}{2\sqrt{1+x^2}}\right)(2x) =\dfrac{x}{1+x^2}.$$\\

    %-------------------- 6.
    \item $f(x)=\log\left(\sqrt{4-x^2}\right)$.\\\\
	Respuesta.-\; Usando la regla de la cadena tenemos que,
	$$f'(x)= \left(\dfrac{1}{\sqrt{4-x^2}}\right)\left( \dfrac{1}{2\sqrt{4-x^2}}\right)(-2x) =\dfrac{x}{x^2-4}.$$\\

    %-------------------- 7.
    \item $f(x)=\log\left(\log(x)\right)$.\\\\
	Respuesta.-\; Usando la regla de la cadena tenemos que,
	$$f'(x)= \dfrac{1}{\log(x)}\cdot \dfrac{1}{x} =\dfrac{1}{x\log(x)}.$$\\

    %-------------------- 8.
    \item $f(x)=\log\left(x^2 \log(x)\right)$.\\\\
	Respuesta.-\; Usando la regla de la cadena y la regla del producto tenemos que,
	$$f'(x)= \left[\dfrac{1}{x^2\log(x)}\right]\left[2x\log(x) + x\right] =\dfrac{1}{x^2\log(x)}\cdot 2x\log(x)+\dfrac{1}{x\log(x)} =\dfrac{2}{x}+\dfrac{1}{x\log(x)}.$$\\

    %-------------------- 9.
    \item $f(x)=\dfrac{1}{4}\log\left(\dfrac{x^2-1}{x^2+1}\right)$.\\\\
	Respuesta.-\; Usando la regla de la cadena y la regla del cociente tenemos que,
	$$
	\begin{array}{rcl}
	    f'(x)&=&\dfrac{1}{4}\left(\dfrac{2x}{x^2-1}-\dfrac{2x}{x^2+1}\right)\\\\
		 &=& \dfrac{1}{4}\left[\dfrac{2x(x^2+1)-2x(x^2-1)}{(x^2-1)(x^2+1)}\right]\\\\
		 &=& \dfrac{1}{4}\left[\dfrac{2x^3+2x-2x^3+2x}{(x^2-1)(x^2+1)}\right]\\\\
		 &=& \dfrac{x}{x^4-1}.
	\end{array}
	$$
	\vspace{.5cm}

    %-------------------- 10.
    \item $f(x)=(x+\sqrt{1+x^2})^n$.\\\\
	Respuesta.-\; Tomando el logaritmo del valor absoluto de la función dada tenemos que,
	$$
	\begin{array}{rcl}
	    g(x) &=& \log|f(x)|\\\\
		 &=& |n|\log\left| (x+\sqrt{1+x^2})\right|.
	\end{array}
	$$
	Derivando $g'(x)$,
	$$
	\begin{array}{rcl}
	    g'(x) &=& \dfrac{f'(x)}{f(x)}\\\\
		  &=& n\left(\dfrac{1}{x+\sqrt{1+x^2}}\right)\left(1+\dfrac{x}{\sqrt{1+x^2}}\right)\\\\
		  &=& \dfrac{n}{\sqrt{1+x^2}}.
	\end{array}
	$$

	Ahora, multiplicando $g'(x)$ por $f(x)$ tenemos que,
	$$
	\begin{array}{rcl}
	    f(x)\cdot \dfrac{f'(x)}{f(x)} &=& \dfrac{n}{\sqrt{1+x^2}}\cdot \left(x+\sqrt{1+x^2}\right)^n\\\\
	    f'(x) &=& \dfrac{n\left(x+\sqrt{1+x^2}\right)^n}{\sqrt{1+x^2}}.
	\end{array}
	$$
	\vspace{.5cm}

    %-------------------- 11.
    \item $f(x)=\sqrt{x+1}-\log(1+\sqrt{x+1})$.\\\\
	Respuesta.-\; Usando la regla de la cadena,
	$$	
	\begin{array}{rcl}
	    f'(x) &=& \dfrac{1}{2\sqrt{x+1}}-\dfrac{1}{1+\sqrt{x+1}}\cdot \dfrac{1}{2\sqrt{x+1}}\\\\
		  &=& \dfrac{1}{2\sqrt{x+1}}\left(1-\dfrac{1}{1+\sqrt{x+1}}\right)\\\\
		  &=& \dfrac{1}{2\left(1+\sqrt{x+1}\right)}.
	\end{array}
	$$
	\vspace{.5cm}


    %-------------------- 12.
    \item $f(x)=x\log\left(x+\sqrt{1+x^2}\right)-\sqrt{1+x^2}$.\\\\
	Respuesta.-\; Usnaod la regla de la cadena y la regla del producto tenemos que,
	$$
	\begin{array}{rcl}
	    f'(x) &=& \log\left(x+\sqrt{1+x^2}\right)+\left(\dfrac{x}{x+\sqrt{1+x^2}}\right)\left(1+\dfrac{x}{\sqrt{1+x^2}}\right)-\dfrac{x}{\sqrt{1+x^2}}\\\\
		  &=& \log\left(x+\sqrt{1+x^2}\right)+\dfrac{x}{\sqrt{1+x^2}}-\dfrac{x}{\sqrt{1+x^2}}\\\\
		  &=& \log\left(x+\sqrt{1+x^2}\right).
	\end{array}
	$$
	\vspace{.5cm}

    %-------------------- 13.
    \item $f(x)=\dfrac{1}{2\sqrt{ab}}\log\left(\dfrac{\sqrt{a}+x\sqrt{b}}{\sqrt{a}-x\sqrt{b}}\right)$.\\\\
	Respuesta.-\; Usando la regla de la cadena y la regla del cociente tenemos que,
	$$
	\begin{array}{rcl}
	    f'(x) &=& \dfrac{1}{2\sqrt{ab}}\left(\dfrac{\sqrt{a}-x\sqrt{b}}{\sqrt{a}+x\sqrt{b}}\right)\left[\dfrac{\sqrt{b}\left(\sqrt{a}-x\sqrt{b}\right)+\sqrt{b}\left(\sqrt{a}+x\sqrt{b}\right)}{\left(\sqrt{a}-x\sqrt{b}\right)^2}\right]\\\\
		  &=& \dfrac{1}{2\sqrt{ab}}\cdot \dfrac{2\sqrt{ab}}{\left(\sqrt{a}+x\sqrt{b}\right)\left(\sqrt{a}-x\sqrt{b}\right)}\\\\
		  &=& \dfrac{1}{a^2-bx^2}.
	\end{array}
	$$
	\vspace{.5cm}

    %-------------------- 14.
    \item $f(x)=x\left[\sen(\log x)-\cos(\log x)\right]$.\\\\
	Respuesta.-\; Usando la regla de la cadena y la regla del producto tenemos que,
	$$
	\begin{array}{rcl}
	    f'(x) &=& \left[\sen(\log x)-\cos(\log x)\right]+x\left[\dfrac{\cos(\log x)}{x}+\dfrac{\sen(\log x)}{x}\right]\\\\
		  &=& 2\sen(\log x).
	\end{array}
	$$
	\vspace{.5cm}

    %-------------------- 15.
    \item $f(x)=\log_xe$.\\\\
	Respuesta.-\; Por definición de logaritmo en base $x$ tenemos que,
	$$f(x)=\log_xe=\dfrac{\log e}{\log x}.$$
	De donde, el logaritmo de $e$ es la constante $1$, por lo tanto
	$$f(x)=\dfrac{1}{\log x}.$$
	Luego, usando la regla del cociente tenemos que,
	$$	
	\begin{array}{rcl}
	    f'(x) &=& \dfrac{-\dfrac{1}{x}}{\left(\log x\right)^2}\\\\
		  &=& -\dfrac{1}{x\left(\log x\right)^2}.
	\end{array}
	$$
	\vspace{.5cm}

\end{enumerate}

En los ejercicios 16 al 26, calcular las integrales.\\

\begin{enumerate}[\bfseries 1.]
\setcounter{enumi}{15}

    %-------------------- 16.
    \item $\displaystyle\int \dfrac{dx}{2+3x}$.\\\\
	Respuesta.-\; 
	$$\int \dfrac{1}{2+3x}\; dx = \dfrac{1}{3}\int \dfrac{3}{2+3x}\; dx=\dfrac{1}{3}\log|2+3x|+C.$$\\

    %-------------------- 17.
    \item $\displaystyle\int \log^2(x)\; dx$.\\\\
	Respuesta.-\; Usando la integración por partes,
	$$
	\begin{array}{rcl}
	    u=\left(\log|x|\right)^2 &\Rightarrow& du=\dfrac{2\log(x)}{x}\; dx\\\\
	    dv=dx &\Rightarrow& v=x.
	\end{array}
	$$
	Notemos por el ejemplo 6.2 de, 
	$$\int \log(x)\; dx = x\log|x|-x+C.$$
	Por lo tanto,
	$$
	\begin{array}{rcl}
	    \displaystyle\int (\log x)^2\; dx &=& \displaystyle\int u\; dv\\\\
					      &=& uv - \displaystyle\int v\; du\\\\
					      &=& x(\log|x|)^2-\displaystyle\int 2\log x \; dx\\\\
					      &=& x(\log|x|)^2 - 2(x\log|x|-x)+C\\\\
					      &=& x(\log|x|)^2-2x\log|x|+2x+c.
	\end{array}
	$$
	\vspace{.5cm}

    %-------------------- 18.
    \item $\displaystyle\int x\log(x)\; dx$.\\\\
	Respuesta.-\; Usando la integración por partes,
	$$
	\begin{array}{rcl}
	    u=\log|x| &\Rightarrow& du=\dfrac{1}{x}\; dx\\\\
	    dv=x\;dx &\Rightarrow& v=\dfrac{x^2}{2}.
	\end{array}
	$$
	Por lo tanto,
	$$
	\begin{array}{rcl}
	    \displaystyle\int x\log x \; dx &=& \displaystyle\int u\; dv\\\\
					    &=& uv-\displaystyle v\; du\\\\
					    &=& \dfrac{1}{2}x^2\log|x|-\dfrac{1}{2}\displaystyle\int x\; dx\\\\
					    &=& \dfrac{1}{2}x^2\log|x|-\dfrac{1}{4}x^2+C.
	\end{array}
	$$
	\vspace{.5cm}

    %-------------------- 19.
    \item $\displaystyle\int x\log^2(x)\; dx$.\\\\
	Respuesta.-\; Usando la integración por partes,
	$$
	\begin{array}{rcl}
	    u=\log^2(x) &\Rightarrow& du=\dfrac{2\log(x)}{x}\; dx\\\\
	    dv=x\; dx &\Rightarrow& v=\dfrac{x^2}{2}.
	\end{array}
	$$
	Entonces,
	$$
	\begin{array}{rcl}
	    \displaystyle\int x\log^2(x) &=& \displaystyle\int u \; dv\\\\
					 &=& uv-\displaystyle\int v\; du\\\\
					 &=& \dfrac{1}{2}x^2\log^2(x)-\displaystyle\int x\log(x)\; dx.
	\end{array}
	$$
	Por el ejercicio anterior,
	$$\displaystyle\int x\log(x)\; dx = \dfrac{1}{x^2}\log|x|-\dfrac{1}{4}x^2+C.$$
	Por lo tanto,
	$$
	\begin{array}{rcl}
	    \displaystyle\int x\log^2(x)\; dx &=&  \dfrac{1}{2}x^2\log^2(x)-\displaystyle\int x\log(x)\; dx\\\\
					      &=& \dfrac{1}{2}x^2\log^2(x)-\left(\dfrac{1}{x^2}\log|x|-\dfrac{1}{4}x^2\right)+C.
	\end{array}
	$$
	\vspace{.5cm}


    %-------------------- 20.
    \item $\displaystyle\int_0^{e^3-1} \dfrac{dt}{1+t}$.\\\\
	Respuesta.-\; Usando la integración por partes,
	$$
	\begin{array}{rcl}
	    u=1+t &\Rightarrow& du=dt.\\\\
	\end{array}
	$$

	Entonces,

	$$
	\begin{array}{rcl}
	    \displaystyle\int_0^{e^3-1}\dfrac{1}{1+t}\; dt&=&\displaystyle\int_0^{e^3-1}\dfrac{1}{u}\; dt.\\\\
	\end{array}
	$$

	Ahora, ajustaremos los límites integrales calculando el límite de la función en $t=0$ y $t=e^3-1$.

	$$
	\begin{array}{rcl}
	    u &=& 1+t\\
	      &=& 1+0\\
	      &=&1.
	\end{array}
	$$

	\begin{center}
	y
	\end{center}

	$$
	\begin{array}{rcl}
	    u&=&1+t\\
	    &=&1+e^3-1\\
	    &=&e^3.
	\end{array}
	$$

	Por lo tanto,

	$$
	\begin{array}{rcl}
	    \displaystyle\int_0^{e^3-1}\dfrac{1}{1+t}\; dt&=&\displaystyle\int_1^{e^3}\dfrac{1}{u}\; du\\\\
							  &=&\left[\log(|u|)\right]\bigg|_0^{e^3}\\\\
							   &=&\left[\log(|1+t|)\right]\bigg|_0^{e^3}\\\\
							   &=&\log\left(|e^3|\right)-\log\left(|1|\right)\\\\
							   &=&3\log\left(|e|\right)\\\\
							   &=&3.
	\end{array}
	$$
	\vspace{.5cm}


    %-------------------- 21.
    \item $\displaystyle\int \cot(x)\; dx$.\\\\
	Respuesta.-\; Reescribiendo la integral se tiene,
	$$\int \cot x \; dx = \int \dfrac{\cos x}{\sen x}\; dx.$$
	Luego, usando la sustitución $u=\sen x$ y $du=\cos x\; dx$. Entonces,
	$$
	\begin{array}{rcl}
	    \displaystyle\int \cot x \; dx &=& \displaystyle\int \dfrac{\cos x}{\sen x}\; dx\\\\
					   &=& \displaystyle\int \dfrac{1}{u}\; du\\\\
					   &=& \log|u|+C\\\\
					   &=& \log|\sen x|+C.
	\end{array}
	$$
	\vspace{.5cm}

    %-------------------- 22.
    \item $\displaystyle\int x^n \log(ax)\; dx$.\\\\
	Respuesta.-\; 

    %-------------------- 23.
    \item $\displaystyle\int x^2\log^2(x)\; dx$.\\\\
	Respuesta.-\; 

    %-------------------- 24.
    \item $\displaystyle\int \dfrac{dx}{x\log(x)}$.\\\\
	Respuesta.-\;

    %-------------------- 25.
    \item $\displaystyle\int_0^{1-e^{-2}} \dfrac{\log(1-t)}{1-t}$.\\\\
	Respuesta.-\;

    %-------------------- 26.
    \item $\displaystyle\int \dfrac{\log|x|}{x\sqrt{1+\log|x|}}$.\\\\
	Respuesta.-\;


\end{enumerate}

