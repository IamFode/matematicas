\chapter{Función Logaritmo, función exponencial y funciones trigonométricas inversas}

\setcounter{section}{1}
\section{definición de logaritmo natural como integral}

Una ecuación tal como 
$$f(xy)=f(x)+f(y)$$ 
que expresa una relación entre los valores de una función en dos o más puntos se denomina \textbf{ecuación funcional}. Una solución de $f(xy)=f(x)+f(y)$ es la función que es cero en todo el eje real; y demás es la única solución que está definida para todos los números reales. En efecto: sea $f$ una función que satisfaga la ecuación funcional, si $0$ pertenece al dominio de $f$ se puede poner $y=0$ obteniéndose $f(0)=f(x)+f(0)$ lo que implica $f(x)=0$ para cada $x$ en el dominio de $f$. Dicho de otra forma, si $0$ pertenece al dominio de $f$, $f$ ha de ser idénticamente nula. Por tanto, una solución no idénticamente nula no puede estar definida en $0$.\\

Si $f$ es una solución de $f(xy)=f(x)+f(y)$ y el dominio de $f$ contiene el punto $1$, se puede poner $x=y=1$ y se obtiene $f(1)=2f(1)$ de donde
$$f(1)=0.$$
Si ambos $1$ y $-1$ pertenecen al dominio de $f$ se puede tomar $x=-1$ e $y=-1$ de donde se deduce $f(1)=2f(-1)$ es decir $f(-1)=0$. Si ahora, $x$, $-x$, $1$, $-1$ pertenecen al dominio de $f$ se puede poner $y=-1$ obteniéndose $f(-x)=f(-1)+f(x)$, y puesto que $f(-1)=0$ se tiene
$$f(-x)=f(x).$$
Es decir, toda solución de $f(xy)=f(x)+f(y)$ es necesariamente una función par.\\

Supóngase ahora,


