\chapter{Relación entre integración y derivación}

\section{La derivada de una integral indefinida. Primer teorema fundamental del cálculo}

\begin{teo}[Primer Teorema fundamental del cálculo]
    Sea $f$ una función que es integrable en $[a,x]$ para cada $x$ en $[a,b]$. Sea $c$ tal que $a\leq c \leq b$ y definamos una nueva función $A$ del siguiente modo:
    $$A(x)=\int_c^x f(t)\;dt \quad \mbox{si}\quad a\leq x \leq b$$
    Existe entonces la derivada $A'(x)$ en cada punto $x$ del intervalo abierto $(a,b)$ donde $f$ es continua, y para tal $x$ tenemos
    \begin{equation}
	A'(x)=f(x)
    \end{equation}
    Interpretación geométrica: La figura 5.1 (Apostol, capítulo 5) muestra la gráfica de una función $f$ en un intervalo $[a,b]$. En la figura, $h$ es positivo y
    $$\int_x^{x+h}f(t)\; dt=\int_0^{x+h}f(t)\; dt -\int_0^x f(t)\; dt= A(x+h)-A(x).$$
    El ejemplo es el de una función continua en todo el intervalo $[x,x+h]$. Por consiguiente, por el teorema del valor medio para integrales, tenemos
    $$A(x+h)-A(x)=hf(z),\quad \mbox{donde}\; x\leq z \leq x+h.$$
    Luego, resulta
    \begin{equation}
	\dfrac{A(x+h)-A(x)}{h}=f(z).
    \end{equation}
    Puesto que $x\leq z \leq x+h,$ encontramos que $f(z)\to f(x)$ cuando $h\to 0$ con valores positivos. Si $h\to 0$ con valores negativos, se razona en forma parecida. Por consiguiente, $A'(x)$ existe y es igual a $f(x)$.\\
	Demostración.-\; Sea $x$ un punto en el que $f$ es continua y supuesta $x$ fija, se forma el cociente:
	$$\dfrac{A(x+h)-A(x)}{h}.$$
	Para demostrar el teorema se ha de probar que este cociente tiende a $f(x)$ cuando $h\to 0$. El numerador es:
	$$A(x+h)-A(x)=\int_0^{x+h}f(t)\; dt-\int_c^x f(t)\; dt = \int_x^{x+h}f(t)\; dt.$$
	Si en la última integral se escribe $f(t)=f(x)-[f(t)-f(x)]$ resulta:
	$$\begin{array}{rcl}
	    A(x+h)-A(x) & = & \displaystyle \int_x^{x+h}f(x)\; dt + \int_x^{x+h}\left[f(t)-f(x)\right]\; dt\\\\
			& = & \displaystyle hf(x) + \int_x^{x+h} \left[f(t)-f(x)\right]\; dt,
	\end{array}$$
	de donde
	\begin{equation}
	    (5.3) \qquad \qquad \dfrac{A(x+h)-A(x)}{h}=f(x)+\dfrac{1}{h}\int_{x}^{x+h}\left[f(t)-f(x)\right]\; dt.
	\end{equation}
	Por tanto, para completar la demostración de (5.1) es necesario demostrar que 
	$$\lim_{h\to 0}\dfrac{1}{h}\int_x^{x+h}\left[f(t)-f(x)\right]\; dt = 0.$$
	En esta parte de la demostración es donde se hace uso de la continuidad de $f$ en $x$.\\
	Si se designa por $G(h)$ el último término del segundo miembro de (5.3), se trata de demostrar que $G(h)\to 0$ cuando $h\to 0$. Aplicando la definición de límite, se ha de probar que para cada $\epsilon>0$ existe un $\delta>0$ tal que
	\begin{equation}
	    G(h)<\epsilon \mbox{ siempre que } 0<h<\delta.
	\end{equation}
	En virtud de la continuidad de $f$ en $x$, dado un $\epsilon$ existe un número positivo $\delta$ tal que:
	\begin{equation}
	|f(t)-f(x)|<\dfrac{1}{2}\epsilon.
	\end{equation}
	siempre que 
	\begin{equation}
	x-\delta<t<x+\delta.
	\end{equation}
	Si se elige $h$ de manera que $0<h<\delta$, entonces cada $t$ en el intervalo $[x,x+h]$ satisface (5.6) y por tanto (5.5) se verifica para cada $t$ de este intervalo. Aplicando la propiedad $|\int_x^{x+h}|\leq \int_x^{x+h}|g(t)|\; dt$, cuando $g(t)=g(t)-f(x)$, de la desigualdad en (5.5) se pasa a la relación:
	$$\left|\int_x^{x+h}\left[f(t)-f(x)\right]\; dt\right|\leq \int_x^{x+h}|f(t)-f(x)|\; dt \leq \int_x^{x+h}\dfrac{1}{2} \epsilon \; dt = \dfrac{1}{2}h\epsilon < h\epsilon.$$
	Dividiendo por $h$ se ve que (5.4) se verifica para $0<h<\delta$. Si $h<0$, un razonamiento análogo demuestra que (5.4) se verifica siempre que $0<|h|<\delta$, lo que completa la demostración.
\end{teo}

%%%%%%%%%%%%%%%%%%%%%%%%%% 5.2. Teorema de la derivada nula %%%%%%%%%%%%%%%%%%%%%%%%%%%%%%%%
\section{Teorema de la derivada nula}

Si una función $f$ es constante en un intervalo $(a, b)$, su derivada es nula en todo el intervalo $(a, b)$. Ya hemos demostrado este hecho como una consecuencia inmediata de la definición de derivada. También se demostró, como parte c) del teorema 4.7, el recíproco de esa afirmación que aquí se presenta como teorema independiente.

\begin{teo}[Teorema de la derivada nula]
    Si $f'(x)=0$ para cada $x$ en un intervalo abierto $I$, es $f$ constante en $I$.
\end{teo}
\vspace{0.5cm}

Este teorema, cuando se utiliza combinando con el primer teorema fundamental del cálculo, nos conduce al segundo teorema fundamental.

\section{Funciones primitivas y segundo teorema fundamental del cálculo}
\begin{def.}[Definición de función primitiva]
    Una función $P$ se llama primitiva (o antiderivada) de una función $f$ en un intervalo abierto $I$ si la derivada de $P$ es $f$, esto es, si $P'(x)=f(x)$ para todo $x$ en $I.$
\end{def.}

Decimos una primitiva y no la primitiva, porque si $P$ es una primitiva de $f$ también lo es $P+k$ para cualquier constante $k$. Recíprocamente, dos primitivas cualesquiera $P$ y $Q$ de la misma función $f$ sólo pueden diferir en una constante por que su diferencia $P-Q$ tiene la derivada
$$P'(x)-Q'(x)=f(x)-f(x)=0.$$
para toda $x$ en $I$ y por tanto, según el teorema $5.2$

%-------------------- Teorema 5.3. Segundo teorema fundamental del cálculo --------------------
\begin{teo}[Segundo teorema fundamental del cálculo]
    Supongamos $f$ continua en un intervalo abierto $I$, y sea $P$ una primitiva cualquiera de $f$ en $I$. Entonces, para cada $c$ y cada $x$ en $I$, tenemos 
    $$P(x)=P(c)+\int_c^x f(t)\; dt.$$
	Demostración.-\; Sea $A(x)=\int_c^x f(t)\; dt$. Puesto que $f$ es continua en cada $x$ de $I$, el primer teorema fundamental nos dice que $A'(x)=f(x)$ para todo $x$ de $I$. Es decir, $A$ es primitiva de $f$ en $I$. Luego, ya que dos primitivas de $f$ pueden diferir tan sólo en una constante, debe ser $A(x)-P(x)=k$ para una cierta constante $k$. Cuando $x=c$ esta fórmula implica $-P(c)=k,$ ya que $A(c)=0.$ Por consiguiente, $A(x)-P(x)=-P(c)$, de lo que obtenemos $P(x)=P(c)+\int_c^x f(t)\; dt.$ es constante en $I$.
\end{teo}

El teorema 5.3 nos indica cómo encontrar una primitiva $P$ de una función continua $f$. Integrando $f$ desde un punto fijo $c$ a un punto arbitrario $x$ y sumando la constante $P(c)$ obtenemos $P(x)$. Pero la importancia real del teorema radica en que poniendo $P(x)=P(c)+\int_c^x f(t)\; dt$ en la forma
$$\int_c^x f(t)\; dt = P(x)-P(c).$$

Se ve que podemos calcular el valor de una integral mediante una simple substracción si conocemos una primitiva $P$.\\\\

Como consecuencia del segundo teorema fundamental, se pueden deducir las siguientes fórmulas de integración.

\begin{ejem}[Integración de potencias recionales]
    La fórmula de integración
    $$\int_a^b x^n\; dx = \dfrac{b^{n+1}-a^{n+1}}{n+1}\qquad (n=0,1,2,\ldots)$$
    se demostró directamente en la Sección 1.23 (Spivak) a partir de la definición de integral. Aplicando el segundo teorema fundamental, puede hallarse de nuevo este resultado y además generalizarlo para exponentes racionales. En primer lugar se observa que la función $P$ definida por
    $$P(x)=\dfrac{x^{n+1}}{n+1}$$
    tiene como derivada $P'(x)=x^n$ para cada $n$ entero no negativo. De esta igualdad válida para todo número real $x$, aplicando $\int_c^x f(t)\; dt = P(x)-P(c)$ se tiene
    $$\int_a^b x^n\; dx  = P(b)-P(a)=\dfrac{b^{n+1}-a^{n+1}}{n+1}$$
    para cualquier intervalo $[a,b]$. Esta fórmula, demostrada para todo entero $n\geq 0$ conserva su validez para todo entero negativo excepto $n=-1$, que se excluye puesto que el denominador aparece $n+1$. Para demostrar $\int_a^b x^n\; dx = \frac{b^{n+1}-a^{n+1}}{n+1}\; (n=0,1,2,\ldots)$ para $n$ negativo, basta probar que $P(x)=\frac{x^{n+1}}{n+1}$ implica $P'(x)=x^n$ cuando $n$ es negativo $x\neq -1$, lo cual es fácil de verificar derivando $P$ como función racional. Hay que tener en cuenta que si $n$ es negativo se deben excluir aquellos intervalos $[a,b]$ que contienen el punto $x=0.$
\end{ejem}

El resultado del ejemplo 3 de la Sección 4.5, permite extender $\int_a^b x^n\; dx = \frac{b^{n+1}-a^{n+1}}{n+1}\; (n=0,1,2,\ldots)$ a todos los exponentes racionales (excepto $-1$) siempre que el integrando esté definido en todos los puntos del intervalo $[a, b]$ en consideración.

\begin{ejem}[Integración de seno y coseno]
    Puesto que la derivada del seno es el coseno y la del coseno menos el seno, el segundo teorema fundamental da las fórmulas siguientes:
    $$\int_a^b \cos x \; dx = \sen x\bigg|_a^b = \sen b -\sen a,$$
    $$\int_a^b \sen x \; dx = -\cos x\bigg|_a^b = -\cos b +\cos a.$$
\end{ejem}
    Estas fórmulas se conocían ya, pues se demostraron en el capítulo 2 a partir de la definición de integral. \\
    Se obtienen otras fórmulas de integración a partir de los ejemplos 1 y 2 tomando sumas finitas de términos de la forma $Ax^n$, $B \sen x$, $C cos x$, donde $A$, $B$, $C$ son constantes.


\section{Propiedades de una función deducida de propiedades de su derivada}
Si una función $f$ tiene derivada continua $f'$ en un intervalo abierto $I$, el segundo teorema fundamental afirma que 
$$f(x)=f(c)+\int_c^x f'(t)\; dt$$
cualesquiera que sean $x$ y $c$ en $I$. \\

\begin{prop}
Supóngase que $f'$ es continua y no negativa en $I$. Si $x>c$, entonces $\int_c^x f'(t)\; dt \geq 0$, y por tanto $f(x)\geq f(c)$. Es decir, si la derivada es continua y no negativa en $I$, la función es creciente en $I$.
\end{prop}

En el teorema 2.9 se demostró que la integral indefinida de una función creciente es convexa. Por consiguiente, si $f'$ es continua y creciente en $I$, la igualdad $f(x)=f(c)+\int_c^x f'(t)\; dt$ demuestra que $f$ es convexa en $I$. Análogamente, $f$ es cóncava en los intervalos en los que $f'$ es continua y decreciente.


\section{Ejercicios}

En cada uno de los Ejercicios del 1 al 10, encontrar una primitiva de $f$; es decir, encontrar una función $P$ tal que $P'(x)=f(x)$ y aplicar el segundo teorema fundamental para calcular $\int_a^b f(x)\; dx.$\\

\begin{enumerate}[\bfseries 1.]

    %-------------------- 1.
    \item $f(x)=5x^3$.\\\\
	Respuesta.-\; Por el ejemplo 5.1 (Apostol) se define a $P$ como,
	$$P(x)=\dfrac{x^{n+1}}{n+1}.$$
	Por lo que la función primitiva de $5x^3$, esta dada por:
	$$P(x)=5\dfrac{x^{3+1}}{3+1}=5\dfrac{x^4}{4}.$$
	Esta función es primitiva de $f$ ya que:
	$$P'(x)=5\dfrac{x^4}{4}=5x^3.$$
	Luego, aplicando el segundo teorema fundamental para calcular $\int_a^b f(x)\; dx$, tenemos
	$$\begin{array}{rcl}
	    \displaystyle\int_a^b 5x^3\; dx &=&P(b)-P(a)\\\\
					    &=&\dfrac{5}{4}b^4-\dfrac{5}{3}a^4.\\\\
					    &=&\dfrac{5}{4}(b^4-a^4).
	\end{array}$$
	\vspace{.5cm}

    %-------------------- 2.
    \item $f(x)=4x^4-12x$.\\\\
	Respuesta.-\; Por el ejemplo 5.1 (Apostol) se define a $P$ como,
	$$P(x)=\dfrac{x^{n+1}}{n+1}.$$
	Por lo que la función primitiva de $4x^4-12x$, esta dada por:
	$$\begin{array}{rcl}
	    P(x)&=&4\dfrac{x^{4+1}}{4+1}-12\dfrac{x^{1+1}}{1+1}\\\\
		&=&\dfrac{4}{5}x^5-6x^2.
	\end{array}$$
	Esta función es primitiva de $f$ ya que:
	$$\begin{array}{rcl}
	    f(x)&=&\dfrac{4}{5}x^5-6x^2\\\\
	    &=&4x^4-12x.
	\end{array}$$

    	Luego, aplicando el segundo teorema fundamental para calcular $\int_a^b f(x)\; dx$, tenemos
	$$\begin{array}{rcl}
	    \displaystyle\int_a^b 4x^4-12x\; dx &=&P(b)-P(a)\\\\
					    &=&\dfrac{4}{5}b^5-6b^2-\left(\dfrac{4}{5}a^5-6a^2\right)\\\\
					    &=&\dfrac{4}{5}(b^5-a^5)-6(b^2-a^2).
	\end{array}$$

    %-------------------- 3.
    \item $f(x)=(x+1)(x^3-2).$\\\\
	Respuesta.-\; Por el ejemplo 5.1 (Apostol) se define a $P$ como,
	$$P(x)=\dfrac{x^{n+1}}{n+1}.$$
	Por lo que la función primitiva de $(x+1)(x^3-2)=x^4+x^3-2x-2$, esta dada por:
	$$\begin{array}{rcl}
	    P(x)&=&\dfrac{x^{4+1}}{4+1}+\dfrac{x^{3+1}}{3+1}-2\dfrac{x^{1+1}}{1+1}-2\dfrac{x^{0+1}}{0+1}\\\\
		&=&\dfrac{x^5}{5}+\dfrac{x^4}{4}-2x^2-2x.
	\end{array}$$
	Esta función es primitiva de $f$ ya que,
	$$\begin{array}{rcl}
	    P'(x)&=&\dfrac{x^5}{5}+\dfrac{x^4}{4}-2x^2-2x\\\\
		&=&x^4+x^3-2x-2\\\\
		&=&(x+1)(x^3-2).
	\end{array}$$
	Luego, aplicando el segundo teorema fundamental para calcular $\int_a^b f(x)\; dx$, tenemos
	$$\begin{array}{rcl}
	    \displaystyle\int_a^b (x+1)(x^3-2)\; dx &=&P(b)-P(a)\\\\
					    &=&\dfrac{b^5}{5}+\dfrac{b^4}{4}-2b^2-2b-\left(\dfrac{a^5}{5}+\dfrac{a^4}{4}-2a^2-2a\right)\\\\
					    &=&\dfrac{1}{5}\left(b^5-a^5\right) + \dfrac{1}{4}\left(b^4-a^4\right)-\left(b^2-a^2\right)-2(b-a).
	\end{array}$$
	\vspace{.5cm}

    %-------------------- 4.
    \item $f(x)=\dfrac{x^4+x-3}{x^3}, \qquad x\neq 0$.\\\\
	Respuesta.-\; Podemos reescribir la función $f$ como:
	$$f(x)=\dfrac{x^4+x-3}{x^3} = \left(x^4+x-3\right)x^{-3} = x+x^{-2}-3x^{-3}.$$
	Por el ejemplo 5.1 (Apostol) se define a $P$ como,
	$$P(x)=\dfrac{x^{n+1}}{n+1}.$$
	Por lo que la función primitiva de $x+x^{-2}-3x^{-3}$, esta dada por:
	$$\begin{array}{rcl}
	    P(x)&=&\dfrac{x^{1+1}}{1+1}+\dfrac{x^{-2+1}}{-2+1}-\dfrac{3x^{-3+1}}{-3+1}\\\\
		&=&\dfrac{x^2}{2}-x^{-1}+\dfrac{3x^{-2}}{2}.
	\end{array}$$
	Esta función es primitiva de $f$ ya que,
	$$P'(x)=x+x^{-2}-3x^{-3}=\dfrac{x^4+x-3}{x^3}=f(x).$$
	Luego, aplicando el segundo teorema fundamental para calcular $\int_a^b f(x)\; dx$, tenemos
	$$\begin{array}{rcl}
	    \displaystyle\int_a^b f(x)\; dx &=& P(b)-P(a)\\\\
					    &=& \dfrac{b^2}{2}-b^{-1}+\dfrac{3b^{-2}}{2}-\left(\dfrac{a^2}{2}-a^{-1}+\dfrac{3a^{-2}}{2}\right)\\\\
					    &=& \dfrac{1}{2}\left(b^2-a^2\right)-\left(b^{-1}-a^{-1}\right)+\dfrac{3}{2}\left(b^{-2}-a^{-2}\right)\\\\
					    &=& \dfrac{1}{2}\left(b^2-a^2\right)-\left(\dfrac{1}{a}-\dfrac{1}{b}\right)+\dfrac{3}{2}\left(\dfrac{1}{a^2}-\dfrac{1}{b^2}\right).
	\end{array}$$
	\vspace{.5cm}

    %-------------------- 5.
    \item $f(x)=\left(1+\sqrt{x}\right)^2,\quad x>0$.\\\\
	Respuesta.-\; Dado que $x>0$. Reescribimos la función $f$ como:
	$$f(x)=1+2\sqrt{x}+|x| \quad \Rightarrow \quad f(x)=1+2x^{\frac{1}{2}}+x$$
	Como $x>0$, entonces $|x|=x$. Luego, por el ejemplo 5.1 (Apostol) tenemos,
	$$\begin{array}{rcl}
	    P(x)&=&1\dfrac{x^{0+1}}{0+1}+\dfrac{2x^{\frac{1}{2}+1}}{\frac{1}{2}+1}+\dfrac{x^{1+1}}{1+1}\\\\
		&=&x+\dfrac{4x^{\frac{3}{2}}}{3}+\dfrac{x^2}{2}.
	\end{array}$$
	Esta función es primitiva de $f$ ya que,
	$$P'(x)=1+\dfrac{3}{2}\dfrac{4x^{\frac{3}{2}-1}}{3}+\dfrac{2x^2}{2}=1+2x^{\frac{1}{2}}+x=\left(1+\sqrt{x}\right)^2=f(x).$$
	Luego, aplicando el segundo teorema fundamental para calcular $\int_a^b f(x)\; dx$, tenemos
	$$\begin{array}{rcl}
	    \displaystyle\int_a^b f(x)\; dx &=& P(b)-P(a)\\\\
					    &=& x+\dfrac{4b^{\frac{3}{2}}}{3}+\dfrac{b^2}{2}-\left(x+\dfrac{4a^{\frac{3}{2}}}{3}+\dfrac{a^2}{2}\right)\\\\
					    &=& \dfrac{4}{3}\left(b^{\frac{3}{2}}-a^{\frac{3}{2}}\right)+\dfrac{1}{2}\left(b^2-a^2\right)+(b-a).
	\end{array}$$
	\vspace{.5cm}

    %-------------------- 6.
    \item $f(x)=\sqrt{2x}+\sqrt{\dfrac{1}{2}x},\qquad x>0$.\\\\
	Respuesta.-\; Por el ejemplo 5.1 (Apostol) tenemos, 
	$$\begin{array}{rcl}
	    P(x)&=&\dfrac{\sqrt{2}x^{\frac{1}{2}+1}}{\frac{1}{2}+1}+\dfrac{\sqrt{\frac{1}{2}}x^{\frac{1}{2}+1}}{\frac{1}{2}+1}\\\\
	    		&=&\dfrac{2\sqrt{2}x^{\frac{3}{2}}}{3}+\dfrac{2\sqrt{\frac{1}{2}}x^{\frac{3}{2}}}{3}\\\\
	\end{array}$$
	Esta función es primitiva de $f$ ya que,
	$$P'(x)=\dfrac{3}{2}\dfrac{2\sqrt{2}x^{\frac{3}{2}-1}}{3}+\dfrac{3}{2}\dfrac{2\sqrt{\frac{1}{2}}x^{\frac{3}{2}-1}}{3}=\sqrt{2x}+\sqrt{\dfrac{1}{2}x}=f(x).$$
	Luego, aplicando el segundo teorema fundamental para calcular $\int_a^b f(x)\; dx$, tenemos
	$$\begin{array}{rcl}
	    \displaystyle\int_a^b f(x)\; dx &=& P(b)-P(a)\\\\
					    &=& \dfrac{2\sqrt{2}b^{\frac{3}{2}}}{3}+\dfrac{2\sqrt{\frac{1}{2}}b^{\frac{3}{2}}}{3}-\left(\dfrac{2\sqrt{2}a^{\frac{3}{2}}}{3}+\dfrac{2\sqrt{\frac{1}{2}}a^{\frac{3}{2}}}{3}\right) \\\\
					    &=& \left(\dfrac{2\sqrt{2}}{3}+\dfrac{2\sqrt{\frac{1}{2}}}{3}\right)b^{\frac{3}{2}}-\left(\dfrac{2\sqrt{2}}{3}+\dfrac{2\sqrt{\frac{1}{2}}}{3}\right)a^{\frac{3}{2}}\\\\
					    &=& \dfrac{2}{\sqrt{2}}\left(b^{\frac{3}{2}}-a^{\frac{3}{2}}\right).

	\end{array}$$
	\vspace{.5cm}

    %-------------------- 7.
    \item $f(x)=\dfrac{2x^2-6x+7}{2\sqrt{x}}\qquad x>0$.\\\\
	Respuesta.-\; Reescribimos la función $f$ como:
	$$\begin{array}{rcl}
	    f(x)&=&\dfrac{x^2}{x^{\frac{1}{2}}}-\dfrac{3x}{x^{\frac{1}{2}}}+\dfrac{7}{2x^{\frac{1}{2}}}\\\\
		&=&x^2x^{-\frac{1}{2}}-3xx^{-\frac{1}{2}}+\dfrac{7}{2}x^{-\frac{1}{2}}\\\\
		&=&x^{\frac{3}{2}}-3x^{\frac{1}{2}}+\dfrac{7}{2}x^{-\frac{1}{2}}.\\\\
	\end{array}$$
	Por el ejemplo 5.1 (Apostol) tenemos,
	$$\begin{array}{rcl}
	    P(x)&=&\dfrac{x^{\frac{3}{2}+1}}{\frac{3}{2}+1}-\dfrac{3x^{\frac{1}{2}+1}}{\frac{1}{2}+1}+\dfrac{7}{2}\dfrac{x^{-\frac{1}{2}+1}}{-\frac{1}{2}+1}\\\\
		&=&\dfrac{2x^{\frac{5}{2}}}{5}-2x^{\frac{3}{2}}+7x^{\frac{1}{2}}\\\\
	\end{array}$$
    	Esta función es primitiva de $f$ ya que,
	$$\begin{array}{rcl}
	    P'(x)&=&\dfrac{5}{2}\cdot\dfrac{2x^{\frac{5}{2}-1}}{5}-\dfrac{3}{2}\cdot{2x^{\frac{3}{2}-1}}+\dfrac{1}{2}\cdot7x^{\frac{1}{2}-1}\\\\
		 &=&\dfrac{2x^{\frac{3}{2}}}{2}-\dfrac{6x^{\frac{1}{2}}}{2}+\dfrac{7}{2}x^{-\frac{1}{2}}.\\\\
		 &=&\dfrac{2x^2-6x+7}{2\sqrt{7}}=f(x).
	\end{array}$$
	Luego, aplicando el segundo teorema fundamental para calcular $\int_a^b f(x)\; dx$, tenemos
	$$\begin{array}{rcl}
	    \displaystyle\int_a^b f(x)\; dx &=& P(b)-P(a)\\\\
					    &=& \dfrac{2b^{\frac{5}{2}}}{5}-2b^{\frac{3}{2}}+7b^{\frac{1}{2}}-\left(\dfrac{2a^{\frac{5}{2}}}{5}-2a^{\frac{3}{2}}+7a^{\frac{1}{2}}\right)\\\\
					    &=&\dfrac{2}{5}\left(b^{\frac{5}{2}}-a^{\frac{5}{2}}\right)-2\left(b^{\frac{3}{2}}-a^{\frac{3}{2}}\right)+7\left(b^{\frac{1}{2}}-a^{\frac{1}{2}}\right).
	\end{array}$$
	\vspace{0.5cm}

    %-------------------- 8.
    \item $f(x)=2x^{\frac{1}{3}}-x^{-\frac{1}{3}}, \qquad x>0$.\\\\
	Respuesta.-\; Por el ejemplo 5.1 (Apostol) tenemos,
	$$\begin{array}{rcl}
	    P(x)&=&2\dfrac{x^{\frac{1}{3}+1}}{\frac{1}{3}+1}-\dfrac{x^{-\frac{1}{3}+1}}{-\frac{1}{3}+1}\\\\
		&=&\dfrac{3x^{\frac{4}{3}}}{2}-\dfrac{3x^{\frac{2}{3}}}{2}\\\\
	\end{array}$$
	Esta función es primitiva de $f$ ya que,
	$$\begin{array}{rcl}
	    P'(x)&=&\dfrac{4}{3}\cdot \dfrac{3x^{\frac{4}{3}-1}}{2}-\dfrac{2}{3}\cdot\dfrac{3x^{\frac{2}{3}-1}}{2}\\\\
		 &=&2x^{\frac{1}{3}}-x^{-\frac{1}{3}}=f(x).
	\end{array}$$
	Luego, aplicando el segundo teorema fundamental para calcular $\int_a^b f(x)\; dx$, tenemos
	$$\begin{array}{rcl}
	    \displaystyle\int_a^b f(x)\; dx &=& P(b)-P(a)\\\\
					    &=& \dfrac{3b^{\frac{4}{3}}}{2}-\dfrac{3b^{\frac{2}{3}}}{2}-\left(\dfrac{3a^{\frac{4}{3}}}{2}-\dfrac{3a^{\frac{2}{3}}}{2}\right)\\\\
					    &=&\dfrac{3}{2}\left(b^{\frac{4}{3}}-a^{\frac{4}{3}}\right)-\dfrac{3}{2}\left(b^{\frac{2}{3}}-a^{\frac{2}{3}}\right).
	\end{array}$$
	\vspace{0.5cm}

    %-------------------- 9.
    \item $f(x)=3\sen x + 2x^5$.\\\\
	Respuesta.-\; Claramente podemos ver que,
	$$P(x)=-3\cos x + \dfrac{1}{3}x^6,$$
	es una función primitiva ya que,
	$$P'(x)=3\sen x + 2x^5 = f(x).$$
	Luego, aplicando el segundo teorema fundamental para calcular $\int_a^b f(x)\; dx$, tenemos
	$$\begin{array}{rcl}
	    \displaystyle\int_a^b f(x)\; dx &=& P(b)-P(a)\\\\
					    &=& -3\cos b + \dfrac{1}{3}b^6 -\left(-3\cos a + \dfrac{1}{3}a^6\right)\\\\
					    &=&-3(\cos b - \cos a) + \dfrac{1}{3}\left(b^6-a^6\right).
	\end{array}$$
	\vspace{0.5cm}

    %-------------------- 10.
    \item $f(x)=x^{\frac{4}{3}}-5\cos x$.\\\\
	Respuesta.-\; Claramente podemos ver que,
	$$P(x)=\dfrac{3}{7}x^{\frac{7}{3}}-5\sin x,$$
	es una función primitiva ya que,
	$$P'(x)=x^{\frac{4}{3}}-5\cos x = f(x).$$
	Luego, aplicando el segundo teorema fundamental para calcular $\int_a^b f(x)\; dx$, tenemos
	$$\begin{array}{rcl}
	    \displaystyle\int_a^b f(x)\; dx &=& P(b)-P(a)\\\\
					    &=& \dfrac{3}{7}b^{\frac{7}{3}}-5\sin b -\left(\dfrac{3}{7}a^{\frac{7}{3}}-5\sin a\right)\\\\
					    &=&\dfrac{3}{7}\left(b^{\frac{7}{3}}-a^{\frac{7}{3}}\right)-5\left(\sin b - \sin a\right).
	\end{array}$$
	\vspace{0.5cm}

    %-------------------- 11.
    \item Demostrar que no existe ningún polinomio $f$ cuya derivada esté dada por la fórmula $f'(x)=\dfrac{1}{x}.$\\\\
	Demostración.-\; Supongamos lo contrario. Sea $f'(x)=\dfrac{1}{x},\; x\in \mathbb{R}$, entonces podemos hallar su primitiva de la siguiente manera:
	$$P(x)=\dfrac{x^{n+1}}{n+1}, n\in \mathbb{N}\quad \Rightarrow \quad f(x)=\dfrac{x^{-1+1}}{-1+1}=\dfrac{x^0}{0}.$$
	Lo cual es un absurdo. Por lo tanto no existe ningún polinomio $f$ cuya derivada esté dada por la fórmula $f'(x)=\dfrac{1}{x}.$\\\\

    %-------------------- 12.
    \item Demostrar que $\int_0^x |t| \; dt = \frac{1}{2}x|x|$ para todo número real $x$.\\\\
	Demostración.-\; Consideremos tres casos.
	\begin{enumerate}[\textit{Caso} 1.]
	    \item Si $x=0$, entonces para ambos lados de la ecuación el resultado será $0$, por lo que el resultado se cumple.
	    \item Si $x>0$, entonces $|t|=t$ para todo $t\in [0,x]$, así
		$$\int_0^x |t|\; dt = \int_0^x t \; dt = \dfrac{1}{2}x^2=\dfrac{1}{2}x|x|.$$
	    \item Si $x<0$, entonces $|t|=-t$ para todo $t\in [x,0]$, así
		$$\int_0^x |t|\; dt = -\int_0^x t \; dt=\int_x^0 t \; dt = \dfrac{1}{2}t^2\bigg|_x^0=-\dfrac{1}{2}x^2=\dfrac{1}{2}x|x|.$$
	\end{enumerate}
	\vspace{0.5cm}

    %-------------------- 13.
    \item Demostrar que 
    $$\int_0^x \left(t+|t|\right)^2\; dt = \dfrac{2x^2}{3}(x+|x|) \mbox{ para todo } x \mbox{ real.}$$\\
    	Demostración.-\; Consideremos dos casos.
	\begin{enumerate}[\textit{Caso} 1.]
	    \item Si $x\geq 0$ entonces $t=|t|$ para todo $t\in[0,x]$ así,
		$$\int_0^x (t+|t|)^2 \; dt= \int_0^x 4t^2\; dt = \dfrac{4}{3}x^3=\dfrac{2}{3}x^2(2x)=\dfrac{2}{3}x^2(x+|x|).$$
	    \item Si $x<0$ entonces $|t|=-t$ para todo $t\in[x,0]$ así,
		$$\int_0^x (t+|t|)^2 \; dt = \int_0^x (t-t)^2\; dt = 0 = \dfrac{2}{3}x^2(x-x)= \dfrac{2}{3}x^2(x+|x|).$$\\ 
	\end{enumerate}

    %-------------------- 14.
    \item Una función $f$ es continua para cualquier $x$ y satisface la ecuación
    $$\int_0^x f(t)\; dt = -\dfrac{1}{2}+x^2+x\sen2x + \dfrac{1}{2}\cos 2x.$$
    para todo $x$. Calcular $f\left(\frac{1}{4}\pi\right)$ y $f'\left(\frac{1}{4}\pi\right).$\\\\
	Respuesta.-\; Sea,
	$$A(x)=\int_0^x f(t)\; dt = -\dfrac{1}{2}+x^2+x\sen2x + \dfrac{1}{2}\cos 2x.$$
	Por el primer teorema fundamental del cálculo la derivada de $A(x)$ existe. Es decir,
	$$A'(x)=f(x).$$
	De donde,
	$$f(x)=A'(x)=2x+\sen(2x)+2x\cos(2x)-\sen(2x)=2x+2x\cos(2x).$$
	Luego, evaluando en $x=\frac{\pi}{4}$ tenemos
	$$f\left(\dfrac{\pi}{4}\right)=A'\left(\dfrac{\pi}{4}\right)=\dfrac{\pi}{2}+\dfrac{\pi}{2}\cos\left(\dfrac{\pi}{2}\right)=\dfrac{\pi}{2}.$$
	Por otro lado, derivamos $f(x)$ y obtenemos
	$$f(x)=2x+2x\cos(2x)\quad \Rightarrow \quad f'(x)=2+2\cos(2x)-4x\sen(2x).$$
	Así, 
	$$f'\left(\dfrac{\pi}{4}\right)=2+2\cos\left(\dfrac{\pi}{2}\right)-\pi\sen\left(\dfrac{\pi}{2}\right)=2-\pi.$$\\

    %-------------------- 15.
    \item Encontrar una función $f$ y un valor de la constante $c$, tal que:
    $$\int_c^x f(t)\; dt = \cos x - \dfrac{1}{2}\mbox{ para todo } x \mbox{ real. }$$\\
	Respuesta.-\; Sea $f(t)=-\sen t$ y $c=\frac{\pi}{3}$. Entonces,
	$$\begin{array}{rcl}
	    \displaystyle\int_c^x f(t)\; dt &=& \displaystyle\int_{\frac{\pi}{3}}^x (-\sen t)\; dt\\\\ 
					    &=&\cos t \bigg|_{\frac{\pi}{3}}^x\\\\
					    &=&\cos x - \cos \left(\dfrac{\pi}{3}\right)\\\\
					    &=&\cos x - \dfrac{1}{2}.
	\end{array}$$
	\vspace{0.5cm}

    %-------------------- 16.
    \item Encontrar una función $f$ y un valor de la constante $c$, tal que:
    $$\int_c^x t f(t)\; dt = \sen x-x\cos x-\dfrac{1}{2}x^2 \mbox{ para todo } x \mbox{ real. }$$\\
	Respuesta.-\; Sea $f(t)=\sen t-1$ y $c=0$. Entonces,
	$$\begin{array}{rcl}
	    \displaystyle\int_c^x tf(t)\; dt &=& \displaystyle \int_0^x (t\sen t - t)\; dt\\\\
					     &=& \left(\sen t - t \cos t - \dfrac{1}{2}t^2\right)\bigg|_0^x\\\\
					     &=& \sen x - x\cos x - \dfrac{1}{2}x^2.
	\end{array}$$
	\vspace{0.5cm}

    %-------------------- 17.
    \item Existe una función $f$ definida y continua para todo número real $x$ que satisface una ecuación de la forma:
    $$\int_0^x f(t)\; dt = \int_x^1 t^2f(t)\; dt + \dfrac{x^{16}}{8}+\dfrac{x^{18}}{9}+c,$$
    donde $c$ es constante. Encontrar una fórmula explícita para $f(x).$ y hallar el valor de la constante $c$.\\\\
	Respuesta.-\; Sea $P(x)=\int_x^1 t^2 f(t)\; dt.$ Entonces, usando los teoremas fundamentales del calculo, se tiene\\
	$$\begin{array}{rcl}
	    f(x)=\left[P(1)-P(x)\right]'+2x^{15}+2x^{17}+0&\Rightarrow&f(x)=-x^2f(x)+2x^{15}+2x^{17}\\\\
							  &\Rightarrow& f(x)\left(x^2+1\right)=2x^{15}\left(x^2+1\right).\\\\
							  &\Rightarrow& f(x)=2x^{15}.
	\end{array}$$

	Ahora, reemplazando $f(x)$ en la ecuación inicial, se tiene
	$$\begin{array}{rcl}
	    \displaystyle\int_0^x 2t^{15}\; dt &=& \displaystyle\int_x^1 t^{2}\cdot 2t^{15}\; dt + \dfrac{x^{16}}{8}+\dfrac{x^{18}}{9}+c\\\\
	    \dfrac{t^{16}}{8}\bigg|_0^x &=& \dfrac{t^{18}}{9}\bigg|_x^1+\dfrac{x^{16}}{8}+\dfrac{x^{18}}{9}+c\\\\
	    \dfrac{16}{8}&=&\dfrac{1}{9}-\dfrac{x^{18}}{9}+\dfrac{x^{16}}{8}+\dfrac{x^{18}}{9}+c\\\\
	    c&=&-\dfrac{1}{9}.
	\end{array}$$
	\vspace{0.5cm}

    %-------------------- 18.
    \item Una función $f$ está definida para todo real $x$ por la fórmula
    $$f(x)=3+\int_0^x \dfrac{1+\sen t}{2+t^2}\; dt.$$
    Sin intentar el cálculo de esta integral, hallar un polinomio cuadrado $p(x)=a+bx+cx^2$ tal que $p(0)=f(0),\; p'(0)$ y $p''(0)=f''(0).$\\\\
	Respuesta.-\; Sea $p(0)=a+b\cdot 0 + c \cdot 0^2=a$. Luego calculemos $f(0)$, como sigue
	$$f(0)=3+\int_0^0 \dfrac{1+\sen t}{2+t^2}\; dt=3.$$
	Dado que $f(0)=p(0)$ tenemos
	$$a=3.$$
	Después, sea 
	$$p'(x)=b+2cx \quad \Rightarrow \quad p'(0)=b$$
	Por el primer teorema fundamental del cálculo, 
	$$f'(x)=\displaystyle\left(3+\int_0^x \dfrac{1+\sen t}{2+t^2}\; dt\right)'=\dfrac{1+\sen x}{2+x^2}.$$
	Luego, $f'(0)=\dfrac{1}{2}$. Ya que $f'(0)=p'(0)$, entonces $b=\dfrac{1}{2}$.\\
	Finalmente, 
	$$p''(x)=2c\quad \Rightarrow \quad f''(0)=2c.$$
	Calculemos $f''(0)$, como sigue
	$$f''(0)=\dfrac{\cos x(2+x^2)-2x(1+\sen x)}{\left(2+x^2\right)^2}$$
	De donde 
	$$f''(0)=\dfrac{1\cdot (2+0)-0(1+0)}{(2+0)^2}=\dfrac{1}{2}.$$
	Dado que $f''(0)=p''(0)$, entonces 
	$$c=\dfrac{1}{4}.$$
	Así,
	$$p(x)=3+\dfrac{1}{2}x+\dfrac{1}{4}x^2.$$\\

    %-------------------- 19.
    \item Dada una función $g$, continua para todo $x$, tal que $g(1)=5$ e $\int_0^1 g(t)\; dt =2$. Póngase $f(x)=\frac{1}{2}\int_0^x (x-t)^2\; dt,$ demostrar que
    $$f'(x)=x\int_0^xg(t)\; dt-\int_0^x tg(t)\; dt,$$
    y calcular $f''(1)$ y $f'''(1)$.\\\\
	Demostración.-\; Podemos reescribir $f$ de la siguiente manera,
	$$\begin{array}{rcl}
	    f(x)&=&\displaystyle\dfrac{1}{2}\int_0^x (x-t)^2\; dt\\\\
		&=&\displaystyle\dfrac{1}{2} \int_0^x\left[x^2g(t)-2xtg(t)+t^2g(t)\right]\; dt\\\\
		&=&\displaystyle\dfrac{1}{2}\int_0^x x^2g(t)\; dt-\int_0^x xtg(t)\; dt+\int_0^x t^2g(t)\; dt\\\\
		&=&\displaystyle\dfrac{x^2}{2}\int_0^x g(t)\; dt-x\int_0^x tg(t)\; dt+\dfrac{1}{2}\int_0^x t^2g(t)\; dt.
	\end{array}$$
	\vspace{0.2cm}
	De donde sacamos $x$ de la integral ya que no depende de $t$. Después, usando la regla del producto y el teorema 1.1, sacamos la derivada de $x$.\\
	$$\begin{array}{rcl}
	    f'(x)&=&\displaystyle \left(\dfrac{x^2}{2}\right)' \displaystyle \int_0^x g(t)+\dfrac{x^2}{2}\left(\int_0^x g(t)\; dt\right)' - x' \int_0^x tg(t)\; dt - x\left(\int_0^x tg(t)\; dt\right)' + \dfrac{1}{2}\left(\int_0^x t^2g(t)\right)'\\\\
		 &=&\displaystyle x\int_0^x g(t)\; dt + \dfrac{x^2}{2} g(x) - \int_0^x tg(t)\; dt - x^2 g(x) + \dfrac{x^2}{2}g(x)\\\\
		 &=&\displaystyle x\int_0^x g(t)\; dt - \int_0^x tg(t)\; dt.
	\end{array}$$
	\vspace{0.2cm}

	Luego, calculamos $f''$ y $f'''$:
	$$\begin{array}{rcl}
	    f''(x)&=&\displaystyle \int_0^x g(t)\; dt + xg(x)-xg(x)\\\\
		  &=&\displaystyle \int_0^x g(t)\; dt\\\\
		  &&\\
	    f'''(x)&=&\displaystyle\left(\int_0^x g(t)\right)'\\\\
		   &=&g(x).
	\end{array}$$
	\vspace{0.2cm}

	Por lo tanto,

	$$\begin{array}{rcl}
	    f''(1)&=&\displaystyle \int_0^1 g(t)\; dt=2\\\\
		  &&\\
	    f'''(1)&=&\displaystyle g(1)=4.
	\end{array}$$
	\vspace{.5cm}

    %-------------------- 20.
    \item Sin calcular las siguientes integrales indefinidas, hallar la derivada $f'(x)$ en cada caso si $f(x)$ es igual a 
	\begin{enumerate}[(a)]

	    %---------- (a)
	    \item $\displaystyle\int_0^x (1+t^2)^{-3}\; dt$.\\\\ 
		Respuesta.-\; Sea $$A(x)=\int_0^x (1+t^2)^{-3}\; dt.$$
		Entonces por el primer teorema fundamental del calculo (teorema 1.1), 
		$$f'(x)=A'(x)=\left(1+x^2\right)^{-3}.$$\\

	    %---------- (b)
	    \item $\displaystyle\int_0^{x^2} (1+t^2)^{-3}\; dt$.\\\\ 
		Respuesta.-\; Sea $$A(x)=\int_0^{x^2} (1+t^2)^{-3}\; dt.$$
		Entonces, por teorema 1.1 
		$$f'(x)=A'\left(x^2\right)=2xA'\left(x^2\right) = (2x)\left[1+\left(x^2\right)^2\right]^{-3}=2x\left(1+x^4\right)^{-3}.$$

	    %---------- (c)
	    \item $\displaystyle\int_{x^3}^{x^2} (1+t^2)^{-3} \; dt$.\\\\ 
		Respuesta.-\; Reescribamos la integral como,
		$$f(x)=\int_{x^3}^{x^2}\left(1+t^2\right)^{-3}\; dt=\int_0^{x^2}-\int_0^{x^3}\left(1+t^2\right)^{-3}\; dt.$$
		Así, 
		    $$f(x)=A\left(x^2\right)-A\left(x^3\right).$$
		Por lo tanto,
		$$\begin{array}{rcl}
		    f'(x)&=&\left[A\left(x^2\right)\right]'-\left[A\left(x^3\right)\right]'\\\\
			 &=&(2x)A'\left(x^2\right) +\left(3x^2\right)A'\left(x^3\right)\\\\
			 &=&(2x)\left(1+x^4\right)^{-3}-\left(3x^2\right)\left(1+x^6\right)^{-3}
		\end{array}$$

	\end{enumerate}
	\vspace{.5cm}

    %-------------------- 21.
    \item Sin calcular la integral, calcular $f'(x)$ si $f$ está definida por la fórmula
    $$f(x)=\int_{x^3}^{x^2}\dfrac{t^6}{1+t^4}\; dt.$$\\
	Respuesta.-\; Sea 
	$$A(x)=\int_0^x \dfrac{t^6}{1+t^4}\; dt.$$
	Entonces, por el primer teorema fundamental del calculo (teorema 1.1),
	$$A'(x)=\dfrac{x^6}{1+x^4}.$$
	Así, usando la linealidad de la integral
	$$\begin{array}{rcl}
	    f(x)&=&\displaystyle\int_{x^3}^{x^2}\dfrac{t^6}{1+t^4}\; dt\\\\
		&=&\displaystyle\int_0^{x^2}\dfrac{t^6}{1+t^4}\; dt - \int_0^{x^3}\dfrac{t^6}{1+t^4}\; dt\\\\
		&=&A\left(x^2\right)-A\left(x^3\right).
	\end{array}$$
	Por lo tanto, usando la regla de la cadena concluimos que,
	$$\begin{array}{rcl}
	    f'(x)&=&\displaystyle\left[A\left(x^2\right)\right]' - \left[A\left(x^3\right)\right]'\\\\
		 &=&\displaystyle\left(2x\right)A'\left(x^2\right) - \left(3x^2\right)\left(x^3\right)\\\\
		 &=&\displaystyle\left(2x\right)\dfrac{x^{12}}{1+x^8} - \left(3x^2\right)\dfrac{x^{18}}{1+x^{12}}\\\\
		 &=&\displaystyle\dfrac{2x^{13}}{1+x^8} - \dfrac{3x^{20}}{1+x^{12}}.
	\end{array}$$
	\vspace{.5cm}

    %-------------------- 22.
    \item En cada caso, calcular $f(2)$ si $f$ es continua y satisface la fórmula dada para todo $x\geq 0$.
	\begin{enumerate}

	    %---------- (a)
	    \item $\displaystyle\int_0^x f(t)\; dt = x^2(1+x)$.\\\\
		Respuesta.-\; Ya que,
		$$\int_0^x f(t)\; dt = x^2(1+x)=x^3+x^2.$$
		Tomemos la derivada de ambos lados,
		$$f(x)=3x^2+2x.$$
		Por lo tanto,
		$$f(2)=16.$$\\

	    %---------- (b)
	    \item $\displaystyle \int_0^{x^2} f(t)\; dt = x^2(1+x)$.\\\\
		Respuesta.-\; Ya que,
		$$\int_0^{x^2}f(t)\; dt = x^2(1+x)=x^3+x^2.$$
		Tomamos la derivada de ambos lados como se observa en la parte (b) del ejercicio 20, Spivak, capítulo 5. Para obtener
		$$2xf\left(x^2\right)=3x^2+2x\quad \Rightarrow \quad f\left(x^2\right)=\dfrac{3}{2}x+1.$$
		Por lo tanto,
		$$f(2)=f\left(\sqrt{2}^2\right)=\dfrac{3\sqrt{2}}{2}+1.$$\\

	    %---------- (c)
	    \item $\displaystyle \int_0^{f(x)} t^2\; dt = x^2(1+x)$.\\\\
		Respuesta.-\; Tenemos,
		$$\int_0^{f(x)}t^2\; dt = x^2(1+x)=x^3+x^2.$$
		Evaluando la integral de la parte izquierda,
		$$\int_0^{f(x)}t^2\; dt = \dfrac{1}{3}\bigg|_0^{(x)} = \dfrac{1}{3}\left[f(x)\right]^3.$$
		Por lo tanto,
		$$f(2)=36^{\frac{1}{3}}.$$\\

	    %---------- (d)
	    \item $\displaystyle \int_0^{x^2(1+x)} f(t)\; dt = x$.\\\\
		Respuesta.-\; Tenemos,
		$$\int_0^{x^2(1+x)}f(t)\; dt = x.$$
		Tomando la derivada de ambos lados,
		$$\left(3x^2+2x\right)f\left(x^3+x^2\right)=1\quad \Rightarrow \quad f\left(x^3+x^2\right)=\dfrac{1}{3x^2+2x}.$$
		Entonces, por el hecho de que
		$$x^3+x^2=2\quad \Leftrightarrow \quad x=1,$$
		ya que que el polinomio cúbico $x^3+x^2-2=0$ tiene sólo una raíz real en $x=1$. Entonces,
		$$f(2)=\dfrac{1}{5}.$$\\

	\end{enumerate}

    %-------------------- 23.
    \item La base de un sólido es el conjunto de ordenadas de una función no negativa $f$ en el intervalo $[0,a]$. Todas las secciones perpendiculares a ese intervalo son cuadrados. El volumen del sólido es 
    $$a^3-2a\cos a + \left(2-a^2\right)\sen a$$
    para todo $a\geq 0$. Suponiendo que $f$ es continua en $[0,a]$, calcular $f(a)$.\\\\
	Respuesta.-\; Tenemos la expresión,
	$$V=a^3-2a\cos a + \left(2-a^2\right)\sen a.$$
	Sabemos que podemos calcular el volumen como la integral de $0$ a $a$ del área de cada sección transversal. Dado que las secciones transversales son cuadrados y la longitud del borde en la base está dada por $f(x)$ (ya que la base es el conjunto ordenado de $f$, la longitud de la base de una sección transversal en un punto $x$ es $f(x)-0=f(x)$). Sabemos que el área de cada sección transversal es $\left[f(x)\right]^2$. Por lo tanto, tenemos otra expresión para el volumen del sólido dada por
	$$V=\int_0^a \left[f(x)\right]^2\; dx.$$
	Igualando estas dos expresiones para el volumen y diferenciando ambos lados tenemos
	$$\begin{array}{rcl}
	    a^3-2a\cos a + \left(2-a^2\right)\sen a = \displaystyle \int_0^x \left[f(x)\right]^2\; dx &\Rightarrow& 3a^2-a^2\cos a = \left[f(a)\right]^2\\
														  &\Rightarrow& f(a)=a(3-\cos a)^{\frac{1}{2}}.
	\end{array}$$
	\vspace{.5cm}

    %-------------------- 24.
    \item Un mecanismo impulsa una partícula a lo largo de una recta. Está concebido de manera que la posición de la partícula en el instante $t$ a partir del punto inicial $0$ en la recta está dado por la fórmula $f(t) = \frac{1}{2}t^2 + 2t \sen t$. El mecanismo trabaja perfectamente hasta el instante $t = \pi$ en surge una avería inesperada. A partir de ese momento la partícula se mueve con velocidad constante (1a velocidad adquirida en el instante $t=\pi$). Calcular : a) su velocidad en el instante $t=\pi$; b) su aceleración en el instante $t=\frac{1}{2}\pi$; c) su aceleración en el instante $\frac{3}{2}\pi$; d) su posición a partir de $O$ en el instante $t=\frac{5}{2}\pi$; e) Hallar el instante $t>\pi$ en el que la partícula vuelve al punto inicial $O$, o bien demostrar que nunca regresa a $O$.\\\\
	Respuesta.-\; 
	\begin{enumerate}[a)]
	    \item Ya que la velocidad de la partícula es dada por la derivada de la posición, tenemos
	$$v(t)=f'(t)=t+2\sen t + 2t\cos t.$$
	$$v(\pi) = \pi + 2\sen(\pi)+2\pi\cos(\pi)=\pi -2\pi = -\pi.$$\\
	    \item La aceleración es la derivad de la velocidad del inciso a). Por lo tanto,
		$$\begin{array}{rcl}
		    a(t) = v'(t) &=& 1+2\cos t + 2\cos t -2t\sen t.\\
			 &=& 1+4\cos t -2t\sen t.
		\end{array}$$
		para $a\left(\dfrac{\pi}{2}\right)$,	

		$$\begin{array}{rcl}
		    a\left(\dfrac{\pi}{2}\right) &=& 1+4\cos\left(\dfrac{\pi}{2}\right) -\pi\sen\left(\dfrac{\pi}{2}\right).\\
						 &=&1+\pi.
		\end{array}$$
		\vspace{.5cm}

	    \item Sabemos por el enunciado del problem que  $v(t)=c$ para $t>\pi$, donde $c$ es una constante. Ya que $v'(t)=0$ para $t>\pi$, se sigue que la aceleración en el tiempo $t=\dfrac{3}{2}\pi$ es $0$.\\\\

	    \item Para encontrar la posición en el tiempo $t=\frac{5}{2}\pi$, consideremos el movimiento de la partícula en dos intervalos de tiempo: El tiempo de $0$ a $\pi$ y el tiempo desde $\pi$ hasta $\frac{5}{2}\pi$. Durante el intervalo de tiempo $[0,\pi]$ la posición es dado por la siguiente función 
		$$f(t)=\dfrac{1}{2}t^2+2t\sen t.$$
		Para el tiempo $t=\pi$, sabemos que la partícula se mueve con velocidad constante de $f'(\pi)$. Por lo que su posición cambia por $\frac{3}{2}\pi f'(\pi)$, durante el intervalo de tiempo $[\pi,\frac{5}{2}\pi]$. Por lo tanto, la posición en el tiempo $t=\frac{5}{2}\pi$ viene dado por 
		$$\begin{array}{rcl}
		    f\left(\dfrac{5\pi}{2}\right) &=& f(\pi) + \left(\dfrac{3\pi}{2}\right) f'(\pi)\\\\
						  &=&\dfrac{\pi^2}{2}-\dfrac{3\pi^2}{2}\\\\
						  &=& -\pi^2.
		\end{array}$$
		\vspace{.5cm}

	    \item Se nos pide encontrar $t>\pi$ tal que $g(t)=0$, donde 
		$$g(t)=f(\pi)+(t-\pi)v(\pi).$$
		En la parte d), obtuvimos $f(\pi)=\frac{1}{2}\pi^2$ y $v(\pi)=-\pi$. Por lo tanto, 
		$$\begin{array}{rcl}
		    \dfrac{1}{2}\pi^2-\pi(t-\pi)&=&0\\\\
		    \dfrac{1}{2}\pi^2-\pi t + \pi^2&=&0\\\\
		    t\pi &=&\dfrac{3}{2}\pi^2\\\\
		    t&=&\dfrac{3}{2}\pi.
		\end{array}$$
		\vspace{.5cm}
	\end{enumerate}

    %-------------------- 25.
    \item Una partícula se desplaza a lo largo de una recta. Su posición en el instante $t$ en $f(t)$. Cuando $0\leq t \leq 1$, la posición viene dada por la integral
    $$f(t)=\int_0^t \dfrac{1+2\sen \pi x \cos \pi x}{1+x^2}\; dx.$$
    (No intente el cálculo de esta integral.) Para $t\geq 1$, la partícula se mueve con aceleración constante (la aceleración adquirida en el instante $t=1$). Calcular: a) su aceleración en el instante $t=2$; b) su velocidad cuando $t=1$; c) su velocidad cuando $t>1$; d) la distancia $f(t)-f(1)$ cuando $t>1$.\\\\
    	Respuesta.-\;
	\begin{enumerate}[a)]
	    \item Dado que la aceleración en el tiempo $t\geq 1$ es constante, la aceleración en el tiempo $t=2$ es la misma que la aceleración en tiempo $t=1$. Para encontrar la aceleración en el tiempo $t=1$, diferenciamos dos veces $f(t)$ y lo evaluamos en $t=1$,
	    $$\begin{array}{rcl}
		f(t)&=&\displaystyle\int_0^t \dfrac{1+2\sen(\pi x)\cos (\pi x)}{1+x^2}\; dx.\\\\
		&&\\
		f'(t) &=& \dfrac{1+2\sen(\pi x)\cos (\pi x)}{1+x^2}\\\\
		      &=& \dfrac{1+\sen(2\pi t)}{1+t^2}.\\\\
		&&\\
		f''(t) &=& \dfrac{\left(1+t^2\right)\left[\cos(2\pi t)-2t\left(1+\sen^2(\pi t)\right)\right]}{\left(1+t^2\right)^2}\\\\
		       &&\\
		       f''(1)&=& \dfrac{4\pi-2}{4}\\\\
			     &=&\pi-\dfrac{1}{2}.
	    \end{array}$$
	    \vspace{.5cm}

	    \item De la parte (a) sabemos que $f'(t)=v(t)=\dfrac{1+\sen(2\pi t)}{1+t^2}$. Por lo que la velocidad en el tiempo $t=1$ es
		$$v(1)=f'(1)=\dfrac{1+\sen(2\pi \cdot 1)}{1+1^2}=\dfrac{1}{2}.$$\\

	    \item Podemos encontrar la velocidad en el tiempo $t>1$, determinando la velocidad en el tiempo $t=1$ y añadiendo la velocidad para moverse con la aceleración constante. Por lo tanto, tenemos
	    $$v(t)=v(1)+(t-1)\cdot a(t),\qquad t>1.$$
	    Sabemos que $t>1$. Así $a(t)=a(1)=\pi-\dfrac{1}{2}.$ También conocemos que $v(t)=\dfrac{1}{2}.$ De donde,
	    $$v(t)=\dfrac{1}{2}+(t-1)\left(\pi-\dfrac{1}{2}\right),\qquad t>1.$$\\

	\item La diferencia $f(t) - f(1)$ es la posición en el tiempo $t$ menos la posición en el tiempo $t = 1$, donde t > 0. Entonces, usando la linealidad de la integral con respecto al intervalo de integración que tenemos, 
	    $$f(t)-f(1)=\int_1^t f'(x)\; dt,\quad f'(t)-f'(1)=\int_1^t f''(x)\; dx.$$
	    Sabemos que $f''(t)$ es la aceleración y para $t>1$ viene dada por 
	    $$f''(t)=\pi-\dfrac{1}{2}.$$
	    Por lo tanto,
	    $$f'(t)-f'(1)=\left(\pi-\dfrac{1}{2}\right)(t-1).$$
	    De donde,
	    $$\begin{array}{rcl}
		f(t)-f(1)&=&\displaystyle\int_1^t (x-1)\left(\pi-\dfrac{1}{2}\right)\; dx\\\\
			 &=&\displaystyle\int_1^t \left(\pi-\dfrac{1}{2}\right)x\; dx - \int_1^t\left(\pi-\dfrac{1}{2}\right)\; dx\\\\
			 &=&\left(\pi-\dfrac{1}{2}\right)\left[\dfrac{1}{2}(t^2-1)\right]_1^t - \left(t-1\right)\left[\pi-\dfrac{1}{2}\right]\\\\
			 &=&\left(\pi-\dfrac{1}{2}\right)\left(\dfrac{t^2-2t+1}{2}\right).
	    \end{array}$$
	    \vspace{.5cm}

	\end{enumerate}

    %-------------------- 26.
    \item En cada uno de los casos siguientes encontrar una función $f$ (con segunda derivada $f''$ continua) que satisfaga a todos las condiciones indicadas, o bien explicar por qué no es posible encontrar una tal función.

	\begin{enumerate}[(a)]

	    %---------- (a)
	    \item $f''(x)>0$ para cada $x$, $f'(0)=1$, $f'(1)=0.$\\\\
		Respuesta.-\; No puede haber una función que cumpla todas estas condiciones, ya que $f''(x)>0$, de donde $f'(x)$ es creciente. Esto porque su derivada, $f''$, es positiva. Luego $f'(0)>f'(1)$ contradice que $f'(x)$ vaya creciendo.\\\\

	    %---------- (b)
	    \item $f''(x)>0$ para cada $x$, $f'(0)=1$, $f'(1)=3$.\\\\
		Respuesta.-\; Sea $f(x)=x^2+x$. Luego
		$$\begin{array}{rcl}
		    f'(x)=2x+1&\Rightarrow & f'(0)=1\\
			      &\Rightarrow& f'(1)=3.
		\end{array}$$
		Además, $f''(x)=2>0$ para cada $x$.\\\\

	    %---------- (c)
	    \item $f''(x)>0$ para cada $x$, $f'(0)=1$, $f(x)\leq 100$ para cada positivo $x$.\\\\
		Respuesta.-\; No puede haber ninguna función que cumpla todas estas condiciones. Una vez más, $f''(x) > 0$ para todos $x$ implica que $f'(x)$ está creciendo para todos $x$. Por lo tanto, $f'(0) = 1$ implica $f'(x) > 1$ para todos $x > 0$. Entonces, por el teorema del valor medio, sabemos que para cualquier $b > 0$ existe algún $c \in (0,b)$ tal que
		$$f(b)-f(0)=f'(c)(c-0)\quad \Rightarrow \quad f(b)>b+f(0),\quad \mbox{ya que }f'(c)>1.$$
		Luego, elija $b>100-f(0)$. Por lo que, $f(b)>100$, el cual contradice $f(x)\leq 100$ para cada $x>0$.\\\\

	    %---------- (d)
	    \item $f''(x)>0$ para cada $x$, $f'(0)=1$, $f(x)\leq 100$ para cada negativo $x$.\\\\
		Respuesta.-\; Definamos $f$ como sigue,
		$$f(x)=\left\{\begin{array}{ll}
		    1+x+x^2&\mbox{ si }x\geq 0\\\\
		    \dfrac{1}{1-x}&\mbox{ si }x<0.
		\end{array}\right.$$
		Notemos que la derivada está definida en $x=0$, por lo que podemos derivar cada parte,
		$$f'(x)=\left\{\begin{array}{ll}
		    1+2x&\mbox{ si }x\geq 0\\\\
		    \dfrac{1}{(1-x)^2}&\mbox{ si }x<0.
		\end{array}\right.$$
		Derivando una vez más, tenemos
		$$f''(x)=\left\{\begin{array}{ll}
		    2&\mbox{ si }x\geq 0\\\\
		    \dfrac{2}{(1-x)^3}&\mbox{ si }x<0.
		\end{array}\right.$$
		Así, $f''(x)>0$ para todo $x$ y $f'(0)=1$. Además, para $x<0$ se tiene,
		$$f(x)=\dfrac{1}{1-x}\leq 100\mbox{ para todo }x<0.$$\\
		
	\end{enumerate}

    %-------------------- 27.
    \item Una partícula se mueve a lo largo de una recta, siendo su posición en el instante $t$, $f(t)$. Parte con una velocidad inicial $f'(0) = O$ Y tiene una aceleración continua $f"(t) \geq 6$ para todo $t$ en el intervalo $O \leq t \leq 1$. Demostrar que la velocidad es $f'(t) \geq 3$ para todo $t$ en un cierto intervalo $[a, b]$, donde $O \leq a < b \leq 1$, siendo $b - a =\frac{1}{2}$.\\\\
	Respuesta.-\; Por el segundo teorema fundamental del cálculo, tenemos
	$$\begin{array}{rcl}
	    \displaystyle f'(t)-f'(0) = \int_0^t f''(x)\; dx&\Rightarrow & \displaystyle f'(t)\geq \int_0^x 6\; dx\\\\
							    &\Rightarrow& f'(t)\geq 6t\\\\
							    &\Rightarrow& f'\left(\dfrac{1}{2}\right)\geq 3.
	\end{array}$$
	Entonces, ya que $f''(x)\geq 6>0$, que implica $f'(x)$ sea creciente para todo $t\in[0,1]$, se tiene
	$$f'(t)\geq 3,\; \forall t\in\left[\dfrac{1}{2},1\right].$$
	Con esto, notemos que tenemos $b=1$ y $a=\frac{1}{2}$, por ende $b-a=\frac{1}{2}$ y $f'(t)\geq 3,\; \forall t\in[a,b]$ para $0\leq a<b\leq 1$ , como se solicitó.\\\\

    %-------------------- 28.
    \item Dada una función $f$ tal que la integral $A(x)=\int_a^x f(t)\; dt$ exista para cada $x$ e un intervalo $[a,b]$. Sea $c$ un punto del intervalo abierto $(a,b)$. Considerar las siguientes afirmaciones relativas a $f$ y $A$:
    \begin{multicols}{2}
	\begin{enumerate}[a)]
	    \item $f$ es continua en $c$.
	    \item $f$ es discontinua en $c$.
	    \item $f$ es creciente en $(a,b)$.
	    \item $f'(c)$ existe.
	    \item $f'$ es continua en $c$.
	\end{enumerate}
	\begin{enumerate}
	    \item[$\alpha$)] $A$ es continia en $c$.
	    \item[$\beta$)] $A$ es discontinua en $c$.
	    \item[$\gamma$)] $A$ es convexa en $(a,b)$.
	    \item[$\delta$)] $A'(c)$ existe.
	    \item[$\epsilon$)] $A'$ es continua en $c$.
	\end{enumerate}
    \end{multicols}
    \vspace{.4cm}
	Respuesta.-\; La tabla estaría dada por:
	$$\begin{array}{c|ccccc}
	    &\alpha&\beta&\gamma&\delta&\epsilon\\
	    \hline
	    a&V&&&V&\\
	    b&V&&&&\\
	    c&V&&V&&\\
	    d&V&&&V&\\
	    e&V&&&V&V\\
	\end{array}$$
	Sabemos que la primera columna tiene todos los $V$ ya que la función $A(x)=\int_a^x f(t)\; dt$ es continua para cualquier función $f$. Por el primer teorema fundamental sabemos que la derivada $A'(x)$ existe en cada punto en el intervalo abierto $(a,b)$. Como diferenciabilidad implica continuidad, entonces sabemos que $A(x)$ es continua en cualquier punto $c\in(a,b)$.\\
	Luego sabemos que la segunda columna no puede tener $V$ en ninguna parte, por el mismo argumento anterior, la función $A(x)$ es continua en todos los puntos $c\in (a,b)$. Por lo tanto, no puede ser discontinuo en ningún punto.\\
	El enunciado $a,b,c,d$ y $e$ no puede implicar que $A(x)$ sea convexo en $(a,b)$, ya que son enunciados sobre continuidad y existencia de derivadas. Ninguna de estas propiedades tiene  que ver con la convexidad. Entonces, el enunciado c) si implica la afirmación $\gamma$, esto porque una función es convexa si su derivada es creciente. Como $f$ es la derivada de $A$ y es creciente tenemos que $A$ es convexa.

\end{enumerate}


\section{La notación de Leibniz para las primitivas}
Hemos definido una primitiva $P$ de una función $f$ como cualquier función para la que $P'(x)=f(x).$ Si $f$ es continua en un intervalo, una primitiva viene dada por una fórmula de la forma
$$P(x)=\int_c^x f(t)\; dt.$$
Leibniz usó el símbolo $\int f(x)\; dx$ para designar una primitiva general de $f$. Con esta notación, una igualdad como
$$\int f(x)\; dx=P(x)+C$$
se considera como otra forma de escribir $P'(x)=f(x).$\\

Ya que la derivada de $\frac{x^{n+1}}{n+1}$ es $x^n$, podemos escribir
$$\int x^n \; dx = \dfrac{x^{n+1}}{n+1}+C,$$
para cualquier potencia racional tal que $n\neq -1$. El símbolo $C$ representa una constante arbitraria.\\

El primer teorema fundamental indica que cada integral indefinida de $f$ es también una primitiva de $f$. Por lo cual, en $\int f(x)\; dt=P(x)+C$, se puede sustituir $P(x)$ por $\int_c^x f(t)\;dt$ donde $c$ es un cierto límite inferior y resulta:
\begin{tcolorbox}[colback=black!3,colframe=white]
$$\int f(x)\; dx=\int_c^x f(t)\; dt + C.$$
\end{tcolorbox}
Esto indica que se puede considerar el símbolo $\int f(x)\; dx$ como representante de una integral indefinida de $f$, más una constante.\\

El segundo teorema fundamental, expresa que para cada primitiva $P$ de $f$ y cada constante $C$, se tiene:
$$\int_a^b f(x)\; dx = \left[P(x)+C\right]\bigg|_a^b.$$
Si se sustituye $P(x)+C$ por $\int f(x)\; dx$, esta fórmula se puede escribir de la forma:
\begin{tcolorbox}[colback=black!3,colframe=white]
$$\int_a^b f(x)\; dx = \int f(x)\; dx \bigg|_a^b.$$
\end{tcolorbox}

Debido a una larga tradición, muchos tratados de Cálculo consideran el símbolo $\int f(x)\; dx$ como representante de una integral indefinida y no de una función primitiva o antiderivada-


\section{Integración por sustitución}
Sea $Q$ la composición de dos funciones $P$ y $g$, es decir $Q(x)=P[g(x)]$ para todo $x$ en un cierto intervalo $I$. Entonces,
\begin{center}
    $P'(x)=f(x)$ implica $Q'(x)=f[g(x)]g'(x)$.
\end{center}

Con la notación de Leibniz, esta afirmación puede escribirse del modo siguiente: Si tenemos la fórmula de integración
$$\int f(x)=P(x)+C,$$
tenemos también la fórmula más general
$$\int f[g(x)]g'(x)\; dx = P[g(x)]+C.$$
Sustituimos $g(x)$ por un nuevo símbolo $u$ y reemplacemos $g'(x)$ por $du/dx,$ según la notación de Leibniz para las derivadas. Entonces,
$$\int f(u) \dfrac{du}{dx}\; dx = P(u)+C \quad \Rightarrow \quad \int f(u)\; du = P(u)+C.$$

\begin{teo}[Teorema de sustitución para integrales]
    Supongamos que $g$ tiene una derivada continua $g'$ en un intervalo $I$. Sea $J$ el conjunto de los valores que toma $g$ en $I$ y supongamos que $f$ es continua en $J$. Entonces para cada $x$ y cada $c$ en $I$, tenemos
    $$\int_c^x f[g(t)]g'(t)\; dt = \int_{g(c)}^{g(x)}f(u)\; du.$$
	Demostración.-\; Sea $g(c)$ y definimos dos nuevas funciones $P$ y $Q$ del siguiente modo:
	$$P(x)=\int_c^x f(u)\; du \quad \mbox{si } x\in J,\qquad Q(x)=\int_c^x f[g(t)]g'(t)\; dt\quad \mbox{si } x\in I.$$
	Puesto que $P$ y $Q$ son integrales indefinidas de funciones continuas, tienen derivadas dada por la fórmulas
	$$P'(x)=f(x),\quad Q'(x)=f[g(x)]g'(x).$$
	Llamemos ahora $R$ a la función compuesta, $R(x)=P[g(x)]$. Con la regla de la cadena, encontramos 
	$$R'(x)=P'[g(x)]g'(x)=f[g(x)]g'(x)=Q'(x).$$
	Aplicando dos veces el segundo teorema fundamental, obtenemos
	$$\int_{g(x)}^{g(x)} f(u)\; du = \int_{g(x)}^{g(x)} P'(u)\; du = P[g(x)]-P[g(c)]=R(x)-R(c),$$
	y
	$$\int_c^x f[g(t)]g'(t)\; dt = \int_c^x Q'(t)\; dt = \int_c^x R'(t)\; dt = R(x)-R(c).$$
	Esto demuestra que las dos integrales $\int_c^x f[g(t)]g'(t)\; dt = \int_{g(c)}^{g(x)}f(u)\; du$ son iguales.
\end{teo}

\section{Ejercicios}

En los ejercicios del 1 al 20, aplicar el método de sustitución para calcular las integrales.\\

\begin{enumerate}[\bfseries 1.]

    %------------------ 1.
    \item $\displaystyle\int \sqrt{2x+1}\; dx$.\\\\
	Respuesta.-\; Sea $u=2x+1 \; \Rightarrow \; 2=\dfrac{du}{dx}.$
	Entonces, usando la sustitución
	$$\begin{array}{rcl}
	    \displaystyle \int \sqrt{2x+1}\; dx &=& \displaystyle \int \dfrac{\sqrt{u}}{2}\; du\\\\
	    &=& \displaystyle \dfrac{1}{2}\int \sqrt{u}\; du\\\\
	    &=& \dfrac{1}{2} \left(\dfrac{u^{\frac{1}{2}+1}}{\dfrac{1}{2}+1}\right)+C\\\\
	    &=& \dfrac{1}{3}(2x+1)^{\frac{3}{2}}+C.
	\end{array}$$
	\vspace{.5cm}

    %------------------ 2.
    \item $\displaystyle \int x\sqrt{1+2x}\; dx$.\\\\
	Respuesta.-\; Sea 
	$$\begin{array}{rcl}
	    u = 1+3x & \Rightarrow & x=\dfrac{u-1}{3}\\\\
	    du=3\; dx & \Rightarrow & dx=\dfrac{1}{3}\; du.
	\end{array}$$

	Entonces,

	$$\begin{array}{rcl}
	    \displaystyle \int x\sqrt{1+3x}\; dx &=& \displaystyle\int \left(\dfrac{u-1}{3}\right)\left(\sqrt{u}\right)\; \dfrac{du}{3}\\\\
						 &=&\dfrac{1}{9}\displaystyle \int \left(u^{\frac{3}{2}}-u^{\frac{1}{2}}\right)\; du\\\\
						 &=& \dfrac{1}{9}\displaystyle \int u^{\frac{3}{2}} - \int u^{\frac{1}{2}}\; du\\\\
						 &=& \dfrac{1}{9}\left(\dfrac{2}{5}u^{\frac{5}{2}}-\dfrac{2}{3}u^{\frac{3}{2}}\right) + C\\\\
						 &=& \dfrac{2}{45}(1+3x)^{\frac{5}{2}} - \dfrac{2}{27}(1+3x)^{\frac{3}{2}} + C.
	\end{array}$$
	\vspace{.5cm}

    %------------------ 3.
    \item $\displaystyle\int x^2\sqrt{x+1}\; dx.$\\\\
	Respuesta.-\; Sea 
	$$\begin{array}{rcl}
	    u = x+1 & \Rightarrow & x=u-1\\\\
	    du=dx. &&
	\end{array}$$

	Entonces,

	$$\begin{array}{rcl}
	    \displaystyle \int x^2\sqrt{x+1}\; dx &=& \displaystyle\int (u-1)^2 \sqrt{u}\; du\\\\
						  &=& \displaystyle \int \left(u^{\frac{5}{2}}-2u^{\frac{3}{2}}+u^{\frac{1}{2}}\right)\; du\\\\
						  &=& \dfrac{2}{7}u^{\frac{7}{2}}-\dfrac{4}{5}u^{\frac{5}{2}}+\dfrac{2}{3}u^{\frac{3}{2}}+C\\\\
						  &=& \dfrac{2}{7}(x+1)^{\frac{7}{2}}-\dfrac{4}{5}(x+1)^{\frac{5}{2}}+\dfrac{2}{3}(x+1)^{\frac{3}{2}}+C.
	\end{array}$$
	\vspace{.5cm}

    %------------------ 4.
    \item $\displaystyle\int_{-\frac{2}{3}}^{\frac{1}{3}} \dfrac{x\; dx}{\sqrt{2-3x}}$.\\\\

	Respuesta.-\; Sea
	$$\begin{array}{rcl}
	    u=2-3x & \Rightarrow & x=\dfrac{2-u}{3}\\\\
	    du=-3\; dx & \Rightarrow & dx=-\dfrac{1}{3}\; du.
	\end{array}$$
	Por el teorema 1.4,
	$$u\left(\dfrac{1}{3}\right)=2-3\cdot \dfrac{1}{3} = 1 \qquad \mbox{y}\qquad u\left(-\dfrac{2}{3}\right)=2-3\cdot \left(-\dfrac{2}{3}\right)=4.$$
	de donde,
	$$\begin{array}{rcl}
	    \displaystyle\int \dfrac{x\; dx}{\sqrt{2-3x}} &=& \displaystyle \int_{u\left(-\frac{2}{3}\right)}^{u\left(\frac{1}{3}\right)} \dfrac{\dfrac{2-u}{3}}{\sqrt{u}}\; \dfrac{du}{3}\\\\
							  &=& -\dfrac{1}{9}\displaystyle \int_{4}^{1} (2-u)u^{-\frac{1}{2}}\; du\\\\
							  &=& -\dfrac{1}{9}\displaystyle \int_{4}^{1} \left(2u^{-\frac{1}{2}}-u^{\frac{1}{2}}\right)\; du\\\\
							  &=&-\dfrac{1}{9}\left[\left(2\dfrac{u^{-\frac{1}{2}}+1}{-\dfrac{1}{2}+1}\right)\bigg|_4^1-\left(\dfrac{u^{\frac{1}{2}+1}}{\dfrac{1}{2}+1}\right)\bigg|_{4}^{1}\right]\\\\
							  &=&-\dfrac{1}{9}\left[(4-8)-\left(\dfrac{2}{3}-\dfrac{16}{3}\right)\right]\\\\
							  &=&-\dfrac{2}{27}.
	\end{array}$$
	\vspace{.5cm}

    %------------------ 5.
    \item $\displaystyle\int \dfrac{(x+1)\; dx}{\left(x^2+2x+2\right)^3}.$\\\\

	Respuesta.-\; Sea
	$$u=x^2+2x+3\quad \Rightarrow \quad du=2(x+1)\; dx$$
	Entonces,
	$$\begin{array}{rcl}
	    \displaystyle \int \dfrac{(x+1)\; dx}{\left(x^2+2x+3\right)^3} &=& \displaystyle\dfrac{1}{2}\int u^{-3}\; du\\\\
									   &=&-\dfrac{1}{4}u^{-2}+C\\\\
									   &=&-\dfrac{1}{4}\left(x^2+2x+3\right)^{-2}+C.
	\end{array}$$
	\vspace{.5cm}

    %------------------ 6.
    \item $\displaystyle\int \sen^3 x\; dx$.\\\\
	Respuesta.-\; Reescribamos la integral como
	$$\begin{array}{rcl}
	    \displaystyle \int \sen^3 \; dx &=& \displaystyle\int \sen x\cdot \sen^2 x \; dx\\\\
					    &=& \displaystyle \int \sen x\left(1-\cos^2 x\right)\; dx\\\\
					    &=& \displaystyle \int \sen x \; dx - \int \sen x \cos^2 x \; dx
	\end{array}$$
	Luego, sea
	$$u=\cos x \quad \Rightarrow \quad du=-\sen x\; dx.$$
	Entonces,
	$$\begin{array}{rcl}
	    \displaystyle \int \sen^3 x \; dx &=& \displaystyle \int \sen x \; dx-\int \sen x\cos^2 x \; dx\\\\
					      &=&-\cos x + \displaystyle \int u^2 \; du\\\\
					      &=& - \cos x + \dfrac{1}{3}u^3+C\\\\
					      &=& \dfrac{1}{3}\cos^3 x - \cos x +C.
	\end{array}$$
	\vspace{.5cm}

    %------------------ 7.
    \item $\displaystyle\int z(z-1)^{\frac{1}{3}}\; dx$.\\\\
	Respuesta.-\; Sea
	$$u=z-1\quad \Rightarrow \quad du=dz.$$
	Entonces,
	$$\begin{array}{rcl}
	    \displaystyle \int z(z-1)^{\frac{1}{3}}\; dx &=& \displaystyle \int (u+1)u^{\frac{1}{3}}\; du\\\\
							 &=& \dfrac{3}{7}u^{\frac{7}{3}} + \dfrac{3}{4}u^{\frac{4}{3}}+C\\\\
							 &=& \dfrac{3}{7}(z-1)^{\frac{7}{3}} + \dfrac{3}{4}(z-1)^{\frac{4}{3}}+C.
	\end{array}$$
	\vspace{.5cm}

    %------------------ 8.
    \item $\displaystyle\int \dfrac{\cos x \; dx}{\sen^3 x}$.\\\\

	Respuesta.-\; Sea,
	$$u=\sen x \quad \Rightarrow \quad du=\cos x\; dx.$$
	Entonces,
	$$\begin{array}{rcl}
	    \displaystyle \int \dfrac{\cos x \; dx}{\sen^3 x}\; dx &=& \displaystyle \int u^{-3}\; du\\\\
								   &=& -\dfrac{1}{2}u^{-2}+C\\\\
								   &=& -\dfrac{1}{2}\cosec^{2} x +C.
	\end{array}$$
	\vspace{.5cm}

    %------------------ 9.
    \item $\displaystyle\int_0^{\pi/4} \cos 2 x\sqrt{4-\sen 2x}\; dx$.\\\\

	Respuesta.- Sea,
	$$u=4-\sen(2x)\quad \Rightarrow \quad du=-2\cos(2x)\; dx.$$
	Por el teorema 1.4,
	$$u\left(\dfrac{\pi}{4}\right)=4-\sen\left(\dfrac{\pi}{2}\right)=4-1=3 \qquad \mbox{y}\qquad u\left(0\right)=4-\sen\left(0\right)=4-0=4.$$

	Entonces,
	$$\begin{array}{rcl}
	    \displaystyle\int_0^{\frac{\pi}{4}} \cos(2x) \sqrt{4-\sen(2x)}\; dx &=& \displaystyle-\dfrac{1}{2}\int_{u(0)}^{u\left(\frac{\pi}{4}\right)} u^{\frac{1}{2}}\; du\\\\
										&=&\displaystyle \int_4^3 u^{\frac{1}{2}}\; du\\\\
										&=& -\dfrac{1}{3}u^{\frac{3}{2}}\bigg|_4^3\\\\
										&=&-\dfrac{1}{3}\left(\sqrt{27}-8\right)\\\\
										&=&\dfrac{8}{3}-\sqrt{3}.
	\end{array}$$
	\vspace{.5cm}


    %------------------ 10.
    \item $\displaystyle\int \dfrac{\sen x\; dx}{(3+\cos x)^2}.$\\\\

	Respuesta.-\; Sea,
	$$u=3+\cos x \quad \Rightarrow \quad du=-\sen x\; dx.$$
	Entonces,
	$$\int \dfrac{\sen x\; dx}{(3+\cos x)^2}=-\int u^{-2}\; du = u^{-1}+C = \dfrac{1}{3+\cos x}+C.$$\\

    %------------------ 11.
    \item $\displaystyle\int \dfrac{\sen x \; dx}{\sqrt{\cos^3 x}}$.\\\\

	Respuesta.-\; Sea,
	$$u=\cos x \quad \Rightarrow \quad du=-\sen x \; dx.$$
	Entonces,
	$$\begin{array}{rcl}
	    \displaystyle\int \sen x(\cos x)^{-\frac{3}{2}} \; dx &=& \displaystyle\int u^{-\frac{3}{2}}\; du\\\\
								  &=& 2u^{-\frac{1}{2}} +C\\\\
								  &=& \dfrac{2}{\sqrt{\cos x}}+C.
	\end{array}$$
	\vspace{.5cm}

    %------------------ 12.
    \item $\displaystyle\int_3^8 \dfrac{\sen \sqrt{x+1}\; dx}{\sqrt{x+1}}$.\\\\

	Respuesta.-\; Sea,
	$$u=\sqrt{x+1}\quad \Rightarrow \quad du=\dfrac{1}{2}(x+1)^{-\frac{1}{2}}\; dx.$$
	Por el teorema 1.4,
	$$u(3)=2\quad \mbox{y}\quad u(8)=3.$$
	Entonces,
	$$\begin{array}{rcl}
	    \displaystyle\int_3^8 \dfrac{\sen\left(\sqrt{x+1}\right)}{\sqrt{x+1}}\;dx &=& 2\displaystyle\int_2^3 \sen u\; du \\\\
										      &=& 2\left(-\cos u \bigg|_2^3\right)\\\\
										      &=& 2(\cos 2 - \cos 3).
	\end{array}$$
	\vspace{.5cm}

    %------------------ 13.
    \item $\displaystyle\int x^{n-1}\sen x^n \; dx$.\\\\

	Respuesta.-\; Sea,
	$$u=x^n \quad \Rightarrow \quad du=nx^{n-1}\; dx.$$
	Entonces,
	$$\begin{array}{rcl}
	    \displaystyle\int x^{n-1} \sen x^n \; dx &=& \dfrac{1}{n} \displaystyle\int \sen u \; du\\\\
						     &=& -\dfrac{1}{n} \cos u + C\\\\
						     &=& -\dfrac{1}{n} \cos x^n + C.
	\end{array}$$
	\vspace{.5cm}

    %------------------ 14.
    \item $\displaystyle\int  \dfrac{x^5\; dx}{\sqrt{1-x^6}}$.\\\\

	Respuesta.-\; Sea,
	$$u=1-x^6\quad \Rightarrow \quad du=-6x^5\; dx.$$
	Entonces,
	$$\begin{array}{rcl}
	    \displaystyle \int \dfrac{x^5}{\sqrt{1-x^6}}&=& -\dfrac{1}{6}\displaystyle\int u^{-\frac{1}{2}}\; du\\\\
						       &=&-\dfrac{1}{3}u^{\frac{1}{2}}+C\\\\
						       &=& -\dfrac{1}{3}\left(1-x^6\right)^{\frac{1}{2}} + C.
	\end{array}$$
	\vspace{.5cm}

    %------------------ 15.
    \item $\displaystyle\int t(1+t)^{\frac{1}{4}}\; dt$.\\\\

	Respuesta.-\; Sea,
	$$u=1+t\quad \Rightarrow \quad du=dt.$$
	Entonces,
	$$\begin{array}{rcl}
	    \displaystyle\int t(1+t)^{\frac{1}{4}}\; dt &=& \displaystyle\int (u-1)u^{\frac{1}{4}}\; du\\\\
							&=& \displaystyle\int u^{\frac{5}{4}}\; du - \int u^{\frac{1}{4}} \; du\\\\
							&=& \dfrac{4}{9}u^{\frac{9}{4}} - \dfrac{4}{5}u^{\frac{5}{4}} + C\\\\
							&=& \dfrac{4}{9}\left(1+t\right)^{\frac{9}{4}} - \dfrac{4}{5}\left(1+t\right)^{\frac{5}{4}} + C.
	\end{array}$$
	\vspace{.5cm}

    %------------------ 16.
    \item $\displaystyle\int (x^2+1)^{-\frac{3}{2}}\; dx$.\\\\

	Respuesta.-\; Sea, 
	$$u=\left(x^2+1\right)^{-\frac{1}{2}}\quad \Rightarrow \quad du=-x\left(x^2+1\right)^{-\frac{3}{2}}\; dx.$$
	De donde,
	$$x=\dfrac{\sqrt{1-u^2}}{u}.$$
	Entonces,
	$$\begin{array}{rcl}
	    \displaystyle\int (x^2+1)^{-\frac{3}{2}}\; dx &=& \displaystyle\int -\dfrac{1}{x}\; du\\\\
							  &=&\displaystyle\int \dfrac{-u}{\sqrt{1-u^2}}\; du.
	\end{array}$$
	Luego, sustituimos por segunda vez
	$$v=1-u^2\quad \Rightarrow \quad dv=-2u\; du$$
	Entonces,
	$$\begin{array}{rcl}
	    \displaystyle\int \dfrac{-u}{\sqrt{1-u^2}}\; du &=& \dfrac{1}{2}\displaystyle\int \dfrac{1}{\sqrt{v}}\; dv\\\\
							    &=&\sqrt{v}+C\\\\
							    &=&\sqrt{1-u^2}+C\\\\
							    &=&\sqrt{1-\dfrac{1}{x^2+1}}+C\\\\
							    &=&\dfrac{x}{\sqrt{x^2+1}}+C.
	\end{array}$$
	\vspace{.5cm}

    %------------------ 17.
    \item $\displaystyle\int x^2(8x^3+27)^{\frac{2}{3}} \; dx$.\\\\

	Respuesta.-\; Sea,
	$$u=8x^3+27\quad \Rightarrow \quad du=24x^2\; dx.$$
	Entonces,
	$$\begin{array}{rcl}
	    \displaystyle\int x^2\left(8x^3+27\right)\; dx &=&\dfrac{1}{24} \displaystyle\int u^{\dfrac{2}{3}}\; du\\\\
							   &=&\dfrac{1}{40}u^{\frac{5}{3}}+C\\\\
							   &=& \dfrac{1}{40}\left(8x^3+27\right)^{\frac{5}{3}}+C.
	\end{array}$$
	\vspace{.5cm}

    %------------------ 18.
    \item $\displaystyle\int \dfrac{(\sen x + \cos x)\; dx}{(\sen x - \cos x)^{\frac{1}{3}}} $.\\\\

	Respuesta.-\; Sea,
	$$u=\sen x - \cos x\quad \Rightarrow \quad du=\sen x + \cos x\; dx.$$
	Entonces,
	$$\begin{array}{rcl}
	    \displaystyle\int \dfrac{\sen x + \cos x}{(\sen x - \cos x)^{\frac{1}{3}}}\; dx &=& \displaystyle\int u^{-\frac{1}{3}}\; du\\\\
											    &=& \dfrac{3}{2}u^{\frac{2}{3}}+C\\\\
											    &=& \dfrac{3}{2}\left(\sen x - \cos x\right)^{\frac{2}{3}}+C.
	\end{array}$$
	\vspace{.5cm}

    %------------------ 19.
    \item $\displaystyle\int \dfrac{x\; dx}{\sqrt{1+x^2+\sqrt{(1+x^2)^3}}} \; dx$.\\\\

	Respuesta.-\; Simplifiquemos la integral como sigue,
	$$\begin{array}{rcl}
	    \displaystyle\int \dfrac{x\; dx}{\sqrt{1+x^2+\sqrt{(1+x^2)^3}}} &=& \displaystyle\int \dfrac{x\; dx}{\sqrt{\left(1+x^2\right)\left(1+\sqrt{1+x^2}\right)}}\\\\
									    &=& \displaystyle\int \dfrac{x\; dx}{\left(\sqrt{1+x^2}\right)\left(\sqrt{1+\sqrt{1+x^2}}\right)}.\\\\
	\end{array}$$
	Ahora, sea
	$$u=\sqrt{1+x^2}\quad \Rightarrow \quad du=\dfrac{x\; dx}{\sqrt{1+x^2}}.$$
	Entonces,
	$$\begin{array}{rcl}
	    \displaystyle\int \dfrac{x\; dx}{\left(\sqrt{1+x^2}\right)\left(\sqrt{1+\sqrt{1+x^2}}\right)} &=& \displaystyle\int u^{-\frac{1}{2}}\; du\\\\
														      &=&2u^{\frac{1}{2}} +C\\\\
																					      &=&2\sqrt{1+\sqrt{1+x^2}} + C.
	\end{array}$$



    %------------------ 20.
    \item $\displaystyle\int \dfrac{\left(x^2+1-2x\right)^{\frac{1}{5}}\; dx}{1-x} $.\\\\

	Respuesta.-\; Simplificando la integral, tenemos
	$$\begin{array}{rcl}
	    \displaystyle\int \dfrac{\left(x^2+1-2x\right)^{\frac{1}{5}}}{1+x}\; dx &=& \displaystyle\int \dfrac{(x-1)^{\frac{2}{5}}}{1-x}\; dx\\\\
										    &=&-\displaystyle\int (x-1)^{-\frac{3}{5}}\; dx.
	\end{array}$$
	Ahora, sea
	$$u=x-1\quad \Rightarrow \quad du=dx.$$
	Entonces,
	$$\begin{array}{rcl}
	    -\displaystyle\int (x-1)^{-\frac{3}{5}}\; dx &=& -\displaystyle\int u^{-\frac{3}{5}}\; du\\\\
	    										      &=&- \dfrac{5}{2}u^{\frac{2}{5}}+C\\\\
											      &=&- \dfrac{5}{2}\left(x-1\right)^{\frac{2}{5}}+C.
	\end{array}$$
	\vspace{.5cm}

    %------------------ 21.
    \item Deducir las fórmulas de los teoremas 1.18 y 1.19 por medio del método de sustitución.\\\\
	Demostración.-\; Deduzcamos el teorema 1.18 (invarianza bajo traslación). Para una función $f$ integrable en un intervalo $[a,b]$ y para todo $c\in \mathbb{R}$, tenemos
	$$\int_a^b f(x)\; dx = \int_{a+c}^{b+c}f(x-c)\; dx.$$
	Si $P$ es una primitiva de $f$, entonces
	$$\int_a^b f(x)\; dx = P(b)-P(a).$$
	Sea,
	$$u=g(x)=x-c\quad \Rightarrow \quad du=g'(x)\; dx = dx.$$
	Así,
	$$\begin{array}{rcl}
	    \displaystyle\int_{a+c}^{b+c} f(x-c)\; dx &=& \displaystyle\int_{a+c}^{b+c} f\left[g(x)\right]g'(x)\; dx\\\\
						      &=& P\left[g(b+c)\right]-P\left[g(a+c)\right]\\\\
						      &=& P(b)-P(a).
	\end{array}$$
	De donde se deduce que
	$$\int_a^b f(x)\; dx = \int_{a+c}^{b+c}f(x-c)\; dx.$$

	Ahora, deduzcamos el teorema 1.19 (Expansión o contracción del intervalo de integración). Para una función $f$ integrable en un intervalo $[a,b]$ y para todo $k\in \mathbb{R}$ distinto de $0$, se tiene
	$$\int_a^b f(x)=\dfrac{1}{k}\int_{ka}^{kb}f\left(\dfrac{x}{k}\right)\; dx.$$
	Sea,
	$$u=g(x)=\dfrac{x}{k}\quad \Rightarrow\quad du=g'(x)\; du=\dfrac{1}{k}\; dx.$$
	Entonces,
	$$\begin{array}{rcl}
	    \dfrac{1}{k}\displaystyle\int_{ka}^{kb} f\left(\dfrac{x}{k}\right)\; dx\\\\
	    &=& \displaystyle\int_{ka}^{kb} f\left[g(x)\right]g'(x)\; dx\\\\
	    &=&\displaystyle\int_{g(ka)}^{g(kb)} f(u)\; du\\\\
	    &=&P\left[g(b)\right]-P\left[g(a)\right]\\\\
	    &=& P(b)-P(a).
	\end{array}$$

	Así,

	$$\int_a^b f(x)=\dfrac{1}{k}\int_{ka}^{kb}f\left(\dfrac{x}{k}\right)\; dx.$$\\


    %------------------ 22.
    \item Sea
    $$F(x,a)=\int_0^x \dfrac{t^q}{(t^2+q^2)^q}\; dt,$$
    donde $a>0$ y $p$ y $q$ son enteros positivos. Demostrar que $F(x,a)=a^{p+1-2q}F(x/a,1)$.\\\\
	Respuesta.-\; Sea,
	$$u=\dfrac{t}{a}\quad \Rightarrow \quad du=\dfrac{1}{a}\; dt.$$
	Entonces,
	$$\begin{array}{rcl}
	    \displaystyle\int_0^x \dfrac{t^p}{(t^2+a^2)^q}\; dt &=&\displaystyle a\int_{u(0)}^{u(x)} \dfrac{u^p a^p}{(u^2a^2+a^2)^q}\; du\\\\
								&=& \displaystyle a^{p+1} \int_{u(0)}^{u(x)}\dfrac{u^p}{a^{2q}(u^2+1)^q}\; du\\\\
								&=&a^{p+1-2q}\displaystyle\int_0^{\frac{x}{a}} \dfrac{u^p}{(u^2+1)^q}\; du\\\\
								&=&a^{p+1-2q} F\left(\dfrac{x}{a},1\right).
	\end{array}$$
	\vspace{.5cm}

    %------------------ 23.
    \item Demostrar que
    $$\int_x^1\dfrac{dt}{1+t^2}=\int_1^{1/x} \dfrac{dt}{1+t^2}, \mbox{ si } x>0.$$\\
    	Demostración.-\; Sea 
	$$u=g(t)=\dfrac{1}{t},\qquad du=\dfrac{1}{t^2}\; dt.$$
	Lo que implica,
	$$t=\dfrac{1}{u},\qquad dt=-t^2\; du = -\dfrac{1}{u^2}\; du.$$
	Entonces,
	$$\begin{array}{rcl}
	    \displaystyle\int_x^1 \dfrac{dt}{1+t^2} &=&\displaystyle -\int_{g(x)}^{g(1)} \dfrac{\dfrac{1}{u^2}\; du}{1+\dfrac{1}{u^2}}\\\\
						    &=&\displaystyle\int_{g(1)}^{g(x)} \dfrac{\dfrac{1}{u^2}\; du}{1+\dfrac{1}{u^2}}\\\\
						    &=&\displaystyle\int_1^{\frac{1}{x}}\dfrac{du}{1+u^2}\\\\
						    &=&\int_1^{\frac{1}{x}} \dfrac{dt}{1+t^2}.
	\end{array}$$
	\vspace{.5cm}

	%------------------ 24.
	\item Demostrar que
	$$\int_0^1 x^m (1-x)^n\; dx = \int_0^1 x^n(1-x)^m\; dx.$$\\
	    Demostración.-\; Sea,
	    $$u=1-x\quad \Rightarrow \quad du=-dx.$$
	    Entonces,
	    $$\begin{array}{rcl}
		\displaystyle\int_0^1 x^m (1-x)^n\; dx &=&-\displaystyle\int_{u(0)}^{u(1)} (1-u)^mu^n\; du\\\\
						      &=&-\displaystyle\int_1^0 (1-u)^m u^n\; du\\\\
						      &=&\displaystyle\int_0^1 (1-u)^m u^n\; du\\\\
						      &=&\displaystyle\int_0^1 x^n (1-x)^m\; dx.
	    \end{array}$$
	    \vspace{.5cm}

	%------------------ 25.
	\item Demostrar que
	$$\int_0^{\pi/2}\cos^m x\sen^m x\; dx = 2^{-m}\int_0^{\pi/2}\cos^m x\; dx.$$
	si $m$ es un entero positivo.\\\\
	    Demostración.-\; Primero, simplifiquemos la integral,
	    $$\begin{array}{rcl}
		\displaystyle \int_0^{\frac{\pi}{2}} \cos^m x \sen^m x\; dx &=& \displaystyle\int_0^{\frac{\pi}{2}} (\cos x \sen x)^m \; dx\\\\
									    &=&\displaystyle\int_0^{\frac{\pi}{2}} \left[\dfrac{1}{2}\sen(2x)\right]^m\; dx\\\\
									    &=&\dfrac{1}{2^m}\displaystyle\int_0^{\frac{\pi}{2}} \sen^m(2x)\; dx.
	    \end{array}$$
	    Luego, usamos el método de sustitución 
	    $$u=2x,\qquad du=2\; dx.$$
	    Entonces,
	    $$\begin{array}{rcl}
		\dfrac{1}{2^m}\displaystyle\int_0^{\frac{\pi}{2}} \sen^m (2x)\; dx &=& \dfrac{1}{2^{m+1}} \displaystyle\int_{u(0)}^{u(\pi/2)} (\sen u)^m \; du\\\\
										   &=& \dfrac{1}{2^{m+1}}\displaystyle \int_0^\pi (\sen u)^m \; du\\\\
										   &=& \dfrac{1}{2^{m+1}} \displaystyle\int_{-\frac{\pi}{2}}^{\frac{\pi}{2}} \left[\sen\left(u+\dfrac{\pi}{2}\right)\right]^m\; du\\\\
										   &=&\dfrac{1}{2^{m+1}} \displaystyle\int_{-\frac{\pi}{2}}^{\frac{\pi}{2}} \cos^m u\; du\\\\
										   &=&\dfrac{1}{2^m}\displaystyle\int_0^{\frac{\pi}{2}}\cos^m x\; dx\\\\
										   &=&2^{-m}\displaystyle\int_0^{\frac{\pi}{2}}\cos^m x\; dx.
	    \end{array}$$
	    \vspace{.5cm}
	    \vspace{.5cm}


	%------------------ 26.
	\item 
	    \begin{enumerate}[(a)]

		%---------- (a)
		\item Demustrar que
		$$\int_0^\pi xf(\sen x)\; dx = \dfrac{\pi}{2}\int_0^\pi f(\sen x)\; dx.$$\\
		    Demostración.-\; Sea,
		    $$u=\pi-x \quad \Rightarrow \quad du=-dx.$$
		    Entonces,
		    $$\begin{array}{rcl}
			\displaystyle\int_0^\pi xf(\sen x)\; dx &=& \displaystyle - \int_{u(0)}^{u(\pi)}(\pi-u)f\left[\sen(\pi-u)\right]\; du\\\\
								&=&\displaystyle \int_\pi^0 (\pi-u)f(\sen u)\; du\\\\
								&=&\displaystyle \int_0^\pi (\pi-u)f(\sen u)\; du\\\\
								&=&\displaystyle \int_0^\pi \pi f(\sen u)\; du - \int_0^\pi uf(\sen u)\; du.
		    \end{array}$$
		    Aquí, cambiamos el nombre de la variable de integración de $u$ a $x$. (Siempre podemos cambiar el nombre de la variable de integración ya que integrar $f(t)\; dt$ es lo mismo que integrar $f(x)\; dx$, por ejemplo). Así que esto significa que tenemos
		    $$\begin{array}{rcl}
			\displaystyle\int_0^\pi xf(\sen x)\; dx &=& \displaystyle \int_0^\pi \pi f(\sen x)\; dx - \int_0^\pi xf(\sen x)\; dx\\\\
								&=&\displaystyle \int_0^\pi xf(\sen x)\; dx\\\\
								&=&\dfrac{\pi}{2} \displaystyle\int_0^\pi f(\sen x)\; dx.
		    \end{array}$$
		    \vspace{.5cm}

		%---------- (b)
		\item Aplicar (a) para deducir la fórmula:
		$$\int_0^\pi \dfrac{x\sen x}{1+\cos^2 x}\; dx=\pi \int_0^1 \dfrac{dx}{1+x^2}.$$\\
		    Demostración.-\; Usemos primero, la identidad trigonométrica $\cos^2 x+\sen^2 x = 1,$ lo que implica $1+\cos^2x=2-\sen^2x,$ para reescribir la integral
		    $$\begin{array}{rcl}
			\displaystyle\int_0^\pi \dfrac{x\sen x}{1+\cos^2 x}\; dx &=& \displaystyle \int_0^\pi \dfrac{x\sen x}{2-\sen^2 x}\; dx\\\\
										 &=&\displaystyle\int_0^\pi xf(\sen x)\; dx \;\mbox{ donde }\; f(x)=\dfrac{2}{2-x^2}
		    \end{array}$$
		    Entonces, por (a) tenemos
		    $$\begin{array}{rcl}
			\displaystyle \int_0^\pi xf(\sen x)\; dx &=& \dfrac{\pi}{2} \displaystyle\int_0^\pi f(\sen x)\; dx\\\\
								 &=& \dfrac{\pi}{2} \displaystyle\int_0^\pi \dfrac{\sen x}{2-\sen^2 x}\; dx\\\\
								 &=& \dfrac{\pi}{2} \displaystyle\int_0^\pi \dfrac{\sen x}{1+\cos^2 x}\; dx.
		    \end{array}$$
		    Luego, sea
		    $$u=\cos x,\qquad du=-\sen x\; dx.$$
		    Así,
		    $$\begin{array}{rcl}
			\dfrac{\pi}{2}\displaystyle\int_0^\pi \dfrac{\sen x}{1+\cos^2 x}\; dx = -\dfrac{\pi}{2}\displaystyle\int_1^{-1}\dfrac{1}{1+u^2}\; du\\\\
			&=&\dfrac{\pi}{2}\displaystyle\int_{-1}^1 \dfrac{1}{1+x^2}\; dx\\\\
			&=&\pi \displaystyle\int_0^1 \dfrac{1}{1+x^2}\; dx.
		    \end{array}$$
	    \end{enumerate}

	%------------------ 27.
	\item Demostrar que $\int_0^1 (1-x^2)^{n-1/2}\; dx = \int_0^{\pi/2}\cos^{2n}u\; du$ si $n$ es un entero positivo. [Indicación: $x=\sen u.$] La integral del segundo miembro se puede calcular por el método de integración por partes.\\\\
	    Demostración.-\; Sea,
	    $$x=\sen x,\qquad dx=\cos u\; du.$$
	    Entonces,
	    $$\begin{array}{rcl}
		\displaystyle\int_0^1 \left(1-x^2\right)^{n-\frac{1}{2}}\; dx &=& \displaystyle\int_0^{\frac{\pi}{2}} \left(1-\sen^2 u\right)^{n-\frac{1}{2}} \cos u\; du\\\\
									      &=&\displaystyle \int_0^{\frac{\pi}{2}}\left(\cos^2 u\right)^{n-\frac{1}{2}} \cos u \; du\\\\
									      &=&\displaystyle\int_0^{\frac{\pi}{2}} \left(\cos^{2n-1}u\right)\cos u\; \; du\\\\
									      &=&\displaystyle\int_0^{\frac{\pi}{2}} \cos^{2n}u\; du.
	    \end{array}$$
	    La última igualdad sigue puesto $\frac{1}{1+(-x)^2} = \frac{1}{1+x^2}$ de modo que esta es una función par. Por lo tanto (por un ejercicio anterior ) la integral de $-1$ a $1$ es el doble de la integral de $0$ a $1$.\\\\

\end{enumerate}


\section{Integración por partes}

Se demostró en el capitulo 4 que la derivada de un producto de dos funciones $f$ y $g$ está dada por la fórmula:
$$h'(x)=f(x)g'(x)+f'(x)g(x),$$
donde $h(x)=f(x)\cdot g(x).$ Traduciendo esto a la notación de Leibiz para primitivas se tiene $\int f(x)g'(x)\; dx + \int f'(x)g(x)\; dx=f(x)g(x)+C$, que se escribe usualmente en la forma
$$\int f(x)g'(x)\; dx = f(x)g(x)-\int f'(x)g(x)\; dx + C.$$
Esta igualdad, conocida por fórmula de integración por partes, da lugar a una nueva técnica de integración.\\

En el caso de integrales definidas, la fórmula anterior se transforma en

$$\int_a^b  f(x)g'(x)\; dx = f(b)g(b)-f(a)g(a)-\int_a^b f'(x)g(x)\; dx.$$
Poniendo $u=f(x)$ y $v=g(x)$ se tiene $du=f'(x)\; dx$ y $dv=g'(x)\; dx$ y la fórmula de integración por partes toma una forma abreviada que parece más fácil de recordar:
$$\int u\; dv = uv-\int v\; du +C.$$

%------------------  Ejemplo 3
\begin{ejem}
    Algunas veces el método falla porque conduce de nuevo a la integral original. Por ejemplo, al intentar calcular por partes la integral $\int x^{-1}\; dx$. Si se hace $u = x$ y $dv = x^{-2}\;dx$, entonces $\int x^{-1}\; dx = \int u\; dv$. Con esta elección de $u$ y $v$ se tiene $du = dx$ y $v = - x^{-1}$ de manera que (5.24) da:
    $$\int x^{-1}\;dx = \int u \; dv = uv-\int c \; du + C = -1 +\int x^{-1}\; dx + C,$$
    y se vuelve al punto de partida. Por otra parte, la situación no mejora si se intenta $u = x^n$ y $dv = x^{-n-1}\; dx.$\\
    Este ejemplo se usa con frecuencia para evidenciar la importancia de no olvidar la constante arbitraria $C$. Si en la fórmula anterior no se hubiera escrito la $C$, se hubiera llegado a la ecuación $\int x^{-1}\; dx = - 1 + \int x^{-1}\; dx$ que se utiliza algunas veces para dar una aparente demostración de que $O = - 1$.
\end{ejem}

Como aplicación del método de integración por partes, se obtiene otra versión del teorema del valor medio ponderado para integrales (teorema 3.16).

%------------------  Teorema 5.5
\begin{teo}[Segundo teorema del valor medio para integrales]
    Supongamos que $g$ es continua en $[a,b]$, y que $f$ tiene derivada continua y que nunca cambia de signo en $[a,b]$. Entonces, para un cierto $c$ de $[a,b]$, tenemos
    $$\int_a^b f(x)g(x)\; dx = f(a)\int_a^c g(x)\; dx + f(b)\int_c^b g(x)\; dx.$$\\
	Demostración.-\; Sea $G(x)=\int_a^x g(t)\; dt$. Como que $g$ es continua, tenemos $G'(x)=g(x)$. Por consiguiente, la integración por partes nos da
	$$\int_a^b f(x)g(x)\; dx = \int_a^b f(x)G'(x)\; dx = f(b)G(b)-\int_a^b f'(x)G(x)\; dx,$$
	puesto que $G(a)=0$ (hipótesis). Según el teorema del valor medio ponderado, se tiene
	$$\int_a^b f'(x)G(x)\; dx = G(c)\int_a^b f'(x)\; dx = G(c)[f(b)-f(a)]$$
	para un cierto $c$ en $[a,b]$. Por consiguiente $\int_a^b f(x)g(x)\; dx = \int_a^b f(x)G'(x)\; dx = f(b)G(b)-\int_a^b f'(x)G(x)\; dx$, se convierte en 
	$$\int_a^b f(x)g(x)\; dx = f(b)G(b)-G(c)[f(b)-f(a)]=f(a)G(c)+f(b)[G(b)-G(c)].$$
	Esto demuestra $\int_a^b f(x)g(x)\; dx = f(a)\int_a^c g(x)\; dx + f(b)\int_c^b g(x)\; dx$ ya que $G(c)=\int_a^c g(x)\; dx$ y $G(b)-G(c)=\int_c^b g(x)\; dx.$
\end{teo}


\section{ejercicios}
Con el método de integración por partes calcular las integrales de los Ejercicios 1 al 6.\\

    \begin{enumerate}[\bfseries 1.]

	%------------------ 1.
	\item $\displaystyle\int x\sen x \; dx.$\\\\
	    Respuesta.-\; Aplicando la fórmula de integración por partes, se tiene
	    $$\begin{array}{rcl} 
		u=x &\Rightarrow&du=dx\\\\
		dv=\sen x\; dx &\Rightarrow& v=-\cos x.
	    \end{array}$$
	    Entonces,
	    $$\begin{array}{rcl}
		\displaystyle\int x\sen x\; dx &=& \displaystyle\int u\; dv\\\\
					       &=& uv - \displaystyle\int v\; du\\\\
					       &=& -x\cos x + \displaystyle\int \cos x\; dx\\\\
					       &=& -x\cos x + \sin x + C.
	    \end{array}$$
	    \vspace{0.5cm}

	%------------------ 2.
	\item $\displaystyle\int x^2\sen x \; dx$.\\\\
	    Respuesta.-\; Aplicando la fórmula de integración por partes, se tiene
	    $$\begin{array}{rcl}
		u=x^2 &\Rightarrow& du=2x\; dx\\\\
		dv=\sen x\; dx &\Rightarrow& v=-\cos x.
	    \end{array}$$
	    Entonces,
	    $$\begin{array}{rcl}
		\displaystyle\int x^2\sen x\; dx &=& \displaystyle\int u\; dv\\\\
					       &=& uv - \displaystyle\int v\; du\\\\
					       &=& -x^2\cos x + 2\displaystyle\int x \cos x\; dx.
	    \end{array}$$
	    Volviendo a utilizar la fórmula de integración por partes,
	    $$\begin{array}{rcl}
		u=x &\Rightarrow& du=dx\\\\
		dv=\cos x\; dx &\Rightarrow& v=\sin x.
	    \end{array}$$
	    Entonces,
	    $$\begin{array}{rcl}
		\displaystyle\int x\cos x\; dx &=& \displaystyle\int u\; dv\\\\
					       &=& uv - \displaystyle\int v\; du\\\\
					       &=& x\sin x - \displaystyle\int \sin x\; dx\\\\
					       &=& x\sin x + \cos x + C.
	    \end{array}$$
	    \vspace{0.5cm}

	%------------------ 3.
	\item $\displaystyle\int x^3\cos x\; dx$.\\\\
	    Respuesta.-\; Aplicando la fórmula de integración por partes, se tiene
	    $$\begin{array}{rcl}
		u=x^3 &\Rightarrow& du=3x^2\; dx\\\\
		dv=\cos x\; dx &\Rightarrow& v=\sin x.
	    \end{array}$$
	    Entonces,
	    $$\begin{array}{rcl}
		\displaystyle\int x^3\cos x\; dx &=& \displaystyle\int u\; dv\\\\
					       &=& uv - \displaystyle\int v\; du\\\\
					       &=& x^3\sin x - 3\displaystyle\int x^2\sin x\; dx.
	    \end{array}$$
	    Por el anterior ejercicio 2,
	    $$\int x^2\sen \; dx = 2x\sen x + 2\cos x -x^2\cos x +C.$$
	    Así, tenemos
	    $$\begin{array}{rcl}
		\displaystyle\int x^3\cos x\; dx &=& x^3\sen x - \displaystyle\int x^2\sen x\; dx\\\\
						 &=&x^3\sen x - 3(2x\sen x + 2\cos x -x^2\cos x)+C\\\\
						 &=& x^3\sen x - 6x\sen x - 6\cos x + 3x^2\cos x + C.
	    \end{array}$$
	    \vspace{0.5cm}

	%------------------ 4.
	\item $\displaystyle\int x^3 \sen x\; dx$.\\\\
	    Respuesta.-\; Aplicando la fórmula de integración por partes, se tiene
	    $$\begin{array}{rcl}
		u=x^3 &\Rightarrow& du=3x^2\; dx\\\\
		dv=\sen x\; dx &\Rightarrow& v=-\cos x.
	    \end{array}$$
	    Entonces,
	    $$\begin{array}{rcl}
		\displaystyle\int x^3\sen x\; dx &=& \displaystyle\int u\; dv\\\\
					       &=& uv - \displaystyle\int v\; du\\\\
					       &=& -x^3\cos x + 3\displaystyle\int x^2\cos x\; dx.
	    \end{array}$$
	    Aplicando la fórmula de integración por partes una vez más,
	    $$\begin{array}{rcl}
		u=x^2 &\Rightarrow& du=2x\; dx\\\\
		dv=\cos x\; dx &\Rightarrow& v=\sin x.
	    \end{array}$$
	    Entonces,
	    $$\begin{array}{rcl}
		\displaystyle\int x^2\cos x\; dx &=& \displaystyle\int u\; dv\\\\
					       &=& uv - \displaystyle\int v\; du\\\\
					       &=& x^2\sin x - 2\displaystyle\int x \sin x\; dx.
	    \end{array}$$
	    Luego, por el ejercicio 1,
	    $$\int x\sen x \; dx = \sen x - x\cos x + C,$$
	    se tiene,
	    $$\begin{array}{rcl}
		\displaystyle\int x^2\cos x\; dx &=& x^2\sin x - 2\displaystyle\int x\sen x\; dx\\\\
						 &=& x^2\sin x - 2(\sen x - x\cos x) + C\\\\
						 &=& x^2\sin x - 2\sen x + 2x\cos x + C.
	    \end{array}$$
	    Por lo tanto,
	    $$\begin{array}{rcl}
		\displaystyle\int x^3\sen x\; dx &=& -x^3\cos x + 3\displaystyle\int x^2\cos x\; dx\\\\
						 &=& -x^3\cos x + 3(x^2\sin x - 2\sen x + 2x\cos x) + C\\\\
						 &=& -x^3\cos x + 3x^2\sin x - 6\sen x + 6x\cos x + C.
	    \end{array}$$
	    \vspace{0.5cm}

	%------------------ 5.
	\item $\displaystyle\int \sen x\cos x\; dx$.\\\\
	    Respuesta.-\; Aplicando la fórmula de integración por partes, se tiene
	    $$\begin{array}{rcl}
		u=\sen x &\Rightarrow& du=\cos x\; dx\\\\
		dv=\cos x\; dx &\Rightarrow& v=-\sen x.
	    \end{array}$$
	    Entonces,
	    $$\begin{array}{rcl}
		\displaystyle\int \sen x\cos x\; dx &=& \displaystyle\int u\; dv\\\\
					       &=& uv - \displaystyle\int v\; du\\\\
					       &=& \sen^2 x - \displaystyle\int \sen x \cos x\; dx.
	    \end{array}$$
	    De donde,
	    $$\begin{array}{rcl}
		2\displaystyle\int \sen x\cos x\; dx &=& \sen^2 x +C \\\\
		\displaystyle\int \sen x\cos x\; dx &=& \frac{\sen^2 x+C}{2}.
	    \end{array}$$
	    \vspace{0.5cm}

	%------------------ 6.
	\item $\displaystyle\int x\sen x \cos x\; dx$.\\\\
	    Respuesta.-\; Por las propiedades trigonométricas se tiene,
	    $$\sen x \cos x = \dfrac{1}{2}\sen 2x,$$
	    por lo que evaluaremos 
	    $$\dfrac{1}{2}\int x\sen 2x\; dx.$$
	    Aplicando la fórmula de integración por partes, se tiene
	    $$\begin{array}{rcl}
		u=x &\Rightarrow& du=dx\\\\
		dv=\sen 2x\; dx &\Rightarrow& v=-\dfrac{1}{2}\cos 2x.
	    \end{array}$$
	    Entonces,
	    $$\begin{array}{rcl}
		\displaystyle\int x\sen 2x\; dx &=& \dfrac{1}{2}\left[x\left(-\dfrac{1}{2}\cos 2x\right)\right]- \dfrac{1}{2}\displaystyle\int \left(-\dfrac{1}{2}\cos 2 x\right)\; dx+C\\\\
					       &=& -\dfrac{1}{4}x\cos 2x - \dfrac{1}{4}\displaystyle\int \cos 2x\; dx+C\\\\
					       &=& -\dfrac{1}{4}x\cos 2x - \dfrac{1}{8}\sen 2x + C.
	    \end{array}$$
	    \vspace{0.5cm}

	%------------------ 7.
	\item Con la integración por partes deducir la fórmula
	$$\int \sen^2 x\; dx = -\sen x \cos x + \int \cos^2 x\; dx.$$
	En la segunda integral, poner $\cos^2 x = 1-\sen^2 x$ y así deducir la fórmula
	$$\int \sen^2 x \; dx = \dfrac{1}{2}-\dfrac{1}{4}\sen 2x.$$\\
	    Respuesta.-\; Primero, usamos la fórmula de integración por partes,
	    $$\begin{array}{rcl}
		u=\sen x &\Rightarrow& du=\cos x\; dx\\\\
		dv=\sen x\; dx &\Rightarrow& v=-\cos x.
	    \end{array}$$
	    Entonces,
	    $$\begin{array}{rcl}
		\displaystyle\int \sen^2 x\; dx &=& \displaystyle\int u\; dv\\\\
					       &=& uv - \displaystyle\int v\; du\\\\
					       &=& -\sen x \cos x + \displaystyle\int \cos^2 x\; dx\\\\
					       &=& -\sen x \cos x + \displaystyle\int (1-\sen^2 x)\; dx\\\\
					       &=& -\sen x \cos x + \displaystyle\int 1\; dx - \int \sen^2 x\; dx\\\\

	    \end{array}$$
	    De donde,
	    $$\begin{array}{rcl}
		2\displaystyle\int \sen^2 x\; dx &=& -\sen x \cos x + x + C\\\\
		\displaystyle\int \sen^2 x\; dx &=& \dfrac{1}{2}-\dfrac{1}{4}\sen(2x)+C.
	    \end{array}$$
	    \vspace{0.5cm}

	%------------------ 8.
	\item Integrando por partes deducir la fórmula
	$$\int \sen^n x\; dx = -\sen^{n-1}x\cos x + (n-1)\int \sen^{n-2}x\cos^2 x\; dx.$$
	En la segunda integral, poner $\cos^2 x= 1-\sen^2 x$ y con eso deducir la fórmula recurrente
	$$\int \sen^n x\; dx = -\dfrac{\sen^{n-1}x\cos x}{n}+\dfrac{n-1}{n}\int \sen^{n-2}x\; dx.$$\\
	    Demostración.-\; Usando la fórmula de integración por partes,
	    $$\begin{array}{rcl}
		u=\sen^{n-1} x &\Rightarrow& du=(n-1)\cos^{n-2} x\; dx\\\\
		dv=\sen x\; dx &\Rightarrow& v=-\cos x.
	    \end{array}$$
	    Entonces, 
	    $$\begin{array}{rcl}
		\displaystyle\int \sen^n x\; dx &=&u\; dx\\\\
						&=&uv- \displaystyle\int v\; du\\\\
						&=& -\sen^{n-1}x\cos x + (n-1)\displaystyle\int \sen^{n-2}x\cos^2 x\; dx\\\\
	    \end{array}$$
	    Luego, usando la identidad $\cos^2x=1-\sen^2 x$, tenemos
	    $$\begin{array}{rcl}
		\displaystyle\int \sen^n x\; dx &=& -\sen^{n-1}x\cos x - (n-1)\displaystyle\int \sen^{n-2}x\left(1-\sen^2 x\right)\; dx\\\\
		\displaystyle\int \sen^n x\; dx &=& -\sen^{n-1}x\cos x + (n-1)\displaystyle\int \sen^{n-2}x\; dx - (n-1)\displaystyle\int \sen^{n}x\; dx\\\\
		n\displaystyle\int \sen^n x\; dx &=& -\sen^{n-1}x\cos x + (n-1)\displaystyle\int \sen^{n-2}x\; dx\\\\
		\displaystyle\int \sen^n x\; dx &=& -\dfrac{\sen^{n-1}x\cos x}{n}+\dfrac{n-1}{n}\displaystyle\int \sen^{n-2}x\; dx.
	    \end{array}$$
	    \vspace{0.5cm}

	%------------------ 9.
	\item Con los resultados de los Ejercicios 7 y 8 demostrar que
	    \begin{enumerate}[(a)]

		%---------- (a)
		\item $\displaystyle\int_0^{\pi/2}\sen^2 x\; dx = \dfrac{\pi}{4}.$\\\\
		    Demostración.-\; Usando la fórmula del Ejercicio 7,
		    $$
			\begin{array}{rcl}
			    \displaystyle\int_0^{\frac{\pi}{2}} \sen^2 x\; dx &=& \left[\dfrac{1}{2}x-\dfrac{1}{4}\sen(2x)\right]\bigg|_0^{\frac{\pi}{2}}\\\\
									      &=&\left(\dfrac{\pi}{4}-0\right)-0\\\\
									      &=&\dfrac{\pi}{4}.

			\end{array}
		    $$
		    \vspace{0.5cm}

		%---------- (b)
		\item $\displaystyle\int_0^{\pi/2} \sen^4 x\; dx = \dfrac{3}{4}\int_{0}^{\pi/2}\sen^2x\; dx = \dfrac{3\pi}{16}.$\\\\
		    Demostración.-\; Usando la fórmula del Ejercicio 8 (b), tenemos
		    $$
			\begin{array}{rcl}
			    \displaystyle\int_0^{\pi} \sen^4 x\; dx &=& -\dfrac{\sen^3x\cos x}{4}\bigg|_0^{\frac{\pi}{2}} + \dfrac{3}{4}\displaystyle\int_0^{\frac{\pi}{2}} \sen^2 x\; dx\\\\
								    &=& 0+\dfrac{3}{4}\left(\dfrac{\pi}{4}\right)\\\\
								    &=& \dfrac{3\pi}{16}.
			\end{array}
		    $$
		    \vspace{0.5cm}

		%---------- (c)
		\item $\displaystyle\int_0^{\pi/2} \sen^6 x\; dx = \dfrac{5}{6}\int_0^{\pi/2} \sen^4x\; dx = \dfrac{5\pi}{32}$.\\\\	
		    Demostración.-\;  Usando la fórmula del Ejercicio 8 (b), tenemos
		    $$
		    \begin{array}{rcl}
			\displaystyle\int_0^{\frac{\pi}{2}} \sen^6 x\; dx &=& \displaystyle -\dfrac{\sen^5x\cos x}{6}\bigg|_0^{\frac{\pi}{2}} + \dfrac{5}{6}\int_0^{\frac{\pi}{2}} \sen^4 x\; dx\\\\
									  &=& 0 + \dfrac{5}{6}\left(\dfrac{3\pi}{16}\right)\\\\
									  &=& \dfrac{5\pi}{32}.
		    \end{array}
		    $$
		    \vspace{0.5cm}

	    \end{enumerate}

    %------------------ 10.
    \item Con los resultados de los ejercicios 7 y 8 deducir las siguientes fórmulas.\\
	\begin{enumerate}[(a)]

	    %---------- (a)
	    \item $\displaystyle\int \sen^3 x\; dx = -\dfrac{3}{4}\cos x + \dfrac{1}{12}\cos 3x.$\\\\
		Respuesta.-\; Por el ejercicio 8 y dado $\cos(3x)=4\cos^3-3\cos x$, que implica $\cos^3x=\dfrac{1}{4}(3x)+\dfrac{3}{4}\cos x$, tenemos
		$$
		\begin{array}{rcl}
		    \displaystyle\int \sen^3 x\; dx &=& -\dfrac{\sen^2x\cos x}{3}+\dfrac{2}{3}\displaystyle\int \sen x\; dx\\\\
						    &=& -\dfrac{1}{3}\left(\sen^2\cos x + 2\cos x\right)\\\\
						    &=& -\dfrac{1}{3}\left[\left(1-\cos^2 x\right)\cos x+2\cos x\right]\\\\
						    &=& - \dfrac{1}{3}\cos x\left(3-\cos^2 x\right)\\\\
						    &=&-\cos x + \dfrac{1}{3}\cos^3 x\\\\
						    &=&-\cos x + \dfrac{1}{3}\left[\dfrac{1}{4}\cos (3x)+\dfrac{3}{4}\cos x\right]\\\\
						    &=& -\dfrac{3}{4}\cos x + \dfrac{1}{12}\cos (3x).
		\end{array}
		$$
		\vspace{.5cm}

	    %---------- (b)
	    \item $\displaystyle\int \sen^4 x\; dx = \dfrac{3}{8}x-\dfrac{1}{4}\sen 2x + \dfrac{1}{32}\sen 4x.$\\\\
		Respuesta.-\; Usando el ejercicio 8 y el ejercicio 7, y dado $\cos(2x)=1-2\sen^2 x$ que implica $\sen^2x=\dfrac{1}{2}-\dfrac{1}{2}\cos(2x)$, se tiene
		$$
		\begin{array}{rcl}
		    \displaystyle\int \sen^4 x\; dx &=& -\dfrac{\sen^3x\cos x}{4} + \dfrac{3}{4}\displaystyle \int \sen^2x\; dx\\\\
						    &=& -\dfrac{1}{4}\left(\sen^3 x \cos x\right) + \dfrac{3}{4}\left[\dfrac{1}{2}x-\dfrac{1}{4}\sen (2x)\right]\\\\
						    &=& -\dfrac{1}{8}\left[\sen^2 x \sen(2x)\right] + \dfrac{3}{8}x-\dfrac{3}{16} \sen(2x)\\\\
						    &=& -\dfrac{1}{16}\left[\sen (2x)-\sen (2x)\cos (2x)\right]+\dfrac{3}{8}x-\dfrac{3}{16}\sen(2x)\\\\
						    &=& \dfrac{3}{8}-\dfrac{1}{4}\sen(2x)+\dfrac{1}{32}\sen(4x).
		\end{array}
		$$
		\vspace{.5cm}

	    %---------- (c)
	    \item $\displaystyle\int \sen^5 x\; dx = -\dfrac{5}{8}x+\dfrac{5}{48}\cos 3x - \dfrac{1}{80}\cos 5x.$\\\\
		Respuesta.-\; Usando el ejercicio 8 y luego la parte (a) de este ejercicio, tenemos
		$$
		\begin{array}{rcl}
		    \displaystyle\int \sen^5x\; dx &=& -\dfrac{\sen^4x\cos x}{5} + \dfrac{4}{5}\displaystyle\int \sen^3x\; dx\\\\
						   &=& - \dfrac{1}{5}\left[\left(1-\cos^2 x\right)\left(1-\cos^2 x\right)\cos x\right] +\dfrac{4}{5}\left[-\dfrac{3}{4}\cos x + \dfrac{1}{12}\cos(3x)\right]\\\\
						   &=& -\dfrac{1}{5}\left(\cos x - 2\cos^3x+\cos^5x\right)+\dfrac{1}{16}\left[10\cos x + 5\cos(3x)+\cos(5x)\right]\\\\
						   &-&\dfrac{3}{5}\cos x + \dfrac{1}{15}\cos (3x)\\\\
						   &=&-\dfrac{1}{5}\left[\dfrac{1}{8}\cos x - \dfrac{3}{16}\cos (3x)+\dfrac{1}{16}\cos (5x)\right]-\dfrac{3}{5}\cos x + \dfrac{1}{15}\cos(3x)\\\\
						   &=& -\dfrac{5}{8}\cos x + \dfrac{5}{48}\cos (3x) - \dfrac{1}{80}\cos (5x).
		\end{array}
		$$
		\vspace{.5cm}

	\end{enumerate}

    %------------------ 11.
    \item Con la integración por partes y los resultados de los ejercicios 7 y 10 deducir las siguientes formulas:

	\begin{enumerate}[(a)]

	    %---------- (a)
	    \item $\displaystyle\int x\sen^2 x\; dx = \dfrac{1}{4}x^2-\dfrac{1}{4}x^2-\dfrac{1}{4}x\sen(2x)-\dfrac{1}{8}\cos (2x)$.\\\\
		Respuesta.-\; Aplicando la integración por partes, tenemos
		$$
		\begin{array}{rcl}
		    u=x&\Rightarrow&du=dx\\\\
		    dv=\sen^2 x\; dx &\Rightarrow&v=\dfrac{x}{2}-\dfrac{\sen(2x)}{4}.\\\\
		\end{array}
		$$
		Usando el ejercicio 7 para la primitiva de $\sen^2 x$, se tiene
		$$
		\begin{array}{rcl}
		    \displaystyle\int x\sen^2 x\; dx&=&\displaystyle\int u\; dv\\\\
				    &=&\displaystyle\int uv - \int v\; du\\\\
				    &=&\dfrac{1}{2}x^2-\dfrac{1}{4}x\sen(2x)-\displaystyle\int\left[\dfrac{x}{2}-\dfrac{\sen(2x)}{2}\right]\; dx\\\\
				    &=&\dfrac{1}{2}x^2-\dfrac{1}{4}x\sen(2x)-\dfrac{1}{8}\cos(2x).
		\end{array}
		$$
		\vspace{.5cm}

	    %---------- (b)
	    \item $\displaystyle\int x\sen^3 x\; dx = \dfrac{3}{4}\sen x - \dfrac{1}{36}\sen(3x)-\dfrac{3}{4}x\cos x + \dfrac{1}{12}x\cos (3x)$.\\\\
		Respuesta.-\; Aplicando la integración por partes, tenemos
		$$
		\begin{array}{rcl}
		    u=x&\Rightarrow&du=dx\\\\
		    dv=\sen^3 x\; dx &\Rightarrow&v=-\dfrac{3}{4}\cos x + \dfrac{1}{12}\cos(3x).
		\end{array}
		$$
		Aplicando el ejercicio 10 para la primitiva de $\sen^3 x$, se tiene
		$$
		\begin{array}{rcl}
		    \displaystyle\int x\sen^3 x\; dx&=&\displaystyle\int u\; dv\\\\
				    &=&\displaystyle\int uv - \int v\; du\\\\
				    &=&-\dfrac{3}{4}x\cos x + \dfrac{1}{12}x\cos (3x)-\displaystyle\int\left[-\dfrac{3}{4}\cos x + \dfrac{1}{12}\cos(3x)\right]\; dx\\\\
				    &=&-\dfrac{3}{4}x\cos x + \dfrac{1}{12}x\cos (3x)+\dfrac{3}{4}\sen x - \dfrac{1}{36}\sen(3x).
		\end{array}
		$$
		\vspace{.5cm}

	    %---------- (c)
	    \item $\displaystyle\int x^2\sen^2 x \; dx = \dfrac{1}{6}x^3 + \left(\dfrac{1}{8}-\dfrac{1}{4}x^2\right)\sen 2x - \dfrac{1}{4} x\cos (2x)$.\\\\
		Respuesta.-\; Aplicando la integración por partes, tenemos
		$$
		\begin{array}{rcl}
		    u=x&\Rightarrow&du=dx\\\\
		    dv=x\sen^2 x\; dx &\Rightarrow&v=\dfrac{1}{4}x^2-\dfrac{1}{4}x\sen(2x)-\dfrac{1}{8}\cos(2x).
		\end{array}
		$$
		Acplicando la parte (a) derivando la primitiva de $x\sen^2 x$, se tiene
		$$
		\begin{array}{rcl}
		    \displaystyle\int x^2\sen^2 x\; dx&=&\displaystyle\int u\; dv\\\\
				    &=&\displaystyle\int uv - \int v\; du\\\\
				    &=&\dfrac{1}{4}x^3 - \dfrac{1}{4}x^2 \sen(2x)-\dfrac{1}{8}x\cos(2x)-\displaystyle\int \left[\dfrac{1}{4}x^2-\dfrac{1}{4}x\sen(2x)-\dfrac{1}{8}(2x)\right]\; dx\\\\
				    &=&\dfrac{1}{4}x^3-\dfrac{1}{4}x^2 \sen(2x)-\dfrac{1}{8}x\cos(2x)-\dfrac{1}{12}x^3+\dfrac{1}{4}\displaystyle\int x\sen(2x)\; dx+\dfrac{1}{16}\sen(2x).
		\end{array}
		$$
		Luego, necesitamos usar nuevamente la integración por partes para $\int x\sen(2x)\; dx$, como sigue
		$$
		\begin{array}{rcl}
		    u=x&\Rightarrow&du=dx\\\\
		    dv=\sen(2x)\; dx &\Rightarrow&v=-\dfrac{1}{2}\cos(2x).
		\end{array}
		$$
		Después, integrando por partes, tenemos
		$$
		\begin{array}{rcl}
		    \displaystyle\int x\sen(2x)\; dx&=&\displaystyle\int u\; dv\\\\
				    &=&\displaystyle\int uv - \int v\; du\\\\
				    &=&-\dfrac{1}{2}x\cos(2x)-\displaystyle\int\left[-\dfrac{1}{2}\cos(2x)\right]\; dx\\\\
				    &=&-\dfrac{1}{2}x\cos(2x)+\dfrac{1}{4}\sen(2x).
		\end{array}
		$$
		Así, la integral queda
		$$
		\begin{array}{rcl}
		    \displaystyle\int x^2\sen^2x\; dx &=& \dfrac{1}{4}x^3-\dfrac{1}{4}x^2 \sen(2x)-\dfrac{1}{8}x\cos(2x)-\dfrac{1}{12}x^3+\dfrac{1}{4}\displaystyle\int x\sen(2x)\; dx +\dfrac{1}{16}\sen(2x)\\\\
		    &=& \dfrac{1}{4}x^3-\dfrac{1}{4}x^2 \sen(2x)-\dfrac{1}{8}x\cos(2x)-\dfrac{1}{12}x^3+\dfrac{1}{4}\left[-\dfrac{1}{2}x\cos(2x)+\dfrac{1}{4}\sen(2x)\right]\\\\
		    &+&\dfrac{1}{16}\sen(2x)\\\\
		    &=& \dfrac{1}{4}x^3-\dfrac{1}{4}x^2\sen(2x)-\dfrac{1}{8}x\cos(2x)-\dfrac{1}{12}x^3-\dfrac{1}{8}x\cos(2x)+\dfrac{1}{16}\sen(2x)\\\\
		    &+&\dfrac{1}{16}\sen(2x)\\\\
		    &=&\left(\dfrac{1}{4}-\dfrac{1}{12}\right)x^3 - \dfrac{1}{4}x^2\sen(2x)-\dfrac{1}{4}x\cos(2x)+\dfrac{1}{8}\sen(2x)\\\\
		    &=&\dfrac{1}{6}x^3+\left(\dfrac{1}{8}-\dfrac{1}{4}x^2\right)\sen(2x)-\dfrac{1}{4}x\cos(2x).
		\end{array}
		$$
		\vspace{.5cm}

	\end{enumerate}

    %-------------------- 12
    \item Integrando por partes deducir la fórmula recurrente
    $$\int \cos^n x\; dx = \dfrac{\cos^{n-1}\sen x}{n}+\dfrac{n-1}{n}\int \cos^{n-2}x\; dx.$$\\
	Respuesta.-\; Aplicando la integración por partes, tenemos
	$$
	\begin{array}{rcl}
	    u=\cos^{n-1} x&\Rightarrow&du=-(n-1)\cos^{n-2} x \sen x\; dx\\\\
	    dv=\cos x\; dx &\Rightarrow&v=\sen x.
	\end{array}
	$$
	Después, usando la identidad $\sen^2 x = 1-\cos^2 x$, se tiene
	$$
	\begin{array}{rcl}
	    \displaystyle\int \cos^n x\; dx&=& \displaystyle\int u\; dv\\\\
					   &=& uv-\displaystyle \int v\; du\\\\
					   &=& \cos^{n-1}x\sen x + (n-1)\displaystyle\int \cos^{n-2}x\; \sen^2x\; dx\\\\
					   &=& \cos^{n-1}x\sen x + (n-1)\displaystyle\int \cos^{n-2}x\left(1-\cos^2 x\right)\; dx\\\\
					   &=& \cos^{n-1}x\sen x + (n-1)\left(\displaystyle\int \cos^{n-2}x\; dx - \int\cos^n x\; dx \right).
	\end{array}
	$$

	De donde,

	$$
	\begin{array}{rcl}
	    \displaystyle (n-1)\int \cos^n x\;dx + \int \cos^n x\; dx &=& \cos^{n-1}x\sen x + (n-1)\displaystyle\int \cos^{n-2}x\; dx\\\\
	    n\displaystyle\int \cos^n x\; dx &=& \cos^{n-1}x\sen x + (n-1)\displaystyle\int \cos^{n-2}x\; dx\\\\
	    \displaystyle\int \cos^n x\; dx &=& \dfrac{\cos^{n-1}x\sen x}{n} + \dfrac{n-1}{n}\displaystyle\int \cos^{n-2}x\; dx.
	\end{array}
	$$
	\vspace{.5cm}

    %-------------------- 13
    \item Utilizar el resultado del Ejercicio 12 para obtener la fórmula siguiente.

	\begin{enumerate}[(a)]

	    %---------- (a)
	    \item $\displaystyle\int \cos^2 x\; dx = \dfrac{1}{2}x+\dfrac{1}{4}\sen(2x)$.\\\\
		Respuesta.-\; Aplicando el ejercicio 12 y la identidad $\sen(2x)=2\sen x \cos x$, se tiene
		$$\int\cos^2 x\; dx = \dfrac{\cos x \sen x}{2}+\dfrac{1}{2}\int \; dx=\dfrac{1}{2}x+\dfrac{1}{4}\sen(2x).$$\\

	    %---------- (b)
	    \item $\displaystyle\int \cos^3 x\; dx = \dfrac{3}{4}\sen x + \dfrac{1}{12}\sen 3x$.\\\\
		Respuesta.-\; Aplicando el ejercicio 12 y la identidad $\cos^2 x = 1-\sen^2 x$, obtenemos
		$$
		\begin{array}{rcl}
		    \displaystyle\int \cos^3 x\; dx &=& \dfrac{\cos^2x\sen x}{3}+\dfrac{2}{3}\displaystyle\int \cos x\; dx\\\\
						    &=& \dfrac{2}{3}\sen x + \dfrac{\sen x-\sen^3x}{3}\\\\
						    &=& \sen x-\dfrac{1}{3}\sen^3 x.
		\end{array}
		$$
		\vspace{.5cm}

	    %---------- (c)
	    \item $\displaystyle\int \cos^4 x\; dx = \dfrac{3}{8}+\dfrac{1}{4}\sen(2x)+\dfrac{1}{32}\sen(4x)$.\\\\
		Respuesta.-\; Aplicando el ejercicio 12 y la solución de la parte (a) de este ejercicio, se tiene
		$$
		\begin{array}{rcl}
		    \displaystyle\int \cos^4 x\; dx &=& \dfrac{\cos^3x\sen x}{4}+\dfrac{3}{4}\displaystyle\int \cos^2 x\; dx\\\\
						    &=& \dfrac{1}{4}\left[\dfrac{1}{2}\sen(2x)\cos^2 x\right]+\dfrac{3}{4}\left[\dfrac{1}{2}x+\dfrac{1}{4}\sen(2x)\right]\\\\
						    &=&\dfrac{3}{8}x+\dfrac{3}{16}\sen(2x)+\dfrac{1}{8}\left[\sen(2x)-\sen(2x)\sen^2 x\right]\\\\
						    &=& \dfrac{3}{8}x+\dfrac{1}{4}\sen(2x)+\dfrac{1}{32}\sen(4x).
		\end{array}
		$$
		\vspace{.5cm}

	\end{enumerate}

    %-------------------- 14
    \item Integrando por partes, demostrar que
    $$\int \sqrt{1+x^2}\; dx = x\sqrt{1-x^2}+\int \dfrac{x^2}{\sqrt{1-x^2}}\; dx.$$
    Sea $x^2=x^2-1+1$ en la segunda integral deducir la fórmula
    $$\int \sqrt{1-x^2}\; dx = \dfrac{1}{2}x\sqrt{1-x^2}+\dfrac{1}{2}\int \dfrac{1}{\sqrt{1-x^2}}\; dx.$$\\
	Respuesta.-\; Integrando por partes, se tiene
	$$
	\begin{array}{rcl}
	    u=\sqrt{1-x^2} &\Rightarrow& du=\dfrac{-x}{\sqrt{1-x^2}}\; dx\\\\
	    dv=dx &\Rightarrow& v=x.
	\end{array}
	$$

	Entonces,

	$$
	\begin{array}{rcl}
	    \displaystyle\int \sqrt{1-x^2}\; dx &=& \displaystyle\int u\; dv \\\\
						&=& uv-\displaystyle\int v\; du\\\\
						&=& x\sqrt{1-x^2}-\displaystyle\int \dfrac{x^2}{\sqrt{1-x^2}}\; dx.
	\end{array}
	$$

	Luego, escribiendo $x^2=x^2-1+1$ se obtiene

	$$
	\begin{array}{rcl}
	    \displaystyle\int \sqrt{1-x^2}\; dx &=& x\sqrt{1-x^2} + \displaystyle\int \dfrac{x^2-1+1}{\sqrt{1-x^2}}\; dx\\\\
						&=& x\sqrt{1-x^2} + \displaystyle\int \sqrt{1-x^2}\; dx + \int \dfrac{1}{\sqrt{1-x^2}}\; dx
	\end{array}
	$$

	Por lo tanto,

	$$2\displaystyle\int \sqrt{1-x^2}\; dx = x\sqrt{1-x^2} + \int \dfrac{1}{\sqrt{1-x^2}}\; dx$$ 
	$$\Downarrow$$  
	$$\int \sqrt{1-x^2}\; dx = \dfrac{1}{2}x\sqrt{1-x^2}+\dfrac{1}{2}\int \dfrac{1}{\sqrt{1-x^2}}\; dx.$$\\

    %-------------------- 15
    \item 
	\begin{enumerate}[a)]

	    %---------- (a)
	    \item Usar la integración por partes para deducir la fórmula
	    $$\int \left(a^2-x^2\right)^n \; dx = \dfrac{x\left(a^2-x^2\right)^n}{2n+1}+\dfrac{2a^2n}{2n+1}\int \left(a^2-x^2\right)^{n-1}\; dx + C.$$\\
		Respuesta.-\; Sea,
		$$
		\begin{array}{rcl}
		    u=\left(a^2-x^2\right)^n &\Rightarrow& du=(-2nx)\left(a^2-x^2\right)^{n-1}\; dx\\\\
		    dv=dx &\Rightarrow& v=x.
		\end{array}
		$$
		Entonces,
		$$
		\begin{array}{rcl}
		    \displaystyle\int \left(\right)^n\; dx &=& \displaystyle\int u\; dv\\\\
							   &=& uv - \int v\; du\\\\
							   &=& x\left(a^2-x^2\right)^n - 2n\displaystyle\int x^2\left(a^2-x^2\right)^{n-1}\; dx+C.
		\end{array}
		$$
		Usando la identidad $x^2=x^2+a^2-a^2 = -\left(a^2-x^2\right)+a^2$, se tiene
		$$\int \left(a^2-x^2\right)^n\; dx = x\left(a^2-x^2\right)^n+2n\left[-\int \left(a^2-x^2\right)^n\; dx+a^2\int\left(a^2-x^2\right)^{n-1}\; dx\right]$$
		$$\Downarrow$$
		$$(2x+1)\int \left(a^2-x^2\right)^n\; dx = x\left(a^2-x^2\right)^n + 2na^2\int \left(a^2-x^2\right)^{n-1}\; dx$$
		$$\Downarrow$$
		$$\int \left(a^2-x^2\right)^n\; dx = \dfrac{x\left(a^2-x^2\right)^n}{2n+1}+\dfrac{2na^2}{2n+1}\int \left(a^2-x^2\right)^{n-1}\; dx.$$\\\\

	    %---------- (b)
	    \item Utilizar la parte a) para calcular $\int_0^a \left(a^2-x^2\right)^{5/2}\; dx.$\\\\
		Respuesta.-\; 
		$$
		\begin{array}{rcl}
		    \displaystyle\int \left(a^2-x^2\right)^{\frac{5}{2}}\; dx &=& \left[\dfrac{x\left(a^2-x^2\right)^{\frac{5}{2}}}{6}+\dfrac{5a^2}{6}\displaystyle\int \left(a^2-x^2\right)^{\frac{3}{2}}\; dx\right]\bigg|_0^a\\\\
									      &=& \dfrac{5a^2}{6} \displaystyle\int_0^a \left(a^2-x^2\right)^{\frac{3}{2}}\; dx \\\\
									      &=&\dfrac{5a^2}{6}\left[\dfrac{x\left(a^2-x^2\right)^{\frac{3}{2}}}{4}+\dfrac{3a^2}{4}\displaystyle\int \left(a^2-x^2\right)^{\frac{1}{2}}\; dx\right]\bigg|_0^a\\\\
									      &=& \dfrac{15a^4}{24}\displaystyle\int_0^a \left(a^2-x^2\right)^{\frac{1}{2}}\; dx\\\\
									      &=&\left(\dfrac{5a^4}{8\cdot 2}\right)\left(\dfrac{1}{2}\right)2\displaystyle\int_{-a}^a \left(a^2-x^2\right)^{\frac{1}{2}}\; dx\\\\
									      &=& \dfrac{5\pi}{32}a^6.
		\end{array}
		$$
		\vspace{.5cm}

	\end{enumerate}

    %-------------------- 16
    \item 
	\begin{enumerate}[a)]

	    %---------- (a)
	    \item Si $I_n(x)=\int_0^x t^n \left(t^2+a^2\right)^{-1/2}\; dt$, aplicar el método de integración por partes para demostrar que
	    $$nI_n(x)=x^{n-1}\sqrt{x^2+a^2}=(n-1)a^2I_{n-2}(x)\qquad \mbox{si}\; n\geq 2.$$\\
		Demostración.-\; Aplicando el método de integración por partes, se tiene
		$$
		\begin{array}{rcl}
		    u = t^{n-1} &\Rightarrow& du = (n-1)t^{n-2}\; dt\\\\
		    dv = t\left(t^2+a^2\right)^{-\frac{1}{2}} &\Rightarrow& v =  \left(t^2+a^2\right)^{\frac{1}{2}}\; dt.
		\end{array}
		$$
		Entonces,
		$$
		\begin{array}{rcl}
		    I_n(x)=\displaystyle \int_0^x t^n\left(t^2+a^2\right)^{-\frac{1}{2}}&=& \displaystyle\int_0^x u\; dv\\\\
											&=& uv\bigg|_0^x - \int_0^x v\; du\\\\
											&=& t^{n-1}\left(t^2+a^2\right)^{\frac{1}{2}}\bigg|_0^x - \displaystyle\int_0^x (n-1) t^{n-2}\left(t^2+a^2\right)^{\frac{1}{2}}\; dt\\\\
											&=& x^{n-1}\left(x^2+a^2\right) - \displaystyle\int_0^x (n-1)t^{n-2} \left(t^2+a^2\right)\left(t^2+a^2\right)^{-\frac{1}{2}}\; dt\\\\
											&=& x^{n-1}\left(x^2+a^2\right) - \displaystyle\int_0^x (n-1)t^{n-2} \left(t^2+a^2\right)^{-\frac{1}{2}}\; dt\\\\
											&-& \displaystyle\int_0^x (n-1)t^{n-2}a^2\left(t^2+a^2\right)^{-\frac{1}{2}}\; dt\\\\
											&=&x^{n-1}\left(x^2+a^2\right)^{\frac{1}{2}}-(u-1)I_n(x)-a^2(n-1)I_{n-2}(x).
		\end{array}
		$$
		Luego añadiendo $(n-1)I_n(x)$ a ambos lados de la ecuación anterior tenemos
		$$nI_n(x)=x^{n-1}\left(x^2+a^2\right)^{\frac{1}{2}}-a^2(n-1)I_{n-2}(x).$$\\

	    %---------- (b)
	    \item Aplicando (a) demostrar que $\int_0^2 x^5\left(x^2+5\right)^{-1/2}\; dx = \dfrac{168}{5}-40\sqrt{\frac{5}{3}}.$\\\\
		Respuesta.-\; Primero por la definición de $I_n(x)$, se tiene
		$$
		\begin{array}{rcl}
		    I_5(2)&=&\displaystyle\int_0^2 t^5\left(t^2+5\right)^{-\frac{1}{2}}\; dt\\\\
		    I_5(2)&=&\displaystyle\int_0^2 t^3\left(t^2+5\right)^{-\frac{1}{2}}\; dt\\\\
		\end{array}
		$$
		Luego, usando la fórmula de la parte (a) obtenemos
		$$5I_5(2)=2^4\sqrt{4+5}-4\cdot 5 \cdot I_3(2) \quad \Rightarrow \quad I_5(2)=\dfrac{48}{5}-4I_3(2).$$
		Usando una vez más la fórmula de la parte (a), tenemos
		$$3I_(2)=2^2\sqrt{4+5}-5\cdot 2 \cdot I_1(2) \quad \Rightarrow \quad I_3(2)=4-\dfrac{10}{3}\cdot I_1(2)=4-\dfrac{10}{3}\int_0^2 t\left(t^2+5\right)^{-\frac{1}{2}\; dt}.$$
		Luego, utilizando el método de sustitución para $u=t~ 2+5$ que implica $du=2t\; dt$, se tiene
		$$
		    \begin{array}{rcl}
			\displaystyle\int_0^2 t\left(t^2+5\right)^{-\frac{1}{2}} \; dt &=& \dfrac{1}{2}\displaystyle\int_{u(0)}^{u(2)} u^{-\frac{1}{2}}\; du\\\\
			&=& \dfrac{1}{2}2u^{\frac{1}{2}}\bigg|_5^9\\\\
			&=& 3 - \sqrt{5}.
		    \end{array}
		$$
		Introduciendo está solución a $I_3(2)$,
		$$I_3(2)=4-\dfrac{10}{3}\left(3-\sqrt{5}\right)=\dfrac{10}{3}\sqrt{5}-6.$$
		Y después, intruduciendo a $I_5(2)$, concluimos que 
		$$I_5(2)=\dfrac{48}{5}-4\left(\dfrac{10}{3}\sqrt{5}-6\right)=\dfrac{168}{5}-\dfrac{40\sqrt{5}}{3}.$$\\

	\end{enumerate}

    %-------------------- 17
    \item Calcular la integral $\int_{-1}^3 t^3\left(4+t^3\right)^{-1/2}\; dt,$ sabiendo que $\int_{-1}^3 \left(4+t^3\right)^{1/2}\; dt = 11.35$ Dejar el resultado en función de $\sqrt{3}$ y $\sqrt{31}$.\\\\
	Respuesta.- Aplicando la integración por partes, se tiene
	$$
	\begin{array}{rcl}
	    u=t &\Rightarrow& du=dt \\
	    dv=\dfrac{3}{2}t^2\cdot \dfrac{1}{\sqrt{4+t^3}}\; dt &\Rightarrow& v = \sqrt{4+t^3}\\
	\end{array}
	$$
	Entonces, 
	$$
	\begin{array}{rcl}
	    \displaystyle\int_{-1}^3 t^3 \left(4+t^3\right)^{-\frac{1}{2}} &=& \dfrac{2}{3}\displaystyle\int_{-1}^3 \dfrac{3}{2}t^3 \left(4+t^3\right)^{-\frac{1}{2}}\; dt\\\\
	    &=& \dfrac{2}{3} \displaystyle\int_{-1}^3 u\; dv\\\\
	    &=& \dfrac{2}{3}\left(uv\bigg|_{-1}^3 - \displaystyle\int_{-1}^3 v\; du\right)\\\\
	    &=& \dfrac{2}{3}\left(t\sqrt{4+t^3}\bigg|_{-1}^3-\displaystyle\int_{-1}^3 \sqrt{4+t^3}\; dt\right)\\\\
	    &=& \dfrac{2}{3}\left(3\sqrt{31}+\sqrt{3}-11.35\right).
	\end{array}
	$$
	\vspace{.5cm}


    %-------------------- 18
    \item Integrando por partes deducir la fórmula
    $$\int \dfrac{\sen^{n+1}x}{\cos^{m+1}x}=\dfrac{1}{m}\dfrac{\sen^n x}{\cos^m x}-\dfrac{n}{m}\int \dfrac{\sen^{n-1}x}{\cos^{m-1}x}\; dx.$$
    Aplicar la fórmula para integrar $\int \tan^2 x\; dx$ y $\int \tan^4 x \; dx.$\\\\
    	Respuesta.-\; Aplicando la integración por partes,
	$$
	\begin{array}{rcl}
	    u=\sen^n x & \Rightarrow & du=n\sen^{n-1} x\cos x\; dx\\\\
	    dv=\dfrac{\sen x}{\cos^{m+1}}\; dx & \Rightarrow & v=\dfrac{1}{m}\cdot \dfrac{1}{\cos^m x}.
	\end{array}
	$$
	Luego, haciendo la sustitución $t = \cos x$ que implica $dt=-\sen x\; dx$. Por lo que tenemos
	$$
	\begin{array}{rcl}
	    \displaystyle\int \dfrac{\sen x}{\cos^{m+1}}\; dx &=& - \displaystyle\int \dfrac{1}{t^{m+1}}\; dt\\\\
	    &=& \dfrac{1}{m}t^{-m}\\\\
	    &=&\dfrac{1}{m}\cdot \dfrac{1}{\cos m}
	\end{array}
	$$
	Después,
	$$
	\begin{array}{rcl}
	    \displaystyle\int \dfrac{\sen^{n+1}x}{\cos^{m+1}x}\; dx &=& \displaystyle\int u \; dv\\\\
	    &=& uv-\displaystyle\int v\; du\\\\
	    &=& \dfrac{1}{m}\dfrac{\sen^n x}{\cos^m x}-\dfrac{n}{m}\displaystyle \int \dfrac{\sen^{n-1}x\cos x}{\cos^m x}\; dx\\\\
	    &=& \dfrac{1}{m}\dfrac{\sen^n x}{\cos^m x}-\dfrac{n}{m}\displaystyle\int \dfrac{\sen^{n-1}x\cos x}{\cos^m x}\; dx.
	\end{array}
	$$
	Ahora, aplicando la fórmula a $\int \tan^2 x$, y dado que $\tan^2 x = \dfrac{\sen^2 x}{\cos^2 x}$. Este es el caso de la formula de arriba con $n+1=m+1=2.$ Por lo que
	$$
	\begin{array}{rcl}
	    \displaystyle\int \tan^2 x\; dx &=& \displaystyle\int \dfrac{\sen^2 x}{\cos^2 x}\; dx\\\\
	    &=& \dfrac{\sen x}{\cos x}-\displaystyle\int \; dx\\\\
	    &=& \tan x - x + C.
	\end{array}
	$$
	Entonces, para $\int \tan^4 x\; dx$ se tiene
	$$
	\begin{array}{rcl}
	    \displaystyle\int \tan^4 x \; dx &=& \displaystyle\int \dfrac{\sen^4}{\cos^4 x}\; dx\\\\
	    &=& \dfrac{1}{3}\dfrac{\sen^3 x}{\cos^3 x} - \displaystyle\int \dfrac{\sen^2 x}{\cos^2 x}\; dx\\\\
	    &=& \dfrac{1}{3}\tan^3 x - \tan x + x + C.
	\end{array}
	$$
	\vspace{.5cm}

    %-------------------- 19
    \item Integrando por partes deducir la fórmula
    $$\int \dfrac{\cos^{m+1}x}{\sen^{n+1}x}\; dx = -\dfrac{1}{n}\dfrac{\cos^m x}{\sen^n x}-\dfrac{m}{n}\int \dfrac{\cos^{m-1}x}{\sen^{n-1}x}\; dx.$$
    Utilizar la fórmula para integrar $\int \cot^2 x \; dx$ y $\int \cot^4 x\; dx$.\\\\
    	Respuesta.-\; Usando la integración por partes,
	$$
	\begin{array}{rcl}
	    u=\cos^m x &\Rightarrow& du=-m\cos^{m-1} x\sen x\; dx\\\\
	    dv=\dfrac{\cos x}{\sen^{n+1}}\; dx & \Rightarrow & v=-\dfrac{1}{n}\cdot \dfrac{1}{\sen^n x}.
	\end{array}
	$$
	Donde la formula para $v$ con $t=\sen x$ y $dt=\cos x$ se tiene
	$$
	\begin{array}{rcl}
	    \displaystyle\int \dfrac{\cos x}{\sen^{n+1}}\; dx &=& \displaystyle\int \dfrac{1}{t^{n+1}}\; dt\\\\
	    &=& -\dfrac{1}{n}t^{-n}\\\\
	    &=& -\dfrac{1}{n}\cdot \dfrac{1}{\sen^n x}.
	\end{array}
	$$
	Por lo que
	$$
	\begin{array}{rcl}
	    \displaystyle\int \dfrac{\cos^{m+1}x}{\sen^{n-1}x}\; dx &=& \displaystyle\int u \; dv\\\\
	    &=& uv - \displaystyle\int v\; du\\\\
	    &=& -\dfrac{1}{n}\cdot \dfrac{\cos^m x}{\sen^n x}-\dfrac{m}{n}\displaystyle\int \dfrac{\cos^{m-1}x\sen x}{\sen^n x}\; dx\\\\
	    &=& -\dfrac{1}{n}\cdot \dfrac{\cos^m x}{\sen^n x}-\dfrac{m}{n}\displaystyle\int \dfrac{\cos^{m-1}x}{\sen^{n-1}x}\; dx.
	\end{array}
	$$
	Luego, usamos esta solución para evaluar $\cot^2 x$ y $\cot^4 x$. Primero para $\cot^2 x$ usamos la formula con $m+1=n+1=2$, como sigue
	$$
	\begin{array}{rcl}
	    \displaystyle\int \cot^2 x\; dx &=& \displaystyle\int \dfrac{\cos^2 x}{\sen^2 x}\; dx\\\\
	    &=& -\dfrac{\cos x}{\sen x} - \displaystyle\int \; dx\\\\
	    &=& - \cot x - x + C.
	\end{array}
	$$
	Para $\cot^4 x$ usamos la formula con $m+1=n+1=4,$
	$$
	\begin{array}{rcl}
	    \displaystyle\int \cot^4 x\; dx &=& \displaystyle \int \dfrac{\cos^4 x}{\sen^4 x}\; dx\\\\
	    &=& -\dfrac{1}{3}\dfrac{\cos^3 x}{\sen^3 x} - \displaystyle \int \cot^2 x\; dx\\\\
	    &=& -\dfrac{1}{3}\cot^3 x + \cot x + x + C.
	\end{array}
	$$
	\vspace{.5cm}
	

    %-------------------- 20
    \item 
	\begin{enumerate}[(a)]

	    %---------- (a)
	    \item Hallar un entero $n$ tal que $n\int_0^1 x f''(2x)\; dx = \int_0^2 tf''(t)\; dt$.\\\\
		Respuesta.-\; Utilicemos la propiedad de expansión contracción de la integral, de donde
		$$
		\begin{array}{rcl}
		    \displaystyle\int t f''(t)\; dt &=& n\displaystyle\int_0^1 xf''(2x)\; dx\\\\
		    &=& \dfrac{n}{2} \displaystyle\int_0^2 \dfrac{x}{2} f''(x)\; dx\\\\
		    &=& \dfrac{n}{4}\displaystyle\int_0^2 xf''(x)\; dx\\\\
		    &=& \dfrac{n}{4}\displaystyle\int_0^2 t f''(x)\; dt.
		\end{array}
		$$
		Esta ecuación se cumple siempre que $n=4$.\\\\

	    %---------- (b)
	    \item Calcular $\int_0^1 xf''(2x)\; dx$, sabiendo que $f(0)=1,$ $f(2)=3$ y $f'(2)=5$.\\\\
		Respuesta.-\; Utilizando la parte (a), se tiene
		$$\int_0^1 xf''(2x)\; dx = \dfrac{1}{4}\int_0^2 xf''(x)\; dx.$$
		Entonces, usamos la integración por partes con
		$$
		\begin{array}{rcl}
		    u=x &\Rightarrow& du=dx\\\\
		    dv=f''(x)\; dx &\Rightarrow& v=f'(x).
		\end{array}
		$$
		Por lo tanto,
		$$
		\begin{array}{rcl}
		    \dfrac{1}{4}\displaystyle\int xf''(x)\; dx &=& \dfrac{1}{4}\left(xf'(x)\bigg|_0^2 - \displaystyle\int_0^2 f'(x)\; dx\right)\\\\
		    &=& \dfrac{1}{4}\left[2f'(2)-f(2)+f(0)\right]\\\\
		    &=& \dfrac{1}{4}(10-3+1)\\\\
		    &=& 2.
		\end{array}
		$$
		\vspace{.5cm}

	\end{enumerate}

    %-------------------- 21
    \item 
	\begin{enumerate}[a)]

	    %---------- (a)
	    \item Si $\phi''$ es continua y no nula en $[a,b]$, y si existe una constante $m>0$ tal que $\phi'(t)\geq m$ para todo $t$ en $[a,b]$, usar el teorema 5.5 para demostrar que
	    $$\left|\int_a^b \sen \phi(t) \; dt\right|\leq \dfrac{4}{m}.$$
	    [Indicación: Multiplicar y dividir el integrando por $\phi'(t).$]\\\\
		Demostración.-\; Ya que $\phi'(t)\geq m > 0$ para todo $t\in [a,b]$, podemos dividir por $\phi'(t)$ de donde obtenemos 
		$$\bigg|\int_a^b \sen \phi(t)\; dt\bigg|=\bigg|\int_a^b \dfrac{\sen \phi(t)}{\phi'(t)}\cdot \phi'(t)\; dt\bigg|$$
		Entonces, para aplicar el segundo teorema de valor medio para integrales (Teorema 5.5, Tom Apostol) se define las funciones siguientes
		\begin{center}
		    $f(t)=\dfrac{1}{\phi'(t)} y g(t)=\phi'(t)\sen \phi(t).$
		\end{center}
		La función $g$ es continua, ya que $\sen \phi(t)$ es una composición de funciones continuas (sabemos que $\phi(t)$ es continua ya que es derivable) y $\phi'(t)$ es continua (es derivable ya que $\phi''(t)$ existe y es continua). Entonces el producto de la función continua también es continuo, lo que establece que $g$ es continua. También sabemos que $f$ cumple las condiciones del teorema, puesto que tiene derivada dada por
		$$f''(t)=-\dfrac{\phi''(t)}{\left[\phi'(t)\right]^2}.$$
		Esta derivada es continua en $[a,b]$, pues $\phi''(t)$ y $\phi'(t)$ son continuas y $\phi'(t)$ no es cero. Además, $\phi''(x)\neq 0\in [a,b]$. Recordemos por el teorema de Bolzano que una función continua que cambia de signo debe tener un cero. Por lo que podemos aplicar el segundo teorema de valor medio:
		$$
		\begin{array}{rcl}
		    \bigg|\displaystyle\int_a^b \dfrac{\phi'(t)\sen \phi(t)}{\phi'(t)}\; dt\bigg| &=& \bigg|\dfrac{1}{\phi'(a)}\displaystyle\int_a^c \phi'(t)\sen \phi(t) \; dt + \dfrac{1}{\phi'(b)}\displaystyle\int_c^b \phi'(t)\sen \phi(t)\; dt\bigg|\\\\
		    &\leq& \bigg|\dfrac{1}{m}\displaystyle\int_a^c \phi'(t)\sen \phi(t)\; dt + \dfrac{1}{m}\displaystyle\int_c^b \phi'(t)\sen \phi(t)\; dt\bigg|\\\\
		    &\leq & \Bigg|\left(-\dfrac{1}{m}\right)\cos \phi(t)\bigg|_a^c\Bigg| + \Bigg|\left(-\dfrac{1}{m}\right)\cos \phi(t)\bigg|_c^b\Bigg|\\\\
		\end{array}
		$$
		Luego, por la desigualdad triangular
		$$
		\begin{array}{rcl}
		    &\leq&\Bigg|-\dfrac{2}{m}\Bigg|+ \Bigg|-\dfrac{2}{m}\Bigg|\\\\
		    &=& \dfrac{2}{m}+\dfrac{2}{m}\\\\
		    &=& \dfrac{4}{m}.
		\end{array}
		$$
		\vspace{.5cm}

	    %---------- (b)
	    \item Si $a>0$, demostrar que $\left|\int_a^x \sen\left(t^2\right)\; dt\right|\leq \frac{2}{a}$ para todo $x>a$.\\\\
		Demostración.-\; Sea $\phi(t)=t^2$, por la parte (a) se tiene
		$$
		\begin{array}{rcl}
		    \phi(t)=t^2 &\Rightarrow& \phi'(t)=2t\\
				&\Rightarrow& \phi''(t)=2.
		\end{array}
		$$
		Por lo tanto, $\phi''(t)$ es continua y no cambia de signo. Además, $\phi'\geq m=2a$ donde $a$ es una constante con $2a>0\; \Rightarrow \; a>0$. Así,
		$$\Bigg|\int_a^x \sen\left(t^2\right)\; dt\Bigg|\leq \dfrac{4}{2a}=\dfrac{2}{a}\quad \mbox{para todo }x>a.$$\\

	\end{enumerate}


    \end{enumerate}


\section{Ejercicios de repaso}

\begin{enumerate}[\bfseries 1.]

    %-------------------- 1.
    \item Sea $f$ un polinomio tal que $f(0)=1$ y sea $g(x)=x^nf(x)$. Calcular $g(0),g'(0),\ldots , g^{n}(0)$.\\\\
	Respuesta.-\; Usando la regla de la cadena, se tiene
	$$
	\begin{array}{rcl}
	    g'(x) &=& nx^{n-1}f(x)+x^nf'(x)\\\\
	    g''(x) &=& n(n-1)x^{n-2}f(x)+2nx^{n-1}f'(x)+x^nf''(x)\\\\
	    g^{(3)}(x) &=& n(n-1)(n-2)x^{n-3}f(x)+\displaystyle{3\choose 2}n(n-1)x^{n-2}f'(x)+\displaystyle{3\choose 1}nx^{n-1}f''(x)+x^nf^{(3)}(x)\\\\
	\end{array}
	$$
	$$
	\begin{array}{rcl}
	    g^{4}(x) &=& n(n-1)(n-2)(n-3)x^{n-4}f(x)+\displaystyle{4\choose 3}n(n-1)(n-2)x^{n-3}f'(x)+\displaystyle{4\choose 2}n(n-1)x^{n-2}f''(x)\\\\
		     &+&\displaystyle{4\choose 1}nx^{n-1}f^{(3)}(x)+x^nf^{(4)}(x)\\\\
	\end{array}
	$$
	Vemos que se cumple un patrón, por lo que podemos generalizarde la siguente manera:
	$$
	\begin{array}{rcl}
	    g^{(n)}(x)&=&n!f(x)+\displaystyle{n\choose n-1}\dfrac{n!}{1!}xf'(x)+\displaystyle{n\choose n-2}x^2f''(x)+\ldots + \displaystyle{n\choose 2}\dfrac{n!}{(n-2)!}x^{n-2}f^{n-2}(x)\\\\
		      &+& \displaystyle{n\choose 1}\dfrac{n!}{(n-1)!}x^{n-1} f^{(n-1)}(x)+x^n f^{(n)}(x).
	\end{array}
	$$
	Luego, usando está fórmula para calcular $g(0),g'(0),\ldots, g^{(n)}(0)$ se tiene
	$$
	\begin{array}{rcl}
	    g'(0)&=&g''(0)=\ldots = g^{n-1}(0)=0\\\\
	    g^{n}(0)&=&n!f(0)=n!.
	\end{array}
	$$
	\vspace{.5cm}


    %-------------------- 2.
    \item Hallar un polinomio $P$ de grado $\leq 5$ tal que $P(0)=1$, $P(1)=2$, $P'(0)=P''(0)=P'(1)=P''(1)=0$.\\\\
	Respuesta.-\; Como debe ser un polinomio de grado $\leq 5$ podemos escribir:
	$$P(x)=a_5x^5+a_4x^4+a_3x^3+a^2x^2+a_1x+a_0$$
	donde cualquiera de las $a_i$ podría ser $0$. Ya que podríamos tener un polinomio de grado menor a $5$. En primer lugar, apliquemos la condición $P(0)=1$ para obtener
	$$P(0)=a_0=1.$$
	Reemplazando esta condición en $P(x)$ y derivando, tenemos 

	$$
	\begin{array}{rcl}
	    P(x)&=&a_5x^5+a_4x^4+a_3x^3+a^2x^2+a_1x+1\\\\
	    P'(x)&=&5a_5x^4+4a_4x^3+3a_3x^2+2a^2x+a_1\\\\
	    P''(x)&=&20a_5x^3+12a_4x^2+6a_3x+2a^2.
	\end{array}
	$$

	Luego, podemos aplicar las condiciones $P'(0)$ y $P''(0)=0$, obteniendo

	$$
	\begin{array}{rcccl}
	    P'(0)&=&a_1&=&0\\\\
	    P''(0)&=&2a_2&=&0.
	\end{array}
	$$

	Así, tenemos $a_0=1$ y $a_1=a_2=0$ de donde
	$$P(x)=a_5x^5+a^4x^4+a^3x^3+1.$$

	Después, necesitamos usar las otras tres condiciones dadas,
	$$
	\begin{array}{rcl}
	    P(1)=2 &\Rightarrow& a_5+a^4+a^3=1\\\\
	    P'(1)=0 &\Rightarrow& 5a_5+4a^4+3a^3=0\\\\
	    P''(1)=0 &\Rightarrow& 20a_5+12a^4+6a^3=0.
	\end{array}
	$$

	Por la primera ecuación se tiene
	$$a_3=1-a_4-a_5$$

	Reemplazando esta ecuación en la segunda

	$$
	\begin{array}{rcl}
	    5a_5+4a_4+3(1-a_4-a_5)=0&\Rightarrow&2a_5+a_4+3=0\\\\
				    &\Rightarrow&a_4=-3-2a_5.
	\end{array}
	$$
	Introduciendo ahora nuestras expresiones para $a_3$ y $a_4$ en la tercera ecuación tenemos
	$$20a_5+12(-3-2a_5)+6\left[1-(-3-2a_5)-a_5\right]\quad \Rightarrow \quad a_5=6.$$
	Luego,
	$$
	\begin{array}{rcr}
	    a_4&=&-15\\
	    a_3&=&10
	\end{array}
	$$
	Ahora que hemos calculado todas las constantes $a_i$, podemos escribir la fórmula del polinomio
	$$P(x)=6x^5-15x^4+10x^3+1.$$\\


    %-------------------- 3.
    \item Si $f(x)=\cos x$ y $g(x)=\sen x$, demostrar que
    $$f^{(n)}(x)=\cos\left(x+\dfrac{1}{2}n\pi\right)\quad \mbox{y}\quad g^{(n)}(x)=\sen\left(x+\dfrac{1}{2}n\pi\right).$$\\
	Demostración.-\; Utilizaremos la prueba por inducción. Para $n=1$ tenemos
	$$f'(x)=-\sen x = \cos \left(x+\dfrac{1}{2}\pi\right),\qquad g'(x)=\cos x = \sen\left(x+\dfrac{1}{2}\pi\right).$$
	Estas igualdades se deducen de las correlaciones del seno y el coseno (Teorema 2.3 parte (d) en la página 96 de Apostol).  Por tanto, las fórmulas son verdaderas para el caso $n=1$. Supongamos entonces que son verdaderas para algún $n=k\in\mathbb{Z}^+$. Para $f(x)$ tenemos
	$$
	\begin{array}{rcl}
	    f^{k+1}(x)&=&\left[f^{k}(x)\right]'\\\\
		      &=&\left[\cos\left(x+\dfrac{1}{2}k\pi\right)\right]'\\\\
		      &=&-\sen\left(x+\dfrac{1}{2}k\pi\right)\\\\
		      &=&\cos\left[\left(x+\dfrac{1}{2} k \pi\right)+\dfrac{1}{2}\pi\right]\\\\
		      &=&\cos\left[x+\dfrac{1}{2}(k+1)\pi\right].
	\end{array}
	$$

	De las misma forma para $g(x)$

	$$
	\begin{array}{rcl}
	    g^{k+1}(x)&=&\left[g^{k}(x)\right]'\\\\
		      &=&\left[\sen\left(x+\dfrac{1}{2}k\pi\right)\right]'\\\\
		      &=&\cos\left(x+\dfrac{1}{2}k\pi\right)\\\\
		      &=&\sen\left[\left(x+\dfrac{1}{2} k \pi\right)+\dfrac{1}{2}\pi\right]\\\\
		      &=&\sen\left[x+\dfrac{1}{2}(k+1)\pi\right].
	\end{array}
	$$
	Por lo tanto, el teorema se sigue por inducción para todos los números enteros positivos.\\\\

    %-------------------- 4.
    \item Si $h(x)=f(x)g(x)$, demostrar que la derivada n-ésima de $h$ viene dada por la fórmula
    $$h^{(n)}(x)=\sum_{k=0}^n {n\choose k}f^{(k)}g^{(n-k)}(x),$$
    en donde ${n\choose k}$ representa el coeficiente binomial. Esta es la llamada fórmula de Leibniz.\\\\
	Demostración.-\; La prueba es por inducción. Sean $h(x)=f(x)g(x)$ y $n=1$. Usemos la regla de la cadena, se tiene
	$$h'(x)=f(x)g'(x)+f'(x)g(x)=\sum_{k=0}^1 {1\choose k}f^{(k)}(x)g^{(1-k)}(x).$$
	Por lo que la fórmula es cierta para $n=1$. Suponiendo que es verdad para $n=m\in \mathbb{Z}^+$. Entonces, consideremos la $(m+1)$ derivada de la siguiente forma:
	$$
	\begin{array}{rcl}
	    h^{(m+1)}(x) &=& \left[h^{(m)}(x)\right]'\\\\
			 &=& \left[\displaystyle\sum_{k=0}^m {m\choose k}f^{(k)}g^{(m-k)}(x)\right]'\\\\
			 &=& \displaystyle\sum_{k=0}^m \left[{m\choose k}f^{(k)}g^{(m-k)}(x)\right]'.
	\end{array}
	$$

	Aquí utilizamos la linealidad de la derivada para diferenciar término a término sobre esta suma finita. Esta propiedad se estableció en el Teorema 4.1 (i) y en los comentarios que siguen al teorema. Luego, aplicamos la regla del producto,

	$$
	\begin{array}{rcl}
	    &=& \displaystyle\sum_{k=0}^m \left[{m\choose k}\left(f^{(k+1)}(x)g^{(m-k)}(x)+f^{(k)}(x)g^{(m-k+1)}(x)\right)\right]\\\\
	    &=& \displaystyle\sum_{k=0}^m {m\choose k} f^{(k+1)}(x)g^{(m-k)}(x)+\sum_{k=0}^m {m\choose k} f^{(k)}(x)g^{(m-k+1)}(x)\\\\
	    &=& \displaystyle\sum_{k=1}^{m+1} {m\choose k-1} f^{(k+1)}(x)g^{(m-k)}(x)+\sum_{k=0}^m {m\choose k} f^{(k)}(x)g^{(m-k+1)}(x),\\\\
	\end{array}
	$$

	donde hemos reindexado la primera suma desde $k=1$ a $m+1$ en lugar de $k=0$ para $m$. Entonces, sacamos el término $k=m+1$ de la primera suma y el término de la segunda,

	$$
	\begin{array}{rcl}
	    &=& f^{(m+1)}(x)g(x)+\displaystyle\sum_{k=0}^m {m\choose k-1} f^{(k)}(x)g^{(m-k+1)}(x)+\displaystyle\sum_{k=0}^m {m\choose k}f^{(k)}(x)g^{(m-k+1)}(x)+f(x)g^{(m+1)}(x)\\\\
	    &=& f^{(m+1)}(x)g(x)+f(x)g^{(m+1)}(x)+\displaystyle\sum_{k=0}^m \left[{m\choose k-1}{m\choose k}\right] f^{(k)}(x)g^{(m-k+1)}(x)\\\\
	\end{array}
	$$

	Recordemos, la ley de triangulo de Pascal
	$${m\choose k}+{m\choose k-1}={m+1\choose k}.$$
	Por lo tanto,

	$$
	\begin{array}{rcl}
	    &=& f^{(m+1)}(x)g(x)+f(x)g^{(m+1)}(x)+\displaystyle\sum_{k=0}^m \left[{m\choose k-1}{m\choose k}\right] f^{(k)}(x)g^{(m-k+1)}(x)\\\\
	    &=& \displaystyle\sum_{k=0}^{m+1} {m+1\choose k} f^{(k)}(x)g^{(m+1-k)}(x).
	\end{array}
	$$
	Por lo tanto, la fórmula es válida para $m+1$ si es válida para $m$.\\\\


    %-------------------- 5.
    \item Dadas las funciones $f$ y $g$ cuyas derivadas $f'$ y $g'$ satisfacen las ecuaciones
    $$f'(x)=g(x),\quad g'(x)=-f(x),\quad f(0)=0,\quad g(0)=1,$$
    para todo $x$ en un cierto intervalo abierto $J$ que contiene el $0$. Por ejemplo, esas ecuaciones se satisfacen cuando $f(x)=\sen x$ y $g(x)=\cos x.$

    \begin{enumerate}[a)]
	
	%---------- a)
	\item Demostrar que $f^2(x)+g^2(x)=1$ para todo $x$ de $J$.\\\\
	    Demostración.-\; Primero probaremos que $f^2(x)+g^2(x)=c$ para alguna constante, y luego probaremos que esa constante es $1.$ Para demostrar $f^2+g^2$ es constante tomemos la derivada, como sigue
	    $$
	    \begin{array}{rcl}
		\left[f^2(x)+g^2(x)\right]' &=& \left[f^2(x)\right]' + \left[g^2(x)\right]'\\\\
			       &=& 2f(x)f'(x)+2g(x)g'(x)\\\\
			       &=& 2f(x)g(x)-2g(x)f(x)\\\\
			       &=& 0.
	    \end{array}
	    $$
	    esto, para todo $x\in J$; por lo tanto, por el teorema 4.7 c en Tom Apostol, tenemos que $f^2(x)+g^2(x)=c,\; x\in I$ para algúna constante $c$. Después, por hipótesis, vemos que $0\in J$ de donde,
	    $$f^2(0)+g^2(0)=0+1=1.$$
	    Así, $f^2(x)+g^2(x)=1$ para todo $x\in J$.\\\\
	
	%---------- b)
	\item Sean $F$ y $G$ otro par de funciones que satisfagan (5.30). Demostrar que $F(x)=f(x)$ y $G(x)=g(x)$, para todo $x$ de $J$.\\\\
	    Demostración.-\; Sea,
	    $$h(x)=\left[F(x)-f(x)\right]^2+\left[G(x)-g(x)\right]^2.$$
	    que implica
	    $$h'(x)=2\left[F(x)-f(x)\right]\left[F'(x)-f'(x)\right]+2\left[G(x)-g(x)\right]\left[G'(x)-g'(x)\right].$$
	    luego, usando las relaciones de la hipótesis
	    $$
	    \begin{array}{rcl}
		h'(x) &=& 2\left[F(x)-f(x)\right]\left[F'(x)-f'(x)\right]+2\left[F'(x)-f'(x)\right]\left[f(x)-F(x)\right]\\\\
		      &=& 2\left[F(x)-f(x)\right]\left[F'(x)-f'(x)\right]-2\left[F(x)-f(x)\right]\left[F'(x)-f'(x)\right]\\\\
		      &=& 0.
	    \end{array}
	    $$
	    para todo $x\in J$. Por lo tanto, $h(x)=c$ para algúna cosntante $c$. Entonces, para $h(0)$, usando las relaciones $f(0)=F(0)=0$ y $g(0)=G(0)=1,$ tenemos
	    $$
	    \begin{array}{rcl}
		h(0) &=& c\\\\
		     &=& \left[F(0)-f(0)\right]^2+\left[G(0)-g(0)\right]^2\\\\
		     &=& 0 + (1-1)^2\\\\
		     &=& 0.
	    \end{array}
	    $$
	    Así, $h(x)=0$ para todo $x\in J$. Luego, como es una suma de cuadrados (que deben ser no negativos), tenemos que $h(x)=0$ sí, y sólo si
	    $$F(x)-f(x)=0\quad \mbox{y}\quad G(x)-g(x)=0$$
	    para todo $x\in J$. Concluimos que
	    $$F(x)=f(x)\quad \mbox{y}\quad G(x)=g(x).$$\\\\
	
	%---------- c)
	\item ¿Qué más se puede decir acerca de las funciones $f$ y $g$ que satisfacen (5.30)?.\\\\
	    Respuesta.-\; Dado que hemos establecido que $\sen x$ y $\cos x$ satisfacen estas propiedades y que cualquier función que satisfaga estas propiedades es única, podemos concluir que $\sen x$ y $\cos x$ son las únicas funciones que satisfacen las propiedades dadas.\\\\

    \end{enumerate}

    %-------------------- 6.
    \item Una función $f$, definida para todo número real positivo, satisface la ecuación $f\left(x^2\right)=x^3$ para cada $x>0$. Determinar $f'(4)$.\\\\
	Respuesta.-\; Tomemos la derivada de ambos lados de la ecuación dada, usando la regla de la cadena.
	$$
	\begin{array}{rcl}
	    f\left(x^2\right)=x^3 &\Rightarrow& \left[f\left(x\right)^2\right]' = \left(x^3\right)'\\\\
	    &\Rightarrow& 2xf'\left(x^2\right) = 3x^2\\\\
	    &\Rightarrow& f'\left(x^2\right) = \dfrac{3}{2} x.
	\end{array}
	$$
	Asumiendo que $x$ es un número no negativo. Entonces,
	$$f'(4)=f'\left(2^2\right)=\dfrac{3}{2}\cdot 2=3.$$\\

    %-------------------- 7.
    \item Una función $g$ definida para todo número real positivo satisface las dos condiciones siguientes: $g(1)=1$ y $g'\left(x^2\right)=x^3$ para todo $x>0$. Calcular $g(4)$.\\\\
	Respuesta.-\; Primero multipliquemos la ecuación dada, por $2x$.
	$$g'\left(x^2\right)=x^3\quad \Rightarrow \quad 2xg'\left(x^2\right)=2x^4.$$
	Donde, 
	$$g'\left[f(x)\right]'=\left[g\left(x^2\right)\right]'=2xg'\left(x^2\right).$$
	Así, podemos integrar ambos lados de la ecuación,
	$$
	\begin{array}{rcl}
	    2xg'\left(x^2\right) = 2x^4 &\Rightarrow & \displaystyle\int 2xg'\left(x^2\right)\; dx = \displaystyle\int 2x^4\; dx\\\\
					&\Rightarrow & \displaystyle\int \left[g\left(x^2\right)\right]'\; dx = \dfrac{2}{5}x^5+C\\\\
					&\Rightarrow & g\left(x^2\right) = \dfrac{2}{5}x^5+C.
	\end{array}
	$$
	Ya que, $g(1)=1$. Entonces
	$$g(1)=1=\dfrac{2}{5}\cdot 1^5+C\quad \Rightarrow \quad C=\dfrac{3}{5}.$$
	De donde,
	$$g\left(x^2\right)=\dfrac{2}{5}x^5+\dfrac{3}{5}.$$
	Así, calculando $g\left(4\right)=g\left(2^2\right)$, tenemos
	$$g\left(4\right)=\dfrac{2}{5}2^5+\dfrac{3}{5}=\dfrac{67}{5}.$$\\


    %-------------------- 8.
    \item Demostrar que
    $$\int_0^x \dfrac{\sen t}{t+1}\; dt \geq 0\mbox{ para todo }x\geq 0.$$\\\\
	Demostración.-\; Recordemos el segundo teorema de valor medio para las integrales (Teorema 5.5). Para una función continua $g$ en el intervalo $[a,b]$. Si $f$ tiene derivadas continuas el cual nunca cambian de signo, en dicho intervalo, entonces existe un $c\in[a,b]$ tal que
	$$\int_a^b f(x)g(x)\;dx =f(a)\int_a^c g(x)\; dx + f(b)\int_c^b g(x)\; dx.$$\\
	Apliquemos ahora lo dicho con $g(t)=\sen t$ y $f(t)=\dfrac{1}{t+1}$. Ya que $\sen t$ es continuo para todo los reales, es continuo para el intervalo $[a,x]$. De donde,
	$$f'(t)=-\dfrac{1}{(1+t)^2}$$
	es continua para todo $t\neq -1$, y por ende continuo para todo $t\geq 0$. Además, ya que $(1+t)^2>0$ para todo $t\geq 0$, tenemos que $f'(t)<0$ para todo $t\geq 0$. Por lo tanto, $f'(t)$ es continua y nunca cambia de signo en el intervalo $[0,x]$. Así, podemos aplicar el teorema de valor medio; Existe un $c\in[0,x]$ para cualquier $x\geq 0$ tal que
	$$
	\begin{array}{rcl}
	    \displaystyle\int_0^x \left(\dfrac{1}{1+t}\right)\sen t\; dt &=& \dfrac{1}{1+0}\displaystyle\int_0^c \sen t\; dt + \dfrac{1}{1+1}\int_c^x \sen t\; dt\\\\
									 &=& (-\cos t)\bigg|_0^c+\dfrac{1}{2}(-\cos t)\bigg|_c^x\\\\
									 &=& 1-\cos c+\dfrac{1}{2}\cos c-\dfrac{1}{2}\cos x\\\\
									 &=&1-\dfrac{1}{2}(\cos c+\cos x).
	\end{array}
	$$
	Pero como $\cos x\leq 1$ para todo $x$, sabemos que $c+\cos x\leq 2$ para cualesquier $c$ y $x$. Por lo tanto,
	$$\int_0^x \dfrac{\sen t}{1+t}\; dt = 1-\dfrac{1}{2}(\cos x+\cos x)\geq 0.$$\\


    %-------------------- 9.
    \item Sean $C_1$ y $C_2$ dos curvas que pasan por el origen tal como se indica en la figura 5.2. Una curva $C$ se dice que biseca en área la región $C_1$ y $C_2$, si para cada punto $P$ de $C$ las dos regiones $A$ y $B$ sombreadas en la figura, tienen la misma área. Determinar la curva superior $C_2$, sabiendo que la curva bisectriz $C$ tiene de ecuación $y=x^2$ y que la curva inferior $C_1$ tiene de ecuación $y=\dfrac{1}{2}x^2.$\\\\
	Respuesta.-\; Primero, calculemos el área $A$ dada por:
	$$
	\begin{array}{rcl}
	    \displaystyle\int_0^t x^2\; dx - \int_0^t \dfrac{x^2}{2}\;dx &=& \displaystyle\int_0^t\left(x^2-\dfrac{x^2}{2}\right)\; dx\\\\
									 &=& \displaystyle\int_0^t\dfrac{x^2}{2}\; dx\\\\
									 &=& \dfrac{x^3}{6}\Bigg|_0^t\\\\
									 &=& \dfrac{t^3}{6}.
	\end{array}
	$$
	Ahora, supongamos que la ecuación para $C_2$ es de la forma $kx^2$ para $k\in \mathbb{R}^+$. Entonces, encontremos el área de la región $B$, encontrando el área de las curvas $C$ y $C_2$ en terminos de $y$, de la siguiente manera:
	$$
	\begin{array}{rcl}
	    C:y=x^2 &\Rightarrow & x=\sqrt{y}\\\\
	    C_2:y=kx^2 &\Rightarrow & x=\dfrac{1}{\sqrt{k}}\sqrt{y}.\\\\
	\end{array}
	$$
	Por lo que, integrando de $0$ a $t^2$. En área de $B$ es dado por:
	$$
	\begin{array}{rcl}
	    \displaystyle\int_0^{t^2}\left(\sqrt{y}-\dfrac{1}{\sqrt{k}}\right)\; dy &=& \left(1-\dfrac{1}{\sqrt{k}}\right)\displaystyle\int_0^{t^2} \sqrt{y}\; dy\\\\
										    &=& \left(1-\dfrac{1}{\sqrt{k}}\right)\left(\dfrac{2}{3}y^{\frac{3}{2}}\bigg|_0^{t^2}\right)\\\\
										    &=& \left(1-\dfrac{1}{\sqrt{k}}\right)\left(\dfrac{2}{3}t^{\frac{3}{2}}\right)\\\\
										    &=& \dfrac{2t^3}{3}-\dfrac{2t^3}{3\sqrt{k}}.
	\end{array}
	$$
	Ahora, igualamos las áreas de las regiones, de la siguiente manera:
	$$
	\begin{array}{rcl}
	    \dfrac{t^3}{6}=\dfrac{2t^3}{3}-\dfrac{2t^3}{3\sqrt{k}} &\Rightarrow & \dfrac{1}{6}=\dfrac{2}{3}-\dfrac{2}{3\sqrt{k}}\\\\
								   &\Rightarrow & \dfrac{2}{3\sqrt{k}}=\dfrac{1}{2}\\\\
								   &\Rightarrow & 4=3\sqrt{k}\\\\
								   &\Rightarrow & k=\dfrac{16}{9}.
	\end{array}
	$$
	Por lo tanto, la ecuación de $C_2$ es 
	$$y=\dfrac{16}{9}x^2,$$\\


    %-------------------- 10.
    \item Una función $f$ está definida para todo $x$ como sigue:
    $$
    f(x)=
    \left\{
	\begin{array}{rcl}
	    x^2 & \mbox{si} & x \mbox{ es racional}\\
	    0 & \mbox{si} & x \mbox{ es irracional}.\\
	\end{array}
    \right.
    $$
    Póngase $Q(h)=\frac{f(h)}{h}$ si $h\neq 0.$ 

    \begin{enumerate}[a)]

	%---------- a)
	\item Demostrar que $Q(h)\to 0$ cuando $h\to 0.$\\\\
	    Demostración.-\; Sea $\epsilon>0$. Entonces, elijamos $0<\delta=\epsilon$. Por lo que si
	$$|h-0|<\delta\quad \Rightarrow \quad |h|<\delta$$
	tenemos
	$$
	\bigg|\dfrac{Q(h)}{h}-0\bigg|=
	\left\{
	    \begin{array}{rcl}
		\dfrac{h^2}{|h|}=|h| &\mbox{si}& h\in \mathbb{Q},\\
		0 &\mbox{si}& h\in \mathbb{R}-\mathbb{Q}.
	    \end{array}
	\right.
	$$
	Así, encontramos un $\delta>0$ tal que $\bigg|\dfrac{Q(h)}{h}\bigg|<\epsilon$ cuando $|h-0|<\delta$. Por lo tanto,
	$$\lim_{h\to 0}\dfrac{Q(h)}{h}=0.$$\\

	%---------- b)
	\item Demostrar que $f$ tiene derivada en $0$ , y calcular $f'(0)$.\\\\
	    Demostración.-\; Para mostrar que $f$ tiene una derivada en $0$ debemos demostrar el límite:
	    $$\lim_{h\to 0}\dfrac{f(0+h)-f(0)}{h}$$
	    existe. Notenemos que $f(0)=0$, ya que $0^2=0$. Entonces por la parte (a),
	    $$
	    \begin{array}{rcl}
		f'(0) &=& \displaystyle\lim_{h\to 0}\dfrac{f(0+h)-f(0)}{h}\\\\
		      &=& \displaystyle\lim_{h\to 0}\dfrac{f(h)-f(0)}{h}\\\\
		      &=& \displaystyle\lim_{h\to 0}\dfrac{f(h)}{h}\\\\
		      &=& 0.
	    \end{array}
	    $$
	    \vspace{.5cm}

    \end{enumerate}

    %-------------------- 11.
\item $\displaystyle\int (2+3x)\sen 5x\; dx.$\\\\
	Respuesta.-\; Sea,
	$$\int (2+3x)\sen(5x)\; dx = 2\int \sen(5x)\; dx + \int x\sen (5x)\; dx.$$
	Luego,
	$$u=5x\; \Rightarrow \; du=5\; dx$$
	Entonces,
	$$
	\begin{array}{rcl}
	    2\displaystyle\int \sen(5x)\; dx &=& \dfrac{2}{5}\displaystyle\int \sen u\; du\\\\
					     &=& -\dfrac{2}{5}\cos u + C\\\\
					     &=& -\dfrac{2}{5}\cos (5x) + C.
	 \end{array}
	$$

	Después,
	$$
	\begin{array}{rcl}
	    u=x &\Rightarrow& du=dx\\\\
	    dv = \sen(5x)\; dx &\Rightarrow& v = -\dfrac{1}{5}\cos(5x)
	\end{array}
	$$
	Entonces,
	$$
	\begin{array}{rcl}
	    3\displaystyle\int x\sen(5x)\; dx &=& 3 \displaystyle\int u \; dv\\\\
					      &=& 3\left(uv-\displaystyle\int v\; du\right)\\\\
					      &=& 3\left(-\dfrac{x}{5}\cos(5x)+\dfrac{1}{25}\sen(5x)\right)+C\\\\
					      &=& \dfrac{4}{25}\sen(5x)-\dfrac{3x}{5}\cos (5x) +C.
	\end{array}
	$$
	Por lo tanto,
	$$\int(2+3x)\sen(5x)\; dx = -\dfrac{2}{5}\cos(5x) + \dfrac{4}{25}\sen(5x)-\dfrac{3x}{5}\cos(5x) +C.$$\\

    %-------------------- 12.
    \item $\displaystyle\int x \sqrt{1+x^2}\; dx.$\\\\
	Respuesta.-\; Sea,
	$$u=1+x^2\quad \Rightarrow \quad du=2x\; dx.$$
	Entonces,
	$$
	\begin{array}{rcl}
	    \displaystyle\int x\sqrt{1+x^2}\; dx &=& \dfrac{1}{2}\displaystyle\int \sqrt{u}\; du\\\\
						 &=& \dfrac{1}{2}\left(\dfrac{1}{3}u^{\frac{3}{2}}\right)+C\\\\
						 &=&\dfrac{1}{3}\left(1+x^2\right)^{\frac{3}{2}}+C.
	 \end{array}
	$$
	\vspace{.5cm}

    %-------------------- 13.
    \item $\displaystyle\int_{-2}^1 x\left(x^2-1\right)^9\; dx.$\\\\
	Respuesta.-\; Sea,
	$$u=x^2-1\quad \Rightarrow \quad du=2x.$$
	Luego,
	$$u(-2)=3\quad \mbox{y} \quad u(1)=0.$$

	Así, por el teorema de sustitución para integrales tenemos

	$$
	\begin{array}{rcl}
	    \displaystyle\int_{-2}^1 x\left(x^2-1\right)^9\; dx &=&\dfrac{1}{2}\displaystyle\int_3^0 u^9\; du\\\\ 
								&=& \dfrac{1}{2}\left(\dfrac{1}{10}u^{10}\bigg|_3^0\right)\\\\
								&=& -\dfrac{3^10}{20}.
	\end{array}
	$$
	\vspace{.5cm}

    %-------------------- 14.
    \item $\displaystyle\int_0^1 \dfrac{2x+3}{(6x+7)^3}\; dx.$\\\\
	Respuesta.-\; Primero, transformemos la integral de la siguiente manera:
	$$
	\begin{array}{rcl}
	    \displaystyle\int_0^1 \dfrac{2x+3}{(6x+7)^3}\; dx &=& \dfrac{1}{3}\displaystyle\int_0^1 \dfrac{6x+9}{(6x+7)^3}\; dx\\\\
							      &=& \dfrac{1}{3}\displaystyle\int_0^1 \dfrac{6x+7}{(6x+7)^3}\;dx+\dfrac{1}{3}\int_0^1 \dfrac{2}{(6x+7)^3}\; dx\\\\
							      &=& \dfrac{1}{18}\displaystyle\int_0^1 \dfrac{6}{(6x+7)^2}\;dx+\dfrac{1}{9}\displaystyle\int_0^1 \dfrac{6}{(6x+7)^3}\; dx\\\\
	\end{array}
	$$
	Ahora, usando la integración por partes, 

	$$u=6x+7\quad \Rightarrow \quad du=6'; dx.$$

	Y por el teorema de sustitución para integrales tenemos

	$$u(0)=7\quad \mbox{y} \quad u(1)=13.$$

	Por lo tanto,

	$$
	\begin{array}{rcl}
	    \displaystyle\int_0^1 \dfrac{2x+3}{(6x+7)^3}\; dx &=& \dfrac{1}{18}\displaystyle\int_7^{13} \dfrac{1}{u^2}\; du  + \dfrac{1}{9}\int_7^{13} \dfrac{1}{u^3}\; dx\\\\
							      &=& \dfrac{1}{18}\left(-\dfrac{1}{u}\bigg|_7^{13}\right) + \dfrac{1}{9}\left(-\dfrac{1}{u^2}\bigg|_7^{13}\right)\\\\
							      &=& \dfrac{1}{18}\left(-\dfrac{u+1}{u^2}\bigg|_7^{13}\right)\\\\
							      &=& \dfrac{1}{18}\left(\dfrac{8}{49}-\dfrac{14}{169}\right)\\\\
							      &=& \dfrac{37}{8281}.
	\end{array}
	$$
	\vspace{.5cm}
	

    %-------------------- 15.
    \item $\displaystyle\int x^4\left(1+x^5\right)^5\; dx.$\\\\
	Respuesta.-\; Utilizando la integración por partes,
	$$u=1+x^5\quad \Rightarrow \quad du=5x^4\; dx.$$
	Entonces,
	$$
	\begin{array}{rcl}
	    \displaystyle\int x^4\left(1+x^5\right)^5\; dx &=& \dfrac{1}{5} \displaystyle\int 5 x^4\left(1+x^5\right)^5\; dx\\\\
							   &=& \dfrac{1}{5}\displaystyle\int u^5\; du\\\\
							   &=& \dfrac{1}{5}\left(\dfrac{1}{6}u^6\right)\\\\
							   &=& \dfrac{1}{30}(1+x^5)^6.

	\end{array}
	$$
	\vspace{.5cm}

    %-------------------- 16.
    \item $\displaystyle\int_0^1 x^4(1-x)^{20}\; dx.$\\\\
	Respuesta.-\; Primero, sustituimos $u=1-x$ y $du=-dx$ para luego aplicar la integración por partes, de la siguiente manera:
	$$u(0)=1\quad \mbox{y}\quad u(1)=0.$$
	Entonces,
	$$
	\begin{array}{rcl}
	    \displaystyle\int_0^1 x^4(1-x)^{20}\; dx &=& \displaystyle\int_1^0 -(1-u)^4u^{20}\; du\\\\
						     &=& \displaystyle\int_0^1 (1-u)^{4}u^{20}\; du.
	\end{array}
	$$
	De donde,
	$$(1-u)^4y^{20}=u^{20}\left(1-4u+6u^2-4u^3+u^4\right)=u^{24}-4u^{23}+6u^{22}-4u^{21}+u^{20}.$$

	Por lo tanto,
	$$
	\begin{array}{rcl}
	    \displaystyle\int_0^1 \left(1-u\right)^4u^{20}\; du &=& \displaystyle\int_0^1 u^{24}-4u^{23}+6u^{22}-4u^{21}+u^{20}\; du\\\\
								&=& \left(\dfrac{u^{25}}{25}-\dfrac{u^{24}}{6}+\dfrac{6u^{23}}{23}-\dfrac{2u^{22}}{11}+\dfrac{u^{21}}{21}\right)\bigg|_0^1\\\\
								&=& \dfrac{1}{25}-\dfrac{1}{6}+\dfrac{6}{23}-\dfrac{2}{11}+\dfrac{1}{21}\\\\
								&=& \dfrac{1}{265650}.
	\end{array}
	\vspace{.5cm}
	$$

    %-------------------- 17.
    \item $\displaystyle\int_1^2 x^{-2} \sen \dfrac{1}{2}\; dx.$\\\\
	Respuesta.-\; Utilizando la integración por partes,

	$$u=\dfrac{1}{x}\quad \Rightarrow \quad du=-\dfrac{1}{x^2}\;dx.$$

	Y luego, utilizando el teorema de sustitución para integrales,

	$$u(1)=1\quad \mbox{y}\quad u(2)=\dfrac{1}{2}.$$

	Entonces,

	$$
	\begin{array}{rcl}
	    \displaystyle\int_1^2 \dfrac{1}{x^2}\sen\left(\dfrac{1}{x}\right)\; dx &=& -\displaystyle\int_1^{\frac{1}{2}} \sen u\; du\\\\
										   &=& \displaystyle\int_{\frac{1}{2}}^1 \sen u\; du\\\\
										   &=& \left(-\cos u\right)\bigg|^{\frac{1}{2}}_1\\\\
										   &=& \cos\dfrac{1}{2}-\cos 1.
	\end{array}
	$$
	\vspace{.5cm}



    %-------------------- 18.
    \item $\displaystyle\int x\sen x^2 \cos x^2\; dx.$\\\\
	Respuesta.-\;

    %-------------------- 19.
    \item $\displaystyle\int x\sen x^2 \cos x^2\; dx.$\\\\
	Respuesta.-\;

    %-------------------- 20.
    \item $\displaystyle\int \sqrt{1+3\cos^2 x \sen 2x}\; dx.$\\\\
	Respuesta.-\;

    %-------------------- 21.
    \item Demostrar que el valor de la integral $\int_0^2 375x^5\left(x^2+1\right)^{-4}\; dx$ es $2^n$ para un cierto entero $n$.\\\\
	Demostración.-\;

    %-------------------- 22.
    \item Determinar un par de números $a$ y $b$ para los cuales $\in_0^1(ax+b)\left(x~ 2+3x+2\right)^{-2}\; dx=\frac{3}{2}$.\\\\
	Respuesta.-\;

    %-------------------- 23.
    \item Sea $I_n = \int_0^1 \left(1-x^2\right)^n\; dx$. Demostrar que $(2n+1)I_n=2nI_{n-1}$, y utilizar esta relación para $I_2$, $I_3$, $I_4$ e $I_5$.\\\\
	Demostración.-\;

    %-------------------- 24.
    \item Sea $F(m,n)=\int_0^x t^m(1+t)^n \; dt,$ $m>0$, $n>0$. Demostrar que
    $$(m+1)F(m,n)+nF(m+1,n-1)=x^{m+1}(1+x)^n.$$
    Utilizar este resultado para calcular $F(10,2)$.\\\\
	Demostración.-\;

    %-------------------- 25.
    \item Sea $f(n)=\int_0^{\pi/4}x\; dx$ donde $n\geq 1$. Demostrar que

    \begin{enumerate}[(a)]

	%---------- (a)
	\item $f(n+1)<f(n)$.\\\\
	    Demostración.-\;

	%---------- (b)
	\item $f(n)+f(n-2)=\dfrac{1}{n-1}\quad \mbox{si} n>2$.\\\\
	    Demostración.-\;

	%---------- (c)
	\item $\dfrac{1}{n+1}<2f(n)<\dfrac{1}{n-1}\quad \mbox{si} n>2$.\\\\
	    Demostración.-\;

    \end{enumerate}

    %-------------------- 26.
    \item Calcular $f(0)$, sabiendo que $f(\pi)=2$ y que $\int_0^\pi \left[f(x)+f''(x)\right]\sen x\; dx = 5.$\\\\
	Respuesta.-\;

    %-------------------- 27.
    \item Designar por $A$ el valor de la integral
    $$\int_0^\pi \dfrac{\cos x}{\left(x+2\right)^2}\; dx.$$
    Y calcular la siguiente inetegral en función de $A$:
    $$\int_0^{\pi/2}\dfrac{\sen x\cos x}{x+1}\; dx.$$
	Respuesta.-\;

    %-------------------- 28.
    \item $\displaystyle\int \dfrac{\sqrt{a+bx}}{x}\; dx = 2\sqrt{a+bx}+a\int \dfrac{dx}{x\sqrt{a+bx}}+C$.\\\\
	Respuesta.-\;

    %-------------------- 29.
    \item $\displaystyle\int x^n \sqrt{ax+b}\; dx = \dfrac{2}{a(2n+3)}\left[x^n(ax+b)^{3/2}-nb\int x^{n-1}\sqrt{ax+b}\; dx\right]+C\left(n\neq -\dfrac{3}{2}\right)$.\\\\
	Respuesta.-\;

    %-------------------- 30.
    \item $\displaystyle\int \dfrac{x^m}{\sqrt{a+bx}}\; dx = \dfrac{2}{(2m+1)b}\left[x^m \sqrt{a+bx}-ma\int \dfrac{x^{m-1}}{\sqrt{a+bx}}\; dx\right]+C \left(m\neq -\dfrac{1}{2}\right)$.\\\\
	Respuesta.-\;

    %-------------------- 31.
    \item $\displaystyle\int \dfrac{dx}{x^n\sqrt{ax+b}}\; dx = -dfrac{\sqrt{ax+b}}{(n-1)bx^{n-1}}-\dfrac{(2n-3)a}{(2n-2)b}\int \dfrac{dx}{x^{n-1}\sqrt{ax+b}}\; dx + C(n\neq 1)$.\\\\
	Respuesta.-\; 

    %-------------------- 32.
    \item $\displaystyle\int \dfrac{\cos^m x}{\sen^n x}\; dx = \dfrac{\cos^{m-1}x}{(m-n)\sen^{n-1}x}+\dfrac{m-1}{m-n}\int \dfrac{\cos^{m-2}}{\sen^n x}\; dx + C (m\neq n)$.\\\\
	Respuesta.-\;

    %-------------------- 33.
    \item $\displaystyle\int \dfrac{\cos^m x}{\sen^n x}\; dx = -\dfrac{\cos^{m+1}x}{(n-1)\sen^{n-1}x}-\dfrac{m-n+2}{n-1}\int \dfrac{\cos^m x}{\sen^{n-2}x}\; dx + C (n\neq 1)$.\\\\
	Respuesta.-\;

    %-------------------- 34.
    \item 
	\begin{enumerate}[a)]

	    %---------- a)
	    \item Encontrar un polinomio $P(x)$ tal que $P'(x)-3P(x)=4-5x+3x^2$. Demostrar que existe una sola solución.\\\\
		Demostración.-\;

	    %---------- b)
	    \item Si $Q(x)$ es un polinomio dado, demostrar que existe uno y sólo un polinomio $P(x)$ tal que $P'(x)-3P(x)=Q(x)$.\\\\
		Demostración.-\;

	\end{enumerate}

    %-------------------- 35.
    \item Una sucesión de polinomios (llamados polinomios de Bernoulli) se define por inducción como sigue:

	\begin{enumerate}[a)]

	    %---------- a)
	    \item Determinar fórmulas explícitas para $P_1(x),P_2(x),\ldots , P_5(x)$.\\\\
		Respuesta.-\;

	    %---------- b)
	    \item Demostrar, por inducción, que $P_n(x)$ es un polinomio en $x$ de grado $n$, siendo el término de mayor grado $x_n$.\\\\
		Demostración.-\;

	    %---------- c)
	    \item Demostrar que $P_n(0)=P_n(1)$ si $n\geq 2$.\\\\
		Demostración.-\;

	    %---------- d)
	    \item Demostrar que $P_n(x+1)-P_n(x)=nx^{n-1}$ si $n\geq 1$.\\\\
		Demostración.-\;

	    %---------- e)
	    \item Demostrar que para $n\leq 2$ tenemos 
	    $$\sum_{r=1}^{k-1}r^n = \int_0^k P_n(x)\; dx = \dfrac{P_{n+1}(k)-P_{n+1}(0)}{n+1}.$$\\
		Demostración.-\;

	    %---------- f)
	    \item  Demostrar que $P_n(1-x)=(-1)^n P_n(x)$ si $n\geq 1$.\\\\
		Demostración.-\;

	    %---------- g)
	    \item Demostrar que $P_{2n+1}(0)=0$ y $P_{2n-1}\left(\dfrac{1}{2}\right)=0$ si $n\geq 1.$\\\\
		Demostración.-\;

	\end{enumerate}

    %-------------------- 36.
    \item Suponiendo que $|f''(x)|\leq m$ para cada $x$ en el intervalo $[0,a]$, y que $f$ toma su mayor valor en un punto interno de este interval, demostrar que $|f'(0)|+|f'(a)|\leq am$. Puede ponerse que $f''$ sea continua en $[0,a]$.
	Demostración.-\; 

\end{enumerate}

