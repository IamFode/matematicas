\chapter{Relación entre integración y derivación}

\section{La derivada de una integral indefinida. Primer teorema fundamental del cálculo}

\begin{teo}[Primer Teorema fundamental del cálculo]
    Sea $f$ una función que es integrable en $[a,x]$ para cada $x$ en $[a,b]$. Sea $c$ tal que $a\leq c \leq b$ y definamos una nueva función $A$ del siguiente modo:
    $$A(x)=\int_c^x f(t)\;dt \quad \mbox{si}\quad a\leq x \leq b$$
    Existe entonces la derivada $A'(x)$ en cada punto $x$ del intervalo abierto $(a,b)$ donde $f$ es continua, y para tal $x$ tenemos
    \begin{equation}
	A'(x)=f(x)
    \end{equation}
    Interpretación geométrica: La figura 5.1 (Apostol, capítulo 5) muestra la gráfica de una función $f$ en un intervalo $[a,b]$. En la figura, $h$ es positivo y
    $$\int_x^{x+h}f(t)\; dt=\int_0^{x+h}f(t)\; dt -\int_0^x f(t)\; dt= A(x+h)-A(x).$$
    El ejemplo es el de una función continua en todo el intervalo $[x,x+h]$. Por consiguiente, por el teorema del valor medio para integrales, tenemos
    $$A(x+h)-A(x)=hf(z),\quad \mbox{donde}\; x\leq z \leq x+h.$$
    Luego, resulta
    \begin{equation}
	\dfrac{A(x+h)-A(x)}{h}=f(z).
    \end{equation}
    Puesto que $x\leq z \leq x+h,$ encontramos que $f(z)\to f(x)$ cuando $h\to 0$ con valores positivos. Si $h\to 0$ con valores negativos, se razona en forma parecida. Por consiguiente, $A'(x)$ existe y es igual a $f(x)$.\\
	Demostración.-\; Sea $x$ un punto en el que $f$ es continua y supuesta $x$ fija, se forma el cociente:
	$$\dfrac{A(x+h)-A(x)}{h}.$$
	Para demostrar el teorema se ha de probar que este cociente tiende a $f(x)$ cuando $h\to 0$. El numerador es:
	$$A(x+h)-A(x)=\int_0^{x+h}f(t)\; dt-\int_c^x f(t)\; dt = \int_x^{x+h}f(t)\; dt.$$
	Si en la última integral se escribe $f(t)=f(x)-[f(t)-f(x)]$ resulta:
	$$\begin{array}{rcl}
	    A(x+h)-A(x) & = & \displaystyle \int_x^{x+h}f(x)\; dt + \int_x^{x+h}\left[f(t)-f(x)\right]\; dt\\\\
			& = & \displaystyle hf(x) + \int_x^{x+h} \left[f(t)-f(x)\right]\; dt,
	\end{array}$$
	de donde
	\begin{equation}
	    (5.3) \qquad \qquad \dfrac{A(x+h)-A(x)}{h}=f(x)+\dfrac{1}{h}\int_{x}^{x+h}\left[f(t)-f(x)\right]\; dt.
	\end{equation}
	Por tanto, para completar la demostración de (5.1) es necesario demostrar que 
	$$\lim_{h\to 0}\dfrac{1}{h}\int_x^{x+h}\left[f(t)-f(x)\right]\; dt = 0.$$
	En esta parte de la demostración es donde se hace uso de la continuidad de $f$ en $x$.\\
	Si se designa por $G(h)$ el último término del segundo miembro de (5.3), se trata de demostrar que $G(h)\to 0$ cuando $h\to 0$. Aplicando la definición de límite, se ha de probar que para cada $\epsilon>0$ existe un $\delta>0$ tal que
	\begin{equation}
	    G(h)<\epsilon \mbox{ siempre que } 0<h<\delta.
	\end{equation}
	En virtud de la continuidad de $f$ en $x$, dado un $\epsilon$ existe un número positivo $\delta$ tal que:
	\begin{equation}
	|f(t)-f(x)|<\dfrac{1}{2}\epsilon.
	\end{equation}
	siempre que 
	\begin{equation}
	x-\delta<t<x+\delta.
	\end{equation}
	Si se elige $h$ de manera que $0<h<\delta$, entonces cada $t$ en el intervalo $[x,x+h]$ satisface (5.6) y por tanto (5.5) se verifica para cada $t$ de este intervalo. Aplicando la propiedad $|\int_x^{x+h}|\leq \int_x^{x+h}|g(t)|\; dt$, cuando $g(t)=g(t)-f(x)$, de la desigualdad en (5.5) se pasa a la relación:
	$$\left|\right|\leq \int_x^{x+h}|f(t)-f(x)|\; dt \leq \int_x^{x+h}\dfrac{1}{2} \epsilon \; dt = \dfrac{1}{2}h\epsilon < h\epsilon.$$
	Dividiendo por $h$ se ve que (5.4) se verifica para $0<h<\delta$. Si $h<0$, un razonamiento análogo demuestra que (5.4) se verifica siempre que $0<|h|<\delta$, lo que completa la demostración.
\end{teo}

%%%%%%%%%%%%%%%%%%%%%%%%%% 5.2. Teorema de la derivada nula %%%%%%%%%%%%%%%%%%%%%%%%%%%%%%%%
\section{Teorema de la derivada nula}

Si una función $f$ es constante en un intervalo $(a, b)$, su derivada es nula en todo el intervalo $(a, b)$. Ya hemos demostrado este hecho como una consecuencia inmediata de la definición de derivada. También se demostró, como parte c) del teorema 4.7, el recíproco de esa afirmación que aquí se presenta como teorema independiente.

\begin{teo}[Teorema de la derivada nula]
    Si $f'(x)=0$ para cada $x$ en un intervalo abierto $I$, es $f$ constante en $I$.
\end{teo}
\vspace{0.5cm}

Este teorema, cuando se utiliza combinando con el primer teorema fundamental del cálculo, nos conduce al segundo teorema fundamental.

\section{Funciones primitivas y segundo teorema fundamental del cálculo}
\begin{def.}[Definición de función primitiva]
    Una función $P$ se llama primitiva (o antiderivada) de una función $f$ en un intervalo abierto $I$ si la derivada de $P$ es $f$, esto es, si $P'(x)=f(x)$ para todo $x$ en $I.$
\end{def.}

Decimos una primitiva y no la primitiva, porque si $P$ es una primitiva de $f$ también lo es $P+k$ para cualquier constante $k$. Recíprocamente, dos primitivas cualesquiera $P$ y $Q$ de la misma función $f$ sólo pueden diferir en una constante por que su diferencia $P-Q$ tiene la derivada
$$P'(x)-Q'(x)=f(x)-f(x)=0.$$
para toda $x$ en $I$ y por tanto, según el teorema $5.2$

%-------------------- Teorema 5.3. Segundo teorema fundamental del cálculo --------------------
\begin{teo}[Segundo teorema fundamental del cálculo]
    Supongamos $f$ continua en un intervalo abierto $I$, y sea $P$ una primitiva cualquiera de $f$ en $I$. Entonces, para cada $c$ y cada $x$ en $I$, tenemos 
    $$P(x)=P(c)+\int_c^x f(t)\; dt.$$
	Demostración.-\; Sea $A(x)=\int_c^x f(t)\; dt$. Puesto que $f$ es continua en cada $x$ de $I$, el primer teorema fundamental nos dice que $A'(x)=f(x)$ para todo $x$ de $I$. Es decir, $A$ es primitiva de $f$ en $I$. Luego, ya que dos primitivas de $f$ pueden diferir tan sólo en una constante, debe ser $A(x)-P(x)=k$ para una cierta constante $k$. Cuando $x=c$ esta fórmula implica $-P(c)=k,$ ya que $A(c)=0.$ Por consiguiente, $A(x)-P(x)=-P(c)$, de lo que obtenemos $P(x)=P(c)+\int_c^x f(t)\; dt.$ es constante en $I$.
\end{teo}

El teorema 5.3 nos indica cómo encontrar una primitiva $P$ de una función continua $f$. Integrando $f$ desde un punto fijo $c$ a un punto arbitrario $x$ y sumando la constante $P(c)$ obtenemos $P(x)$. Pero la importancia real del teorema radica en que poniendo $P(x)=P(c)+\int_c^x f(t)\; dt$ en la forma
$$\int_c^x f(t)\; dt = P(x)-P(c).$$

Se ve que podemos calcular el valor de una integral mediante una simple substracción si conocemos una primitiva $P$.\\\\

Como consecuencia del segundo teorema fundamental, se pueden deducir las siguientes fórmulas de integración.

\begin{ejem}[Integración de potencias recionales]
    La fórmula de integración
    $$\int_a^b x^n\; dx = \dfrac{b^{n+1}-a^{n+1}}{n+1}\qquad (n=0,1,2,\ldots)$$
    se demostró directamente en la Sección 1.23 (Spivak) a partir de la definición de integral. Aplicando el segundo teorema fundamental, puede hallarse de nuevo este resultado y además generalizarlo para exponentes racionales. En primer lugar se observa que la función $P$ definida por
    $$P(x)=\dfrac{x^{n+1}}{n+1}$$
    tiene como derivada $P'(x)=x^n$ para cada $n$ entero no negativo. De esta igualdad válida para todo número real $x$, aplicando $\int_c^x f(t)\; dt = P(x)-P(c)$ se tiene
    $$\int_a^b x^n\; dx  = P(b)-P(a)=\dfrac{b^{n+1}-a^{n+1}}{n+1}$$
    para cualquier intervalo $[a,b]$. Esta fórmula, demostrada para todo entero $n\geq 0$ conserva su validez para todo entero negativo excepto $n=-1$, que se excluye puesto que el denominador aparece $n+1$. Para demostrar $\int_a^b x^n\; dx = \frac{b^{n+1}-a^{n+1}}{n+1}\; (n=0,1,2,\ldots)$ para $n$ negativo, basta probar que $P(x)=\frac{x^{n+1}}{n+1}$ implica $P'(x)=x^n$ cuando $n$ es negativo $x\neq -1$, lo cual es fácil de verificar derivando $P$ como función racional. Hay que tener en cuenta que si $n$ es negativo se deben excluir aquellos intervalos $[a,b]$ que contienen el punto $x=0.$
\end{ejem}

El resultado del ejemplo 3 de la Sección 4.5, permite extender $\int_a^b x^n\; dx = \frac{b^{n+1}-a^{n+1}}{n+1}\; (n=0,1,2,\ldots)$ a todos los exponentes racionales (excepto $-1$) siempre que el integrando esté definido en todos los puntos del intervalo $[a, b]$ en consideración.

\begin{ejem}[Integración de seno y coseno]
    Puesto que la derivada del seno es el coseno y la del coseno menos el seno, el segundo teorema fundamental da las fórmulas siguientes:
    $$\int_a^b \cos x \; dx = \sen x\bigg|_a^b = \sen b -\sen a,$$
    $$\int_a^b \sen x \; dx = -\cos x\bigg|_a^b = -\cos b +\cos a.$$
\end{ejem}
    Estas fórmulas se conocían ya, pues se demostraron en el capítulo 2 a partir de la definición de integral. \\
    Se obtienen otras fórmulas de integración a partir de los ejemplos 1 y 2 tomando sumas finitas de términos de la forma $Ax^n$, $B \sen x$, $C cos x$, donde $A$, $B$, $C$ son constantes.


\section{Propiedades de una función deducida de propiedades de su derivada}
Si una función $f$ tiene derivada continua $f'$ en un intervalo abierto $I$, el segundo teorema fundamental afirma que 
$$f(x)=f(c)+\int_c^x f'(t)\; dt$$
cualesquiera que sean $x$ y $c$ en $I$. \\

\begin{prop}
Supóngase que $f'$ es continua y no negativa en $I$. Si $x>c$, entonces $\int_c^x f'(t)\; dt \geq 0$, y por tanto $f(x)\geq f(c)$. Es decir, si la derivada es continua y no negativa en $I$, la función es creciente en $I$.
\end{prop}

En el teorema 2.9 se demostró que la integral indefinida de una función creciente es convexa. Por consiguiente, si $f'$ es continua y creciente en $I$, la igualdad $f(x)=f(c)+\int_c^x f'(t)\; dt$ demuestra que $f$ es convexa en $I$. Análogamente, $f$ es cóncava en los intervalos en los que $f'$ es continua y decreciente.


\section{Ejercicios}

En cada uno de los Ejercicios del 1 al 10, encontrar una primitiva de $f$; es decir, encontrar una función $P$ tal que $P'(x)=f(x)$ y aplicar el segundo teorema fundamental para calcular $\int_a^b f(x)\; dx.$\\

\begin{enumerate}[\bfseries 1.]

    %-------------------- 1.
    \item $f(x)=5x^3$.\\\\
	Respuesta.-\; Por el ejemplo 5.1 (Apostol) se define a $P$ como,
	$$P(x)=\dfrac{x^{n+1}}{n+1}.$$
	Por lo que la función primitiva de $5x^3$, esta dada por:
	$$P(x)=5\dfrac{x^{3+1}}{3+1}=5\dfrac{x^4}{4}.$$
	Esta función es primitiva de $f$ ya que:
	$$P'(x)=5\dfrac{x^4}{4}=5x^3.$$
	Luego, aplicando el segundo teorema fundamental para calcular $\int_a^b f(x)\; dx$, tenemos
	$$\begin{array}{rcl}
	    \displaystyle\int_a^b 5x^3\; dx &=&P(b)-P(a)\\\\
					    &=&\dfrac{5}{4}b^4-\dfrac{5}{3}a^4.\\\\
					    &=&\dfrac{5}{4}(b^4-a^4).
	\end{array}$$
	\vspace{.5cm}

    %-------------------- 2.
    \item $f(x)=4x^4-12x$.\\\\
	Respuesta.-\; Por el ejemplo 5.1 (Apostol) se define a $P$ como,
	$$P(x)=\dfrac{x^{n+1}}{n+1}.$$
	Por lo que la función primitiva de $4x^4-12x$, esta dada por:
	$$\begin{array}{rcl}
	    P(x)&=&4\dfrac{x^{4+1}}{4+1}-12\dfrac{x^{1+1}}{1+1}\\\\
		&=&\dfrac{4}{5}x^5-6x^2.
	\end{array}$$
	Esta función es primitiva de $f$ ya que:
	$$\begin{array}{rcl}
	    f(x)&=&\dfrac{4}{5}x^5-6x^2\\\\
	    &=&4x^4-12x.
	\end{array}$$

    	Luego, aplicando el segundo teorema fundamental para calcular $\int_a^b f(x)\; dx$, tenemos
	$$\begin{array}{rcl}
	    \displaystyle\int_a^b 4x^4-12x\; dx &=&P(b)-P(a)\\\\
					    &=&\dfrac{4}{5}b^5-6b^2-\left(\dfrac{4}{5}a^5-6a^2\right)\\\\
					    &=&\dfrac{4}{5}(b^5-a^5)-6(b^2-a^2).
	\end{array}$$

    %-------------------- 3.
    \item $f(x)=(x+1)(x^3-2).$\\\\
	Respuesta.-\; Por el ejemplo 5.1 (Apostol) se define a $P$ como,
	$$P(x)=\dfrac{x^{n+1}}{n+1}.$$
	Por lo que la función primitiva de $(x+1)(x^3-2)=x^4+x^3-2x-2$, esta dada por:
	$$\begin{array}{rcl}
	    P(x)&=&\dfrac{x^{4+1}}{4+1}+\dfrac{x^{3+1}}{3+1}-2\dfrac{x^{1+1}}{1+1}-2\dfrac{x^{0+1}}{0+1}\\\\
		&=&\dfrac{x^5}{5}+\dfrac{x^4}{4}-2x^2-2x.
	\end{array}$$
	Esta función es primitiva de $f$ ya que,
	$$\begin{array}{rcl}
	    P'(x)&=&\dfrac{x^5}{5}+\dfrac{x^4}{4}-2x^2-2x\\\\
		&=&x^4+x^3-2x-2\\\\
		&=&(x+1)(x^3-2).
	\end{array}$$
	Luego, aplicando el segundo teorema fundamental para calcular $\int_a^b f(x)\; dx$, tenemos
	$$\begin{array}{rcl}
	    \displaystyle\int_a^b (x+1)(x^3-2)\; dx &=&P(b)-P(a)\\\\
					    &=&\dfrac{b^5}{5}+\dfrac{b^4}{4}-2b^2-2b-\left(\dfrac{a^5}{5}+\dfrac{a^4}{4}-2a^2-2a\right)\\\\
					    &=&\dfrac{1}{5}\left(b^5-a^5\right) + \dfrac{1}{4}\left(b^4-a^4\right)-\left(b^2-a^2\right)-2(b-a).
	\end{array}$$
	\vspace{.5cm}

    %-------------------- 4.
    \item $f(x)=\dfrac{x^4+x-3}{x^3}, \qquad x\neq 0$.\\\\
	Respuesta.-\; Podemos reescribir la función $f$ como:
	$$f(x)=\dfrac{x^4+x-3}{x^3} = \left(x^4+x-3\right)x^{-3} = x+x^{-2}-3x^{-3}.$$
	Por el ejemplo 5.1 (Apostol) se define a $P$ como,
	$$P(x)=\dfrac{x^{n+1}}{n+1}.$$
	Por lo que la función primitiva de $x+x^{-2}-3x^{-3}$, esta dada por:
	$$\begin{array}{rcl}
	    P(x)&=&\dfrac{x^{1+1}}{1+1}+\dfrac{x^{-2+1}}{-2+1}-\dfrac{3x^{-3+1}}{-3+1}\\\\
		&=&\dfrac{x^2}{2}-x^{-1}+\dfrac{3x^{-2}}{2}.
	\end{array}$$
	Esta función es primitiva de $f$ ya que,
	$$P'(x)=x+x^{-2}-3x^{-3}=\dfrac{x^4+x-3}{x^3}=f(x).$$
	Luego, aplicando el segundo teorema fundamental para calcular $\int_a^b f(x)\; dx$, tenemos
	$$\begin{array}{rcl}
	    \displaystyle\int_a^b f(x)\; dx &=& P(b)-P(a)\\\\
					    &=& \dfrac{b^2}{2}-b^{-1}+\dfrac{3b^{-2}}{2}-\left(\dfrac{a^2}{2}-a^{-1}+\dfrac{3a^{-2}}{2}\right)\\\\
					    &=& \dfrac{1}{2}\left(b^2-a^2\right)-\left(b^{-1}-a^{-1}\right)+\dfrac{3}{2}\left(b^{-2}-a^{-2}\right)\\\\
					    &=& \dfrac{1}{2}\left(b^2-a^2\right)-\left(\dfrac{1}{a}-\dfrac{1}{b}\right)+\dfrac{3}{2}\left(\dfrac{1}{a^2}-\dfrac{1}{b^2}\right).
	\end{array}$$
	\vspace{.5cm}

    %-------------------- 5.
    \item $f(x)=\left(1+\sqrt{x}\right)^2,\quad x>0$.\\\\
	Respuesta.-\; Dado que $x>0$. Reescribimos la función $f$ como:
	$$f(x)=1+2\sqrt{x}+|x| \quad \Rightarrow \quad f(x)=1+2x^{\frac{1}{2}}+x$$
	Como $x>0$, entonces $|x|=x$. Luego, por el ejemplo 5.1 (Apostol) tenemos,
	$$\begin{array}{rcl}
	    P(x)&=&1\dfrac{x^{0+1}}{0+1}+\dfrac{2x^{\frac{1}{2}+1}}{\frac{1}{2}+1}+\dfrac{x^{1+1}}{1+1}\\\\
		&=&x+\dfrac{4x^{\frac{3}{2}}}{3}+\dfrac{x^2}{2}.
	\end{array}$$
	Esta función es primitiva de $f$ ya que,
	$$P'(x)=1+\dfrac{3}{2}\dfrac{4x^{\frac{3}{2}-1}}{3}+\dfrac{2x^2}{2}=1+2x^{\frac{1}{2}}+x=\left(1+\sqrt{x}\right)^2=f(x).$$
	Luego, aplicando el segundo teorema fundamental para calcular $\int_a^b f(x)\; dx$, tenemos
	$$\begin{array}{rcl}
	    \displaystyle\int_a^b f(x)\; dx &=& P(b)-P(a)\\\\
					    &=& x+\dfrac{4b^{\frac{3}{2}}}{3}+\dfrac{b^2}{2}-\left(x+\dfrac{4a^{\frac{3}{2}}}{3}+\dfrac{a^2}{2}\right)\\\\
					    &=& \dfrac{4}{3}\left(b^{\frac{3}{2}}-a^{\frac{3}{2}}\right)+\dfrac{1}{2}\left(b^2-a^2\right)+(b-a).
	\end{array}$$
	\vspace{.5cm}

    %-------------------- 6.
    \item $f(x)=\sqrt{2x}+\sqrt{\dfrac{1}{2}x},\qquad x>0$.\\\\
	Respuesta.-\; Por el ejemplo 5.1 (Apostol) tenemos, 
	$$\begin{array}{rcl}
	    P(x)&=&\dfrac{\sqrt{2}x^{\frac{1}{2}+1}}{\frac{1}{2}+1}+\dfrac{\sqrt{\frac{1}{2}}x^{\frac{1}{2}+1}}{\frac{1}{2}+1}\\\\
	    		&=&\dfrac{2\sqrt{2}x^{\frac{3}{2}}}{3}+\dfrac{2\sqrt{\frac{1}{2}}x^{\frac{3}{2}}}{3}\\\\
	\end{array}$$
	Esta función es primitiva de $f$ ya que,
	$$P'(x)=\dfrac{3}{2}\dfrac{2\sqrt{2}x^{\frac{3}{2}-1}}{3}+\dfrac{3}{2}\dfrac{2\sqrt{\frac{1}{2}}x^{\frac{3}{2}-1}}{3}=\sqrt{2x}+\sqrt{\dfrac{1}{2}x}=f(x).$$
	Luego, aplicando el segundo teorema fundamental para calcular $\int_a^b f(x)\; dx$, tenemos
	$$\begin{array}{rcl}
	    \displaystyle\int_a^b f(x)\; dx &=& P(b)-P(a)\\\\
					    &=& \dfrac{2\sqrt{2}b^{\frac{3}{2}}}{3}+\dfrac{2\sqrt{\frac{1}{2}}b^{\frac{3}{2}}}{3}-\left(\dfrac{2\sqrt{2}a^{\frac{3}{2}}}{3}+\dfrac{2\sqrt{\frac{1}{2}}a^{\frac{3}{2}}}{3}\right) \\\\
					    &=& \left(\dfrac{2\sqrt{2}}{3}+\dfrac{2\sqrt{\frac{1}{2}}}{3}\right)b^{\frac{3}{2}}-\left(\dfrac{2\sqrt{2}}{3}+\dfrac{2\sqrt{\frac{1}{2}}}{3}\right)a^{\frac{3}{2}}\\\\
					    &=& \dfrac{2}{\sqrt{2}}\left(b^{\frac{3}{2}}-a^{\frac{3}{2}}\right).

	\end{array}$$
	\vspace{.5cm}

    %-------------------- 7.
    \item $f(x)=\dfrac{2x^2-6x+7}{2\sqrt{x}}\qquad x>0$.\\\\
	Respuesta.-\; Reescribimos la función $f$ como:
	$$\begin{array}{rcl}
	    f(x)&=&\dfrac{x^2}{x^{\frac{1}{2}}}-\dfrac{3x}{x^{\frac{1}{2}}}+\dfrac{7}{2x^{\frac{1}{2}}}\\\\
		&=&x^2x^{-\frac{1}{2}}-3xx^{-\frac{1}{2}}+\dfrac{7}{2}x^{-\frac{1}{2}}\\\\
		&=&x^{\frac{3}{2}}-3x^{\frac{1}{2}}+\dfrac{7}{2}x^{-\frac{1}{2}}.\\\\
	\end{array}$$
	Por el ejemplo 5.1 (Apostol) tenemos,
	$$\begin{array}{rcl}
	    P(x)&=&\dfrac{x^{\frac{3}{2}+1}}{\frac{3}{2}+1}-\dfrac{3x^{\frac{1}{2}+1}}{\frac{1}{2}+1}+\dfrac{7}{2}\dfrac{x^{-\frac{1}{2}+1}}{-\frac{1}{2}+1}\\\\
		&=&\dfrac{2x^{\frac{5}{2}}}{5}-2x^{\frac{3}{2}}+7x^{\frac{1}{2}}\\\\
	\end{array}$$
    	Esta función es primitiva de $f$ ya que,
	$$\begin{array}{rcl}
	    P'(x)&=&\dfrac{5}{2}\cdot\dfrac{2x^{\frac{5}{2}-1}}{5}-\dfrac{3}{2}\cdot{2x^{\frac{3}{2}-1}}+\dfrac{1}{2}\cdot7x^{\frac{1}{2}-1}\\\\
		 &=&\dfrac{2x^{\frac{3}{2}}}{2}-\dfrac{6x^{\frac{1}{2}}}{2}+\dfrac{7}{2}x^{-\frac{1}{2}}.\\\\
		 &=&\dfrac{2x^2-6x+7}{2\sqrt{7}}=f(x).
	\end{array}$$
	Luego, aplicando el segundo teorema fundamental para calcular $\int_a^b f(x)\; dx$, tenemos
	$$\begin{array}{rcl}
	    \displaystyle\int_a^b f(x)\; dx &=& P(b)-P(a)\\\\
					    &=& \dfrac{2b^{\frac{5}{2}}}{5}-2b^{\frac{3}{2}}+7b^{\frac{1}{2}}-\left(\dfrac{2a^{\frac{5}{2}}}{5}-2a^{\frac{3}{2}}+7a^{\frac{1}{2}}\right)\\\\
					    &=&\dfrac{2}{5}\left(b^{\frac{5}{2}}-a^{\frac{5}{2}}\right)-2\left(b^{\frac{3}{2}}-a^{\frac{3}{2}}\right)+7\left(b^{\frac{1}{2}}-a^{\frac{1}{2}}\right).
	\end{array}$$
	\vspace{0.5cm}

    %-------------------- 8.
    \item $f(x)=2x^{\frac{1}{3}}-x^{-\frac{1}{3}}, \qquad x>0$.\\\\
	Respuesta.-\; Por el ejemplo 5.1 (Apostol) tenemos,
	$$\begin{array}{rcl}
	    P(x)&=&2\dfrac{x^{\frac{1}{3}+1}}{\frac{1}{3}+1}-\dfrac{x^{-\frac{1}{3}+1}}{-\frac{1}{3}+1}\\\\
		&=&\dfrac{3x^{\frac{4}{3}}}{2}-\dfrac{3x^{\frac{2}{3}}}{2}\\\\
	\end{array}$$
	Esta función es primitiva de $f$ ya que,
	$$\begin{array}{rcl}
	    P'(x)&=&\dfrac{4}{3}\cdot \dfrac{3x^{\frac{4}{3}-1}}{2}-\dfrac{2}{3}\cdot\dfrac{3x^{\frac{2}{3}-1}}{2}\\\\
		 &=&2x^{\frac{1}{3}}-x^{-\frac{1}{3}}=f(x).
	\end{array}$$
	Luego, aplicando el segundo teorema fundamental para calcular $\int_a^b f(x)\; dx$, tenemos
	$$\begin{array}{rcl}
	    \displaystyle\int_a^b f(x)\; dx &=& P(b)-P(a)\\\\
					    &=& \dfrac{3b^{\frac{4}{3}}}{2}-\dfrac{3b^{\frac{2}{3}}}{2}-\left(\dfrac{3a^{\frac{4}{3}}}{2}-\dfrac{3a^{\frac{2}{3}}}{2}\right)\\\\
					    &=&\dfrac{3}{2}\left(b^{\frac{4}{3}}-a^{\frac{4}{3}}\right)-\dfrac{3}{2}\left(b^{\frac{2}{3}}-a^{\frac{2}{3}}\right).
	\end{array}$$
	\vspace{0.5cm}

    %-------------------- 9.
    \item $f(x)=3\sen x + 2x^5$.\\\\
	Respuesta.-\; Claramente podemos ver que,
	$$P(x)=-3\cos x + \dfrac{1}{3}x^6,$$
	es una función primitiva ya que,
	$$P'(x)=3\sen x + 2x^5 = f(x).$$
	Luego, aplicando el segundo teorema fundamental para calcular $\int_a^b f(x)\; dx$, tenemos
	$$\begin{array}{rcl}
	    \displaystyle\int_a^b f(x)\; dx &=& P(b)-P(a)\\\\
					    &=& -3\cos b + \dfrac{1}{3}b^6 -\left(-3\cos a + \dfrac{1}{3}a^6\right)\\\\
					    &=&-3(\cos b - \cos a) + \dfrac{1}{3}\left(b^6-a^6\right).
	\end{array}$$
	\vspace{0.5cm}

    %-------------------- 10.
    \item $f(x)=x^{\frac{4}{3}}-5\cos x$.\\\\
	Respuesta.-\; Claramente podemos ver que,
	$$P(x)=\dfrac{3}{7}x^{\frac{7}{3}}-5\sin x,$$
	es una función primitiva ya que,
	$$P'(x)=x^{\frac{4}{3}}-5\cos x = f(x).$$
	Luego, aplicando el segundo teorema fundamental para calcular $\int_a^b f(x)\; dx$, tenemos
	$$\begin{array}{rcl}
	    \displaystyle\int_a^b f(x)\; dx &=& P(b)-P(a)\\\\
					    &=& \dfrac{3}{7}b^{\frac{7}{3}}-5\sin b -\left(\dfrac{3}{7}a^{\frac{7}{3}}-5\sin a\right)\\\\
					    &=&\dfrac{3}{7}\left(b^{\frac{7}{3}}-a^{\frac{7}{3}}\right)-5\left(\sin b - \sin a\right).
	\end{array}$$
	\vspace{0.5cm}

    %-------------------- 11.
    \item Demostrar que no existe ningún polinomio $f$ cuya derivada esté dada por la fórmula $f'(x)=\dfrac{1}{x}.$\\\\
	Demostración.-\; Supongamos lo contrario. Sea $f'(x)=\dfrac{1}{x},\; x\in \mathbb{R}$, entonces podemos hallar su primitiva de la siguiente manera:
	$$P(x)=\dfrac{x^{n+1}}{n+1}, n\in \mathbb{N}\quad \Rightarrow \quad f(x)=\dfrac{x^{-1+1}}{-1+1}=\dfrac{x^0}{0}.$$
	Lo cual es un absurdo. Por lo tanto no existe ningún polinomio $f$ cuya derivada esté dada por la fórmula $f'(x)=\dfrac{1}{x}.$\\\\

    %-------------------- 12.
    \item Demostrar que $\int_0^x |t| \; dt = \frac{1}{2}x|x|$ para todo número real $x$.\\\\
	Demostración.-\; Consideremos tres casos.
	\begin{enumerate}[\textit{Caso} 1.]
	    \item Si $x=0$, entonces para ambos lados de la ecuación el resultado será $0$, por lo que el resultado se cumple.
	    \item Si $x>0$, entonces $|t|=t$ para todo $t\in [0,x]$, así
		$$\int_0^x |t|\; dt = \int_0^x t \; dt = \dfrac{1}{2}x^2=\dfrac{1}{2}x|x|.$$
	    \item Si $x<0$, entonces $|t|=-t$ para todo $t\in [x,0]$, así
		$$\int_0^x |t|\; dt = -\int_0^x t \; dt=\int_x^0 t \; dt = \dfrac{1}{2}t^2\bigg|_x^0=-\dfrac{1}{2}x^2=\dfrac{1}{2}x|x|.$$
	\end{enumerate}
	\vspace{0.5cm}

    %-------------------- 13.
    \item Demostrar que 
    $$\int_0^x \left(t+|t|\right)^2\; dt = \dfrac{2x^2}{3}(x+|x|) \mbox{ para todo } x \mbox{ real.}$$\\
    	Demostración.-\; Consideremos dos casos.
	\begin{enumerate}[\textit{Caso} 1.]
	    \item Si $x\geq 0$ entonces $t=|t|$ para todo $t\in[0,x]$ así,
		$$\int_0^x (t+|t|)^2 \; dt= \int_0^x 4t^2\; dt = \dfrac{4}{3}x^3=\dfrac{2}{3}x^2(2x)=\dfrac{2}{3}x^2(x+|x|).$$
	    \item Si $x<0$ entonces $|t|=-t$ para todo $t\in[x,0]$ así,
		$$\int_0^x (t+|t|)^2 \; dt = \int_0^x (t-t)^2\; dt = 0 = \dfrac{2}{3}x^2(x-x)= \dfrac{2}{3}x^2(x+|x|).$$\\ 
	\end{enumerate}

    %-------------------- 14.
    \item Una función $f$ es continua para cualquier $x$ y satisface la ecuación
    $$\int_0^x f(t)\; dt = -\dfrac{1}{2}+x^2+x\sen2x + \dfrac{1}{2}\cos 2x.$$
    para todo $x$. Calcular $f\left(\frac{1}{4}\pi\right)$ y $f'\left(\frac{1}{4}\pi\right).$\\\\
	Respuesta.-\; Sea,
	$$A(x)=\int_0^x f(t)\; dt = -\dfrac{1}{2}+x^2+x\sen2x + \dfrac{1}{2}\cos 2x.$$
	Por el primer teorema fundamental del cálculo la derivada de $A(x)$ existe. Es decir,
	$$A'(x)=f(x).$$
	De donde,
	$$f(x)=A'(x)=2x+\sen(2x)+2x\cos(2x)-\sen(2x)=2x+2x\cos(2x).$$
	Luego, evaluando en $x=\frac{\pi}{4}$ tenemos
	$$f\left(\dfrac{\pi}{4}\right)=A'\left(\dfrac{\pi}{4}\right)=\dfrac{\pi}{2}+\dfrac{\pi}{2}\cos\left(\dfrac{\pi}{2}\right)=\dfrac{\pi}{2}.$$
	Por otro lado, derivamos $f(x)$ y obtenemos
	$$f(x)=2x+2x\cos(2x)\quad \Rightarrow \quad f'(x)=2+2\cos(2x)-4x\sen(2x).$$
	Así, 
	$$f'\left(\dfrac{\pi}{4}\right)=2+2\cos\left(\dfrac{\pi}{2}\right)-\pi\sen\left(\dfrac{\pi}{2}\right)=2-\pi.$$\\

    %-------------------- 15.
    \item Encontrar una función $f$ y un valor de la constante $c$, tal que:
    $$\int_c^x f(t)\; dt = \cos x - \dfrac{1}{2}\mbox{ para todo } x \mbox{ real. }$$\\
	Respuesta.-\; Sea $f(t)=-\sen t$ y $c=\frac{\pi}{3}$. Entonces,
	$$\begin{array}{rcl}
	    \displaystyle\int_c^x f(t)\; dt &=& \displaystyle\int_{\frac{\pi}{3}}^x (-\sen t)\; dt\\\\ 
					    &=&\cos t \bigg|_{\frac{\pi}{3}}^x\\\\
					    &=&\cos x - \cos \left(\dfrac{\pi}{3}\right)\\\\
					    &=&\cos x - \dfrac{1}{2}.
	\end{array}$$
	\vspace{0.5cm}

    %-------------------- 16.
    \item Encontrar una función $f$ y un valor de la constante $c$, tal que:
    $$\int_c^x t f(t)\; dt = \sen x-x\cos x-\dfrac{1}{2}x^2 \mbox{ para todo } x \mbox{ real. }$$\\
	Respuesta.-\; Sea $f(t)=\sen t-1$ y $c=0$. Entonces,
	$$\begin{array}{rcl}
	    \displaystyle\int_c^x tf(t)\; dt &=& \displaystyle \int_0^x (t\sen t - t)\; dt\\\\
					     &=& \left(\sen t - t \cos t - \dfrac{1}{2}t^2\right)\bigg|_0^x\\\\
					     &=& \sen x - x\cos x - \dfrac{1}{2}x^2.
	\end{array}$$
	\vspace{0.5cm}

    %-------------------- 17.
    \item 
	



\end{enumerate}
