\chapter{Funciones continuas}

\setcounter{section}{2}
\section{Definición de límite de una función}
La continuidad existe si existe  continuidad por la izquierda y por la derecha.

\begin{tcolorbox}
    \begin{def.}[Definición de entorno de un punto]
	Cualquier intervalo abierto que contenga un punto $p$ como su punto medio se denomina entorno de $p$.
    \end{def.}
\end{tcolorbox}

\textbf{Notación.-} Designemos los entornos con $N(p), N_1(p), N_2(p),$ etc. Puesto que un entorno $N(p)$ es un intervalo abierto simétrico respecto a $p$, consta de todos los números reales $x$ que satisfagan $p-r<x<p+r$ para un cierto $r>0$. El número positivo $r$ se llama radio del entorno. En lugar de $N(p)$ ponemos $N(p;r)$ si deseamos especificar su radio. Las desigualdades $p-r<x<p+r$ son equivalentes a $-r<x-p<r$, y a $|x-p|<r$. Así pues, $N(p;r)$ consta de todos los puntos $x$ , cuya distancia a $p$ es menor que $r$.\\
En la definición que sigue suponemos que $A$ es un número real y que $f$ es una función definida en un cierto entorno de un punto $p$ (excepción hecha acaso del mismo $p$). La función puede estar definida en $p$ pero esto no interviene en la definición.\\

\begin{tcolorbox}
    \begin{def.}[Definición de límite de una función]
	El simbolismo $$\lim_{x\to p}f(x)=A\qquad [\mbox{o}\; {f(x)\to A}\quad {x\to p}]$$
	significa que para todo entorno $N_1(A)$ existe un cierto entorno $N_2(p)$ tal que 
	\begin{center}
	    $f(x)\in N_1(A)$ siempre que $x\in N_2(p)$ y $x\neq p$
	\end{center}
	\vspace{0.3cm}
	El entorno $N_1(A)$ se cita en primer lugar, e indica cuán próximo queremos que sea $f(x)$ a su límite $A$. El segundo entorno, $N_2(p)$, nos indica lo próximo que debe estar $x$ de $p$ para que $f(x)$ sea interior al primer entorno $N_1(p)$. El entorno $N_2(p)$ dependerá del $N_1(A)$ elegido. Un entorno $N_2(p)$ que sirva para un $N_1(A)$ determinado servirá también, naturalmente, para cualquier $N_1(A)$ mayor, pero puede no ser útil para todo $N_1(A)$ más pequeño.\\
	Decir que $f(x)\in N_1(A)$ es equivalente a la desigualdad $|f(x)-A|<\epsilon$ y poner que $x\in N_2(p),\; x\neq p$ es lo mismo que escribir $0<|x-p|<\delta$. Por lo tanto, la definición de límite puede también expresarse así:\\\\
	El símbolo $\lim\limits_{x\to p}f(x)=A$ significa que para todo $\epsilon > 0$, existe un $\delta >0$ tal que $$|f(x)-A|<\epsilon \quad \mbox{siempre que}\quad 0<|x-p|<\delta.$$
    \end{def.}
\end{tcolorbox}

Observamos que las tres desigualdades,

\begin{center}
    $\lim\limits_{x\to p}f(x)=A$, $\lim\limits_{x\to p}(f(x)-A)=0$, $\lim\limits_{x\to p}[f(x)-A]=0$
\end{center}

Son equivalentes. También son equivalentes las desigualdades,

\begin{center}
    $\lim\limits_{x\to p} f(x)=A,$ $\lim\limits_{h\to 0}f(p+h)=A.$
\end{center}

Todas estas se derivan de la definición de límite.\\
\begin{tcolorbox}
    \begin{def.}[Límites laterales]
	Los límites laterales pueden definirse en forma parecida. Por ejemplo, si ${f(x)\to A}$ cuando ${x\to p}$ con valores mayores que $p$, decimos que $A$ es el límite por la derecha de $f$ en $p$, indicamos esto poniendo
	$$\lim_{x\to p^+}f(x)=A.$$
	En la terminología de los entornos esto significa que para todo entorno $N_1(A),$ existe algún entorno $N_2(p)$ tal que 
	$$f(x)\in N_1(A)\; \mbox{siempre que}\; x \in N_2(p)\quad \mbox{y}\quad x>p.$$
	Los límites a la izquierda, que indican poniendo ${x\to p^-}$, se definen del mismo modo restringiendo $x$ a valores menores que $p$.
    \end{def.}
\end{tcolorbox}

\section{Definición de continuidad de una función}

\begin{tcolorbox}
    \begin{def.}[Definición de continuidad de una función en un punto]
	Se dice que una función $f$ es continua en un punto $p$ si 
	\begin{enumerate}[\bfseries a)]
	    \item $f$ está definida en $p$, y
	    \item $\lim\limits_{x\to p}f(x)=f(p)$.
	\end{enumerate}
	Esta definición también puede formularse con entornos. Una función $f$ es continua en $p$ si para todo entorno $N_1[f(p)]$ existe un entorno $N_2(p)$ tal que 
	$$f(x)\in N_1[f(p)]\; \mbox{siempre que}\; x \in N_2(p).$$
	Puesto que $f(p)$ pertenece siempre a $N_1[f(p)]$, no se precisa la condición $x\neq p$.\\\\
	Especificando los radios de los entornos, la definición de continuidad puede darse como sigue:
	\begin{center}
	    $|f(x)-f(p)|<\epsilon$ siempre que $|x-p|<\delta$.
	\end{center}
    \end{def.}
\end{tcolorbox}

\section{Teoremas fundamentales sobre límites. Otros ejemplos de funciones continuas.}

%-------------------- teorema 3.1 --------------------------
\begin{tcolorbox}
    \begin{teo}
	Sean $f$ y $g$ dos funciones tales que 
	$$\lim_{x\to p} f(x)=A,\qquad \lim_{x\to p}g(x)=B$$
	Se tiene entonces
	\begin{enumerate}[\bfseries (i)]
	    \item $\lim\limits_{x\to p} [f(x)+g(x)]=A+B$,
	    \item $\lim\limits_{x\to p} [f(x)-g(x)=A-B]$,
	    \item $\lim\limits_{x\to p} f(x)\cdot g(x) = A\cdot B$,
	    \item $\lim\limits_{x\to p} \dfrac{f(x)}{g(x)} = \dfrac{A}{B}\qquad \mbox{si}\; B\neq 0$.\\
	\end{enumerate}
	Las demostraciones estarán dadas en la sección 3.5.
    \end{teo}
\end{tcolorbox}


Observemos primero que las afirmaciones del teorema pueden escribirse en forma un poco distinta. Por ejemplo, (i) puede ponerse como sigue:
$$\lim_{x\to p}[f(x)+g(x)]=\lim_{x\to p}f(x) + \lim_{x\to p}g(x).$$\\
Es costumbre indicar por $f+g$, $f-g$, $f\cdot g$ y $f/g$ las funciones cuyos valores para cada $x$ son:
$$f(x)+g(x),\quad f(x)-g(x),\quad f(x)\cdot g(x),\quad y \quad f(x)/g(x).$$\\

\begin{teo}
    Sean $f$ y $g$ dos funciones continuas en un punto $p$. La suma $f+g$, la diferencia $f-g$,  el producto $f\cdot g$ y  $g(p)\neq 0$ siempre que $g(p)\neq 0$ son también continuas en $p$.\\\\
    Demostración.-\; Puesto que $f$ y $g$ son continuas en $p$, se tiene $\lim\limits_{x\to p} f(x)=f(p)$ y $\lim\limits_{x\to p}g(x)=g(p)$. Aplicando las propiedades para límites, dadas en el teorema 3.1, cuando $A=f(p)$ y $B=g(p)$, se deduce el teorema 3.2.\\\\
\end{teo}

El teorema que sigue demuestra que si una función $g$ está intercalada entre otras dos funciones que tienen el mismo límite cuando $x\to p,$ $g$ tiene también este límite cuando $x\to p.$\\\\

% -------------------- teorema 3.3 --------------------------
\begin{teo}[Principio de intercalación]
    Supongamos que $f(x)\leq g(x)\leq h(x)$ para todo $x\neq p$ en un cierto entorno $N(p)$. Supongamos también que 
    $$\lim_{x\to p}f(x)=\lim_{x\to p}h(x)=a.$$
    Se tiene entonces $\lim_{x\to p}g(x)=a.$\\\\
	Demostración.-\; Sean $G(x)=g(x)-f(x)$, y $H(x)=h(x)-f(x)$. Las desigualdades $f\leq g\leq h$ implican $0\leq g-f\leq h-f$ o
	$$0\leq G(x)\leq H(x)$$
	para todo $x\neq p$ en $N(p)$. Para demostrar el teorema, basta probar que $G(x)\to 0$ cuando $x\to p$, dado que $H(x)\to 0$ cuando $x\to p$.\\
	Sea $N_1(0)$ un entorno cualquier de $0$. Puesto que $H(x)\to 0$ cuando $x\to p,$ existe un entorno $N_2(p)$ tal que 
	\begin{center}
	    $H(x) \in N_1(0)$ siempre que $x\in N_2(p)$ y $x\neq p.$
	\end{center}
	    Podemos suponer que $N_2(p) \subseteq N(p)$. Entonces la desigualdad $0\leq G \leq H$ establece que $G(x)$ no está más lejos de $0$ si $x$ está en $N_2(p), x\neq p$. Por consiguiente $G(x)\in N_1(0)$ para tal valor $x$ y por tanto $G(x)\to 0$ cuando $x\to p$. La misma demostración es válida si todos los límites son límites a un lado.\\\\
\end{teo}

% -------------------- teorema 3.4 --------------------------
\begin{teo}[Continuidad de las integrales indefinidas]
    Supongamos que $f$ es integrable en $[a,x]$ para todo $x$ en $[a,b]$, y sea
    $$A(x)=\int_a^x f(t)\; dt$$
    Entonces la integral indefinida $A$ es continua en cada punto de $[a,b]$ (En los extremos del intervalo tenemos continuidad a un lado.)\\\\
	Demostración.-\; Elijamos $p$ en $[a,b]$. Hay que demostrar que $A(x)\to A(p)$ cuando $x\to p.$ Tenemos
	$$A(x)-A(p)=\int_p^x f(t)\; dt$$
	Puesto que $f$ está acotada en $[a,b]$, existe una constante $M>0$ tal que $-M\leq f(t)\leq M$ para $t$ en $[a,b]$. Si $x>p$, integramos esas desigualdades en el intervalo $[p,x]$ obteniendo
	$$\int_p^x -M \;dt \leq \int_p^x f(t)\; dt\leq \int_p^x M \; dt \quad \Longrightarrow \quad -M(x-p)\leq A(x)-A(p)\leq M(x-p).$$
	Si $x<p$, obtenemos las mismas desigualdades con $x-p$ sustituida por $p-x$. Por consiguiente, en uno u otro caso podemos hacer que $x\to p$ y aplicar el principio de intercalación encontrando que $A(x)\to A(p)$. Esto prueba el teorema. Si $p$ es un extremo de $[a,b]$, tenemos que hacer que $x\to p$ desde el interior del intervalo, con lo que los límites son a un lado.
\end{teo}

