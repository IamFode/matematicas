\chapter{Límites y continuidad}

%--------------------definición 2.1

\begin{tcolorbox}[colframe = white]
    \begin{def.}[La razón promedio de cambio] de $y=f(x)$ con respecto a $x$ en el intervalo $[x_1,x_2]$ sabiendo que $\triangle x = x_2 - x_1 = h$ es $$\dfrac{\triangle y}{\triangle x} = \dfrac{f(x_2) - f(x_1)}{x_2-x_1} = \dfrac{f(x_1+h) - f(x_1)}{h}, \quad h\neq 0$$
    \end{def.}
\end{tcolorbox}

%-------------------- Ejercicios 2.1
\section{Ejercicios}

\textbf{Razones promedio de cambio}\\\\\
En los ejercicios 1 a 6, determine la razón promedio de cambio de la función en el intervalo o intervalos dados.\\\\

\begin{enumerate}[\large\bfseries 1.]

%--------------------1.
\item $f(x) = x^3 + 1$ 
\begin{enumerate}[\bfseries a)]
    
    %----------a)
    \item $[2,3]$\\\\
	Respuesta.-\; $\dfrac{\triangle y}{\triangle x} = \dfrac{(3^3 + 1) - (2^3 + 1)}{3 - 2} = 19$\\\\

    %----------b)
    \item $[-1,1]$\\\\
	Respuesta.-\; $\dfrac{\triangle y}{\triangle x} = \dfrac{(1^3 + 1)-((-1)^3 + 1)}{1-(-1)} = 1$\\\\

\end{enumerate}

%--------------------2.
\item $g(x) = x^2 - 2x$
\begin{enumerate}[\bfseries a)]
    
    %----------a)
    \item $[1,3]$\\\\
	Respuesta.-\; $\dfrac{\triangle y}{\triangle x} = \dfrac{(3^2 - 2\cdot 3) - (1^2 - 2\cdot 1)}{3-1} = 2 $\\\\ 
    
    %----------b)
    \item $[-2,4]$\\\\
	Respuesta.-\; $\dfrac{\triangle y}{\triangle x} = \dfrac{(4^2 - 2\cdot 4) - ((-2)^2 - 2\cdot (-2))}{4-(-2)} = 0$\\\\

\end{enumerate}

%--------------------3.
\item 

\end{enumerate}
