\chapter{Límites y continuidad}

%--------------------definición 2.1

\begin{tcolorbox}[colframe = white]
    \begin{def.}[La razón promedio de cambio] de $y=f(x)$ con respecto a $x$ en el intervalo $[x_1,x_2]$ sabiendo que $\triangle x = x_2 - x_1 = h$ es $$\dfrac{\triangle y}{\triangle x} = \dfrac{f(x_2) - f(x_1)}{x_2-x_1} = \dfrac{f(x_1+h) - f(x_1)}{h}, \quad h\neq 0$$
    \end{def.}
\end{tcolorbox}

%-------------------- Ejercicios 2.1
\section{Ejercicios}

\textbf{Razones promedio de cambio}\\\\\
En los ejercicios 1 a 6, determine la razón promedio de cambio de la función en el intervalo o intervalos dados.\\\\

\begin{enumerate}[\large\bfseries 1.]

%--------------------1.
\item $f(x) = x^3 + 1$ 
\begin{enumerate}[\bfseries a)]
    
    %----------a)
    \item $[2,3]$\\\\
	Respuesta.-\; $\dfrac{\triangle y}{\triangle x} = \dfrac{(3^3 + 1) - (2^3 + 1)}{3 - 2} = 19$\\\\

    %----------b)
    \item $[-1,1]$\\\\
	Respuesta.-\; $\dfrac{\triangle y}{\triangle x} = \dfrac{(1^3 + 1)-((-1)^3 + 1)}{1-(-1)} = 1$\\\\

\end{enumerate}

%--------------------2.
\item $g(x) = x^2 - 2x$
\begin{enumerate}[\bfseries a)]
    
    %----------a)
    \item $[1,3]$\\\\
	Respuesta.-\; $\dfrac{\triangle y}{\triangle x} = \dfrac{(3^2 - 2\cdot 3) - (1^2 - 2\cdot 1)}{3-1} = 2 $\\\\ 
    
    %----------b)
    \item $[-2,4]$\\\\
	Respuesta.-\; $\dfrac{\triangle y}{\triangle x} = \dfrac{(4^2 - 2\cdot 4) - ((-2)^2 - 2\cdot (-2))}{4-(-2)} = 0$\\\\

\end{enumerate}

%--------------------3.
\item $h(t) = cot \; t$
\begin{enumerate}[\bfseries (a)]

    %----------(a)
    \item $[\pi/4,3\pi/4]$\\\\
	Respuesta.-\; $\dfrac{\triangle y}{\triangle x} = \dfrac{cot(\pi/4)-cot(3\pi/4)}{\pi/4-3\pi/4} = \dfrac{1+1}{\dfrac{\pi - 3\pi}{4}} = \dfrac{8}{-2\pi}$\\\\

    %----------(b)
    \item $[\pi/6,\pi/2]$\\\\
	Respuesta.-\; $\dfrac{\triangle y}{\triangle x} = \dfrac{cot(\pi/6)-cot(\pi/2)}{\pi/6-\pi/2} = \dfrac{-3\sqrt{3}}{\pi}$\\\\

\end{enumerate}

%--------------------4.
\item $g(t) = 2 + \cos t$ 
\begin{enumerate}[\bfseries (a)]

    %----------(a)
    \item $[0,\pi]$\\\\
	Respuesta.-\; $\dfrac{2+\cos \pi - (2+\cos 0)}{\pi - 0} = -\dfrac{2}{\pi}$\\\\

    %----------(b)
    \item $[-\pi,\pi]$\\\\
	Respuesta.-\; $\dfrac{2 + \cos \pi - (2 - \cos \pi)}{\pi+\pi} = \dfrac{3-3}{2\pi} = 0$\\\\

\end{enumerate}

%--------------------5.
\item $R(\theta) = \sqrt{4\theta + 1}; \; [0,2]$\\\\ 
    Respuesta.-\; $\dfrac{\sqrt{4*2+1}+1 - (\sqrt{4*0+1}+1)}{2-0} = \dfrac{2}{2} = 1$\\\\

%--------------------6.
\item $P(\theta) = \theta^3 - 4\theta^2 + 5\theta; \; [1,2]$\\\\
    Respuesta.-\; $\dfrac{2^3 - 4\cdot 2^2 + 5\cdot 2 - (1^3 - 4^2 + 5)}{2-1} = 0$\\\\

\textbf{Pendiente de una curva en un punto}\\\\
En los ejercicios 7 a 14, utilice el método del ejemplo 3 para determinar $a)$ la pendiente de la curva en el punto $P$ dado, y $b)$ la ecuación de la recta tangente en $P$\\\\

%--------------------7.
\item $y = x^2 - 5, \quad P(2,-1)$

\begin{enumerate}[\bfseries a)]
    
    %----------a)
    \item Iniciamos con una recta secante que pasa por el punto $(2,-1)$ y el punto cercano $(2+h,(2+h)^2 - 5)$, luego hallamos la pendiente de la secante, $$\dfrac{\triangle y}{\triangle x} = \dfrac{(2+h)^2 - 5 - (2^2 - 5)}{2+h - 2} = \dfrac{4h+h^2}{h} = 4+h$$ 
    Luego aproximamos  $h$ a $0$ siendo la pendiente $m=4$.\\\\

    %----------b)
    \item La ecuación de la recta tangente viene dado por $y=mx + c$  de donde $y=4x+c$, luego reemplazamos $(2,-1)$, quedándonos $-1=4\cdot 2 + c \; \Longrightarrow\; c = -9$. Por lo tanto $$y=4x-9$$\\

\end{enumerate}

%--------------------8.
\item $y=7-x^2, \quad P(2,3)$ 

\begin{enumerate}[\bfseries a)]
    
    %----------a)
    \item Sea la recta secante que pasa por el punto $P(2,3)$ y el punto cercano $Q\left[2+h,7-(2+h)^2\right]$, hallamos la pendiente de la secante, $$\dfrac{\triangle y}{\triangle x} = \dfrac{7-(2+h)^2 - (7-2^2)}{2+h-2} = \dfrac{7-(2+h)^2 - 3}{h} = \dfrac{h(-h-4)}{h} = -h-4$$
    Con forme $Q$ se aproxima a $P$ a lo largo de la curva, $h$ se aproxima a cero, y la pendiente de la secante $-h-4$ se aproxima a $-4$. Tomamos $-4$ como la pendiente de la parábola en $P$.\\\\

    %----------b)
    \item La ecuación de la recta tangente viene dado por $y = mx + c$ de donde $y=-4x+c$, luego reemplazamos $(2,3)$, así $3 = -4(2) + c \Longrightarrow c = 11$. Por lo tanto $$y = -4x + 11$$.\\

\end{enumerate}

%--------------------9.
\item $y=x^2-2x-3, \quad P(2,-3)$

\begin{enumerate}[\bfseries a)]
    
    %----------a)
    \item Sea la recta secante que pasa por le punto $P(2,-3)$  y el punto cercano $Q\left[2+h, (2+h)^2 - 2(2+h) - 3\right]$, hallamos la pendiente de la secante, $$\dfrac{\triangle y}{\triangle x} = \dfrac{(2+h)^2 - 2(2+h) - 3 - (-3)}{h} = \dfrac{h^2 + 2h}{h} = h+2$$
    Con forme $Q$ se aproxima a $P$ a lo largo de la curva, $h$ se aproxima a cero, y la pendiente de la secante $h+2$ se aproxima a $2$. Tomamos $2$ como la pendiente de la parábola en $P$.\\\\

    %----------b)
    \item La ecuación de la recta tangente viene dado por $y = mx + c$ de donde $y=2x+c$, luego reemplazamos $(2,-3)$, así $-3 = 2(2) + c \Longrightarrow c = -7$. Por lo tanto $$y = -2x - 7$$.\\

\end{enumerate}

%--------------------10.
\item $y=x^2-4x, \quad P(1,-3)$ 

\begin{enumerate}[\bfseries a)]
    
    %----------a)
    \item Sea la recta secante que pasa por le punto $P(1,-3)$  y el punto cercano $Q\left[1+h, (1+h)^2 - 4(1+h) \right]$, hallamos la pendiente de la secante, $$\dfrac{\triangle y}{\triangle x} = \dfrac{(1+h)^2 - 4(1+h) - (-3)}{1+h - 1} = \dfrac{h^2 - 2h}{h} = h-2$$
    Con forme $Q$ se aproxima a $P$ a lo largo de la curva, $h$ se aproxima a cero, y la pendiente de la secante $h-2$ se aproxima a $-2$. Tomamos $-2$ como la pendiente de la parábola en $P$.\\\\

    %----------b)
    \item Tenemos $$\dfrac{\triangle y}{\triangle x} = m = \dfrac{y - y_1}{x - x_1} \; \Longrightarrow \; y-y_1 = m(x-x_1) \; \Longrightarrow \; y = -2(x-1) + (-3) \; \Longrightarrow \; y = -2x+2 - 3$$ por lo tanto $$y = -2x-1$$\\
     

\end{enumerate}

%--------------------11.
\item $y=x^3, \quad P(2,8)$

\begin{enumerate}[\bfseries a)]
    
    %----------a)
    \item Sea la recta secante que pasa por le punto $P(2,8)$  y el punto cercano $Q\left[2+h, (2+h)^3 \right]$, hallamos la pendiente de la secante, $$\dfrac{\triangle y}{\triangle x} = \dfrac{(2+h)^3 - 8}{2+h - 2} = \dfrac{h^3 + 6h^2 + 12h}{h} = h^2+6h+12$$
    Con forme $Q$ se aproxima a $P$ a lo largo de la curva, $h$ se aproxima a cero, y la pendiente de la secante $h^2 + 6h + 12$ se aproxima a $12$. Tomamos $12$ como la pendiente de la parábola en $P$.\\\\

    %----------b)
    \item Tenemos $$\dfrac{\triangle y}{\triangle x} = m = \dfrac{y - y_1}{x - x_1} \; \Longrightarrow \; y = m(x-x_1) + y_1\; \Longrightarrow \; y = 12(x-2) + 8 \; \Longrightarrow \; y = 12x - 24 + 8$$ por lo tanto $$y = 12x-16$$\\

\end{enumerate}

%--------------------12.
\item $y=2-x^3, \quad P(1,1)$

\begin{enumerate}[\bfseries a)]
    
    %----------a)
    \item Sea la recta secante que pasa por le punto $P(1,1)$  y el punto cercano $Q\left[1+h, 2-(1+h)^3 \right]$, hallamos la pendiente de la secante, $$\dfrac{\triangle y}{\triangle x} = \dfrac{2-(1+h)^3 - 1}{1+h - 1} = \dfrac{h^3 + 3h^2 + 3h}{h} = h^2+3h+3$$
    Con forme $Q$ se aproxima a $P$ a lo largo de la curva, $h$ se aproxima a cero, y la pendiente de la secante $h^2 + 3h + 3$ se aproxima a $3$. Tomamos $3$ como la pendiente de la parábola en $P$.\\\\

    %----------b)
    \item Tenemos $$\dfrac{\triangle y}{\triangle x} = m = \dfrac{y - y_1}{x - x_1} \; \Longrightarrow \; y = m(x-x_1) + y_1\; \Longrightarrow \; y = 3(x-1) + 1 \; \Longrightarrow \; y = 3x - 3 + 1$$ por lo tanto $$y = 3x-2$$\\

\end{enumerate}

%--------------------13.
\item $y=x^3 - 12x, \quad P(1,-11)$

\begin{enumerate}[\bfseries a)]
    
    %----------a)
    \item Sea la recta secante que pasa por le punto $P(1,-11)$  y el punto cercano $Q\left[1+h, (1+h)^3 -12(1+h)\right]$, hallamos la pendiente de la secante, $$\dfrac{\triangle y}{\triangle x} = \dfrac{(1+h)^3 - 12(1+h) + 11}{1+h - 1} = \dfrac{h^3 + 3h^2 - 9h}{h} = h^2+3h-12$$
    Con forme $Q$ se aproxima a $P$ a lo largo de la curva, $h$ se aproxima a cero, y la pendiente de la secante $h^2 + 3h - 9$ se aproxima a $-9$. Tomamos $-9$ como la pendiente de la parábola en $P$.\\\\

    %----------b)
    \item Tenemos $$\dfrac{\triangle y}{\triangle x} = m = \dfrac{y - y_1}{x - x_1} \; \Longrightarrow \; y = m(x-x_1) + y_1\; \Longrightarrow \; y = -9(x-1) - 11 \; \Longrightarrow \; y = -9x + 9 - 11$$ por lo tanto $$y = -9x-2$$\\

\end{enumerate}

%--------------------14.
\item $y=x^3-3x^2+4,\quad P(2,0)$

\begin{enumerate}[\bfseries a)]
    
    %----------a)
    \item Sea la recta secante que pasa por le punto $P(2,0)$  y el punto cercano $Q\left[2+h, (2+h)^3 -3(2+h)^2 + 4\right]$, hallamos la pendiente de la secante, $$\dfrac{\triangle y}{\triangle x} = \dfrac{(2+h)^3 - 3(2+h)^2 + 4 - 0}{2+h - 2} = \dfrac{h^3 + 3h^2}{h} = h^2+3h$$
    Con forme $Q$ se aproxima a $P$ a lo largo de la curva, $h$ se aproxima a cero, y la pendiente de la secante $h^2 + 3h$ se aproxima a $0$. Tomamos $0$ como la pendiente de la parábola en $P$.\\\\

    %----------b)
    \item Tenemos $$\dfrac{\triangle y}{\triangle x} = m = \dfrac{y - y_1}{x - x_1} \; \Longrightarrow \; y = m(x-x_1) + y_1\; \Longrightarrow \; y = 0(x-2) - 0$$ por lo tanto $$y = 0$$\\

\end{enumerate}

\textbf{Razones instantáneas de cambio}\\\\

%--------------------15.
\item Rapidez de un automóvil. La siguiente figura muestra la gráfica tiempo-distancia de un automóvil deportivo que acelera desde el reposo.
\begin{enumerate}[\bfseries a)]

    %----------a)
    \item Determine las pendientes de la secante $PQ_1, PQ_2, PQ_3$ y $PQ_4$, ordenándolas en una tabla como la de la figura 2.6.\\\\
	Respuesta.-\; 
	\begin{center}
	    \begin{tabular}{c|c}
		Q & PQ \\\\	
		\hline\\
		$Q_1(10,220)$ & $\dfrac{650-220}{20-10} = 43$\\\\
		$Q_2(14,380)$ & $\dfrac{650-380}{20-14} = 45$\\\\
		$Q_3(17,480)$ & $\dfrac{650-480}{20-16} = 43$\\\\
		$Q_4(17,550)$ & $\dfrac{650-550}{20-18} = 50$\\\\
	    \end{tabular}
	\end{center}
	Los resultados anteriores son redondeados.\\\\

    %----------b)
    \item Después estime la rapidez del automóvil para el tiempo $t=20s$.\\\\
	Respuesta.-\; Tomamos el punto mas cercano a $P$, en este caso $Q_4$ de donde la velocidad vendrá dado aproximadamente por $50m/s$.\\\\

\end{enumerate}

%--------------------16.
\item La siguiente figura muestra la gráfica de la distancia de caída libre contra el tiempo para un objeto que cae desde un módulo espacial que se encuentra a una distancia de 80 m de la superficie de la Luna.
\begin{enumerate}[\bfseries a)]
    
    %----------a)
    \item Estime las pendientes de las secantes $PQ_1$, $PQ_2$, $PQ_3$ y $PQ_4$, ordenándolas en una tabla como la de la figura 2.6.\\\\
	Respuesta.-\; 
	\begin{center}
	    \begin{tabular}{c|c}
		Q & PQ \\\\	
		\hline\\
		$Q_1(5,20)$ & $\dfrac{80-20}{10-5} = 12$\\\\
		$Q_2(7,38)$ & $\dfrac{80-38}{10-7} = 14$\\\\
		$Q_3(8.5,56)$ & $\dfrac{80-56}{10-8.5} = 16$\\\\
		$Q_4(9.5,71)$ & $\dfrac{80-71}{10-9.5} = 18$\\\\
	    \end{tabular}
	\end{center}

    %----------b)
    \item ¿Cuál será la rapidez aproximada del objeto cuando choca con la superficie de la Luna?\\\\
	Respuesta.-\; Será de  $18m/s$.\\\\ 

\end{enumerate}

%--------------------17.
\item En la siguiente tabla se registran las utilidades de una pequeña empresa en cada uno de sus primeros cinco años de operación:
\begin{enumerate}[\bfseries a)]

    %----------a)
    \item Trace los puntos que representan las utilidades como una función del año, y únalos mediante una curva suave.\\\\

    %----------b)
    \item ¿Cuál es la razón promedio de incremento de las utilidades entre $2012$ y $2014$?\\\\
	Respuesta.-\; $\dfrac{\triangle y}{\triangle x} = \dfrac{175-63}{2014-20112} = 56$\\\\ 

    %----------c)
    \item Use su gráfica para estimar la razón a la que cambiaron las utilidades en $2012$.\\\\
	Respuesta.-\; Sea $\dfrac{63-26}{2012-2011} = 37$ y $\dfrac{112-63}{2013-2012} = 49$ entonces $\dfrac{37+49}{2} = 43$.\\\\

\end{enumerate}

%--------------------18.
\item Elabore una tabla de valores para la función $F(x) = (x + 2)/(x - 2)$ en los puntos $x = 1.2, x = 11/10, x = 101/100, x = 1001/1000, x = 10001/10000$ y $x = 1$\\

	\begin{center}
	\begin{tabular}{ccccccc}
	    $x$&$1.2$&$11/10$&$101/100$&$1001/1000$&$10001/10000$&$1$\\
	    \hline\\
	    $F(x)$&$-4$&$-31/9$&$-301/99$&$3001/999$&$30001/9999$&$-3$\\\\
	\end{tabular}
	\end{center}

\begin{enumerate}[\bfseries a)]
    
    %----------a)
    \item Determine la razón promedio de cambio de $F(x)$ en los intervalos $[1, x]$ para cada $x \neq 1$ de su tabla.\\\\


    \item Si es necesario, amplíe su tabla para intentar determinar la razón de cambio de $F(x)$ en $x = 1$.\\\\

\end{enumerate}


%--------------------19.
\item Sea $g(x)=\sqrt{x}$ para $x\geq 0$\\\\
\begin{enumerate}[\bfseries a)]
    
    %----------a)
    \item Obtenga la razón promedio de cambio de $g(x)$ con respecto a $x$ en los intervalos $[1,2]$, $[1,1.5]$, $[1,1+h]$\\\\
	Respuesta.-\; Para $[1,2]$  se tiene $\dfrac{\triangle y}{\triangle x} = \dfrac{\sqrt{2} - 1}{2-1} = \sqrt{2}-1$.\\
	Para $[1,1.5]$ se tiene $\dfrac{\sqrt{1.5}-1}{1.5-1} = \dfrac{\sqrt{1.5}-1}{0.5}$.\\
	Para $[1,1+h]$ se tiene $\dfrac{\triangle y}{\triangle x} = \dfrac{\sqrt{1+h}-1}{1+h - 1} = \dfrac{\sqrt{1+h}-1}{h}$.\\\\

    %----------b)
    \item Elabore una tabla de valores de la razón promedio de cambio de $g$ con respecto a $x$ en el intervalo $[1, 1 + h]$ para algunos valores de $h$ cercanos a cero, digamos, $h = 0.1$, $0.01$, $0.001,$ $0.0001$, $0.00001$ y $0.000001.$\\\\
	Respuesta.-\; 
	\begin{center}
	    \begin{tabular}{c|c|c|c|c|c|c}
		$h$&$0.1$&$0.01$&$0.001$&$0.0001$&$0.00001$&$0.000001$\\
		\hline
		$\dfrac{\sqrt{1+h}-1}{h}$&$0.488$&$0.4987$&$0.4998$&$0.49998$&$0.499998$&$0.4999998$\\\\
	    \end{tabular}
	\end{center}

    %----------c)
    \item De acuerdo con su tabla, ¿cuál es la razón de cambio de $g(x)$ con respecto a $x$ en $x = 1$?\\\\
	Respuesta.-\; 0.5\\\\

    %----------d)
    \item Calcule el límite, cuando $h$ se aproxima a cero, de la razón promedio de cambio de $g(x)$ con respecto a $x$ en el intervalo $[1, 1 + h].$\\\\

	Respuesta.-\; 0.5\\\\

\end{enumerate}

%--------------------20.
\item Sea $f(t)=1/t$ para $t\neq 0$.
\begin{enumerate}[\bfseries a)]
    
    %----------a)
    \item Obtenga la razón promedio de cambio de $f$ con respecto a $t$ en los intervalos $i.$ de $t = 2$ a $t = 3$, y $ii.$ de $t = 2$ a $t = T$.\\\\
	Respuesta.-\; Para $i.$ se tiene $\dfrac{\triangle y}{triangle x} = \dfrac{1/3 - 1/2}{3-2} = -\dfrac{1}{6}$. Para $ii.$ se tiene $\dfrac{\triangle y}{\triangle x} = \dfrac{1/T - 1/2}{T-2} = \dfrac{2-T}{2T(T-2)}$\\\\

    %----------b)
    \item Elabore una tabla de valores de la razón promedio de cambio de $f$ con respecto a $t$ en el intervalo $[2, T]$ para algunos valores de $T$ cercanos a $2$, digamos, $T = 2.1, 2.01, 2.001, 2.0001, 2.00001 y 2.000001.$\\\\
	Respuesta.-\; 
	\begin{center}
	    \begin{tabular}{c|c|c|c|c|c}
		$2.1$&$2.01$&$2.001$&$2.0001$&$2.00001$&$2.000001$\\
		\hline
		$-0.238$&$-0.2487$&$-0.2498$&$-0.24998$&$-0.24999$&$-0.25$\\\\
	    \end{tabular}
	\end{center}

    %----------c)
    \item De acuerdo con su tabla, ¿cuál es la razón de cambio de $f$ con respecto a $t$ en $t = 2$?\\\\
	Respuesta.-\; -0.25.\\\\

    %----------d)
    \item Calcule el límite, cuando $T$ se aproxima a $2$, de la razón promedio de cambio de $f$ con respecto a $t$ en el intervalo de $2$ a $T$. Tendrá que hacer algo de álgebra antes de que pueda sustituir $T = 2.$\\\\
	Respuesta.-\; Por la parte $a), ii.$ tenemos $\dfrac{2-T}{2T(T-2)}$ de donde $-\dfrac{1}{2T}$y por lo tanto $-\dfrac{1}{4}$.\\\\

\end{enumerate}
    
%--------------------21.
\item La siguiente gráfica muestra la distancia total $s$, que recorre un ciclista después de $t$ horas.
\begin{enumerate}[\bfseries a)]

    %----------a)
    \item Estime la velocidad promedio del ciclista en los intervalos de tiempo $[0, 1], [1, 2.5] y [2.5, 3.5].$\\\\
	Respuesta.-\; 
	\begin{center}
	    \begin{tabular}{c|c|c}
		$[0,1]$&$[1,2.5]$&$[2.5,3.5]$\\
		\hline
		$15$&$3.33$&$10$\\\\
	    \end{tabular}
	\end{center}

    %----------b)
    \item Estime la velocidad instantánea del ciclista en los tiempos $t = \dfrac{1}{2}, t = 2$ y $t = 3.$
	Respuesta.-\; 
	\begin{center}
	    \begin{tabular}{c|c|c}
		$1/2$&$2$&$3$\\
		\hline
		$12$&$0$&$4$\\\\
	    \end{tabular}
	\end{center}

    %----------c)
    \item Estime la velocidad máxima del ciclista y el tiempo específico en que ésta se registra.\\\\
	Respuesta.-\; Tenemos la velocidad máxima dada por $\dfrac{30-20}{3.5-3} = 20$ en la hora $3.5$.\\\\

\end{enumerate}

%--------------------22.
\item La siguiente gráfica muestra la cantidad total de gasolina $A$ en el tanque de un automóvil después de conducirlo $t$ días.

\begin{enumerate}[\bfseries a)]

    %----------a)
    \item Estime la razón promedio del consumo de gasolina en los intervalos de tiempo $[0, 3], [0, 5] y [7, 10].$\\\\
	Respuesta.-\; 
	\begin{center}
	    \begin{tabular}{c|c|c}
	     $[0,3]$&$[0,5]$&$[7,10]$\\
	     \hline
	     $-5/3$&$-2.24$&$0.5$\\\\
	    \end{tabular}
	\end{center}

    %----------b)
    \item Estime la razón instantánea de consumo de gasolina en los tiempos $t = 1, t = 4$ y $t = 8$.
	Respuesta.-\; 
	\begin{center}
	    \begin{tabular}{c|c|c}
	     $1$&$4$&$8$\\
	     \hline
	     $-1$&$-4$&$-1/2$\\\\
	    \end{tabular}
	\end{center}

    %----------c)
    \item Estime la razón máxima de consumo de gasolina y el tiempo específico en que ésta se registra.\\\\
	Respuesta.-\; con una razón de $-4$, en el día $4$.\\\\ 

\end{enumerate}
\end{enumerate}


\section{Límites de una función y leyes de los límites}

%--------------------teorema 1
\begin{tcolorbox}[colframe=white]
\begin{teo} Si $L,M,c$ y $k$ son números reales y $$\lim_{x\to c} f(x) = L \quad y \quad \lim_{x\to c} g(x) = M, \quad entonces$$\\
\begin{enumerate}[\bfseries 1.]
    
    %----------1.
    \item Regla de la suma: $\displaystyle\lim_{x\to c} \left[f(x)+g(x)\right]=L+M$\\\\

    %----------2.
    \item Regla de la diferencia: $\displaystyle\lim_{x\to c} \left[f(x)-g(x)\right] = L-M$\\\\

    %----------3.
    \item Regla del múltiplo constante: $\displaystyle\lim_{x\to c} \left[k\cdot f(x)\right] = k\cdot L$\\\\

    %----------4.
    \item Regla del producto: $\displaystyle\lim_{x\to c} \left[f(x)\cdot g(x)\right] = L\cdot M$\\\\

    %----------5.
    \item Regla del cociente: $\displaystyle\lim_{x\to c} = \dfrac{L}{M}, \; M\neq 0$\\\\

    %----------6.
    \item Regla de la potencia: $\displaystyle\lim_{x\to c} \left[f(x)\right]^n = L^n, \;n$ es un entero positivo\\\\

    %----------7.
    \item Regla de la raíz: $\displaystyle\lim_{x\to c} \sqrt[n]{f(x)} = \sqrt[n]{L} = L^{1/n}, \; n$ es un entero positivo.\\\\

\end{enumerate}
\end{teo}
\end{tcolorbox}

%--------------------teorema 2
\begin{tcolorbox}[colframe = white]
    \begin{teo}[Límites de las funciones polinomiales]
	Si $P(x) = a_nx^n + a_{n-1}x^{n-1}+...+a_0$ entonces, 
	$$\lim_{x\to c}P(x) = P(c) = a_nc^n + a_{n-1}c^{n-1} + ... + a_0$$.
    \end{teo}
\end{tcolorbox}

%--------------------teorema 3 
\begin{tcolorbox}[colframe = white]
    \begin{teo}[Límites de las funciones racionales]
	Si $P(x)$ y $Q(x)$ son polinomios, y $Q(c)\neq 0$, entonces, $$\lim_{x\to c}\dfrac{P(x)}{Q(x)} = \dfrac{P(c)}{Q(c)}.$$
    \end{teo}
\end{tcolorbox}

%--------------------teorema 4 
\begin{tcolorbox}[colframe = white]
    \begin{teo}[El teorema del sándwich]
	Suponga que $g(x)\leq f(x)\leq h(x)$ para toda $x$ en algún intervalo abierto que contenga a $c$, excepto posiblemente en $x=c$. Suponga también que $$\lim_{x\to c} g(x) = \lim_{x\to c} h(x) = L$$
	entonces $\lim\limits_{x\to c} f(x) = L$.
    \end{teo}
\end{tcolorbox}

%--------------------teorema 5 
\begin{tcolorbox}[colframe = white]
    \begin{teo} Si $f(x)\leq g(x)$ para toda $x$ en un intervalo abierto que contiene a $c$, excepto posiblemente en $x=c$, y los límites de $f$ y $g$ cuando $x$ se aproxima a $c$, entonces, 
    $$\lim_{x\to c} f(x) \leq \lim_{x\to c} g(x).$$
    \end{teo}
\end{tcolorbox}


\setcounter{section}{1}
\section{Ejercicios}

\textbf{Límites a partir de gráficas}\\

\begin{enumerate}[\Large\bfseries 1.]

%--------------------1.
\item Para la función $g(x)$ cuya gráficas a continuación, determine los siguientes límites o explique por qué no existen.
\begin{enumerate}[\bfseries a)]

    %----------a)
    \item $\lim\limits_{x\to 1} g(x)$.\\\\
    Respuesta.-\; No existe. Cuando $x$ se aproxima a $1$ por la derecha, $g(x)$ se aproxima a $0$. Cuando $x$ se aproxima a $1$ por la izquierda, $g(x)$ se aproxima a $1$. No hay un número único $L$ para el que todos los valores de $g(x)$ estén arbitrariamente cerca cuando ${x \to 1}$.\\\\

    %----------b)
    \item $\lim\limits_{x\to 2} g(x) = 2$.\\\\

    %----------c)
    \item $\lim\limits_{x\to 3} g(x) = 0$.\\\\

    %----------d)
    \item $\lim_{x\to 2.5} g(x) = 0.5$.\\\\

\end{enumerate}

%--------------------2.
\item Para la función $f(t)$ cuya gráfica aparece a continuación, determine los siguientes límites o explique por qué no existen.
\begin{enumerate}[\bfseries a)]
    
    %----------a)
    \item $\lim\limits_{t\to -2} f(t) = 0$.\\\\

    %----------b)
    \item $\lim\limits_{t\to -1} f(t) = -1$.\\\\

    %----------c)
    \item $\lim\limits_{t\to 0} f(t) = 0$.\\\\

    %----------d)
    \item $\lim\limits_{t\to -0.5} f(t) = -1$.\\\\

\end{enumerate}

%--------------------3.
\item ¿Cuáles de los siguientes enunciados acerca de la función $y = f(x)$, cuya gráfica aparece a continuación, son verdaderos y cuáles son falsos?
\begin{enumerate}[\bfseries a)]

    %----------a)
    \item $\lim\limits_{x\to 0}f(x)$ existe.\\\\
	Respuesta.-\; Verdadero.\\\\

    %----------b)
    \item $\lim\limits_{x\to 0}f(x) = 0$\\\\
	Respuesta.-\; Verdadero.\\\\

    %----------c)
    \item $\lim\limits_{x\to 0}f(x) = 1$\\\\
	Respuesta.-\; Falso.\\\\

    %----------d)
    \item $\lim\limits_{x\to 1}f(x) = 1$\\\\
	Respuesta.-\; Falso.\\\\

    %----------e)
    \item $\lim\limits_{x\to 1}f(x) = 0$\\\\
	Respuesta.-\; Falso.\\\\

    %----------f)
    \item $\lim\limits_{x\to c}f(x)$ existe en todos los puntos $c$ en $(-1,1)$\\\\
	Respuesta.-\; Verdadero.\\\\

    %----------g)
    \item $\lim\limits_{x\to 1}f(x)$ no existe.\\\\
	Respuesta.-\; Verdadero.\\\\

\end{enumerate}

%--------------------4.
\item ¿Cuáles de los siguientes enunciados acerca de la función $y = f(x)$, cuya gráfica aparece a continuación, son verdaderos y cuáles son falsos?
\begin{enumerate}[\bfseries a)]

    %----------a)
    \item $\lim\limits_{x\to 2}f(x)$ no existe.\\\\
	Respuesta.-\; Falso.\\\\

    %----------b)
    \item $\lim\limits_{x\to 2}f(x) = 2$\\\\
	Respuesta.-\; Falso.\\\\

    %----------c)
    \item $\lim\limits_{x\to 1}f(x)$ no existe.\\\\
	Respuesta.-\; Verdadero.\\\\

    %----------d)
    \item $\lim\limits_{x\to c}f(x)$ existe en todos los puntos $c$ en $(-1,1).$\\\\
	Respuesta.-\; Verdadero.\\\\

    %----------e)
    \item $\lim\limits_{x\to c}f(x)$ existe en todos los puntos $c$ en $(1,3).$\\\\
	Respuesta.-\; Verdadero.\\\\

\end{enumerate}

\textbf{Existencia de límites}\\\\
En los ejercicios 5 y 6, explique por qué los límites no existen.\\\\

%--------------------5.
\item $\lim\limits_{x\to 0} \dfrac{x}{|x|}$.\\\\
    Respuesta.-\; No existe ya que si $x$ se aproxima a $1$ por la derecha, $g(x)$ se aproxima a $0$. Y cuando $x$ se aproxima a $1$ por la izquierda $g(x)$ se aproxima a $1$. Por lo tanto no hay un número único $L$ para que todos los valores de $g(x)$ estén arbitrariamente cerca cuando $x\to 1$.\\\\

%--------------------6.
\item $\lim\limits_{x\to 1} \dfrac{1}{x-1}$\\\\
    Respuesta.-\; De similar manera al anterior ejercicio se tiene $L=\infty$ cuando $x$ se aproxima por la izquierda. Y $L=-\infty$ cuando $x$ se aproxima por la derecha.\\\\

%--------------------7.
\item Suponga que una función $f(x)$ está definida para todos los valores reales de $x$, excepto para $x = c$. ¿Qué puede decirse con respecto a la existencia de $\lim\limits_{x\to c} f(x)$? Justifique su respuesta.\\\\
    Respuesta.-\; El límite existe debido a que $x$ se aproxima a $c$ tanto por la parte derecha como por la parte izquierda lo más que pueda, esto siempre y cuando sea un único $L$.\\\\ 

%--------------------8.
\item Suponga que una función $f(x)$ está definida para toda $x$ en $[-1, 1]$. ¿Qué puede decirse con respecto a la existencia de $\lim\limits_{x\to 0} f(x)$? Justifique su respuesta.\\\\
    Respuesta.-\; De similar forma al ejercicio anterior el límite existe.\\\\ 

%--------------------9.
\item Si $\lim\limits_{x\to 1} f(x) = 5$, ¿debe estar definida $f$ en $x = 1$? Si es así, ¿$f(1)$ debe ser igual a $5$? ¿Es posible concluir algo con respecto a los valores de $f$ en $x = 1$? Explique.\\\\
    Respuesta.-\; No necesariamente debe estar definida en $x=1$. No necesariamente debe ser igual a $5$. No es posible concluir algo concreto.\\\\

%--------------------10.
\item Si $f(1)=5$ ¿debe existir el $\lim\limits_{x\to 1} f(x)$? Si es así. ¿$\lim\limits_{x\to 1} f(x)$ debe ser igual a $5$? ¿Es posible concluir algo respecto del $\lim\limits_{x\to 1} f(x)$? Explique.\\\\
    Respuesta.-\; Si debería existir. Si debería ser igual a $5$. Se puede concluir que $\lim\limits_{x\to 1} f(x) = 5$.\\\\

\textbf{Cálculo de límites}\\\\
En los ejercicios 11 a 22, encuentre los límites.\\\\

\begin{multicols}{2}

%--------------------11.
\item $\lim\limits_{x\to -3} (x^2 - 13) = -4$.\\\\

%--------------------12.
\item $\lim\limits_{t\to 2} (-x^2+5x-2) = 12$.\\\\

%--------------------13.
\item $\lim\limits_{t\to 6} (t-5)(t-7) = -1$.\\\\

%--------------------14.
\item $\lim\limits_{x\to -2} (x^3 - 2x^2 + 4x + 8) = -16$.\\\\

%--------------------15.
\item $\lim\limits_{x\to 2} \dfrac{2x+5}{11-x^3} = 3$.\\\\

%--------------------16.
\item $\lim\limits_{s\to 2/3} (8-3s)(2s-1) = 2$.\\\\

%--------------------17.
\item $\lim\limits_{x\to -1/2} 4x(3x+4)^2 = -\dfrac{25}{2}$.\\\\

%--------------------18.
\item $\lim\limits_{y\to 2} \dfrac{y+2}{y^2+5y+6} = \dfrac{1}{5}$.\\\\

%--------------------19.
\item $\lim\limits_{y\to -3} (5-y)^{4/3} = 16$.\\\\

%--------------------20.
\item $\lim\limits_{z\to 4} \sqrt{z^2-10} = \sqrt{6}$.\\\\

%--------------------21.
\item $\lim\limits_{h\to 0}\dfrac{3}{\sqrt{3h+1}+1} = \dfrac{3}{2}$.\\\\

\end{multicols}

%--------------------22.
\item $\lim\limits_{h\to 0} \dfrac{\sqrt{5h+4}-2}{h} = \lim\limits_{h\to 0} \dfrac{\sqrt{5h+4}-2}{h} \cdot \dfrac{\sqrt{5h+4}+2}{\sqrt{5h+4}+2} = \lim\limits_{h\to 0}  \dfrac{5h}{h\sqrt{5h+4}+2} = \dfrac{5}{4}$.\\\\

\textbf{Límites de cocientes} Encuentre los límites en los ejercicios 23 a 42.\\\\

%--------------------23.
\item $\lim\limits_{x\to 5} \dfrac{x-5}{x^2 - 25} \Longrightarrow \lim\limits_{x\to 5}\dfrac{x-5}{(x+5)(x-5)} = \dfrac{1}{10}$.\\\\

%--------------------24.
\item $\lim\limits_{x\to -3} \dfrac{x+3}{x^2 + 4x + 3} \Longrightarrow \lim\limits_{x\to -3} \dfrac{x+3}{(x+3)(x+1)} = -\dfrac{1}{2}$\\\\

%--------------------25.
\item $\lim\limits_{x\to -5}\dfrac{x^2 + 3x - 10}{x+5} \Longrightarrow \lim\limits_{x\to -5}\dfrac{(x+5)(x-2)}{x+5} = -7$\\\\

%--------------------26.
\item $\lim\limits_{x\to 2} = \dfrac{x^2-7x + 10}{x-2} \Longrightarrow \lim\limits_{x\to 2}\dfrac{(x-5)(x-2)}{x-2} = -3$\\\\

%--------------------27.
\item $\lim\limits_{t\to 1} \dfrac{t^2+t-2}{t^2-1} \Longrightarrow \lim\limits_{t\to 1}\dfrac{(t+2)(t-1)}{(t+1)(t-1)} = \dfrac{3}{2}$\\\\

%--------------------28.
\item $\lim\limits_{t\to -1} \dfrac{t^2+3t+2}{t^2-t-2} \Longrightarrow\lim\limits_{x\to -1}\dfrac{(t+2)(t+1)}{(t-2)(t+1)} = -\dfrac{1}{3}$\\\\

%--------------------29.
\item $\lim\limits_{x\to -2}\dfrac{-2x-4}{x^3+2x^2} \Longrightarrow \lim\limits_{x\to -2}\dfrac{-2(x+2)}{x^2(x+2)} = -\dfrac{1}{2}$\\\\

%--------------------30.
\item $\lim\limits_{y\to 0}\dfrac{5y^3+8^2}{3y^3-16y^2} \Longrightarrow \lim\limits_{y\to 0} \dfrac{y^2(5y+8)}{y^2(3^2-16)} = -\dfrac{1}{2}$\\\\

%--------------------31.
\item $\lim\limits_{x\to 1} \dfrac{x^{-1}-1}{x-1} \Longrightarrow \lim\limits_{x\to 1} -\dfrac{x+1}{x(x+1)} = -1$\\\\ 

%--------------------32.
\item $\lim\limits_{x\to 0} \dfrac{\frac{1}{x-1}+\frac{1}{x+1}}{x} \Longrightarrow \lim\limits_{x\to 0} \dfrac{2x}{x(x^2-1)} = -2$\\\\

%--------------------33.
\item $\lim\limits_{u\to 1}\dfrac{u^4 - 1}{u^3 - 1} \Longrightarrow \lim\limits_{u\to 1} \dfrac{(u-1)(u+1)(u^2+1)}{(u-1)(u^2+u+1)} = \dfrac{4}{3}$\\\\

%--------------------34.
\item $\lim\limits_{u\to 2} \dfrac{u^3-8}{u^4-16} \Longrightarrow \lim\limits_{u\to 2} \dfrac{(x-2)(x^2+2x+4)}{(x-2)(x+2)(x^2+4)} = \dfrac{3}{8}$\\\\

%--------------------35.
\item $\lim\limits_{u\to 1} \dfrac{\sqrt{x}-3}{x-9} \Longrightarrow \lim\limits_{u\to 1}\dfrac{\sqrt{x}-3}{x-9}\cdot \dfrac{\sqrt{2}+3}{\sqrt{x}+3} \Longrightarrow \lim\limits_{u\to 1}\dfrac{x-9}{(x-9)\sqrt{x}+3} = \dfrac{1}{4}$\\\\

%--------------------36.
\item $\lim\limits_{x\to 4}\dfrac{4x-x^2}{2-\sqrt{x}} \Longrightarrow \lim\limits_{x\to 4}\dfrac{-x(x-4)}{2-\sqrt{x}}\cdot \dfrac{2+\sqrt{x}}{2+\sqrt{x}} \Longrightarrow \lim\limits_{x\to 4} \dfrac{-x(x-4)(2+\sqrt{x})}{4-x} = -16$\\\\

%--------------------37.
\item $\lim\limits_{x\to 1} \dfrac{x-1}{\sqrt{x+3}-2} \Longrightarrow \lim\limits_{x\to 1} \dfrac{x-1}{\sqrt{x+3}-2} \cdot \dfrac{\sqrt{x+3}+2}{\sqrt{x+3}+2} \Longrightarrow \lim\limits_{x\to 1} \dfrac{(x-1)(\sqrt{x+3}+2)}{x-1} = 4 $\\\\

%--------------------38.
\item $\lim\limits_{x\to -1} \dfrac{\sqrt{x^2+8}-3}{x+1} \Longrightarrow \lim\limits_{x\to -1} \dfrac{\sqrt{x^2+8}-3}{x+1} \cdot \dfrac{\sqrt{x^2+8}+3}{\sqrt{x^2+8}+3} \Longrightarrow \lim\limits_{x\to -1}\dfrac{(x - 1)(x+1)}{(x+1)(\sqrt{x^2+8}+3)} = -\dfrac{1}{3}$\\\\

%--------------------39.
\item $\lim\limits_{x\to 2}\dfrac{\sqrt{x^2+12}-4}{x-2} \Longrightarrow \lim\limits_{x\to 2} \dfrac{\sqrt{x^2+12}-4}{x-2} \cdot \dfrac{\sqrt{x^2+12}+4}{\sqrt{x^2+12}+4} \Longrightarrow \lim\limits_{x\to 2} \dfrac{(x-2)(x+2)}{(x-2)(\sqrt{x^2+12}+4)} = \dfrac{1}{2}$\\\\

%--------------------40.
\item $\lim\limits_{x\to -2} \dfrac{x+2}{\sqrt{x^2+5}-3} \Longrightarrow \lim\limits_{x\to -2} \dfrac{x+2}{\sqrt{x^2+5}-3}\cdot \dfrac{\sqrt{x^2+5}+3}{\sqrt{x^2+5}+3} \Longrightarrow \lim\limits_{x\to -2} \dfrac{(x+2)(\sqrt{x^2+5}+3)}{(x-2)(x+2)} = \dfrac{3}{2}$\\\\

%--------------------41.
\item $\lim\limits_{x\to -3}\dfrac{2-\sqrt{x^2-5}}{x+3} \Longrightarrow \lim\limits_{x\to -3}\dfrac{2-\sqrt{x^2-5}}{x+3}\cdot \dfrac{2+\sqrt{x^2-5}}{2+\sqrt{x^2-5}} \Longrightarrow \lim\limits_{x\to -3}\dfrac{-(x-3)(x+3)}{(x+3)(2+\sqrt{x^2-5})} = \dfrac{3}{2}$\\\\

%--------------------42.
\item $\lim\limits_{x\to 4} \dfrac{4-x}{5-\sqrt{x^2+9}} \Longrightarrow \lim\limits_{x\to 4} \dfrac{4-x}{5-\sqrt{x^2+9}}\cdot \dfrac{5+\sqrt{x^2+9}}{5+\sqrt{x^2+9}} \Longrightarrow \lim\limits_{x\to 4}\dfrac{(4-x)(5+\sqrt{x^2+9})}{(4-x)(x+4)} = \dfrac{5}{4}$\\\\

\textbf{Límites con funciones trigonométricas} En los ejercicios 43 a 50, encuentre los límites.\\\\

%--------------------43.
\item 

\end{enumerate}
