\chapter{Límites y continuidad}

%--------------------definición 2.1

\begin{tcolorbox}[colframe = white]
    \begin{def.}[La razón promedio de cambio] de $y=f(x)$ con respecto a $x$ en el intervalo $[x_1,x_2]$ sabiendo que $\triangle x = x_2 - x_1 = h$ es $$\dfrac{\triangle y}{\triangle x} = \dfrac{f(x_2) - f(x_1)}{x_2-x_1} = \dfrac{f(x_1+h) - f(x_1)}{h}, \quad h\neq 0$$
    \end{def.}
\end{tcolorbox}

%-------------------- Ejercicios 2.1
\section{Ejercicios}

\textbf{Razones promedio de cambio}\\\\\
En los ejercicios 1 a 6, determine la razón promedio de cambio de la función en el intervalo o intervalos dados.\\\\

\begin{enumerate}[\large\bfseries 1.]

%--------------------1.
\item $f(x) = x^3 + 1$ 
\begin{enumerate}[\bfseries a)]
    
    %----------a)
    \item $[2,3]$\\\\
	Respuesta.-\; $\dfrac{\triangle y}{\triangle x} = \dfrac{(3^3 + 1) - (2^3 + 1)}{3 - 2} = 19$\\\\

    %----------b)
    \item $[-1,1]$\\\\
	Respuesta.-\; $\dfrac{\triangle y}{\triangle x} = \dfrac{(1^3 + 1)-((-1)^3 + 1)}{1-(-1)} = 1$\\\\

\end{enumerate}

%--------------------2.
\item $g(x) = x^2 - 2x$
\begin{enumerate}[\bfseries a)]
    
    %----------a)
    \item $[1,3]$\\\\
	Respuesta.-\; $\dfrac{\triangle y}{\triangle x} = \dfrac{(3^2 - 2\cdot 3) - (1^2 - 2\cdot 1)}{3-1} = 2 $\\\\ 
    
    %----------b)
    \item $[-2,4]$\\\\
	Respuesta.-\; $\dfrac{\triangle y}{\triangle x} = \dfrac{(4^2 - 2\cdot 4) - ((-2)^2 - 2\cdot (-2))}{4-(-2)} = 0$\\\\

\end{enumerate}

%--------------------3.
\item $h(t) = cot \; t$
\begin{enumerate}[\bfseries (a)]

    %----------(a)
    \item $[\pi/4,3\pi/4]$\\\\
	Respuesta.-\; $\dfrac{\triangle y}{\triangle x} = \dfrac{cot(\pi/4)-cot(3\pi/4)}{\pi/4-3\pi/4} = \dfrac{1+1}{\dfrac{\pi - 3\pi}{4}} = \dfrac{8}{-2\pi}$\\\\

    %----------(b)
    \item $[\pi/6,\pi/2]$\\\\
	Respuesta.-\; $\dfrac{\triangle y}{\triangle x} = \dfrac{cot(\pi/6)-cot(\pi/2)}{\pi/6-\pi/2} = \dfrac{-3\sqrt{3}}{\pi}$\\\\

\end{enumerate}

%--------------------4.
\item $g(t) = 2 + \cos t$ 
\begin{enumerate}[\bfseries (a)]

    %----------(a)
    \item $[0,\pi]$\\\\
	Respuesta.-\; $\dfrac{2+\cos \pi - (2+\cos 0)}{\pi - 0} = -\dfrac{2}{\pi}$\\\\

    %----------(b)
    \item $[-\pi,\pi]$\\\\
	Respuesta.-\; $\dfrac{2 + \cos \pi - (2 - \cos \pi)}{\pi+\pi} = \dfrac{3-3}{2\pi} = 0$\\\\

\end{enumerate}

%--------------------5.
\item $R(\theta) = \sqrt{4\theta + 1}; \; [0,2]$\\\\ 
    Respuesta.-\; $\dfrac{\sqrt{4*2+1}+1 - (\sqrt{4*0+1}+1)}{2-0} = \dfrac{2}{2} = 1$\\\\

%--------------------6.
\item $P(\theta) = \theta^3 - 4\theta^2 + 5\theta; \; [1,2]$\\\\
    Respuesta.-\; $\dfrac{2^3 - 4\cdot 2^2 + 5\cdot 2 - (1^3 - 4^2 + 5)}{2-1} = 0$\\\\

\textbf{Pendiente de una curva en un punto}\\\\
En los ejercicios 7 a 14, utilice el método del ejemplo 3 para determinar $a)$ la pendiente de la curva en el punto $P$ dado, y $b)$ la ecuación de la recta tangente en $P$\\\\

%--------------------7.
\item $y = x^2 - 5, \quad P(2,-1)$

\begin{enumerate}[\bfseries a)]
    
    %----------a)
    \item Iniciamos con una recta secante que pasa por el punto $(2,-1)$ y el punto cercano $(2+h,(2+h)^2 - 5)$, luego hallamos la pendiente de la secante, $$\dfrac{\triangle y}{\triangle x} = \dfrac{(2+h)^2 - 5 - (2^2 - 5)}{2+h - 2} = \dfrac{4h+h^2}{h} = 4+h$$ 
    Luego aproximamos  $h$ a $0$ siendo la pendiente $m=4$.\\\\

    %----------b)
    \item La ecuación de la recta tangente viene dado por $y=mx + c$  de donde $y=4x+c$, luego reemplazamos $(2,-1)$, quedándonos $-1=4\cdot 2 + c \; \Longrightarrow\; c = -9$. Por lo tanto $$y=4x-9$$\\

\end{enumerate}

%--------------------8.
\item $y=7-x^2, \quad P(2,3)$ 

\begin{enumerate}[\bfseries a)]
    
    %----------a)
    \item Sea la recta secante que pasa por el punto $P(2,3)$ y el punto cercano $Q\left[2+h,7-(2+h)^2\right]$, hallamos la pendiente de la secante, $$\dfrac{\triangle y}{\triangle x} = \dfrac{7-(2+h)^2 - (7-2^2)}{2+h-2} = \dfrac{7-(2+h)^2 - 3}{h} = \dfrac{h(-h-4)}{h} = -h-4$$
    Con forme $Q$ se aproxima a $P$ a lo largo de la curva, $h$ se aproxima a cero, y la pendiente de la secante $-h-4$ se aproxima a $-4$. Tomamos $-4$ como la pendiente de la parábola en $P$.\\\\

    %----------b)
    \item La ecuación de la recta tangente viene dado por $y = mx + c$ de donde $y=-4x+c$, luego reemplazamos $(2,3)$, así $3 = -4(2) + c \Longrightarrow c = 11$. Por lo tanto $$y = -4x + 11$$.\\

\end{enumerate}

%--------------------9.
\item $y=x^2-2x-3, \quad P(2,-3)$

\begin{enumerate}[\bfseries a)]
    
    %----------a)
    \item Sea la recta secante que pasa por le punto $P(2,-3)$  y el punto cercano $Q\left[2+h, (2+h)^2 - 2(2+h) - 3\right]$, hallamos la pendiente de la secante, $$\dfrac{\triangle y}{\triangle x} = \dfrac{(2+h)^2 - 2(2+h) - 3 - (-3)}{h} = \dfrac{h^2 + 2h}{h} = h+2$$
    Con forme $Q$ se aproxima a $P$ a lo largo de la curva, $h$ se aproxima a cero, y la pendiente de la secante $h+2$ se aproxima a $2$. Tomamos $2$ como la pendiente de la parábola en $P$.\\\\

    %----------b)
    \item La ecuación de la recta tangente viene dado por $y = mx + c$ de donde $y=2x+c$, luego reemplazamos $(2,-3)$, así $-3 = 2(2) + c \Longrightarrow c = -7$. Por lo tanto $$y = -2x - 7$$.\\

\end{enumerate}

%--------------------10.
\item $y=x^2-4x, \quad P(1,-3)$ 

\begin{enumerate}[\bfseries a)]
    
    %----------a)
    \item Sea la recta secante que pasa por le punto $P(1,-3)$  y el punto cercano $Q\left[1+h, (1+h)^2 - 4(1+h) \right]$, hallamos la pendiente de la secante, $$\dfrac{\triangle y}{\triangle x} = \dfrac{(1+h)^2 - 4(1+h) - (-3)}{1+h - 1} = \dfrac{h^2 - 2h}{h} = h-2$$
    Con forme $Q$ se aproxima a $P$ a lo largo de la curva, $h$ se aproxima a cero, y la pendiente de la secante $h-2$ se aproxima a $-2$. Tomamos $-2$ como la pendiente de la parábola en $P$.\\\\

    %----------b)
    \item Tenemos $$\dfrac{\triangle y}{\triangle x} = m = \dfrac{y - y_1}{x - x_1} \; \Longrightarrow \; y-y_1 = m(x-x_1) \; \Longrightarrow \; y = -2(x-1) + (-3) \; \Longrightarrow \; y = -2x+2 - 3$$ por lo tanto $$y = -2x-1$$\\
     

\end{enumerate}

%--------------------11.
\item $y=x^3, \quad P(2,8)$

\begin{enumerate}[\bfseries a)]
    
    %----------a)
    \item Sea la recta secante que pasa por le punto $P(2,8)$  y el punto cercano $Q\left[2+h, (2+h)^3 \right]$, hallamos la pendiente de la secante, $$\dfrac{\triangle y}{\triangle x} = \dfrac{(2+h)^3 - 8}{2+h - 2} = \dfrac{h^3 + 6h^2 + 12h}{h} = h^2+6h+12$$
    Con forme $Q$ se aproxima a $P$ a lo largo de la curva, $h$ se aproxima a cero, y la pendiente de la secante $h^2 + 6h + 12$ se aproxima a $12$. Tomamos $12$ como la pendiente de la parábola en $P$.\\\\

    %----------b)
    \item Tenemos $$\dfrac{\triangle y}{\triangle x} = m = \dfrac{y - y_1}{x - x_1} \; \Longrightarrow \; y = m(x-x_1) + y_1\; \Longrightarrow \; y = 12(x-2) + 8 \; \Longrightarrow \; y = 12x - 24 + 8$$ por lo tanto $$y = 12x-16$$\\

\end{enumerate}

%--------------------12.
\item $y=2-x^3, \quad P(1,1)$

\begin{enumerate}[\bfseries a)]
    
    %----------a)
    \item Sea la recta secante que pasa por le punto $P(1,1)$  y el punto cercano $Q\left[1+h, 2-(1+h)^3 \right]$, hallamos la pendiente de la secante, $$\dfrac{\triangle y}{\triangle x} = \dfrac{2-(1+h)^3 - 1}{1+h - 1} = \dfrac{h^3 + 3h^2 + 3h}{h} = h^2+3h+3$$
    Con forme $Q$ se aproxima a $P$ a lo largo de la curva, $h$ se aproxima a cero, y la pendiente de la secante $h^2 + 3h + 3$ se aproxima a $3$. Tomamos $3$ como la pendiente de la parábola en $P$.\\\\

    %----------b)
    \item Tenemos $$\dfrac{\triangle y}{\triangle x} = m = \dfrac{y - y_1}{x - x_1} \; \Longrightarrow \; y = m(x-x_1) + y_1\; \Longrightarrow \; y = 3(x-1) + 1 \; \Longrightarrow \; y = 3x - 3 + 1$$ por lo tanto $$y = 3x-2$$\\

\end{enumerate}

%--------------------13.
\item $y=x^3 - 12x, \quad P(1,-11)$

\begin{enumerate}[\bfseries a)]
    
    %----------a)
    \item Sea la recta secante que pasa por le punto $P(1,-11)$  y el punto cercano $Q\left[1+h, (1+h)^3 -12(1+h)\right]$, hallamos la pendiente de la secante, $$\dfrac{\triangle y}{\triangle x} = \dfrac{(1+h)^3 - 12(1+h) + 11}{1+h - 1} = \dfrac{h^3 + 3h^2 - 9h}{h} = h^2+3h-12$$
    Con forme $Q$ se aproxima a $P$ a lo largo de la curva, $h$ se aproxima a cero, y la pendiente de la secante $h^2 + 3h - 9$ se aproxima a $-9$. Tomamos $-9$ como la pendiente de la parábola en $P$.\\\\

    %----------b)
    \item Tenemos $$\dfrac{\triangle y}{\triangle x} = m = \dfrac{y - y_1}{x - x_1} \; \Longrightarrow \; y = m(x-x_1) + y_1\; \Longrightarrow \; y = -9(x-1) - 11 \; \Longrightarrow \; y = -9x + 9 - 11$$ por lo tanto $$y = -9x-2$$\\

\end{enumerate}

%--------------------14.
\item $y=x^3-3x^2+4,\quad P(2,0)$

\begin{enumerate}[\bfseries a)]
    
    %----------a)
    \item Sea la recta secante que pasa por le punto $P(2,0)$  y el punto cercano $Q\left[2+h, (2+h)^3 -3(2+h)^2 + 4\right]$, hallamos la pendiente de la secante, $$\dfrac{\triangle y}{\triangle x} = \dfrac{(2+h)^3 - 3(2+h)^2 + 4 - 0}{2+h - 2} = \dfrac{h^3 + 3h^2}{h} = h^2+3h$$
    Con forme $Q$ se aproxima a $P$ a lo largo de la curva, $h$ se aproxima a cero, y la pendiente de la secante $h^2 + 3h$ se aproxima a $0$. Tomamos $0$ como la pendiente de la parábola en $P$.\\\\

    %----------b)
    \item Tenemos $$\dfrac{\triangle y}{\triangle x} = m = \dfrac{y - y_1}{x - x_1} \; \Longrightarrow \; y = m(x-x_1) + y_1\; \Longrightarrow \; y = 0(x-2) - 0$$ por lo tanto $$y = 0$$\\

\end{enumerate}

\textbf{Razones instantáneas de cambio}\\\\

%--------------------15.
\item

\end{enumerate}
