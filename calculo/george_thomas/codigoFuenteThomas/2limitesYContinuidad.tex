\chapter{Límites y continuidad}

%--------------------definición 2.1

\begin{tcolorbox}[colframe = white]
    \begin{def.}[La razón promedio de cambio] de $y=f(x)$ con respecto a $x$ en el intervalo $[x_1,x_2]$ sabiendo que $\triangle x = x_2 - x_1 = h$ es $$\dfrac{\triangle y}{\triangle x} = \dfrac{f(x_2) - f(x_1)}{x_2-x_1} = \dfrac{f(x_1+h) - f(x_1)}{h}, \quad h\neq 0$$
    \end{def.}
\end{tcolorbox}

%-------------------- Ejercicios 2.1
\section{Ejercicios}

\textbf{Razones promedio de cambio}\\\\\
En los ejercicios 1 a 6, determine la razón promedio de cambio de la función en el intervalo o intervalos dados.\\\\

\begin{enumerate}[\large\bfseries 1.]

%--------------------1.
\item $f(x) = x^3 + 1$ 
\begin{enumerate}[\bfseries a)]
    
    %----------a)
    \item $[2,3]$\\\\
	Respuesta.-\; $\dfrac{\triangle y}{\triangle x} = \dfrac{(3^3 + 1) - (2^3 + 1)}{3 - 2} = 19$\\\\

    %----------b)
    \item $[-1,1]$\\\\
	Respuesta.-\; $\dfrac{\triangle y}{\triangle x} = \dfrac{(1^3 + 1)-((-1)^3 + 1)}{1-(-1)} = 1$\\\\

\end{enumerate}

%--------------------2.
\item $g(x) = x^2 - 2x$
\begin{enumerate}[\bfseries a)]
    
    %----------a)
    \item $[1,3]$\\\\
	Respuesta.-\; $\dfrac{\triangle y}{\triangle x} = \dfrac{(3^2 - 2\cdot 3) - (1^2 - 2\cdot 1)}{3-1} = 2 $\\\\ 
    
    %----------b)
    \item $[-2,4]$\\\\
	Respuesta.-\; $\dfrac{\triangle y}{\triangle x} = \dfrac{(4^2 - 2\cdot 4) - ((-2)^2 - 2\cdot (-2))}{4-(-2)} = 0$\\\\

\end{enumerate}

%--------------------3.
\item $h(t) = cot \; t$
\begin{enumerate}[\bfseries (a)]

    %----------(a)
    \item $[\pi/4,3\pi/4]$\\\\
	Respuesta.-\; $\dfrac{\triangle y}{\triangle x} = \dfrac{cot(\pi/4)-cot(3\pi/4)}{\pi/4-3\pi/4} = \dfrac{1+1}{\dfrac{\pi - 3\pi}{4}} = \dfrac{8}{-2\pi}$\\\\

    %----------(b)
    \item $[\pi/6,\pi/2]$\\\\
	Respuesta.-\; $\dfrac{\triangle y}{\triangle x} = \dfrac{cot(\pi/6)-cot(\pi/2)}{\pi/6-\pi/2} = \dfrac{-3\sqrt{3}}{\pi}$\\\\

\end{enumerate}

%--------------------4.
\item $g(t) = 2 + \cos t$ 
\begin{enumerate}[\bfseries (a)]

    %----------(a)
    \item $[0,\pi]$\\\\
	Respuesta.-\; $\dfrac{2+\cos \pi - (2+\cos 0)}{\pi - 0} = -\dfrac{2}{\pi}$\\\\

    %----------(b)
    \item $[-\pi,\pi]$\\\\
	Respuesta.-\; $\dfrac{2 + \cos \pi - (2 - \cos \pi)}{\pi+\pi} = \dfrac{3-3}{2\pi} = 0$\\\\

\end{enumerate}

%--------------------5.
\item $R(\theta) = \sqrt{4\theta + 1}; \; [0,2]$\\\\ 
    Respuesta.-\; $\dfrac{\sqrt{4*2+1}+1 - (\sqrt{4*0+1}+1)}{2-0} = \dfrac{2}{2} = 1$\\\\

%--------------------6.
\item $P(\theta) = \theta^3 - 4\theta^2 + 5\theta; \; [1,2]$\\\\
    Respuesta.-\; $\dfrac{2^3 - 4\cdot 2^2 + 5\cdot 2 - (1^3 - 4^2 + 5)}{2-1} = 0$\\\\

\textbf{Pendiente de una curva en un punto}\\\\
En los ejercicios 7 a 14, utilice el método del ejemplo 3 para determinar $a)$ la pendiente de la curva en el punto $P$ dado, y $b)$ la ecuación de la recta tangente en $P$\\\\

%--------------------7.
\item 

\end{enumerate}
