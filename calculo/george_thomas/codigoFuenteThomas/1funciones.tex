\chapter{Funciones}

\section{Las funciones y sus gráficas}

    %--------------------definición 1.1.
    \begin{tcolorbox}[colframe=white]
	\begin{def.}
	    Una función $f$ de un conjunto $D$ a un conjunto $Y$ es una regla que asigna a cada elemento $x \in D$ un solo o único elemento $f(x) \in Y$\\
	\end{def.}
    \end{tcolorbox}

    %--------------------definición 1.2.
    \begin{tcolorbox}[colframe=white]
	\begin{def.}
	    Cuando definimos una función $y = f(x)$ mediante una fórmula, y el dominio no se establece de forma explícita o se restringe por el contexto, se supondrá que el dominio será el mayor conjunto de números reales $x$ para los cuales la fórmula proporciona valores reales para $y$ , el llamado \textbf{dominio natural}.\\\\
	    Cuando el rango de una función es un subconjunto de números reales, se dice que la función tiene \textbf{valores reales (o que es real valuada)}\\
	\end{def.}
    \end{tcolorbox}

    %--------------------definición 1.3.
    \begin{tcolorbox}[colframe=white]
	\begin{def.}[Valor absoluto]
	    $f(x)=\left\{\begin{array}{rcl}
		x&si&x\geq 0\\
		\\ -x&si&x<0\\
	    \end{array} \right.$
	\end{def.}
    \end{tcolorbox}

    %--------------------definición 1.4.
    \begin{tcolorbox}[colframe=white]
	\begin{def.}
	    Sea una funcion definida en un intervalo $I$ y sean $x_1$ y $x_2$ cualesquiera dos puntos en $I$ 
	    \begin{enumerate}[\bfseries 1.]
		\item Si $f(x_2) > f(x_1)$, siempre que $x_1<x_2$ entonces se dice que $f$ es \textbf{creciente} en $I$.
		\item Si $f(x_2) < f(x_1)$, siempre que $x_1<x_2$ entonces se dice que $f$ es $\textbf{decreciente}$ en $I$.\\
	    \end{enumerate}
	\end{def.}
    \end{tcolorbox}

    %--------------------definición 1.5.
    \begin{tcolorbox}[colframe=white]
	\begin{def.}
	Una función $y=f(x)$ es una 
	    \begin{enumerate}[\bfseries 1.]
		\item Función par de $x$ si $f(-x)=f(x)$.
		\item Función impar de $x$ si $f(-x)=-f(x)$.
	    \end{enumerate}
	Para toda $x$ en el dominio de la función.\\\\
	(Los nombres par e impar provienen de las potencias de $x$).\\
	\end{def.}
    \end{tcolorbox}

    %--------------------definición 1.6.
    \begin{tcolorbox}[colframe=white]
	\begin{def.}
	    Dos variables $x$ e $y$ son \textbf{proporcionales} (una con respecto a la otra) si una siempre es un múltiplo constante de la otra; esto es, si $y=kx$ para alguna constante $k$ distinta de $0$.\\\\
	    Si la variable $y$ es proporcional al recíproco $1/x$, entonces algunas veces se dice que $y$ es \textbf{inversamente proporcional} a $x$ (puesto que $1/x$ es el inverso multiplicativo de $x$).\\

	\end{def.}
    \end{tcolorbox}

\setcounter{section}{0}
\section{Ejercicios}

    \begin{enumerate}[\Large \bfseries 1.]

	%--------------------1.
	\item $f(x)=1+x^2$ \\\\
	Respuesta.-\;

    \end{enumerate}
