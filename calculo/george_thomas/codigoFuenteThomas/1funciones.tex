\chapter{Funciones}

\section{Las funciones y sus gráficas}

    %--------------------definición 1.1.
    \begin{tcolorbox}[colframe=white]
	\begin{def.}
	    Una función $f$ de un conjunto $D$ a un conjunto $Y$ es una regla que asigna a cada elemento $x \in D$ un solo o único elemento $f(x) \in Y$\\
	\end{def.}
    \end{tcolorbox}

    %--------------------definición 1.2.
    \begin{tcolorbox}[colframe=white]
	\begin{def.}
	    Cuando definimos una función $y = f(x)$ mediante una fórmula, y el dominio no se establece de forma explícita o se restringe por el contexto, se supondrá que el dominio será el mayor conjunto de números reales $x$ para los cuales la fórmula proporciona valores reales para $y$ , el llamado \textbf{dominio natural}.\\\\
	    Cuando el rango de una función es un subconjunto de números reales, se dice que la función tiene \textbf{valores reales (o que es real valuada)}\\
	\end{def.}
    \end{tcolorbox}

    %--------------------definición 1.3.
    \begin{tcolorbox}[colframe=white]
	\begin{def.}[Valor absoluto]
	    $f(x)=\left\{\begin{array}{rcl}
		x&si&x\geq 0\\
		\\ -x&si&x<0\\
	    \end{array} \right.$
	\end{def.}
    \end{tcolorbox}

    %--------------------definición 1.4.
    \begin{tcolorbox}[colframe=white]
	\begin{def.}
	    Sea una funcion definida en un intervalo $I$ y sean $x_1$ y $x_2$ cualesquiera dos puntos en $I$ 
	    \begin{enumerate}[\bfseries 1.]
		\item Si $f(x_2) > f(x_1)$, siempre que $x_1<x_2$ entonces se dice que $f$ es \textbf{creciente} en $I$.
		\item Si $f(x_2) < f(x_1)$, siempre que $x_1<x_2$ entonces se dice que $f$ es $\textbf{decreciente}$ en $I$.\\
	    \end{enumerate}
	\end{def.}
    \end{tcolorbox}

    %--------------------definición 1.5.
    \begin{tcolorbox}[colframe=white]
	\begin{def.}
	Una función $y=f(x)$ es una 
	    \begin{enumerate}[\bfseries 1.]
		\item Función par de $x$ si $f(-x)=f(x)$.
		\item Función impar de $x$ si $f(-x)=-f(x)$.
	    \end{enumerate}
	Para toda $x$ en el dominio de la función.\\\\
	(Los nombres par e impar provienen de las potencias de $x$).\\
	\end{def.}
    \end{tcolorbox}

    %--------------------definición 1.6.
    \begin{tcolorbox}[colframe=white]
	\begin{def.}
	    Dos variables $x$ e $y$ son \textbf{proporcionales} (una con respecto a la otra) si una siempre es un múltiplo constante de la otra; esto es, si $y=kx$ para alguna constante $k$ distinta de $0$.\\\\
	    Si la variable $y$ es proporcional al recíproco $1/x$, entonces algunas veces se dice que $y$ es \textbf{inversamente proporcional} a $x$ (puesto que $1/x$ es el inverso multiplicativo de $x$).\\

	\end{def.}
    \end{tcolorbox}


\setcounter{section}{0}
\section{Ejercicios}

\begin{enumerate}[\Large \bfseries 1.]

    %--------------------1.
    \item $f(x)=1+x^2$ \\\\
	Respuesta.-\; Al evaluar $1+x^2$ vemos que $x$ se cumple para todos los reales, por lo tanto $f_D=\lbrace x /; \forall \; x \in \mathbb{R} \rbrace$. Luego el rango viene dado por $f_R=\lbrace y=f(x) / y \geq 1 \rbrace$\\\\

    %--------------------2.
    \item $f(x)=1-\sqrt{x}$\\\\
       Respuesta.-\; El dominio viene dado por $f_D=\lbrace x / x \geq 0 \rbrace$. Y el rango viene dado por $f_R = \lbrace y = f(x) / y \leq 1 \rbrace$.\\\\

    %--------------------3.
    \item $F(x)=\sqrt{5x + 10}$\\\\
	Respuesta.-\; Sea $5x + 10 \geq 0$ ya que una raíz par no puede ser no negativo, entonces $x \geq 2$, por lo tanto el dominio viene dado por $f_D=\lbrace x / x \geq -2 \rbrace$. Luego el rango viene dado por $f_R = \lbrace y=f(x) / y \geq 0 \rbrace$.\\\\

    %--------------------4.
    \item $g(x)=\sqrt{x^2 - 3x}$\\\\
	Respuesta.-\; De igual forma al anterior ejercicio, evaluaremos $x^2 - 3x \geq 0$, de donde $x(x-3)\geq 0$, por lo tanto el dominio es $f_D=\lbrace x/\leq x \leq 0 \cup x \geq 3 \rbrace$. Luego el rango viene definido por $f_R=\lbrace y=f(x) / y \geq 0 \rbrace$.\\\\

    %--------------------5.
    \item $f(t)=\dfrac{4}{3-t}$ \\\\
	Respuesta.-\; Sabemos que no se puede dividir un número por $0$. Por lo tanto para hallar el dominio de la función debemos evaluar $3-t=0$, de donde $t=3$, así $f_D=\lbrace t / t\neq 3\rbrace$. Luego el rango viene dado por $f_R=\lbrace y=f(x) / y\neq 0\rbrace$.\\\\

    %--------------------6.
    \item $G(t)=\dfrac{2}{t^2 - 16}$ \\\\
	Respuesta.-\; De igual forma al anterior ejercicio evaluamos $t^2 - 16 = 0$, de donde $(t - 4)(t + 4)=0$, por lo tanto el dominio de la función viene dado por $f_D=\lbrace t / t \neq 4 \land t \neq -4 \rbrace$. Luego el rango viene dado por $f_R=\lbrace y=f(x) / 0 < y \leq - \dfrac{1}{8} \rbrace$ ya que al despejar $x$ nos queda  $x=\sqrt{\dfrac{2}{y} + 16}$ de donde se debe evaluar por un lado $\dfrac{2}{y}$ y por otro $\dfrac{2}{y} - 16 \geq 0$.\\\\

    En los ejercicios $7$ y $8$ ¿Cuál de las gráficas representa la gráfica de una función de $x$? ¿Cuáles no representan a funciones de $x$? Dé razones que apoyen sus respuestas.\\\\

    %--------------------7.
    \item El inciso $a.$ no es una función ya que no cumple con la prueba de la recta vertical ya una función sólo puede tener un valor $f(x)$ para cada $x$ en su dominio. Y el inciso $b.$ no representa la gráfica de una función.\\\\

    %--------------------8.
    \item Los incisos $a.$ y $b.$ no representan a funciones de $x$. El único que no representa una gráfica de una función es el inciso $b.$\\\\

    Determinación de fórmulas para funciones.\\\\

    %--------------------9.
    \item Exprese el área y el perímetro de un triángulo equilátero como una función del lado $x$ del triángulo.\\\\
	Respuesta.-\; El área se representa por $f(x)=\dfrac{\sqrt{3}a^2}{4}$ y el perímetro por $f(x)=3x$\\\\

    %--------------------10.
    \item Exprese la longitud del lado de un cuadrado como una función de la longitud $d$ de la diagonal del cuadrado. Exprese el área como una función de la longitud de la diagonal.\\\\
	Respuesta.-\; La longitud del lado de un cuadrado como función de longitud esta dado por $d=\sqrt{2a^2}$. El área es expresado por $A=\dfrac{d^2}{2}$\\\\

    %--------------------11.
    \item Exprese la longitud del lado de um cubo como una función de la longitud de la diagonal $d$ del cubo. Exprese el área de la superficie y el volumen del cubo como una función de la longitud de la diagonal.\\\\
	Respuesta-.\; La expresión de la longitud del lado del cubo como función de la longitud de la diagonal $d$ del cubo es  $$ L(d) = (\sqrt{2}/2)\cdot d $$ 
	Las expresiones del área de la superficie y el volumen del cubo como función de la longitud de la diagonal $d$ del cubo son:
	$$A(d) = 3\cdot d² \quad    y \quad  V(d) = (\sqrt{2}/4)\cdot d³$$

    %--------------------12.
    \item Un punto $P$ en el primer cuadrante pertenece a la gráfica de la función $f(x)=\sqrt{x}$. Exprese las coordenadas de $P$ como funciones de la pendiente de la recta que une a $P$ con el origen.\\\\
	Respuesta.-\; Sea el punto en el origen $(0,0)$ y el punto $P$ tenga las coordenadas $(z,z^{'})$. Sabemos que una recta viene definido por $f(x)=ax+b$ entonces formando un sistema de ecuaciones tenemos:
	$$0=0x + b \quad y \quad z^{'} = az + b$$
	Luego $z^{'}=az$ de donde $a=\dfrac{z^{'}}{z}$, y así nos queda la función  
	$$f(x)=\dfrac{z^{'}}{z} x$$\\\\

    %--------------------13.
    \item Considere el punto $(x,y)$ que está en la gráfica de la recta $2x + 4y = 5$. Sea $L$ la distancia del punto $(x, y)$ al origen $(0, 0)$. Escriba $L$ como función de $x$.\\\\
	Respuesta.-\; Dado $(x,y) \in 2x+4y=5 ; (0,0)$ entonces $$x=\dfrac{5-4y}{2} \qquad \dfrac{5-2x}{4}$$
	Luego $L=\sqrt{(y-0)^2+(x-0)^2} = \sqrt{y^2 + \left( \dfrac{5-4y}{2}\right)}=\sqrt{y^2 + \dfrac{25+40y+16y^2}{4}}=\sqrt{\dfrac{4y^2}{4} + \dfrac{25 - 40y + 16y^2}{4}} = \dfrac{1}{2} \sqrt{20y^2 + 40y + 25}$\\\\
 
    %--------------------14.
    \item Considere el punto $(x, y)$ que está en la gráfica de $y = \sqrt{x - 3}$. Sea $L$ la distancia entre los puntos $(x,y)$ y $(4,0)$. Escriba $L$ como función de $y$.\\\\
	Respuesta.-\; $y=\sqrt{x-3}, (x,y)\in y=\sqrt{x-3}$ entonces calculamos la distancia entre $y=\sqrt{x-3}$ y $(4,0)$.\\
	$$y^2=x-3 \Longrightarrow x=y^2 + 3 \quad y \quad y=\sqrt{x-3}$$ 
	Así $L=\sqrt{(y-o)^2 + (x-y)^2} = \sqrt{y^2 + (y^2 + 3)^2} = \sqrt{y^2 + y^4 + 6y^2 + 9} = \sqrt{y^4+7y^2+9}$\\\\

\end{enumerate}
