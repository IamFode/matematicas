\chapter{Límite, continuidad de la función}

\section{Limite de la magnitud variable, variable infinitamente grande}

    %--------------------definición 2.1
    \begin{tcolorbox}[colframe = white]
	\begin{def.}
	    El número constante $a$ se denomina límite de la variable $x$, si para cualquier número infinitesimal positivo $\epsilon$ prefijado, se puede indicar tal valor de la variable $x$, a partir del cual todos los valores posteriores de la misma satisfacen la desigualdad: $$|x-a| < \epsilon$$ Si el número $a$ es el límite de la variable $x$, se dice que $x$ tiende al límite $a$; su notación es: $$x \longrightarrow a \quad ó \quad \lim x = a$$ En términos geométricos la definición de limite puede enunciarse así: El número constante $a$ es el limite de la variable $x$, si para cualquiera vecindad infinitesimal prefijada de radio $\epsilon$ y centro en el punto $a$, existe un valor de $x$ tal que todo los puntos correspondientes a los valores posteriores de la variable se encuentren dentro de la misma vecindad.
	\end{def.}
    \end{tcolorbox}

    %--------------------teorema 2.1
    \begin{teo}
	Una magnitud variable no puede tener dos límites.\\\\
	    Demostración.-\; En efecto, si $\lim x = a$ y $\lim x = b (a<b)$, entonces $x$ debe satisfacer las dos desigualdades simultáneamente: $|x-a|<\epsilon$ y $|x-b|<\epsilon$ siendo $\epsilon$ arbitrariamente pequeño, pero esto es imposible, si $\epsilon < \dfrac{b-a}{2}$\\\\
    \end{teo}

    %--------------------definición 2.2
    \begin{tcolorbox}[colframe = white]
	\begin{def.}
	    La variable $x$ tiende al infinito, si para cualquier número positivo $M$ prefijado se puede elegir un valor de $x$ tal que, a partir de él todos los valores posteriores de la variable satisfagan la desigualdad $|x|>M$.\\
	    La variable $x$ que tiende al infinito, se denomina infinitamente grande y esta tendencia se expresa así: $x \longrightarrow \infty$. 
	\end{def.}
    \end{tcolorbox}

\section{Limite de la función}
    
    %-------------------definición 2.3
    \begin{tcolorbox}[colframe = white]
	\begin{def.}
	    Supongamos que la función $y=f(x)$ está definida en determinada vecindad del punto $a$ en ciertos puntos de la misma.\\
	    La función $y=f(x)$ tiende al límite $b$ $(y\rightarrow b)$ cuando $x$ tienda a $a$ $(x\rightarrow a)$, si para cada número positivo $\epsilon$, por pequeño que éste sea, es posible indicar un número positivo $\delta$ tal que para todos los valores $x$, diferentes de $a$, que satisfacen la desigualdad: $|x-a|<\delta$ , se verificará la desigualdad: $$|f(x)-b|< \epsilon$$ Si $b$ es el límite de la función $f(x)$, cuando $x\rightarrow a$, su notación es: $$\lim_{x \to a} f(x)$$ o bien $f(x)\rightarrow b,$ cuando $x\rightarrow a.$\\
	    Si la variable $y=f(x)$ tiende a un límite $b$, cuando $x$ tiende a $a$, escribimos: $$\lim_{x to a} f(x) = b$$
	\end{def.}
    \end{tcolorbox}

    %--------------------definición 2.4
    \begin{tcolorbox}[colframe = white]
	\begin{def.}
	    La función $f(x)$ tiende al límite $b$ cuando $x\rightarrow \infty$, si para cualquier número positivo $\epsilon$ arbitrariamente pequeño existe un número positivo $N$ tal que para todos los valores de $x$ que satisfacen la desigualdad $|x| > N$, se cumpla la desigualdad $$|f(x) - b| < \epsilon.$$
	\end{def.}
    \end{tcolorbox}

\section{Función que tiende al infinito. Funciones acotadas}

    %--------------------definición 2.5
    \begin{tcolorbox}[colframe=white]
	\begin{def.}
	    La función $f(x)$ tiende al infinito cuando $x \rightarrow a,$ es decir, es una magnitud infinitamente grande cuando $x\rightarrow a,$ si para cualquier número positivo $M$, por grande que sea, existe un valor $\delta > 0$ tal que para todos los valores de $x$ diferentes de $a$ y que satisfacen la condición $|x-a|< \delta$, se cumpla la desigualdad $|f(x)|>M$.\\
	    Si $f(x)$ tiende al infinito cuando $x \rightarrow a,$ se escribe $$\lim_{x \to a} f(x) = \infty$$
	\end{def.}
    \end{tcolorbox}

    %--------------------definición 2.5
    \begin{tcolorbox}[colframe=white]
	\begin{def.}
	    La función $y=f(x)$ se denomina acotada en el dominio dado de variación del argumento $x$, si existe un número positivo $M$ tal que para todos los valores de $x$ pertenecientes al dominio considerado se cumpla la desigualdad $|f(x)|\leq M.$ Si el número $M$ no existe, se dice que la función $f(x)$ no está acotada en el dominio dado.
	\end{def.}
    \end{tcolorbox}

    %--------------------definición 2.6
    \begin{tcolorbox}[colframe=white]
	\begin{def.}
	    La función $f(x)$ se denomina acotada, cuando $x \to a,$ si existe una vecindad con centro en el punto $a$ en la cual dicha función está acotada.
	\end{def.}
    \end{tcolorbox}

    %--------------------definición 2.7
    \begin{tcolorbox}[colframe=white]
	\begin{def.}
	    La función $y=f(x)$ se denomina acotada, cuando $x \to \infty$, si existe un número $N>0$ tal que para todos los valores de $x$ que satisfacen la desigualdad $|x| > N$, la función $f(x)$ esté acotada.
	\end{def.}
    \end{tcolorbox}

    %--------------------teorema 2.2
    \begin{teo}
	Si $\displaystyle\lim_{x \to a} f(x) = b,$ siendo $b$ un número finito, la función $f(x)$ está acotada cuando $x\to a$.\\\\
	    Demostración.-\; Por definición de límite se deduce que para $\epsilon > 0$ existe un número $\delta$ tal que $a-\delta<x<a+\delta$ se cumple la desigualdad $$|f(x) - b| < \epsilon$$ es decir $$|f(x)|<|b|+\epsilon$$.\\\\
    \end{teo}

