\chapter{Límite, continuidad de la función}

\section{Limite de la magnitud variable, variable infinitamente grande}

    %--------------------definición 2.1
    \begin{tcolorbox}[colframe = white]
	\begin{def.}
	    El número constante $a$ se denomina límite de la variable $x$, si para cualquier número infinitesimal positivo $\epsilon$ prefijado, se puede indicar tal valor de la variable $x$, a partir del cual todos los valores posteriores de la misma satisfacen la desigualdad: $$|x-a| < \epsilon$$ Si el número $a$ es el límite de la variable $x$, se dice que $x$ tiende al límite $a$; su notación es: $$x \longrightarrow a \quad ó \quad \lim x = a$$ En términos geométricos la definición de limite puede enunciarse así: El número constante $a$ es el limite de la variable $x$, si para cualquiera vecindad infinitesimal prefijada de radio $\epsilon$ y centro en el punto $a$, existe un valor de $x$ tal que todo los puntos correspondientes a los valores posteriores de la variable se encuentren dentro de la misma vecindad.
	\end{def.}
    \end{tcolorbox}

    %--------------------teorema 2.1
    \begin{teo}
	Una magnitud variable no puede tener dos límites.\\\\
	    Demostración.-\; En efecto, si $\lim x = a$ y $\lim x = b (a<b)$, entonces $x$ debe satisfacer las dos desigualdades simultáneamente: $|x-a|<\epsilon$ y $|x-b|<\epsilon$ siendo $\epsilon$ arbitrariamente pequeño, pero esto es imposible, si $\epsilon < \dfrac{b-a}{2}$\\\\
    \end{teo}

    %--------------------definición 2.2
    \begin{tcolorbox}[colframe = white]
	\begin{def.}
	    La variable $x$ tiende al infinito, si para cualquier número positivo $M$ prefijado se puede elegir un valor de $x$ tal que, a partir de él todos los valores posteriores de la variable satisfagan la desigualdad $|x|>M$.\\
	    La variable $x$ que tiende al infinito, se denomina infinitamente grande y esta tendencia se expresa así: $x \longrightarrow \infty$. 
	\end{def.}
    \end{tcolorbox}

\section{Limite de la función}

