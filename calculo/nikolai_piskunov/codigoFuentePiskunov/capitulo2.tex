\chapter{Límite, continuidad de la función}

\section{Limite de la magnitud variable, variable infinitamente grande}

    %--------------------definición 2.1
    \begin{tcolorbox}[colframe = white]
	\begin{def.}
	    El número constante $a$ se denomina límite de la variable $x$, si para cualquier número infinitesimal positivo $\epsilon$ prefijado, se puede indicar tal valor de la variable $x$, a partir del cual todos los valores posteriores de la misma satisfacen la desigualdad: $$|x-a| < \epsilon$$ Si el número $a$ es el límite de la variable $x$, se dice que $x$ tiende al límite $a$; su notación es: $$x \longrightarrow a \quad ó \quad \lim x = a$$ En términos geométricos la definición de limite puede enunciarse así: El número constante $a$ es el limite de la variable $x$, si para cualquiera vecindad infinitesimal prefijada de radio $\epsilon$ y centro en el punto $a$, existe un valor de $x$ tal que todo los puntos correspondientes a los valores posteriores de la variable se encuentren dentro de la misma vecindad.
	\end{def.}
    \end{tcolorbox}

    %--------------------teorema 2.1
    \begin{teo}
	Una magnitud variable no puede tener dos límites.\\\\
	    Demostración.-\; En efecto, si $\lim x = a$ y $\lim x = b (a<b)$, entonces $x$ debe satisfacer las dos desigualdades simultáneamente: $|x-a|<\epsilon$ y $|x-b|<\epsilon$ siendo $\epsilon$ arbitrariamente pequeño, pero esto es imposible, si $\epsilon < \dfrac{b-a}{2}$\\\\
    \end{teo}

    %--------------------definición 2.2
    \begin{tcolorbox}[colframe = white]
	\begin{def.}
	    La variable $x$ tiende al infinito, si para cualquier número positivo $M$ prefijado se puede elegir un valor de $x$ tal que, a partir de él todos los valores posteriores de la variable satisfagan la desigualdad $|x|>M$.\\
	    La variable $x$ que tiende al infinito, se denomina infinitamente grande y esta tendencia se expresa así: $x \longrightarrow \infty$. 
	\end{def.}
    \end{tcolorbox}

\section{Limite de la función}
    
    %-------------------definición 2.3
    \begin{tcolorbox}[colframe = white]
	\begin{def.}
	    Supongamos que la función $y=f(x)$ está definida en determinada vecindad del punto $a$ en ciertos puntos de la misma.\\
	    La función $y=f(x)$ tiende al límite $b$ $(y\rightarrow b)$ cuando $x$ tienda a $a$ $(x\rightarrow a)$, si para cada número positivo $\epsilon$, por pequeño que éste sea, es posible indicar un número positivo $\delta$ tal que para todos los valores $x$, diferentes de $a$, que satisfacen la desigualdad: $|x-a|<\delta$ , se verificará la desigualdad: $$|f(x)-b|< \epsilon$$ Si $b$ es el límite de la función $f(x)$, cuando $x\rightarrow a$, su notación es: $$\lim_{x \to a} f(x)$$ o bien $f(x)\rightarrow b,$ cuando $x\rightarrow a.$\\
	    Si la variable $y=f(x)$ tiende a un límite $b$, cuando $x$ tiende a $a$, escribimos: $$\lim_{x to a} f(x) = b$$
	\end{def.}
    \end{tcolorbox}

    %--------------------definición 2.4
    \begin{tcolorbox}[colframe = white]
	\begin{def.}
	    La función $f(x)$ tiende al límite $b$ cuando $x\rightarrow \infty$, si para cualquier número positivo $\epsilon$ arbitrariamente pequeño existe un número positivo $N$ tal que para todos los valores de $x$ que satisfacen la desigualdad $|x| > N$, se cumpla la desigualdad $$|f(x) - b| < \epsilon.$$
	\end{def.}
    \end{tcolorbox}

\section{Función que tiende al infinito. Funciones acotadas}

    %--------------------definición 2.5
    \begin{tcolorbox}[colframe=white]
	\begin{def.}
	    La función $f(x)$ tiende al infinito cuando $x \rightarrow a,$ es decir, es una magnitud infinitamente grande cuando $x\rightarrow a,$ si para cualquier número positivo $M$, por grande que sea, existe un valor $\delta > 0$ tal que para todos los valores de $x$ diferentes de $a$ y que satisfacen la condición $|x-a|< \delta$, se cumpla la desigualdad $|f(x)|>M$.\\
	    Si $f(x)$ tiende al infinito cuando $x \rightarrow a,$ se escribe $$\lim_{x \to a} f(x) = \infty$$
	\end{def.}
    \end{tcolorbox}

    %--------------------definición 2.5
    \begin{tcolorbox}[colframe=white]
	\begin{def.}
	    La función $y=f(x)$ se denomina acotada en el dominio dado de variación del argumento $x$, si existe un número positivo $M$ tal que para todos los valores de $x$ pertenecientes al dominio considerado se cumpla la desigualdad $|f(x)|\leq M.$ Si el número $M$ no existe, se dice que la función $f(x)$ no está acotada en el dominio dado.
	\end{def.}
    \end{tcolorbox}

    %--------------------definición 2.6
    \begin{tcolorbox}[colframe=white]
	\begin{def.}
	    La función $f(x)$ se denomina acotada, cuando $x \to a,$ si existe una vecindad con centro en el punto $a$ en la cual dicha función está acotada.
	\end{def.}
    \end{tcolorbox}

    %--------------------definición 2.7
    \begin{tcolorbox}[colframe=white]
	\begin{def.}
	    La función $y=f(x)$ se denomina acotada, cuando $x \to \infty$, si existe un número $N>0$ tal que para todos los valores de $x$ que satisfacen la desigualdad $|x| > N$, la función $f(x)$ esté acotada.
	\end{def.}
    \end{tcolorbox}

    %--------------------teorema 2.2
    \begin{teo}
	Si $\displaystyle\lim_{x \to a} f(x) = b,$ siendo $b$ un número finito, la función $f(x)$ está acotada cuando $x\to a$.\\\\
	    Demostración.-\; Por definición de límite se deduce que para $\epsilon > 0$ existe un número $\delta$ tal que $a-\delta<x<a+\delta$ se cumple la desigualdad $$|f(x) - b| < \epsilon$$ es decir $$|f(x)|<|b|+\epsilon$$.\\\\
    \end{teo}

    %--------------------teorema 2.3
    \begin{teo}
	Si $\lim\limits_{x \to a} f(x) - b \neq 0$, la función $y=\dfrac{1}{f(x)}$ está acotada, cuando ${x \to a}$.\\\\
	    Demostración.-\; De la hipótesis del teorema se deduce que para cualquier $\epsilon > 0$ arbitrario, en cierta vecindad del punto $x=a$ tendremos: $|f(x)-b|<\epsilon$, ó $||f(x)|-|b||<\epsilon$, ó $-\epsilon < |f(x)|-|b|< \epsilon$ ó $|b|-\epsilon < |f(x)| < |b|+\epsilon.$ De las últimas desigualdades se deduce: $$\dfrac{1}{|b|-\epsilon}>\dfrac{1}{|f(x)|}>\dfrac{1}{|b|+\epsilon}$$
	    Al tomar, por ejemplo, $\epsilon = \dfrac{1}{10} |b|$ tenemos $$\dfrac{10}{9|b|}>\dfrac{1}{|f(x)|}>\dfrac{10}{11|b|}.$$
	    lo que significa que la función $\dfrac{1}{f(x)}$ está acotada.\\\\
    \end{teo}

\section{Infinitesimales y sus principales propiedades}

%--------------------definición 2.8
\begin{tcolorbox}[colframe=white]
    \begin{def.}
	La función $\alpha = \alpha(x)$ se denomina infinitamente pequeña (infinitesimal), cuando ${x \to a}$ o cuando ${x \to \infty}$, si $\lim\limits_{x \to a} \alpha(x) = 0$ ó $\lim\limits_{x \to \infty} \alpha(x)=0$
    \end{def.}
\end{tcolorbox}

    %--------------------teorema2.4
    \begin{teo}
	Si la función $y=f(x)$ puede ser representada como suma del número constante $b$ y la magnitud infinitamente pequeña $\alpha$: $$y=b+\alpha$$ se tiene que $$\lim y = b \;(cuando \; {x \to a} \; ó \; {x \to \infty})$$ Recíprocamente, si $\lim y = b$, se puede escribir $y=b+\alpha$, de donde $\alpha$ es una magnitud infinitamente pequeña.\\\\
	    Demostración.-\; De la igualdad se deduce que $|y-b|=|\alpha|$. Pero cuando $epsilon$ es arbitrario todos los valores de $\alpha$, a partir de uno de ellos, satisfacen la desigualdad $|\alpha|<\epsilon$; entonces, para todos los valores de $y$, a partir de alguno de ellos, se cumplirá la desigualdad $|y-b|<\epsilon$, lo que significa que $\lim y = b$.\\
	    Recíprocamente: si $\lim y=b$, entonces para $epsilon$ arbitrario para todos los valores de $y$, a partir de uno de ellos, se verificará la desigualdad $|y-b|<\epsilon$. Pero, si designamos $y-b=\alpha$, entonces para todos los valores de $\alpha$, a partir de alguno de ellos, tendremos $|\alpha|<\epsilon$, de lo que significa que $\alpha$ es una magnitud infinitamente pequeña.\\\\
    \end{teo}

    %--------------------teorema2.5
    \begin{teo}
	Si $\alpha = \alpha(x)$ tiende a cero, cuando ${x \to a}$ (o cuando ${x \to \infty}$), sin reducirse a cero, se tendrá que $y=\dfrac{1}{\alpha}$  tiende a infinito.\\\\
	    Demostración.-\; Por grande que sea $M>0$ se cumplirá la desigualdad $\dfrac{1}{|\alpha|}>M$, siempre que se cumpla $|\alpha|>\dfrac{1}{M}$. La última desigualdad se cumplirá para todos los valores de $\alpha$, a partir de algunos de ellos, puesto que $\alpha(x) \to 0.$\\\\
    \end{teo}

    %-------------------- teorema 2.6
    \begin{teo}
	La suma algebraica de dos, tres o un número determinado de infinitesimales es una función infinitamente pequeña.\\\\
	    Demostración.-\; Nos limitaremos a dos sumando ya que la demostración es análoga para cualquier número de ellos.\\
	    Supongamos que $u(x)=\alpha(x) + \beta(x)$, donde $\lim\limits_{x \to a} \alpha(x)=0$ y $\lim\limits_{x \to a} \beta(x)=0$. Demostraremos que para cualquier $\epsilon >0$ tan pequeño como se quiera, se encontrará $\delta >0$ tal que, al satisfacer la desigualdad $|x-a|<\delta$, se verifica $|u|<\epsilon$. Puesto que $\alpha (x)$ es una magnitud infinitamente pequeña se encontrará $\delta$ tal que en la vecindad de radio $\delta_1$ y centro ubicado en el punto $a$, se verificará, también $|\alpha(x)|<\dfrac{\epsilon}{2}$.\\
	    Luego puesto que $\beta (x)$ es una magnitud infinitamente pequeña, en la vecindad del punto $a$ de radio $\delta_2$ tendremos $|\beta(x)|<\dfrac{\epsilon}{2}$.\\
	    Tomemos $\delta$ igual a la menor de las magnitudes $\delta_1$ y $\delta_2$. Entonces, en la vecindad del punto $a$ de radio $\delta$ se cumplirán las desigualdades $|\alpha|<\dfrac{\epsilon}{2}; \; |\beta|<\dfrac{\epsilon}{2}$. Por tanto, en esta vecindad tendremos: $$|u|=|\alpha (x) + \beta (x)|\leq |\alpha (x)| + |\beta (x)| < \dfrac{\epsilon}{2} + \dfrac{\epsilon}{2} = \epsilon$$ es decir, $|u|<\epsilon$, lo que se trataba de demostrar.\\
	    De modo análogo se demuestra el caso: $$\lim\limits_{x \to \infty} \alpha (x)=0, \qquad \lim\limits_{x \to \infty} \beta (x) =0$$\\\\
    \end{teo}

    %-------------------- teorema 2.7
    \begin{teo}
	El producto de una función infinitamente pequeña $\alpha = \alpha(x)$ por una función acotada $z=z(\alpha)$, cuando $x \to a$ ó $x \to \infty$ es una magnitud (Función) infinitamente pequeña.\\\\
	    Demostración.-\; Demostremos el teorema para el caso en que $x \to a$. Dado un número $M>0$, se encontrará tal vecindad del punto $x=a$ en la que se verificará la desigualdad $|z|<M$. Para cualquier $\epsilon > 0$ se encontrará una vecindad en la que se cumplirá la desigualdad $|\alpha|< \dfrac{\epsilon}{M}.$ En la menor de estas dos vecindades se cumplirá la desigualdad $$|\alpha z|< \dfrac{\epsilon}{M}M = \epsilon.$$ Esto quiere decir que $\alpha z$ es una magnitud infinitamente pequeña. Se demuestra análogamente para $x \to \infty$.\\\\
    \end{teo}

	%----------corolario 2.1
	\begin{cor}
	    Si $\lim \alpha = 0$ y $\lim \beta = 0$, entonces $\lim \alpha \beta = 0$, puesto que $\beta (x)$ es una magnitud acotada. Esto se cumple para cualquier número finito de factores.\\\\
	\end{cor}

	%----------corolario 2.2.
	\begin{cor}
	    Si $\lim \alpha = 0$ y $c=const$, entonces $\lim c\alpha =0$.\\\\
	\end{cor}

    %-------------------- teorema 2.8
    \begin{teo} El cociente $\dfrac{\alpha(x)}{z(x)}$ de la división de una magnitud infinitamente pequeña $\alpha(x)$ por una función, cuyo límite es diferente de cero, es una magnitud infinitamente pequeña.\\\\
	Demostración.-\; Supongamos que $\lim \alpha(x)=0$ y $\lim x(x)=b\neq 0$. Basándose en el teoremas se deduce que $dfrac{1}{z(\alpha)}$ es una magnitud acotada. Por consiguiente la fracción $\dfrac{\alpha(x)}{z(x)}=\alpha(x)\dfrac{1}{z(x)}$ es el producto de una magnitud infinitamente pequeña por otra acotada, es decir, una infinitesimal.\\\\
    \end{teo}


\section{Teoremas fundamentales sobre limites}
