\chapter{Número, variable y función}

\section{Números reales. Representación de número reales por medio de puntos en el eje numérico}

    %--------------------Definición 1.1.
    \begin{tcolorbox}[colframe=white]
	\begin{def.}
	    El número racional puede expresarse como la razón $\dfrac{p}{q}$ de dos números enteros $p$ y $q$.\\\\
	    El número entero $p$ se puede considerar como la razón de dos números enteros $\dfrac{p}{1}$.\\
	\end{def.}
    \end{tcolorbox}
       
    %--------------------Definición 1.2.
    \begin{tcolorbox}[colframe = white]
	\begin{def.}
	    Los números en forma de fracciones decimales indefinidas no periódicas, se denominan números irracionales.\\
	\end{def.}
    \end{tcolorbox}

    \begin{tcolorbox}[colframe=white]
	\begin{def.}
	    Para cualquier par de números reales $x$ e $y$ existen una correlación, y sólo una, de las siguientes:
	    $$x<y, \qquad x=y, \qquad x>y$$
	\end{def.}
    \end{tcolorbox}

	%--------------------teorema 1.1
	\begin{teo}
	    Todo número irracional $\alpha$ se puede expresar con cualquier grado de precisión por medio de números racionales.\\\\
	    Demostración.-\; En efecto, siendo el número irracional $\alpha>0$, calculamos $\alpha$ con un error no mayor de $\dfrac{1}{n}$ (por ejemplo,, de $\dfrac{1}{10}, \dfrac{1}{100},$ etc.)\\
	    Cualquiera que sea el número $\alpha$, está comprendido entre dos números enteros consecutivos $N$ y $N+1$. Dividamos el segmento comprendido entre $N$ y $N+1$ en n partes, entonces el número $\alpha$ resulta comprendido entre los número racionales $N + \dfrac{m}{n}$ y $N + \dfrac{m+1}{n}$. Dado que la diferencia entre estos números es $\dfrac{1}{n}$, cada uno de ellos expresa $\alpha$ con un grado de precisión predeterminado: El primero por defecto y el segundo por exceso.\\\\
	\end{teo}

\section{Valor absoluto del número real}

    %--------------------definición 1.3.
    \begin{tcolorbox}[colframe=white]
	\begin{def.}
	    Un número real no negativo, que satisface las condiciones: $$|x|=x, \; si \; x\geq 0;$$ $$|x| = -x, \; si \; x<0$$ se llama valor absoluto (o módulo) de un número real $x$.\\ 
	\end{def.}
    \end{tcolorbox}

    %-------------------- propiedades de valor absoluto

    %propiedad 1.1.
    \begin{tcolorbox}[colframe=white]
	\begin{prop}	
	    El valor absoluto de la suma albegraica de varios números reales no es mayor que la suma de los valores absolutos de los sumandos:
	    $$|x+y| \leq |x| +|y|$$\\\\
	    Demostración.-\; Sea $x+y\geq 0$, entonces:
	    $$|x+y| = x+y \leq |x| + |y| $$ (ya que $x\leq |x|$ e $y\leq |y|$). \\
	    Supongamos ahora que $x+y<0$, entonces:
	    $$|x+y| = -/x+y) = (-x) + (-y) \leq |x| + |y|,$$ como se trataba de demostrar.\\\\
	\end{prop}
    \end{tcolorbox}

    %propiedad 1.2.
    \begin{tcolorbox}[colframe=white]
	\begin{prop}
	  El valor absoluto de la diferencia de dos números no es mejor que la diferencia de los valores absolutos del minuendo y sustraendo:
	  $$|x-y|\geq |x| - |y|$$\\\\
	  Demostración\; Supongamos que $x-y=x$. Entonces $x=y+z$, y según lo demostrado anteriormente, se tiene:
	  $$|x|=|y+z| \leq |y| + |z| = |z| + |x-y|,$$ de donde $$|x| - |y| \leq |x-y|,$$ como se trataba de demostrar.\\\\
	\end{prop}
    \end{tcolorbox}

    %propiedades 1.3.
    \begin{tcolorbox}[colframe=white]
	\begin{prop}	
	    El valor absoluto del producto es igual al producto de los valores absolutos de los factores: $$|xyz| = |x||y||z|.$$
	\end{prop}
    \end{tcolorbox}

    %propiedad 1.4.
    \begin{tcolorbox}[colframe=white]
	\begin{prop}
	    El valor absoluto del cociente es igual al cociente de dividir el valor absoluto del dividendo por el del divisor:
	    $$\left| \dfrac{x}{y} \right| = \dfrac{|x|}{|y|}.$$\\
	\end{prop}
    \end{tcolorbox}

\section{Magnitudes variables y constantes}
