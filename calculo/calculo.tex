\documentclass[10pt]{book}
\usepackage[text=17cm,left=2.5cm,right=2.5cm, headsep=20pt, top=2.5cm, bottom = 2cm,letterpaper,showframe = false]{geometry} %configuración página
\usepackage{latexsym,amsmath,amssymb,amsfonts} %(símbolos de la AMS).7
\parindent = 0cm  %sangria
\usepackage[T1]{fontenc} %acentos en español
\usepackage[spanish]{babel} %español capitulos y secciones
\usepackage{graphicx} %gráficos y figuras.

%---------------FORMATO de letra--------------------%

\usepackage{lmodern} % tipos de letras
\usepackage{titlesec} %formato de títulos
\usepackage[backref=page]{hyperref} %hipervinculos
\usepackage{multicol} %columnas
\usepackage{tcolorbox, empheq} %cajas
\usepackage{enumerate} %indice enumerado
\usepackage{marginnote}%notas en el margen
\tcbuselibrary{skins,breakable,listings,theorems}
\usepackage[Bjornstrup]{fncychap}%diseño de portada de capitulos
\usepackage[all]{xy}%flechas
\counterwithout{footnote}{chapter}
\usepackage{xcolor}
\spanishdecimal{.}
\usepackage{pgfplots}
\usepgfplotslibrary{polar}

%------------------------------------------

\newtheorem{axioma}{\large\textbf{Axioma}}[part]
\newtheorem{teo}{\textbf{TEOREMA}}[chapter]%entorno para teoremas
\newtheorem{ejem}{{\textbf{EJEMPLO}}}[chapter]%entorno para ejemplos
\newtheorem{def.}{\textbf{Definición}}[chapter]%entorno para definiciones
\newtheorem{post}{\large\textbf{Postulado}}[chapter]%entorno de postulados
\newtheorem{col.}{\textbf{Corolario}}[chapter]
\newtheorem{ej}{\textbf{Ejercicio}}[chapter]
\newtheorem{prop}{\large\textbf{Propiedad}}[part]
\newtheorem{lema}{\textbf{LEMA}}[chapter]
\newtheorem{prob}{\textbf{problema}}[chapter]
\newtheorem{nota}{\textbf{Nota}}[chapter]

%--------------------GRÀFICOS--------------------------

\usepackage{tkz-fct}

%---------------------------------

\titleformat*{\section}{\LARGE\bfseries\rmfamily}
\titleformat*{\subsection}{\Large\bfseries\rmfamily}
\titleformat*{\subsubsection}{\large\bfseries\rmfamily}
\titleformat*{\paragraph}{\normalsize\bfseries\rmfamily}
\titleformat*{\subparagraph}{\small\bfseries\rmfamily}

\renewcommand{\thechapter}{\Roman{chapter}}
\renewcommand{\thesection}{\arabic{chapter}.\arabic{section}}
%------------------------------------------

\renewcommand{\labelenumi}{\Roman{enumi}.}%primer piso II) enumerate
\renewcommand{\labelenumii}{\arabic{enumii}$)$}%segundo piso 2)
\renewcommand{\labelenumiii}{\alph{enumiii}$)$}%tercer piso a)
\renewcommand{\labelenumiv}{$\bullet$}%cuarto piso (punto)

%----------Formato título de capítulos-------------

\usepackage{titlesec}
\renewcommand{\thechapter}{\arabic{chapter}}
\titleformat{\chapter}[display]
{\titlerule[2pt]
\vspace{4ex}\bfseries\sffamily\huge}
{\filleft\Huge\thechapter}
{2ex}
{\filleft}

\usepackage[htt]{hyphenat}


\begin{document}

%--------------------tom_apostol--------------------

    %----------caratula
	%\begin{tabular}{r l }
Universidad: & \textbf{Mayor de San Ándres.}\\
Asignatura: & \textbf{Álgebra Lineal I}\\
Ejercicio: & \textbf{Práctica 1.}\\ 
Alumno: & \textbf{PAREDES AGUILERA CHRISTIAN LIMBERT.}
\end{tabular}
\begin{flushleft}
\begin{tikzpicture}
\draw(0,1)--(16.5,1);
\end{tikzpicture}
\end{flushleft}




    %----------introducción
	%\chapter*{Introducción}
\setcounter{chapter}{1}
\setcounter{section}{2}

\pagenumbering{arabic} 

\section{El método de exhaución para el área de un segmento de parábola}

El método consiste simplemente en lo siguiente: se divide la figura en un cierto número de bandas y se obtienen dos aproximaciones de la región, una por defecto y otra por exceso, utilizando dos conjuntos de rectángulos.\\
Se subdivide la base en $n$ partes iguales, cada una de longitud $b/n$. Los puntos de subdivisión corresponden a los siguientes valores de $x$: $$0,\dfrac{b}{n},\dfrac{2b}{n},\dfrac{3b}{n},...,\dfrac{(n-1)b}{n},\dfrac{nb}{n}=b$$
La expresión general de un punto de la subdivisión es $x=\frac{kb}{n},$ donde $k$ toma los valores sucesivos $k=0,01,2,3,...,n.$ En cada punto $\frac{kb}{n}$ se construye el rectángulo exterior de altura $(kb/n)^2$. El área de este rectángulo es el producto de la base por la altura y es igual a:
$$\left(\dfrac{b}{n}\right)\left(\dfrac{kb}{n}\right)^2=\dfrac{b^3}{3}k^3$$
Si se designa por $S_n$ la suma de las áreas de todos os rectángulos exteriores, puesto que el área del rectángulo k-simo es $(b^3/n^3)k^3$ se tiene la formula. $$S_n=\dfrac{b^3}{n^3}(1^2+2^2+3^2+...+n^2) \qquad (1)$$
De forma análoga se obtiene la fórmula para la suma $s_n$ de todos los rectángulos interiores:
$$s_n=\dfrac{b^3}{n^3}[1^2+2^2+3^2+...+(n-1)^2] \qquad (2)$$
Luego se tiene la identidad $$1^2+2^2+3^2+...+n^2=\dfrac{n^3}{3}+\dfrac{n^2}{2}+\dfrac{6}{n} \qquad (3)$$
Como también $$1^2+2^2+...+(n-1)^2=\dfrac{n^2}{3}-\dfrac{n^2}{2}+\dfrac{n}{6} \qquad (4)$$
Las expresiones exactas dadas no son necesarias para el objeto que aquí se persigue, pero sirven para deducir fácilmente las dos desigualdades que interesan $$1^2+2^2+3^2+...+(n-1)^2<\dfrac{n^3}{3}<1^2+2^2+...+n^2$$
que son válidas para todo entero $n\geq 1$. Multiplicando ambas desigualdades por $b^3/n^3$ y haciendo uso de $(1)$ y $(2)$ se tiene: $$s_n<\dfrac{b^3}{3}<S_n$$
Probemos que $b^3/3$ es el único número que goza de esta propiedad, es decir, que si $A$ es un número que verifica las desigualdades $$s_n<A<S_n \qquad (7)$$
para cada entero positivo $n$, ha de ser necesariamente $A=b^3/3$. Por esta razón dedujo Arquímedes que el área del segmento parabólico es $b^3/3$.\\
Para probar que $A=b^3/3$ se utilizan una vez más las desigualdades $(5)$. Sumando $n^2$ a los dos miembros de la desigualdad de la izquierda en $(5)$ se obtiene $$1^2+2^2+3^2+...+n^2=\dfrac{n^3}{3}+n^2$$
Multiplicando por $b^3/3$ y utilizando $(1)$ se tiene $$S_n<\dfrac{b^3}{3}+\dfrac{b^3}{n} \qquad (8)$$
Análogamente, restando $n^2$ de los dos miembros de la desigualdad de la derecha en $(5)$ y multiplicando por $b^3/n^3$ se llega a la desigualdad:
$$\dfrac{b^3}{3}-\dfrac{b^3}{n}<s_n \qquad (9)$$
Por tanto,cada número $A$ que satisfaga $(7)$ ha de satisfacer también: $$\dfrac{b^3}{3}-\dfrac{b^3}{n}<A<\dfrac{b^3}{3}+\dfrac{b^3}{n} \qquad (10)$$ para cada entero $n\geq 1$. Ahora también, hay sólo tres posibilidad: $$A>\dfrac{b^3}{3}, \quad A<\dfrac{b^3}{3} \quad A=\dfrac{b^3}{3}$$
 Si se prueba que las dos primeras conducen a una contradicción habrá de ser $A=\frac{b^3}{3}$\\
 Supongamos que la desigualdad $A>b^3/3$ fuera cierta. De la segunda desigualdad en $(10)$ se obtiene $$A-\dfrac{b^3}{3}<\dfrac{b^3}{n} \qquad (11)$$ para cada entero $n\geq 1$. Puesto que $A-b^3/3$ es positivo, se puede dividir ambos miembros de $(11)$ por $A-b^3/3$ y multiplicando después por $n$ se obtiene la desigualdad $$n<\dfrac{b^3}{A-b^3/3}$$ para cada $n$. Pero esta desigualdad es evidentemente para $n>b^3/(A-b^3/3).$ Por tanto la desigualdad es una contradicción. De forma análoga se demuestra para $A<b^3/3$ de donde concluimos que $A=b^3/3$.\\

\newpage
\addcontentsline{toc}{chapter}{Introducción}
\setcounter{chapter}{3}
\setcounter{section}{1}
\section{Axiomas de cuerpo}
%axioma 1
\begin{axioma}[Propiedad conmutativa] $x+y=y+x, \; xy=yx$.\\
\end{axioma}

%axioma 2
\begin{axioma}[Propiedad asociativa] $x+(y+z)=(x+y)+z, \; x(yz)=(xy)z$. \\
\end{axioma}

%axioma 3
\begin{axioma}[Propiedad distributiva] $x(y+z)=xy+xz$. \\
\end{axioma}

%axioma 4
\begin{axioma}[Existencia de elementos neutros] Existen dos números reales distintos que se indican por $0$ y $1$ tales que para cada número real $x$ se tiene:
$0+x=x+0=x \;$ y $1\cdot x = x\cdot 1 = 1$. \\
\end{axioma}

%axioma 5
\begin{axioma}[Existencia de negativos] Para cada número real $x$ existe un número real y tal que $x+y=y+x=0$. \\
\end{axioma}

%axioma 6
\begin{axioma}[Existencia del recíproco] Para cada número real $x\neq 0$ existe un número real y tal que $xy=yx=1$. \\
\end{axioma}

%teorema 1.1
\begin{teo}[Ley de simplificación para la suma]
Si $a+b=a+c$ entonces $b=c$ (En particular esto prueba que el número 0 del axioma 4 es único)\\\\
Demostración.- \;
Dado a+b=a+c. En virtud de la existencia de negativos, se puede elegir y de manera que $y+a=0$, con lo cual $y+(a+b)=y+(a+c)$ y aplicando la propiedad asociativa tenemos $(y+a)+b=(y+a)+c$ entonces, $0+b=0+c$. En virtud de la existencia de elementos neutros, se tiene $b=c$.\\
por otro lado este teorema demuestra que existe un solo número real que tiene la propiedad del 0 en el axioma 4. En efecto, si $0$ y $0^{'}$ tuvieran ambos esta propiedad, entonces $0+0^{'}=0$ y $0+0=0$; por lo tanto, $0+0^{'}=0+0$ y por la ley de simplificación para la suma $0=0'$.
\end{teo}

%teorema 1.2
\begin{teo}[Posibilidad de la sustracción]
Dado $a$ y $b$ existe uno y sólo un $x$ tal que $a+x=b$. Este $x$ se designa por $b-a$. En particular $0-a$ se escribe simplemente $-a$ y se denomina el negativo de $a$.\\\\
Demostración.- \;
Dados $a$ y $b$ por el axioma 5 se tiene $y$ de manera que $a + y = 0$ ó $y=-a$, por hipótesis y teorema tenemos que $x=b-a$ sustituyendo $y$ tenemos $x=b+y$ y propiedad conmutativa $x=y+b$, entonces $a+x=a+(y+b)=(a+y)+b=0+b=b$ esto por sustitución, propiedad asociativa y propiedad de neutro, Por lo tanto hay por lo menos un $x$ tal que $a+x=b$. Pero en virtud del teorema 1.1, hay a lo sumo una. Luego hay una y sólo una $x$ en estas condiciones.
\end{teo}

%teorema 1.3
\begin{teo}
$b-a=b+(-a)$\\\\
Demostración.- \; Sea $x=b-a$ y sea $y=b+(-a)$. Se probará que $x=y$. por definición de $b-a$, $x+a=b$ y $y+a=\left[ b+(-a)\right]+a=b+\left[ (-a)+a \right]=b+0=b$, por lo tanto, $x+a=y+a$ y en virtud de teorema 1.1 $x=y$.
\end{teo}

%teorema 1.4
\begin{teo}
$-(-a)=a$\\\\
Demostración.- \;
Se tiene $a+(-a)=0$ por definición de $-a$ incluido en el teorema 1.1. Pero esta igualdad dice que $a$ es el opuesto de $-a$, es decir, que si $a+(-a)=0$ entonces $a=0-(-a)=a=-(-a)$.
\end{teo}

\section{Ejercicios}
\begin{enumerate}[\bfseries 1.]

%----------------------------------1--------------------------
\item Demostrar los teoremas del 1.5 al 1.15, utilizando los axiomas 1 al 6 y los teoremas I.1 al I.4\\\\

%teorema 1.5
\begin{teo}
$a(b-c)=ab-ac$\\\\
Demostración.- \; Sea $a(b-c)$ por teorema tenemos que $a\left[ b+(-c)\right]$  y por la propiedad distributiva $\left[ ab + a(-c) \right]$, y en virtud de anteriores teoremas nos queda $ ab - ac $. 
\end{teo}

%teorema 1.6
\begin{teo}
$0\cdot a = a\cdot 0 =0$\\\\
Demostración.- \;
Sea $0\cdot a$ por la propiedad conmutativa $a\cdot 0$, $a\cdot 0 + 0$ y $a\cdot 0 + \left[a+(-a) \right]$ y en virtud la propiedad asociativa y distributiva $a(0 + 1)+(-a)$ después $1(a)+(-a)$, luego por elemento neutro y existencia de negativos tenemos $0$, Así queda demostrado que cualquier número multiplicado por cero es cero.
\end{teo}

%teorema 1.7
\begin{teo}[Ley de simplificación para la multiplicación] Si $ab=ac$ y $a \neq 0$, entonces $b=c$. (En particular esto demuestra que el número 1 del axioma 4 es único)\\\\
Demostración.- \;
Sea $b$, $a\neq 0$, y por el existencia del recíproco tenemos $a\cdot a^{'}=1 $  luego,  $b=b\cdot 1=b\left[a(a^{'})\right]=(ab)(a^{'})=(ac)(a^{'})=c(a\cdot a^{'})=c\cdot 1=c$ por lo tanto queda demostrado la ley de simplificación.
\end{teo}

%teorema 1.8
\begin{teo}[Posibilidad de la división] Dados $a$ y $b$ con $a\neq 0$, existe uno y sólo un $x$ tal que $ax=b$. La $x$ se designa por $b/a$ ó $\displaystyle\frac{b}{a}$ y se denomina cociente de $b$ y $a$. En particular $1/a$ se escribe también $a^{-1}$ y se designa recíproco de $a$\\\\
Demostración .- \;
Sea $a$ y $b$ por axióma 6 se tiene un $y$ de manera que $a\cdot y = 1$ ó $y = a^{-1}$. Por hipótesis y teorema se tiene $x=b\cdot a^{-1}$, sustituyendo tenemos $x=y\cdot b$ entonces $ax=a(y\cdot b)=(a\cdot y)b=1\cdot b = b$  por lo tanto hay por lo menos un $x$ tal que $ax=b$ pero en virtud del teorema 1.7 hay por lo mucho uno, luego hay una y sólo una $x$ en estas condiciones.
\end{teo}

%teorema 1.9
\begin{teo}
Si $a\neq 0$, entonces $b/a=b\cdot a^{-1}$\\\\
Demostración.- \;
Sea $x =b/a$ y sea $y=b\cdot a^{-1}$ se probará que $x=y$, por definición de $b/a$, $ax=b$ y $ya=(b\cdot a^{-1})a=b(a^{-1}a)=b\cdot 1 = 1$, entonces $ya=xa$ y por la ley de simplificación para la multiplicación $y=x$. 
\end{teo}

%teorema 1.10
\begin{teo}
Si $a\neq 0$, entonces $(a^{-1})^{-1}=a$\\\\
Demostración.- \;
Si $a\neq 0$ entonces  $(a^{-1})^{-1} = 1\cdot (a^{-1})^{-1} = \displaystyle\frac{1}{a^{-1}}=a$ esto por axioma de neutro, definición de $a^{-1}$ y teorema 1.9, así concluimos que $(a^{-1})^{-1}=a$.
\end{teo}

%teorema 1.11
\begin{teo}
Si $ab = 0$, entonces ó $a=0$ ó $b=0$\\\\
Demostración.- \;
Veamos dos casos, cuando $x\neq 0$ y cuando $x=0$\\
Si $x\neq 0$ y  $ab = 0$ entonces  $b=b\cdot 1 = b (a\cdot a^{-1}) = (ab)a^{-1}=0a^{-1}=0$, ahora si $a=0$ y virtud del teorema 1.6 nos queda demostrado que la multiplicación de dos números cualesquiera es igual a cero si $a=0$ ó $b=0$. 
\end{teo}

%teorema 1.12
\begin{teo}
$(-a)b=-(ab) \; y \; (-a)(-b) = ab$\\\\
Demostración.- \;
Empecemos demostrando la primera proposición, Por la ley de simplificación para la suma podemos escribir como $(-a)b+ab=0$ entonces por la propiedad distributiva $b\left[ (-a)+a \right]$ por lo tanto $b\cdot 0$, luego por el teorema 1.6 queda demostrado la primera proposición.\\
Para demostrar la segunda proposición acudimos a la primera proposición, $(-a)(-b)=-\left[ a(-b)\right]$ y luego, \\ $-\left[ a(-b)+b+(-b)\right]=-\left[ (-b)(a+1)+b\right]=-\left[ (-b)(a+1)-1(-b)\right]=-\left[( -b)(a+1-1)\\\right]=-\left[ (-b)a\right]=-\left[ -(ab)\right]$ y en virtud  del el teorema 1.4 $(-a)(-b)=ab$ así queda demostrado la proposición. 
\end{teo}

%teorema 1.13
\begin{teo}
$\left(a/b \right) + \left(c/d \right) = \left( ad+bc  \right) / \left( bd \right) $ si $b\neq 0$ y $d\neq 0$.\\\\
Demostración.- \;
Si $\left(a/b \right) + \left(c/d \right)$ entonces por definición de $a/b$, $a\cdot b^{-1}+c\cdot d^{-1}=(a\cdot b^{-1})\cdot 1+(c\cdot d^{-1})\cdot 1=(a\cdot b^{-1})\cdot 1+(c\cdot d^{-1})\cdot 1=(a\cdot b^{-1})\cdot dd^{-1}+(c\cdot d^{-1})\cdot bb^{-1}$ por las propiedades asociativa  conmutativa y distributiva, $(b^{-1}d^{-1})(ad)+(b^{-1}d^{-1})(cb)=(b^{-1}d^{-1})(ad+cb)$, por lo tanto $(ad+bc)/bd$ esto por definición.
\end{teo}

%teorema 1.14
\begin{teo}
$(a/b)(c/d)=(ac)/(bd)$ si $b\neq 0$ y $d\neq 0$\\\\
Demostración.- \;
Por definición, $(ab^{-1})(cd^{-1})$, propiedades conmutativa y asociativa $(ac)(b^{-1}d^{-1})$, y por definición queda demostrado la proposición.
\end{teo}

%corolario 1.1
\begin{cor}
Si $c\neq 0$ y $d\neq 0$ entonces $(cd^{-1})=c^{-1}d$\\\\
Demostración.- \;
Por definición de $a^{-1}$ tenemos que $(cd^{-1})^{-1}=\displaystyle\frac{1}{cd^{-1}}$, por el teorema de posibilidad de la división $1=(c^{-1}d)(cd^{-1})$ y en virtud de los axiomas de conmutatividad y asociatividad $1=(c^{1}c)(dd^{-1})$, luego $1=1$. Quedando demostrado el corolario.
\end{cor}

%teorema 1.15 
\begin{teo}
$(a/b)/(c/d)=(ad)/(bc)$ si $b\neq 0$,  $c\neq 0$ y $d\neq 0$\\\\
Demostración.- \;
Sea $(a/b)/(c/d)$ entonces por definición $(ab^{-1})(cd^{-1})^{-1}$, en virtud del corolario 1 se tiene que $(ab^{-1})(c^{-1}d)$, y luego por axioma conmutativa y asociativa $(ad)(c^{-1}b^{-1})$, así por definición concluimos que $(ad)/(cd)$.
\end{teo}

%-------------------------------2--------------------------------------
\item $-0=0$\\\\
Demostración.- \;
Sabemos que por el axioma 5 $a+(-a)=0$, $- \left[ a+(-a) \right] = 0$ y $-a+-(-a)=0$ en virtud de teorema 1.12 y propiedad conmutativa $a+(-a)=0$, por lo tanto $0=0$.\\\\ 


%---------------------------------3-------------------------------
\item 
$1^{-1}=1$\\\\
Demostración.- \;
Por la existencia de elementos nuestros tenemos $1^{-1}\cdot 1$ y por axioma de existencia de reciproco $1=1$\\\\

%-------------------------------4-----------------------------
\item El cero no tiene reciproco\\\\
Demostración.- \;
Supongamos que el cero tiene reciproco es decir $0\cdot 0^{-1}=1$ pero por el teorema 1.6 se tiene que $0\cdot 0^{-1}=0$ y $0=1$ esto no es verdad, por lo tanto el cero no tiene reciproco.\\\\

%-----------------------------------5---------------------------------
\item $-(a+b)=-a-b$\\\\
Demostración.- \;
Por existencia de reciproco $-\left[1(a+b)\right]$ y teorema 1.12 $(-1)(a+b)$ luego por la propiedad distributiva $\left[ (-1)b \right] + \left[ (-1)b \right]$, una vez mas por el teorema 1.12 $-(1a)+ \left[-(1b)\right]$, en virtud del axioma 4 $-a+(-b)$, y teorema 1.3, $-a-b$ \\\\ 

%------------------------------------6----------------------------
\item $-(-a-b)=a+b$\\\\
Demostración\\\\
Si $-(-a-b)$ entonces por axioma $-\left[1(-a-b)\right]$, luego $(-1)(-a-b)=(-1)(-a)-\left[(-1)b\right])= (1\cdot a)-\left[-(1\cdot b)\right]$ y por axioma $a - \left[ - (b)\right]$, así por teorema $a+b$.\\\\

%------------------------------------7--------------------------
\item $(a-b)+(b-c)=a-c$\\\\
Demostración.- \;
Por definición tenemos $\left[ a+(-b) \right]+\left[ b+(-c) \right]$, y axiomas de asociatividad y conmutatividad $\left[ a+(-c) \right]+\left[ b+(-b) \right]$, luego por existencia de negativos  $\left[ a+(-c) \right] + 0$,, así $a+(-c)$ y $a-c$. \\\\

%-----------------------------------8--------------------------------
\item Si $a\neq 0$ y $b\neq 0$, entonces $(ab)^{-1} = a^{-1} b^{-1}$\\\\
Demostración.- \; Por hipótesis $\dfrac{1}{a} \dfrac{1}{b}$ luego $\dfrac{1}{ab}$ por lo tanto $(ab)^{-1}$\\\\

%------------------------------------9--------------------------------
\item $-(a/b)=(-a/b)=a/(-b)$ si $b\neq 0$\\\\
Demostración.- \;
Primero demostremos que $-(a/b)=(-a/b)$, Sea $b\neq 0$, en virtud de definición de la división y teorema 1.12 no queda que $-(a/b)=(-a)\cdot b^{-1}=-a/b$.\\
Ahora demostramos que $-(a/b)=(-a/b)=a/(-b)$, sea $b\neq0$, luego $-(b^{-1}\cdot a)=\left[-(b^-1)\right]\cdot a = a/-b$. \\\\

%------------------------------------10---------------------------------
\item $(a/b)-(c/d)=(ad-bc)/(bd)$ si $b\neq 0$ y $d\neq 0$\\\\
Demostración.- \;
Sea $b\neq 0$ y $d\neq 0$ y por definición $ab^{-1}-cd^{-1}$, luego por axiomas $(ab^{-1})(d\cdot d^{-1})-(cd^{-1})(b\cdot b^{-1})$, y en virtud del teorema 1.5 y propiedad asociativa $b^{-1}\cdot d^{-1}(ad-bc)$ y $(ad-bc)/bd$\\\\
\end{enumerate}

\section{Axiomas de orden}
\begin{axioma}Si $x$ e $y$ pertenecen a $\mathbb{R}^+$, lo mismo ocurre a $x+y$ y $xy$\\
\end{axioma}
\begin{axioma}
Para todo real $x\neq 0$, ó $x \in \mathbb{R}^+$ ó $-x \in \mathbb{R}^+$, pero no ambos.\\
\end{axioma}
\begin{axioma}
$0 \not\subset \mathbb{R}^+$\\
\end{axioma}

\begin{def.}
$x<y $ significa que $y-x$ es positivo. \\
\end{def.}
\begin{def.}
$y>x$ significa que $x<y$\\
\end{def.}
\begin{def.}
$x \geq y$ significa que ó $x<y$ ó $x=y$\\
\end{def.}
\begin{def.}
$y \leq x$ significa que $x \leq y$ 
\end{def.}

\section{Ejercicios}
\begin{enumerate}[\bfseries  1.]
%-------------------------1----------------------------
\item Demostrar los teoremas 1.22 al 1.25 utilizando los teoremas anteriores y los axiomas del 1 al 9
%teorema 1.16
\begin{teo}[Propiedad de Tricotomía]
Para $a$ y $b$ números reales cualesquiera se verifica se verifica una y sólo una de las tres relaciones $a<b$, $b<a$, $a=b$\\\\
demostración.- \;
Sea $x=b-a$. Si $x=0$, entonces $x=a-b=b-a$, por axioma 9, $0\notin \mathbb{R}^+$ es decir:
$$a<b, \; \; b-a \in \mathbb{R}^+$$
$$b<a, \; \;  a-b \in \mathbb{R}^+$$
pero como $a-b=b-a=0$ entonces no ser $a<b$ ni $b<a$\\
Si $x0\neq 0,$ el axioma 8 afirma que ó $x>0$ ó $ x<0$, pero no ambos, por consiguiente, ó es $a<b$ ó es $b<a$, pero no ambos. Pro tanto se verifica una y sólo una de las tres relaciones $a=b,$ $a<b,$ $b<a$.\\\\
\end{teo}

%teorema 1.17
\begin{teo}[Propiedad Transitiva]
Si $a<b$ y $b<c$, es $a<c$\\\\
Demostración.- \;
Si $a<b$ y $b<c$, entonces por definición $b-a>0$ y $c-b>0$. En virtud de axioma  $(b-a)+(c-b)>0$, es decir, $c-a>0,$ y por lo tanto $a<c$.\\\\
\end{teo}

%teorema 1.18
\begin{teo}
Si $a<b$ es $a+c<b+c$\\\\
Demostración.- \;
Sea $x=a+c$
\end{teo}

%teorema 1.19
\begin{teo}
Si $a<b$ y $c>0$ es $ac<bc$\\\\
Demostración.- \;
Si $a<b$ por definición $b-a>0$, dado que  $c>0$ y por el axioma  $(b-a)c>0$ y $bc-ac>0$, por lo tanto $ac<bc$\\\\
\end{teo}

%teorema 1.20
\begin{teo}
Si $a\neq0$ es $a^2>0$\\\\
Demostremos por casos..- \;
Si $a>0$, entonces por axioma   \; $a\cdot a >0$ \; y \; $a^2>0$. Si $a<0$, entonces por axioma   \; $(-a)(-a)>0$ \; y  \; $a^2>0$\\\\
\end{teo}

%teorema 1.21
\begin{teo}
1>0\\\\
Demostración.- \;
Por el anterior teorema, si $1>0$ ó $1<0$ entonces $1^2>0$, y $1^2=1$, por lo tanto que $1>0$\\\\
\end{teo}

%teorema 1.22
\begin{teo}
Si $a<b$ y $c<0$, es $ac>bc$\\\\
Demostración.- \;
Si $c<0$, por definición $-c>0$, en virtud del axioma  $-c(b-a)>0$, y $ac-cb>0$, por lo tanto $ab<ac=ac>bc$\\\\
\end{teo}

%teorema 1.23
\begin{teo}
Si $a<b$, es $-a>-b$. En particular si $a<0$, es $-a>0$\\\\
Demostración.- \;
Si $1>0$, por la existencia de negativos $-1<0$ y por teorema  tenemos que $-1a>-1b$ por lo tanto $-a>-b$ \\\\
\end{teo}

%teorema 1.24
\begin{teo}
Si $ab>0$ entonces $a$ y $b$ son o ambos positivos o ambos negativos\\\\
Demostración.- \;
Sea $a>0$ y $b>0$, por axioma  \;  $ab>0$, y sea $a<0$ y $b<0$, por definición $-a>0$ y $-b>0$, por lo tanto $(-a)(-b)>0$ y por teorema 1.12 \; $ab>0$.\\\\
\end{teo}

%teorema 1.25
\begin{teo}
Si $a<c$ y $b<d,$ entonces $a+b<c+d$\\\\
Demostración.- \;
Si $a<c$ \; y \; $b<d$ por definición $c-a>0$ \; y \; $d-b>0$, en virtud del axioma 6:
$$(c-a)+(d-b)>0 \Rightarrow c-a+d-b>0 \Rightarrow (c+d)-(a+b)>0$$ 
por lo tanto $a+b<c+d$.\\\\ 
\end{teo}

%---------------------------------2---------------------------
\item No existe ningún número real tal que $x^2+1=0$\\\\
Demostración.- \; Sea $Y=x^2+1=0$ de acuerdo con la propiedad de tricotomía:
\begin{itemize}
\item Si $x>0$ entonces por teorema 2.5 \; $x^2>0$ y por axioma 7 \; $x^2+1>0$ esto es $Y > 0$ y no satisface $Y=0$ para $x>0$.
\item Si $x=0$ entonces $x^2=0$ y $x^2+1=1$ esto es $Y=1$ pero no satisface a $Y=1$ para $x=0$.
\item Si $x<0$ entonces $-x>0$ y $x^2+1>0$, esto es $Y=0$ pero tampoco satisface a $y=0$ para $x<0$.\\\\ 
\end{itemize}

%------------------------------3--------------------------
\item La suma de dos números negativos es un número negativo.\\\\
Demostración.- \;   Si $a<0$ y $b<0$ entonces $-a>0$ y $-b>0$ por axioma 7 \; $(-a)+(-b)>0$ y en virtud del teorema 1.19 \; $-(a+b)>0$ es decir $a+b<0$\\

%-------------------------------4--------------------------
\item Si $a>0$, también $1/a>0;$ Si $a<0$ entonces $1/a<0$\\\\
Demostración.- \;
\begin{itemize}
\item Si $a>0$ entonces $(2a)^{-1}\cdot a > 0 \cdot (2a)^{-1}$ por lo tanto $1/a>0$
\item Si $a<0$ entonces $-a>0$ y $(-2a)^{-1}\cdot (-a)>0\cdot (-2a)^{-1}$ por lo tanto $1/a>0$ y $-1/a<0$\\\\
\end{itemize}

%-------------------------------5--------------------------
\item 
Si $0<a<b,$ entonces, $0<b^{-1}<a^{-1}$\\\\
Demostración.- \; Si $b>0$ entonces por el teorema anterior  $b^{-1}>0$ ó $0<b^{-1}$.\\
Si $a>0$ entonces $a^{-1}>0$, dado que $a<b$ y por teorema  \; $a\cdot a^{-1}< a^{-1}b$ así $1<a^{-1}b$, luego $b{-1}<a^{-1}\cdot bb^{-1}$, por lo tanto $b^{-1}<a^{-1}$. Y por la propiedad transitiva queda demostrado que $0<b^{-1}<a^{-1}$.\\\\

%--------------------------6----------------------------
\item Si $a \leq b $ y $b \leq c$ es $a\leq c$\\\\
Demostración.- \; Si $a<b$ ó $a=b$ y $b<c$ ó $b=c$ demostremos por casos: Si $a<b$ y $b<c$ por la propiedad transitiva $a<c$, después si $a<b$ y $b=c$ entonces $a<c$, luego si $a=b$ y $b<c$ entonces $a<c$,  por último si $a=b$ y $b=c$ entonce $a=c$, por lo tanto $a\leq c$\\\\

%corolario 
\begin{cor}
Si $c\leq b$ y $b \leq c$ entonces $c=b$\\\\
Demostración .- \; Si $b-c>0$ y $c-b>0$ entonces $(b-c)+(c-b)>0$ y $0<0$ es Falso, entonces queda que $c=b$ (Usted puede comprobar para cada uno de los casos que se suscita parecido al teorema anterior.)\\\\
\end{cor}

%---------------------------7----------------------------
\item Si $a\leq b$ y $b \leq c$ y $a=c$ entonces $b=c$\\\\
Demostración.- \; Si $a\leq b$\;  y \; $a=c$ entonces $c\leq b$. Sea $c\leq b$ \; y \; $b\leq c$ y por corolario anterior \; $b=c$.\\\\

%----------------------------8-------------------------
\item Para números reales $a$ y $b$ cualquiera, se tiene $a^2+b^2\leq 0$. Si $ab\geq 0$, entonces es $a^2+b^2>0$.\\\\
Demostración.- \; Si $ab>0$ por teorema  ($a>0$ y $b>0$ ) ó ($a<0$ y $b<0$) luego por teorema  $a^2>0$ y $b^2>0$ por lo tanto por axioma 7 y ley de tricotomía $a^2+b^2>0$.\\\\ 

%-------------------------9------------------------------
\item No existe ningún número real $a$ tal que $x\leq a$ para todo real $x$\\\\
Demostración.- \; Supongamos que existe un número real $" a "$ tal que $y \leq a$. Sea $n\in \mathbb{R}$ \; y \; $x=y+n$ entonces por teorema 1.18 \; $y+n \leq a+n$ \; y \; $x\leq a+n$ esto contradice que existe un número real $a$ tal que $y\leq a$, por lo tanto no existe ningún número real tal que para todo x, $x\leq a$.\\\\

%--------------------------10-----------------------------
\item Si $x$ tiene la propiedad que $0\leq x < h$ para cada número real positivo $h$, entonces $x=0$\\\\
Demostración .- \; Por el teorema anterior ni $0<x$ ni $x<h$ satisfacen la proposición por lo tanto queda $x=0$\\\\

%lema 1.1
\begin{lema}
Para $b\geq 0$ $a^2>b$ $\Rightarrow$ $a>\sqrt{b}$ ó $a<-\sqrt{b}$\\\\
Demostración.- \;
Por hipótesis $a^2>b$ y $a^2-b>0$, luego \; $(a-\sqrt{b})(a+\sqrt{b})>0$, y por teorema  \; $a-\sqrt{b}>0$ \; y \; $a+\sqrt{b}<0$ \; ó \; $a-\sqrt{b}<0$ \; y \; $a+\sqrt{b}<0$, por lo tanto $a-\sqrt{b}<0$ \; ó \; $a<-\sqrt{b}$\\\\
\end{lema}
\end{enumerate}

\section{Números enteros y racionales}
%definición de conjunto inductivo
\begin{def.}[Definción de conjunto inductivo]
Un conjunto de números reales se denomina conjunto inductivo si tiene las propiedades siguientes:
\begin{enumerate}[\bfseries a)]
\item El número $1$ pertenece al conjunto.
\item Para todo $x$ en el conjunto, el número $x+1$ pertenece también al conjunto.
\end{enumerate}
\end{def.}
\begin{def.}[Definición de enteros positivos]
Un número real se llama entero positivo si pertenece a todo conjunto inductivo.
\end{def.}

\setcounter{section}{7}
\section{Cota superior de un conjunto, elemento máximo, extremo superior}
\begin{def.}[definición de extremo superior]
Un número $B$ se denomina extremo superior de un conjunto no vacío $S$ si $B$ tiene las dos propiedades siguientes:
\begin{enumerate}[\bfseries a)]
\item $B$ es una cota superior de $S$.
\item Ningún número menor que $B$ es cota superior para $S$.
\end{enumerate}
\end{def.}

%teorema 1.26
\begin{teo}
Dos números distintos no pueden ser extremos superiores para el mismo conjunto.\\\\
Demostración.- \; Sean $B$ y $C$ dos extremos superiores para un conjunto $S$. La propiedad $b)$ de la definición 3.1 implica que $C\geq B$ puesto que $B$ es extremo superior; análogamente, $B \geq C$ ya que $C$ es extremo superior. Luego $B = C$ \\\\
\end{teo}

\section{Axioma del extremo superior (axioma de completitud)}

%axioma 10
\begin{axioma}
Todo conjunto no vacío $S$ de números reales acotado superiormente posee extremo superior; esto es, existe un número real $B$ tal que $B=sup S$
\end{axioma}

%definición de extremo inferior
\begin{def.}[Definición de extremo inferior (Ínfimo)]
Un número $L$ se llama extremo inferior (o ínfimo) de $S$ si:
\begin{enumerate}[\bfseries a)]
\item $L$ es una cota inferior para $S$,
$$L\leq x, \, \, \forall x \in S$$
\item Ningún número mayor que $L$ es cota inferior para $S$.
$$Si \; t\leq x, \, \; \forall x \in S, \, \, entonces \, t\leq L $$
\end{enumerate}
El extremo inferior de $S$, cuando existe, es único y se designa por $infS$. Si $S$ posee mínimo, entonces $minS=infS$\\
\end{def.}

%teorema 1.27
\begin{teo}
Todo conjunto no vacío $S$ acotado inferiormente posee extremo inferior  o ínfimo; esto es, existe un número real $L$ tal que $L=infS$.\\\\
Demostración.- \; Sea $-S$ el conjunto de los números opuestos de los de $S$. Entonces $-S$ es no vacío y acotado superiormente. El axioma 10 nos dice que existe un número $B$ que es extremo superior de $-S$. Es fácil ver que $-B=infS$.\\\\
\end{teo}

\section{La propiedad Arquimediana del sistema de los números reales}
%teorema 1.28
\begin{teo}
El conjunto $P$ de los enteros positivos 1,2,3,... no está acotado superiormente.\\\\
Demostración.- \; Supóngase $P$ acotado superiormente. Demostraremos que esto nos conduce a una contradicción. Puesto que P no es vacío, el axioma 10 nos dice que $P$ tiene supremo, sea este $b$. El número $b-1$, siendo menor que $b$, no puede ser cota superior de $P$. Por b) de la definición 3.1 existe $n>b-1$ es decir $b-1 \in P, \; b-1<b \; \; \exists n \in P:\; b-1<n$. Para este $n$ tenemos $n+1>b$. Puesto que $n+1$ pertenece a $P$, esto contradice el que $b$ sea una cota superior para $P$. \\\\
\end{teo}

%teorema 1.29
\begin{teo}
Para cada real $x$ existe un entero positivo $n$ tal que $n>x$\\\\
Demostración.- \; Si no fuera así, $x$ sería una cota superior de $P$, en contradicción con el teorema 3.3.\\\\
\end{teo}

%teorema 1.30
\begin{teo}
Si $x>0$ e $y$ es un número real arbitrario, existe un entero positivo $n$ tal que $nx>y$\\\\
Demostración.- \; Aplicar teorema 3.4 cambiando $x$ por $y/x$.\\\\
\end{teo}

%teorema 1.31
\begin{teo}
Si tres números reales $a$, $x$, e $y$ satisfacen las desigualdades $a\leq x \leq a+\displaystyle\frac{y}{n}$ para todo entero $n \geq 1$, entonces $x=a$\\\\
Demostración.- \; Si $x>a$, el teorema 3.5 nos menciona que existe un entero positivo que satisface $n(x-a)>y$, en contradicción de la hipótesis, luego $x>a$ no satisface para todo número real $x$ y $a$, con lo que deberá ser $x=a$.\\\\
\end{teo} 
\setcounter{section}{11}
\section*{Propiedades fundamentales del extremo superior ó supremo}
%teorema 1.32
\begin{teo}
Sea $h$ un número positivo dado y $S$ un conjunto de números reales.
\begin{enumerate}[\bfseries a)]
\item Si $S$ tiene extremo superior o supremo, para un cierto $x$ de $S$ se tiene 
$$x>supS-h$$
Demostración.- \; Si es $x\leq supS -h$ para todo $x$ de $S$, entonces $supS-h$ sería una cota superior de $S$ menor que su supremo. Por consiguiente debe ser $x>supS-h$ por lo menos para un $x$ de $S$.
\item Si $S$ tiene extremo inferior o ínfimo, para un cierto $x$ de $S$ se tiene 
$$x<infS+h$$
Demostración.- \; Si es $x \geq supS+h$ para todo $x$ de $S$, entonces $supS+h$ sería una cota inferior de $S$ mayor que su ínfimo. Por consiguiente debe ser $x<supS+h$ por lo menos para un $x$ de $S$.
\end{enumerate}
\end{teo}

%teorema 1.33
\begin{teo}[Propiedad aditiva]
Dados dos subconjuntos no vacíos $A$ y $B$ de $\mathbb{R}$, sea $C$ el conjunto
$$C=\lbrace a+b / a\in A, \; b \in B   \rbrace$$
\begin{enumerate}[\bfseries a)]
\item Si $A$ y $B$ poseen supremo, entonces $C$ tiene supremo, y 
$$supC= supA + supB$$
Demostración.- \; Supongamos que $A$ y $B$ tengan supremo. Si $c \in C$, entonces $c=a+b$, donde $a\in A$ y $b\in B.$ Por consiguiente $c \leq supA +supB$; de modo que $supA + supB$ es una cota superior de $C$. esto demuestra por el axioma 10 que $C$ tiene supremo y que 
$$supC \leq supA + supB$$
Sea ahora $n$ un entero positivo cualquiera. Según el teorema 3.7 $\left( con \; h=1/n \right)$ existen un $a$ en $A$ y un $b$ en $B$ tales que:
$$a>supA - \displaystyle\frac{1}{n} \; y \; b>supB -\frac{1}{n}$$
Sumando estas desigualdades, se obtiene 
$$a+b>supA +supB -\displaystyle\frac{2}{n}, \; \; ó  \; \; supA + supB < a+b+\frac{2}{n} \leq supC +\frac{2}{n}$$
puesto que $a+b \leq supC.$ Por consiguiente hemos demostrado que
$$supC \leq supA + supB < supC + \displaystyle\frac{2}{n}$$
para todo entero $n \geq 1.$ En virtud del teorema 3.6, debe ser $supC = supA+supB.$ Esto demuestra $a)$\\
\item Si $A$ y $B$ tienen ínfimo, entonces $C$ tiene ínfimo, e
$$infC = infA+ infB$$ 
Demostración.- \; Supongamos que $A$ y $B$ tengan ínfimo. Si $c \in C$, entonces $c=a+b$, donde $a\in A$ y $b\in B.$ Por consiguiente $c \geq infA +infB$; de modo que $infA + infB$ es una cota inferior de $C$. esto demuestra por el axioma 10 que $C$ tiene ínfimo y que 
$$infC \geq infA + infB$$
Sea ahora $n$ un entero positivo cualquiera. Según el teorema 3.7 $\left( con \; h=1/n \right)$ existen un $a$ en $A$ y un $b$ en $B$ tales que:
$$a<infA + \displaystyle\frac{1}{n} \;\;  y \; \; b<infB + \frac{1}{n}$$
Sumando estas desigualdades, se obtiene 
$$a+b<infA +infB +\displaystyle\frac{2}{n}, \; \; ó \; \;  infA+infB \leq infC \leq a+b < infA+infB + \displaystyle\frac{2}{n} $$
puesto que $a+b \geq infC$ Por consiguiente hemos demostrado que
$$infA+ infB \leq infC <infA + infB + \displaystyle\frac{2}{n}$$
para todo entero $n \geq 1.$ En virtud del teorema 3.6, debe ser $supA+supB = infC.$ Esto demuestra $b)$\\\\
\end{enumerate}
\end{teo}

%teorema 1.34
\begin{teo}
Dados dos subconjuntos no vacíos $S$ y $T$ de $\mathbb{R}$ tales que $$s\leq t$$ para todo $s$ en $S$ y todo $t$ en $T$. Entonces $S$ tiene supremo, $T$ ínfimo, y se verifica $$supS\leq infT$$\\\\
Demostración.- \; Cada $t$ de $T$ es corta superior para $S$. Por consiguiente $S$ tiene supremo que satisface la desigualdad $supS\leq T$ para todo $t$ de $S$. Luego $supS$ es una cota inferior de $T$, con lo cual $T$ tiene ínfimo que no puede ser menor que $supS$. Dicho de otro modo, se tiene $supS\leq infT$, como se afirmó.\\\\
\end{teo}

\section{Ejercicios}
\begin{enumerate}[\bfseries  1.]
%-----------------------1-----------------------------
\item Si $x$ e $y$ son números reales cualesquiera, $x<y$, demostrar que existe por lo menos un número real $z$ tal que $x<z<y$\\\\
Demostración.- \; Sea $S$ un conjunto no vacío de $\mathbb{R}$, por axioma 10 se tiene un supremo llamemosle $z$, por definición $x\leq z$ para todo $x\in S$, ahora si $y\in \mathbb{R}$ \; que cumple $x\leq y$, para todo $x\in S$, entonces $z\leq y$, por lo tanto $x\leq z \leq y$ esto  nos muestra que existe por lo menos un número real que cumple la condición $x<z<y$. \\\\

%----------------------------2------------------------------
\item Si x es un número real arbitrario, probar que existen enteros $m$ y $n$ tales que $m<x<n$\\\\
Demostración.- \;  Sea $n\in \mathbb{Z}^+$ en virtud del axioma 5 se verifica $n+m=0$, donde $m$ es el opuesto de $n$, esto nos dice que $m<n$ y por teorema anterior se tiene $m<x<n$.\\\\

%-----------------------------3---------------------
\item Si $x>0$, demuestre que existe un entero positivo $n$ tal que $1/n<x$\\\\
Demostración.- \; Sea $y=1$ entonces por teorema 3.5 \; $nx>1$, por lo tanto $1/n<x$ \\\\

%-----------------------------4------------------------------
\item Si $x$ es un número real arbitrario, demostrar que existe un entero $n$ único que verifica las desigualdades $n\leq x < n+1$. Este $n$ se denomina la parte entera de $x$, se delega por $[x]$. Por ejemplo, $[5]=5,\; \left[ \frac{5}{2}\right] =2, \; \left[ -\frac{8}{2} \right]=-3$\\\\
Demostración.-\; Primero probemos la existencia de $n$,
\begin{itemize}
\item  Sea $1\leq a $  \; y  \; $\; S=\lbrace m \in \mathbb{N}/\; m \leq a \rbrace$\\ 
Vemos que $S$ es no vacío pues contiene a 1, y \; $a$ \; es un cota superior de $S$, luego por axioma del supremo, existe un número $s=supS$, entonces por teorema 3.7 \; con $h=1$ resulta:
$$n>s-1 \; \; ó \; \; s<n+1, \; \; para \; algún n \; de \; S \; \; \; (1) $$
Como $z \in S$, se cumple $z\leq a$ y solo falta probar que $a<z+1$. En efecto, si fuese $z+1\leq a$, entonces $n+1 \in S$ y por la propiedad a), se tendría $n+1\leq s$, en contradicción con (1).   \\
Por tanto, el número entero positivo $n$ cumple con $n\leq a < n+1$
\item $0\leq a < 1$\\
En este caso, el entero $n=0$ cumple con la propiedad requerida.
\item $a<0$
Entonces $-a>0$ \; y por los dos casos anteriores, existe un entero u tal que $u\leq a< u+1$ de donde $-u-1<a\leq -u$.\\
Definiendo $n$ por  
\begin{equation}
n = \left\lbrace
\begin{array}{lcr}
-u-1 & si & a<-u\\
\textup{si } & x\leq 5 & a=-u
\end{array}        
\right.
\end{equation}
se prueba fácilmente que $z\leq a < z+1$
\end{itemize}
Luego demostramos la unicidad. Sea $w$ y $z$ dos números enteros tal que, $w\leq a < w + 1$ y $z \leq a < z + 1$ debemos probar que $w=z$. Si fuesen distintos, podemos suponer que $z<w$. Entonces $w-z\geq 1$, esto es $z+1\leq w$, y de $a<z+1\leq w \leq a$ resulta una contradicción ya que $a<a$ luego se cumple que $w=z$. \\\\ 

%---------------------------5------------------------------
\item Si $x$ es un número real arbitrario, $x<y$, probar que existe un entero único $n$ que satisface la desigualdad $n \geq x < n+1$\\\\
Demostración.- \; sabemos que para $x \in \mathbb{R}$ hay exactamente un $n \in \mathbb{Z}$ tal que $n \neq x < n+1$\\
Si $n=x$ entonces $x \neq n < x+1.$. Por otro lado si $n \neq x,$ entonces tenemos $n<x,$ así $n+1 < x+1$. Pero sabemos que $x<n+1$, por lo tanto, $$x<n+1<x+1 \rightarrow x \leq n+1 < x+1$$\\\\

%---------------------------6-------------------------------
\item Si $a$ y $b$ son números reales arbitrarios, $a<b$, probar que existe por lo menos un número racional $r$ tal que $a<r<b$ y deducir de ello que existen infinitos. Esta propiedad se expresa diciendo que el conjunto de los números racionales es denso en el sistema de los números reales.\\\\
Demostración.- \, Por la propiedad arquimediana, para el número $\displaystyle\frac{1}{b-a}$ existe un número natural $d$ tal que $\displaystyle\frac{1}{b-a}$, de donde 
$$db-da>1 \; \; ó \; \, da+1<da \; \, \, (1)$$
y también si $z$=parte entera de $da$
$$z\leq da < z+1 \, \, \, (2)$$
Sea $q=\displaystyle\frac{n}{d}$, con $n=z+1$. Entonces $q$ es un número racional y cumple $x<q<y$ pues:
$$a= d\displaystyle\frac{a}{d} < \frac{z+1}{d}<q=\frac{z+1}{d}\leq \frac{da+1}{d}<d\frac{b}{d}=b$$. y por ser $\displaystyle\frac{z+1}{d}$ deducimos que existen infinitos números racionales entre $a$ e $b$\\

%----------------------------7----------------------------------
\item Si $x$ es racional, $x\neq 0$, e $y$ es irracional, demostrar que $x+y$, $x-y$, $xy$, $x/y$, son todos irracionales.
\begin{itemize}
\item $x+y$, \, $x-y$\\
Supongamos que la suma nos da un racional, es decir $\displaystyle\frac{q}{p}+y=\frac{s}{t}\; para \; s,t\neq 0$, por lo tanto $y = \displaystyle\frac{qt+sp}{tp}$, así llegamos a una contradicción, en virtud del axioma 7 (la suma y multiplicación de dos racionales nos da otros racionales).\\
$x-y$ Se puede comprobar de similar manera a la anterior demostración.\\
\item $xy$, \, $x/y$, \; $y/x$\\
Supongamos que el producto nos da un número racional, por lo tanto $\frac{p}{q}\cdot y = \displaystyle\frac{t}{s}$ para $q,s \neq 0$ y $ \displaystyle y = \frac{sq}{pt}$ en contradicción con la hipótesis. De igual manera se comprueba que $x/y$ es irracional.\\\\
\end{itemize}

%-----------------------------8----------------------------
\item ¿La suma o el producto de dos números irracionales es siempre irracional?\\\\
Demostración.- \, No siempre se cumple la proposición, veamos dos contra ejemplos.\\
Sea $a$ un número irracional entonces por teorema anterior $1-a$ es irracional, así $a+(1-a)=1$, sabiendo que $1\in \mathbb{R}$. Por otro lado sabemos que $\displaystyle\frac{1}{a}$ es irracional, por lo tanto $1\in \mathbb{R}$.\\\\

%------------------------------9-----------------------------
\item Si $x$ e $y$ son números reales cualesquiera, $x<y$, demostrar que existe por lo menos un número irracional $z$ tal que $x<z<y$ y deducir que existen infinitos\\\\
Demostración.- \; Sea $0<x<y$ e $i$ un número irracional, por propiedad arquimediana  $y-x>\displaystyle\frac{i}{n}$ \; ó \; $\displaystyle x+ \frac{i}{n}<y$. \\\\
por teorema 3.15 \; se tiene que $\dfrac{i}{n}$ es irracional llamemosle $z$ por lo tanto  $x+z>x$, luego existe $x<z<y$. Y de $\dfrac{i}{n}$  deducimos que existen infinitos números irracionales que cumplen la condición.\\\\

%-------------------------------10----------------------------
\item Un entero $n$ se llama par si $n=2m$ para un cierto entero $m$, e impar si $n+1$ es par demostrar las afirmaciones siguientes:
\begin{enumerate}[\bfseries a)]
\item Un entero no puede ser a la vez par e impar.\\\\
Demostración.- \; Sean $2k$ y $2i+1$ dos enteros par e impar a la vez  entonces $2k=2i+1$ ó  $(k-i)=\dfrac{1}{2}$  lo cual no es cierto, ya que la resta de dos números pares siempre da par, por lo tanto es par o es impar pero no los dos al mismo tiempo.\\
\item Todo entero es par o es impar.\\\\
Demostración.- \; Por inciso a) \; $2k\neq 2k-1$ para $k\in \mathbb{Z}$, por la tricotomía ó $2k < 2k-1$ ó $2k > 2k-1$ lo cual se cumple pero no ambos a la vez.\\  
\item La suma o el producto de dos enteros pares es par. ¿ Qué se puede decir acerca de la suma o del producto de dos enteros impares ?\\\\
Demostración.- \; Sea $k\in \mathbb{R}$ entonces $2k+2k=4k=2(2k)$. Luego para el producto $2k\cdot 2k = 4k^2=2(2k^2)$\\
Por otra parte $(2k-1)+(2k-1)=4k-2=2(2k-1)$. No pasa lo mismo para el producto ya que  $(2k-1)(2k-1)=2k^2-4k+1=2(2k^2-2k)+1$\\\\
\item Si $n^2$ es par, también lo es $n$. Si $a^2=2b^2$, siendo $a$ y $b$ enteros, entonces $a$ y $b$ son ambos pares.\\\\
Demostración.- \;  Si $n$ es impar entonces $n^2$ es impar, reciprocamente hablando, entonces sea $n^2=(2k-1)^2$ para $k\in \mathbb{R}$, por lo tanto $2(2k^2+4k)-1$ es impar.\\
Por otro lado, sea $a=2k$, $b=2k-1$ '; y \; $k\in \mathbb{Z}$ entonces $(2k)^2=2(2k-1)^2$, por lo tanto $k=\dfrac{1}{2}$, esto contradice $k\in \mathbb{Z}$.\\
\item Todo número racional puede expresarse en la forma $a/b$, donde $a$ y $b$ enteros, uno de los cuales por lo menos es impar.\\\\
demostración.- \; Sea $r$ un número racional con $r=\dfrac{a}{b}$. Si $a$\; y \; $b$ son ambos pares, entonces tenemos $$a=2c \; y \; b=2d \,\; \Rightarrow \,\; \dfrac{a}{b}=\dfrac{2c}{2d}=\dfrac{c}{d},$$ con $c<a$ \; y \; $d<b$. ahora, si $c$ \; y \; d ambos son pares, repita el proceso. Esto dará una secuencia estrictamente decreciente de enteros positivos, por lo que el proceso debe terminar por el principio de buen orden. Por lo tanto debemos tener algunos enteros $r$ \; y \; $s$, no ambos con $n=\dfrac{a}{b}=\dfrac{r}{s}$.
\end{enumerate}

%-----------------------------11------------------------
\item Demostrar que no existe número racional cuyo cuadrado sea 2.\\\\
Demostración.- \; Utilizaremos el método de reducción al absurdo. Supongamos que n es impar, es decir, $n=2k+1\; k \in \mathbb{Z}$, ahora operando:
$$n^2=(2k+1)^2 \Rightarrow  n^2 = 4k^2 +4k + 1 \Rightarrow n^2=2(2k^2+2k)+1$$
Sabemos que $2k^2+2k$ es un número entero cualquiera, por lo tanto podemos realizar un cambio de variable, $2k^2+2k = k^{'}$, entonces:
$$n^2=2k^{'} +1$$
Se tiene una contradicción ya por teorema anterior se de dijo que $n^2$ es par, por lo tanto queda demostrado la proposición.  
Ahora si estamos con la facultad de demostrar que  $\sqrt{2}$ es irracional.\\
Supongamos que $\sqrt{2}$ es racional, es decir, existen números enteros tales que:
$$\displaystyle\frac{p}{q}=\sqrt{2}$$
Supongamos también que $p$ y $q$ no tienen divisor común mas que el 1. Se tiene:
$$p^2=2q^2$$
Esto nos muestra que $p^2$ es par y  por la previa demostración tenemos que $p$ es par. En otras palabras $p = 2k, \; \forall k \in \mathbb{Z}$, entonces:
$$(2k)^2 = 2q^2 \Rightarrow 4k^2 = 2q2 \Rightarrow 2k^2 = q^2 $$
Esto demuestra que $q^2$ es par y en consecuencia que $q$ es par. Así pues, son pares tanto $p$ como $q$ en contradicción con el hecho de que $p$ y $q$ no tienen divisores comunes. Esta contradicción completa la demostración.\\\\

%-----------------------------11--------------------------
\item La propiedad arquimediana del sistema de números reales se dedujo como consecuencia del axioma del supremo. Demostrar que el conjunto de los números racionales satisface la propiedad arquimediana pero no la del supremo. Esto demuestra que la propiedad arquimediana no implica el axioma del supremo.\\\\
Demostración.- \; Está claro que que el conjunto de los racionales satisface la propiedad arquimediana ya que si $x=\dfrac{p}{q}$ e $y=\dfrac{s}{t}$ para $q,\; t \neq 0$ entonces $\dfrac{p}{q}\cdot n>\dfrac{s}{t}$.\\
Por otra parte sea $S$ el conjunto de todos los racionales y supongase que esta acotado superiormente, por axioma 10 se tiene supremo, llamemosle $B$, entonces $x\leq B, \; \; x \in S$, luego existe $t\in \mathbb{R}$ tal que $B\leq t$, así por teorema 3.14 \; $B<x<t$,  esto contradice que $B$ sea supremo.\\\\ 

   
\section{Existencia de raíces cuadradas de los números reales no negativos}
\paragraph{Nota}
Los números negativos no pueden tener raíces cuadradas, pues si $x^2=a$, al ser $a$ un cuadrado ha de ser no negativo (en virtud del teorema 2.5). Además, si $a=0$, $x=0$ es la única raíz cuadrada (por el teorema 1.11). Supóngase, pues $a>0$. Si $x^2=a$ entonces $x\leq 0$ y $(-x)^2=a$, por lo tanto, $x$ y su opuesto son ambos raíces cuadradas. Pero a lo sumo tiene dos, porque si $x^2=a$ e $y^2=a$, entonces $x^2=y^2$ \; y \; $(x+y)(x-y)=0$, en virtud del teorema 1.11, \; ó $x=y$ ó $x=-y$. Por lo tanto, si $a$ tiene raíces cuadradas, tiene exactamente dos.
\begin{def.}
Si $a\geq 0$, su raíz cuadrada no negativa se indicará por $a^{1/2}$ o por $\sqrt{a}$. Si $a>0$, la raíz cuadrada negativa es $-a^{1/2}$ ó $-\sqrt{a}$
\end{def.}
%teorema 1.35
\begin{teo}
Cada número real no negativo $a$ tiene una raíz cuadrada no negativa única.\\\\
Demostración.- \; Si $a=0$, entonces $0$ es la única raíz cuadrada. Supóngase pues que $a>0$. Sea $S$ el conjunto de todos los números reales positivos $x$ tales que $x^2\leq a$. Puesto que $(1+a)^2>a$, el número $(a+1)$ es una cota superior de $S$. Pero, $S$ es no vacío, pues $a/(1+a)$ pertenece a $S$; en efecto $a^2\leq a(1+a)^2$ y por lo tanto $a^2/(1+a)^2\leq a$. En virtud del axioma 10, $S$ tiene un supremo que se designa por $b$. Nótese que $b\geq a/(1+a)$ y por lo tanto $b>0$. Existen sólo tres posibilidades: $b^2>a$, $b^2<a$, $b^2=a$.\\
Supóngase $b^2>a$ y sea $c=b-(b^2-a)/(2b)/(2b)=\dfrac{1}{2}(b+a/b)$. Entonces $a<c<b$ \; y \; $c^2=b^2-(b^2-a)+(b^2-a)^2/4b^2=a+(b^2-a)^2/(4b^2)>a$. Por lo tanto, $c^2>x^2$ para todo $x \in S$, es decir, $c>x$ para cada $x \in S$; luego $c$ es una cota superior de $S$, y puesto que $c<b$ se tiene una contradicción con el hecho de ser $b$ el extremo superior de $S$. Por tanto, la desigualdad $b^2>a$ es imposible.\\
Supóngase $b^2<a$. Puesto que $b>0$ se puede elegir un número positivo $c$ tal que $c<b$ y tal que $c<(a-b^2)7(3b). Se tiene entonecs$ $$(b+c)^2=b^2+c(2b+c)< b^2 +3bc < b^2 + (a-b^2)=a$$ es decir, $b+c$ pertenece a $S$. Como $b+c>b,$ esta desigualdad está en contradicción con que $b$ sea una cota superior de $S$. Por lo tanto, la desigualdad $b^2<a$ es imposible y sólo queda como posible $b^2=a$\\\\
\end{teo}
\end{enumerate}
 
\section{Raíces de orden superior. Potencias racionales}
El axioma del extremo superior se puede utilizar también para probar la existencia de raíces de orden superior. Por ejemplo, si $n$ es un entero positivo impar, para cada real $x$ existe un número real $y$, y uno sólo tal que $x^n=x$. Esta $y$ se denomina raíz n-sima de $x$ y se indica por:
\begin{def.}
$$y=x^{\frac{1}{n}} \; \; ó \; \; y=\sqrt[n]{x}$$
\end{def.}
Si $n$ es par, la situación es un poco distinta. En este caso, si $x$ es negativo, no existe un número real $y$ tal que $y^n = x$, puesto que $y^n\geq 0$ para cada número real $y$. Sin embargo, si $x$ es positivo, se puede probar que existe un número positivo y sólo uno tal que $y^n = x$. Este $y$ se denomina la raíz n-sima positiva de $x$ y se indica por los símbolos anteriormente mencionados. Puesto que $n$ es par, $(-y)^n = y^n$ y, por tanto, cada $x > 0$ tiene dos raíces n-simas reales, $y$ e $-y$. Sin embargo, los símbolos $x^{\frac{1}{n}}$ y $\sqrt[n]{x}$; se reservan para la raíz n-sima positiva.\\
\begin{def.}
Si $r$ es un número racional positivo, sea $r = m/n$, donde $m$ y $n$ son enteros positivos, se define como: $$x^r= x^{m/n }=(x^m)^{\frac{1}{n}},$$ es decir como raíz n-sima de $x^m$, siempre que ésta exista.\\ 
\end{def.}
\begin{def.}  
Si $x\neq 0$, se define $$x^{-r} = \dfrac{1}{x^r},$$ con tal que $x^r$ esté definida.
\end{def.}
 Partiendo de esas definiciones, es fácil comprobar que las leyes usuales de los exponentes son válidas para exponentes racionales: 
\begin{prop} Propiedades de potencia.\\
\begin{center}
\begin{enumerate}[\bfseries 1.]
\item $x^r \cdot x^s =  x^{r+s}$
\item $(x^r)^s=x^{rs}$
\item $(xy)^r=x^r \cdot y^r$
\item $\left( \dfrac{x}{y} \right)^r=\dfrac{x^r}{y^r}$
\end{enumerate}
\end{center}
\end{prop}

\setcounter{chapter}{4}
\setcounter{section}{2}
\section{El principio de la inducción matemática}
\paragraph{Método de demostración por inducción}Sea $A(n)$ una afirmación que contiene el entero $n$. Se puede concluir que $A(n)$ es verdadero para cada $n\geq n_1$ si es posible:
\begin{enumerate}[\bfseries a)]
\item Probar que $A(n_1)$ es cierta.
\item Probar, que supuesta $A(k)$ verdadera, siendo $k$ un entero arbitrario pero fijado $\geq n_1$, que $A(k+1)$ es verdadera.\\
\end{enumerate}
En la práctica, $n_1$ es generalmente igual a $1$.

%teorema 1.36
\begin{teo}[Principio de inducción matemática]
Sea $S$ un conjunto de enteros positivos que tienen las dos propiedades siguientes:
\begin{enumerate}[\bfseries a)]
\item El número 1 pertenece al conjunto $S$.
\item Si un entero $k$ pertenece al conjunto $S$, también $k+1$ pertenece a $S$.
\end{enumerate}
Entonces todo entero positivo pertenece al conjunto $S$.\\\\
Demostración.- \; Las propiedades $a)$ y $b)$ nos dicen que $S$ es un conjunto inductivo. Por consiguiente $S$ tiene cualquier entero positivo.\\\\ 
\end{teo}

%teorema 1.37
\begin{teo}[principio de buena ordenación]
Todo conjunto no vacío de enteros positivos contiene uno que es el menor \\\\
Demostración.- \; Sea $T$ una colección no vacía de enteros positivos. Queremos demostrar que $t_0$ tiene un número que es el menor, esto es, que hay en T un entero positivo t.; tal que $t_0\leq t$ para todo $t$ de $T$.\\
Supongamos que no fuera así. Demostraremos que esto nos conduce a una contradicción. El entero $1$ no puede pertenecer a $T$ (de otro modo él sería el menor número de $T$). Designemos con $S$ la colección de todos los enteros positivos $n$ tales que $n<t$ para todo $t$ de $T$. Por tanto $1$ pertenece a $S$ porque $1 < t$ para todo $t$ de $T$. Seguidamente, sea $k$ un entero positivo de $S$. Entonces $k < t$ para todo $t$ de $T$. Demostraremos que $k + 1$ también es de $S$. Si no fuera así, entonces para un cierto $t$, de $T$ tendríamos $t_1 \leq k+1$. Puesto que $T$ no posee número mínimo, hay un entero $t_2$ en $T$ tal que $t_2 < t_1$ Y por tanto $t_2 < k + 1$. Pero esto significa que $t_2 \leq k$, en contradicción con el hecho de que $k < t$ para todo $t$ de $T$. Por tanto $k + 1$ pertenece a $S$. Según el principio de inducción, $S$ contiene todos los enteros positivos. Puesto que $T$ es no vacío, existe un entero positivo $t$ en $T$. Pero este $t$ debe ser también de $S$ (ya que $S$ contiene todos los enteros positivos). De la definición de $S$ resulta que $t < t$, lo cual es absurdo. Por consiguiente, la hipótesis de que $T$ no posee un número mínimo nos lleva a una contradicción. Resulta pues que $T$ debe tener un número mínimo, y a su vez esto prueba que el principio de buena ordenación es una consecuencia del de inducción.\\\\
\end{teo}

\section{Ejercicios}
\begin{enumerate}[\bfseries  1.]
%----------------------------1----------------------------
\item Demostrar por inducción las fórmulas siguientes:
\begin{enumerate}[\bfseries (a)]
%(a)
\item $1+2+3+...+n=n(b+1)/2$\\\\
Demostración.- \; Sea $n=k$ entonces $1+2+3+...+k=k(k+1)/2$.\\
Para $k=1$ se tiene $1=1(1+1)/2$. \\ 
Por ultimo si  $k=k+1$ nos queda probar que  $1+2+3+...+k+(k+1)=\dfrac{(k+1)(k+2)}{2}$, luego $\dfrac{k(k+1)}{2}+(k+1)=\dfrac{k(k+1)+2(k+1)}{2}$. Así $\dfrac{k^2+3k+2}{2}$ \; y \; $\dfrac{(x+1)(x+2)}{2}$\\\\
%(b)
\item $1+3+5+...+(2n-1)=n^2$\\\\
Demostración.- \; Sea $n=k$ entonces $1+3+5+...+(2k-1)=k^2$.\\
Para $k=1$ se tiene $[2(1)-1]=1^2$, así $1=1$\\
Luego, si $k=k+1$ entonces $1+3+5+...+[2(k+1)-1]=(k+1)^2$. Por lo tanto $k^2+2k+1=(k+1)^2$.\\\\
%(c)
\item $1^3+2^3+3^3+...+n^3=(1+2+3+...+n)^2$\\\\
Demostración.- \; Sea $n=k$ entonces, $$1^3+2^3+3^3+...+k^3=(1+2+3+...+k)^2$$
Para $k=1$ $$1=1,$$
Luego $k=k+1$, $$1^3+2^3+3^3+...+k^3+(k+1)^3=(1+2+3+...+k)^2,$$
Así,
\begin{center}
\begin{tabular}{r c l}
$\left(\dfrac{k(k+1)}{2} \right)^2+(k+1)^3=$&=&$\left( \dfrac{k(k+1)}{2}+(k+1)\right) ^2$\\\\
$\dfrac{k^2(k+1)^2}{4}+(k+1)^3$&=&$\dfrac{k^2(k+1)^2}{4}+k(k+1)^2+(k+1)^2$\\\\
$\dfrac{k^2(k+1)^2+4(k+1)^3}{4}$&=&$\dfrac{k^2(k+1)^2+4k(k+1)^2+4(k+1)^2}{4}$\\\\
$\dfrac{(k+1)^2 (k^2+4k+4)}{4}$&=&$\dfrac{(k+1)^2 (k^2 + 4k +4)}{4}$\\\\
\end{tabular}   
\end{center}
%(d)
\item $1^3+2^3+...+(n-1)^3<n^4/4<1^3+2^3+...+n^3$\\\\
Demostración.- \; Sea $n=k$ entonces $$1^3+2^3+...+(k-1)^3<k^4/4 \; \; \; (1)$$ 
Para $k=1$, \; $0<1/4$ se observa que se cumple.\\
Después, para $k=k+1$, $$1^3+2^3+...+k^3<(k+1)^4/4,$$ sumando $k^3$ a $(1)$, 
$$1^3+2^3+...+k^3<k^4/4+k^3$$ y para deducir como consecuencia de $k+1$, basta demostrar, $$k^4/4+k^3<(k+1)^4/4$$, Pero esto es consecuencia inmediata de la igualdad $$(k+1)^4/4=(k^4+4k^3+6k^2+4k+1)/4=k^4/4+k^3+(3k^2)/2+k+1$$ Por tanto se demostró que $k+1$ es consecuencia de $k$\\\\
\end{enumerate}

%-----------------------------2---------------------------------
\item Obsérvese que: 
\begin{center}
\begin{tabular}{r c l}
$1$&$=$&$1$\\
$1-4$&$=$&$-(1+2)$\\
$1-4+9$&$=$&$1+2+3$\\
$1-4+9-16$&$=$&$-(1+2+3+4)$\\
\end{tabular}
\end{center}
Indúzcase la ley general y demuéstrese por inducción\\\\
Demostración.- \: Verificando tenemos que la ley general es $1-4+9-16+...+(-1)^{n+1}\cdot n^2=(-1)^{n+1}(1+2+3+...+n)$. \\
Ahora pasemos a demostrarlo. Sea $n=k$ entonces, $$1-4+9-16+...+(-1)^{k+1}\cdot k^2=(-1)^{k+1}(1+2+3+...+k)$$ Si $k=1$, se sigue, $(-1)^2 \cdot 1^2 = (-1)^2\cdot 1$, vemos que satisface para $k=1.$
Luego $k=k+1$,  $$1-4+9-16+...+(-1)^{k+2}\cdot (k+1)^2=(-1)^{k+2}\left[\dfrac{(k+1)(k+2)}{2}\right]$$.\\
Sumando $(-1)^{k+2}\cdot (k+1)^2$  a la segunda igualdad dada, se tiene,
$$1-4+9-16+...+ (-1)^{k+2}\cdot (k+1)^2 = (-1)^{k+1}\left(\dfrac{k(k+1)}{2}\right) + (-1)^{k+2}\cdot (k+1)^2$$ Por lo tanto, basta demostrar que $(-1)^{k+1}\left(\dfrac{k(k+1)}{2}\right) + (-1)^{k+2}\cdot (k+1)^2=(-1)^{k+2}\left[\dfrac{(k+1)(k+2)}{2}\right]$\\
\begin{center}
\begin{tabular}{r c l}
$(-1)^{k+1}\left(\dfrac{k(k+1)}{2}\right) + (-1)^{k+2}\cdot (k+1)^2$&$=$&$(-1)^{k+2}\left\lbrace \dfrac{[(-1)(k+1)k]+2(k^2+2k+1)}{2} \right\rbrace$\\\\
&$=$&$(-1)^{k+2} \left( \dfrac{-k^2 -k +2k^2 +4k +2}{2} \right)$\\\\
&$=$&$(-1)^{k+2} \left( \dfrac{k^2+3k+2}{2} \right)$\\\\
&$=$&$(-1)^{k+2} \left[ \dfrac{(x+1)(x+2)}{2} \right]$\\\\
\end{tabular}
\end{center}
%-------------------------------3--------------------------------
\item Obsérvese que: 
\begin{center}
\begin{tabular}{r c l}
$1+\frac{1}{2}$&=&$2-\frac{1}{2}$\\\\
$1+\frac{1}{2}+\frac{1}{4}$&=&$2-\frac{1}{4}$\\\\
$1+\frac{1}{2}+\frac{1}{4}+\frac{1}{8}$&=&$2-\frac{1}{8}$\\\\
\end{tabular}
\end{center}
Demostración.- \; Se verifica que $1+\frac{1}{2}+\frac{1}{4}+...+\dfrac{1}{2^n}=2-\dfrac{1}{2^n}$.\\\\
Para $n=k$ $$1+\frac{1}{2}+\frac{1}{4}+...+\dfrac{1}{2^k}=2-\dfrac{1}{2^k}$$\\
$k=1$ $$1+\dfrac{1}{2^1}=2-\dfrac{1}{2^1}$$\\
Luego $k=k+1$ $$1+\frac{1}{2}+\frac{1}{4}+...+\dfrac{1}{2^{k+1}}=2-\dfrac{1}{2^{k+1}}$$\\
Así solo falta demostrar que, $$2-\dfrac{1}{2^k}+\dfrac{1}{2^{k+1}}=2-\dfrac{1}{2^{k+1}}$$
\begin{center}
\begin{tabular}{r c l}
$2-\dfrac{1}{2^k}+\dfrac{1}{2^{k+1}}$&=&$2+\dfrac{-2+1}{2^{k+1}}$\\\\
&=&$2-\dfrac{1}{2^{k+1}}$\\\\
\end{tabular}
\end{center}

%-----------------------------4-----------------------------
\item Obsérvese que:
\begin{center}
\begin{tabular}{r c l}
$1-\frac{1}{2}$&=&$\frac{1}{2}$\\\\
$(1-\frac{1}{2})(1-\frac{1}{3})$&=&$\frac{1}{3}$\\\\
$(1-\frac{1}{2})(1-\frac{1}{3})(1-\frac{1}{4})$&=&$\frac{1}{4}$\\\\
\end{tabular}
\end{center}
Indúzcase la ley general y demuéstrese por inducción.\\\\
Demostración.- \; Se induce que $\left( 1-\dfrac{1}{2} \right)\left( 1-\dfrac{1}{3} \right)...\left(1- \dfrac{1}{n} \right)=\dfrac{1}{n}$ para todo $n>1$.\\
Sea $n=k$, entonces $\left( 1-\dfrac{1}{2} \right)\left( 1-\dfrac{1}{3} \right)...\left(1- \dfrac{1}{k} \right)=\dfrac{1}{k}$. Después para $k=2$, \; $1-\dfrac{1}{2}=\dfrac{1}{2}$. Si $k=k+1$ tenemos $\left( 1-\dfrac{1}{2} \right) \left( 1-\dfrac{1}{3} \right)...\left(1- \dfrac{1}{k+1} \right)=\dfrac{1}{k+1}$. Luego es fácil comprobar que 
$\dfrac{1}{k}\left(1 - \dfrac{1}{k+1} \right)=\dfrac{1}{k+1}$. \\\\

%----------------------------5--------------------------------
\item Hallar la ley general que simplifica al producto $$\left( 1-\dfrac{1}{4} \right)\left( 1-\dfrac{1}{9} \right)\left( 1- \dfrac{1}{16} \right)..\left( 1- \dfrac{1}{n^2} \right)$$ y demuéstrese por inducción.\\\\
Demostración.- \; Inducimos que $\left( 1-\dfrac{1}{4} \right)\left( 1-\dfrac{1}{9} \right)\left( 1- \dfrac{1}{16} \right)..\left( 1- \dfrac{1}{n^2} \right)=\dfrac{n+1}{2n}$,para todo $n>1$. Después $n=k=1$, $$1-\dfrac{1}{2^2}=\dfrac{2+1}{2\dot 2} \Rightarrow \dfrac{3}{4}=\dfrac{3}{4}$$ 
Luego $k+1$, $$\left( 1-\dfrac{1}{4} \right)\left( 1-\dfrac{1}{9} \right)\left( 1- \dfrac{1}{16} \right)..\left( 1- \dfrac{1}{(k+1)^2} \right)=\dfrac{(k+1)+1}{2(k+1)}$$
Así,
\begin{center}
\begin{tabular}{r c l}
$\left( \dfrac{k+1}{2k}\right) \left( 1- \dfrac{1}{(k+1)^2} \right)$&=&$\dfrac{(k+1)+1}{2(k+1)}$\\\\
$\left( \dfrac{k+1}{2k} - \dfrac{1}{2k(k+1)} \right)$&=&$\dfrac{k+2}{2k+2}$\\\\
$\dfrac{(k+1)^2-1}{2k(k+1)}$&=&$\dfrac{k+2}{2k+2}$\\\\
$\dfrac{k+2}{2k+2}$&=&$\dfrac{k+2}{2k+2}$\\\\
\end{tabular}
\end{center}

%--------------------------6-------------------------------
\item Sea $A(n)$ la proporción: $1+2+...+n=\dfrac{1}{8} (2n+1)^2$.
\begin{enumerate}[\bfseries a)]
%(a)
\item Probar que si $A(k)$, $A(k+1)$ también es cierta.\\\\
Demostración.- \; Para $A(k+1)$, 
\begin{center}
\begin{tabular}{r c l}
$\dfrac{1}{8}(2k+1)^2+(k+1)$&=&$\dfrac{1}{8}\left[2(k+1)+1\right]^2$\\\\
$\dfrac{4k^2+12k+9}{8}$&=&$\dfrac{4k^2+12k+9}{8}$\\\\
\end{tabular}
\end{center}
%(b)
\item Critíquese la proposición $"$de la inducción se sigue que $A(n)$ es cierta para todo $n$ $"$.\\\\
Se ve que no se cumple para ningún entero $A(n)$ pero si para $A(n+1)$.\\\\ 
%(c)
\item Transfórmese $A(n)$ cambiando la igualdad por una desigualdad que es cierta para todo entero positivo $n$\\\\
Primero comprobemos para $A(1)$, \; $1<\dfrac{9}{8}$.\\
Luego para $A(k),$ $$1+2+...+(k+1)<\dfrac{1}{8}(2k+1)^2$$
Después para $A(k+1)$ $$1+2+...+(k+1)<\dfrac{1}{8}(2k+1)^2$$
Remplazando $(k+1)$ a $A(k)$ $$1+2+...+(k+1)<\dfrac{1}{8}(2k+1)^2+(k+1)$$
por último solo nos queda demostrar $$\dfrac{1}{8}(2k+1)^2<\dfrac{1}{8}(2k+1)^2+(k+1)$$
Así $\dfrac{4k^2+12k+9}{8}<\dfrac{4k^2+12k+9}{8} +(k+1)$, vemos que la inecuación se cumple para cualquier número natural.\\\\ 
\end{enumerate} 

%--------------------------7-------------------------------
\item Sea $n_1$ el menor entero positivo $n$ para el que la desigualdad $(1+x)^n>1+nx+nx^2$ es cierta para todo $x>0$. Calcular $n_1$, y demostrar que la desigualdad es cierta para todos los enteros $n\geq n_1$\\\\
Demostración.- \; vemos que la proposición es validad para $n_1=3$, $$(1+x)^3>1+3x+3x^2,$$ y no así para $n=1$ \; y \;$n=2$ entonces $A(n)=A(k)\geq 3$, $(1+x)^k>1+kx+kx^2.$ Después para un $A(k+1)$, $(1+x)^{k+1}>1+(k+1)x+(k+1)x^2,$ así \; $(1+kx+kx^2)(1+k)>1+(k+1)x+(k+1)x^2,$ luego se cumple la desigualdad $x(kx^2)+x^2+kx+x+1+x^2>kx^2+x^2+kx+x+1$.\\\\

%----------------------------8-------------------------------
\item Dados números reales positivos $a_1,a_2,a_3,...,$ tales que $a_n\leq ca_{a-1}$ para todo $n\geq 2$. Donde $c$ es un número positivo fijo, aplíquese el método de inducción para demostrar que $a_n \leq a_1 c^{n-1}$ para cada $n \geq 1$\\\\
Demostración.- \; Primero, para el caso $n=1$, tenemos $a_1c^0=a_1$, por lo tanto la desigualdad es validad. Ahora supongamos que la desigualdad es válida para algún número entero $k$: $a_k\leq a_1c^{k-1}$, luego multiplicamos por $c$, $ca_k\leq a_1c^k$, pero dado que se asume por hipótesis $a_{k+1} \leq ca_k$, entonces $a_{k+1}\leq a_1c^k$, por lo tanto, la declaración es válida para todo $n$.\\\\

%-------------------------------9----------------------------
\item Demuéstrese por inducción la proposición siguiente: Dado un segmento de longitud unidad, el segmento de longitud $\sqrt{n}$ se puede construir con regla y compás para cada entero positivo $n$.\\\\
Demostración.- \; Dada una línea de longitud 1, podemos construir una línea de longitud $\sqrt{2}$ tomando la hipotenusa del triángulo rectángulo con patas de longitud 1.\\
Ahora, supongamos que tenemos una línea de longitud 1 y una línea de longitud $\sqrt{k}$ para algún número entero $k$. Luego podemos formar un triángulo rectángulo con patas de longitud 1 y longitud $\sqrt{k}$. La hipotenusa de este triángulo es $\sqrt{k+1}$. Por lo tanto, si podemos construir una línea de longitud $\sqrt{k}$, entonces podemos construir una línea de longitud $\sqrt{k+1}$. Como podemos construir una línea de longitud $\sqrt{2}$ en el caso base, podemos construir una línea de longitud $\sqrt{n}$ para todos los enteros $n$.\\\\

%------------------------------10------------------------------
\item Sea $b$ un entero positivo. Demostrar por inducción la proposición siguiente: Para cada entero $n\geq 0$ existen enteros no negativos $q$ \; y \; $r$ tales que:
$$n=qb+r, \; \; \; 0\leq r < b$$ \\
Demostración.- \; Sea $b$ ser un entero positivo fijo. Si $n=0$ , luego $q=r=0$, la afirmación es verdadera (ya que $0=0b+0$).\\
Ahora suponga que la afirmación es cierta para algunos $k \in \mathbb{N}$. Por hipótesis de inducción sabemos que existen enteros no negativos $q$ \; y \; $r$ tales que $$k=qb+r, \; \; \; 0\leq r < b,$$ Por lo tanto, sumando 1 a ambos lados tenemos, $$k+1=qb+(r+1).$$ Pues $0\leq r < b$ entonces sabemos que $0\leq r \leq (b-1)$. Si $0\leq r < b-1,$ entonces $0\leq r+1<b,$ y la declaración aún se mantiene con la misma elección $q$ \; y \; $r+1$ en lugar de $r$.\\
Por otro lado, si $r=b-1,$ entonces $r+1=b$ y tenemos, $$k+1=qb+b=(q+1)b+0$$
Por lo tanto, la declaración se mantiene de nuevo, pero con $q+1$ en lugar de $q$ \; y con $r=0$ (que es válido ya que si $r=0$ tenemos $a\leq r < b$). Por ende, si el algoritmo de división es válido para $k$, entonces también es válido para $k+1.$ Entonces, es válido para todos $n \in \mathbb{N}$.\\\\

%-------------------------------11-------------------------------
\item Sea $n$ \; y \; $d$ enteros. Se dice que $d$ es un divisor de $n$ si $n=cd$ para algún entero $c$. Un entero $n$ se denomina primo si $n>1$ y los únicos divisores de $n$ son $1$ \; y \; $n$. Demostrar por inducción que cada entero $n>1$ es o primo o producto de primos.\\\\
Demostración.- \; LA prueba se hará por inducción. Si $n=2$ ó $n=3$ entonces $n$ es primo, entonces la proposición es verdadera.\\
Ahora supongamos que la afirmación es verdadera para todos los enteros desde $2$ hasta $k$. Se debe demostrar que esto implica $k+1$ es primo o un producto de primos. Si $k+1$ es primo, entonces no hay nada que demostrar. Por otro lado, si $k+1$ no es primo, entonces sabemos que hay enteros $c$ \; y \; $d$ tal que $1<c,$ $d<k+1$ en otras palabras decimos que $n$ es divisible por números disntintos de $1$ y de sí mismo.\\
Por hipótesis de inducción, sabemos que $2\leq c$, $d\leq k$ entonces $c$ \; y \; $d$ son primos o son producto de primos.\\
Por lo tanto, si la declaración es verdadera para todos los enteros mayores que $1$ hasta $k$ entonces, también es verdadera para $k+1$, así es cierto para todo $n\in \mathbb{Z}^+$\\\\

%--------------------------12------------------------------
\item Explíquese el error en la siguiente demostración por inducción.\\
Proposición.- Dado un conjunto de n niñas rubias, si por 10 menos una de las niñas tiene ojos azules, entonces las n niñas tienen ojos azules.\\
Demostración.-\; La proposición es evidentemente cierta si $n = 1$. El paso de $k$ a $k + 1$ se puede ilustrar pasando de $n = 3$ a $n = 4$. Supóngase para ello que la proposición es cierta para $n=3$ Y sean $G_1, G_2, G_3, G_4$ cuatro niñas rubias tales que una de ellas, por lo menos, tenga ojos azules, por ejemplo, la $G_1$, Tomando $G_1,G_2, G_3,$ conjuntamente y haciendo uso de la proposición cierta para $n =3$, resulta que también $G_2$ y $G_3$ tienen ojos azules. Repitiendo el proceso con $G_1, G_2$ y $G_4,$ se encuentra igualmente que $G_4$ tiene ojos azules. Es decir, las cuatro tienen ojos azules. Un razonamiento análogo permite el paso de $k$ a $k + 1$ en general.\\
\textbf{Corolario.} Todas las niñas rubias tienen ojos azules.\\
Demostración.- \; Puesto que efectivamente existe una niña rubia con ojos azules, se puede aplicar el resultado precedente al conjunto formado por todas las niñas rubias.\\\\
Esta prueba supone que la afirmación es cierta n = 3, es decir, supone que si hay tres chicas rubias, una de las cuales tiene ojos azules, entonces todas tienen ojos azules. Claramente, esta es una suposición falsa.\\\\
\end{enumerate}

\setcounter{section}{6}
\section{Ejercicios}
\begin{enumerate}[\bfseries  1.]
%---------------------------1---------------------------------
\item Hallar los valores numéricos de las sumas siguientes:
\begin{enumerate}[\bfseries a)]
%(a)
\item $\displaystyle\sum_{k=1}^{4} k = 1 + 2 + 3 + 4 = 10$\\\\
%(b)
\item $\displaystyle\sum_{n=2}^{5} 2^{n-2} = 2^{0} + 2^{1} + 2^{2} + 2^{3} = 15$\\\\
%(c)
\item $\displaystyle\sum_{r=0}^{3} 2^{2r+1} = 2^{1} + 2^{3} +  2^{5} + 2^{7} = 2 + 8 + 32 + 128 = 170$ \\\\
%(d)
\item $\displaystyle\sum_{n=1}^{4} n^n = 1^1 + 2^2 + 3^3 + 4^4 = 1 + 4 + 27 + 256= 288$ \\\\
%(e)
\item $\displaystyle\sum_{i=0}^{5} (2i + 1) = 1 + 3 + 5 + 7 + 9 + 11 = 36$ \\\\
%(f)
\item $\displaystyle\sum_{k=1}^{5} \dfrac{1}{k(k+1)} = \dfrac{1}{2} + \dfrac{1}{6} + \dfrac{1}{12} + \dfrac{1}{20} + \dfrac{1}{30} = 0,83333.....$ \\\\
\end{enumerate}

%-------------------------------2---------------------------
\item Establecer las siguientes propiedades del símbolo sumatorio. 
\begin{enumerate}[\bfseries a)]
%(a)
\item $\displaystyle\sum_{k=1}^{n}(a_k + b_k) = \sum_{k=1}^n a_k + \sum_{k=1}^{n} b_k$ (propiedad aditiva)
\begin{center}
\begin{tabular}{r c l l}
$\displaystyle\sum_{k=1}^{n}(a_k + b_k)$&$=$&$(a_1 + b_1) + (a_2 + b_2) + ... + (a_n + b_n)$&\\\\
&$=$&$(a_1 + a_2 + ... + a_n) + (b_1 + b_2 + ... + b_n)$&Asociatividad y conmutatividad\\\\
&$=$&$\displaystyle\sum_{k=1}^{n} a_k + \sum_{k=1}^{n} b_k$&\\\\
\end{tabular}
\end{center}
%(b)
\item $\displaystyle\sum_{k=1}^{n} (ca_k) = c \sum_{k=1}^{n} a_k$ (Prpiedad homogénea)
\begin{center}
\begin{tabular}{r c l l}
$\displaystyle\sum_{k=1}^{n} (ca_k)$&$=$&$ca_1 + ca_2 + ... + ca_n$&\\\\
&$=$&$c(a_1 + a_2 +...+a_n)$&distributividad\\\\
&$=$&$ c \displaystyle\sum_{k=1}^{n} a$&\\\\
\end{tabular}
\end{center}
%(c)
\item $\displaystyle\sum_{k=1}^{n} (a_k - a_{k-1}) = a_n - a_0$ (Propiedad telescópica)
\begin{center}
\begin{tabular}{r c l l}
$\displaystyle\sum_{k=1}^{n} (a_k - a_{k-1})$&$=$&$(a_1-a_0)+(a_2-a_1)+...+(a_n-a_{n-1})$&\\
&$=$&$(a_1+a_2+...+a_{n-1})-(a_0+a_1+...+a_{n-1})+a_n$&\\\\
&$=$&$a_n + \displaystyle\sum_{k=1}^{n-1} a_k - \sum_{k=0}^{n-1} a_k$&Reindexar la 2da suma\\\\
&$=$&$a_n + \displaystyle\sum_{k=1}^{n-1} a_k - a_0 - \sum_{k=1}^{n-1} a_k$&\\\\
&$=$&$a_n - a_0$&\\\\
\end{tabular}
\end{center}
\end{enumerate}

%--------------------------3-------------------------------
\item $\displaystyle\sum_{k=1}^{n} 1 = n$ (El sentido de esta suma es $\displaystyle\sum_{k=1}^{n} a_k$, cuando $a_k=1$)\\\\
Demostración.- \; Probemos por inducción, Si $n=1$, entonces la proposición es verdadera ya que $\displaystyle\sum_{k=1}^{1} 1 = 1.$ Ahora supongamos que el enunciado es verdadero para $n=m \in \mathbb{Z}^+.$ Luego, $$\displaystyle\sum_{k=1}^m 1 = m \Rightarrow \left( \sum_{k=1}^m 1 = m \right) + 1 = m + 1 \Rightarrow \sum_{k=1}^{m+1} 1 = m+1$$\\\\

%---------------------------4---------------------------------
\item $\displaystyle\sum_{k=1}^{n} (2k - 1) = n^2$ $[Indicación, \; 2k-1 = k^2 - (k-1)^2]$\\\\
Demostración.- \; Sea $2k-1 = k^2 - (k^2 - 2k + 1) = k^2 - (k-1)^2$ entonces  por la propiedad telescópica se tiene $\sum\limits_{k=1}^n (2k-1)= \sum\limits_{k=1}^{n} (k^2 - (k-1)^2) = n^2 + 0 = n^2$\\\\

%------------------------------5--------------------------
\item $\displaystyle\sum_{k=1}^{n} k = \dfrac{n^2}{2} + \dfrac{n}{2} $ \; \; [indicación. \; Úsese el ejercicio 3 y el 4.]\\\\
Demostración.- \;Por aditividad y homogeneidad se tiene $\sum\limits_{k=1}^{n}(2k-1) = 2\sum\limits_{k=1}^{n} k - \sum\limits_{k=1}^{n} 1 = n^2$ entonces $ 2 \sum\limits_{k=1}^{n} k - n =n^2$ ya que $\sum\limits_{k=1}^{n} 1 = n$, luego $\sum\limits_{k=1}^{n} k = \dfrac{n^2}{2} + \dfrac{n}{2}$\\\\

%------------------------------6----------------------------
\item $\displaystyle\sum_{k=1}^{n} k^2 = \dfrac{n^3}{3} + \dfrac{n^2}{2} + \dfrac{n}{6}$ $[Indicación. \; k^3 - (k-1)^3 = 3k^2 - 2k + 1]$\\\\
Demostración.- \;  Por la propiedad telescópica se tiene $\sum\limits_{k=1}^n k^3 - (k-1)^3 = n^3$, luego 
\begin{center}
\begin{tabular}{r c l l}
$n^3 = \sum\limits_{k=1}^n 3k^2-3k+1$&$\Rightarrow$&$n^3 = \sum\limits_{k=1}^n k^2 - 3 \sum\limits_{k=1}^n k + \sum\limits_{k=1}^n 1$&propiedad aditiva\\\\
&$\Rightarrow$&$\sum\limits_{k=1}^n k^2 = \dfrac{n^3}{3} + \sum\limits_{k=1}^n k - \dfrac{1}{3} \sum\limits_{k=1}^n 1$&\\\\
&$\Rightarrow$&$\sum\limits_{k=1}^n k^2 = \dfrac{n^3}{3} + \dfrac{n^2}{2} + \dfrac{n}{2} - \dfrac{n}{3}$&por los anteriores ejercicios\\\\
&$\Rightarrow$&$\sum\limits_{k=1}^n k^2 = \dfrac{n^3}{3} + \dfrac{n^2}{2} + \dfrac{n}{6}$&\\\\
\end{tabular}
\end{center}

%----------------------------7--------------------------------
\item $\displaystyle\sum_{k=1}^n k^3 = \dfrac{n^4}{4} + \dfrac{n^3}{2} + \dfrac{n^2}{4}$\\\\
Demostración.- \; Sea $k^4 - (k-1)^4 = 4k^3 - 6k^2 + 4k -1$ entonces  
\begin{center}
\begin{tabular}{r c l}
$n^4 = 4 \sum\limits_{k=1}^n k^3 - 6 \sum\limits_{k=1}^n k^2 + 4 \sum\limits_{k=1}^n k - \sum\limits_{k=1}^n 1$&$\Rightarrow$&$n^4 = 4 \sum\limits_{k=1}^n k^3 - 6 \left( \dfrac{n^3}{3} + \dfrac{n^2}{2} + \dfrac{n}{6} \right) + 4 \left( \dfrac{n^2}{2} + \dfrac{n}{2} \right) - n$\\\\
&$\Rightarrow$&$4 \sum\limits_{k=1}^n k^3 = n^4 + 2n^3 + 3n^2 - 2n^2$\\\\
&$\Rightarrow$&$\sum\limits_{k=1}^n k^3 = \dfrac{n^4}{4} + \dfrac{n^3}{2} + \dfrac{n^2}{4}$\\\\
\end{tabular}
\end{center}

%----------------------------8------------------------------------
\item 
\begin{enumerate}[\bfseries a)]
%a)
\item $\displaystyle\sum_{k=0}^n x^k = \dfrac{1 - x^{n+1}}{1-x}$ si $x\neq 1$. Nota: Por definición $x^0 = 1$ [Indicación. Aplíquese el ejercicio 2 a $(1-x) \sum\limits_{k=0}^n x^k.$]\\\\
Demostración.- Por propiedades  aditiva y telescópica, 
$$(1-x) \sum\limits_{k=0}^n x^k =\sum\limits_{k=0}^n (x^k - x^{k+1}) = - \sum\limits_{k=0}^n (x^{k+1} - x^k)=-(x^{n+1}-1) = 1-x^{n+1}$$ 
pro lo tanto, $$\sum\limits_{k=0}^n x^k = \dfrac{1 - x^{n+1}}{1-x}$$\\

%b)
\item ¿Cuál es la suma cuando $x=1$?\\\\
Respuesta.- \; 	Si $x=1$ por la parte 3 y el hecho de que $1^k=1$ para $k=0,...,n$ tenemos $$\sum\limits_{k=0}^n x^k = \sum\limits_{k=0}^n 1 = n+1$$\\ 
\end{enumerate}

%----------------------------9-----------------------------
\item Demostrar por inducción, que la suma $\sum\limits_{k=1}^{2n} (-1)^k (2k+1)$ es proporcional a $n$, y hallar la constante de proporcionalidad.\\\\
Demostración.- \; Sea $n=1$ entonces $\sum\limits_{k=1}^2 (-1)^k (2k+1) = 2$. Así vemos que es valida para este caso. Observemos que  $2n=2$, lo mismo pasa para $n=2$ donde $2n=4.$
Luego supongamos que la fórmula es válida para $n=m \in \mathbb{Z}^+$, es decir $$\sum\limits_{k=0}^2m (-1)^k (2k+1) = 2m,$$ entonces
\begin{center}
\begin{tabular}{r c l}
$\sum\limits_{k=1}^{2(m+1)} (-1)^k(2k+1)$&$=$&$(-1)(2k+1)$\\\\
&$=$&$2m-[2(2m+1)+1] + [2(2m+2)+1]$\\\\
&$=$&$2m-4m-3+4m+5$\\\\
&$=$&$2(m+1)$\\\\
\end{tabular}
\end{center}

%---------------------------10-----------------------------
\item 
\begin{enumerate}[\bfseries a)]
%a)
\item Dar una definición razonable del símbolo $\sum\limits_{k=m}^{m+n} a_k$\\\\
$$\sum\limits_{k=m}^{m+n} a_k = a_m + a_{m+1} + ... + a_{m+n}$$\\

%b)
\item Demostrar por inducción que para $n\geq 1$ se tiene
$$\displaystyle\sum_{k=n+1}^{2n} \dfrac{1}{k} = \sum_{m=1}{2n}\dfrac{(-1)^{m+1}}{m}$$\\\\
Demostración.- \; Sea $n=1$ se tiene,
$$\sum\limits_{k=n+1}^{2n} \dfrac{1}{k} = \sum\limits_{k=2}^{2} \dfrac{1}{k} = \dfrac{1}{2}$$
luego, $$\sum\limits_{m=1}^{2n} \dfrac{(-1)^{m+1}}{m} = \sum\limits_{m=1}^{2} \dfrac{(-1)^{m+1}}{m} = 1 - \dfrac{1}{2} = \dfrac{1}{2}$$
Ahora supongamos que se cumple para $n=j \in \mathbb{Z}^+$ entonces,
\begin{center}
\begin{tabular}{r c l}
$\sum\limits_{k=n+1}^{2n} \dfrac{1}{k} = \sum\limits_{m=1}^{2n}\dfrac{(-1)^{m+1}}{m}$&$\Rightarrow$&$\left( \sum\limits_{k=j+1}^{2j} \dfrac{1}{k} \right) + \dfrac{1}{2j+1} + \dfrac{1}{2j+2} = \left( \sum\limits_{m=1}^{2j} \dfrac{(1)^{m+1}}{m} \right) + \dfrac{1}{2j+1} + \dfrac{1}{2j+2}$\\\\
&$\Rightarrow$&$\left( \sum\limits_{k=j+1}^{2(j+1)} \dfrac{1}{k} \right)  = \left( \sum\limits_{m=1}^{2j} \dfrac{(1)^{m+1}}{m} \right) + \dfrac{1}{2j+1} + \dfrac{1}{2j+2}$\\\\
&$\Rightarrow$&$\dfrac{1}{j+1} + \left( \sum\limits_{k=j+2}^{2(j+1)} \dfrac{1}{k} \right) = \left( \sum\limits_{m=1}^{2j} \dfrac{(1)^{m+1}}{m} \right) + \dfrac{1}{2j+1} + \dfrac{1}{2j+2}$\\\\
&$\Rightarrow$&$\sum\limits_{k=j+2}^{2(j+1)} \dfrac{1}{k} = \left( \sum\limits_{m=1}^{2j} \dfrac{(1)^{m+1}}{m} \right) + \dfrac{1}{2j+1} + \dfrac{1}{2j+2} - \dfrac{1}{j+1}$\\\\
&$\Rightarrow$&$\sum\limits_{k=j+2}^{2(j+1)} \dfrac{1}{k} = \left( \sum\limits_{m=1}^{2j} \dfrac{(1)^{m+1}}{m} \right)  - \dfrac{1}{2(j+1)}$\\\\
&$\Rightarrow$&$\sum\limits_{k=j+2}^{2(j+1)} \dfrac{1}{k} = \left( \sum\limits_{m=1}^{2(j+1)} \dfrac{(1)^{m+1}}{m} \right)$\\\\
\end{tabular}
\end{center}
\end{enumerate}

%----------------------------11---------------------------
\item Determinar si cada una de las igualdades siguientes es cierta o falsa. En cada caso razonar la decisión.
\begin{enumerate}[\bfseries (a)]
%(a)
\item $\displaystyle\sum_{n=0}^{100} n^4= \sum_{n=1}^{100} n^4$\\\\
Razonamiento.- \; Dado que se se evalúa para $n=0$ entonces se tiene que $0^4 = 0$ entonces la desigualdad es cierta.\\\\

%(b)
\item $\displaystyle\sum_{j=0}^{100} 2 = 200$\\\\
Razonamiento.- \; En vista que de si se evalúa para $n=0$ el resultado es $2$ entonces para $n=100$ será $202$, por lo tanto la igualdad es falsa\\\\

%(c)
\item $\displaystyle\sum_{k=0}^{100} (2+k) = 2 + \sum_{k=0}^{100} k$\\\\
Razonamiento.- \; La igualdad es falsa debido a que el lado derecho de la igualdad se añade el $2$ solo una vez, a diferencia de la igualdad de la izquierda que se añade a cada iteración de la suma.\\\\

%(d)
\item $\displaystyle\sum_{i=1}^{100} (i+1)^2 = \sum_{i=0}^{99} i^2$\\\\
Razonamiento.- \; La igualdad es falsa ya que al cuadrado de la serie de la izquierda se añade en $1$ a cada iteración, contemplando que $n=100$\\\\ 

%(e)
\item $\displaystyle\sum_{k=1}^{100} k^3 = \left( \sum_{k=1}^{100} k \right)\cdot \left(  \sum_{k=1}^{100} k^2 \right)$\\\\
Razonamiento.- \; La igualdad es falsa, ya que anteriormente se dijo que $$\sum\limits_{k=1}^{n} \dfrac{n^4}{4} + \dfrac{n^3}{2} + \dfrac{n^2}{4},$$ mientras que $$\left( \sum\limits_{k=1}^{n} k \right)\left( \sum\limits_{k=1}^{n} k^2 \right)$$ \\\\

%(f)
\item $\displaystyle\sum_{k=0}^{100} k^3 = \left( \sum_{k=0}^{100} k \right) ^3$\\\\
Razonamiento.- \; Similar a  la parte $(e)$ se tiene $$\sum\limits_{k=1}^{n} \dfrac{n^4}{4} + \dfrac{n^3}{2} + \dfrac{n^2}{4},$$ mientras que $$\left( \sum\limits_{k=0}^{n} k \right)^3 = \left( \dfrac{n^2}{2} + \dfrac{n}{2}\right)^3$$\\\\
\end{enumerate} 

%------------------------------12-------------------------------------
\item Inducir y demostrar una regla general que simplifique la suma $$\displaystyle\sum_{k=1}^{n} \dfrac{1}{k(k+1)}$$\\\\
Demostración.- \; Esta claro ver que $\sum\limits_{k=1}^{n} \dfrac{1}{k(k+1)} = \dfrac{k}{k+1}$ ya que si sumamos para $n=2$ el resultado es $\dfrac{2}{3}$, para $n=3$ $\dfrac{3}{4}$ y así sucesivamente.\\
Luego efectivamente se cumple para $n=1$,  $\dfrac{1}{1(1+1)} = \dfrac{1}{2} = \dfrac{1}{1+1} = \dfrac{1}{2}$. 
Así asumimos que se cumple para un entero fijo $n=m$,
$$\sum\limits_{k=1}^{m} \dfrac{1}{k(k+1)} = \dfrac{k}{k+1}$$
Ahora supongamos que se cumple para $m+1$ por lo tanto $$\sum\limits_{k=1}^{m+1} \dfrac{1}{k(k+1)} = \sum\limits_{k=1}^{m} \dfrac{1}{k(k+1)} + \dfrac{1}{(k+1)(k+2)} = \dfrac{k+1}{k+2}$$
Por último solo falta demostrar que $ \dfrac{k}{k+1} + \dfrac{1}{(k+1)(k+2)} = \dfrac{k+1}{k+2}$.\\\\
$$\dfrac{k(k+2)+1}{(k+1)(k+2)} = \dfrac{k^2 + 2k +1}{(k+1)(k+2)} = \dfrac{(k+1)(k+1)}{(k+1)(k+2)} = \dfrac{k+1}{k+2}$$\\\\

%------------------------------13----------------------------------
\item Demostrar que $2(\sqrt{n+1} - \sqrt{n}) < \dfrac{1}{\sqrt{n}}< 2(\sqrt{n} - \sqrt{n-1})$ si $n\geq 1.$ Utilizar luego este resultado para demostrar que
$$2\sqrt{m} - 2 < \sum\limits_{n=1}^m \dfrac{1}{\sqrt{n}} < 2 \sqrt{m} - 1$$
si $m \geq 2.$ En particular, cuando $m=10^6$, la suma está comprendida entre $1998$ y $1999.$\\\\
Demostración.- \; Sea  $0<1$  entonces 
\begin{center}
\begin{tabular}{r c l}
$0$&$<$&$1$\\
$4n^2 + 4n$&$<$&$4n^2 + 4n + 1$\\
$4n(n+1)$&$<$&$4n^2 + 4n + 1$\\
$2\sqrt{n} \sqrt{n+1}$&$<$&$2n+1$\\
$2\sqrt{n+1} - 2\sqrt{n}$&$<$&$\dfrac{1}{\sqrt{n}}$\\
$2(\sqrt{n+1} - \sqrt{n})$&$<$&$\dfrac{1}{\sqrt{n}}$\\\\
\end{tabular}
\end{center} 
Análogamente se puede demostrar para $\dfrac{1}{\sqrt{n}} < 2(\sqrt{n} - \sqrt{n-1})$\\\\
Ahora demostramos la segunda parte del enunciado. Consideremos la desigualdad de la izquierda. Para $m=2$ tenemos,
$$2\sqrt{2} -2 ; \; \; 1 + \dfrac{1}{\sqrt{2}}$$ 
\begin{center}
\begin{tabular}{r c l l}
$2\sqrt{2} - 2 $&$<$&$2 \sqrt{2} - \sqrt{2}$& ya que $2 > \sqrt{2}$\\
&$=$&$\sqrt{2}$&\\
&$=$&$\dfrac{1}{\sqrt{2}} + \dfrac{1}{\sqrt{2}}$&ya que $\sqrt{2} = 2\sqrt{2}$\\
&$<$&$1 + \dfrac{1}{\sqrt{2}}$&\\
\end{tabular}
\end{center} 
donde la inecuación se cumple. Luego asumimos que la inecuación se cumple para algún $k \in \mathbb{Z}$ para $k\geq 2,$ entonces sea
\begin{center}
\begin{tabular}{r c l l}
$2\sqrt{2} -2 < \sum\limits_{n=1}^k \dfrac{1}{n}$&$\Rightarrow$&$2\sqrt{2} -2 + \dfrac{1}{\sqrt{k+1}}< \sum\limits_{n=1}^{k+1} \dfrac{1}{n}$&\\\\
&$\Rightarrow$&$2\sqrt{2} - 2 + 2(\sqrt{k+2} - \sqrt{k+1}) < \sum\limits_{n=1}^{k+1} \dfrac{1}{\sqrt{n}}$& por la parte 1 y $\sum\limits_{n=1}^{k+1} \dfrac{1}{n}<\sum\limits_{n=1}^{k+1} \dfrac{1}{\sqrt{n}}$\\\\
&$\Rightarrow$&$2(\sqrt{k+1} - \sqrt{k+1} + 2\sqrt{k} -2) < \sum\limits_{n=1}^{k+1} \dfrac{1}{\sqrt{n}}$&usando la parte 1\\\\
&$\Rightarrow$&$2\sqrt{k+1} - 2 < \sum_{n=1}^{k+1} \dfrac{1}{\sqrt{n}}$&\\\\
\end{tabular}
\end{center}
\end{enumerate}
Por lo tanto la inecuación es verdadera para todo $m \in \mathbb{Z}$, $k\geq 2.$\\
Ahora veamos la inecuación de la derecha. Para el caso de $m=2$ tenemos,
$$+1\dfrac{1}{\sqrt{2}}; \; \; 2\sqrt{2} - 1$$
luego como $\sqrt{2} = \dfrac{2\sqrt{2}}{2}< \dfrac{3}{2}$ tenemos,
\begin{center}
\begin{tabular}{rcll}
$\sqrt{2}< \dfrac{3}{2}$&$\Rightarrow$&$2\sqrt{2}<3$&\\
&$\Rightarrow$&$2\sqrt{2} + 1 < 4$&\\
&$\Rightarrow$&$2 + \dfrac{1}{\sqrt{2}}< 2\sqrt{2}$&\\
&$\Rightarrow$&$1 + \dfrac{1}{\sqrt{2}}<2 \sqrt{2} -1$&\\\\
\end{tabular}
\end{center} 
por lo tanto la desigualdad es cierta. Asumamos entonces que es cierto para algunos $m = k \in \mathbb{Z}, \; k\geq 2$, luego
\begin{center}
\begin{tabular}{rcll}
$\sum\limits_{n=1}^k \dfrac{1}{\sqrt{n}} < 2\sqrt{k} - 1$&$\Rightarrow$&$\left( \sum\limits_{n=1}^k \dfrac{1}{\sqrt{n}} + \dfrac{1}{\sqrt{k+1}} < 2 \sqrt{2} -1 + \dfrac{1}{\sqrt{k+1}}\right)$&\\\\
&$\Rightarrow$&$\sum\limits_{n=1}^{k+1} \dfrac{1}{\sqrt{n}} < 2\sqrt{k} - 1 + 2(\sqrt{k+1} - \sqrt{k})$&parte 1\\\\
&$\Rightarrow$&$\sum\limits_{n=1}^{k+1} \dfrac{1}{\sqrt{n}} < 2\sqrt{k+1} - 1$&parte 1 \\\\
\end{tabular}
\end{center}
Por lo tanto la desigualdad correcta se aplica a todo $m \in \mathbb{Z} k \geq 2$.
\textbf{Cabe recalcar que el libro menciona que $m\geq 2$} pero este último solo se cumple para los extremos de la desigualdad y no así para $\sum\limits_{n=1}^m \dfrac{1}{\sqrt{2}}$\\\\

\section{Valor absoluto y desigualdad triangular}
%definición valor absoluto
\begin{def.}
$$|x| = \left\lbrace
\begin{array}{crr}
\textup{si } & x, & x\geq 0\\
\textup{si }& -x, & x \leq 0
\end{array}
\right.$$
\end{def.}

%teorema 1.38
\begin{teo}
Si $a\geq 0$, es \; $|x|\leq a$ \; si y sólo si \; $-a\leq x \leq a$\\\\
Demostración.- \; Debemos probar dos cuestiones: primero, que la desiguadad $|x|\leq a$ implica las dos desigualdades $-a \leq x \leq a$ y recíprocamente, que $-a \leq x \leq a$ implica $|x|\leq a$.\\ 
Ya supuesto $|x|\leq a$ se tiene también $-a \leq - |x|$. Pero ó \; $x=|x|$ \; ó \; $x = -|x|$ y, por lo tanto, $x\leq a$ \; y \; $-a \leq x$, lo cual prueba la primera parte del teorema.\\
Para probar el recíproco, supóngase $-a\leq x \leq a$. Si $x\leq 0$ se tiene $|x|=x\leq a$; si por el contrario es $x\leq 0$ , entonces $|x|=-x \leq a$. En ambos casos se tiene $|x|\leq a$, lo que demuestra el teorema.\\\\   
\end{teo}

%teorema 1.39
\begin{teo}[Desigualdad triangular]
Para $x$ e $y$ números reales cualesquiera se tiene $$|x+y| \leq |x| + |y|$$\\\\
Demostración.- \; Puesto que $x=|x|$ \; ó \; $x=-|x|$, se tiene $-|x|\leq x \leq |x|$. Análogamente $-|y| \leq y \leq |y|$. Sumando ambas desigualdades se tiene: $$-(|x|+|y|)\leq x+y \leq |x|+|y|$$ y por tanto en virtud del teorema anterior  se concluye que: $|x+y|\leq |x|+|y|$\\\\  
\end{teo}

%teorema 1.40
\begin{teo}
Si $a_1, a_2, ..., a_n$ son números reales cualesquiera $$\left|\displaystyle\sum_{k=1}^n a_k\right| \leq \sum_{k=1}^n |a_k|$$\\\\
Demostración.- \; Para $n=1$ la desigualdad es trivial y para $n=2$ es la desigualdad triangular. Supuesta cierta para $n$ números reales, para $n+1$ números reales $a_1,a_2,...,a_{n+1}$ se tiene:
$$\left| \sum\limits_{k=1}^{n+1} a_k \right|= \left| \sum\limits_{k=1}^n + a_{n+1}\right| \leq \left| \sum\limits_{k=1}^n  \right| + |a_{n+1}| \leq \sum\limits_{k=1}^n |a_k| + |a_{n+1}| = \sum\limits_{k=1}^{n+1} |a_k|$$
Por tanto, el teorema es cierto para $n+1$ números si lo es para $n;$ luego, en virtud deñ principio de inducción, es cierto para todo número positivo $n.$\\\\
\end{teo}

%teorema 1.41
\begin{teo}[Desigualdad de Cauchy-Schwarz]
Si $a_1,...,a_n$ y $b_1,...,b_n$ son números reales cualesquiera, se tiene
$$\left( \displaystyle\sum_{k=1}^n a_k b_k \right)^2 \leq \left( \sum_{k=1}^n a_k^2 \right) \left( \sum_{k=1}^n b_k^2 \right)$$
El signo de igualdad es válido si y sólo si hay un npumero real $x$ tal que $a_k x + b_k = 0$ ára cada vañpr de $k=1,2,...,n$\\\\
Demostración.- \; Para $\forall x \in \mathbb{R}$ se tiene $\sum\limits_{k=1}^n (a_kx + b_k)^2 \geq 0$. Esto se puede poner en la forma $$Ax^2 + 2Bx + C \geq 0,$$ donde $$A=\sum\limits_{k=1}^n a_k^2 , \; \; B=\sum\limits_{k=1}^n a_k b_k, \; \; C = \sum\limits_{k=1}^n b_k^2.$$ Queremos demostrar que $B^2 \leq AC.$ Si $A=0$ cada $a_k=0$, con lo que $B=0$ y el resultado es trivial. Si $A \neq 0,$ podemos completar el cuadrado y escribir $$Ax^2 + 2Bx + C = A \left( x + \dfrac{B}{A} \right)^2 + \dfrac{AC - B^2}{A}$$
El segundo miembro alcanza su valor mínimo cuando $x=-\dfrac{B}{A}$. Poniendo $x=- \dfrac{B}{A}$ en la primera ecuación, obtenemos $B^2 \leq AC.$ Esto demuestra la desigualdad dada.\\\\
\end{teo}

\section{Ejercicios}
\begin{enumerate}[ \bfseries 1.]
%----------------------------1.---------------------------------
\item Probar cada una de las siguientes propiedades del valor absoluto.\\
\begin{enumerate}[\bfseries (a)]
%(a)
\item $|x|=0$ si y sólo si $x=0$\\\\
Demostración.- \; Si $|x|=0$, por definición $x=0$. Luego, si $x=0$, entonces por teorema $\sqrt{x^2}=\sqrt{0^2}=0$\\\\

%(b)
\item $|-x|=|x|$\\\\
Demostración.- \; Por definición $|-x|=-(-x)$ si $x\leq 0$ y $|-x|=x$ si $x\geq 0$ por lo tanto $|-x|=x$\\\\ 

%(c)
\item $|x-y|=|y-x|$\\\\
Demostración.- \; Por teorema $|x-y|=\sqrt{(x-y)^2}=\sqrt{x^2-2xy+y^2}=\sqrt{y^2-2yx+x^2}=\sqrt{(y-x)^2}=|y-x|$\\\\ 

%(d)
\item $|x|^2=x^2$\\\\
Demostración.- \; Si $|x|^2$, por teorema $\left( \sqrt{x^2} \right)^2$, por propiedad de potencia $x^2$\\\\

%(e)
\item $|x|=\sqrt{x^2}$\\\\
Demostración.- \; Sea $x^2\geq 0$ entonces por teorema $\sqrt{x^2}$, luego $(x^2)^{\frac{1}{2}}$, por lo tanto por definición de valor absoluto $x=|x|$.\\\\ 

%(f)
\item $|xy|=|x||y|$\\\\
Demostración.- \; Por el teorema anterior $|xy|=\sqrt{(xy)^2}=\sqrt{x^2} \sqrt{y^2} = |x||y|$\\\\ 

%(g)
\item $|x/y|=|x|/|y|$ si $y \neq 0$\\\\
Demostración.- \; Similar al anterior problema se tiene $|x/y|=\sqrt{(x/y)^2}$, luego por propiedades de  potencia y raíces $\sqrt{x^2}/\sqrt{y^2}$, así nos queda $|x|/|y|$ si $y \neq 0$\\\\

%(h)
\item $|x-y|\leq |x|+|y|$\\\\
Demostración.- \; Por la desigualdad triangular y la parte $(b)$ se tiene $|x-y| = |x+(-y)| = |x|+ |-y| = |x|+|y|$\\\\

%(i)
\item $|x|-|y| \leq |x-y|$\\\\
Demostración.- \; Por la desigualdad triangular tenemos  que $|x|= |(x-y)+y| \leq |x-y|+|y|$ entonces $|x|-|y| \leq |x-y|$\\\\

%(j)
\item $\left| |x| - |y| \right| \leq |x-y|$\\\\
Demostración.- \; Parecido al anterior ejercicio se tiene $|y|=|x+(y-x)| \leq |x|+|y-x|$ entonces $|y|-|x| \leq |x-y|$ y por lo tanto $|x|-|y|\geq |x-y|$. Luego por el inciso (i) y teorema $-|x-y|\leq |x|-|y| \leq |x-y| \Rightarrow \left||x|-|y|\right| \leq |x-y|$\\\\ 

\end{enumerate}

%---------------------------------------------2----------------------------------------------
\item Cada desigualdad $(a_i)$, de las escritas a continuación, equivale exactamente a una desigualdad $(b_j)$. Por ejemplo, $|x|<3$ si y sólo si $-3<x<3$ y por tanto $(a_1)$ es equivalente a $(b_2)$. Determinar todos los pares equivalentes.\\
\begin{center}
\begin{tabular}{l c l}
$|x|<3$&$\longrightarrow$&$-3<x<3$\\\\
$|x-1|<3$&$\longrightarrow$&$-2<x<4$\\\\
$|3-2x|<1$&$\longrightarrow$&$1<x<2$\\\\
$|1+2x|\leq 1$&$\longrightarrow$&$-1\leq x \leq 0$\\\\
$|x-1|>2$&$\longrightarrow$&$x>4 \; \lor \; x<-1$\\\\
$|x+2| \geq 5$&$\longrightarrow$&$x\geq 3 \; \lor \; x\leq -7$\\\\
$|5-x^{-1}|<1 $&$\longrightarrow$&$4<x<6$\\\\
$|x-5|<|x+1|$&$\longrightarrow$&$x>2$\\\\
$|x^2-2|\leq 1$&$\longrightarrow$&$-\sqrt{3} \leq x \leq -1 \; ó \; 1\leq x \leq \sqrt{3}$\\\\
$x<x^2-12<4x$&$\longrightarrow$&$\dfrac{1}{6}<x<\dfrac{1}{4}$\\\\
\end{tabular}
\end{center}

%---------------------------------------3----------------------------------------
\item Decidir si cada una de las siguientes afirmaciones es cierta o falsa. En cada caso razonar la decisión.
\begin{enumerate}[\bfseries (a)]
\item $x<5$ implica $|x|<5$\\\\
Es falso ya que $|-6|<5$ entonces $6>5$.\\\\
\item $|x-5|<2$ implica $3<x<7$\\\\
Es verdad ya que por teorema $-2 < x-5 < 2$ entonces $3<x<7$.\\\\
\item $|1 + 3x| \leq 1$ implica $x > - \dfrac{2}{3}$ \\\\
es verdad ya que $-1\leq 1 + 3x \leq 1$ entonces $-\dfrac{2}{3} \leq x \leq 0$.\\\\
\item No existe número real $x$ para el que $|x-1|= |x-2|$\\\\
Es falso ya que se cumple para $\dfrac{3}{2}$.\\\\
\item Para todo $x>0$ existe un $y>0$ tal que $|2x + y|=5$\\\\
Es falso ya que si tomas $x=3$ será $y<0$\\\\
\end{enumerate}

%---------------------------------------4----------------------------------------------
\item Demostrar que el signo de igualdad es válido en la desigualdad de Cauchy-Schwarz si y sólo si existe un número real tal que $a_k x + b_k = 0$ para todo $k=1,2,...,n$\\\\
Demostración.- $(\Rightarrow)$ \; Si $a_k=0$ para todo $k$ entonces la igualdad es verdadera así asumimos que $a_k\neq 0$ para al menos un $k$, $$\left( \displaystyle\sum_{k=1}^n a_k b_k \right)^2 = \left( \sum_{k=1}^n a_k^2 \right) \left( \sum_{k=1}^n b_k^2 \right)$$, Luego, sea $\sum\limits_{k=1}^n (a_k x + b_k)^2 \geq 0$, entonces $$A=\sum\limits_{k=1}^n a_k^2, \; \; B=\sum\limits_{k=1}^n b_k a_k, \; \; C= \sum\limits_{k=1}^n b_k^2$$, así se tiene $$Ax^2+2b^x + C \geq 0 \Rightarrow x= \dfrac{-2B \pm \sqrt{4B^2 - 4AC}}{2A} = \dfrac{-B \pm \sqrt{B^2 - AC}}{A}$$ pero sabemos por suposición que $B^2=AC$, donde $x=- \dfrac{B}{A}$ el cual esta en $\mathbb{R}$ ya que $A\neq 0$ y por lo tanto $a_k \neq 0$ para algún $k$ y $a_k^2$ es no negativo. asís que la suma es estrictamente positivo como se vio en la desigualdad de Cauchy-Schwarz.\\\\
$(\Leftarrow)$ Supongamos que existe $x\in \mathbb{R}$ tal que $a_k x + b_k=0$ para cada $k=1,...,n.$ Luego $a_k x + b_k = 0 \Rightarrow b_k =(-x)a_k.$ Entonces,
\begin{center}
\begin{tabular}{r c l}
$\left( \sum\limits_{k=1}^n a_k b_k \right)$&$=$&$\left[ -x \left( \sum\limits_{k=1}^n a_k^2 \right)\right]^2$\\\\
&$=$&$x^2 \left( \sum\limits_{k=1}^n a_k^2 \right)\left( \sum\limits_{k=1}^n a_k^2 \right)$\\\\
&$=$&$\left( \sum\limits_{k=1}^n a_k^2 \right) \left( \sum\limits_{k=1}^n x^2 a_k^2 \right)$\\\\
&$=$&$\left( \sum\limits_{k=1}^n a_k^2\right) \left( \sum\limits_{k=1}^n  (-xa_k)^2\right)$\\\\
&$=$&$\left( \sum\limits_{k=1}^n a_k^2 \right)\left( \sum\limits_{k=1}^n b_k^2\right)$\\\\
\end{tabular}
\end{center}
\end{enumerate}

\section{Ejercicios varios referentes al método de inducción}
%definición coeficiente factorial y binomial
\begin{def.}[Coeficiente factorial y binomial]
El símbolo $n!$ (que se lee $n$ factorial) se puede definir por inducción como sigue: $0!=1, \; n!=(n-1!n$ si $n\geq 1$\\
Obsérvese que $n!=1\cdot 2 \cdot 3 \cdot \cdot \cdot n.$\\
Si $0\leq k \leq n$ el coeficiente binomial ${n \choose k}$ se define por 
$${n \choose k} = \dfrac{n!}{k!(n-k)!}$$
\end{def.}
\begin{enumerate}[ \bfseries 1.]

%------------------------------------------1--------------------------------------
\item Calcúlese los valores de los siguientes coeficientes binomiales:
\begin{enumerate}[\bfseries (a)]
%------------------a---------------------
\item ${5 \choose 3} = \dfrac{5!}{3!(5-3)!} = \dfrac{4\cdot 5}{2\cdot 1} = 10$\\\\

%------------------b---------------------
\item ${ 7 \choose 0 } = \dfrac{7!}{0!(7-0)!} = \dfrac{7!}{7!} = 1$\\\\

%------------------c---------------------
\item ${ 7 \choose 1 } = \dfrac{7!}{1!(7-1)!} = \dfrac{7!}{6!} = 7$\\\\

%------------------d---------------------
\item ${ 7 \choose 2 } = \dfrac{7!}{2!(7-2)!} = \dfrac{6\cdot 7}{2} = 21$\\\\

%------------------e---------------------
\item ${ 17 \choose 14 } = \dfrac{17!}{14!(17-14)!} = \dfrac{15 \cdot 16 \cdot 17}{3\cdot 2 \cdot 1} = 680$\\\\

%------------------f---------------------
\item ${ 0 \choose 0 } = \dfrac{0!}{0!(0-0)!} = 1$\\\\
\end{enumerate}

%--------------------------------2---------------------------------------------
\item 
\begin{enumerate}[\bfseries (a)]
%-------------------(a)-----------------------------
\item Demostrar que: ${n \choose k} = {n \choose n-k}$\\\\
Demostración.- \; Sea $\dfrac{n!}{(n-k)!\left[ n - (n-k) \right]!}$ entonces $\dfrac{n!}{(n-k)!k}$ por lo tanto ${n \choose k}$\\\\

%-------------------(b)-----------------------------
\item Sabiendo que ${n \choose 10} = {n \choose 7}$ calcular $n$\\\\
Respuesta.- \;  Usando la parte $(a)$ sabemos que $k=10$ y $n-k=7$ por lo tanto $n=17$\\\\

%-------------------(c)-----------------------------
\item Sabiendo que ${14 \choose k} = {14 \choose k - 4}$ calcular $k$\\\\ 
Respuesta.- \; Similar a la parte $(b)$ $k=14 - (k-4) \Rightarrow 2k=18 \Rightarrow k =9$\\\\

%-------------------(d)-----------------------------
\item ¿Existe un $k$ tal que ${12 \choose k} = {12 \choose k-3}$\\\\
Respuesta.- \; No existe ya que $k=12 - (k-3) \Rightarrow 2k=15$ no es un entero.\\\\
\end{enumerate}

%-------------------------------------3------------------------------------------
\item Demostrar que ${n+1 \choose k} = {n \choose k-1} + {n \choose k}.$ Esta propiedad se denomina fórmula aditiva de los coeficientes combinatorios o ley del triángulo de Pascal y proporciona un método rápido para calcular sucesivamente los coeficientes binomiales. A continuación se da el triángulo de Pascal para $n \leq 6.$
\begin{center}
\begin{tabular}{>{$n=}l<{$\hspace{12pt}}*{13}{c}}
0 &&&&&&&1&&&&&&\\
1 &&&&&&1&&1&&&&&\\
2 &&&&&1&&2&&1&&&&\\
3 &&&&1&&3&&3&&1&&&\\
4 &&&1&&4&&6&&4&&1&&\\
5 &&1&&5&&10&&10&&5&&1&\\
6 &1&&6&&15&&20&&15&&6&&1\\\\
\end{tabular}
\end{center}
Demostración.- \; 
\begin{center}
\begin{tabular}{r c l}
${n \choose k-1}$&$=$&$\dfrac{n!}{(k-1)![n-(k-1)]!} + \dfrac{n!}{k!(n-k)!}$\\\\
&$=$&$\dfrac{(n!)k + (n!)(n-k+1)}{k!(n-k+1)!}$\\\\
&$=$&$\dfrac{(n!)(k+n-k+1)}{k!(n+1-k)!}$\\\\
&$=$&$\dfrac{(n!(n+1))}{k!(n+1-k)!}$\\\\
&$=$&$\dfrac{(n+1)!}{k![(n+1)-k]!}$\\\\
&$=$&${n+1 \choose k}$\\\\
\end{tabular}
\end{center}

%-------------------------------------4-----------------------------------------
\item Demuéstrese por inducción la fórmula de la potencia del binomio:
$$(a+b)^n = \displaystyle\sum_{k=0}^n {n \choose k} a^k b^{n-k}$$
Y utilícese el teorema para deducir las fórmulas:
$$\displaystyle\sum_{k=0}^n {n \choose k} = 2^n \; \; y \; \; \sum_{k=0}^n (-1)^k {n \choose k} = 0 \; \; si \; \; n>0$$\\\\
Demostración.- \; Sea $n=1$ entonces $$(a+n)^1 = a + b = \sum\limits_{k=0}^1 {n \choose k} a^k b^{n-k} = a^0 b + a b^0 = b + a = a + b$$
Por lo tanto la formula es cierta para $n=1$\\ 
Luego supongamos que la formula es cierta para algún $n=m \in \mathbb{Z}^+$ entonces,
$$(a+b)^m = \sum\limits_{k=0}^m {m \choose k} a^k b^{m-k}$$ Luego suponemos que se cumple para $m+1$
$$(a+b)^{m+1} = \sum\limits_{k=0}^{m+1} {m+1 \choose k} a^k b^{m+1-k}$$
\begin{center}
\begin{tabular}{r c l}
$(a+b)^{m+1}$&$=$&$\left[ \sum\limits_{k=0}^m {m \choose k} a^k b^{m-k} \right] (a+b)$\\\\
&$=$&$\sum\limits_{k=0}^m {m \choose k} a^{k+1} b^{m-k} + \sum\limits_{k=0}^m {m \choose k}a^k b^{m+1-k}$\\\\
&$=$&$a^{m+1} + \sum\limits_{k=0}^{m-1} {m \choose k} a^{k+1} b^{m-k} + \sum\limits_{k=0}^m {m \choose k}a^k b^{m+1-k}$\\\\
&$=$&$a^{m+1} + \sum\limits_{k=1}^{m} {m \choose k-1} a^k b^{m+1-k} + \sum\limits_{k=0}^{m} a^k b^{m+1-k}$\\\\
&$=$&$a^{m+1} + b^{m+1} + \sum\limits_{k=1}^{m} {m \choose k-1} a^k b^{m+1-k} + \sum\limits_{k=0}^m {m \choose k} a^k b^{m+1-k}$\\\\
&$=$&$a^{m+1} + b^{m+1} + \sum\limits_{k=1}^m \left\lbrace \left[ {m \choose k-1} + {m \choose k} \right] \left( a^k b^{m+1-k} \right)\right\rbrace$\\\\
&$=$&$a^{m+1} + b^{m+1} + \sum_{k=1}^m {m+1 \choose k} a^k b^{m+1-k}$\\\\
&$=$&$\sum\limits_{k=0}^{m+1} {m+1 \choose k} a^k b^{m+1-k}$\\\\
\end{tabular}
\end{center}
Por lo tanto, si la fórmula es verdadera para el caso $m$ entonces es verdadera para el caso $m+1$.\\\\
Luego aplicando el teorema del binomio con $a=1, \; b=1$, entonces,
$$(a+b)^n = \sum\limits_{k=0}^n a^k b^{n-k} \Rightarrow (1+1)^n = 2^n = \sum\limits_{k=0}^n {n \choose k}$$
Para la segunda fórmula aplicamos una vez mas pero con $a=-1$ y $b=1$, entonces,
$$(a+b)^n = (-1+1)^n = 0^n = 0 = \sum\limits_{k=0}^n {n \choose k} (-1)^k$$\\\\

%definición 4.15
\begin{def.}[Símbolo producto] El producto de $n$ números reales $a_i, a_2,...,a_n$ se indica por el símbolo $\prod\limits_{k=1}^n a_k,$ que se puede definir por inducción. El símbolo $a_1 a_2 \cdot \cdot \cdot a_n$ es otra forma de escribir este producto. Obsérvese que:
$$n! = \displaystyle\prod_{k=1}^n k$$\\
\end{def.}

%-------------------------------------------5---------------------------------------
\item Dar una definición por inducción del producto $\displaystyle\prod_{k=1}^n a_k$\\\\
Definición.- \; $$\prod\limits_{k=1}^0 a_k = 1; \; \; \prod\limits_{k=1}^{n+1} a_k = a_{n+1} \cdot  \prod\limits_{k=1}^n a_k$$\\\\

Demostrar por inducción las siguientes propiedades de los productos:
%-------------------------------------------6---------------------------------------
\item $\displaystyle \prod_{k=1}^n (a_k b_k) = \left( \prod_{k=1}^n a_k \right) \left( \prod_{k=1}^n b_k \right)$ (Propiedad multiplicativa)\\
Un caso importante es la relación: $\displaystyle\prod_{k=1}^n (ca_k) = c_n \prod_{k=1}^n a_k$\\\\
Demostración.- \; Sea $n=1$ entonces $a_1 b_1 = a_1 b_1$. Supongamos que se cumple para $n = m \in \mathbb{Z}^+$, $$\prod\limits_{k=1}^m (a_k b_k) = \left( \prod\limits_{k=1}^m a_k \right) \left( \prod\limits_{k=1}^m b_k \right)$$
Luego sea $m+1$, por lo tanto $$\prod\limits_{k=1}^{m+1} (a_k b_k) = \left( \prod\limits_{k=1}^{m+1} a_k \right) \left( \prod\limits_{k=1}^{m+1} b_k \right)$$
Así, \\
\begin{center}
\begin{tabular}{r c l}
$\prod\limits_{k=1}^{m+1} (a_k b_k)$&$=$&$\left( \prod\limits_{k=1}^m a_k \right) \cdot a_{m+1} \left( \prod\limits_{k=1}^m b_k \right) \cdot b_{m+1}$\\\\
&$=$&$\left( \prod\limits_{k=1}^{m+1} a_k \right) \left( \prod\limits_{k=1}^{m+1} b_k \right)$\\\\
\end{tabular}
\end{center}
El caso $m+1$ es cierto por lo tanto la propiedad es válida para todo $n\in \mathbb{Z}^+$\\\\

%------------------------------------------7-----------------------------------------
\item $\displaystyle\prod_{k=1}^n \dfrac{a_k}{a_{k-1}} = \dfrac{a_n}{a_0}$ si cada $a_k\neq 0$ (propiedad telescópica)\\\\
Demostración.- \; Sea $n=1$ entonces $\prod\limits_{k=1}^1 \dfrac{a_k}{a_{k-1}} = \dfrac{a_n}{a_0} \Rightarrow \dfrac{a_1}{a_0} = \dfrac{a_1}{a_0}$ si $a_k \neq 0$. De ésta manera se cumple para $n=1$.\\
Luego supongamos que se cumple para $n = m \in \mathbb{Z}^+ $ así nos queda, $$\prod\limits_{k=1}^m \dfrac{a_k}{a_{k-1}} = \dfrac{a_m}{a_0}$$
Ahora supongamos que es cierto para $m+1$, luego $$\prod\limits_{k=1}^{m+1} \dfrac{a_k}{a_{k-1}} = \dfrac{a_{m+1}}{a_0}$$ de donde,\\
\begin{center}
\begin{tabular}{rcl}
$\prod\limits_{k=1}^{m+1} \dfrac{a_k}{a_{k-1}}$&$=$&$\dfrac{a_{m+1}}{a_m} \cdot \prod\limits_{k=1}^{m} \dfrac{a_k}{a_{k-1}}$\\\\
&$=$&$\dfrac{a_{m+1}}{a_m} \cdot \dfrac{a_m}{a_0}$\\\\
&$=$&$\dfrac{a_{m+1}}{a_0}$\\\\
\end{tabular}
\end{center}
Vimos que el caso $m+1$ es cierto, por lo tanto la propiedad es válida para $\forall n \in \mathbb{Z}^+$\\\\

%----------------------------------------8---------------------------------------------
\item Si $x\neq 1,$ demostrar que: $\displaystyle\prod_{k=1}^n (1 + x^{2^{k-1}}) = \dfrac{1 - x^{2^n}}{1-x}$\\
¿Cuál es el valor del producto cuando $x=1$?\\\\
Demostración.- \; Se cumple la condición para $n=1$ ya que $\prod\limits_{k=1}^1 (1 + x^{2^{k-1}}) = \dfrac{1 - x^{2^{n}}}{1-x} \Rightarrow 1 + x = \dfrac{1 - x^2}{1-x} = \dfrac{(1-x)(1+x)}{1-x} = 1+x$.\\
Supongamos que se cumple para $n= m \in \mathbb{Z}^+$ entonces $$\prod\limits_{k=1}^m (1 + x^{2^{k-1}}) = \dfrac{1 - x^{2^m}}{1-x}$$, luego se cumple para $m+1$, $$\prod\limits_{k=1}^{m+1} (1 + x^{2^{k-1}}) = \dfrac{1 - x^{2^{m+1}}}{1-x}$$, por lo tanto nos queda,\\
\begin{center}
\begin{tabular}{rcl}
$\prod\limits_{k=1}^{m+1} (1 + x^{2^{k-1}})$&$=$&$(1 + x^{2^{m}})\cdot \prod\limits_{k=1}^{m} (1 + x^{2^{k-1}})$\\\\
&$=$&$(1 + x^{2^{m}})\cdot \dfrac{1 - x^{2^m}}{1-x}$\\\\
&$=$&$\dfrac{(1+x^{2^m})(1-x^{2^m})}{1-x}$\\\\
&$=$&$\dfrac{1 - x^{2^{m+1}}}{1-x}$\\\\
\end{tabular}
\end{center}
Por lo tanto se cumple para $\forall n \in \mathbb{Z}^+$\\\\

%-----------------------------------------9----------------------------------------------
\item Si $a_k < b_k$ para cada valor de $k=1,2,...,n$, es fácil demostrar por inducción $\displaystyle\sum_{k=1}^n < \sum_{k=1}^n b_k.$\\
Discutir la desigualdad correspondiente para productos:
$$\displaystyle\prod_{k=1}^n a_k < \prod_{k=1}^n b_k$$\\\\
Demostración.- \; Es fácil ver que es la afirmación es cierta para $n=1$. Supongamos que es cierto para $n=m \in \mathbb{Z}^+$, luego por hipótesis sabemos que $b_{m+1}>a_{m+1} \geq 0$ de donde, $$\prod\limits_{k=1}^{m+1} a_k = a_{m+1} \cdot \prod\limits_{k=1}^m a_k < a_{m+1} \cdot \prod\limits_{k=1}^m b_k<b_{m+1} \cdot \prod\limits_{k=1}^m b_k = \prod\limits_{k=1}^{m+1} b_k$$\\\\

Algunas desigualdades notables\\\\
%---------------------------------------10----------------------------------------------
\item Si $x>1,$ demostrar por inducción que $x^n >x$ para cada $n\geq 2.$ Si $0<x<1$, demostrar que $x^n<x$ para cada $x \geq 2.$\\\\
Demostración.- \; Es fácil probar que la afirmación se cumple para $n=2$. Supongamos que se cumple para alguna $n=m \geq 2$, así $$x^m \geq x,$$ de donde $m+1$ se cumple para $$x^{m+1}\geq x,$$ así solo nos queda probar que $$x \cdot x \geq x,$$ el cuál se cumple a simple vista.\\\\
Por otra parte sea $n=2$ por lo tanto $x^2<x$, como $0<x<1$ vemos se cumple la desigualdad. Supongamos que se cumple para $n=m \geq 2$, entonces,
\begin{center}
\begin{tabular}{rcl}
$x^m<x$&$\Rightarrow$&$x^m < x\cot x$\\
&$\Rightarrow$&$x^{m+1}<x^2$\\
&$\Rightarrow$&$x^{m+1}<x^2<x$\\
&$\Rightarrow$&$x^{m+1}<x$\\
\end{tabular}
\end{center}
Así la desigualdad es válida para $m+1 \in \mathbb{Z}_{\geq 2}$\\\\


%---------------------------------------11-------------------------------------------------
\item Determínense tofos los enteros positivos $n$ para los cuales $2^n < n!$\\\\
Demostración.- \; Podemos observar que no se cumple para $n=1,2,3$. Luego observemos que la afirmación es válida para $n=4$,   $$2^4 < 4! \Rightarrow 16<24.$$
Ahora demostremos que la afirmación es válida para $n=m+1$ suponiendo que se cumple para algún $n=m \in \mathbb{Z}_{\geq 4}$, entonces,
$$2^m < m! \Longrightarrow 2^{m} (m+1) < m! (m+1)! \Longrightarrow 2^{m+1} < (m+1)! $$
Ya que $m\geq 4 > 2$ entonces $2^m(m+1) > 2^m \cdot 2 = 2^{m+1}$. Así la inecuación es verdadera para $m+1 \; \forall n \in \mathbb{Z}_{\geq 4}$\\\\

%--------------------------------------12---------------------------------------------------
\item 
\begin{enumerate}[\bfseries (a)]
%---------------------------------(a)--------------------------------
\item Con el teorema del binomio demostrar que para $n$ entero positivo se tiene 
$$\left( 1 + \dfrac{1}{n}  \right)^n = 1 + \displaystyle\sum_{k=1}^n \left[ \dfrac{1}{k!} \prod_{r=0}^{k-1} \left( 1 - \dfrac{r}{n} \right) \right]$$\\\\
Demostración.- \; Sea $a=\dfrac{1}{n}$ y $b=1$ tenemos,
\begin{center}
\begin{tabular}{rcll}
$\left( \dfrac{1}{n} + 1 \right)^n$&$=$&$\sum\limits_{k=0}^n {n \choose k} \left( \dfrac{1}{n} \right)^k$&\\\\
&$=$&$1 + \sum\limits_{k=1}^n \dfrac{n!}{k!(n-k)!} \left( \dfrac{1}{n} \right)^k$&\\\\
&$=$&$1+ \sum\limits_{k=1}^n \dfrac{1}{k!} \left[ \dfrac{\prod\limits_{r=1}^n r}{\prod\limits_{r=1}^{n-k} r}\cdot \left( \dfrac{1}{n} \right)^k \right]$&\\\\
&$=$&$1 + \sum\limits_{k=1}^n \dfrac{1}{k!} \left( \prod\limits_{r=n-k+1}^n r \right) \left( \dfrac{1}{n} \right) ^k$&\\\\
&$=$&$1 + \sum\limits_{k=1}^n \dfrac{1}{k!} \left[ \prod\limits_{r=0}^{k-1} (n-r) \right] \left( \prod\limits_{r=0}^{k-1} \dfrac{1}{n} \right)$&\\\\
&$=$&$1 + \sum\limits_{k=1}^n \dfrac{1}{k!} \left[ \prod\limits_{r=0}^{k-1} \left( 1 - \dfrac{r}{n} \right) \right] $&\\\\
\end{tabular}
\end{center}

%---------------------------------(b)-----------------------------
\item Si $n>1,$ aplíquese la parte $(a)$ y el Ejercicio 11 para deducir las desigualdades 
$$2 < \left( 1 + \dfrac{1}{n} \right)^n < 1 + \displaystyle\sum_{k=1}^n \dfrac{1}{k!} < 3 $$ \\\\
Demostración.- \; Si $n>1$ y $n\geq 2$ por el teorema bonomial tenemos:
$$\left( 1 + \dfrac{1}{n} \right)^n = \sum\limits_{k=0}^n {n \choose k} \left( \dfrac{1}{n} \right) ^k = 1 + n\left( \dfrac{1}{n} \right) + \sum\limits_{k=2}^n {n \choose k} \left( \dfrac{1}{n} \right)^k > 2$$
Donde la desigualdad es estricta ya que $\sum\limits_{k=2}^n {n \choose k} \left( \dfrac{1}{n} \right)^k $ para $n \geq 2.$\\
Luego por la parte (a) sabemos que $$\left( 1 + \dfrac{1}{n} \right) ^n = 1 + \sum\limits_{k=1}^n \dfrac{1}{k!} \left[ \prod\limits_{r=0}^{k-1} \left( 1 - \dfrac{r}{n} \right) \right] < 1 + \sum\limits_{k=1}^n \dfrac{1}{k!}$$
ya que si $n>1$ entonce $\left( 1 - \dfrac{r}{n} \right) < 1,$ para todo $ r = 0,...,n-1$ por lo tanto $\prod\limits_{r=0}^{k-1} \left( 1 - \dfrac{r}{n} \right) < \prod\limits{r=0}^{k-1} 1 = 1$ y en consecuencia $\dfrac{1}{k!} \left[ \prod\limits[{r=0}^{k-1} \right) \left( 1 - \dfrac{r}{n} \right) < \dfrac{1}{k!}$\\
Por último demostraremos que $1+ \sum\limits_{k=1}^n \dfrac{1}{k!} < 3$, 
\begin{center}
\begin{tabular}{rcll}
$1 + \sum\limits{k=1}^n \dfrac{1}{k!}$&$=$&$1+ 1 +\dfrac{1}{2} + \dfrac{1}{6} + \sum\limits_{k=4}^n \dfrac{1}{k!}$&\\\\
&$<$&$\dfrac{8}{3} + \sum\limits_{k=4}^n \dfrac{1}{2^k}$&por el problema 11\\\\
&$=$&$\dfrac{8}{3} + \dfrac{1}{16} \left( \sum\limits_{k=0}^{n-4} \dfrac{1}{2^k} \right)$&\\\\
&$=$&$\dfrac{8}{3} + \dfrac{1}{16} \left( 2 - \dfrac{1}{2^{n-4}} \right)$&por el problema 8\\\\
&$=$&$\dfrac{8}{3} + \dfrac{1}{8} - \dfrac{1}{2^n} $&\\\\
&$<$&$3$&\\\\
\end{tabular}
\end{center}
\end{enumerate}

%-----------------------------------13--------------------------------------
\item 
\begin{enumerate}[\bfseries (a)]
%--------------------------(a)----------------------------
\item Sea $p$ un entero positivo. Demostrar que:
$$b^p - a^p = (b-a)(b^{p-1} + b^{p-2}a + b^{p-3}a^2 + ... + ba^{p-2} + a^{p-1})$$\\\
Demostración.- \; Sea $(b-a)(b^{p-1} + b^{p-2}a + b^{p-3}a^2 + ... + ba^{p-2} + a^{p-1})$ entonces,
\begin{center}
\begin{tabular}{rcl}
&$=$&$(b-a)\left( \sum\limits_{k=0}^{p-1} b^{p-k-1} a^k \right)$\\\\
&$=$&$b \left( \sum\limits_{k=0}^{p-1} b^{p-k-1} a^k \right) - a\left( \sum\limits_{k=0}^{p-1} b^{p-k-1} a^k \right)$\\\\
&$=$&$ \sum\limits_{k=0}^{p-1} b^{p-k} a^k - \sum\limits_{k=0}^{p-1} b^{p-k} a^{k-1}  $\\\\
&$=$&$b^p + \sum\limits_{k=1}^{p-1} b^{p-k}a^k - a^p - \sum\limits_{k=1}^{p-1} b^{p-k}a^k$\\\\
&$=$&$b^p - a^p$\\\\
\end{tabular}
\end{center}

%----------------------(b)---------------------------
\item Si $p$ y $n$ son enteros positivos, demostrar que $$n^p < \dfrac{(n+1)^{p+1} - n^{p+1}}{p+1} < (n+1)^p$$\\\\
Demostración.- \; Por el teorema binomial se tiene:
\begin{center}
\begin{tabular}{rcl}
$\dfrac{(n+1)^{p+1} - n^{p+1}}{p+1}$&$=$&$\left( \dfrac{1}{p+1} \right) \left[ \left( \sum\limits_{k=0}^{p+1} {p+1 \choose k} n^k \right) - n^{p+1} \right]$\\\\
&$=$&$\left( \dfrac{1}{p+1} \right) \left[ n^{p+1} + (p+1)n^p + \left( \sum\limits_{k=0}^{p-1} {p+1 \choose k} n^k \right) - n^{p+1} \right]$\\\\
&$=$&$n^p + \left( \dfrac{1}{p+1} \right) \left( {p+1 \choose k} n^k \right)$\\\\
&$>$&$n^p$\\\\
\end{tabular}
\end{center}
Luego para la desigualdad de la derecha usaremos la parte $(a)$, de la siguiente manera:
\begin{center}
\begin{tabular}{rcl}
$\dfrac{(n+1)^{p+1} - n^{p+1}}{p+1}$&$=$&$\left( \dfrac{1}{p+1} \right)\left[ (n+1)^p + n(n+1)^{p-1} + ... + n^{p-1}(n+1) + n^p \right]$\\\\
&$<$&$\left( \dfrac{1}{p+1} \right)\left[ (n+1)^p + (p+1)^p + ... + (n+1)^p + (n+1)^p \right]$\\\\
&$=$&$\left( \dfrac{1}{p+1} \right) \left[ (p+1)(n+1)^p \right]$\\\\
&$=$&$(n+1)^p$\\\\
\end{tabular}
\end{center}

%------------------------(c)--------------------------------
\item Demuéstrese por inducción que:
$$\displaystyle\sum_{k=1}^{n-1} k^p < \dfrac{n^{p+1}}{p+1} < \sum_{k=1}^n k^p$$\\\\
Demostración.- \; Sea $n=1$ entonces $\sum\limits_{k=1}^{0} k^p = 0 < \dfrac{1^{p+1}}{p+1}< \sum\limits_{k=1}^1 k^p =1$ por lo tanto la inecuación se cumple. Luego asumimos que es verdad para algún $n=m \in \mathbb{Z}^{>0}$. Así para la inecuación de la izquierda se tiene:
$$\sum\limits_{k=1}^{m+1} k^p < \dfrac{m^{p+1}}{p+1}$$ Luego,
\begin{center}
\begin{tabular}{rcl}
$\sum\limits_{k=1}^m k^p$&$<$&$\dfrac{m^{p+1}}{p+1} + m^p$\\\\
&$<$&$\dfrac{m^{p+1}}{p+1} + \dfrac{(m+1)^{p+1} - m^{p+1}}{p+1}$\\\\
&$=$&$\dfrac{(m+1)^{p+1}}{p+1}$\\\\
\end{tabular}
\end{center}
Ahora veamos para la inecuación de la derecha. Asumiendo que es verdad para algún $n= m \in \mathbb{Z}_{>0},$ tenemos:
$$\dfrac{m^{p+1}}{p+1} < \sum\limits_{k=1}^m k^p$$, entonces
\begin{center}
\begin{tabular}{rcl}
$\dfrac{m^{p+1}}{p+1} + (m+1)^p$&$<$&$\sum\limits_{k=1}^{m+1} k^p$\\\\
$\dfrac{m^{p+1} + (m+1)^{p+1} - m^{p+1}}{p+1}$&$<$&$\sum\limits_{k=1}^{m+1} k^p$\\\\
$\dfrac{(m+1)^{p+1}}{p+1}$&$<$&$\sum\limits_{k=1}^{m+1} k^p$\\\\
\end{tabular}
\end{center}
Así vemos que se cumple para todo $n \in \mathbb{Z}^{>0}$\\\\
\end{enumerate}

%---------------------------------------------14----------------------------------------------------
\item Sean $a_1,...,a_n$ $n$ números reales, todos del mismo signo y todos mayores que $-1$. Aplicar el método de inducción para demostrar que:
$$(1+a_1)(1+a_2)\cdot \cdot \cdot (1+ a_n) \geq 1 + a_1 + a_2 + ... + a_n$$
en particular, cuando $a_1=a_2=...=a_n=x$, donde $x>-1$, se transforma en:
$$(1+x)^n \geq 1+nx \; \; \mbox{(desigualdad de Bernoulli)}$$
Probar que si $n>1$ el signo de igualdad se presenta en (1.25) sólo para $x=0$\\\\
Demostración.- \; Sea $n=1$ entonces $1+a_1 \geq 1 + a_1$ por lo que la desigualdad es válida. Ahora supongamos que la desigualdad es válida para algún $n=k \in \mathbb{Z}_{>0}$. así,
$$(1+a_1)...(1+ a_k) \geq 1 + a_1 + ... + a_k$$ de donde,
\begin{center}
\begin{tabular}{rcl}
$(1+a_1)...(1+a_k)(1+a_{k+1})$&$\geq$&$(1+a_1 + .. + a_k)(1+a_{k+1})$\\
&$\geq$&$(1+a_1 + ... + a_{k+1})+ a_{k+1} (a_1+...+a_k)$\\
\end{tabular}
\end{center}
dado que $a_i$ deben ser del mismo signo por lo tanto $a_{k+1}$ y $(a_1+...+a_k)$ debe ser positivo. Así, $$(1+a_1)\cdot \cdot \cdot (1+a_{k+1}) \geq 1 + a_1 + ...+ a_{k+1}$$ que cumple la desigualdad para $n \in \mathbb{Z}_{>0}$\\
Vemos ahora la desigualdad de bernoulli. Si $x=0$ entonces $(1+0)^n = 1 = 1 + n\cdot 0$, por lo tanto se cumple la igualdad si sólo si $x=0$\\
Para el caso de $n=2$ tenemos $(1+x)^2 0 1+2x +x^2 > 1 + 2x$ entonces la desigualdad se cumple para $n=2$. Supongamos ahora que la desigualdad es estricta para algún $n=k \in \mathbb{Z}_{>1}$ por lo tanto$$(1+x)^k > 1 + kx$$ así,
\begin{center}
\begin{tabular}{rcl}
$(1+x)^k (1+x)$&$>$&$(1+kx)(1+x)$\\
$(1+x)^{k+1}$&$>$&$(k+1)x + kx^2$\\
$(1+x)^{k+1}$&$>$&$1 + (k+1)x$\\
\end{tabular}
\end{center}
Ya que $k>0$ y $x>0$ implica que $kx^2>0$ por lo tanto la desigualdad es estricta para todo $n>1$ si $x\neq 0$. Por lo tanto, la igualdad es válida si y sólo si $x=0$\\\\

%------------------------------------------------15-------------------------------------------------
\item Si $n \geq 2,$ demostrar que $n!/n^n \leq \left( \dfrac{1}{2} \right)^k,$ siendo $k$ la parte entera de $n/2.$\\\\
Demostración.- \; Demostremos por inducción. Sea $n=2$ entonces $$\dfrac{2!}{2^2} = \dfrac{1}{2} = \left( \dfrac{1}{2} \right)^2 = \dfrac{1}{2}$$ ya que $n/2 = 2/2 = 1 = k$ siendo $k$ la parte entera de $n/2$. Supongamos que la desigualdad se cumple para algún $n=m \in \mathbb{Z} \geq 2$. Entonces,
\begin{center}
\begin{tabular}{rcl}
$\dfrac{m!}{m^m} \leq \left( \dfrac{1}{2} \right)^k$ & $\Rightarrow$ & $\left( \dfrac{m!}{m^m} \right)\left[ \dfrac{(m+1)m^m}{(m+1)^{m+1}} \right] \leq \left( \dfrac{1}{2} \right)^k  \left[ \dfrac{(m+1)m^m}{(m+1)^{m+1}} \right]$\\\\
&$\Rightarrow$&$\dfrac{(m+1)!}{(m+1)^{m+1}} \leq \left( \dfrac{1}{2} \right)^k \left( \dfrac{m}{m+1} \right)^m $\\\\
\end{tabular}
\end{center} 
Luego por el problema $13$ se tiene $$m^p < \dfrac{(m+1)^{p+1} - m^{p+1}}{p+1} \;\; \forall \; p \in \mathbb{Z}^{+}$$
Supongamos que $p=m-1$ entonces,
\begin{center}
\begin{tabular}{rcll}
$(p+1)m^p$&$\Rightarrow$&$(m+1)^{p+1} - m^{p+1}$&\\
&$\Rightarrow$&$m^{p+1} + (p+1)m^p < (m+1)^{p+1}$&\\
&$\Rightarrow$&$m^m + m^m < (m+1)^m$& ya que $p=m-1$\\
&$\Rightarrow$&$2m^m < (m+1)^m$\\
&$\Rightarrow$&$\left( \dfrac{m}{m+1}\right) ^m < \dfrac{1}{2}$\\
\end{tabular}
\end{center}
Así nos queda que $$\dfrac{(m+1)!}{(m+1)^{m+1}} \leq \left( \dfrac{1}{2} \right)^k \left( \dfrac{m}{m+1}\right)^m \leq \left( \dfrac{1}{2} \right)^{k+1}$$
Por último recordemos que $k$ es la parte entera de $m/2$, para completar la demostración debemos demostrar que si la desigualdad se  cumple para $m$ entonces e cumplirá para $m+1$, en efecto si $m$ es par, entonces $k=m/2$ y $k+1 = (m+1)/2$, por otro lado si $m$ es impar entonces $k=(m+1)/2$. En cualquier caso $k+1  \geq (m+1)/2$ entonces $$\dfrac{(m+1)!}{(m+1)^{m+1}} \leq \left(\dfrac{1}{2} \right)^2 \leq \left( \dfrac{1}{2} \right) ^{\dfrac{m+1}{2}}$$
Por lo tanto la desigualdad es válida para el caso $m+1$ y en consecuencia es cierta para $n \in \mathbb{Z}_{\geq 2}$\\\\

%--------------------------------------------16-----------------------------------------------
\item Los números $1,2,3,5,8,13,21,...$ tales que cada uno después del segundo es la suma de los dos anteriores, se denomina números de Fibonacci. Se pueden definir por inducción como sigue:
$$a_1=1, \; \; a_2=2, \; \; a_{n+1} = a_n + a_{n-1}, \; \; si \; n \geq 2.$$
Demostrar que $$a_n < \left( \dfrac{1 + \sqrt{5}}{2} \right)^n$$
para cada $n \geq 1.$\\\\
Demostración.- \; Para el caso de $n=1$ tenemos $$1 < \dfrac{1+\sqrt{5}}{2}$$ ya que $\sqrt{5}>2$ se cumple la desigualdad para $n=1$. Ahora supongamos que es verdad para algún $n= k \in \mathbb{Z}_{\geq 1}$, luego 
\begin{center}
\begin{tabular}{crcll}
&$a_k + a_{k-1}$&$<$&$ \left( \dfrac{1+\sqrt{5}}{2} \right)^k +  \left( \dfrac{1+\sqrt{5}}{2} \right)^{k-1}$&\\\\
$\Rightarrow$&$a_{k+1}$&$<$&$\left( \dfrac{1 + \sqrt{5}}{2} \right)^{k-1} \cdot \left( \dfrac{1 + \sqrt{5}}{2} + 1 \right)$&\\\\ 
&&$=$&$\left( \dfrac{1 + \sqrt{5}}{2} \right)^{k-1} \cdot \left( \dfrac{3 + \sqrt{5}}{2} \right)$&\\\\
&&$=$&$\left( \dfrac{1 + \sqrt{5}}{2} \right)^{k-1} \cdot \left( \dfrac{1 + \sqrt{5}}{2} \right)^2$&ya que $\left( \dfrac{3 + \sqrt{5}}{2} \right) = \left( \dfrac{1+\sqrt{5}}{2}\right)^2$\\\\
&&$=$&$\left( \dfrac{1 + \sqrt{5}}{2} \right)^{k-1}$&\\\\
\end{tabular}
 
\end{center}
Por lo tanto la desigualdad es válida para $n \in \mathbb{Z}_{\geq 1}$\\\\

\begin{def.}[Desigualdades que relacionan distintos tipos de promedios]
Sean $x_1,x_2,...,x_n$ $n$ números reales positivos. Si $p$ es un entero no nulo, la media de potencias p-énesimas $M_p$ se define como sigue.
$$M_p = \left( \dfrac{x_1^{p} + ... + x_{n}^{p}}{n} \right)^{1/p}$$
El número $M_1$ se denomina media aritmética, $M_2$ media cuadrática y $M_{-1}$ media armónica.\\
\end{def.}
\vspace{1cm}
%---------------------------------17------------------------------------
\item Si $p>0$ demostrar que $M_p< M_{2p}$ cuando $x_1,x_2,...,x_n$ no son todos iguales.\\\\
Demostración.- \; Sea $a_k=x^p_k$ y $b_k=1$ entonces por la desigualdad de Cauchy-Schwarz, $$\left( \sum\limits_{k=1}^n x_k^p \cdot 1 \right)^2 \leq \left( \sum\limits_{k=1}^n a_k^{2p} \right) \left( \sum\limits_{k=1}^n 1^2 \right) $$ luego, vemos que la desigualdad es estricta ya que si se sostuviera la igualdad existiría alguna $y\in \mathbb{R}$ tal que $x_k^p \cdot y + 1 = 0, \; \forall k$, pero esto implicaría que $x_k = \left( -\dfrac{1}{y} \right)^{1/p}, \; \forall k,$ contradiciendo nuestro supuesto de que $x_k$ no todos son iguales. Luego sabemos que $\sum\limits_{k=1}^n 1 = n$ entonces
\begin{center}
\begin{tabular}{rcll}
$\sum\limits_{k=1}^n x_k^p$&$<$&$\left( \sum\limits_{k=1}^n a_k^{2p} \right)^{1/2} \cdot n^{1/2}$&\\\\
$\left( \sum\limits_{k=1}^n x_k^p \right)^{1/p}$&$<$&$ \left( \sum\limits_{k=1}^n a_k^{2p} \right)^{1/2p} \left( \dfrac{1}{n}\right)^{1/2p}$&elevado por $1^{1/p}$\\\\
$\left( \dfrac{1}{p} \right)^{1/p} \left( \sum\limits_{k=1}^n x_k^p \right)^{1/p}$&$<$&$\left( \sum\limits_{k=1}^n a_k^{2p}\right)^{1/2p}  \left( \dfrac{1}{n}\right)^{1/2p}$&por $\left( \dfrac{1}{n}\right)^{1/2}$\\\\
$\left( \dfrac{\sum\limits_{k=1}^n x_k^p}{n} \right)^{1/p}$&$<$&$\left( \dfrac{\sum\limits_{k=1}^{2p} x_k^{2p}}{n} \right)^{1/2p}$&\\\\
\end{tabular}
\end{center}

%-------------------------------------18------------------------------------
\item Aplíquese	el resultado del Ejercicio 17 para demostrar que
$$a^4 + b^4 + c^4 \geq \dfrac{64}{3}$$ si $a^2 + b^2 +c^2 = 8$ y $a, \; b, \;c>0.$\\\\
Demostración.- \; Sea $a + b + c \in \mathbb{R}$ distintos entre si, aplicamos el problema 17 con $p=2$, 
\begin{center}
\begin{tabular}{rcl}
$\left( \dfrac{a^2 + b^2 + c^2}{3} \right)^{1/2}< \left( \dfrac{a^2 + b^2 + c^2}{3} \right)^{1/4}$&$\Rightarrow$&$\left(\dfrac{8}{3}\right)^2 < \dfrac{a^4 + b^4 + c^4}{3}$\\\\
&$\Rightarrow$&$a^4+b^4+c^4 > \dfrac{64}{3}$\\\\
\end{tabular}
\end{center} 
Luego si $a=b=c,$ entonces 
\begin{center}
\begin{tabular}{rcl}
$a^2+b^2+c^2$&$\Rightarrow$&$3a^2=8$\\\\
&$\Rightarrow$&$a^2=b^2=c^2=\dfrac{8}{3}$\\\\
&$\Rightarrow$&$a^4=b^4=c^4 = \dfrac{64}{9}$\\\\
&$\Rightarrow$&$a^4+b^4+c^4 = \dfrac{64}{3}$\\\\
\end{tabular}
\end{center}
Por lo tanto la desigualdad se cumple para cualquier $a,b,c \in \mathbb{R}$\\\\

%---------------------------------19-----------------------------------
\item Sean $a_1,...,a_n$  $n$ números reales positivos cuyo producto es igual a $1$. Demostrar que $a_1+...+a_n \geq n$ y que el signo de igualdad se presenta sólo cuando cada $a_k =1$\\\\
Demostración.- \; Consideremos dos casos:
\begin{enumerate}[C1]
\item Si $a_1=...=a_n=1,$ entonces 
$$\sum\limits_{k=1}^n = \sum\limits_{k=1}^n 1 = n \geq \prod\limits_{k=1}^n a_n = \prod\limits_{k=1}^n 1 = 1 $$
Por lo tanto la desigualdad se cumple.
\item Sea $a_k \neq 1$ y $n=2$ por inducción vemos que no se cumple la desigualdad $$a_1 \cdot a_2 \neq 1$$. Si uno de ellos no es uno tampoco lo es el siguiente es decir,
\begin{center}
\begin{tabular}{rcl}
$a_1 \cdot a_2 = 1$&$\Rightarrow$&$a_2 = \dfrac{1}{a_1}$\\\\
&$\Rightarrow$&$a_1 + a_2 = a_1 + \dfrac{1}{a_1}$\\\\
&$\Rightarrow$&$a_1 + a_2=\dfrac{a_1^2 + 1}{a_1}$\\\\
&$\Rightarrow$&$a_1 + a_2=\dfrac{a_1^2 - 2a_1 + 1 +2a_1}{a_1}$\\\\
&$\Rightarrow$&$a_1 + a_2=\dfrac{(a_1 - 1)^2}{a_1} + 2$\\\\
&$\Rightarrow$&$a_1 + a_2 > 2$\\\\
\end{tabular}
\end{center}
Donde la desigualdad final sigue desde $\dfrac{(a_1-1)^2}{a_1}$ para $a_1>0,$ por lo tanto la desigualdad es válida para el caso $n=2$\\
Supongamos que la desigualdad se cumple para $n=k \in \mathbb{Z}^+$. Luego $a_1,...,a_{k+1} \in \mathbb{R}^+$ con $a_i \neq 1$ para al menos un $i=1,...,k+1.$ Si $a_i < 1$, entonces debe haber algún $j\neq i$ tal que $a_j > 0$ porque de lo contrario si $a_j \leq 1$ para  todo $j\neq i$, entonces $a_1 \cdot \cdot \cdot a_{k+1}<1.$ De manera similar, si $a_i>1,$ entonces hay algún $j\neq i$ tal que $a_j <1.$ Por lo tanto tenemos un par $a_i, \; a_j$ con un miembro del par mayor que $1$ y el otro menor que $1$. Sea este par $a_1$ y $a_{k+1}$ entonces definamos $a_1 \cdot a_{k+1}$, entonces,
$$b\cdot a_2 \cdot \cdot \cdot a_k = 1 \Rightarrow b_1 + a_2 + ... - a_k \geq k$$
Además, dado que $(1-a_i)(1-a_{k+1})<0$ (dado que uno de $a_1, a_{k+1}$ es mayor que $1$ y el otro es menor que $1$, uno de $(1-a_1), (1-a_{k+1})$ es positivo y el otro es negativo, luego $$1-a_1-a_{k+1}+a_1a_{k+1}<0 \Rightarrow b<a_1 + a_{k+1}-1,$$
Así
$$b+a_2+...+a_k\geq k \Rightarrow a_1 + a_2 +...+a_k + a_{k+1} \geq k+1$$
Por lo tanto, la desigualdad es válida para $k+1$ y en consecuencia es verdadera para $n\in \mathbb{Z}^+$\\\\
\end{enumerate}

\begin{def.}
La media geométrica $G$ de $n$ números reales positivos $x_1,...,x_n$ está definida por la fórmula $G=(x_1 x_2 \cdot \cdot \cdot x_n)^{1/n}$
\end{def.}
%----------------------------------20-----------------------------------
\item 
\begin{enumerate}[\bfseries (a)]
\item Desígnese con $M_p$ la media de potencias p-ésimas. Demostrar que $G \leq M_1$ y que $G=M_1$ solo cuando $x_1 = x_2 = \cdot \cdot \cdot = x_n.$\\\
Demostración.- \; Si $x_1, ... ,x_n$ no todos iguales, entonces $$G_n = \left[ (x_1 \cdot \cdot \cdot x_n)^{1/n} \right]^n = x_1 \cdot \cdot \cdot x_n,$$ así, 
\begin{center}
\begin{tabular}{rcll}
$\left( \dfrac{1}{G_n} \right)(x_1\cdot \cdot \cdot x_n)$&$=$&$1$&\\\\
$\left( \dfrac{x_1}{G} \right) \left( \dfrac{x_2}{G} \right) \cdot \cdot \cdot \left( \dfrac{x_n}{G} \right)$&$=$&$1$&\\\\
$\dfrac{x_1}{G} + \dfrac{x_2}{G} + ... + \dfrac{x_n}{G}$&$>$&$n$& por problema anterior\\\\
$x_1+x_2+...+x_n$&$>$&$n \cdot G$\\\\
$M_1$&$>$&$G$\\\\
\end{tabular}
\end{center}
luego si $x_1,...x_n$ son todos iguales , entonces, $$G=(x_1\cdot \cdot \cdot x_n)^{1/n} = (x_1^n)^{1/n}=x_1 =  \dfrac{n x_1}{n} = \dfrac{x_1 + ... + x_n}{n} = M_1$$\\
\item Sean $p$ y $q$ enteros, $q<0<p$. A partir de $(a)$ deducir que $M_q< G < M_p$ si $x_1,x_2,...,x_n$ no son todos iguales.\\\\
Demostración.- \; Probemos primero que $G < M_p$. Para ello primero veamos que si $x_1,...,x_n$ son números reales positivos, no todos iguales entonces $x_P,...,x_n^P$ también son números reales positivos también no todos iguales. A partir de la definición de $M_p$ y dejando $M_p(x_1^p,...,x_n^p)$ denotar la p-enésima potencia media de los números $x_i^p,...,x_n^p$ tenemos,
\begin{center}
\begin{tabular}{rcl}
$M_1(x_1^p,...,x_n^p)$&$=$&$\dfrac{x_1^p + ... + x_n^p}{n}$\\\\
&$=$&$\left[ \left( \dfrac{x_1^p + ... + x_n^p}{n} \right)^{1/p} \right]^p$\\\\
&$=$&$[M_p(x_1,...,x_n)]^p$\\\\
\end{tabular}
\end{center}
así se observa que $[M_p(x_1,...,x_n)]^p = M_1(x_i^p,...,x_n^p)$ luego por la parte $(a)$,
\begin{center}
\begin{tabular}{rcl}
$G(x_1,...,x_n)$&$=$&$(x_1\cdot \cdot \cdot x_n)^{p/n}$\\
&$=$&$(x_i^p \cdot \cdot \cdot x_n^p)^{1/n}$\\
&$=$&$G(x_1^p,...,x_n^p)$\\
&$<$&$M_1(x_1^p,...,x-n^p)$\\
&$=$&$[M_p(x_1,...,x_n)]^p$\\
\end{tabular}
\end{center}
Por lo tanto implica que  $G<M_p$\\
ahora debemos  $M_q<G$ para $q<0$, visto de otra forma $-q>0$ entonces $G<M_{-q}$ y por la desigualadad que demostramos se tiene, $$G^{-q}<(M_{-q})^{-q} \Rightarrow G^q > M_q^q \Rightarrow G > M_q$$\\\\
\end{enumerate}

%---------------------------------21-----------------------------------
\item Aplíquese los resultados del Ejercicio 20 para probar la siguiente proposición: Si $a,b$ y $c$ son números reales y positivos tales que $abc=8$, entonces $a+b+c \geq 6$ y $ab +ac +bc \geq 12.$\\\\
Demostración.- \; Sea $a,b \in \mathbb{R}^+$ tenemos 
\begin{center}
\begin{tabular}{rcl}
$(a\cdot b \cdot c)^{1/3}$&$\leq$&$ \dfrac{a+b+c}{3} $\\\\
$8$&$\leq$&$\dfrac{(a+b+c)^3}{3^3}$\\\\
$2^3 \cdot 3^3$&$\leq$&$(a+b+c)^3$\\\\
$a+b+c$&$\geq$&$6$\\\\
\end{tabular}
\end{center}
Luego para la segunda desigualdad utilizamos la parte $(b)$ del anterior problema,
\begin{center}
\begin{tabular}{rcl}
$\left( \dfrac{a^{-1} + b^{-1} + c^{-1}}{3} \right)^{-1}$&$\leq$&$(a \cdot b \cdot c)^{1/3}$\\\\
$\left[ \dfrac{3^3}{\left( \dfrac{1}{a} + \dfrac{1}{b} + \dfrac{1}{c} \right)^3} \right] ^3$&$\leq$&$2^3$\\\\
$\left( \dfrac{1}{a} + \dfrac{1}{b} + \dfrac{1}{c} \right)^3$&$\geq $&$\dfrac{3^3}{2^3}$\\\\
$\dfrac{1}{a} + \dfrac{1}{b} + \dfrac{1}{c}$&$\geq$&$\dfrac{3}{2}$\\\\
$\dfrac{bc + ac + ab}{abc}$&$\geq$&$\dfrac{3}{2}$\\\\
$\dfrac{bc + ac + ab}{8}$&$\geq$&$\dfrac{3}{2}$\\\\
$ab+ac+bc$&$\geq$&$12$\\\\
\end{tabular}
\end{center}
%---------------------------------22------------------------------------
\item Si $x_1,...,x_n$ son números positivos y si $y_k = 1/x_k,$ demostrar que $$\displaystyle \left( \sum_{k=1}^n x_k \right) \left( \sum_{k=1}^n y_k \right) \geq n^2.$$\\\\
Demostración.- \; Sea $\sqrt{x_k} \sqrt{y_k} = 1$ ya que $y_k = \dfrac{1}{x_k}$ entonces por la desigualdad de Cauchy-Schwarz tenemos,
\begin{center}
\begin{tabular}{rcl}
$\left( \sum\limits_{k=1}^n a_k b_k \right)^2$&$\leq$&$\left( \sum\limits_{k=1}^n a_k^2 \right) \left( \sum\limits_{k=1}^n  b_k^2 \right)$\\\\
$\left( \sum\limits_{k=1}^n \sqrt{x_k} \sqrt{y_k} \right)^2$&$\leq$&$\left( \sum\limits_{k=1}^n \sqrt{x_k}^2 \right)\left( \sum\limits_{k=1}^n \sqrt{y_k}^2 \right)$\\\\
$\left( \sum\limits_{k=1}^n 1 \right)^2$&$\leq$&$\left(\sum\limits_{k=1}^n x_k \right)\left( \sum\limits_{k=1}^n  y_k\right)$\\\\
$\left(\sum\limits_{k=1}^n x_k \right)\left( \sum\limits_{k=1}^n  y_k\right)$&$\geq$&$n^2$\\\\
\end{tabular}
\end{center}

%--------------------------------23------------------------------------
\item Si $a,b$ y $c$ son números positivos y si $a+b+c =1$, demostrar que $(1-a)(1-b)(1-c)\geq 8abc$\\\\
Demostración.- \; Sea $M_{-1}(a,b,c) \leq M_1(a,b,c)$ entonces $$\left( \dfrac{a_{-1} + b_{-1} + c_{-1}}{3}\right)^{-1} \leq \dfrac{a+b+c}{3}$$ por lo tanto,
\begin{center}
\begin{tabular}{rcl}
$\dfrac{3}{\dfrac{1}{a}+ \dfrac{1}{b} + \dfrac{1}{c}}$&$\leq$&$\dfrac{1}{3}$\\\\
$9$&$\leq$&$\dfrac{1}{a}+ \dfrac{1}{b} + \dfrac{1}{c}$\\\\
$9abc$&$\leq$&$bc+ac+ab$\\\\
$8abc$&$\leq$&$bc+ac+ab-abc$\\\\
$8abc$&$\leq$&$1-(a+b+c)+ab+ac+bc-abc$\\\\
\end{tabular}
\end{center}
ya que $1=a+b+c$, luego $1-(a+b+c)+ab+ac+bc-abc = (1-a)(1-b)(1-c)$ así nos queda $$8abc\leq (1-a)(1-b)(1-c)$$\\\\
\end{enumerate}





    %---------- conceptos de calculo integral 
	%\setcounter{chapter}{0}
\chapter{Los conceptos del Cálculo Integral}

\setcounter{chapter}{1}
\setcounter{section}{2}

\section{Funciones. Definición formal como conjunto de pares ordenados}
En cálculo elemental tiene interés considerar en primer lugar, aquellas funciones en las que el dominio y el recorrido son conjuntos de números reales. Estas funciones se llaman 
\textbf{Funciones de variable real} o funciones reales.\\

        %-----------------------------1.1. definición par ordenado-----------------------------
        \begin{def.}[Par ordenado]
            Dos pares ordenados $(a,b)$ y $(c,d)$ son iguales si y sólo si sus primeros elementos son iguales y sus segundos elementos son iguales.
            $$(a,b) = (c,d) \; \; \mbox{si y sólo si} \; \; a=c \; y \; b=d$$
        \end{def.}

        
        \begin{def.}[Definición de función]
            Una función $f$ es un conjunto de pares ordenados $(x,y)$ ninguno de los cuales tiene el mismo primero elemento.\\\\
            Debe cumplir las siguientes condiciones de existencia y unicidad:
            \begin{enumerate}[\bfseries (i)]
                \item $\forall x \in D_f, \exists y / (x,y) \in f(x) \; ó \; y=f(x)$
                \item $(x,y) \in  f \land (x,z) \in  f \Rightarrow y = z$
            \end{enumerate}
        \end{def.}

        %----------------------------1.3 definición dominio recorrido---------------------------
        \begin{def.}[Dominio y recorrido]
            Si $f$ es una función, el conjunto de todos los elementos $x$ que aparecen como primeros elementos de pares $(x,y)$ de $f$ se llama el \textbf{dominio} de $f$.  El conjunto de los segundos elementos y se denomina \textbf{recorrido} de $f$, o conjunto de valores de $f$.
        \end{def.}

        %----------------------teorema 1.1------------------------
        \begin{teo}
            Dos funciones $f$ y $g$ son iguales si y sólo si 
            \begin{enumerate}[\bfseries (a)]
                \item $f$ y $g$ tienen el mismo dominio, y
                \item $f(x) = g(x)$ para todo $x$ del dominio de $f.$\\
            \end{enumerate}
            Demostración.- \; Sea $f$ función tal que $x \in  D_f,\exists y \; / \; y=f(x)$ es decir $(x,f(x))$, $g$ una función talque $\forall  z \in  D_g , \exists  y \; / \; y=g(z)$ es decir $(z,g(z))$, entonces por definición de par ordenado tenemos que $(x,f(x)) = (z,g(z)) $ si y sólo si $x=z$ y $f(x)=g(z)$\\\\
        \end{teo}

        %--------------1.4 definición de sumas productos y cocientes de una función----------------------
        \begin{def.}[Sumas, productos y cocientes de funciones]
            Sean $f$ y $g$ dos funciones reales que tienen el mismo dominio $D$. Se puede construir nuevas funciones a partir de $f$ y $g$ por adición, multiplicación o división de sus valores. La función $u$ definida por,
            $$u(x) = f(x) + g(x) \; \; si \; x \in D$$
            se denomina suma de $f$ y $g$, se representa por $f+g.$ Del mismo modo, el producto $v=f cdot g$ y el cociente $w=f/g$ están definidos por las fórmulas
            $$v(x) = f(x) g(x) \; \; si \; x \in D, \; \; \, \, w(x) = f(x)/g(x) \; \; si \, x \, \in D \; y \; g(x) \neq 0$$
        \end{def.}
    
\setcounter{section}{4}
\section{Ejercicios}
    \begin{enumerate}[ \bfseries 1.]
        %--------------------1.-------------------
        \item Sea $f(x)=x+1$ para todo real $x$. Calcular:
            \begin{itemize}
                \item $f(2) = 2+1 = 3$\\\\
                \item $f(-2) = -2 +1 = -1$\\\\
                \item $-f(2) = -(2+1)=-3$\\\\
                \item $f \left( \dfrac{1}{2} \right) = \dfrac{1}{2} + 1 = \dfrac{3}{2}$\\\\
                \item $\dfrac{1}{f(2)}= \dfrac{1}{3}$\\\\
                \item $f(a+b) = a+b+1$\\\\
                \item $f(a)+f(b)= (a+1) + (b+1) = a+b+2$\\\\
                \item $f(a) \cdot f(b) = (a+1)(b+1) = ab + a + b + 1$\\\\
            \end{itemize}

        %--------------------2.--------------------
        \item Sean $f(x)= 1+x$ \; y \; $g(x)=1-x$ para todo real $x$. calcular:
            \begin{itemize}
                \item $f(2)+g(2) = (1+2) + (1-2) = 2$\\\\
                \item $f(2)-g(2) = (1+2) - (1-2) = 4$\\\\
                \item $f(2)\cdot g(2) = (1+2) \cdot (1-2) = 3 \cdot (-1) = -3$\\\\
                \item $\dfrac{f(2)}{g(2)}= \dfrac{1+2}{1-2} = \dfrac{3}{-1} = -3$\\\\
                \item $f\left[ g(2)\right] = f(1-2) = f(-1) = 1+(-1)= 0$\\\\
                \item $g\left[ f(2)\right] = f(1+2) = g(3) = 1 - 3 = -2$\\\\
                \item $f(a) + g(-a) = (1+a) + (1 - a) = 2$\\\\
                \item $f(t)\cdot g(-t) = (1+t) \cdot (1+t) = 1 + t + t + t^2 = t^2 +2t + 1 = (t+1)^2$\\\\
            \end{itemize}
        
        %--------------------3.---------------------
        \item Sea $f(x)=|x-3|+|x-1|$ para todo real $x$. Calcular:\\
            \begin{itemize}
                \item $f(0) = |0-3|+|0-1| = 3 + 1 = 4$
                \item $f(1) = |1-3|+|1-1| = 2$
                \item $f(2) = |2-3|+|2-1| = -1 + 1 = 2$
                \item $f(3) = |3-3|+|3-1| = 2$
                \item $f(-1) = |-1-3|+|-1-1| = 4 + 2 = 6$
                \item $f(-2) = |-2-3|+|-2-1| = 5 + 3 = 8$\\
            \end{itemize}
            Determinar todos los valores de $t$ para los que $f(t+2)=f(t)$\\
            \begin{center}
                \begin{tabular}{r c l}
                    $|t+2-3| + |t+2-1|$&=&$|t-3| + |t-1|$\\
                    $|t-1|+|t+1|$&=&$|t-3|+|t-1|$\\
                    $|t+1|$&=&$t-3$\\
                \end{tabular}
            \end{center}
            Por lo tanto  $t=1$\\\\

        %--------------------4.-------------------
        \item Sea $f(x)=x^2$  para todo real $x$. Calcular cada una de las fórmulas siguientes. En cada caso precisar los conjuntos de números erales $x, \; y \; t,$ etc., para los que la fórmula dada es válida.
            \begin{enumerate}[\bfseries (a)]
                %----------(a)----------
                \item $f(-x)=f(x)$ \\\\
                Demostración.- \; Se tiene $f(-x) = (-x)^2 = x^2 = f(x) \; \forall x \in \mathbb{R}$\\\\

                %----------(b)----------
                \item $f(y)-f(x)=(y-x)(y+x)$\\\\
                Demostración.- \; $f(y)-f(x)= y^2 - x^2 = (x-y)(x+y), \; \forall x, \; y \in \mathbb{R}$\\\\

                %----------(c)----------
                \item $f(x+h) - f(x) = 2xh + h^2$\\\\
                Demostración.- \; $f(x+h) - f(x) = (x+h)^2 -x^2 = x^2 + 2xh +h^2 - x^2 = 2xh + h^2, \; \forall x \in \mathbb{R}$\\\\

                %----------(d)----------
                \item $f(2y) = 4f(y)$\\\\
                Demostración.- \; $f(2y) = (2y)^2 = 4y^2 = 4 f(y), \; \forall y \in \mathbb{R}$\\\\
                %----------(e)----------
                \item $f(t^2)=f(t)^2$\\\\
                Demostración.- \; $f(t^2) = (t^2)^2 = f(t)^2$\\\\
                %----------(f)----------
                \item $\sqrt{f(a)} = |a|$\\\\
                Demostración.- \; $\sqrt{f(a)} = \sqrt{a^2} = |a|$\\\\
            \end{enumerate}

        %--------------------5.--------------------
        \item Sea $g(x) = \sqrt{4-x^2}$ para $|x| \leq 2$. Comprobar cada una de las fórmulas siguientes e indicar para qué valores de $x, \; y, s$ y $t$ son válidas.
            \begin{enumerate}[\bfseries (a)]
                %----------(a)----------
                \item $g(-x) = g(x)$\\\\
                Se tiene $g(-x)=\sqrt{2-(-x)^2} = \sqrt{2-(x)^2} = g(x), \; \; para \; |x| \leq 2$\\\\

                %----------(b)----------
                \item $g(2y) = 2\sqrt{1-y^2}$\\\\
                $g(2y)=\sqrt{4-(2y)^2}= \sqrt{4(1-y^2)} = 2 \sqrt{1-y^2}, \; \; para \; |y|\leq 1$ Se obtiene $|y| \leq 1$  de $\sqrt{1-y^2}$ es decir $1-y^2 \geq 0$ entonces $\sqrt{y^2} \leq \sqrt{1}$ \; y \; $|y|\leq 1$\\\\

                %----------(c)----------
                \item $g\left( \dfrac{1}{t} \right) = \dfrac{\sqrt{4t^2-1}}{|t|}$\\\\
                $g\left( \dfrac{1}{t} \right) = \sqrt{4 - \left( \dfrac{1}{t} \right)^2} = \sqrt{\dfrac{4t^2 - 1}{t^2}} =\dfrac{\sqrt{4t^2 - 1}}{|t|}, \; \; para \; |t| \geq \dfrac{1}{2}$\\\\
                Para hallar los valores correspondientes debemos analizar $\sqrt{4t^2 - 1}$. Es decir $$4t^2-1 \geq 0 \Rightarrow 4t^2 \geq 1 \Rightarrow t^2 \geq \dfrac{1}{2^2} \Rightarrow |t| \geq \dfrac{1}{2}$$\\

                %----------(d)----------
                \item $g(a-2) = \sqrt{4a-a^2}$\\\\
                $g(a-2) = \sqrt{4 - x^2} = \sqrt{4 - (a-2)^2} = \sqrt{4a - a^2}, \; \; para \; 0\leq a \leq 4.$ Basta probar  $4a-a^2 \geq 0$\\\\

                %----------(e)----------
                \item $g \left( \dfrac{s}{2} \right) = \dfrac{1}{2} \sqrt{16 - s^2}$\\\\
                $s\left( \dfrac{s}{2} \right) = \sqrt{4 - \left( \dfrac{s}{2} \right)^2} = \dfrac{\sqrt{16 - s^2}}{2}, \; \; para \; |s| \leq 4$. ya que solo basta comprobar que $\sqrt{16 - s^2} \geq 0$\\\\

                %----------(f)----------
                \item $\dfrac{1}{2 +g(x)} = \dfrac{2-g(x)}{x^2}$\\\\ 
                $\dfrac{1}{2 +g(x)} = \dfrac{1}{2+ \sqrt{4-x^2}} \cdot \dfrac{2 - \sqrt{4-x^2}}{2 - \sqrt{4-x^2}} = \dfrac{2 - g(x)}{x^2}\; para \; \; |x| \leq 2 \; y \; x \neq 0$\\\\ 
                Evaluemos $\sqrt{4-x^2}$. Sea $4-x^2 \geq 0$ entonce $\sqrt{x^2} \leq 2$. Por otro lado tenemos que la función no puede ser $0$ por $\dfrac{1}{x^2}$, por lo tanto debe ser $x^2\neq 0$.\\\\ 
            \end{enumerate}

        %--------------------6.-------------------
        \item Sea $f$ la función definida como sigue: $f(x)=1$ para $0 \leq x \leq 1;$ $f(x)=2$ para $1 < x \leq 2$. La función no está definida si $x<0$ ó si $x >2.$
            \begin{enumerate}[\bfseries (a)]
                %----------(a)----------
                \item Trazar la gráfica de $f$ 
                    \begin{center}
                        \begin{tikzpicture}[scale=1,draw opacity = 0.6]
                            % abscisa y ordenada
                            \tkzInit[xmax= 3,xmin=0,ymax=3,ymin=0]
                            \tiny\tkzLabelXY[opacity=0.6,step=1, orig=false]
                            % etiqueta x, f(x)
                            \tkzDrawX[opacity=0.6,label=x,right=0.3]
                            \tkzDrawY[opacity=0.6,label=f(x),below = -0.6]
                            %dominio y función
                            \draw [domain=0:1,thick,gray] plot(\x,{1});
                            \tkzText[opacity=0.6,above](0.5,1){\tiny $f(x)=1$}
                            \draw [domain=1:2,thick,gray] plot(\x,{2});
                            \tkzText[opacity=0.6,above](1.5,2){\tiny $f(x)=2$}
                            % intervalos
                            \draw[fill=black] (0,1) circle (1.5pt);
                            \draw[fill=black] (1,1) circle (1.5pt);
                            \draw[          ] (1,2) circle (1.5pt);
                            \draw[fill=black] (2,2) circle (1.5pt);
                        \end{tikzpicture}
                    \end{center}

                %----------(b)----------
                \item Poner $g(x) = f(2x).$ Describir el dominio de $g$ y dibujar su gráfica.
                    \begin{center}
                        \begin{tikzpicture}[scale=1,draw opacity = 0.6]
                            % abscisa y ordenada
                            \tkzInit[xmax= 3,xmin=0,ymax=3,ymin=0]
                            \tiny\tkzLabelXY[opacity=0.6,step=1, orig=false]
                            % label x, f(x)
                            \tkzDrawX[opacity=0.6,label=x,right=0.3]
                            \tkzDrawY[opacity=0.6,label=f(x),below = -0.6]
                            %dominio y función
                            \tkzFct[opacity=1,domain = 0:0.5]{1}
                            \draw [domain=0:0.5,thick,gray] plot(\x,{1}); 
                            \tkzFct[opacity=1,domain = 0.5:1]{2} 
                            % intervalos
                            \tkzSetUpPoint[shape=circle, size = 3, color=black, fill=black]
                            \tkzDefPointByFct[draw,with = a](0)
                            \tkzDefPointByFct[draw,with = a](0.5)
                            \tkzDefPointByFct[draw,with = b](1)
                            \tkzSetUpPoint[shape=circle, size = 3, color=black, fill=white]
                            \tkzDefPointByFct[draw,with = b](0.5)
                        \end{tikzpicture}
                    \end{center}
                Debido a que $1\leq 2x \leq 1$ \; y \; $1 < 2x \leq 2$ el dominio de $g(x)$ es $0\leq x \leq 1$\\\\ 

                %----------(c)----------
                \item Poner $h(x) = f(x-2).$ Describir el dominio de $k$ \; y dibujar su gráfica.
                    \begin{center}
                        \begin{tikzpicture}[scale=1,draw opacity = 0.6]
                            % abscisa y ordenada
                            \tkzInit[xmax= 4,xmin=0,ymax=3,ymin=0]
                            \tiny\tkzLabelXY[opacity=0.6,step=1, orig=false]
                            % label x, f(x)
                            \tkzDrawX[opacity=0.6,label=x,right=0.3]
                            \tkzDrawY[opacity=0.6,label=f(x),below = -0.6]
                            %dominio y función
                            \tkzFct[opacity=1,domain = 2:3]{1} 
                            \tkzFct[opacity=1,domain = 3:4]{2} 
                            % intervalos
                            \tkzSetUpPoint[shape=circle, size = 3, color=black, fill=black]
                            \tkzDefPointByFct[draw,with = a](2)
                            \tkzDefPointByFct[draw,with = a](3)
                            \tkzDefPointByFct[draw,with = b](4)
                            \tkzSetUpPoint[shape=circle, size = 3, color=black, fill=white]
                            \tkzDefPointByFct[draw,with = b](3)
                        \end{tikzpicture}
                    \end{center}
                Debido a que $1\leq x-2 \leq 1$ \; y \; $1 < x-2 \leq 2$ el dominio de $h(x)$ es $2\leq x \leq 4$\\\\ 

                %----------(d)----------
                \item Poner $k(x) = f(2x) + f(x-2).$ Describir el dominio de $k$ \; y dibujar su gráfica.\\\\
                El dominio está vacío ya $f(2x)$ que solo está definido para $0 \leq x \leq 1$ \; y \;  $f(x-2)$ solo está definido para $2 \leq x \leq 4$. Por lo tanto no hay ninguno $x$ que satisfaga ambas condiciones. \\\\
            \end{enumerate}
        
        %--------------------7.-------------------
        \item Las gráficas de los dos polinomios $g(x)=x$ \; y \; $f(x)=x^3$ se cortan en tres puntos. Dibujar una parte suficiente de sus gráficas para ver cómo se cortan.
            \begin{center}
                \begin{tikzpicture}[scale=0.9, draw opacity = .6]
                    % abscisa y ordenada
                    \tkzInit[xmax= 3,xmin=-2,ymax=3,ymin=-2]
                    \tiny\tkzLabelXY[opacity=0.6,step=1, orig=false]
                    % label x, f(x)
                    \tkzDrawX[opacity= .6,label=x,right=0.3]
                    \tkzDrawY[opacity= .6,label=f(x),below = -0.6]
                    %dominio y función
                    \draw [domain=-2:3,thick] plot(\x,{\x}); 
                    \tkzText[above,opacity=0.6](3.3,3){\tiny $f(x)=x$}
                    \draw [domain=-1.3:1.45,thick] plot(\x,{\x^3}); 
                    \tkzText[above,opacity=0.6](1.2,3){\tiny $f(x)=2x^3$}
                \end{tikzpicture}
            \end{center}  

        %--------------------8.-------------------
        \item Las gráficas de los dos polinomios cuadráticos $f(x) = x^2-2$ \; y \; $g(x)=2x^2+4x+1$ se cortan en dos puntos. Dibujar las porciones de sus gráficas comprendidas entre sus intersecciones.
            \begin{center}
                \begin{tikzpicture}[scale=.6, draw opacity = .6]
                    % abscisa y ordenada
                    \tkzInit[xmax= 4,xmin=-4,ymax=10,ymin=-3]
                    \tiny\tkzLabelXY[opacity=0.6,step=1, orig=false]
                    % label x, f(x)
                    \tkzDrawX[opacity= .6,label=x,right=0.3]
                    \tkzDrawY[opacity= .6,label=f(x),below = -0.6]
                    %dominio y función
                    \draw [domain=-3.5:3.2,thick] plot(\x,{\x*\x - 2});  
                    \tkzText[above,opacity=0.6](4,8.2){\tiny $f(x)=x^2 - 2$}
                    \draw [domain=-3.4:1.4,thick] plot(\x,{2*\x*\x + 4*\x + 1});  
                    \tkzText[above,opacity=0.6](3,10.5){\tiny $f(x)=2x^2 +4x + 1$}
                \end{tikzpicture}
            \end{center} 

        %--------------------9.-------------------
        \item Este ejercicio desarrolla ciertas propiedades fundamentales de los polinomios. Sea $f(x)=\displaystyle\sum_{k=0}^{n} c_k x^k$ un polinomio de grado $n$. Demostrar cada uno de los siguientes apartados:
            \begin{enumerate}[\bfseries (a)]
                %----------(a)----------
                \item Si $n \geq 1$ \; y \; $f(0)=0,$ $f(x)=xg(x)$, siendo $g$ un polinomio de grado $n-1.$\\\\
                Para entender lo que nos quiere decir Apostol pongamos un ejemplo. Supongamos que tenemos un polinomio donde $f(x)=2x^2+3x-x$ entonces notamos que $f(x)=x(2x+3-1)$ donde $g(x)=2x+3-1$, esto quiere decir que si $0 = f(0)=c_0 \Rightarrow c_1 x + c_2 x^2 + ... + c_n x^n = x(c_1 + c_2 x + ... + c_n x^{n-1})$ 
                Así que debemos demostrar que $f(x)$ es un polinomio arbitrario de grado $n \geq 1$ tal que $f(0)=0$, entonces debe haber un polinomio de grado $n-1, \; g(x)$, tal que $f(x)=xg(x)$\\\\
                Demostración.- \; Sabemos que $$f(0) = c_n \cdot 0^n + c_{n-1} \cdot 0^{n-1} + ... + c_1 \cdot 0 + c_0 =c_0,$$ como $f(0)=0$ se concluye que $c_0=0$. Así tenemos $$f(x)=\displaystyle\sum_{k=1}^{n} c_k x^k.$$ Ahora crearemos una función $g(x).$ Dada la función $f(x)$ como la anterior, definamos, $$f(x)=\displaystyle\sum_{k=0}^{n} c_k x^{k-1} = \sum_{k=1}^{n} c_k x^{k-1}$$ 
                Ahora crearé una función $g(x)$. Dada una función $f(x)$ como la anterior, definamos  $$g(x) = \displaystyle\sum_{k=1}^{n} c_k x^{k-1}$$
                donde $c_k$ son los mismos que los dados por la función $f(x)$. Primero notemos que el grado de $g(x)$ es $n-1$. Finalmente, tenemos que $$ xg(x) = x \displaystyle\sum_{k=1}^{n} c_k x^{k-1} = \sum_{k=1}^{n} c_k x^k = f(x).$$ \\\\

                %----------(b)----------
                \item Para cada real $a$, la función $p$ dada por $p(x)=f(x+a)$ es un polinomio de grado $n.$\\\\
                Demostración.- \; Usando el teorema del binomio,
                    \begin{center}
                        \begin{tabular}{r c l}
                            $f(x+a)$&=&$ \displaystyle\sum_{k=0}^{n} (x+a)^k c_k$\\\\
                            &=&$c_o + (x+a)c_1 + (x+a)^2 c_2 + ...+(x+a)^n c_n$\\\\
                            &=&$ c_o + c_1 \left( \displaystyle\sum_{j=0}^{1} {1 \choose j} a^j x^{1-j} \right) + c_2 \left( \displaystyle\sum_{j=0}^2 {2 \choose j} a^j x^{2-j} \right) 
                            + ... + c_n \left( \displaystyle\sum_{j=0}^{n} {n \choose j} a^j x^{n-j} \right)$\\\\
                            &=&$(c_o + ac_1 + a^2 c_2 + ... + a^n c_n) + x(c_1 + 2ac_2 + ... + na^{n-1} c_n)$\\\\
                            &=&$\displaystyle\sum_{k=0}^n \left( x^k \left( \displaystyle\sum_{j=k} {j \choose j-k} c_j a{j-k}\right) \right)$\\
                        \end{tabular}
                    \end{center}
                En la linea final reescribimos los coeficientes como sumas para verlos de manera más concisa. De cualquier manera, dado que todos los $c_i$ son constantes, tenemos $\displaystyle\sum_{j=k}^n {j \choose j-k} c_j a^{j-k}$ es alguna constante para cada $k,$ de $d_k$ y tenemos, $$p(x) = \displaystyle\sum_{k=0}^n d_k x^k$$\\\\

                %----------(c)----------
                \item Si $n \geq 1$ \; y \; $f(a)=0$ para un cierto valor real $a$, entonces $f(x) = (x-a) \, h(x),$ siendo $h$ un polinomio de grado $n-1$. (considérese $p(x)=f(x+a).$)\\\\
                Demostración.- \; Por la parte $b)$ se sabe que $f(x)$ es un polinomio de grado $n$, entonces $p(x)=f(x+a)$ también es un polinomio del mismo grado. Ahora si $f(a)=0$ entonces por hipótesis $p(0)=f(a)=0$. Luego por la parte $a)$, tenemos $$p(x)=x \cdot g(x)$$ donde $g(x)$ es un polinomio de grado $n-1$. Así,
                $$p(x-a)=f(x)= f(x) = (x-a) \cdot g(x-a)$$ ya que $p(x)=f(x+a)$. Pero, si $g(x)$ es un polinomio de grado $n-1$, entonces por la parte $b)$ nuevamente, también lo es $h(x)=g(x+(-a)) = g(x-a)$. Por lo tanto, $$f(x)=(x-a) \cdot h(x)$$ para $h$ un grado $n-1$ polinomial, según lo solicitado.\\\\\

                %----------(d)----------
                \item Si $f(x)=0$ para $n+1$ valores reales de $x$ distintos, todos los coeficientes $c_k$ son cero y $f(x)=0$ para todo real de $x$\\\\
                Demostración.- \;  La prueba se realizara por inducción. Sea $n=1$, entonces $f(x)=c_o + c_1 x$. Dado que la hipótesis es que existen $n+1$ distintos  $x$ de tal manera que $f(x)=0$, sabemos que existen $a_1$, $a_2$ $\in \mathbb{R}$ tal que $$f(a_1)=f(a_2)=0, \; \; \; a_1 \neq a_2,$$ Así, 
                    \begin{center}
                        \begin{tabular}{r c l c r c l l}
                            $c_0 + c_1 a_1$&$=$&$0$&$\Rightarrow$&$c_1 a_1 - c_1 a_2$&$=$&$0$&\\
                            &&&$\Rightarrow$&$c_1(a_1 - a_2)$&$=$&$0$&\\
                            &&&$\Rightarrow$&$c_1$&$=$&$0$& ya que $a_1 \neq a_2$\\
                            &&&$\Rightarrow$&$c_0$&$=$&$0$&ya que $c_0 + c_1 a_1 = 0$\\
                        \end{tabular}
                    \end{center}
                Por lo tanto, la afirmación es verdadera. Suponga que es cierto para algunos $n=k \in \mathbb{Z}^{+}$. Luego Sea $f(x)$ un polinomio de grado $k+1$ con $k+2$ distintos de $0$, $a_1,...,a_{k+2}.$ ya que $f(a_{k+2}) = 0$, usando la parte $c)$, tenemos, $$f(x)=(x- a_{k+2})h(x)$$
                donde $h(x)$ es un polinomio de grado $k$. Sabemos que hay $k+1$ valores distintos $a_1,... a_{k+1}$ tal que  $h(a_i) = 0.$ Dado que $f(a_i)=0$ para $1< i < k+2y$ y $(x- a_{k+2}) \neq 0$ para $x=a_i$ con $1 < i < k+1$ ya que todos los $a_1$ son distintos), por lo tanto, según la hipótesis de inducción, cada coeficiente de $h$ es $0$ y $h(x)=0$ para todo $x \in \mathbb{R}.$ Así,
                    \begin{center}
                        \begin{tabular}{r c r c l}
                            $f(x)$&$=$&$(x - a_{k+2})h(x)$&$=$&$(x- a_{k+2}) \cdot \displaystyle\sum_{j=0}^k c_j x^j$\\\\
                            &&&$=$&$\displaystyle\sum_{j=0}^k (x - a_{k+2})c_j x^j$\\\\
                            &&&$=$&$c_k x^{k+1} + (c_{k-1} - a_k + 2c_k)x^k + ... + (c_1 -a_{k+2} c_0)x + a_{k+2} c_0$\\
                        \end{tabular}
                    \end{center}
                Pero dado que todos los coeficientes de $h(x)$ son cero y $f(x) = 0$ para todo $x \in \mathbb{R}$. Por lo tanto, la afirmación es verdadera para el caso $k+1$ y para todo $n \in \mathbb{Z}^+$\\\\

                %----------(e)----------
                \item Sea $g(x) = \displaystyle\sum_{k=0}^m b_k x^k$ un polinomio de grado $m$, siendo $m \geq n.$ Si $g(x)=f(x),$ para $m+1$ valores reales de $x$ distintos, entonces $m=n$, $b_k =c_k$ para cada valor de $k$, y $g(x)=f(x)$ para todo real $x$\\\\
                Demostración.- \; Sea $$p(x)=g(x)-f(x) = \displaystyle\sum_{k=0}^m b_k x^k - \sum_{k=0}^n c_k x^k = \sum_{k=0}^m (b_k - c_k)x^k$$ donde $c_k=0$ para $n< k \leq m$, cabe recordar que tenemos $m \geq n$.\\
                Entonces, hay $m+1$ distintos reales $x$ para los cuales $p(x)=0$. Dado que hay $m+1$ valores reales distintos para lo cuál $g(x)=f(x),$ así en cada uno de estos valores $p(x)=g(x)-f(x)=0$. Por lo tanto, por la parte $d)$, $b_k-c_k =0$ para $k=0,...,m$ y $p(x)=0$ para todo $x \in \mathbb{R}$. Es decir $$b_k - c_k =0 \; \; \; \Rightarrow \; \; \; b_k=c_k \; \; \; para \; k=0,...,m$$ y $$p(x)=0 \; \; \; \Rightarrow \; \; \; g(x)-f(x)=0 \; \; \; \Rightarrow \; \; \; f(x)=g(x),$$ para todo $x \in \mathbb{R}$. Ademas desde $b_k -c_k =0$ para $k=0,..., m$ y por supuesto $c_k=0$ para $k=n+1, ... , m,$ tenemos $b_k=0$ para $k= n+1, ... , m.$ Pero entonces, $$g(x)=\displaystyle\sum_{k=0}^n b_k x^k + \sum_{k=n+1}^m 0 \cdot x^k = \sum_{k=0}^n b^k x^k$$ significa que $g(x)$ es un polinomio de grado $n$ también.\\\\

            \end{enumerate}

        %--------------------10.-------------------
        \item En cada caso, hallar todos los polinomios $p$ de grado $\leq 2$ que satisfacen las condiciones dadas.\\\\
        Sabemos que para un polinomio de grado $\leq 2$ es:
        $$p(x) = ax^2 + bx + c$$ 
        para todo $a,b,c \in \mathbb{R}$.\\\\
            \begin{enumerate}[\bfseries (a)]
                %----------(a)-----------
                \item $p(x) = p(1-x)$\\\\
                Sea $f(x) = p(x) -1$, entonces $f$ es de grado como máximo $2$ por la parte $d)$ del problema $9$ tenemos que  todos los coeficientes de $f$ son $0$ \; y \; $f(x)=0$ para todo $x \in \mathbb{R}$, así,
                $$p(x)-1 = 0 \; \; \Rightarrow \; \; p(x)=1 \; \; \forall x \in \mathbb{R}$$\\\\ 

                %----------(b)-----------
                \item $p(x) = p(1+x)$\\\\
                Tenemos $p(0)=1 \Rightarrow c=1$ luego, $p(1)=1 \Rightarrow a+b=0 \Rightarrow b=-a$ y finalmente, con $c=1$ \; y \; $b=-a$, tenemos: $p(2)=2 \Rightarrow 4a-2a=1 \Rightarrow a=\dfrac{1}{2}, \; b=-\dfrac{1}{2}$. por lo tanto $$p(x)=\dfrac{1}{2}x^2 - \dfrac{1}{2}x + 1 = \dfrac{1}{2}x(x-1) + 1$$\\\\

                %----------(c)-----------
                \item $p(x) = p(0) = p(1) =1$\\\\
                Una vez mas, desde $p(0)=1$ tenemos: $a+b=0 \Rightarrow b=-a$ así, $p(x) = ax^2 - ax + 1 = ax(x-1) + 1$\\\\

                %----------(d)-----------
                \item $p(0) = p(1)$\\\\
                Simplemente sustituyendo estos valores que tenemos, $p(0)=p(1) \Rightarrow c=a+b+c \Rightarrow b=-a$ entonces, $$p(x) = ax^2 -ax + c = ax(x-1) + c$$\\\\

            \end{enumerate}
            
        %--------------------11.-------------------
        \item En cada caso, hallar todos los polinomios $p$ de grado $\leq 2$ que para todo real $x$ satisfacen las condiciones que se dan.
        Como $p$ es un polinomio de grado por lo mucho $2$, podemos escribir
        $$p(x) = ax^2 + bx + c, \; \; \; para \; a,b,c \in \mathbb{R}$$
            \begin{enumerate}[\bfseries (a)]
                %----------(a)----------
                \item $p(x) = p(1-x)$\\\\
                Sustituyendo se tiene $p(x) = p(1-x) = ax^2 + bx + c = a(1-x)^2 + b(1-x) + c \Rightarrow a - 2ax + ax^2 + b - bx + c$ por lo tanto $$ax^2 + (-2a-b)x + (a+b+c)$$
                Así para $a=a$, $b = -2a - b \Rightarrow a=-b$, $c=a+b+c$ entonces $$p(x) = -bx^2 + bx + c = bx(1-x)+c$$\\\\

                %----------(b)----------
                \item $p(x) = p(x) = p(1+x)$\\\\
                Una vez más sustituyendo, $p(x) = p(1+x) \Rightarrow ax^2 + bx + c = a(1+x)^2 + b(1+x) + c = ax^2 + (2a+b)x + (a+b+c)$. Luego, igualando como potencias de $x$, $a=a$, $b=2a+b \Rightarrow a = 0$, $c=a+b+c \Rightarrow b=0$. Por lo tanto $p(x) = c$ donde $c$ es una constante arbitraria.\\\\

                %----------(c)----------
                \item $p(2x) = 2p(x)$\\\\
                Sustituyendo, $p(2x) = 2p(x) \Rightarrow 4ax^2 + 2bx + c = 2ax^2 + 2bx +2c.$. Igualando a las potencias de $x$, $4a=2a \Rightarrow a=0,$ $2b=2b \Rightarrow  b \; arbitrario,$ $c=2c \Rightarrow c=0$.\\
                Así $$p(x)bx, \; b \; arbitrario$$\\\\

                %----------(d)----------
                \item $p(2x) = p(x+3)$\\\\
                Sustituyendo $p(3x) = p(x+3) \Rightarrow 9ax^2 + 3bx + c = ax^2 + (6a+b)x + (9a+3b+c)$. Igualando como potencias de $x,$ $9a=a \Rightarrow a=0,$ $3b = 6a+b \Rightarrow b=0,$ $c=9a + 3b + c = c \Rightarrow c \; arbitrario$. Por lo tanto $$p(x)=c \mbox{para c constante arbitrario.}$$\\\\

            \end{enumerate}

        %--------------------corolario 1.1-----------------------
        \begin{cor}Probar que:
            $$\displaystyle\sum_{k=0}^n x^k = \dfrac{1 - x^{n+1}}{1-x} \; \; para x\neq 1$$\\\\
            Demostración.- \; Usando propiedades de suma, $$(1-x)\displaystyle\sum_{k=0}^n x^k = \sum_{k=0}^n (x^k - x^{k+1}) = - \sum_{k=0}^n (x^{k+1} - x^k) = -(x^{n+1} -1 = 1 - x{n+1})$$
            En la penultima igualdad se deriva de la propiedad telescópica, por lo tanto nos queda,
            $$\displaystyle\sum_{k=0}^n x^k = \dfrac{1 - x^{n+1}}{1-x}$$ \\\\
        \end{cor}
                
        %--------------------corolario 1.2.-----------------------
        \begin{cor}Probar la identidad
            $$\displaystyle\prod_{k=1}^n \left( 1 + x^{2^{k-1}} \right) = \dfrac{1 - x^{2^n}}{1-x}, \; \; para \; x\neq 1$$\\\\
            Demostración.- \; Para $n=1$ a la izquierda tenemos,
            $$\displaystyle\prod_{k=1}^n \left( 1 + x^{2^{k-1}} \right) = \prod_{k=0}^1 \left( 1 + x^{2^{k-1}} = 1 + x^{2^0} = 1 + x  \right)$$
            Por otro lado a la derecha se tiene, $$\dfrac{1- x^{2^n}}{1-x} = \dfrac{1 - x^2}{1-x} = \dfrac{(1-x)(1+x)}{1-x} = 1+x$$
            Concluimos que la identidad se mantiene para $n=1$. Ahora supongamos que es válido para algunos $n=m \in \mathbb{Z}^+,$
                \begin{center}
                    \begin{tabular}{r c l}
                    $\prod\limits_{k=1}^{m+1}$&$=$&$\left( +x^{2^m} \right) \cdot \prod\limits_{k=1}^m \left( 1 + x^{2^{k+1}} \right)$\\\\
                    &$=$&$\left( 1 + x^{2^m} \right) \cdot  \left( \dfrac{1-x^{2^m}}{1-x} \right)$\\\\
                    &$=$&$\dfrac{(1+x^{2^m}) (1- x^{2^m})}{1-x}$\\\\
                    &$=$&$\dfrac{1 - x^{2^{m+1}}}{1-x}$\\\\
                    \end{tabular}
                \end{center} 
            Por lo tanto, la afirmación es verdadera para $m+1$, y así para todo $n \in \mathbb{Z}^+$\\\\
        \end{cor}

        %--------------------12.-------------------
        \item Demostrar que las expresiones siguientes son polinomios poniéndolas en la forma $\displaystyle\sum_{k=0}^m a_k x^k$ para un valor de $m$ conveniente. En cada caso $n$ es entero positivo.
            \begin{enumerate}[\bfseries (a)]
                %----------(a)----------
                \item $(1+x)^{2n}$\\\\
                Demostración.- \; Usando el teorema binomial $(1+x)^{2n} = \displaystyle\sum_{k=0}^2n {2n \choose k} x^k$, sea $m=2n$ entonces $\displaystyle\sum_{k=0}^m {m \choose n} x^k$, por lo tanto $\displaystyle\sum_{k=0}^m c_k x^k$ si $c_k = {m \choose k} $ para cada $k$.\\\\

                %----------(b)----------
                \item $\dfrac{1- x^{n+1}}{1-x}, \; x \neq 1$\\\\
                Demostración.- \; Por el corolario anterior 
                    \begin{center}
                        \begin{tabular}{r c l}
                            $\dfrac{1 - x^{n+1}}{1-x}$&=&$\dfrac{(1-x)(1+x+...+x^n)}{1-x}$\\\\
                            &=&$1 + x + ... + x^n$\\\\
                            &=&$\displaystyle\sum_{k=0}^n 1 \cdot x^k$\\\\
                        \end{tabular}
                    \end{center}

                %----------(c)----------
                \item $\displaystyle\prod_{k=0}^n (1+x^{2^k})$\\\\
                Demostración.- \; Por le corolario anterior,
                    \begin{center}
                        \begin{tabular}{r c l}
                            $\displaystyle\prod_{k=0}^n \left( 1+x^{2^k} \right)$&=&$\dfrac{(1 - x^{2^{n+1}})}{1-x}$\\
                            &=&$\dfrac{(1-x^{2^n})(1+x^{2^n})}{1-x}$\\\\
                            &=&$\left( \dfrac{1 - x^{2^n}}{1-x} \right) (1+x^{2^n})$\\\\
                            &=&$(1+x+...+x^{2^n + 1})(1 + x^{2^n})$\\\\
                            &=&$(1+x+...+x^{2^n + 1})(x^{2^n} + x^{2^n +1} + ... + x^{2^{n+1} - 1})$\\\\
                            &=&$\displaystyle\sum_{k=0}^{2^{n+1} - 1} 1 \cdot x^k$\\\\
                            &=&$\displaystyle\sum_{k=0}^{m} 1 \cdot x^k$ si $m = 2^{n+1} - 1$\\\\
                        \end{tabular}
                    \end{center}

            \end{enumerate}

    \end{enumerate}

    %-------------------axioma .1
	\begin{axioma}[Definición axiomática de área]
	    Supongamos que existe una clase $M$ de conjuntos del plano medibles y una función de conjunto $a$, cuyo dominio es $M$, con las propiedades siguientes:
	    \begin{enumerate}[\bfseries 1.]
		\item \textbf{Propiedad de no negatividad}. Para cada conjunto $S$ de $M$, se tiene $a(S)\geq 0$
		\item \textbf{Propiedad aditiva}. Si $S$ y $T$ pertenecen a $M$, también pertenecen a $M$, $S \cup T$ y $S \cap T,$ y se tiene $$a(S \cup T)=a(S)+a(T)-a(S\cap T)$$
		\item \textbf{Propiedad de la diferencia}. Si $S$ y $T$ pertenecen a $M$ siendo $S \subseteq T$ entonces $T - S$ está en $M$, y se tiene $a(T-S)=a(T)-a(S)$ 
		\item \textbf{Invariancia por congruencia}. Si un conjunto $S$ pertenece a $M$ y $T$ es congruente a $S$, también $T$ pertenece a $M$ y tenemos $a(S)=a(T)$
		\item \textbf{Elección de escala} Todo rectángulo $R$ pertenece a $M$. Si los lados de $R$ tienen longitudes $h$ y $k$, entonces $a(R)=hk$
		\item \textbf{Propiedad de exhaución}. Sea $Q$ un conjunto que puede encerrarse entre dos regiones $S$ y $T$ de modo que $$S\subseteq Q \subseteq T.$$ Si existe uno y sólo un número $c$ que satisface las desigualdades $$a(S)\leq c \leq a(T)$$ para todas la regiones escalonadas $S$ y $T$ que satisfacen $S\subseteq Q \subseteq T$, entonces $Q$ es medible y $a(Q)=c$

	    \end{enumerate}

	\end{axioma}
    

\setcounter{section}{6}
\section{Ejercicios}

    \begin{enumerate}[ \bfseries 1.]

	%--------------------1.
	\item Demostrar que cada uno de los siguientes conjuntos es medible y tiene área nula:

	    \begin{enumerate}[\bfseries (a)]

	    %----------(a)
	    \item Un conjunto que consta de un solo punto.\\\\
	    Demostración.-\; Un sólo punto se puede medir con un área $0$, ya que un punto es un rectángulo con $h=k=0$\\\\

	    %----------(b)
	    \item El conjunto de un número finito de puntos.\\\\
		Demostración.-\; Demostraremos por inducción en $n$, el número de puntos. Para el caso de $n=1$ ya quedo demostrado en el anterior inciso. Supongamos que es cierto para algunos $n=k\in \mathbf{Z}^+$. Entonces, tenemos un conjunto $S \in M$ de $k$ puntos en el plano y $a(S)=0$. Sea $T$ un punto en el plano. Por $(a)$ $T \in M$ y $a(T)=0$, por tanto por la propiedad aditiva, $$S\cup T \in M \;\; y \;\; a(S\cup T)=a(S)+a(T) - a(S\cap T).$$ pero $S\cap T \subseteq S,$ entonces $$a(S \cap T)\leq a(S) \Rightarrow a(S \cap T)\leq 0 \Rightarrow a(S \cap T)=0.$$\\ El axioma 1 nos garantiza que $a(S \cap T)$ no puede ser negativo. Por lo tanto, $a(S \Cup T)=0,$ Por tanto, el enunciado es verdadero para $k+1$ puntos en un plano y, por tanto, para todo $n \in Z_{>0}$\\\\

	    %----------(c)
	    \item La reunión de una colección finita de segmentos de recta en un plano.\\\\
	    Demostración.-\; Por inducción, sea $n$ el número de segmentos en un plano. Para $n=1$, dejamos $S$ ser un conjunto con una línea en un plano. Dado que una línea es un rectángulo y todos los rectángulos son medibles, tenemos $S\in M$ ademas, $a(S)=0$ ya que una línea es un rectángulo con $h=0$ ó $k=0$, y así en cualquier caso $hk=0$. Por lo tanto, el enunciado es verdadero para una sola línea en el plano, el caso $n=1$.\\
		    Asuma entonces que es cierto para $n=k \in \mathbf{Z}^+$. Sea $S$ un conjunto de rectas en el plano. Luego, por la hipótesis de inducción, $S\in M$ y $a(S)=0$. Sea $T$ una sola línea en el plano. Por el caso $n=1$ en $T\in M$ y $a(T)=0$. Por lo tanto $S\cup T \in M$ y $a(S\cup T)=0$ (ya que $a(S)=a(T)a(S\cap T)=0$). Por tanto, la afirmación es verdadera para $k+1$ líneas en un plano, y así para todos $n\in \mathbf{Z}^+$\\\\

	    \end{enumerate}

	%--------------------2.
	\item Toda región en forma de triángulo rectángulo es medible pues puede obtenerse como intersección de dos rectángulos. Demostrar que toda región triangular es medible y que su área es la mitad del producto de su base por su altura.\\\\
	    Demostración.-\; Dado que cada triángulo rectángulo es medible, por el axioma 2 del área su unión es medible, denotando los dos triángulos rectángulos $A$ y $B$, y la región triangular $T$, tenemos $$a(T)=a(A)+a(B)$$ ya que  $A$ y $B$ son disjuntos $a(A\cap B)=0$.\\
	    Dejando que la altitud de la región triangular se denote por $h$, y su base por $b$, tendremos,
	    $$a(A)=\dfrac{1}{2}(hb_1)\;\;\;\; a(B)=\dfrac{1}{2}hb_2 \;\;\; con \;\;\; b_1+b_2=b,$$ entonces $$a(T)=\dfrac{1}{2}hb_1+ \dfrac{1}{2}hb_2=\dfrac{1}{2}h(b_1+b_2)=\dfrac{1}{2}hb$$\\\\

	%--------------------3.
	\item Demostrar que todo trapezoide y todo paralelogramo es medible y deducir las fórmulas usuales para calcular su área.\\\\
	    Demostración.-\; Todo trapecio es medible ya que, la unión de un rectángulo y dos triángulos rectángulos (disjuntos por pares y cada uno de los cuales es medible ppor los axiomas y el ejercicio anterior.)\\
	    Luego su área es la suma de las áreas de los triángulos rectángulos y el rectángulo (dado que están separados por pares, su intersección tiene un área cero). Para calcular esta área, especificamos las longitudes de los dos lados desiguales del trapezoide para que sean $b_1$ y $b_2$. La altura está indicada por $a$. Entonces, el área del rectángulo es de $1$. El área de los triángulos es $\dfrac{1}{2} a \cdot b_3$ y $\frac{1}{2} a \cdot b_4$ dónde $b_1 + b_3 + b_4 = b_2$. Entonces, denotando el trapezoide por $T$, tenemos $$a(T)=ab_1+\dfrac{1}{2}ab_3 + \dfrac{1}{2}ab_4=\dfrac{1}{2}ab_1 + \dfrac{1}{2}a(b_1 + b_3 + b_3)=\dfrac{1}{2}a(b_1+b_2)$$ A continuación, un paralelogramo es solo un caso especial de un trapezoide, en el que $b_1 = b_2$; por lo tanto, por la fórmula anterior, y denotando el paralelogramo por $P$, $$a(P)=\dfrac{1}{2}a(2b)=ab$$\\\\

	%--------------------4.
	\item Un punto $(x,y)$ en el plano se dice que es un punto de una red, si ambas coordenadas $x$ e $y$ son enteras. Sea $P$ un polígono cuyos vértices son puntos de una red. El área de $P$ es $I+\dfrac{1}{2} B - 1$ donde $I$ es el número de puntos de la red interiores a $P$, y $B$ el de los de la frontera.\\\\

	    \begin{enumerate}[\bfseries (a)]

		%----------(a)
		\item Probar que esta fórmula es correcta para rectángulos de lados paralelos a los ejes coordenados.\\\\
		Demostración.-\; Sea $R$ un $h \times k$ rectángulo con lados paralelos a los ejes de coordenadas. Entonces, $R$ es medible (ya que es un rectángulo) y $a(R) = hk.$
A continuación, dado que los vértices están en puntos de celosía, $B = 2 (h + 1) + 2 (k + 1) - 4$ y $I = (h-1) (k-1)$. Por lo tanto,
		\begin{center}
		    \begin{tabular}{rcl}
		    $I+\dfrac{1}{2}B-1$ & $=$ & $(h-1)(k-1)+\dfrac{1}{2}\left[2(h+1)+2(k+1)-4\right]-1$ \\\\
		     & $=$ & $hk-h-k+1+h+1+k+1-2-1$ \\\\
		     & $=$ & $hk$ \\\\
		    \end{tabular}
		\end{center}

		%----------(b)
		\item Probar que la fórmula es correcta para triángulos rectángulos y paralelogramos.\\\\
		Demostración.-\; Sabemos que cualquier triángulo rectángulo puede encerrarse en un rectángulo con bordes cuyas longitudes sean iguales a las longitudes de los catetos del triángulo rectángulo. Además, este rectángulo está compuesto por dos triángulos rectángulos congruentes unidos a lo largo de su diagonal. Cada uno de estos triángulos rectángulos tiene un área la mitad de la del rectángulo y se cruzan a lo largo de la diagonal (que tiene un área cero (1.7, problema 1) ya que es una línea en el plano). Dado un triángulo rectángulo $T$, $R$ sea tal rectángulo, y $S$ sea el triángulo rectángulo que forma la otra mitad de $R$, entonces $S\cup T = R$.\\ 
		Dado que $R$ es un rectángulo, sabemos por la parte $(a)$ que $$a(R)=I_R + \dfrac{1}{2} B_R-1.$$
		Además, cualquier punto interior $R$ será un punto interior de cualquiera $S$ o $T$, o se acuesta sobre su frontera compartida. Por lo tanto, $$I_R = I_S + I_T + H_P$$ donde $H_P$ denota los puntos en la hipotenusa (compartida) de los dos triángulos rectángulos. Entonces, también tenemos para los puntos límite, $$B_R = B_S + B_T - 2 - 2H_P. $$  Finalmente, dado que $S$ y $T$ son congruentes, conocemos $B_S = B_T$ y $I_S = I_T.$ Entonces, poniendo todo esto junto, tenemos,
		\begin{center}	    
		    \begin{tabular}{rcl}
			$a(R)$ & $=$ & $I_R + \dfrac{1}{2} B_R -1$ \\\\
			& $=$ & $2I_S + H_P + \dfrac{1}{2} (2B_S - 2 - 2H_P) - 1 $ \\\\ 
			& $=$ & $2 (I_S + \dfrac{1}{2} B_S - 1)$\\\\
		    \end{tabular}
		\end{center}
		ó, $$I_S + \dfrac{1}{2} B_S - 1 = \dfrac{1}{2} a(R).$$
		Pero, sabemos que $\dfrac{1}{2} a(R) = a (S)$; por lo tanto, $$a(S)= I_S + \dfrac{1}{2} B_S -1.$$
		Esto prueba el resultado para triángulos rectángulos con vértices en puntos de una red.\\\\

		%----------(c)
		\item Emplear la inducción sobre el número de lados para construir una demostración para polígonos en general.\\\\
		Respuesta.-\; Ya tenemos esto de la parte $(b)$ ya que podemos realizar cualquier polígono simple como la unión de un número finito de triángulos rectángulos (es decir, cada polígono simple es triangularizable)\\\\

	    \end{enumerate}

	%--------------------5.
	\item Demostrar que un triángulo cuyos vértices son puntos de una red no puede ser equilátero.\\\\
	Demostración.-\; Supongamos que existe tal triángulo equilátero $T$. Entonces, $$T = A \ cup B$$ 
	Para dos triángulos rectángulos congruentes y disjuntos $A$, $B$. Dado que los vértices de $T$ están en puntos de una red, sabemos que la altitud desde el vértice hasta la base debe pasar por $h$ puntos de red (donde $h$ es la altura de $T$). Por lo tanto, al denotar los puntos de red en esta altitud por $V_B = h + 1$, tenemos
	$$B_T = B_A + B_B -V_B + 2, \qquad I_T = I_A + I_B + V_B - 2.$$ 
	Dado que $T$ es un polígono con vértices de puntos de red, sabemos por el ejercicio anterior que $a(T) = I_T + \dfrac{1}{2} B_T -1$. Además, por el problema $2$, sabemos que $a(T) = \dfrac{1}{2} bh$. Así que,
	\begin{center}
	    \begin{tabular}{crclr}
		& $I_T + \dfrac{1}{\2} B_T - 1$ & $=$ & $(I_A + I_B + V_B - 2) + \dfrac{1}{2}(B_A + B_B - V_B + 2)$ &\\\\
		$\Rightarrow$ & $I_T + \dfrac{1}{2} B_T - 1$ & $=$ & $2I_A + B_A - 2 + \dfrac{1}{2} V_B$ & $(B_A=B_B, \,\, I_A=I_B)$\\\\
		$\Rightarrow$ & $I_T + \dfrac{1}{2} B_T - 1$ & $=$ & $2I_A + B_A - 2 + \dfrac{1}{2}(h+1)$ & $(V_B = h + 1)$\\\\
		$\Rightarrow$ & $I_T + \dfrac{1}{2} B_T - 1$ & $=$ & $2(a(A)) + \dfrac{1}{2}(h+1)$ &\\\\
	    \end{tabular}
	\end{center}
	Pero, $\dfrac{1}{2} a(T)=a(A)=a(B)$ así,
	$$I_T + \dfrac{1}{2} B_T - 1 = a(T) + \dfrac{1}{2}(h+1) \qquad \Rightarrow \qquad a(T)=a(T) + \dfrac{1}{2}(h+1)$$
	Pero, $h > 0$ entonces esto es una contradicción. Por lo tanto,$T$ no puede tener sus vértices en puntos de red y ser equilátero.\\\\

	%--------------------6.
	\item Sean $A=\lbrace 1,2,3,4,5 \rbrace$ y $M$ la clase de todos los subconjuntos de $A$. (Son en número de $32$ contando el mismo $A$ y el conjunto vacio $\emptyset$.) Para cada conjunto $S$ de $M$, representemos con $n(S)$ el número de elemento distintos de $S$. Si $S=\lbrace 1,2,3,4 \rbrace$ y $T=\lbrace 3,4,5 \rbrace$, calcular $n(S \cup T)$, $n(S\cup T)$, $n(S-T)$ y $n(T_S)$. Demostrar que la función de conjunto $n$ satisface los tres primeros axiomas del área.\\\\
	Demostración.-\; Calculemos,
	\begin{center}
	    \begin{tabular}{rcrcl}
		$n(S \cup T)$ & $=$ & $n\left(\lbrace 1,2,3,4,5\rbrace \right)$ & $=$ & $5$\\
		$n(S \cap T)$ & $=$ & $n\left( \lbrace 3,4 \rbrace \right)$ & $=$ & $2$\\
		$$ & $=$ & $n\left(\lbrace 1,2 \rbrace\right)$ & $=$ & $2$\\
		$$ & $=$ & $n\left(\lbrace 5 \rbrace\right)$ & $=$ & $1$\\
	    \end{tabular}
	\end{center}
	Ahora demostremos que esto satisface los primeros tres axiomas de área.\\
	\textbf{Axioma 1}. (Propiedad no negativa) Esto se satisface para cualquier conjunto, $S$ ya que el número de elementos distintos en un conjunto no es negativo. Entonces, $n(S) \geq 0$ para todos $S$.\\
\textbf{Axioma 2}. (Propiedad aditiva) Primero, si $S$, $T \in \ mathcal{M}$, luego $S \subseteq A$, $T \subseteq A$ por definición de $\mathcal{M}$. Entonces, para cualquiera $x \in S$ que tengamos $x \in A$ y para cualquiera $y \in T$, tenemos $y \in A$.\\
	Así, si $x \in S \cup T$, entonces $x \in A$; por lo tanto $S \cup T \subseteq A$, entonces $S \cup T \in \mathcal{M}$.\\
	Entonces, $S \cap T \subseteq S$ implica $S \cap T \subseteq A$ (desde $S \subseteq A$). Por lo tanto, $S \cap T \in \mathcal{M}.$\\
	Entonces, para cualquiera $S, T \in \mathcal{M}$ que tengamos $S \cup T \in \mathcal{M},$  $S \cap T \in \mathcal{M}$.\\
	Luego, debemos mostrar $n(S \cup T) = n(S) + n(T) - n(S \cap T)$. Para cualquier $x \in S \cup T$ tenemos $x \in S$, $x \in T$, ó $x \in S$ y $T$. Entonces, esto significa $x \in (S - T)$, ó $x \in (T - S)$ ó $x \in (S \cap T)$. Por lo tanto,
	$$n(S\cup T)=n(S-T) + n(T-S) + n(S \cap T)$$
	Del mismo modo observamos,
	\begin{center}
	    \begin{tabular}{r c l}
		$n(S) = (S-T) + n(S\cap T)$ & $\Rightarrow$ & $n(S-T) = n(S) -s(S\cap T)$\\
		$n(T) = n(T-S) + n(T\cap S)$ & $\Rightarrow$ & $n(T-S)=n(T) - n(S\cap T)$\\
	    \end{tabular}
	\end{center}
	Así que, 
	\begin{center}
	    \begin{tabular}{rcl}
		$n(S\cup T)$ & $=$ & $n(S) -n(S\cap T) + n(T) -n(S\cap T) +n(S\cap T)$\\
		 & $=$ & $n(S) +n(T) -n(S\cap T)$\\
	    \end{tabular}
	\end{center}
	\textbf{Axioma 3} (Propiedades de la diferencia). Si $S,T \in \mathcal{M}$ y $S\subseteq T$, entonces desde arriba tenemos $$n(T-S)=n(T) -n(T\cap S)$$
	Pero porque $S\subseteq T$ sabemos $T\cap S = S,$ entonces,
	$$n(T-S)=n(T)-n(S)$$\\\\

    \end{enumerate}

\section{Intervalos y conjuntos ordenados}


        %-----------------------------1.5. definición Intervalo cerrado-----------------------------
        \begin{def.}[Intervalo cerrado]
	    Si $a<b$, se indica por $\left[a,b\right]$ el conjunto de todos los $x$ que satisfacen las desigualdades $a\leq x \leq b$.\\\\
        \end{def.}

	%-----------------------------1.6. definición Intervalo abierto
	\begin{def.}[Intervalo abierto]
	    El intervalo abierto correspondiente, indicado por $(a,b)$ es el conjunto de todos los $x$ que satisfacen $a<x<b$\\\\
	    El intervalo abierto $(a,b)$ se denomina también el interior de $\left[a,b\right]$ \\\\
	\end{def.}

	%-----------------------------1.7. definición Intervalo semiabiertos
	\begin{def.}[Intervalo semiabiertos]
	    Los intervalos semiabiertos $(a,b]$ y $[a,b)$ que incluyen sólo un extremo están definidos por las desigualdades $a<x\leq b$ y $a\leq x <b$, respectivamente.\\\\
	\end{def.}


\section{Particiones y funciones escalonadas}

	%-----------------------------1.8 definición
	\begin{def.}
	    Un conjunto de puntos que satisfaga $$a<x_1 < x_2 < ... < x_{n-1}<b$$ se denomina una partición $P$ de $\left[a,b\right]$, y se utiliza el símbolo: $$P=\lbrace x_0,x_1,...,x_n\rbrace$$ para designar tal partición, la partición $P$ determina $n$ subinterválos cerrados $$\left[x_0,x_1\right], \left[x_1,x_2\right],...,\left[x_{n-1}, x_n\right]$$\\\\
	\end{def.}

	%-----------------------------1.9 definición de función escalonada

	\begin{def.}[Definición de función escalonada]
	    Una función $s$ cuyo dominio es el intervalo cerrado $\left[a,b\right]$ se dice que es una función escalonada, si existe una partición $P=\lbrace x_o,x_1,...,x_n \rbrace$ de $\left[ a,b \right]$ tal que $s$ es constante en cada subintervalo abierto de $P$. Es decir, para cada $k=1,2,...,n$ existe un número real $S_k$ tal que: $$s(x)=s_K \qquad si \qquad x_{k-1}<x<x_k$$ A veces las funciones escalonadas se llaman funciones constantes a trozos.\\\\
	\end{def.}

\setcounter{section}{10}
\section{Ejercicios}

    En este conjunto de Ejercicios, $\left[x\right]$ representa el mayor entero $\leq x_i$ es decir, la parte entera de $x$.\\

    \begin{enumerate}[ \bfseries 1.]

	%--------------------1.
	\item Sean $f(x)=[x]$ y $g(x)=[2x]$ para todo real $x$. En cada caso, dibujar la gráfica de la función $h$ definida en el intervalo $[-1,2]$ por la fórmula que se da.

	\begin{enumerate}[\bfseries (a)]

	    %----------(a)
	    \item $h(x)=f(x)+g(x)$.\\\\
		Respuesta.-\; $h(x)=[x] + [2x]$
		\begin{center}
		    \begin{tikzpicture}
			% abscisa y ordenada
			\tkzInit[xmax= 2,xmin=-2,ymax=4,ymin=-4]
			\tiny\tkzLabelXY[opacity=0.6,step=1, orig=false]
			% label x, f(x)
			\tkzDrawX[opacity= .6,label=x,right=0.3]
			\tkzDrawY[opacity= .6,label=f(x),below = -0.6]
			%funciones
			\draw(-1,-3)--(-.53,-3);
			\draw(-.5,-2)--(-0.03,-2);
			\draw(0,0)--(.53,0);
			\draw(.5,1)--(.97,1);
			\draw(1,3)--(1.47,3);
			\draw(1.5,4)--(1.97,4);
			%puntos
			\filldraw[black](-1,-3) circle(1pt);
			\draw(-.5,-3)node[]{$\circ$};
			\filldraw[black](-.5,-2) circle(1pt);
			\draw(0,-2)node[]{$\circ$};
			\filldraw[black](0,0) circle(1pt);
			\draw(.5,0)node[]{$\circ$};
			\filldraw[black](.5,1) circle(1pt);
			\draw(1,1)node[]{$\circ$};
			\filldraw[black](1,3) circle(1pt);
			\draw(1.5,3)node[]{$\circ$};
			\filldraw[black](1.5,4) circle(1pt);
			\draw(2,4)node[]{$\circ$};
		    \end{tikzpicture}
		\end{center}

	    %----------(b)
	    \item $h(x)=f(x)+g(x/2)$.\\\\
		Respuesta.-\; $h(x)=[x]+[x]=2[x]$
		\begin{center}
		    \begin{tikzpicture}
			% abscisa y ordenada
			\tkzInit[xmax= 2,xmin=-2,ymax=3,ymin=-3]
			\tiny\tkzLabelXY[opacity=0.6,step=1, orig=false]
			% label x, f(x)
			\tkzDrawX[opacity= .6,label=x,right=0.3]
			\tkzDrawY[opacity= .6,label=f(x),below = -0.6]
			%funciones
			\draw(-1,-2)--(-0.03,-2);
			\draw(0,0)--(.97,0);
			\draw(1,2)--(1.97,2);
			%puntos
			\filldraw[black](-1,-2) circle(1pt);
			\draw(0,-2)node[]{$\circ$};
			\filldraw[black](0,0) circle(1pt);
			\draw(1,0)node[]{$\circ$};
			\filldraw[black](1,2) circle(1pt);
			\draw(2,2)node[]{$\circ$};
		    \end{tikzpicture}
		\end{center}

	    %----------(c)
	    \item $h(x)=f(x)g(x)$.\\\\
		Respuesta.-\; $h(x)=[x]\cdot [2x]$
		\begin{center}
		    \begin{tikzpicture}
			% abscisa y ordenada
			\tkzInit[xmax= 2,xmin=-2,ymax=4,ymin=-1]
			\tiny\tkzLabelXY[opacity=0.6,step=1, orig=false]
			% label x, f(x)
			\tkzDrawX[opacity= .6,label=x,right=0.3]
			\tkzDrawY[opacity= .6,label=f(x),below = -0.6]
			%funciones
			\draw(-1,2)--(-0.53,2);
			\draw(-.5,1)--(-.03,1);
			\draw(0,0)--(0.97,0);
			\draw(1,2)--(1.47,2);
			\draw(1.5,3)--(1.97,3);
			%puntos
			\filldraw[black](-1,2) circle(1pt);
			\draw(-.5,2)node[]{$\circ$};
			\filldraw[black](-.5,1) circle(1pt);
			\draw(0,1)node[]{$\circ$};
			\filldraw[black](0,0) circle(1pt);
			\draw(1,0)node[]{$\circ$};
			\filldraw[black](1,2) circle(1pt);
			\draw(1.5,2)node[]{$\circ$};
			\filldraw[black](1.5,3) circle(1pt);
			\draw(2,3)node[]{$\circ$};
		    \end{tikzpicture}
		\end{center}

	    %----------(d)
	    \item $h(x)=\frac{1}{4}f(2x) g(x/2)$.\\\\
		Respuesta.-\; $\frac{1}{4}[x][2x]$
		\begin{center}
		    \begin{tikzpicture}
			% abscisa y ordenada
			\tkzInit[xmax= 2,xmin=-2,ymax=1,ymin=-1]
			\tiny\tkzLabelXY[opacity=0.6,step=1, orig=false]
			% label x, f(x)
			\tkzDrawX[opacity= .6,label=x,right=0.3]
			\tkzDrawY[opacity= .6,label=f(x),below = -0.6]
			%funciones
			\draw(-1,.5)--(-0.53,.5);
			\draw(-.5,.25)--(-.03,.25);
			\draw(0,0)--(0.97,0);
			\draw(1,.5)--(1.47,.5);
			\draw(1.5,.75)--(1.97,.75);
			%puntos
			\filldraw[black](-1,.5) circle(1pt);
			\draw(-.5,.5)node[]{$\circ$};
			\filldraw[black](-.5,.25) circle(1pt);
			\draw(0,.25)node[]{$\circ$};
			\filldraw[black](0,0) circle(1pt);
			\draw(1,0)node[]{$\circ$};
			\filldraw[black](1,.5) circle(1pt);
			\draw(1.5,.5)node[]{$\circ$};
			\filldraw[black](1.5,.75) circle(1pt);
			\draw(2,.75)node[]{$\circ$};
		    \end{tikzpicture}
		\end{center}

	\end{enumerate}

	%--------------------2.
	\item En cada uno de los casos, $f$ representa una función definida en el intervalo $[-2,2]$ por la fórmula que se indica. Dibújense las gráficas correspondientes a cada una de las funciones $f$. Si $f$ es una función escalonada, encontrar la partición $P$ de $[-2,2]$ tal que $f$ es constante en los subintervalos abierto de $P$.

	\begin{enumerate}[\bfseries (a)]
	    
	    %----------(a)
	    \item  $f(x)=x+[x]$\\\\
		Respuesta.-\; No es una función escalonada.
		\begin{center}
		    \begin{tikzpicture}
			% abscisa y ordenada
			\tkzInit[xmax= 2,xmin=-2,ymax=3,ymin=-4]
			\tiny\tkzLabelXY[opacity=0.6,step=1, orig=false]
			% label x, f(x)
			\tkzDrawX[opacity= .6,label=x,right=0.3]
			\tkzDrawY[opacity= .6,label=f(x),below = -0.6]
			%funciones
			\draw(-2,-4)--(-.97,-3);
			\draw(-1,-2)--(-.03,-1);
			\draw(0,0)--(0.97,1);
			\draw(1,2)--(1.97,3);
			%puntos
			\filldraw[black](-2,-4) circle(1pt);
			\draw(-1,-3)node[]{$\circ$};
			\filldraw[black](-1,-2) circle(1pt);
			\draw(0,-1)node[]{$\circ$};
			\filldraw[black](0,0) circle(1pt);
			\draw(1,1)node[]{$\circ$};
			\filldraw[black](1,2) circle(1pt);
			\draw(2,3)node[]{$\circ$};
		    \end{tikzpicture}
		\end{center}
	    
	    %----------(b)
	    \item  $f(x)=x-[x]$ \\\\
	    Respuesta.-\; No es una función escalonada.
	    \begin{center}
		\begin{tikzpicture}
		    % abscisa y ordenada
		    \tkzInit[xmax= 2,xmin=-2,ymax=1,ymin=-1]
		    \tiny\tkzLabelXY[opacity=0.6,step=1, orig=false]
		    % label x, f(x)
		    \tkzDrawX[opacity= .6,label=x,right=0.3]
		    \tkzDrawY[opacity= .6,label=f(x),below = -0.6]
		    %funciones
		    \draw(-2,0)--(-.97,1);
		    \draw(-1,0)--(-0.03,1);
		    \draw(0,0)--(0.97,1);
		    \draw(1,0)--(1.97,1);
		    %puntos
		    \filldraw[black](-2,0) circle(1pt);
		    \draw(-1,1)node[]{$\circ$};
		    \filldraw[black](-1,0) circle(1pt);
		    \draw(0,1)node[]{$\circ$};
		    \filldraw[black](0,0) circle(1pt);
		    \draw(1,1)node[]{$\circ$};
		    \filldraw[black](1,0) circle(1pt);
		    \draw(2,1)node[]{$\circ$};
		\end{tikzpicture}
	    \end{center}
	    
	    %----------(c)
	    \item  $f(x)=[-x]$ \\\\
	    Respuesta.-\; Esta es una función de paso y es constante en los subintervalos abiertos de la partición, $P=\lbrace -2,-1,0,1,2\rbrace$.
	    \begin{center}
		\begin{tikzpicture}
		    % abscisa y ordenada
		    \tkzInit[xmax= 2,xmin=-2,ymax=1,ymin=-2]
		    \tiny\tkzLabelXY[opacity=0.6,step=1, orig=false]
		    % label x, f(x)
		    \tkzDrawX[opacity= .6,label=x,right=0.3]
		    \tkzDrawY[opacity= .6,label=f(x),below = -0.6]
		    %funciones
		    \draw(-2,1)--(-.97,1);
		    \draw(-1,0)--(-0.03,0);
		    \draw(0,-1)--(0.97,-1);
		    \draw(1,-2)--(1.97,-2);
		    %puntos
		    \filldraw[black](-2,1) circle(1pt);
		    \draw(-1,1)node[]{$\circ$};
		    \filldraw[black](-1,0) circle(1pt);
		    \draw(0,0)node[]{$\circ$};
		    \filldraw[black](0,-1) circle(1pt);
		    \draw(1,-1)node[]{$\circ$};
		    \filldraw[black](1,-2) circle(1pt);
		    \draw(2,-2)node[]{$\circ$};
		\end{tikzpicture}
	    \end{center}
	    
	    %----------(d)
	    \item  $f(x)=2[x]$ \\\\
	    Respuesta.-\; Esta es una función de paso y es constante en los subintervalos abiertos de la partición, $P=\lbrace -2,-1,0,1,2 \rbrace$.
	    \begin{center}
		\begin{tikzpicture}
		    % abscisa y ordenada
		    \tkzInit[xmax= 2,xmin=-2,ymax=2,ymin=-4]
		    \tiny\tkzLabelXY[opacity=0.6,step=1, orig=false]
		    % label x, f(x)
		    \tkzDrawX[opacity= .6,label=x,right=0.3]
		    \tkzDrawY[opacity= .6,label=f(x),below = -0.6]
		    %funciones
		    \draw(-2,-4)--(-.97,-4);
		    \draw(-1,-2)--(-0.03,-2);
		    \draw(0,0)--(.97,0);
		    \draw(1,2)--(1.97,2);
		    %puntos
		    \filldraw[black](-2,-4) circle(1pt);
		    \draw(-1,-4)node[]{$\circ$};
		    \filldraw[black](-1,-2) circle(1pt);
		    \draw(0,-2)node[]{$\circ$};
		    \filldraw[black](0,0) circle(1pt);
		    \draw(.97,0)node[]{$\circ$};
		    \filldraw[black](1,2) circle(1pt);
		    \draw(2,2)node[]{$\circ$};
		\end{tikzpicture}
	    \end{center}
	    
	    %----------(e)
	    \item  $f(x)=[x+ \frac{1}{2}]$ \\\\
	    Respuesta.-\; Esta es una función de paso y es constante en los subintervalos abiertos de la partición, $P=\lbrace -2,-3/2,-1/2,1/2,3/2,2 \rbrace$
	    \begin{center}
		\begin{tikzpicture}
		    % abscisa y ordenada
		    \tkzInit[xmax= 2,xmin=-2,ymax=2,ymin=-2]
		    \tiny\tkzLabelXY[opacity=0.6,step=1, orig=false]
		    % label x, f(x)
		    \tkzDrawX[opacity= .6,label=x,right=0.3]
		    \tkzDrawY[opacity= .6,label=f(x),below = -0.6]
		    %funciones
		    \draw(-2,-2)--(-1.53,-2);
		    \draw(-1.5,-1)--(-0.53,-1);
		    \draw(-.5,0)--(0.53,0);
		    \draw(.5,1)--(1.53,1);
		    \draw(1.5,2)--(1.97,2);
		    %puntos
		    \filldraw[black](-2,-2) circle(1pt);
		    \draw(-1.5,-2)node[]{$\circ$};
		    \filldraw[black](-1.5,-1) circle(1pt);
		    \draw(-0.5,-1)node[]{$\circ$};
		    \filldraw[black](-.5,0) circle(1pt);
		    \draw(0.5,0)node[]{$\circ$};
		    \filldraw[black](.5,1) circle(1pt);
		    \draw(1.5,1)node[]{$\circ$};
		    \filldraw[black](1.5,2) circle(1pt);
		    \filldraw[black](2,2) circle(1pt);
		\end{tikzpicture}
	    \end{center}
	    
	    %----------(f)
	    \item  $f(x)=[x]+[x+\frac{1}{2}]$ \\\\
	    Respuesta.-\; Esta es una función de paso y es constante en los subintervalos abiertos de los partición, $P=\lbrace -2,-3/2,-1,-1/2,0,1/2,1,3/2,2 \rbrace$
	    \begin{center}
		\begin{tikzpicture}
		    % abscisa y ordenada
		    \tkzInit[xmax= 2,xmin=-2,ymax=3,ymin=-4]
		    \tiny\tkzLabelXY[opacity=0.6,step=1, orig=false]
		    % label x, f(x)
		    \tkzDrawX[opacity= .6,label=x,right=0.3]
		    \tkzDrawY[opacity= .6,label=f(x),below = -0.6]
		    %funciones
		    \draw(-2,-4)--(-1.53,-4);
		    \draw(-1.5,-3)--(-1.03,-3);
		    \draw(-1,-2)--(-0.53,-2);
		    \draw(-.5,-1)--(-0.03,-1);
		    \draw(0,0)--(.47,0);
		    \draw(.5,1)--(.97,1);
		    \draw(1,2)--(1.47,2);
		    \draw(1.5,3)--(1.97,3);
		    %puntos
		    \filldraw[black](-2,-4) circle(1pt);
		    \draw(-1.5,-4)node[]{$\circ$};
		    \filldraw[black](-1.5,-3) circle(1pt);
		    \draw(-1,-3)node[]{$\circ$};
		    \filldraw[black](-1,-2) circle(1pt);
		    \draw(-0.5,-2)node[]{$\circ$};
		    \filldraw[black](-.5,-1) circle(1pt);
		    \draw(0,-1)node[]{$\circ$};
		    \filldraw[black](0,0) circle(1pt);
		    \draw(.5,0)node[]{$\circ$};
		    \filldraw[black](.5,1) circle(1pt);
		    \draw(1,1)node[]{$\circ$};
		    \filldraw[black](1,2) circle(1pt);
		    \draw(1.5,2)node[]{$\circ$};
		    \filldraw[black](1.5,3) circle(1pt);
		    \draw(2,3)node[]{$\circ$};
		\end{tikzpicture}
	    \end{center}
	    \vspace{.5cm}

	\end{enumerate}

	%--------------------3.
	\item  En cada caso, dibujar la gráfica de la función definida por la fórmula que se da.\\\\

	\begin{enumerate}[\bfseries (a)]

	    %----------(a)
	    \item $f(x)=[\sqrt{x}]$ para $0\leq x \leq 10$\\\\
		Respuesta.-\;
		\begin{center}
		    \begin{tikzpicture}
			% abscisa y ordenada
			\tkzInit[xmax= 10,xmin=0,ymax=3,ymin=0]
			\tiny\tkzLabelXY[opacity=0.6,step=1, orig=false]
			% label x, f(x)
			\tkzDrawX[opacity= .6,label=x,right=0.3]
			\tkzDrawY[opacity= .6,label=f(x),below = -0.6]
			%funciones
			\draw(0,0)--(.53,0);
			\draw(.5,1)--(4.03,1);
			\draw(4,2)--(9.03,2);
			\draw(9,3)--(10,3);
			%puntos
			\filldraw[black](0,0) circle(1pt);
			\draw(.5,0)node[]{$\circ$};
			\filldraw[black](.5,1) circle(1pt);
			\draw(4,1)node[]{$\circ$};
			\filldraw[black](4,2) circle(1pt);
			\draw(9,2)node[]{$\circ$};
			\filldraw[black](9,3) circle(1pt);
			\filldraw[black](10,3) circle(1pt);
		    \end{tikzpicture}
		\end{center}
		\vspace{.5cm}

	    %----------(b)
	    \item $f(x)=[x^2]$\\\\
		Respuesta.-\;
		\begin{center}
		    \begin{tikzpicture}
			% abscisa y ordenada
			\tkzInit[xmax= 3,xmin=0,ymax=9,ymin=0]
			\tiny\tkzLabelXY[opacity=0.6,step=1, orig=false]
			% label x, f(x)
			\tkzDrawX[opacity= .6,label=x,right=0.3]
			\tkzDrawY[opacity= .6,label=f(x),below = -0.6]
			%funciones
			\draw(0,0)--(0.97,0);
			\draw(1,1)--(1.47,1);
			\draw(1.5,2)--(1.73,2);
			\draw(1.75,3)--(1.97,3);
			\draw(2,4)--(2.23,4);
			\draw(2.25,5)--(2.37,5);
			\draw(2.4,6)--(2.57,6);
			\draw(2.6,7)--(2.73,7);
			\draw(2.75,8)--(2.87,8);
			%puntos
			\filldraw[black](0,0) circle(1pt);
			\draw(1,0)node[]{$\circ$};
			\filldraw[black](-2,-4) circle(1pt);
			\draw(-1.5,-4)node[]{$\circ$};
			\filldraw[black](1,1) circle(1pt);
			\draw(1.5,1)node[]{$\circ$};
			\filldraw[black](1.5,2) circle(1pt);
			\draw(1.75,2)node[]{$\circ$};
			\filldraw[black](1.75,3) circle(1pt);
			\draw(2,3)node[]{$\circ$};
			\filldraw[black](2,4) circle(1pt);
			\draw(2.25,4)node[]{$\circ$};
			\filldraw[black](2.25,5) circle(1pt);
			\draw(2.4,5)node[]{$\circ$};
			\filldraw[black](2.4,6) circle(1pt);
			\draw(2.6,6)node[]{$\circ$};
			\filldraw[black](2.6,7) circle(1pt);
			\draw(2.75,7)node[]{$\circ$};
			\filldraw[black](2.75,8) circle(1pt);
			\draw(2.9,8)node[]{$\circ$};
			\filldraw[black](2.9,9) circle(1pt);
		    \end{tikzpicture}
		\end{center}

	    %----------(c)
	    \item $f(x)=\sqrt{[x]}$ para $0\leq x \leq 10.$\\\\
		Respuesta.-\;
		\begin{center}
		    \begin{tikzpicture}
			% abscisa y ordenada
			\tkzInit[xmax= 10,xmin=0,ymax=3,ymin=0]
			\tiny\tkzLabelXY[opacity=0.6,step=1, orig=false]
			% label x, f(x)
			\tkzDrawX[opacity= .6,label=x,right=0.3]
			\tkzDrawY[opacity= .6,label=f(x),below = -0.6]
			%funciones
			\draw(0,0)--(.97,0);
			\draw(1,1)--(1.97,1);
			\draw(2,1.3)--(2.97,1.3);
			\draw(3,1.8)--(3.97,1.8);
			\draw(4,2)--(4.97,2);
			\draw(5,2.2)--(5.97,2.2);
			\draw(6,2.4)--(6.97,2.4);
			\draw(7,2.6)--(7.97,2.6);
			\draw(8,2.8)--(8.97,2.8);
			\draw(9,3)--(9.97,3);
			%puntos
			\filldraw[black](0,0) circle(1pt);
			\draw(1,0)node[]{$\circ$};
			\filldraw[black](1,1) circle(1pt);
			\draw(2,1)node[]{$\circ$};
			\filldraw[black](2,1.3) circle(1pt);
			\draw(3,1.3)node[]{$\circ$};
			\filldraw[black](3,1.8) circle(1pt);
			\draw(4,1.8)node[]{$\circ$};
			\filldraw[black](4,2) circle(1pt);
			\draw(5,2)node[]{$\circ$};
			\filldraw[black](5,2.2) circle(1pt);
			\draw(6,2.2)node[]{$\circ$};
			\filldraw[black](6,2.4) circle(1pt);
			\draw(7,2.4)node[]{$\circ$};
			\filldraw[black](7,2.6) circle(1pt);
			\draw(8,2.6)node[]{$\circ$};
			\filldraw[black](8,2.8) circle(1pt);
			\draw(9,2.8)node[]{$\circ$};
			\filldraw[black](9,3) circle(1pt);
			\draw(10,3)node[]{$\circ$};
		    \end{tikzpicture}
		\end{center}
		\vspace{.5cm}

	    %----------(d)
	    \item $f(x)=[x]^2$ para $0 \leq x \leq 3.$\\\\
		Respuesta.-\;
		\begin{center}
		    \begin{tikzpicture}
			% abscisa y ordenada
			\tkzInit[xmax= 3,xmin=0,ymax=4,ymin=0]
			\tiny\tkzLabelXY[opacity=0.6,step=1, orig=false]
			% label x, f(x)
			\tkzDrawX[opacity= .6,label=x,right=0.3]
			\tkzDrawY[opacity= .6,label=f(x),below = -0.6]
			%funciones
			\draw(0,0)--(0.97,0);
			\draw(1,1)--(1.97,1);
			\draw(2,4)--(2.97,4);
			%puntos
			\filldraw[black](0,0) circle(1pt);
			\draw(1,0)node[]{$\circ$};
			\filldraw[black](1,1) circle(1pt);
			\draw(2,1)node[]{$\circ$};
			\filldraw[black](2,4) circle(1pt);
			\draw(3,4)node[]{$\circ$};
		    \end{tikzpicture}
		\end{center}
		\vspace{.5cm}

	\end{enumerate}
	
	%--------------------4.
	\item demostrar que la función parte entera  tiene las propiedades que se indican:\\\\

	\begin{enumerate}[\bfseries (a)]

	    %-----------(a)
	    \item $[x+n] = [x] + n$ para cada entero $n$.\\\\
		Demostración.-\; Por definición sea $[x+n]=m$ para $m\in \mathbb{Z}$,
		\begin{center}
		    \begin{tabular}{rcl}
			$m \leq x+n < m+1$ & $\Longrightarrow$ & $m-1 \leq x < m-n+1$\\
			 & $\Longrightarrow$ & $[x]=m-n$\\
			 & $\Longrightarrow$ & $[x]+n=m$\\\\
		    \end{tabular}
		\end{center}

	    %-----------(b)
	    \item $[-x] = \left\{ 
		    \begin{array}{ll} 
			-[x] & si \; x \; es \; entero \\ 
			-[x] - 1 & en \; otro \; caso\\
		    \end{array} 
		\right.$ \\\\
		Demostración.-\; Si $x \in \mathbb{Z},$ entonces $x=n$ para algunos $n \in \mathbb{Z}$. Por tanto, $[x]=n$, luego $$-x=-n \Longrightarrow [-x]=-n \Longrightarrow [-x] = -[x]$$
		Por otro lado, si $x \neq \mathbb{Z}$, entonces $[x]=n$. Luego $$n\leq x < n+1 \Longrightarrow -n-1 < -x < n \mbox{ ya que } n\neq x \Longrightarrow [-x] = -n -1 = -[x] - 1$$\\\\
 
	    %-----------(c)
	    \item $[x+y] = [x] + [y]$ ó $[x] + [y] + 1$\\\\
		Demostración.-\; Sea $[x]=m$ y $[y]=n$, luego,
		$$m\leq x < m+1 \qquad y \qquad n\leq y < n+1$$ Entonces, sumando obtenemos $$m+n \leq x+y < m+n+2$$  por lo tanto $$[x+y]=m+n=[x]+[y] \qquad o \qquad [x+y] = m+n+1=[x]+[y]+1$$ Esto ya que si $x+y$ está entre $m+n$ y $m+n+1$ entonces $[x+y]=[x] + [y] + 1$ y cuando $x+y$ está entre $m+n+1$ y $m+n+2$ entonces $[x+y] = [x] + [y] + 1.$\\\\ 

	    %-----------(d)
	    \item  $[2x] = [x] + [x + \frac{1}{2}$\\\\
		Demostración.-\; Por la parte $(c)$  $$[2x]=[x+x] = [x] + [x] \qquad ó \qquad [x]+[x]+1$$ \\
		Para $[2x] = [x] + [x]$, sea $[x]=n$, entonces 
		\begin{center}
		    \begin{tabular}{rcl}
			$[2x]=2n$&$\Longrightarrow$&$2n\leq 2x \leq 2n+1$\\\\
			&$\Longrightarrow$&$n\leq x < n + \dfrac{1}{2}$\\\\
			&$\Longrightarrow$&$n \leq x + \dfrac{1}{2}<n+1$\\\\
			&$\Longrightarrow$&$\left[x + \dfrac{1}{2}\right] = n$\\\\
		    \end{tabular}
		\end{center}
		de donde, $[2x]=2n=n+n=[x]+\left[x + \dfrac{1}{2}\right]$\\\\

		Por otro lado, para $[2x] = [x] + [x] + 1,$ sea $[x]=n$, entonces:
		\begin{center}
		    \begin{tabular}{rcl}
			$[2x]=2n+1$&$\Longrightarrow$&$2n+1 \leq 2n + 2$\\\\
			&$\Longrightarrow$&$n+\dfrac{1}{2}\leq x < n+1$\\\\
			&$\Longrightarrow$&$n+1 \leq x+\dfrac{1}{2} < n+2$\\\\
			&$\Longrightarrow$&$\left[x+ \dfrac{1}{2}\right]$= n+1\\\\
		    \end{tabular}
		\end{center}
		de donde $[2x]=n+n+1=[x]+\left[x+\dfrac{1}{2}\right]$\\\\

	    %-----------(e)
	    \item $[3x] = [x] + [x + \frac{1}{2}] + [x + \frac{2}{3}]$\\\\
		Demostración.-\; Por la parte $(c)$ tenemos $$[3x]=[x+x+x]=[x+x]+[x] \quad \mbox{o} \quad [x+x] + [x] + 1$$ 
		$$\mbox{y}$$ 
		$$[x+x]=[x]+[x] \quad \mbox{o} \quad [x]+[x]+1.$$   
		de donde al juntarlos obtenemos:
		$$[3x]=[x]+[x]+[x] \qquad ó \qquad [x]+[x]+[x]+1 \qquad ó \qquad [x]+[x]+[x]+2$$ 
		Para $3x=[x]+[x]+[x]$ sea $[x]=n$ entonces
		\begin{center}
		    \begin{tabular}{rcl}
			$3n\leq 3x<3x+1$&$\Longrightarrow$&$n\leq x<n+\dfrac{1}{3}$\\\\
			&$\Longrightarrow$&$n\leq x+\dfrac{1}{3} < n+1 \qquad y \qquad n \leq x+\dfrac{2}{3} < n+1$\\\\
			&$\Longrightarrow$&$[x]=\left[x+\dfrac{1}{3}\right] = \left[x+\dfrac{2}{3}\right] = n$\\\\
		    \end{tabular}
		\end{center}
		Por lo tanto, $[x]=3n=[x]+\left[x+\dfrac{1}{3}\right] + \left[x +\dfrac{2}{3} \right]$\\\\
		Luego para $[3x]=[x]+[x]+[x] + 1$ sea $[x]=n$ entonces,

		\begin{center}
		    \begin{tabular}{rcl}
			$3n+1 \leq 3x < 3n + 1$&$\Longrightarrow$&$n+\dfrac{1}{3} \leq x < n + \dfrac{2}{3}$\\\\
			&$\Longrightarrow$&$n+\dfrac{1}{3} \leq x+\dfrac{1}{3} < n+1 \quad \Rightarrow \quad \left[x + \dfrac{1}{3} \right] = n$\\\\
			y&$\Longrightarrow$&$n+1\leq x + \dfrac{2}{3} < n+2 \quad \Rightarrow \quad \left[x+\dfrac{2}{3}\quad \right]= n+1$\\\\
		    \end{tabular}
		\end{center}
		Por lo tanto $[3x]=3n+1 \quad \Longrightarrow \quad [3x]=[x] + \left[x+\dfrac{1}{3}\right] + \left[x+\dfrac{2}{3}\right]$\\\\

		Finalmente, para $[3x]=[x]+[x]+[x]+2$ sea $[x]=n$ entonces
		\begin{center}
		    \begin{tabular}{rcl}
			$3x+2\leq 3x < 3x+3$&$\Longrightarrow$&$n+\dfrac{2}{3} \leq x < n+1$\\\\
			&$\Longrightarrow$&$n+1 \leq x +\dfrac{1}{3}<n+2 \qquad y \qquad n+1\leq x+\dfrac{2}{3}<n+2$\\\\
		    \end{tabular}
		\end{center}
		Así que $[3x]=3n+2=[x]+\left[x+\dfrac{1}{3}\right]+\left[x+\dfrac{2}{3}\right]$\\\\

	\end{enumerate}

	%--------------------5.
    \item  Las fórmulas de los Ejercicios $4(d)$ y $4(c)$ sugieren una generalización para $[nx]$. Establecer y demostrar una  generalización.\\\\
	Demostración.-\; Se puede afirmar que: $$[nx] =\sum\limits_{k=0}^{n-1} \left[ x + \dfrac{k}{n}\right],$$ luego sea $[x]=m,$ entonces $$m\leq x < m+1 \quad \Longrightarrow \quad nm \leq nx < nm+n$$ Por lo tanto existen algunos $j \in \mathbf{Z}$ con $0\leq j < n$ tales que $$nm+j\leq nx < nm + j + 1$$ ya que sabemos que para cualquier número $nx$ está entre $k$ y $k+1$ para un entero. Pero como $nm\leq nx$ es un número entero $k$ donde está en algún lugar entre $nm$ y $nx$, lo que significa que es $nm+j$ para algún número entero $j$. Entonces tenemos que $nx$ está entre $nm+j$ y $nm+j+1$: Básicamente, solo decimos que conocemos $k\leq nx<k+1$ para algún entero $k$. Pero es mas conveniente escribirlo como $nm+j\leq nx>nm+j+1$.\\
	Luego, $[nx]=nm+j$. y por lo tanto, $$m+\dfrac{j}{n}\leq x < m + \dfrac{j+1}{n}$$ 
	para cada $k\in \mathbf{Z}$ con $0\leq k <n-j$ tenemos
	\begin{center}
	    \begin{tabular}{cll}
		& $m+\dfrac{k+k}{n} \leq x + \dfrac{k}{n} < m + \dfrac{j+k+1}{n}$ & sumando $\dfrac{k}{n}$ \\\\
		 $\Longrightarrow$ & $m\leq x+\dfrac{k}{n}< m+1$ & $\dfrac{j+k}{n} <  1$ ya que $k<n-j$\\\\
		 $\Longrightarrow$ & $\left[x+\dfrac{k}{n}=m=[x]\right]$ & para $0\leq k < n-j$ \\\\
	    \end{tabular}
	\end{center}
	por otro lado $n-j\leq k < n,$ entonces
	\begin{center}
	    \begin{tabular}{cll}
		& $m+\dfrac{k+k}{n} \leq x + \dfrac{k}{n} < m + \dfrac{j+k+1}{n}$ & \\\\
		 $\Longrightarrow$ & $m+1\leq x+\dfrac{k}{n}< m+2$ & $\dfrac{j+k}{n} \geq  1$ ya que $n-j\leq k$\\\\
		 $\Longrightarrow$ & $\left[x+\dfrac{k}{n}=m+1=[x]+1\right]$ & para $n-j \leq k < n$ \\\\
	    \end{tabular}
	\end{center}
	Así que 
	\begin{center}
	    \begin{tabular}{rcl}
		$\sum\limits_{k=0}^{n-1} \left[x+\dfrac{k}{n}\right]$ & $=$ & $\sum\limits_{k=0}^{n-j-1} \left[x+\dfrac{k}{n}\right] + \sum\limits_{k=n-j}^{n-1} \left[x+\dfrac{k}{n}\right]$\\\\
		& $=$ & $(n-j)[x]+j([x]+1)$\\\\
		& $=$ & $n[x]+j$\\\\
		& $=$ & $nm+j$\\\\
		& $=$ & $[nx]$\\\\
	    \end{tabular}
	\end{center}

	%--------------------6.
	\item Recuérdese que un punto de red $(x,y)$ en el plano es aquel cuyas coordenadas son enteras. Sea $f$ una función no negativa cuyo dominio es el intervalo $[a,b]$, donde $a$ y $b$ son enteros, $a<b$. Sea $S$ el conjunto de puntos $(x,y)$ que satisfacen $a\leq x\leq b$, $a\leq y \leq f(x)$. Demostrar que el número de puntos de la red pertenecientes a $S$ es igual a la suma $$\sum\limits_{n=a}^{b} \left[f(n)\right]$$\\\\
	    Demostración.-\; Sea $n\in \mathbb{Z}$ con $a\leq n < b$: Sabemos que tal $n$ existe desde $a,b \in \mathbb{Z}$ con $a<b$. Entonces, el número de puntos de red $S$ con el primer elemento $n$ es el número de enteros $y$ tales que $0<y\leq f(n)$. Pero, por definición, esto es $[f(n)]$ Sumando todos los números enteros $n$. $a\leq n\leq b$ tenemos, $$S=\sum\limits_{n=a}^{b} \left[f(n)\right]$$

	%--------------------7.
	\item Si $a$ y $b$ son enteros positivos primos entre sí, se tiene la fórmula $$\sum\limits_{n=1}^{b-1} \left[\dfrac{na}{b}\right] = \dfrac{(a-1)(b-1)}{2}.$$ Se supone que para $b=1$ la suma del primer miembro es $0$.\\\\
	
	\begin{enumerate}[\bfseries (a)]

	    %----------(a)
	    \item Dedúzcase este resultado analíticamente contando los puntos de la red en un triángulo rectángulo.\\\\
		Respuesta.-\; Sabemos por el ejercicio anterior que $$\sum\limits_{n=1}^{b-1}\left[\dfrac{na}{b}\right]$$ (puntos de red de un triángulo rectángulo con base $b$ y altura $a$). Ademas del ejercicio $1.7 n 4$ sabemos que $$a(T)=I_t + \dfrac{1}{2}B_T - 1$$ donde $I_T$ es el número de puntos de red interior y $B_T$ es el número de puntos de red límite. También sabemos por la fórmula del área de un triángulo rectángulo que $$a(T)=\dfrac{1}{2}(ab)$$ por lo tanto tenemos $$I=\sum\limits_{n=1}^{b-1}\left[\dfrac{na}{b}\right]=\dfrac{1}{2}(ab)-\dfrac{1}{2}B_T + 1$$
		Luego para calcular $B_T$ notamos que no hay puntos límite en la hipotenusa de nuestro triángulo rectángulo ya que $a$ y $b$ no tienen un factor común. Esto se deduce ya que si no fuera tal punto a continuación, $\dfrac{na}{b} \in \mathbb{Z}$ para algunos $n<b,$ tendríamos que $a$ divide a $b$, lo que contradice que no tienen ningún factor común. Por lo tanto $B_T=a+b+1$, de donde,
		\begin{center}
		    \begin{tabular}{rcl}
			$\sum\limits_{n=1}^{b-1} \left[\dfrac{na}{b}\right]$&$=$&$\dfrac{1}{2}(ab) - \dfrac{1}{2}(a+b+1) + 1$\\\\
			&$=$&$\dfrac{1}{2} (ab-a-b+1)$\\\\
			&$=$&$\dfrac{1}{2}(a-1)(b-1)$\\\\
		    \end{tabular}
		\end{center}

	    \item Dedúzcase este resultado analíticamente de la manera siguiente. Cambiando el índice de sumación, obsérvese que $\sum\limits_{a=1}^{b-1}[na/b]=\sum\limits_{n=1}^{b-1}[a(b-n)/b]$. Aplíquese luego los ejercicios $4(a)$ y $(b)$  al corchete de la derecha.\\\\
		Respuesta.-\; Sea $$ \sum\limits_{n=1}^{b-1} \left[\dfrac{na}{b}\right] =  \sum\limits_{n=1}^{b-1} \left[\dfrac{a(b-n}{b}\right] $$
		entonces,
		\begin{center}
		    \begin{tabular}{crcl}
			&$\sum\limits_{n=1}^{b-1} \left[\dfrac{a(b-n)}{b}\right]$&$=$&$\sum\limits_{n=1}^{b-1} \left[a\right] - \sum\limits_{n=1}^{b-1} \left[\dfrac{na}{b}\right] - \sum\limits_{n=1}^{b-1} 1$\\\\
			$\Longrightarrow$&$2\sum\limits_{n=1}^{b-1} \left[\dfrac{a(b-n}{b}\right]$&$=$&$\sum\limits_{n=1}^{b-1} - \sum\limits_{n=1}^{b-1} = (b-1)a - (b-1)$\\\\
			&$\sum\limits_{n=1}^{b-1} \left[\dfrac{na}{b}\right]$&$=$&$\dfrac{1}{2}(a-1)(b-1)$\\\\
		    \end{tabular}
		\end{center}

	\end{enumerate} 

	%--------------------8.
    \item Sea $S$ un conjunto de puntos en la recta real. La función característica de $S$ es, por definición, la función $Xs(x)=1$ para todo $x$ de $S$ y $Xs(x)=0$ para aquellos puntos que o pertenecen a $S$. Sea $f$ una función escalonada que toma el valor constante $c_k$ en el k-simo subintervalo $I_k$ de una cierta partición de un intervalo $[a,b]$. Demostrar que para cada $x$ de la reunión $I_1 \cup I_2 \cup ... \cup I_n$ se tiene $$f(x)=\sum\limits_{k=1}^{n} c_k X_{I_k} (x)$$ Esta propiedad se expresa diciendo que toda función escalonada es una combinación lineal de funciones características del intervalos.\\\\
	Demostración.-\; Definimos una función característica, $Xs$, en un conjunto $S$ de puntos en $\mathbf{R}$ por 
	$$Xs(x)=\left\{\begin{array}{c c c} 
	    1&si&x\in S\\ 
	    \\0&si&x \notin S \\
	\end{array}\right.$$
	Primero, observemos que los subintervalos abiertos de alguna partición de $[a,b]$ son necesariamente disjuntos desde $P=\lbrace x_0,x_1,...,x_n \rbrace \quad \Longrightarrow \quad x_0<x_1<...<x_n$. Por lo tanto si $x\in I_1 \cup ... \cup I_n$ entonces $x \in I_j$ exactamente por uno $j, 1\leq j \leq n$. Por lo tanto,
	$$XIs(x)=\left\{\begin{array}{c c c} 
	    1&si&xj=k\\ 
	    \\0&si&j\neq k \\
	\end{array}\right.$$
	para todos $1\leq k\leq n$ y para cualquiera $x$. Además, por definición de $f$, sabemos $f(x)=c_k$ si $x\in I_k,$. Así que, 
	$$\sum\limits_{k=1}^{n} c_k XI_k(x) = c_1\cdot 0 + c_2 \cdot 0 + ... + c_k\cdot 1 + ... + c_n\cdot 0 = c_k = f(x)$$
	para cada $x\in I_i \cup ... \cup I_n$\\\\

    \end{enumerate}
   
\section{Definición de integral para funciones escalonadas}

    %--------------------definición de intergal de funciones escalonadas
	\begin{def.}[Definición de integral de funciones escalonadas]
	    La integral de $s$ de $a$ a $b$, que se designa por el símbolo $\int_{a}^{b} s(x) dx$, se define mediante la siguiente fórmula: $$\int_{a}^{b} s(x) dx = \sum\limits_{k=1}^{n} s_k \cdot (x_k - x_{k-1})$$
	    Es decir, para obtener el valor de la integra, se multiplica cada valor $s_k$ constante, por la longitud de intervalo $k-simo$ correspondiente, formando el producto $s_k\cdot (x_k - x_{k-1})$ y se suman luego todos los productos obtenidos.
	\end{def.}

    %-------------------- definición 1.11
	\begin{def.}
	    Si $x$ es constante en el intervalo abierto $(a,b)$, es decir, $s(x)=c$ si $a<x<b,$ se tiene entonces: $$\int_{a}^{b} s(x) dx = c \sum\limits_{k=1}^{n} (x_k - x_{k-1}) = c(b-a)$$
	\end{def.}

\section{Propiedades de la integral de una función escalonada}

    %--------------------teorema 1.2
    \begin{teo}[Propiedad aditiva]
	$\displaystyle\int_{a}^{b} [s(x) + t(x)] dx = \int_{a}^{b} s(x) dx + \int_{a}^{b} t(x) dx$\\\\
    \end{teo}

    %--------------------teorema 1.3
    \begin{teo}[Propiedad Homogénea]
	$\displaystyle\int_{a}^{b} c \cdot s(x) dx = c\int_{a}^{b} s(x) dx$\\\\
	    Demostración.-\; Sea $P=\lbrace x_0,x_1,...,x_n \rbrace$ una partición de $[a,b]$ tal que $s$ es constante en los subintervalos abiertos de $P$. Sea $s(x)=s_k$ si $x_{k-1}<x<x_k$ para $k=1,2,...,n$. Entonces, $c \cdot s(x) = c \cdot s_k$ si $x_{k-1}<x<x_k$, y por tanto en virtud de la definición de integral se tiene $$\displaystyle\int_{a}^{b} c\cdot s(x) \; dx = \sum\limits_{k=1}^n c \cdot s_k (x_k - x_{k-1}) = c \sum\limits_{k=1}^n s_k(x_k - x_{k-1}) = c\cdot \int_{a}^{b} s(x) \; dx$$\\
    \end{teo}

    %--------------------teorema 1.4
    \begin{teo}[Propiedad de la linealidad]
	$\displaystyle\int_{a}^{b} [c_1x(x) + c_2t(x)] dx = c_1 \int_{a}^{b} s(x) dx + c_2 \int_{a}^{b} dx$\\\\
    \end{teo}

    %--------------------teorema 1.5
    \begin{teo}[Teorema de comparación] Si $s(x)<t(x)$ para todo $x$ de $[a,b]$, entonces $\displaystyle\int_{a}^{b} s(x) dx = \int_{a}^{b} t(x) dx$\\\\
    \end{teo}

    %--------------------teorema 1.6
    \begin{teo}[Aditividad respecto al intervalo de integración]
	$\displaystyle\int_{a}^{c} s(x) dx +  \int_{c}^{b} s(x) dx = \int_{a}^{b} s(x) dx$  si $a<c<b$ \\\\
    \end{teo}

    %--------------------teorema 1.7
    \begin{teo}[Invariancia frente a una traslación] 
	$\displaystyle\int_{a}^{b} s(x) dx = \int_{a+c}^{b+c} s(x-c) dx \mbox{ para todo real } c$\\\\
    \end{teo}

    %--------------------teorema 1.8
    \begin{teo}[Dilatación o contracción del intervalo de integración]
	$\displaystyle\int_{ka}^{kb} s\left(\dfrac{x}{k}\right) dx$ para todo $k>0$\\\\
	Es conveniente considerar integrales con el límite inferior mayor que el superior. Esto se logra definiendo: $$\displaystyle\int_{a}^{b} s(x) dx = - \int_{a}^{b} s(x) dx \qquad a<b$$\\\\
	    Demostración.-\; Sea $P=\lbrace x_0,x_1,...,x_n \rbrace$ una partición del intervalo $[a,b]$ tal que $s$ es constante en los subintervalos abiertos de $P$. Supóngase que $s(x)=s_i$ si $x_{i-1}<x<x_i$. Sea $t(x)=s(x/k)$ si $ka<x<kb$. Entonces $t(x)=s_i$ si $x$ pertenece al intervalo abierto $(kx_{i-1},kx_i)$;, por tanto $P{'}=\lbrace kx_0,kx_1,...,kx_n \rbrace$ es una partición de $[ka,kb]$ y $t$ es constante en los subintervalos abiertos de $P^{'}$. Por tanto $t$ es una función escalonada cuya integral es: $$\displaystyle\int_{ka}^{kb} t(x) \; dx = \sum\limits_{k=1}^{n} s_i \cdot (kx_1 - kx_{i-1}) k\sum\limits_{k=1}^n s_i \cdot (x_i - x_{i-1}) = k \int_{a}^{b} s(x) \; dx$$\\
    \textbf{Propiedad de reflexión de la integral}
	$\displaystyle\int_{a}^{b} s(x) dx = \int_{-b}^{-a} s(-x) dx$\\\\
    \end{teo}

\setcounter{section}{14}
\section{Ejercicios}

\begin{enumerate}[ \bfseries 1.]

    %--------------------1.
    \item Calcular el valor de cada una de las siguientes integrales. Se pueden aplicar los teoremas dados en la Sección $1.13$ siempre que convenga hacerlo. La notación $[x]$ indica la parte entera de $x$.

    \begin{enumerate}[\bfseries (a)]
	
	%----------(a)
	\item $\displaystyle\int_{-1}^{3} [x] dx.$\\\\ 
	    Respuesta.-\; Sea $P=\lbrace -1,0,1,2 \rbrace$ ya que 
	    $$[x] = \left\{ \begin{array}{rcl}
		-1&si& -1\leq x <0\\
		\\ 0&si& 0\leq x < 1 \\
		\\ 1&si& 1\leq x < 2 \\
		\\ 2&si& 2\leq x < 3 \\
		\end{array}\right.$$
	    entonces por definición $\displaystyle\int_{-1}^{3} [x] dx = -1\cdot [0-(-1)] + 0\cdot (1-0) + 1\cdot (2-1) + 2(3-2) = 2$\\\\

	%----------(b)
	\item $\displaystyle\int_{-1}^{3} \left[ x + \dfrac{1}{2} \right] dx$\\\\
	    Respuesta.-\; La partición viene dada por $P=\lbrace -1,-1/2,1/2,3/2,5/2\rbrace$ ya que 
	    $$\left[x + \dfrac{1}{2} \right] = \left\{ \begin{array}{rclcl}
		-1&si&-1\leq x + 1/2 < 0&\Longrightarrow&-3/2 \leq x < -1/2\\
		\\ 0&si&0\leq x + 1/2 < 1&\Longrightarrow&-1/2 \leq x < 1/2\\
		\\ 1&si&1\leq x + 1/2 < 2&\Longrightarrow&1/2 \leq x < 3/2\\
		\\ 2&si&2 \leq x + 1/2 < 3&\Longrightarrow&3/2 \leq x < 5/2\\
		\\ 3&si &3 \leq x + 1/2 < 4&\Longrightarrow&5/2 \leq x < 7/2 \\
	    \end{array}\right.$$
	     entonces por definición $\displaystyle\int_{-1}^{3} \left[x + \dfrac{1}{2}\right] dx = -1\left(-\dfrac{1}{2} + 1\right) + 0 \left( \dfrac{1}{2} + \dfrac{1}{2}\right) + 1\left(\dfrac{3}{2} - \dfrac{1}{2}\right) + 2\left( \dfrac{5}{2} - \dfrac{3}{2}\right) + 3\left( 3 - \dfrac{5}{2}\right) = -1\cdot \dfrac{1}{2} + 0\cdot 1 + 1\cdot 1 + 2\cdot 1 + 3\cdot \dfrac{1}{2} = 4$\\\\
	%----------(c)
	 \item $\displaystyle\int_{-1}^{3} \left([x] + [x + 1/2]\right)$\\\\
	     Respuesta.-\; Por la propiedad aditiva se tiene $\displaystyle\int_{-1}^{3} [x] dx + \int_{-1}^{3} \left[x + \dfrac{1}{2}\right] dx = 2+4=6$.\\\\  

	%----------(d)
	\item $\displaystyle\int_{-1}^{3} 2[x] dx$\\\\
	    Respuesta.-\; Por la propiedad homogénea se tiene $2\cdot \displaystyle\int_{-1}^{3} [x] dx = 2\cdot 2 = 4$\\\\

	%----------(e)
	\item $\displaystyle\int_{-1}^{3} [2x] dx$\\\\
	    Respuesta.-\; Por la propiedad de parte entera donde $[2x] = [x] + [x + frac{1}{2}]$ y $(c)$ resulta que $\displaystyle\int_{-1}^{3} [2x] dx = 6$\\\\

	%----------(f)
	\item $\displaystyle\int_{-1}^{3} [-x] dx$\\\\
	    Respuesta.-\; Por propiedad de parte entera se tiene que $[-x]=-[x]-1$ ya que el valor de los subintervalos no es entero, luego $$\displaystyle\int_{-1}^{3} -[x] - 1 = \int_{-1}^{3} -[x] + \int_{-1}^{3} = -\int_{-1}^{3} [x] - \int_{-1}^{3} 1 = -2 - \lbrace 1\cdot \left[3-(-1)\right]\rbrace$$\\\\

    \end{enumerate}


    %--------------------2.
    \item Dar un ejemplo de función escalonada $s$ definida en el intervalo cerrado $[0,5]$, que tenga las siguientes propiedades $\displaystyle\int_{0}^{2} s(x) dx = 5,\; \int_{0}^{5}s(x) dx = 2$\\\\
	Respuesta.-\; Existen infinitas funciones escalonadas que deben cumplir lo siguiente: 
	$$\displaystyle\int_{2}^{5} s(x) dx = \int_{0}^{5} s(x) dx - \int_{0}^{2} s(x) dx=-3$$\\\\

    %--------------------3.
    \item Probar que $\displaystyle\int_{a}^{b} [x] \; dx + \int_{a}^{b}[-x] \; dx = a - b$\\\\
	Demostración.-\; Por propiedad de parte entera se tiene que $[-x]=-[x]-1$ si $x$ no es entero. Entonces $\displaystyle\int_{a}^{b} [x] \; dx + \int_{a}^{b} -[x]-1 \; dx$ luego por la propiedad aditiva $\displaystyle\int_{a}^{b} -1 \; dx$, de donde por la propiedad dilatación $\displaystyle\int_{b}^{a} 1 \; dx$ y por lo tanto $a-b$.\\\\ 

    %--------------------4.
    \item 
    \begin{enumerate}[\bfseries (a)]
	
	%----------(a)
	\item Si $n$ es un entero positivo, demostrar que $\displaystyle\int_{0}^{n} [t] \; dt = n(n-1)/2$.\\\\
	    Demostración.-\; Según la definición de la función de número entero mayor, $[t]$ es constante en los subintervalos abiertos, por lo que $[t]=k-1$ si $k-1<t<k$ entonces  $$\displaystyle\int_{0}^{n} = \sum_{k=1}^{n} (k-1)(k-(k-1)) = \sum_{k=1}^{n} (k-1)$$ Luego sabemos que $\sum\limits_{k=1}^{n} k = \dfrac{n^2}{2}+\dfrac{n}{2}$ por lo tanto $\displaystyle\int_{0}^{n}=\dfrac{n(n-1)}{2}$. \\\\
	    También se podría demostrar de la siguiente manera.
	    $$\displaystyle\int_{0}^{n} [t] \; dt = 0\cdot (1-0) + 1\cdot (2-1) + ... + (n-1)\cdot (n-(n-1))=1+2+...+(n-1)=\dfrac{n(n-1)}{2}$$\\\\

	%----------(b)
	\item Si $f(x)=\displaystyle\int_{0}^{x} [t] \; dt$ para $x\geq 0,$ dibujar la gráfica de $f$ sobre el intervalo $[0,4]$.\\\\
	    Respuesta.-\; Ya que $x \in \mathbf{R}^+$ la gráfica será continua, pero por motivos prácticos dibujemos puntos en los números enteros del intervalo $[0,4]$
	    \begin{itemize}
		\item $f(0)=\displaystyle\int_{0}^{0} [t] \; dt = 0(0-0)=0$
		\item $f(1)=\displaystyle\int_{0}^{1} [t] \; dt = 0(1-0)=0$
		\item $f(2)=\displaystyle\int_{0}^{2} [t] \; dt = 0(1-0) + 1(2-1)=1$
		\item $f(3)=\displaystyle\int_{0}^{3} [t] \; dt = 0(1-0) + 1(2-1) + 2(3-2) = 3$
		\item $f(4)=\displaystyle\int_{0}^{4} [t] \; dt = 0(1-0) + 1(2-1) + 2(3-2) + 3(4-3) = 6$\\\\
	    \end{itemize}
		\begin{center}
		    \begin{tikzpicture}
			% abscisa y ordenada
			\tkzInit[xmax= 4,xmin=0,ymax=6,ymin=0]
			\tiny\tkzLabelXY[opacity=0.6,step=1, orig=false]
			% label x, f(x)
			\tkzDrawX[opacity= .6,label=x,right=0.3]
			\tkzDrawY[opacity= .6,label=f(x),below = -0.6]
			%funciones
			\draw(0,0)--(1,0);
			\draw(1,0)--(2,1);
			\draw(2,1)--(3,3);
			\draw(3,3)--(4,6);
		    \end{tikzpicture}
		\end{center}
		\vspace{1cm}

    \end{enumerate}

    %--------------------5.
    \item
    \begin{enumerate}[\bfseries (a)]
	
	%----------(a)
	\item Demostrar que $\displaystyle\int_{0}^{2} [t^2] \; dt = 5 - \sqrt{2} - \sqrt{3}.$\\\\
	    Demostración.-\; 
	    $$\left[t^2\right]=\left\{\begin{array}{rclcl}
		  0&si&0\leq t^2 < 1&\Rightarrow&0\leq t < 1\\
		\\1&si&1\leq t^2 < 2&\Rightarrow&1\leq t < \sqrt{2}\\
		\\2&si&2\leq t^2 < 3&\Rightarrow&\sqrt{2}\leq t < \sqrt{3}\\
		\\3&si&3\leq t^2 < 4&\Rightarrow&\sqrt{3} \leq t<2\\\\
	    \end{array}\right.$$
	    Vemos que la partición esta dada por $P=\lbrace 0,1,\sqrt{2},\sqrt{3},2 \rbrace$ y por lo tanto:
	    $$\displaystyle\int_{0}^{2} \left[t^2\right] \; dt = \sum\limits_{k=1}^{5} s_k \cdot (x_k - x_{k-1}) = 0(1-0)+1(\sqrt{2}-1) + 2(\sqrt{3}-\sqrt{2}) + 3(2-\sqrt{3})=5 - \sqrt{2} - \sqrt{3}$$\\\\
	
	%----------(b)
	\item Calcular $\displaystyle\int_{-3}^{3} \left[ t^2 \right] \; dt$\\\\
	    Respuesta.-\; Sea $\displaystyle\int_{-3}^{3} [t^{2}] \ dx = \int_{0}^{3} [t^{2}] \ dx + \int_{-3}^{0} [t^{2}] \ dx = \int_{0}^{3} [t^{2}] \ dx + \int_{0}^{3} [(-t)^{2}] \ dx = 2 \int_{0}^{3} [t^{2}] \ dx$ entonces:\\

	    $$\begin{array}{rcl}
		\displaystyle\int_{-3}^{3} \left[t^2\right] \ dt&=& \displaystyle2\cdot \int_{0}^{3} \left[t^2\right] \ dt\\\\
		&=&\displaystyle2\cdot \left( \int_{0}^{2} \left[t^2\right] \ dt + \int_{2}^{3} \left[t^2\right] \ dt\right)\\\\
		&=& 2\cdot \left(5 - \sqrt{2} - \sqrt{3} + 4(\sqrt{5}-2) + 5(\sqrt{6}-\sqrt{5}) + 6(\sqrt{7} - \sqrt{6}) + 7(\sqrt{8} - \sqrt{7})\right.\\\\ 
		&+&\left. 8(3 - \sqrt{8})\right)\\\\
		&=&2\left(21 - 3\sqrt{2} - \sqrt{3} - \sqrt{5} \sqrt{6} - \sqrt{7}\right)\\\\
	    \end{array}$$

    \end{enumerate}

    %-------------------6.
    \item 
    \begin{enumerate}[\bfseries (a)]
	
	%-----------(a)
	\item Si $n$ es un entero positivo demostrar que $\displaystyle\int_{0}^{n} [t]^2 /; dt = \dfrac{n(n-1)(2n-1)}{6}$\\\\
	    Demostración.-\; Sea $P=\lbrace 0,1,...,n \rbrace$. Entonces $P$ es una partición de $[0,n]$ y $[t]^2$ es constante en los subintervalos abiertos de $P$. Ademas, $[t]^2=(k-1)^2$ para $k-1<t<k$, entonces 
	    \begin{center}
		\begin{tabular}{rcl}
		    $\displaystyle\int_{0}^{n} [t]^2 \; dt$&$=$&$\sum\limits_{k=1}^{n}(k-1)^2 \cdot (k-(k-1))$\\\\
		    &$=$&$\sum\limits_{k=1}^{n} (k-1)^2$\\\\
		    &$=$&$\sum\limits_{n-1}^{k=0} k^2$\\\\
		    &$=$&$\dfrac{n^3}{3}-\dfrac{n^2}{2}+\dfrac{n}{6}$\\\\
		    &$=$&$\dfrac{n(n-1)(2n-1)}{6}$\\\\
		\end{tabular}
	    \end{center}

	%----------(b)
	\item Si $f(x)=\displaystyle\int_{0}^{x} [t]^2 \; dt$ para $x\geq 0$, dibujar la gráfica de $f$ en el intervalo $[0,3]$.\\\\
	    Respuesta.-\; 
	    \begin{center}
		\begin{tikzpicture}
		    % abscisa y ordenada
		    \tkzInit[xmax= 4,xmin=0,ymax=6,ymin=0]
		    \tiny\tkzLabelXY[opacity=0.6,step=1, orig=false]
		    % label x, f(x)
		    \tkzDrawX[opacity= .6,label=x,right=0.3]
		    \tkzDrawY[opacity= .6,label=f(x),below = -0.6]
		    %funciones
		    \draw(0,0)--(1,0);
		    \draw(1,0)--(2,1);
		    \draw(2,1)--(3,5);
		\end{tikzpicture}
	    \end{center}
	    \vspace{.5cm}

	%----------(c)
	\item Hallar todos los valores de $x>0$ para los que $\displaystyle\int_{0}^{x} [t]^2 \; dt=2(x-1)$\\\\
	    Respuesta.-\; Los valores son $x=1,\frac{5}{2}$ ya que son los que se intersectan con $2(x-1)$.\\\\ 

    \end{enumerate}

    %---------------------7.
    \item 
	\begin{enumerate}[\bfseries (a)]
		
	    %----------(a)
	    \item Calcular $\displaystyle\int_{0}^{9} [\sqrt{t}] \; dt$.\\\\
		Respuesta.-\;
		$$[\sqrt{t}] = \left\{ \begin{array}{c c l c r}
		    0 & si & 0\leq \sqrt{t}<1 & \Rightarrow & 0\leq t < 1\\
		    \\1 & si & 1 \leq \sqrt{t} < 2 & \Rightarrow & 1 \leq t < 4\\
		    \\2 & si & 2 \leq \sqrt{t}<3 & \Rightarrow & 4\leq t < 9\\\\
		\end{array}\right.$$
		    $\displaystyle\int_{0}^{9} \left[\sqrt{t}\right]\; dt = 0\cdot(1-0) + 1\cdot (4-1) + 2(9-4)=13$\\\\

	    %----------(b)
		\item Si $n$ es un entero positivo, demostrar que $\displaystyle\int_{0}^{n^2} \left[ \sqrt{t}\right] \; dt = \dfrac{n(n-1)(4n+1)}{6}$\\\\
		    Demostración.-\; Sea $P=\lbrace 0,1,3,9,...,n^2\rbrace$. Entonces $P$ es una partición de $\left[0,n^2\right]$ y $\left[\sqrt{t}\right]$ es constante en los subintervalos abiertos de $P$. Además, porque $(k-1)^2 < t < k^2$ tenemos $\left[\sqrt{t}\right]=(k-1)$. Por lo tanto, se tiene:
		    \begin{center}
			\begin{tabular}{rcl}
			    $\displaystyle\int_{0}^{n^2} \left[\sqrt{t}\right]\; dt$&$=$&$\sum\limits_{k=1}^{n} (k-1)(k^2-(k-1)^2)$\\\\
			    &$=$&$\sum\limits_{k=1}^{n}(k-1)(2k-1)$\\\\
			    &$=$&$\sum\limits_{k=1}^{n} (2k^2 - 3k + 1)$\\\\
			    &$=$&$2\sum\limits_{k=1}^{n} k^2 - 3\sum\limits_{k=1}^{n} k + \sum\limits_{k=1}^{n} 1$\\\\
			    &$=$&$\dfrac{2n^3}{3} + n^2 + \dfrac{n}{3} - \dfrac{3n^2}{2} - \dfrac{3n}{2} + n$\\\\
			    &$=$&$\dfrac{n(n-1)(4n+1)}{6}$\\\\
			\end{tabular}
		    \end{center}
	\end{enumerate}

    %--------------------8.
    \item Pruébese que la propiedad de traslación (teorema 1.7) se puede expresar en la forma siguiente. $$\displaystyle\int_{a+c}^{b+c}f(x) \; dx = \int_{a}^{b}f(x+c)\; dx$$\\\\
	Demostración.-\; Sea $d=a+c$ y $e=b+c$ entonces por el teorema de invariancia frente a una traslación $$\displaystyle\int_{d}^{e} f(x) \; dx = \int_{e+(-c)}^{d+(-c)} f(x-(-c)) \; dx$$ para $-c \in \mathbb{R}$. Luego, $a=d-c$ y $b=c-e$ por lo tanto $$\displaystyle\int_{b+c}^{a+c} f(x) \; dx = \int_{a}^{b} f(x+c) \; dx$$\\\\

    %--------------------9.
    \item Probar que la propiedad siguiente es equivalente al teorema 1.8 $$\displaystyle\int_{ka}^{kb} f(x) \; dx = k \int_{a}^{b} f(kx) \; dx$$\\\\
	Demostración.-\; Sea $\displaystyle\int_{a}^{b} f\left(\dfrac{x}{j}\right) \; dx$, por el teorema tenemos que para cualquier $j>0$ se tiene $$\displaystyle\int_{b/j}^{a/j}f\left(j\dfrac{x}{j}\right)\; dx = \dfrac{1}{j} \int_{a}^{b} f \left(\dfrac{x}{j}\right) \; dx \quad \Rightarrow \quad j\int_{a/j}^{b/j} f(x) \; dx = \int_{a}^{b} f\left(\dfrac{x}{j} \right) \; dx$$ Luego, si $j \in \mathbb{R}^+$ entonces $\dfrac{1}{j}\in \mathbb{R}^+$, de donde podemos aplicar el teorema como $k=\dfrac{1}{j}$: $$\dfrac{1}{k}\displaystyle\int_{ka}^{kb} f(x)\; dx = \int_{a}^{b} \quad \Rightarrow \quad k \int_{a}^{b} f(kx) \; dx = \int_{ka}^{kb} f(x) \; dx$$\\\\

    %--------------------10.
    \item Dado un entero positivo $p$. Una función escalonada $s$ está definida en el intervalo $[0,p]$ como sigue $s(x)=(-1)^n n$ si $x$ está en el intervalo $n\leq x < n+1$ siendo $n=0,1,2,...,p-1$; $s(p)=0$. Póngase $f(p)=\displaystyle\int_{0}^{p} s(x)\; dx.$\\\\

    \begin{enumerate}[\bfseries (a)]

	%----------(a)
	\item Calcular $f(3), f(4)$ y $f(f(3))$.\\\\
	    Respuesta.-\; Sea $$s(x) = \left\{ \begin{array}{lcl}
		(-1)^n n&si&n\leq x < n+1, \; n=0,1,...p-1\\
		\\0& si &x=p\\
	    \end{array}\right.$$
	    Entonces calculamos para $f(x)=\displaystyle\int_{0}^{p} s(x)\; dx$:
	    \begin{center}
	    \begin{tabular}{rclcrcr}
		$f(3)$ & $=$ & $\displaystyle\int_{0}^{3} s(x) \; dx$ & $=$ & $(-1)^0 0 (1-0) + (-1)\cdot 1\cdot(2-1) + (-1)^2 2 (3-2)$ & $=$ & $1$\\\\
		$f(4)$ & $=$ & $\displaystyle\int_{0}^{4} s(x) \; dx$ &$=$ & $\displaystyle\int_{0}^{3} s(x) \;dx + \int_{3}^{4} s(x) \; dx = 1 + (-1)^3 3 (4-3)$&$=$&$-2$\\\\
		$f(f(3))$&$=$&$f(1)$&$=$&$\displaystyle\int_{0}^{1} s(x) \; dx = (-1)^0 0 (1-0)$&$=$&$0$\\\\
	    \end{tabular}
	    \end{center}

	%----------(b)
	\item ¿Para qué valor o valores de $p$ es $|f(p)|=7$?\\\\
	    Respuesta.-\; Luego de completarlo por un bucle llegamos a la conclusión de que los números que cumplen la condición dada son $14,15$.\\\\

    \end{enumerate}

    %--------------------11.
    \item  Si en lugar de definir la integral de una función escalonada utilizando la fórmula $(1.3)$ se tomara como definición: $$\displaystyle\int_{a}^{b} s(x)\; dx = \sum\limits_{k=1}^{n} s_k^3 \cdot (x_k - x^{k-1})$$ se tendría una nueva teoría de la integración distinta de la dada. ¿Cuáles de las siguientes propiedades seguirán siendo válidas en la nueva teoría?

    \begin{enumerate}[\bfseries (a)]
	
	%----------(a).
	\item $\displaystyle\int_{a}^{b} s + \int_{b}^{c} s = \int_{a}^{c} s.$\\\\
	    Respuesta.-\; Sea $P_1 =\lbrace x_0,...,x_m\rbrace$ una partición de $[a,b]$ y $P_2 = \lbrace y_0,...,y_n \rbrace$ sea una partición de $[b,c]$ tal que $s(x)$ sea constante en los intervalos abiertos de $P_1$ y $P_2$, entonces, $P=P_1\cup P_2 =\lbrace x_0,...,x_m,x_{m+1},...,x{m+n} \rbrace$ de donde $x_m = y_0, \; x_{m+1} = y_1, \; x_{m+n} = y_n$, así $P$ es una partición de $[a,c]$ y $s(x)$ es constante en los intervalos abiertos de $P$. Luego,
	    \begin{center}
		\begin{tabular}{rcll}
		    $\displaystyle\int_{a}^{c} \; dx + \int_{c}^{b} s(x) \; dx$&$=$&$\sum\limits_{k=1}^{m} s_k^3 (x_k - x_{k-1}) + \sum\limits_{k=1}^{n} s_k^3 (y_k - y_{k-1})$&$def \; de \displaystyle\int_{a}^{b} s$\\\\
		    &$=$&$\sum\limits_{k=1}^{m} s_k^3 (x_k - x_{k-1}) + \sum\limits_{k=m+1}^{n+m} s_k^3 (x_k - x_{k-1})$&\\\\
		    &$=$&$\sum\limits_{k=1}^{m+n} s_k^3(x_k - x_{k-1})$&\\\\
		    &$=$&$\displaystyle\int_{a}^{c} s(x) \; dx$&\\\\
		\end{tabular}
	    \end{center}

	%----------(b).
	\item $\displaystyle\int_{a}^{b} (s+t) = \int_{a}^{b}s + \int_{a}^{b} t$\\\\
	    Respuesta.-\; Sea $\displaystyle\int_{0}^{1} \left( s(x) + t(x) \right) \; dx = 2^3(1-0) = 8$, por otro lado $$\displaystyle\int_{0}^{1} s(x) \; dx = \int_{0}^{1} t(x) \; dx = 1(1-1) + 1(1-1) = 2$$ por lo tanto no se cumple la definición para esta propiedad.\\\\

	%----------(c).
	\item $\displaystyle\int_{a}^{b} c\cdot s = c\int_{a}^{b} s$\\\\
	    Respuesta.-\; Sea $s(x)=1$ para todo $x \in [0,1]$ y $c=2$ entonces $$\displaystyleîn_{0}^{1} s(x) \; dx = 2^3 = 2$$, por otro lado $$c\cdot \displaystyle\int_{0}^{1} s(x) \; dx = 2\cdot 1 = 2$$ por lo tanto es falso para esta propiedad.\\\\

	%----------(d)
	\item $\displaystyle\int_{a+c}^{b+c} s(x) \; dx = \int_{a}^{b} s(x+c) \; dx$\\\\
	    Respuesta.-\; Sea $P=\lbrace x_0,...,x_n \rbrace$ una partición de $[a,b]$ tal que $s(x))s_k$ en el que $k$ es el subintervalo abierto de $P$. Luego, $$P=\lbrace x_o+c.x_1+c,...,x_n+c \rbrace$$ es una partición de $[a+c,b+c]$ y $s(x-c)=x_k$ sobre $x_{k-1} + c < x < s_k + c$ entonces:
	    $$\displaystyle\int_{a}^{b} s(x) \; dx = \sum\limits_{k=1}^{n} s_k^3 (x_k - x_{k-1}) \qquad y \qquad \int_{a+c}^{b+c} s(x-c) \; dx = \sum\limits_{k=1}^{n} s_k^3 (x_k- x_{k-1})$$ por lo tanto $$\displaystyle\int_{a}^{b} s(x) \; dx = \int_{a+c}^{b+c} s(x-c) \; dx$$\\

	%----------(e)
	\item Si $s(x)<t(x)$ para cada $x$ en $[a,b]$, entonces $\displaystyle\int_{a}^{b} s < \int_{a}^{b} t.$\\\\
	    Respuesta.-\; Como $s(x)<t(x)$ entonces $s(x)^3 < t(x)^3$ de donde el resultado se sigue inmediatamente.\\\\

    \end{enumerate}

    %--------------------12.
\item Resolver el ejercicio $11$ utilizando la definición $$\displaystyle\int_{a}^{b} s(x) \; dx = \sum\limits_{k=1}^{n} s_k \cdot (x_k^2 - x_{k-1}^2)$$\\\\
    
    \begin{enumerate}[\bfseries (a)]

	%----------(a)
	\item $\displaystyle\int_{a}^{b} s + \int_{b}^{c} s = \int_{a}^{c} s$.\\\\
	    Respuesta.-\; Sea $P_1=\lbrace x_0,...,x_m \rbrace$ una partición de $[a,b]$ y $P_2=\lbrace y_0,...,y_n \rbrace$ sea una partición de $[b,c]$ tal que $s(x)$ es constante en los subintervalos abiertos de $P_1$ y $P_2$. Entonces $P=P_1 \cup P_2 = \lbrace x_0,..., x_m,x_{m+1},...,x_{m+n} \rbrace$ donde $x_m = y_0, x_{m+1}=y_1, x_{m+n}=y_n$, así $P$ es un partición de $[a,c]$ y $s(x)$ es constante en los subintervalos abiertos de $P$. Luego, 
	    \begin{center}
		\begin{tabular}{rcl}
		    $\int_{a}^{b} s(x) \; dx + \int_{b}^{a} s(x) \; dx$&$=$&$\sum\limits_{k=1}^{m} s_k (x_k^2 - x_{k-1}^2) + \sum\limits_{k=1}^n s_k (y_k^2 - y_{k-1}^2) $\\\\
		    &$=$&$\sum\limits_{k=1}^m s_k(s_k^2 - x_{k-1}^2) + \sum\limits_{k=m+1}^{m+n} x_k (x_k^2 - x_{k-1}^2)$\\\\
		    &$=$&$\sum\limits_{k=1}^{m+n} x_k(x_k^2 - x_{k-1}^2)$\\\\
		    &$=$&$\displaystyle\int_{a}^{c} s(x) \; dx$\\\\
		\end{tabular}
	    \end{center}

	%----------(b)
	\item $\displaystyle\int_{a}^{b} (s+t)=\int_{a}^{b} s + \int_{a}^{b} t$\\\\
	    Respuesta.-\; Sea $P=\lbrace x_0,...,x_n \rbrace$ partición del intervalo $[a,b]$ tal que $s(x)$ es constante en los subintervalos abiertos de $P$. Supongamos que $s(x) + t(x) = s_k + t_k$ si $x_{k-1}<x<x_k$ para $k=1,2,...n$. Luego,
	    \begin{center}
		\begin{tabular}{rcl}
		    $\displaystyle\int_{a}^{b} s(x) + t(x) \; dx$&$=$&$\sum\limits_{k=1}^n (s_k + t_k)(x_k^2 - x_{k-1}^2)$\\\\
		    &$=$&$\sum\limits_{k=1}^{n} s_k(x_k^2 - x_{k-1}^2) + t_k(x_k^2 - x_{k-1}^2)$\\\\
		    &$=$&$\sum\limits_{k=1}^{n} s_k(x_k^2 - x_{k-1}^2) + \sum\limits_{k=1}^{n} t_k(x_k^2 - x_{k-1}^2)$\\\\
		    &$=$&$\displaystyle\int_{a}^{b} s(x) \; dx + \int_{a}^{b} t(x) \; dx$\\\\
		\end{tabular}
	    \end{center}

	%----------(c)
	\item $\displaystyle\int_{a}^{b} c\cdot s = c\int_{a}^{b} s$\\\\
	    Respuesta.-\; Sea $P=\lbrace x_0,...x_n \rbrace$ una partición del intervalo $[a,b]$ tal que $s(x)$ es constante en los subintervalos  abiertos de $P$. Suponga que $s(x)=s_k$ si $x_{k-1} < x< s_{k}$ para $k=1,2,...n$, entonces $c\cdot s(x) = c\cdot s_k$ si $x_{k-1}<x<x_k$ de donde 
	    \begin{center}
		\begin{tabular}{rcl}
		    $\displaystyle\int_{a}^{b} s(x) \; dx$&$=$&$\sum\limits_{k=1}^{n} c\cdot s_k(x_k^2 - x_{k-1}^2)$\\\\
		    &$=$&$c\cdot \sum\limits_{k=1}^{n} s_k(x_k^2 - x_{k-1}^2)$\\\\
		    &$=$&$c\cdot \displaystyle\int_{a}^{b} s(x) \; dx$\\\\
		\end{tabular}
	    \end{center}

	%----------(d)
	\item $\displaystyle\int_{a}^{b} c\cdot s = c\int_{a}^{b} s$\\\\
	    Respuesta.-\; En particular se da un contraejemplo dejando $s(x)=1$ para todo $x\in [1,2]$ y $c=1$ Luego,
	    \begin{center}
		\begin{tabular}{rcc}
		    $\displaystyle\int_{0+1}^{1+1} x(x) \;dx$&$=$&$1\cdot(2^2 - 1^2) = 3$\\\\
		    $\displaystyle\int_{0}^{1} (s+1) \;dx$&$=$&$1\cdot (1^2 - 0^2) = 1$\\\\
		\end{tabular}
	    \end{center}
	    Por lo que se concluye que es falso para dicha propiedad.\\\\

	%----------(e)
	\item Si $s(x)<t(x)$ para cada $x$ en $[a,b]$, entonces $\displaystyle\int_{a}^{b} s < \int_{a}^{b} t.$\\\\
	    Respuesta.-\; Se da un contraejemplo considerando $s(x)=0$ y $t(x)=1$ en el intervalo $[-1,0]$. Luego $s<t$ en el intervalo, pero $$\displaystyle\int_{-1}^{0} s(x) \;dx = 0 \not< \int_{-1}^{0} t(x) \; dx = 1\cdot(0^2 - (-1)^2) = -1$$\\

    \end{enumerate}

    %--------------------13.
    \item Demostrar el teorema $1.2$ (Propiedad aditiva).\\\\
	Demostración.-\; Sea $P=\lbrace x_o,...,x_n \rbrace$ una partición del intervalo $[a,b]$, tal que $s(x) + t(x)$ es constante en los intervalos abiertos de $P$. Sea $s(x)+(x)=s_k + t_k$ si $x_{k-1}<x<x_k$ para $k=1,2,...,n$, luego
	\begin{center}
	    \begin{tabular}{rcl}
		$\displaystyle\int_{a}^{b} \left[ s(x) + t(x) \right] \; dx$&$=$&$\sum\limits_{k=1}^n (s_k + t_k)(x_{k-1} - x_k)$\\\\
		&$=$&$\sum\limits_{k=1}^n s_k(s_k + t_k) + t_k(s_k + t_k)$\\\\
		&$=$&$\sum\limits_{k=1}^n s_k(s_k + t_k) + \sum\limits_{K=1}^n t_k(s_k + t_k)$\\\\
		&$=$&$\displaystyle\int_{a}^{b} s(x) \; dx + \int_a^b t(x) \; dx$\\\\
	    \end{tabular}
	\end{center}

    %--------------------14.
    \item Demostrar el teorema $1.4$(Propiedad lineal).\\\\
	Demostración.-\; Por el teorema $1.2$ y $1.3$ se tiene $$\displaystyle\int_a^b \left[c_1 s(x) + c_2 t(x)\right] \; dx = \int_a^b c_1 s(x) \; dx + \int_a^b c_2 t(x) \; dx = c_1 \int_a^b s(x) \;dx  + c_2 \int_a^b t(x) \; dx$$\\ 

    %--------------------15.
    \item Demostrar el teorema $1.5$ (teorema de comparación).\\\\
	Demostración.-\; Sea $P\lbrace x_0,...,x_n \rbrace$ una partición de $[a,b]$ tal que $s(x)$ y $t(x)$ sean constantes en los subintervalos abiertos de $P$. Suponga que $s(x)=s_k$ y $t(x)=t_k$ si $x_{k-1}<x<x_k$ de donde por definición de función escalonada de integrales tenemos:
	$$\displaystyle\int_a^b s(x) \; dx = \sum\limits_{k=1}^n s_k(x_{k-1} - x_k) \qquad y \qquad \int_a^b t(x) \;dx = \sum\limits_{k=1}^n t_k(x_{k-1} - x_k)$$
	Luego $s_k<t_k$ para cada $x\in [a,b]$, lo que implica:
	$$\displaystyle  \sum\limits_{k=1}^n s_k(x_{k-1} - x_k)  < \sum\limits_{k=1}^n t_k(x_{k-1} - x_k) \qquad \Rightarrow \qquad \int_a^b s(x) \; dx < \int_a^b t(x) \;dx$$\\

    %--------------------16.
    \item Demostrar el teorema $1.6$ aditividad con respecto al intervalo.\\\\
	Demostración.-\; Sea $P_1=\lbrace x_0,...,x_m\rbrace$ una partición de $[a,b]$ y $P_2=\lbrace y_o,...,y_n \rbrace$ sea una partición de $[c,b]$ tal que $s(x)$ es constante en los subintervalos abiertos de $P_1$ y $P_2$ luego $P=P_1 \cup P_2$ una partición de $[a,b]$ siendo $y_0=x_m$, $y_1=x_{m+1},...,y_n=x_{m+n}$ y $s(x)$ constante en los subintervalos abiertos de $P$, así
	\begin{center}
	    \begin{tabular}{rcl}
		$\displaystyle\int_{a}^{c} s(x) \; dx \int_c^b s(x) \;dx$&$=$&$\sum\limits_{k=1}^{m} s_k (x_k-x_{k-1}) + \sum\limits_{k=1}^n s_k(k_y - y_{k-1})$\\\\
		&$=$&$\sum\limits_{k=1}^m s_k(x_k - x_{k-1}) + \sum\limits_{k=m+1}^{m+n} s_k(x_k - x_{k-1})$\\\\
		&$=$&$\sum\limits_{k=1}^m s_k(s_k - x_{k-1})$\\\\
		&$=$&$\displaystyle\int_{a}^b s(x) \; dx$\\\\
	    \end{tabular}
	\end{center}

    %--------------------17.
    \item Demostrar el teorema $1.7$ invariancia frente a una traslación.\\\\
	Demostración.-\;  Sea $P=\lbrace x_0,...,x_n \rbrace$ una partición $[a,b]$ tal que $s(x)=s_k$ constante en el subintervalo abierto de la partición. Por otro lado sea $P=\lbrace x_0+c,x_1+c,...,x_n+c \rbrace$ en una partición de $[a+c,b+c]$ y $s(x-c)=s_k$ para $x_{k-1}+c<x<x_k+c$ Entonces,
	$$\displaystyle\int_a^b s(x) \; dx = \sum\limits_{k=1}^n s_k(x_k-x_{k-1}) = \int_{a+c}^{b+c} s(x-c)\; dx $$\\

\end{enumerate}

\section{La integral de funciones más generales}

%--------------------definición 1.12
    \begin{def.}[Definición de integral de una función acotada] Sea $f$ una función definida y acotada en $[a,b]$. Sean $s$ y $t$ funciones escalonadas arbitrarias definidas en $[a,b]$ tales que $$s(x)\leq f(x) \leq t(x)$$
	para cada $x$ en $[a,b]$. Si existe un número $I$, y sólo uno, tal que $$\displaystyle\int_a^b s(x)\; dx \leq I \leq \int:a^b t(x) \; dx$$
	Para cada par de funciones escalonadas $s$ y $t$ que verifiquen $s(x) \leq f(x) \leq t(x)$, entonces este número $I$ se denomina la integral de $f$ desde $a$ hasta $b$ y se indica por el símbolo $\displaystyle\int_a^b f(x) \; dx$. Cuando $I$ existe se dice que $f$ es integrable en $[a,b]$.\\\\
	Si $a<b$ se define $\displaystyle\int_a^b f(x) \; dx = - \int_b^a f(x) \; dx$ supuesta integrable $f$ en $[a,b]$.\\\\
	También se define $\displaystyle\int_a^a f(x) \; dx=0$. Si $f$ es integrable en $[a,b]$, se dice que la integral $\displaystyle\int_a^b f(x) \; dx x$ existe.\\\\
	La función $f$ se denomina integrando, los número $a$ y $b$ los límites de integración, y el intervalo $[a,b]$ el intervalo de integración.
    \end{def.}

\section{Integral superior e inferior}

%--------------------1.13
    \begin{def.}
	Supongamos la función $f$ acotada en $[a,b]$. Si $s$ y $t$ son funciones escalonadas que satisfacen $s(x)<f(x)<t(x)$ se dice que $s$ es inferior a $f$ y que $t$ es superior a $f$
    \end{def.}

%--------------------teorema 1.9
\begin{teo}
    Toda función $f$ acotada en $[a,b]$ tiene una integral inferior $\underbar{I}(f)$ y una integral superior $\bar{I}(f)$ que satisfacen las desigualdades $$\int_a^b s(x) \; dx \leq \underbar{I}(f) \leq \bar{I}(f) \leq \int_a^b t(x) \; dx$$ 
    para todas las funciones $s$ y $t$ tales que $s\leq f\leq t$. La función $f$ es integrable en $[a,b]$ si y sólo si sus integrables superior e inferior son iguales, en cuyo caso se tiene $$\int_a^b f(x) \; dx = \underbar{I}(f)=\bar{I}(f)$$\\
    Demostración.-\; Sea $S$ el conjunto de todos los números $\int_a^b s(x)\; dx$ obtenidos al tomar como $s$ todas las funciones escalonadas inferiores a $f$, y sea $T$ el conjunto de todos los números $\int_a^b t(x)\; dx$ al tomar como $t$ todas las funciones escalonadas superiores a $f$. Esto es $$ S=\left\{ \int_a^b s(x)\; dx | s\leq f\right\}, \qquad T=\left\{ \int_a^b t(x)\; dx | f\leq t\right\}$$ 
    Los dos conjuntos $S$ y $T$ son no vacíos puesto que $f$ es acotada. Asimismo, $\int_a^b s(x)\; dx \leq \int_a^b t(x)\; dx$ si $s\leq f \leq t$, de modo que todo número de $S$ es menor que cualquiera de $T$. Por consiguiente según el teorema I.34, $S$ tiene extremo superior, y $T$ tiene extremo inferior, que satisfacen las desigualdades $$\int_a^b s(x)\; dx \leq \sup S \leq \inf T \leq \int_a^b y(x)\; dx$$
    para todas las $s$ y $t$ que satisfacen $s\leq f\leq t$. Esto demuestra que tanto $\sup S$ como el $\inf T$ satisfacen $\int_a^b s(x)\; dx \leq I\leq \int_a^b t(x)\; dx$. Por lo tanto $f$ es integrable en $[a,b]$ si y sólo si $\sup S = \inf T$, en cuyo caso se tiene $$\int_a^b f(x)\; dx = \sup S = \inf T.$$
    El número $\sup S$ se llama integral inferior de $f$ y se presenta por $\underbar{I}(f)$. El número $\inf T$ se llama integral superior de $f$ y se presenta por $\bar{I}(f)$. Así que tenemos 
    $$\underbar{I}(f)=\sup \left\{\int_a^b s(x) \; dx \; | \; s\leq f\right\}, \qquad \bar{I}(f)=\inf \left\{\int_a^b t(x) \; dx \; | \; f\leq t\right\}$$\\\\
\end{teo}

\section{El área de un conjunto de ordenadas expresada como una integral}

%--------------------teorema 1.10
\begin{teo}
    Sea $f$ una función no negativa, integrable en un intervalo $[a,b]$, y sea $Q$ el conjunto de ordenadas de $f$ sobre $[a,b]$. Entonces $Q$ es medible y su área es igual a la integral $\int_a^b f(x) \; dx$\\\\
    Demostración.-\; Sea $S$ y $T$ dos regiones escalonadas que satisfacen $S\subseteq Q \subseteq T.$. Existen dos funciones escalonadas $s$ y $t$ que satisfacen $s\leq f\leq t$ en $[a,b]$, tales que $$a(S)=\int_a^b s(x) \; dx \quad y \quad a(T)=\int_a^b t(x) \; dx$$
    Puesto que $f$ es integrable en $[a,b]$ el número $I=\int_a^b f(x) \; dx$ es el único que satisface las desigualdades $$\int_a^b s(x) \leq I \leq \int_a^b t(x) \; dx$$
    para todas las funciones escalonadas $s$ y $t$ que cumplen $s\leq f \leq t$. Por consiguiente ése es también el único número que satisface $a(S)\leq I \leq a(T)$ para todas las regiones escalonadas $S$ y $T$ tales que $S\subseteq Q \subseteq T.$ Según la propiedad de exhaución, esto demuestra que $Q$ es medible y que $a(Q)=I$\\\\
\end{teo}

%--------------------teorema 1.11
\begin{teo}
    Sea $f$ una función no negativa, integrable en un intervalo $[a,b]$. La gráfica de $f$, esto es el conjunto $$\lbrace (x,y)/a\leq x \leq b, y=f(x) \rbrace$$ es medible y tiene área igual a $0$.\\\\
    Demostración.-\; Sean $Q$ el conjunto de ordenadas del teorema 1.11 y $Q^{'}$ el conjunto que queda si se quitan de $Q$ los puntos de la gráfica de $f$. Esto es, $$Q^{'} = \lbrace (x,y)/a\leq x \leq b, 0\leq y \leq f(x)\rbrace$$
    El razonamiento utilizado para demostrar el teorema 1.11 también demuestra que $Q^{'}$ es medible y que $a(Q^{'}) = a(Q)$. Por consiguiente, según la propiedad de la diferencia relativa al área, el conjunto $Q-Q^{'}$ es medible y $$a(Q-Q^{'})=a(Q)-a(Q^{'})=0$$.\\
\end{teo}


\setcounter{section}{19}
\section{Funciones monótomas y monótonas a trozos. Definiciones y ejemplos}

%--------------------definición 1.14
    \begin{def.}[Funciónes crecientes y decreciente]
	Una función $f$ se dice que es creciente en un conjunto $S$ si $f(x)\leq f(y)$ para cada par de puntos $x$ e $y$ de $S$ con $x<y$. Si se verifica la desigualdad estricta $f(x)<f(y)$ par todo $x<y$ en $S$ se dice que la función es creciente en sentido estricto en $S$.\\\\
	Una función se dice decreciente en $S$ si $f(x)\geq f(y)$ para todo $x<y$ en $S$. Si $f(x)>f(y)$ para todo $x<y$ en $S$ la función se denomina decreciente en sentido estricto en $S$.
    \end{def.}

%--------------------definición 1.15
    \begin{def.}[Función monótona]
	Una función se denomina monótona en $S$ si es creciente en $S$ o decreciente en $S$. Monótona en sentido estricto significa que $f$ o es estrictamente creciente en $S$ o estrictamente decreciente en $S$. En general el conjunto $S$ es un intervalo abierto o cerrado.
    \end{def.}

%--------------------definición 1.16
    \begin{def.}[Función monótona a trozos]
	Una función $f$ se dice que es monótona a trozos en un intervalo si su gráfica está formada por un número finito de trozos monótonos. Es decir, $f$ es monótona a trozos en $[a,b]$ si existe una partición $P$ de $[a,b]$ tal que $f$ es monótona en cada uno de los subintervalos abiertos de $P$. 
    \end{def.}

\section{Integrabilidad de funciones monótonas acotadas}

%--------------------teorema 1.12
\begin{teo} Si $f$ es monótona en un intervalo cerrado $[a,b]$, $f$ es integrable en $[a,b]$\\\\
    Demostración.-\; Demostraremos el teorema para funciones decrecientes. El razonamiento es análogo para funciones decrecientes. Supongamos pues $f$ decreciente y sean $\underbar{I}(f)$ e $\overline{I}(f)$ sus integrales inferior y superior. Demostraremos que $\underbar{I}(f)=\overline{I}(f)$. \\
    Sea $n$ un número entero positivo y construyamos dos funciones escalonadas de aproximación $s_n$ y $t_n$ del modo siguiente: $P=\lbrace x_0,x_1,...,x_n \rbrace$ una partición de $[a,b]$ en $n$ subintervalos iguales, esto es, subintervalos $[x_{k-1},x_k]$ tales que $x_k-x_{k-1} = (b-a)/n$ para cada valor de $k$. Definamos ahora $s_n$ y $t_n$ por las fórmulas $$s_n(x)=f(x_{k-1}), \quad t_n(x)=f(x_k) \quad si \quad x_{k-1}<x<x_k$$ en los puntos de división, se definen $s_n$ y $t_n$ de modo que se mantengan las relaciones $s(x)\leq f(x) \leq t_n(x)$ en todo $[a,b]$. Con esta elección de funciones escalonadas, tenemos 
    $$\int_a^b t_n - \int_a^b s_n = \sum\limits_{k=1}^n f(x_k)(x_k-x_{k-1}) - \sum\limits_{k=1}^n f(x_{k-1})(x_k-x_{k-1}) = \dfrac{b-a}{n}\sum\limits_{k=1}^n \left[f(x_{k}) - f(x_{k-1})\right]=$$ $$=\dfrac{(b-a)\left[f(b)-f(a)\right]}{n}$$
    siendo la última igualdad una consecuencia de la propiedad telescópica de las sumas finitas. Esta última relación tiene una interpretación geométrica muy sencilla. La diferencia $\int_a^n t_n - \int_a^b s_n$ es igual a la suma de las áreas de los rectángulos. Deslizando esos rectángulos hacia la derecha, vemos que completan un rectángulo de base $(b-a)/n$ y altura $f(b)-f(a)$; la suma de las áreas es por tanto, $C/n$, siendo $C=(b-a)\left[f(b)-f(a)\right]$.\\
    Volvamos a escribir la relación anterior en la forma $$\int_a^b t_n - \int_a^b s_n = \dfrac{C}{n} \qquad (1)$$
    Las integrales superior e inferior de $f$ satisfacen las desigualdades $$\int_a^b s_n \leq \underbar{I}(f) \leq \int_a^b t_n \quad y \quad \int_a^b s_n \leq \bar{I}(f)\leq \int_a^b t_n$$
    Multiplicando las primeras igualdades por $(-1)$ y sumando el resultado a las segundas,es decir: $$-\underbar{I}(f)\leq - \int_a^b s_n \;\;\land \;\;  \bar{I}(f) \leq \int_a^b t_n \quad \lor \quad -\bar{I}(f)\leq -\int_a^b s_n \;\;\land\;\; \underbar{I}(f)\leq \int_a^b t_n$$ obtenemos $$\bar{I}(f)-\underbar{I}(f) \leq \int_a^b t_n - \int_a^b s_n$$ 
    Utilizando $(1)$ y la relación $\underbar{I}(f) \leq \bar{I}(f)$ obtenemos $$0\leq \bar{I}(f) - \bar{I}(f) \leq \dfrac{C}{n}$$
    para todo entero $n\geq 1$. Por consiguiente, según el teorema I.31 se tiene $$\underbar{I}(f)\leq \bar{I}(f)\leq \underbar{I}(f) + \dfrac{C}{n}$$
    por lo tanto $\underbar{I}(f)=\bar{I}(f)$. Esto demuestra que $f$ es integrable en $[a,b]$.\\\\ 
\end{teo}


\section{Cálculo de la integral de una función monótona acotada}

%--------------------teorema 1.13
\begin{teo} Supongamos $f$ creciente en un intervalo cerrado $[a,b]$. Sea $x_k = a + k(b-a)/n$ para $k=0,1,...,n$. Si $I$ es un número cualquiera que satisface las desigualdades $$\dfrac{b-a}{n}\sum\limits_{k=0}^{n-1} f(x_k)\leq I \leq \dfrac{b-a}{n}\sum\limits_{k=1}^n f(x_k) \qquad (2)$$
    para todo entero $n\geq 1$, entonces $I=\int_a^b f(x) \; dx$\\\\
    Demostración.-\; Sean $s_n$ y $t_n$ las funciones escalonadas de aproximación especial obtenidas por subdivisión del intervalo $[a,b]$ en $n$ partes iguales, como se hizo en la demostración del teorema 1.13. Entonces, las desigualdades (1.9) establecen que $$\int_a^b s_n \leq I \leq \int_a^b t_n$$
    para $n\geq 1$. Pero la integral $\int_a^b f(x) \; dx$ satisface las mismas desigualdades que $I$. Utilizando la igualdad $(1)$ tenemos $I\leq \int_a^b t_n$ \, como también \, $\int_a^b s_n \leq \int_a^b f(x) \; dx \quad \Longrightarrow\quad  - \int_a^b f(x) \; dx \leq -\int_a^b s_n$ entonces $$I-\int_a^b f(x)\; dx \leq \int_a^b t_n - \int_a^b s_n$$
    Similarmente usando las inecuaciones $\int_a^b s_n \leq I$ \, y \, $\int_a^b f(x) \; dx \leq \int_a^b t_n$ resulta que $$\int_a^b f(x) \; dx - I \leq \int_a^b t_n - \int_a^b s_n \quad \Longrightarrow \quad I - \int_a^b f(x) \; dx \geq - \left(\int_a^b t_n - \int_a^b s_n\right)$$ 
    Donde se concluye que $$0\leq \left| I - \int_a^b f(x) \; dx \right| \leq \int_a^b t_n - \int_a^b s_n = \dfrac{C}{n}$$
    par todo $n\geq 1$. Por consiguiente, según el teorema I.31, tenemos $I=\int_a^b f(x) \; dx$\\\\
\end{teo}

%--------------------teorema 1.14
\begin{teo} Supongamos $f$ decreciente en $[a,b]$. Sea $x_k=c+k(b-a)/n$ para $k=0,1,...,n$. Si $I$ es un número cualquiera que satisface las desigualdades $$\dfrac{b-a}{n} \sum\limits_{k=1}^n f(x_k) \leq I \leq \dfrac{b-a}{n} \sum\limits_{k=0}^{n-1} f(x_k)$$
    para todo entero $n\geq 1,$ entonces $I=\int_a^b f(x) \; dx$\\\\
\end{teo}


\section{Cálculo de la integral $\int_0^b x^p \; dx$ siendo $p$ entero positivo}

%--------------------teorema 1.15
\begin{teo} Si $p$ es un entero positivo y $b>0$, tenemos $$\int_0^b x^p \; dx = \dfrac{b^{p+1}}{p+1}$$\\
    Demostración.-\; Comencemos con las desigualdades $$\sum\limits_{k=1}^{n-1} k^p < \dfrac{n^{p+1}}{p+1}<\sum\limits_{k=1}^n k^p$$
    válidas para todo entero $n\geq 1$ y todo entero $p\geq 1$. Estas desigualdades se demostraron anteriormente. La multiplicación de esas desigualdades por $b^{p+1}/n^{p+1}$ nos da $$\dfrac{b}{n} \sum\limits_{k=1}^{n-1} \left(\dfrac{kb}{n}\right)^p<\dfrac{b^{p+1}}{p+1}<\dfrac{b}{n}\sum\limits_{k=1}^{n}\left(\dfrac{kb}{n}\right)^p$$
    Si ponemos, las desigualdades (2) del teorema 1.14 se satisfacen poniendo $f(x)=x^p, \;a=0$, entonces $I=\dfrac{b^{p+1}}{p+1}$. Resulta pues que $$\int_0^b x^p \; dx=\dfrac{b^{p+1}}{p+1}$$\\\\
\end{teo}


\section{Propiedades fundamentales de la integral}

%--------------------teorema 1.16
\begin{teo}[Linealidad respecto al integrando] Si $f$ y $g$ son ambas integrables en $[a,b]$ también lo es $c_1f+c_2g$ para cada par de constantes $c_1$ y $c_2$. Además, se tiene
    $$\int_a^b \left[c_1 f(x) + c_2g(x)\right]\; dx = c_1\int_a^b f(x) \; dx + c_2 \int_a^b g(x) \; dx$$
    Nota.- \; Aplicando el método de inducción, la propiedad de linealidad se puede generalizar como sigue: Si $f_1,...,f_n$ son integrables en $[a.b]$ también lo es $c_1f_1+...+c_nf_n$ para $c_1,...,c_n$ reales cualesquiera y se tiene $$\int_a^b \sum\limits_{k=1}^n c_kf_k(x)\; dx = \sum\limits_{k=1}^n c_k \int_a^b f_k(x)\; dx$$\\

    Demostración.-\; Descompongamos esa propiedad en dos partes:
    $$\int_a^b (f+g) = \int_a^b f + \int_a^b g \qquad (A)$$
    $$\int_a^b cf = c \int_a^b f \qquad (B)$$
    Para demostrar $(A)$, pongamos $I(f) = \int_a^b f$ e $I(g) = \int_a^b g$. Demostraremos que $\underbar{I}(f+g) = \bar{I}(f+g) = I(f) + I(g).$
    Sean $s_1$ y $s_2$ funciones escalonadas cualesquiera inferiores a $f$ y $g$, respectivamente. Puesto que $f$ y $g$ son integrables, se tiene $$I(f) = \sup\left\{ \int_a^b s_1 | s_1 \leq f \right\}, \quad I(g) =\sup\left\{ \int_a^b s_2 | s_2 \leq g\right\}$$
    Por el teorema I.33 aditiva del extremo superior, también se tiene $$I(f) + I(g) = \sup\left\{\int_a^b s_1 + \int_a^b s_2 | s_1 \leq f, s_2\leq g\right\} \qquad (1.11)$$
    Pero si $s_1\leq f$ y $s_2 \leq g,$ entonces la suma $s=s_1+s_2$ es una función escalonada inferior a $f+g$, y tenemos 
    $$\int_a^b s_1 + \int_a^b s_2 = \int_a^b s \leq \underbar{I}(f+g)$$
    Por lo tanto, el número $\underbar{I}(f+g)$ es una cota superior para el conjunto que aparece en el segundo miembro de (1.11). Esta cota superior no puede ser menor que el extremo superior del conjunto de manera que $$I(f) + I(g) \leq \underbar{I}(f+g) \qquad (1.12)$$ 
    Del mismo modo, si hacemos uso de las relaciones 
    $$I(f) = \inf\left\{\int_a^b t_1 | f\leq t_1\right\}, \qquad I(g)= \inf\left\{\int_a^b t_2 | g\leq t_2\right\}$$
    donde $t_1$ y $t_2$ representan funciones escalonadas arbitrarias superiores a $f$ y $g$, respectivamente, obtenemos la desiguald
    $$\bar{I}(f+g) \leq I(f) + I(g).$$
    Las desigualdades (1.12) y (1.13) juntas demuestran que $\underbar{I}(f+g) = \bar{I} (f+g) = I(f) + I(g)$. Por consiguiente $f+g$ es integrable y la relación $(A)$ es cierta.\\
    La relación $(B)$ es trivial si $c=0$. Si $c>0$, observemos que toda función escalonada $s_1=cf$ es de la forma $s_1=cs,$ siendo $s$ una función escalonada inferior a $f$. Análogamente, cualquier función escalonada $t_1$ superior a $cf$ es de la forma $t_1=ct$, siendo $t$ una función escalonada superior a $f$. Por hipótesis se tenemos, 
    $$\bar{I}(cf) = \sup\left\{ \int_a^b s_1 | s_1 \leq cf\right\} = \sup\left\{c\int_a^b s | s \leq f\right\} = cI(f)$$
    y 
    $$\underbar{I}(cf) = \inf\left\{\int_a^b t_1 | cf \leq t_1\right\} = \inf\left\{c\int_a^b t | f \leq t\right\} = cI(f)$$
    Luego $\underbar{I}(cf)=\bar{I}(cf)=cI(f).$ Aquí hemos utilizado las propiedades siguientes del extremo superior y del extremo inferior:
    $$\sup \lbrace cx | x\in A \rbrace = c \sup \lbrace x|x\in A \rbrace,\qquad \inf \lbrace c x | x \in A\rbrace = c \inf\lbrace x|x\in A \rbrace \qquad \mbox{(1.14)}$$
    que son válidas si $c>0$. Esto demuestra $(B)$ si $c>0$.\\
    Si $c<0$, la demostración de $(B)$ es básicamente la misma, excepto que toda función escalonada $s_1$ inferior a $cf$ es de la forma $s_1=ct$, siendo $t$ una función escalonada superior a $f$ y toda función escalonada $t_1$ superior a $cf$ es de la forma $t_1=cs$, siendo $s$ una función escalonada inferior a $f$. Asimismo, en lugar de (1.14) utilizamos las relaciones 
    $$\sup{cx | x\in A} = c\inf{x | x \in A}, \quad \inf{cx | x \in A} = c \sup {x | x \in A},$$
    que son ciertas si $c<0$. Tenemos pues 
    $$\underbar{I}(cf) = \sup\left\{\int_a^b s_1 | s_1 \leq cf\right\}= \sup \left\{c \int_a^b t | f \leq t\right\} = \inf \left\{\int_a^b t | f \leq t\right\} = cI(f).$$
    Análogamente, encontramos $\bar{I}(f)=cI(f)$. Por consiguiente $(B)$ es cierta para cualquier valor real de $c$.\\\\

\end{teo}

%--------------------teorema 1.17
\begin{teo}[Aditividad respecto al intervalo de integración] Si existen dos de las tres integrales siguientes, también existe la tercera y se tiene: $$\int_a^b f(x) \; dx + \int_b^c f(x) \; dx = \int_a^c f(x) \; dx$$
    Nota.- \; En particular, si $f$ es monótona en $[a,b]$ y también en $[b,c]$, existen las dos integrales $\int_a^b f$ e $\int_b^c f,$ con lo que también existe $\int_a^c f$ y es igual a la suma de aquellas.\\\\
    Demostración.-\; Supongamos que $a<b<c$, y que las dos integrales $\in_a^b f$ e $\int_b^c f$ existen. Designemos con $\underbar{I}(f)$ e $\bar{I}(f)$ las integrales superior e inferior de $f$ en el intervalo $[a,c]$. Demostraremos que $$\underbar{I}(f) = \bar{I}(f) = \int_a^b f + \int_b^c f. \qquad (1.15)$$
    Si $s$ es una función escalonada cualquiera inferior a $f$ en $[a,c]$, se tiene $$\int_a^c = \int_a^b + \int_b^c s.$$
    Recíprocamente, si $s_1$ y $s_2$ son funciones escalonadas inferiores a $f$ en $[a,b]$ y $[b,c]$ respectivamente, la función $s$ que coincide con $s_1$ en $[a,b]$ y con $s_2$ en $[b,c]$ es una función escalonada inferior a $f$ para lo que $$\int_a^c = \int_a^b s_1 + \int_b^c s_2.$$
    Por consiguiente, en virtud de la propiedad aditiva del extremo superior, tenemos 
    $$\underbar{I}(f) = \sup\left\{\int_a^c s | s\leq f\right\} = \sup \left\{\int_a^b s_1 | s_1 \leq f\right\} + \sup \left\{\int_b^c s_2 | s_2 \leq f\right\} = \int_a^b f + \int_b^c f.$$
    Análogamente, encontramos $$\bar{I}(f) = \int_a^b f + \int_b^c f,$$
    lo que demuestra (1.15) cuando $a<b<c$. La demostración es parecida para cualquier otra disposición de los puntos $a,b,c$.\\\\

\end{teo}

%--------------------teorema 1.18
\begin{teo}[Invariancia frente a una traslación] Si $f$ es integrable en $[a,b]$, para cada número real $c$ se tiene: $$\int_a^b f(x) \; dx = \int_{a+c}^{b+c} f(x-c)\; dx$$\\
    Demostración.-\; Sea $g$ la función definida en el intervalo $[a+c,b+c]$ por la ecuación $g(x) = f(x-c)$. Designemos por $\underbar{I}(g)$ e $\bar{g}$ las integrales superior e inferior de $g$ en el intervalo $[a+c,b+c]$. Demostraremos que $$\underbar{I}(g) = \bar{I}(g) = \int_a^b f(x) \; dx.$$
    Sea $s$ cualquier función escalonada inferior a $g$ en el intervalo $[a+c,b+c]$. Entonces la función $s_1$ definida en $[a,b]$ por la ecuación $s_1(x) = s(x+c)$ es una función escalonada inferior a $f$ en $[a,b]$. Además, toda función escalonada $s_1$ inferior a $f$ en $[a,b]$ tiene esta forma para un cierta $s$ inferior a $g$. También, por la propiedad de traslación para las integrales de las funciones escalonadas, tenemos $$\int_{a+c}^{b+c} \; dx = \int_a^b s(x+c)\; dx = \int_a^b s_1(x)\; dx. \qquad (1.16)$$
    Por consiguiente se tiene $$\underbar{I}(g) = \sup\left\{\int_{a+c}^{b+c} s | s \leq g\right\} =  \sup \left\{\int_a^b s_1 | s_1 \leq f\right\} = \int_a^b f(x) \; dx.$$
    Análogamente, encontramos $\underbar{I} = \int_a^b f(x)\; dx$, que prueba (1.16).\\\\ 
    
\end{teo}

%--------------------teorema 1.19
\begin{teo}[Dilatación o contracción del intervalo de integración] Si $f$ es integrable en $[a,b]$ para cada número real $k\neq 0$ se tiene: $$\int_a^b f(x) = \dfrac{1}{k} \int_{ka}^{kb} f\left(\dfrac{x}{k}\right)\; dx$$
    Nota.-\; En los dos teoremas 1.19 y 1.20 la existencia de una de las integrables implica la existencia de la otra. Cuando $k=-1$, el teorema 1.19 se llama propiedad de reflexión.\\\\
    Demostración.-\; Supongamos $k>0$ y definamos $g$ en el intervalo $[ka,kb]$ para la igualdad $g(x)=f(x/k)$. Designemos por $\underbar{I}(g)$ e $\bar{I}(g)$ las integrales inferiores y superiores de $g$ en $[ka,kb]$. Demostraremos que $$\underbar{I}(g) = \bar{I}(g) = k \int_a^b f(x) \; dx. \qquad (1.17)$$
    Sea $s$ cualquier función escalonada inferior a $g$ en $[ka,kb]$. Entonces la función definida en $[a,b]$ por la igualdad $s_1(x) = (kx)$ es una función escalonada inferior a $f$ en $[a,b]$. Además, toda función escalonada $s_1$ a $f$ en $[a,b]$ tiene esta forma. También, en virtud de la propiedad de dilatación para las integrales de funciones escalonadas, tenemos 
    $$\int_{ka}^{kb} s(x)\; dx = k \int_a^b s(kx)\; dx = k \int_a^b s_1(x) \; dx.$$
    Por consiguiente $$\underbar{I}(g) = \sup\left\{\int_{ka}^{kb} s | s\leq g\right\} = \sup\left\{ k \int_a^b s_1 | s_1\leq f\right\} = k \int_a^b f(x) \; dx.$$
    Análogamente, encontramos $\underbar{I}(g) = k \int_a^b f(x)\; dx$, que demuestra (1.17) si $k>0.$ El mismo tipo de demostración puede utilizarse si $k<0$.\\\\

\end{teo}

%--------------------teorema 1.20
\begin{teo}[teorema de comparación] Si $f$ y $g$ son ambas integrables en $[a,b]$ y si $g(x)\leq f(x)$ para cada $x$ en $[a,b]$ se tiene: $$\int_a^b g(x)\; dx \leq \int_a^b f(x)\; dx$$\\\\
    Demostración.-\; Supongamos $g\leq f$ en el intervalo $[a,b]$. Sea $s$ cualquier función escalonada inferior a $g$, y sea $t$ cualquier función escalonada superior a $f$. Se tiene entonces $\int_a^b s \leq \int_a^b t,$ y por tanto el teorema I.34 nos da 
    $$\int_a^b = \sup \left\{\int_a^b s | s\leq g \right\} \leq \inf \left\{\int_a^b t | f\leq t\right\} = \int_a^b f.$$
    Esto demuestra que $\int_a^b g \leq \int_a^b f,$ como deseábamos.\\\\

\end{teo}


\section{Integración de polinomios}
Podemos usar el teorema 1.20 para demostrar que (1.11) también es válida para $b$ negativo. Tomemos $k=-1$ en el teorema 1.20 y se obtiene $$\int_0^{-b} = - \int_0^b (-x)^p \; dx = (-1)^{p+1} \int_0^b x^p \; dx = \dfrac{(-b)^{p+1}}{p+1}$$ lo cual prueba la validez de (1.11) para $b$ negativo. La propiedad aditiva $\int_a^b x^p \; dx = \int_0^b x^p \; dx - \int_0^a x^p \;dx$ nos conduce a la formula general: $$\int_a^b x^p \; dx = \dfrac{b^{p+1}-a^{p+1}}{p+1}$$
válida para todos los valores reales de $a$ y $b$, y todo entero $p\geq 0$.\\
Algunas veces el símbolo $$P(x)\left.\right|_a^b$$
se emplea para designar la diferencia $P(b)-P(a)$. De este modo la fórmula anterior puede escribirse así: $$\int_a^b x^p \; dx = \left.\dfrac{x^{p+1}}{p+1}\right|_a^b = \dfrac{b^{p+1}-a^{p+1}}{p+1}$$
Esta fórmula y la propiedad de linealidad, nos permiten integrar cualquier polinomio.\\
Con mayor generalidad, para calcular la integral de cualquier polinomio integramos término a término: $$\int_a^b \sum_{k=0}^n c_k x^k \; dx = \sum_{k=0}^n c_k \int_a^b x^k  \; dx = \sum\limits_{k=0}^n c_k \dfrac{b^{k+1} - a^{a+1}}{k+1}$$\\


\section{Ejercicios}
Calcular cada una de las integrales siguientes:\\\\

\begin{enumerate}[\bfseries 1.]

    %--------------------1.
    \item $\displaystyle\int_0^3 x^2 \;dx = \dfrac{3^{3}}{3} = 9$\\\\

    %--------------------2.
    \item $\displaystyle\int_{-3}^3 x^2 \; dx = \left.\dfrac{x^3}{3}\right|_{-3}^3 = \dfrac{3^3 - (-3)^3}{3} = \dfrac{54}{3} = 18 $\\\\

    %--------------------3.
    \item $\displaystyle\int_0^2 4x^3 \;dx = 4\dfrac{2^4}{4} = 16$.\\\\

    %--------------------4.
    \item $\displaystyle\int_{-2}^2 4x^3 \; dx = 4 \left.\dfrac{x^5}{5}\right|_{-2}^{2} = 4 \dfrac{2^4 - (-2)^4}{4} = 0$\\\\ 

    %--------------------5.
    \item $\displaystyle\int_0^1 5t^4 \; dt = 5\dfrac{1^5}{5} = 1$\\\\

    %--------------------6.
    \item $\displaystyle\int_{-1}^1 5t^4 \; dt = 5\left.\dfrac{t^5}{5}\right|_{-1}^1 = 5\dfrac{1^5 - (-1)^5}{5} = 5$\\\\

    %--------------------7.
    \item $\displaystyle\int_0^1 (5x^4 - 4x^3) \; dx = 5\dfrac{1^5}{5} - 4\dfrac{1^4}{4} = 0$\\\\

    %--------------------8.
    \item $\displaystyle\int_{-1}^1 (5x^4 - 4x^3) \; dx = 5\int_{-1}^1 x^4 \; dx - 4\int_{-1}^1 x^3 \; dx = 5\cdot \left.\dfrac{x^5}{5}\right|_{-1}^1 - 4\cdot \left.\dfrac{x^4}{4}\right|_{-1}^1 = 5\cdot \dfrac{1^5 - (-1)^5}{5} - 4\cdot \dfrac{1^4 - (-1)^4}{4} = 2$\\\\

    %--------------------9.
    \item $\displaystyle\int_{-1}^2 (t^2 + 1)\; dt = \dfrac{t^3}{3}\bigg|_{-1}^{2} + t\bigg|_{-1}^2 = \dfrac{2^3 - (-1)^3}{3} + 2 - (-1) = 6$\\\\

    %--------------------10.
    \item $\displaystyle\int_2^3 (3x^2 - 4x + 2) \; dx = 3\cdot \dfrac{x^3}{3}\bigg|_2^3 - 4\cdot \dfrac{x^2}{2}\bigg|_2^3 + 2\cdot t\bigg|_2^3 = 3\dfrac{3^3-2^3}{3} - 4\dfrac{3^2-2^2}{2} + 2(3-2) = 11$\\\\

    %--------------------11.
    \item $\displaystyle\int_0^{1/2} (8t^3 + 6 t^2 - 2t +5)\; dt  = 8\cdot \dfrac{(1/2)^4}{4} + 6\cdot\dfrac{(1/2)^3}{3} - 2\cdot \dfrac{(1/2)^2}{2} + 5\cdot (1/2) = \dfrac{21}{8} $\\\\

    %--------------------12.
    \item $\displaystyle\int_{-2}^4 (u-1)(u-2)\; du \quad \Longrightarrow \quad \int_{-1}^4 u^2 - 3u + 2\; dx = \dfrac{u^3}{3}\bigg|_{-2}^4 - 3\cdot \dfrac{u^2}{2}\bigg|_{-2}^4 + 2u\bigg|_{-2}^4 = $ $$=\dfrac{4^3 - (-2)^3}{3}-3\dfrac{4^2 - (-2)^2}{2} + 2[4 - (-2)] = \dfrac{73}{3} - 18 + 12 = 18$$\\\\

    %--------------------13.
    \item $\displaystyle\int_{-1}^0 (x+1)^2 \; dx \Longrightarrow \int_{-1+1}^{0+1} (x-1+1)^2 \; dx \; \;(traslación)  \Longrightarrow  \int_{1}^0 x^2 \; dx = \dfrac{x^3}{3}\bigg|_0^1 = \dfrac{1^2 - 0^2}{3} =\dfrac{1}{3}$\\\\

    %--------------------14.
    \item $\displaystyle\int_0^{-1} (x+1)^2\; dx$\\\\
	Respuesta.-\; Por ser $-1<0$ entonces por el teorema de dilatación o contracción del intervalo de integración se tiene que $$\int_{0}^{-1} (x+1)^2\; dx = -\int_{-1}^0 (x+1)^2 \; dx = - \left( \dfrac{x^3}{3}\bigg|_{-1}^0 + 2\cdot \dfrac{x^2}{2}\bigg|_{-1}^0 + x \bigg|_{-1}^0 \right) = - \left[\dfrac{1}{3} + 1 + (-1)\right]  = - \dfrac{1}{3}$$\\

    %--------------------15.
    \item $\displaystyle\int_0^2 (x-1)(3x-1) \; dx = 3\cdot \dfrac{2^3}{3} - 4\cdot \dfrac{2^2}{2} + 2 = 2$\\\\

    %--------------------16.
    \item $\displaystyle\int_0^2 |(x-1)(3x-1)| \; dx$\\\\
	Respuesta.-\; Se evaluará en intervalos en los que sea siempre sea positiva o siempre negativa, como también evaluar por separado. Examinando el polinomio se tiene ceros en $x=1,\frac{1}{3}$, donde la expresión va a cambiar de signo en estos puntos, entonces:
	\begin{center}
	    \begin{tabular}{rcccl}
		$x<\dfrac{1}{3}$ & $\Longrightarrow$ & $(x-1)(3x-1)>0$ & $\Longrightarrow$ & $|(x-1)(3x-1)| = (x-1)(3x-1)$\\\\
		$\dfrac{1}{3}<x<1$ & $\Longrightarrow$ & $(x-1)(3x-1)<0$ & $\Longrightarrow$ & $|(x-1)(3x-1)| = -(x-1)(3x-1)$\\\\
		$x>1$ & $\Longrightarrow$ & $(x-1)(3x-1)>0$ & $\Longrightarrow$ & $|(x-1)(3x-1)| = (x-1)(3x-1)$\\\\
	    \end{tabular}
	\end{center}
	Luego por el teorema  aditivo de integración $$\int_0^2 (x-1)(3x-1)\; dx = \int_0^{1/3}(x-1)(3x-1)\; dx + \int_{1/3}^1 -(x-1)(3x-1)\; dx + \int_1^2 (x-1)(3x-1)\; dx$$
	Así, nos queda 
	$$\int_0^2 3x^2 - 4x + 1\; dx = \int_0^{1/3}3x^2 - 4x + 1\; dx + \int_{1/3}^1 -(3x^2 - 4x + 1)\; dx + \int_1^2 3x^2 - 4x + 1\; dx$$
	por lo tanto
	$$\left(3\dfrac{x^3}{3}-3\dfrac{x^2}{2}+x\right)\bigg|_0^{1/3} - \left(3\dfrac{x^3}{3}-3\dfrac{x^2}{2} + x\right)\bigg|_{1/3}^1 + \left(3\dfrac{x^3}{3} - 4\dfrac{x^2}{2} + x\right) \bigg|_1^2 = $$
	$$ = \left(\dfrac{1}{27}-\dfrac{2}{9}+\dfrac{1}{3}\right)-0 - \left[ (1-2+1)-\left(\dfrac{1}{27}-\dfrac{2}{9}+\dfrac{1}{3}\right)\right] = \dfrac{4}{27}+\dfrac{4}{27}+2=\dfrac{62}{27}$$\\

    %--------------------17.
    \item $\displaystyle\int_0^3 (2x-5)^3 \; dx$\\\\
	Respuesta.-\; Por el teorema de invariancia frente a una traslación se tiene
	$$\int_0^3 (2x-5)^3 \; dx \int_{-5/2}^{1/2} \left[2\left(x+\dfrac{5}{2}\right)-5\right]^3 \; dx = \int_{-5/2}^{1/2} (2x)^3 \; dx = 80\dfrac{x^4}{4}\bigg|_{-5/2}^{1/2}\; dx = \dfrac{1}{8}-\dfrac{625}{8}=-78$$\\

    %--------------------18.
    \item $\displaystyle\int_{-3}^3 (x^2-3)^3 \; dx = \int_{-3}^3 x^6 - 9x^4 + 27x^2 - 27 \; dx = \left(\dfrac{x^7}{7}-9\dfrac{x^5}{5}+27\dfrac{x^3}{3}-27x\right)\bigg|_{-3}^3 =$
	$$=\left(\dfrac{2187}{7}-\dfrac{2187}{5}+243-81\right)-\left(-\dfrac{2187}{7}+\dfrac{2187}{5}-243+81\right)=\dfrac{2592}{35}$$\\


    %--------------------19.
    \item $\displaystyle\int_0^5 (x-5)^4 \; dx = \int_0^5 x^4 - 20x^3\cdot + 150 x^2\cdot + - 500x + 625 \; dx = \left(\dfrac{5^5}{5} - 3125 + 6250 - 6250 + 3125 \right) = 625$\\\\

    %--------------------20.
    \item $\displaystyle\int_{-2}^{-4} (x + 4)^{10} \; dx$\\\\
	Respuesta.-\; Por la invariancia de la integral se tiene 
	$$\int_{-2}^{-4} (x+4)^{10} \; dx = \int_2^0 x^{10} \; dx = - \int_0^2 x^{10} \; dx = - \dfrac{2^{11}}{11}$$\\

    %--------------------21.
    \item Hallar todos los valores de $c$ para los que

    \begin{enumerate}[\bfseries (a)]

	%----------(a)
	\item $\displaystyle\int_0^c x(1-x) \; dx = 0$\\\\
	    Respuesta.-\; Al integrar tenemos que $\displaystyle\int_0^c x - x^2 \; dx = \dfrac{c^2}{2} - \dfrac{c^3}{3}$ igualando el resultado a $0$ se tiene $3c^2 - 2c^3 = 0$ de donde $$c=0 \quad y \quad c=\dfrac{3}{2}$$.\\\\

	%----------(b)
	\item $\displaystyle\int_0^c |x(1-x)| \; dx$\\\\
	    Respuesta.-\; Dado que $|x(1-x)|\geq 0$ para todo $x$, podemos usar el teorema de comparación para ver si $|x(1-x)|>0$ para cualquier $x$, de donde $$\int_0^c |x(1-x)| \; dx > \int_0^c 0 = 0$$
	    Así, para que la ecuación se mantenga debemos tener $|x(1-x)|=0$ para todo $0\leq x \leq c$ Dado que la expresión será distinta de cero para cualquier $0<x\leq c$, debemos tener $c=0$\\\\\\\\
    \end{enumerate}

    %-------------------22.
    \item Calcular cada una de las integrales siguientes. Dibújese la gráfica $f$ en cada caso.
	\begin{enumerate}[\bfseries (a)]

	    %----------(a)
	    \item $\displaystyle\int_0^2 f(x) \; dx$ donde 
		$f(x) =  \left\{ \begin{array}{rcl}
		    x^2&si&0\leq x \leq 1\\
		    2-x&si&1\leq x\leq 2\\\\
		\end{array}\right.$
		Respuesta.-\; Por el teorema de aditividad respecto al intervalo de integración se tiene 
		\begin{center}
		    \begin{tabular}{rcl}
			$\displaystyle\int_0^2 f(x) \; dx$&$=$&$\displaystyle\int_0^1 f(x) \; dx + \int_1^2 f(x) \; dx$\\
			\\&$=$&$\displaystyle\int_0^1 x^2 \; dx + \int_1^2 (2-x) \; dx$\\
			\\&$=$&$\dfrac{1^3}{3} + 2x\bigg|_1^2  - \dfrac{x^2}{2}\bigg|_1^2$\\
			\\&$=$&$\dfrac{1}{3} + (2\cdot 2 - 2\cdot 1) - \dfrac{2^2 - 1^2}{2}$\\
			\\&$=$&$\dfrac{5}{6}$\\\\
		    \end{tabular}
		\end{center}

	    %----------(b)
	    \item $\displaystyle\int_0^1 f(x)\; dx$ donde 
		$f(x) = \left\{ \begin{array}{rcl}
		    x&si&0\leq x \leq c\\
		    \\c\dfrac{1-x}{1-c}&si&c\leq x \leq 1\\
		\end{array}\right.$
		$c$ es un número real fijado, $0<c<1.$\\\\
		Respuesta.-\; dividimos la integral en $\displaystyle\int_0^c x \; dx  + \int_c^1 \left( \dfrac{c}{1-c} \right) (1-x) \; dx$ \; de donde $$\dfrac{x^2}{2}\bigg|_0^c + \dfrac{c}{1-c} \left(x \bigg|_c^1 - \dfrac{x^2}{2}\bigg|_c^1\right) = \dfrac{c^2}{2} + \dfrac{c}{1-c}(1 - c) - \dfrac{c}{1-c}\left(\dfrac{1^2 - c^2}{2}\right) = \dfrac{c}{2}$$\\

	\end{enumerate}

    %--------------------23.
    \item Hallar un polinomio cuadrático $P$ para el cual $P(0) = P(1) = 0$ y $\displaystyle\int_0^1 P(x) \; dx = 1$.\\\\
	Respuesta.-\; Sea $P(x)=ax^2+bx+c$ entonces para $P(0)=0$ nos queda $c=0$ y para $P(1)$ nos da 
	$$a+b+c = 0 \; \Longrightarrow \; a=-b$$
	De donde $$\int_0^1 P(x) \; dx = 1 \; \Longrightarrow \; \int_0^1 (ax^2 - ax) \; dx \; \Longrightarrow \; a\left(\dfrac{x^3}{3} - \dfrac{x^2}{2}\right)\bigg|_0^1 = 1 \; \Longrightarrow \; a = -6$$
	Por lo tanto se tiene $$P(x)=6x-6x^2$$\\

    %--------------------24.
    \item Hallar un polinomio cúbico $P$ para el cual $P(0)=P(-2)=0,\; P(1)=15$ y $\displaystyle\int_{-2}^0 P(x) \; dx = 4$\\\\
	Respuesta.-\; Sea $ax^3 + bx^2 + cx + d$ si $P(0)$ entonces $d=0$. Luego para $P(-2)=0$ se tiene $$a(-2)^3 + b(-2)^2 + c(-2) = 0 \; \Longrightarrow \; -8a + 4b -2c = 0 \; \Longrightarrow \; c=2b-4a \qquad (1)$$ 
	Después si $P(1)=15$ entonces $a + b + 2b-4a = 15 \; \Longrightarrow \; b = a + 5 \qquad (2)$\\\\
	Remplazando en la ecuación cuadrática e integrando se tiene $$ \int_{-2}^0 \left[ax^3 + (5+a)x^2 + (10-2a)x\right] \; dx = 4$$ 
	de donde 
	$$\begin{array}{rcl}
	    a\dfrac{x^4}{4}\bigg|_{-2}^0 + (5+a)\dfrac{x^3}{3}\bigg|_{-2}^0 + (10-2a)\dfrac{x^2}{2}\bigg|_{-2}^0 = 4  &\Longrightarrow& a\left( \dfrac{- (-2)^4}{4}\right) + (5+a)\left(\dfrac{-(-2)^3}{3}\right)\\\\
																  &+& (10-2a)\left(\dfrac{-(-2)^2}{2}\right) = 4
	\end{array}$$

	Así nos queda $$-4a + \dfrac{40}{3} + \dfrac{8}{3}a - 20 + 4a = 4 \; \Longrightarrow \; a=4$$
	Por lo tanto $$4x^3 + 9x^2 + 2x$$\\

    %--------------------25.
    \item Sea $f$ una función cuyo dominio contiene $-x$ siempre que contiene $x$. Se dice que $f$ es una función par si $f(-x)=f(x)$ y una función impar si $f(-x)=-f(x)$ para todo $x$ en el dominio de $f$. Si $f$ es integrable en $[0,b]$ demostrar que:
	\begin{enumerate}[\bfseries (a)]

	    %----------(a)
	    \item $\displaystyle\int_{-b}^b f(x)\; dx = 2\int_0^b f(x)\; dx$ si $f$ es par.\\\\
		Demostración.-\; Por el teorema de adición de una integral tenemos $$\int_{-b}^b f(x) \; dx = \int_{-b}^0 f(x) \; dx + \int_0^b f(x) \; dx$$
		De la primera integral y por el teorema de dilatación o contracción del intervalo de integración, con $k=-1$, tenemos, $$\int_{-b}^0 f(x) \; dx = -\int_b^0 f(-x)\; dx$$ 
		Siendo la función par, y el hecho de que $-\int_b^0 = \int_0^b$ entonces  $$-\displaystyle\int_{b}^0 f(-x)\; dx = \int_0^b f(x) \; dx$$
		Así, nos queda $$\int_{-b}^b f(x) \; dx = \int_0^b f(x) \; dx + \int_0^b f(x) \; dx = 2\int_0^b f(x) \; dx$$\\

	    %----------(b)
	    \item $\displaystyle\int_a^b f(x) \; dx = 0$ si $f$ es impar.\\\\
		Demostración.-\; Sabemos que $f$ es impar por lo tanto $\displaystyle\int_0^b f(-x)\; dx = -\int_0^b f(x) \; dx$. Luego Similar a la parte $(a)$ nos queda  
		$$\int_{-b}^b f(x)\; dx = \int_{-b}^0 f(x) \; dx + \int_0^b f(x) \;dx = \int_0^b f(-x)\; dx + \int_0^b f(x) \; dx = 0$$\\

	\end{enumerate}

    %--------------------26.
    \item Por medio de los teoremas 1.18 y 1.19 deducir la fórmula $$\int_a^b f(x) \; dx = (b-a)\int_0^1 f[a+(b-a)x] \; dx$$\\
	Demostración.-\; Por los teoremas mencionados en la hipótesis se tiene $$\int_a^b f(x) \; dx = \int_0^{b-a} f[x-(-a)] \; dx = (b-a) \int_0^1 f[x(b-a)+a] \; dx = \int_0^1 f[a+(b-a)x] \; dx$$\\ 

    %--------------------27.
    \item Los teoremas 1.18 y 1.19 sugieren una generalización de la integral $\int_a^b f(Ax+B) \; dx$. Obtener esa fórmula y demostrarla con el auxilio de los citados teoremas. Discutir también el caso $A=0$.\\\\
	Demostración.-\; Sea 
	$$\int_a^b f(Ax+B)\;dx = \left\{ \begin{array}{rcl} 
	    \dfrac{1}{A}\displaystyle\int_{Ab+B}^{Aa+B}&si&A\neq 0\\\\
	    (b-a)f(B)&si&A = 0\\\\
	\end{array}\right.$$
	Para el caso $A=0$, tenemos $$\int_a^b f(Ax+B)\; dx = \int_{a}^b f(B)\; dx = f(B)\int_a^b \; dx = (b-a)f(B)$$.
	Luego para el caso $A\neq 0$ usamos el teorema de expansión o contracción del intervalo de integración, para obtener, 
	$$\int_a^b f(Ax+B)\; dx = \dfrac{1}{A}\int_{Ab}{Aa} f(x+B)\; dx$$
	así por la invariancia de la integral concluimos que 
	$$\dfrac{1}{A}\int_{Aa}^{Ab} f(x+B)\; dx = \dfrac{1}{A}\int_{Aa+b}^{Ab+B} f(x)\; dx$$\\\\

    %-------------------28.
    \item Mediante los teoremas 1.18 y 1.19 demostrar la fórmula 
	$$\int_a^b f(c-x)\; dx = \int_{c-a}^{c-b}f(x) \; dx$$\\\\
	Demostración.-\; Por el teorema de expansión o contracción con $k=-1$ y el teorema de invariancia frente a la traslación, 
	$$\int_a^b f(c-x)\; dx = - \int_{-a}{-b} f(x+c)\; dx = \int_{-b}^{-a} f(x+c)\; dx = \int_{c-b}^{c-a} f(x) \; dx$$\\\\

\end{enumerate}


    %---------- Aplicaciones De La Integracion
	%\chapter{Algunas aplicaciones de la integración}

\setcounter{section}{1}
\section{El área de una región comprendida entre dos gráficas expresada como una integral}

%--------------------2.1
\begin{teo} Supongamos que $f$ y $g$ son integrables y que satisfacen $f\leq q$ en $[a,b]$. La región $S$ entre sus gráficas es medible y su área $a(S)$ viene dada por la integral $$a(S) = \int_a^b [g(x)-f(x)] \; dx$$
    Demostración.- \; 
	Demostración.- \; Supongamos primero que $f$ y $g$ son no negativas,. Sean $F$ y $G$ los siguientes conjuntos:
	$$F = {(x,y)|a\leq x \leq b, 0\leq y \leq f(x)}, \quad G = {(x,y) | a\leq x \leq b, 0\leq y \leq g(x)}.$$
	Esto es, $G$ es el conjunto de ordenadas de $g$, y $F$ el de $f$, menos la gráfica de $f$. La región $S$ es la diferencia $G-F$. Según los teoremas 1.10 y 1.11, $F$ y $G$ son ambos medibles. Puesto que $F \subseteq G$ la diferencia $S = G-F$ es también medible, y se tiene 
	$$a(S) = a(G) - a(F) = \int_a^b g(x) \; dx = \int_a^b [g(x)-f(x)] \; dx$$
	Consideremos ahora el caso general cuando $f\leq q$ en $[a,b]$, pero no son necesariamente no negativas. Este caso lo podemos reducir al anterior trasladando la región hacia arriba hasta que quede situada encima del eje $x$. Esto es, elegimos un número positivo $c$ suficientemente grande que asegure que $0 \leq f(x) + c \leq g(x) + c$ para todo $x$ en $[a,b]$. Por lo ya demostrado la nueva región $T$ entre las gráficas de $f+c$ y $g+c$ es medible, y su parea viene dad por la integral
	$$a(T) = \int_a^b [(g(x)+c) - (f(x)+c)] = \int_[g(x)-f(x)] \; dx$$
	Pero siendo $T$ congruente a $S$, ésta es también medible y tenemos $$a(S) = a(T) = \int_a^b [g(x) - f(x)]\; dx$$
	Esto completa la demostración.\\\\
\end{teo}

%----------nota 2.1
\begin{tcolorbox}[colframe = white]
    \begin{nota} En los intervalos $[a,b]$ puede descomponerse en un número de subintervalos en cada uno de los cuales $f\leq g$ o $g\leq f$ la fórmula (2.1) del teorema 2.1 adopta la forma 
    $$a(S) = \int_a^b |g(x) -f(x)| \; dx$$
    \end{nota}
\end{tcolorbox}

    %--------------------lema 2.1
    \begin{lema}[Área de un disco circular] Demostrar que $A(r) = r^2 A(1)$. Esto es, el área de un disco de radio $r$ es igual al producto del área de un disco unidad (disco de radio $1$) por $r^2$.\\\\
	Demostración.-\; Ya que $g(x) - f(x) = 2g(x),$ el teorema 2.1 nos da 
	    $$A(r) = \int_{-r}^r g(x) \; dx = 2 \int_{-r}^r \sqrt{r^2 - x^2} \; dx$$
	    En particular, cuando $r = 1$, se tiene la fórmula $$A(1) = 2\int_{-1}^1 \sqrt{1 - x^2} \; dx$$
	    Cambiando la escala en el eje $x$, y utilizando el teorema 1.19 con $k=1/r$, se obtiene
	    $$A(r) = 2\int_{-r}^r g(x) \; dx = 2r \int_{-1}^1 g(rx) \; dx = 2r\int_{-1}^1 \sqrt{r^2 - (rx)^2} \; dx = 2r^2 \int_{-1}^1 \sqrt{1-x^2} \; dx = r^2 A(1)$$
	    Esto demuestra que $A(r) = r^2 A(1)$, como se afirmó.\\\\
    \end{lema}

%---------------------definición 2.1
\begin{tcolorbox}[colframe = white]
    \begin{def.} Se define el número $\pi$ como el área de un disco unidad.\\\\
    \end{def.}
\end{tcolorbox}
\begin{center}
    La formula que se acaba de demostrar establece que $A(r) = \pi r^2$\\
\end{center}

\begin{tcolorbox}[colframe = white]
Generalizando el anterior lema se tiene 
\begin{center}
    $$a(kS) = \int_{ka}^{kb} g(x)\; dx = k \int_{ka}^{kb} f(x/k) \; dx = k^2 \int_a^b f(x) \; dx$$
\end{center}
\end{tcolorbox}

%--------------------teorema 2.1
\begin{teo} Para $a>0$, $b>0$ y $n$ entero positivo, se tiene $$\int_a^b x^{\frac{1}{n}} \; dx = \dfrac{b^{1-1/n} - a^{1-\frac{1}{n}}}{1+\frac{1}{n}}$$\\
    Demostración.-\; Sea $\int_0^a x^{\frac{1}{n}}$. El rectángulo de base $a$ y altura $a^{\frac{1}{n}}$ consta de dos componentes: el conjuntos de ordenadas de $f(x) = x^{\frac{1}{n}}$ a $a$ y el conjuntos de ordenadas $g(y) = y^n$ a $a^{\frac{1}{n}}$. Por lo tanto,
    $$a\cdot a^{\frac{1}{n}} = a^{1+\frac{1}{n}} = \int_0^a x^{\frac{1}{n}} \; dx + \int_0^{a^{\frac{1}{n}}} y^n \; dy \; \Longrightarrow \; \int_0^a x^{\frac{1}{n}} \; dx = a^{1+\frac{1}{n}} - \dfrac{y^{n+1}}{n+1}\bigg|_0^{a^{\frac{1}{n}}} = a^{1+\frac{1}{n}} -  \dfrac{a^{1+\frac{1}{n}}}{n+1} = \dfrac{a^{1+\frac{1}{n}}}{1 + 1/n}$$
    Análogamente se tiene $$\int_0^b x^{\frac{1}{n}}\; dx = \dfrac{b^{1 + \frac{1}{n}}}{1 + 1/n}$$
    Luego notemos que $$\int_a^b x^{\frac{1}{n}} \; dx = \int_0^b x^{\frac{1}{n}}\; dx - \int_0^a x^{\frac{1}{n}} \;dx$$
    por lo tanto $$\int_a^b x^{\frac{1}{n}}\; dx = \dfrac{b^{1+\frac{1}{n}} - b^{1 + \frac{1}{n}}}{1 + 1/n}$$\\\\
\end{teo}



\setcounter{section}{3}
\section{Ejercicios}

En los ejercicios del 1 al 14, calcular el área de la región $S$ entre las gráficas de $f$ y $g$ para el intervalo $[a,b]$ que en cada caso se especifica. Hacer un dibujo de las dos gráficas y sombrear $S$.\\

\begin{enumerate}[\Large\bfseries 1.]

%--------------------1.
\item 

\end{enumerate}




    %---------- Funciones continuas
	%\input{codigoFuente/tom_apostol/3-FuncionesContinuas.tex}

    %---------- Cálculo diferencial
	%\chapter{Cálculo diferencial}

\begin{comment}
\setcounter{section}{3}
\section{Derivada de una función}

\begin{tcolorbox}
    \begin{def.}[Definición de derivada]
	La derivada $f'(x)$ está definida por la igualdad 
	$$f'(x)=\lim_{h\to 0}\dfrac{f(x+h)-f(x)}{h},$$
	siempre que exista el límite. El número $f'(x)$ también se denomina coeficiente de variación de $f$ en $x$.
    \end{def.}
\end{tcolorbox}
\vspace{.7cm}

\begin{ejem}[Derivada de una función potencial de exponente entero positivo]
    Consideremos el caso $f(x)=x^n$, siendo $n$ un entero positivo. El cociente de diferencias es ahora\\
    $$\dfrac{f(x+h)-f(x)}{h}=\dfrac{(x+h)^n-x^n}{h}$$\\
    Para estudiar este cociente al tender $h$ a cero, podemos proceder de dos maneras, o por la descomposición factorial del numerador considerado como diferencia de dos potencias n-simas o aplicando el teorema del binomio para el desarrollo de $(x + h)^n$. Seguiremos con el primer método.\\

    En álgebra elemental se tiene la identidad
    $$a^n - b^n = (b-a)\sum_{k=0}^{n-1} a^k b^{n-1-k}$$
    Si se toma $a=x+h$ y $b=x$ y dividimos ambos miembros por $h$, esa identidad se transforma en \\
    $$\dfrac{(x+h)^n-x^n}{h} = \sum_{k=0}^{n-1}(x+h)^k x^{n-1-k}$$\\
    En la suma hay $n$ términos. Cuando $h$ tiende a $0$, $(x+h)^k$ tiende a $x^k$, el k-ésimo término tiende a $x^k x^{n-1-k}=x^{n-1},$ y por tanto la suma de los $n$ términos tiende a $nx^{n-1}$. De esto resulta que \\
    $$f'(x)=nx^{n-1}\quad \forall \; x.$$\\
\end{ejem}

\begin{ejem}[Derivada de la función seno]
    Sea $s(x)=\sen x$. El cociente de diferencias es
    $$\dfrac{s(x+h)-s(x)}{h}=\dfrac{\sen(x+h)-\sen x}{h}$$
    Para transformarlo de modo que haga posible calcular el límite cuando $h\to 0$, utilizamos la identidad trigonométrica
    $$\sen y -\sen x = 2\sen \dfrac{y-x}{2}\cos \dfrac{y+x}{2}$$
    poniendo $y=x+h$. Esto conduce a la fórmula
    $$\dfrac{\sen(x+h)-\sen x}{h}=\dfrac{\sen\frac{h}{2}}{\frac{h}{2}} \cos \left(x+\dfrac{h}{2}\right)$$
    Como $h\to 0$, el factor $\cos(x+\frac{1}{2}h)\to \cos x$ por la continuidad del coseno. Así mismo, la fórmula 
    $$\lim_{x\to 0}\dfrac{\sen x}{x}=1,$$
    demuestra que
    $$\dfrac{\sen \frac{h}{2}}{\frac{h}{2}}\to 1\;\mbox{ para todo } \; h\to 0$$
    Por lo tanto el cociente de diferencias tiene como límite $\cos x$ cuando $h\to 0$. Dicho de otro modo, $s'(x)=\cos x;$ para todo $x$; la derivada de la función seno es la función coseno.\\\\ 
\end{ejem}

\begin{ejem}[Derivada de la función coseno]
    Sea $c(x)=\cos x$. Demostraremos que $c'(x)=-\sen x$; esto es, la derivada de la función coseno es menos la función seno. Partamos de la identidad
    $$\cos y - \cos x = -2\sen \dfrac{y-x}{2}\sen \dfrac{y+x}{2}$$
    y pongamos $y=x+h$. Esto nos conduce a la fórmula
    $$\dfrac{\cos(x+h)-\cos x}{h}=-\dfrac{\sen \frac{h}{2}}{\frac{h}{2}}\sen \left(x+\dfrac{h}{2}\right).$$
    La continuidad del seno demuestra que $\left(x+\frac{1}{2}h\right)\to x$ cuando $h\to 0$; luego ya que 
    $$\dfrac{\sen \frac{h}{2}}{\frac{h}{2}}\to 1\;\mbox{ para todo } \; h\to 0$$
    obtenemos $c'(x)=-\sen x$.\\
\end{ejem}

\begin{ejem}[Derivada de la función raíz n-esima]
    Si $n$ es un entero positivo, sea $f(x)=x^{1/n}$ para $x>0$. El cociente de diferencias para $f$ es
    $$\dfrac{f(x+h)-f(x)}{h}=\dfrac{(x+h)^{1/n}-x^{1/n}}{h}.$$
    Pongamos $u=(x+h)^{1/n}$ y $v=x^{1/n}$. Tenemos entonces $u^n = x+h$ y $v^n=x,$ con lo que $h=u^n-v^n$, y el cociente de diferencias toma la forma
    $$\dfrac{f(x+h)-f(x)}{h}=\dfrac{u-v}{u^n -v^n}=\dfrac{1}{u^{n-1}+u^{n-2}v + \ldots + uv^{n-2}v^{n-1}}$$
    La continuidad de la función raíz n-sima prueba que $u \to v$ cuando $h \to O$. Por consiguiente cada término del denominador del miembro de la derecha tiene límite $v^{n-1}$ cuando $h \to O$. En total hay $n$ términos, con lo que el cociente de diferencias tiene como límite $v^{1-n}/n$. Puesto que  $v = x^{1/n}$, esto demuestra que
    $$f'(x)=\dfrac{1}{n}x^{1/n-1}.$$\\
\end{ejem}

\begin{tcolorbox}
    \begin{ejem}[Continuidad de las funciones que admiten derivada]
	Si una función $f$ tiene derivada en un punto $x$, es también continua en $x$. Para demostrar, empleamos la identidad
	$$f(x+h)=f(x)+h\left(\dfrac{f(x+h)-f(x)}{h}\right)$$
	que es válida para $h\neq 0$. Si hacemos que $h\to 0$, el cociente de diferencias del segundo miembro tiende a $f'(x)$, puesto que este cociente está multiplicando por un factor que tiende a $0$, el segundo término del segundo miembro tiende a $0\cdot f'(x)$. Esto demuestra que $f(x+h)\to f(x)$ cuando $h\to 0$ y por tanto que $f$ es continua en $x$.
    \end{ejem}
\end{tcolorbox}

\section{Álgebra de las derivadas}

\begin{teo}
    Sean $f$ y $g$ dos funciones definidas en un intervalo común. En cada punto en que $f$ y $g$ tienen derivadas, también las tienen la suma $f+g$, la diferencia $f-g$, el producto $f\cdot g$ y el cociente $f/g$. (Para $f/g$ hay que añadir también que $g$ ha de ser distinta de cero en el punto considerado). Las derivadas de estas funciones están dadas por las siguientes fórmulas:
    \begin{enumerate}[(i)]
	\item $(f+g)' = f'+g',$
	\item $(f-g)' = f'-g',$
	\item $(f\cdot g)'=f\cdot g' + g\cdot f',$
	\item $\left(\dfrac{f}{g}\right)' = \dfrac{g\cdot f' - f\cdot g'}{g^2}$ en puntos $x$ donde $g(x)\neq 0$.\\\\
    \end{enumerate}
	Demostración.-\; Demostración de (i). Sea $x$ un punto en el que existen ambas derivadas $f'(x)$ y $g'(x)$: El cociente de diferencias para $f+g$ es
	$$\dfrac{(f+g)(x+h)-(f+g)(x)}{h}=\dfrac{[f(x+h)+g(x+h)]-[f(x)+g(x)]}{h}=\dfrac{f(x+h)-f(x)}{h}+\dfrac{g(x+h)-g(x)}{h}$$
	Cuando $h\to 0$, el primer cociente del segundo miembro tiende a $f'(x)$ y el segundo a $g'(x)$ y por tanto la suma tiende a $f'(x)+g'(x)$. La demostración de (ii) es análoga.\\

	Demostración de (iii). El cociente de diferencias para el producto $f\cdot g$ es:
	$$\dfrac{f(x+h)g(x+h)-f(x)g(x)}{h}.$$
	Para estudiar este cociente cuando $h\to 0$ se suma y resta al numerador un término conveniente para que se puede escribir la fórmula dada como la suma de dos términos en los que aparezca los cocientes de diferencias de $f$ y $g$. Sumando y restando $f(x)f(x+h)$ se convierte en
	$$\dfrac{f(x+h)g(x+h)-f(x)g(x)}{h}=g(x)\dfrac{f(x+h)-f(x)}{h}+f(x+h)\dfrac{g(x+h)-g(x)}{h}.$$
	Cuando $h\to 0$ el primer término del segundo miembro tiende a $g(x)f'(x)$, y puesto que $f(x+h)\to f(x),$ el segundo término tiende a $f(x)g'(x)$, lo que demuestra (iii).\\

	Demostración de (iv). Un caso particular de (iv) se tiene cuando $f(x)=1$ para todo $x$. En este caso $f'(x)=0$ y (iv) se reduce a la fórmula
	$$\left(\dfrac{1}{g}\right)'=-\dfrac{g'}{g^2}$$
	suponiendo que $f(x)\neq 0$ partir de este caso particular, se puede deducir la fórmula general (iv) escribiendo $f/g$ como producto y aplicando (iii), con lo cual se tiene:
	$$\left(f\cdot \dfrac{1}{g}\right)'=\dfrac{1}{g}\cdot f' + f\cdot \left(\dfrac{1}{g}\right)'=\dfrac{f'}{g}-\dfrac{f\cdot g'}{g^2}=\dfrac{g\cdot f'-f\cdot g'}{g^2}$$
	Por tanto, queda solamente por probar $\left(\dfrac{1}{g}\right)'=-\dfrac{g'}{g^2}$. El cociente de diferencias de $1/g$ es:
	$$\dfrac{\frac{1}{g(x+h)}-\frac{1}{g(x)}}{h} = -\dfrac{g(x+h)-g(x)}{h}\cdot \dfrac{1}{g(x)}\dfrac{1}{g(x+h)}$$
	Cuando $h\to 0,$ el primer cociente de la derecha tiende a $g'(x)$ y el tercer factor tiende a $\frac{1}{g(x)}$. Se requiere la continuidad de $g$ en $x$ ya que se hace uso del hecho que $g(x+h)\to g(x)$ cuando $h\to 0$. Por tanto, el cociente tiende a $-\dfrac{g'(x)}{g(x)^2}$.\\\\

\end{teo}

Un caso particular de (iii) se tiene cuando una de las dos funciones es constante, por ejemplo, $g(x) = c$ para todo valor de $x$. En este caso, (iii) se transforma en: $(c \cdot f)' = c\cdot f'$; es decir, la derivada del producto de una función por una constante es el producto de la derivada de la función por la constante. Combinando esta propiedad con la de la derivada de una suma [propiedad (i)] se tiene, que para cada par de constantes $c_1$ y $c_2$, es:
$$(c_1f+c_2g)'=c_1f'+c_2g'$$
Esta propiedad se denomina propiedad lineal de la derivada, y es análoga a la propiedad lineal de la integral.\\
Aplicando el método de inducción se puede extender la propiedad lineal a un número cualquiera finito de sumandos:
$$\left(\sum_{i=1}^n c_i\cdot f_i\right)' = \sum_{i=1}^n c_i\cdot f_i',$$
donde $c_1, c_2, c_3,\ldots , c_n$ son constantes y $f_1, f_2, \ldots , f_n$ son funciones cuyas derivadas son $f_1',f_2',\ldots, f_n'$.\\
Cuando se consideran los valores de estas funciones en un punto x, se obtienen fórmulas entre números; así la fórmula (i) implica
$$(f+g)(x)=f'(x)+g'(x)$$

\begin{ejem}[funciones Racionales]
    Si $r$ es el cociente de dos polinomios, es decir, $r(x)=p(x)/q(x)$, la derivada $r'(x)$ se puede calcular por medio de la fórmula del cociente (iv) del teorema 4.1. La derivada existe para todo $x$ en el que $g(x)\neq 0$. Obsérvese que la función $r'$ así definida es a su vez una función racional. En particular, si $r(x)=1/x^m$ donde $m$ es un entero positivo y $x\neq 0$ se tiene:
    $$r'(x)=\dfrac{x^m\cdot 0 - mx^{m-1}}{x^{2m}}=\dfrac{-m}{x^{m+1}}$$
    Escribiendo este resultado en la forma: $r'(x) = - mx^{-m-1}$ se obtiene una extensión a exponentes negativos de la fórmula dada para la derivación de potencias n-simas para $n$ positivo.\\\\
\end{ejem}

\section{Ejercicios}

\begin{enumerate}[\bfseries 1.]

    %--------------------1.
    \item Si $f(x)=2+x-x^2$, calcular $f'(0),f'(\frac{1}{2}),f'(1),f'(-10)$.\\\\
	Respuesta.-\; Por definición sea,
	$$\begin{array}{rcl}
	    f'(x)&=&\lim\limits_{h\to 0}\dfrac{\left[2+(x+h)-(x+h)^2\right]-(2+x-x^2)}{h}\\\\
		 &=&\lim\limits_{h\to 0}\dfrac{2+x+h-x^2-2xh-h^2-2-x+x^2}{h}\\\\
		 &=&\lim\limits_{h\to 0} \dfrac{h(1-2x-h)}{h}\\\\
		 &=&1-2x\\\\
	\end{array}$$
	De donde 
	$$\begin{array}{rclcr}
	    f'(0)&=&1-2\cdot 0 &=&1\\\\
	    f'(\frac{1}{2})&=&1-2\cdot\frac{1}{2} &=& 0\\\\
	    f'(1)&=&1-2\cdot 1 &=& -1\\\\
	    f'(-10)&=&1-2\cdot (-10) &=& 21\\\\
	\end{array}$$
	\vspace{.7cm}

    %--------------------2.
    \item Si $f(x)=\frac{1}{3}x^3+\frac{1}{2}x^2-2x$, encontrar todos los valores de $x$ para los que \\
	\begin{enumerate}[(a)]

	    %---------- (a)
	    \item $f'(x)=0.$\\\\
		Respuesta.-\; Sea $f'(x)=x^2+x-2$, entonces 
		$$x^2+x-2=0\quad \Rightarrow \quad x_1=1\quad \mbox{y} \quad x_2=-2.$$\\

	    %---------- (b)
	    \item $f'(x)=-2$.\\\\
		Respuesta.-\;  Sea $f'(x)=x^2+x-2$, entonces
		$$x^2+x-2=-2\quad \Rightarrow \quad x^2+x = 0  \quad \Rightarrow \quad x_1=0 \quad \mbox{y} \quad x_2=-1.$$\\

	    %---------- (c)
	    \item $f'(x)=10$.\\\\
		Respuesta.-\; Sea $f'(x)=x^2+x-2$, entonces
		$$x^2+x-2=10\quad \Rightarrow \quad x^2+x -12 = 0  \quad \Rightarrow \quad x_1=3 \quad \mbox{y} \quad x_2=-4.$$\\\\

	\end{enumerate}

    En los ejercicios del 3 al 12, obtener una fórmula para $f'(x)$ si $f(x)$ es la que se indica.\\\\

    %--------------------3.
    \item $f(x)=x^2+2x+2$.\\\\
	Respuesta.-\; Por definición,
	$$\begin{array}{rcl}
	    f'(x)&=&\lim\limits_{h\to 0}\dfrac{(x+h)^2+2(x+h)+2-(x^2+2x+2)}{h}\\\\
		 &=&\lim\limits_{h\to 0}\dfrac{x^2+2xh+h^2+2x+2h+2-x^2-2x-2}{h}\\\\
		 &=&\lim\limits_{h\to 0}\dfrac{h^2+2xh+2h}{h}\\\\
		 &=&\lim\limits_{h\to 0}\dfrac{h(h+2x+2)}{h}\\\\
		 &=&2x+2\\\\
	\end{array}$$

    %--------------------4.
    \item $f(x)=x^4+\sen x$.\\\\
	Respuesta.-\; Ya que $(f+g)'=f'+g'$ y sabiendo que la derivada de $\sen x$ es $\cos x$, entonces
	$$f'(x) = 3x^2+\cos x.$$\\

    %--------------------5.
    \item $f(x)=x^4\sen x$.\\\\
	Respuesta.-\; Ya que $(fg)'=f\cdot g'+g\cdot f'$ y la derivada de $\sen x$ es $\cos x$, entonces
	$$f'(x) = x^4\cos x + 4x^3 \sen x.$$\\

    %--------------------6.
    \item $f(x)=\dfrac{1}{x+1},\quad x\neq -1.$\\\\
	Respuesta.-\; Sean $g(x)=1$ y $h(x)=x+1$. Sabemos que $\left(\dfrac{g}{h}\right)'=\dfrac{h\cdot g'-g\cdot h'}{h^2}$, entonces
	$$f'(x)=\dfrac{(x+1)\cdot 1' -1\cdot(x+1)'}{(x+1)^2}=\dfrac{1\cdot0 - 1\cdot 1}{(x+1)^2}=\dfrac{1}{(x+1)^2}.$$\\

    %--------------------7.
    \item $f(x)=\dfrac{1}{x^2+1}+x^5\cos x$.\\\\
	Respuesta.-\; Sean, $k(x)=1,\; g(x)=x^2+1,\; h(x)=x^5$ y $j(x)=\cos x$. Ya que $\left(\dfrac{k}{g}\right)'=\dfrac{g\cdot k'-k\cdot g'}{g^2}$, $(hj)'=h\cdot j'+j\cdot h'$ y la derivada de $\cos x$ es $-\sen x$, entonces 
	$$\begin{array}{rcl}
	    &=&\lim\limits_{h\to 0} \dfrac{(x^2+1)\cdot1'-1\cdot (x^2+1)'}{(x^2+1)^2}+x^5\cdot (\cos x)'+\cos x\cdot (x^5)'\\\\
	    &=&\lim\limits_{h\to 0}\dfrac{- 2x}{(x^2+1)^2}+x^5\cdot (-\sen x) + \cos x\cdot 5x^4\\\\
	\end{array}$$

    %--------------------8.
    \item $f(x)=\dfrac{x}{x-1},\quad x\neq 1$.\\\\
	Respuesta.-\; Sean $k(x)=x$ y $g(x)=x-1$. Ya que $\left(\dfrac{k}{g}\right)'=\dfrac{g\cdot k'-k\cdot g'}{g^2}$ para $g\neq 0$, entonces
	$$f'(x) = \dfrac{x-1 - x}{(x-1)^2} = -\dfrac{1}{(x-1)^2}.$$\\

    %--------------------9.
    \item $f(x)=\dfrac{1}{2+\cos x}.$\\\\
	Respuesta.-\; $$f'(x)=\dfrac{(2+\cos x)\cdot 0+\sen x}{(2+\cos x)^2}=\dfrac{\sen x}{(2+\cos x)^2}.$$\\

    %--------------------10.
    \item $f(x)=\dfrac{x^2+3x+2}{x^4+x^2+1}$.\\\\
	Respuesta.-\; $$f'(x)=\dfrac{(x^4+x^2+1)(2x+3)-(x^2+3x+2)(4x^3+2x)}{(x^4+x^2+1)^2} = \dfrac{2x^5+9x^4+8x^3+3x^2+2x-3}{(x^4+x^2+1)^2}.$$\\

    %--------------------11.
    \item $f(x)=\dfrac{2-\sen x}{2-\cos x}$.\\\\
	Respuesta.-\; $$\begin{array}{rcl}
	    f'(x)&=&\dfrac{(-\cos x)(2-\cos x)-(2-\sen x)(\sen x)}{(2-\cos x)^2}\\\\
		 &=&\dfrac{-2\cos x + \cos^2 x - 2\sen x +\sen^2 x}{(2-\cos x)^2}\\\\
		 &=&\dfrac{1-2(\sen x + \cos x)}{(2-\cos x)^2}
	\end{array}$$
	\vspace{.7cm}

    %--------------------12.
    \item $f(x)=\dfrac{x\sen x}{1+x^2}$.\\\\
	Respuesta.-\; Primero escribimos 
	$$f(x)=\dfrac{x\sen x}{1+x^2}=x\left(\dfrac{\sen x}{1+x^2}\right)$$
	Luego usando el la regla del producto y del cociente para derivadas tenemos,
	$$f'(x)=\dfrac{\sen x}{1+x^2}+x\left[\dfrac{(1+x^2)\cos x - 2x\sen x}{(1+x^2)^2}\right] =\dfrac{\sen x + x \cos x}{1+x^2}- \dfrac{2x^2\sen x}{(1+x^2)^2}.$$\\

    %--------------------13.
    \item Se supone que la altura $f(t)$ de un proyectil, $t$ segundos después de haber sido lanzado hacia arriba a partir del suelo con una velocidad inicial de $v_0$ metros por segundo, está dada por la fórmula:
	$$f(t)=v_0t-16t^2.$$

	\begin{enumerate}[(a)]

	    %---------- (a)
	    \item Aplíquese el método descrito en la Sección 4.2 para probar que la velocidad media del proyectil durante el intervalo de tiempo de $t$ a $t+h$ es $v_0-32t-16h$ pies sobre segundo, y que la velocidad instantánea en el instante $t$ es $v_0-32t$ pies por segundo.\\\\
		Respuesta.-\; La velocidad media es dada por,
		$$\dfrac{f(t+h)-f(t)}{h} = \dfrac{v_0(t+h)-16(t+h)^2-v_0t+16t^2}{h} = v_0-32t-16h.$$\\
		Luego, la velocidad instantánea está dada por,
		$$\begin{array}{rcl}
		\lim\limits_{h\to 0}\dfrac{v_0(t+h)-16(t+h)^2-v_0t+16t^2}{h}&=&\lim\limits_{h\to 0}\dfrac{v_0t+v_0h-16t^2-32th-16h^2-v_0t+16t^2}{h}\\\\
									    &=& \lim\limits_{h\to 0}\dfrac{v_0h-32th-16t^2}{h}\\\\
									    &=&\lim_{h\to 0}v_0-32t-16h\\\\
									    &=&v_0-32t.\\\\
		\end{array}$$

	    %---------- (b)
	    \item Calcúlese (en función de $v_0$) el tiempo necesario para que la velocidad se anule.\\\\
		Respuesta.-\; Para ello igualamos $v_0-32t$ a cero como sigue,
		$$v_0-32t=0\quad \Rightarrow \quad t=\dfrac{v_0}{32}.$$\\

	    %---------- (c)
	    \item ¿Cuál es la velocidad de retorno a la Tierra?.\\\\
		Respuesta.-\; Sea $f'(t)=0$, entonces 
		$$v_0t-16t^2=0\quad \Rightarrow \quad t(v_0-16t)=0\quad \Rightarrow \quad t=\dfrac{v_0}{16}.$$
		Esto significa que el proyectil regresa a la tierra luego de $\frac{v_0}{16}$ segundos. Luego la velocidad de retorno será:
		$$v\left(\dfrac{v_0}{16}\right) = v_0-32\cdot \dfrac{v_0}{16}=v_0-2v_0=-v_0.$$\\

	    %---------- (d)
	    \item ¿Cuál debe ser la velocidad inicial del proyectil para que regrese a la tierra al cabo de $1$ segundo? ¿y al cabo de $10$ segundos? ¿y al cabo de $T$ segundos?.\\\\
		Respuesta.-\; La velocidad inicial para que el proyectil regrese a la tierra luego de un segundo será:
		$$f(1)=0\quad \Rightarrow \quad v_0\cdot 1 -16\cdot 1^2 = 0 \quad \Rightarrow \quad v_0=16.$$
		Después la velocidad inicial para que el proyectil regrese a la tierra luego de 10 segundos será:
		$$f(10)=0\quad \Rightarrow \quad v_0\cdot 10-16\cdot 10^2 = 0\quad \Rightarrow \quad v_0=160.$$
		Y para que vuelva luego de $T$ segundos será:
		$$f(T)=0\quad \Rightarrow \quad v_0\cdot T-16\cdot T^2 = 0\quad \Rightarrow \quad v_0=16T.$$\\

	    %---------- (e)
	    \item Pruébese que el proyectil se mueve con aceleración constante.\\\\
		Demostración.-\; La aceleración constante viene dada por $f''(t)$, es decir,
		$$f''(t)=v'(t)=\dfrac{d}{dt}(v_0-32t)=-32.$$\\

	    %---------- (f)
	    \item Búsquese un ejemplo de otra fórmula para la altura que dé lugar a una aceleración constante de $-20$ pies por segundo al cuadrado.\\\\
		Respuesta.-\; Sea $f(t)=v_0t-10t^2$, entonces $f'(t)=v_0-20t$ y $f''(t)=-20\; ft/sec^2$.\\\\

	\end{enumerate}

    %--------------------14.
    \item ¿Cuál es el coeficiente de variación del volumen de un cubo con respecto a la longitud de cada lado?.\\\\
	Respuesta.-\; El volumen de un cubo viene dado por $V(x)=x^3$, por lo tanto el coeficiente de variación vendrá dado por,
	$$V'(x)=3x^2.$$\\

    %--------------------15.
    \item 
	\begin{enumerate}[(a)]

	    %---------- (a)
	    \item El área de un círculo de radio $r$ es $\pi r^2$ y su circunferencia es $2\pi r$. Demostrar que el coeficiente de variación del área respecto al radio es igual a la circunferencia.\\\\
		Demostración.-\; Sea $C(r)=\pi r^2$. Dado que el coeficiente de variación es la derivada de $C(r)$ entonces,
		$$C'(r)=2\pi r.$$
		Tal como se quiere.\\\\

	    %---------- (b)
	    \item El volumen de una esfera de radio $r$ es $4\pi r^3/3$ y su área es $4\pi r^2.$ Demostrar que el coeficiente de variación del volumen respecto al radio es igual al área.\\\\
		Demostración.-\; Sea $V(r)=4\pi r^3/3$. Dado que el coeficiente de variación es la derivada de $V(r)$ entonces,
		$$V'(r)=4\pi r^2.$$\\

	\end{enumerate}

    En los ejercicios del 16 al 23, obtener una fórmula para $f'(x)$ si $f(x)$ es la que se indica.\\\\

    %--------------------16.
    \item $f(x)=\sqrt{x},\qquad x>0$. \\\\
	Respuesta.-\; Usando la definición de derivada para número con coeficiente racional se tiene,
	$$f'(x)=\dfrac{1}{2}x^{-\frac{1}{2}},\qquad x>0.$$\\

    %--------------------17.
    \item $f(x)=\dfrac{1}{1+\sqrt{x}},\qquad x>0$. \\\\
	Respuesta.-\; Usando la derivada para cocientes y la derivada de una potencia racional se tiene,
	$$f'(x)=\dfrac{(1+\sqrt{x})\cdot 0 - 1\cdot \frac{1}{2}x^{-\frac{1}{2}}}{(1+\sqrt{x})^2}=-\dfrac{1}{2\sqrt{x}(1+\sqrt{x})^2}\qquad x>0.$$\\

    %--------------------18.
    \item $f(x)=x^{3/2},\qquad x>0$. \\\\
	Respuesta.-\; Usando la derivada para potencias racionales se tiene,
	$$f'(x)=\dfrac{3}{2}x^{\frac{1}{2}},\qquad x>0.$$\\

    %--------------------19.
    \item $f(x)=x^{-3/2},\qquad x>0$. \\\\
	Respuesta.-\; Usando la derivada para potencias racionales se tiene,
	$$f'(x)=-\dfrac{3}{2}x^{-\frac{5}{2}},\qquad x>0.$$\\

    %--------------------20.
    \item $f(x)=x^{1/2}+x^{1/3}+x^{1/4},\qquad x>0$. \\\\
	Respuesta.-\; Usando la derivada para potencias racionales se tiene,
	$$f'(x)=\dfrac{1}{2}x^{-\frac{1}{2}}+\dfrac{1}{3}x^{-\frac{2}{3}}+\dfrac{1}{4}x^{-\frac{3}{4}},\qquad x>0.$$\\

    %--------------------21.
    \item $x^{-1/2}+x^{-1/3}+x^{-1/4},\qquad x>0$. \\\\
	Respuesta.-\; Usando la derivada para potencias racionales se tiene,
	$$f'(x)=-\dfrac{1}{2}x^{-\frac{3}{2}}-\dfrac{1}{3}x^{-\frac{4}{3}}-\dfrac{1}{4}x^{-\frac{5}{4}},\qquad x>0.$$\\

    %--------------------22.
    \item $f(x)=\dfrac{\sqrt{x}}{1+x},\qquad x>0$. \\\\
	Respuesta.-\; Usando la derivada para cocientes y la derivada de una potencia racional se tiene,
	$$f'(x)=\dfrac{\frac{1}{2}x^{-\frac{1}{2}(1+x)-\sqrt{x}}}{(1+x)^2} = \dfrac{1+x-2x}{2\sqrt{x}(1+x)^2} = \dfrac{1-x}{2\sqrt{x}(1+x)^2}.$$\\

    %--------------------23.
    \item $f(x)=\dfrac{x}{1+\sqrt{x}},\qquad x>0$. \\\\
	Respuesta.-\; Usando la derivada para cocientes y la derivada de una potencia racional se tiene,
	$$f'(x)=\dfrac{1+\sqrt{x}-x\left(\frac{1}{2}x^{-\frac{1}{2}}\right)}{(1+\sqrt{x})^2} = \frac{2+\sqrt{x}}{2(1+\sqrt{x})^2}.$$\\

    %--------------------24.
    \item Sean $f_1,\ldots,f_n$ funciones que admiten derivadas $f_1',\ldots, f_n'$. Dar una regla para la derivación del producto $g=f_1\ldots f_n$ y demostrarla por inducción. Demostrar que para aquellos puntos $x$, en los que ninguno de los valores $f_1(x),\ldots,f_n(x)$ es cero, tenemos
	$$\dfrac{g'(x)}{g(x)}=\dfrac{f_1'(x)}{f_1(x)}+\ldots + \dfrac{f_n'(x)}{f_n(x)}.$$\\
	Demostración.-\; Según la derivada para productos se tendría que demostra,
	$$g' = f_1'\cdot f_2\cdot f_3 \cdots f_n + f_1\cdot f_2' \cdot f_3 \cdots f_n + \ldots + f_1\cdot f_2 \cdot f_{n-1} \cdot f_n'.$$
	Par ello tomamos $n=2$ por lo que la derivada nos queda,
	$$f'=f_1'\cdot f_2 + f_1\cdot f_2'$$
	Por lo que es cierto para $n=2$. Supongamos luego que la afirmación:
	$$g'=f_1'\cdot f_2\cdot f_3 \cdots f_n + f_1\cdot f_2'\cdot f_3 \cdots f_n + \ldots + f_1\cdot f_2 \cdots f_{n-1}\cdot f_n'.$$
	es cierta para $n$. Por lo que demostraremos que es cierta para $n+1$ de la siguiente manera,
	$$\begin{array}{rcl}
	    g&=&(f_1\cdot f_2\cdots f_n)\cdot f_{n+1}\\\\
	     &=&(f_1\cdot f_2 \cdots f_n)'\cdot f_{n+1}+(f_1\cdot f_2\cdot \cdots f_n)\cdot f_{n+1}'\\\\
	     &=&f_1'\cdot f_2 \cdot f_3 \cdots f_n\cdot f_{n+1}+f_1\cdot f_2'\cdot f_3 \cdots f_n\cdot f_{n+1}\\\\
	     &+& \ldots + f_1\cdot f_2 \cdots f_{n-1}\cdot f_n'\cdot f_{n+1}+f_1\cdot f_2 \cdots f_n\cdot f_{n+1}'.\\\\
	\end{array}$$	
	Por lo tanto es cierto para $n+1$, así para $g=f_1\cdot f_2 \cdots f_n$ tenemos la derivada,
	$$g' = f_1'\cdot f_2\cdot f_3 \cdots f_n + f_1\cdot f_2' \cdot f_3 \cdots f_n + \ldots + f_1\cdot f_2 \cdot f_{n-1} \cdot f_n'.$$
	Por otro lado. Sea $x$ un punto tal que $f_1(x)\neq 0$ para $i=1,\ldots,n$, entonces
	$$\dfrac{g'(x)}{g(x)}= \dfrac{f_1'\cdot f_2 \cdots f_n}{f_1 \cdots f_n} + \ldots  + \dfrac{f_1 \cdots f_{n-1}\cdot f_n'}{f_1\cdots f_n} = \dfrac{f_1'\cdots f_n}{f_1 \cdots f_2}+\ldots + \dfrac{f_1\cdots f_n'}{f_1\cdots f_n} = \dfrac{f_1'}{f_1}+\ldots + \dfrac{f_n'}{f_n}$$

    %--------------------25.
    \item Comprobar la pequeña tabla de derivadas que sigue. Se sobreentiende que las fórmulas son válidas para aquellos valores de $x$ para los que $f(x)$ está definida.\\
	    $$\begin{array}{cc|cc}
		f(x) & f'(x) & f(x) & f'(x) \\
		\hline
		\tan x & \sec^2x & \sec x & \tan x \sec x\\
		       \cot x&-\csc^2 x&\csc x&-\cot x \csc x\\
	    \end{array}$$\\

	Demostración.-\; Verificaremos que si $f(x)=\tan x$ entonces $f'(x)=\sec^2 x$. Sabemos que por la identidad trigonométrica 
	$$\tan x = \dfrac{\sen x}{\cos x}$$
	Luego, aplicando las distintas reglas de derivación,
	$$f'(x) = \dfrac{\cos x \sen x - \sen x (-\sen x)}{\cos^2x} = \dfrac{\cos^2 x + \sen^2 x}{\cos^2 x}=\dfrac{1}{\cos^2 x} = \sec^2 x.$$\\
	Ahora, verificaremos que si $f(x)=\cot x$ implica que $f'(x)=-\csc^2x$. Sea $f(x)=\cot x =\dfrac{\cos x}{\sen x}$. De donde,
	$$f'(x)=\dfrac{(-\sen^2x)\sen x-\cos x\cos x}{\sen^2x} = \dfrac{-\sen^2 x - \cos^2 x}{\sen^2 x} = -\dfrac{1}{\sen^2 x} = -\csc^2x.$$\\
	Después verificamos que si $f(x)=\csc x$ entonces $f'(x)=\tan x \sec x$. Sea $f(x)=\sec x =\dfrac{1}{\cos x}$. Usando las reglas de derivación tenemos,
	$$f'(x)=\dfrac{\sen x}{\cos^2 x} = \dfrac{\sen x}{\cos x}\cdot \dfrac{1}{\cos x} = \tan x \sec x.$$\\
	Por último verificaremos que si $f(x)=\csc x$ entonces $f'(x)=-\cot \csc x.$ Sea $f(x)=\csc x = \dfrac{1}{\sen x}$. De donde,
	$$f'(x)=\dfrac{-\cos x}{\sen^2 x}=-\dfrac{\cos x}{\sen^2 x}\cdot \dfrac{1}{\sen x} = -\cot x \csc x.$$\\\\

En los Ejercicios del 26 al 35. Calcular la derivada $f'(x)$. Se sobrentiende que cada fórmula será válida para aquellos valores de $x$ para los que $f(x)$ esté definida.\\\\

    %--------------------26.
    \item $f(x)=\tan x \sec x.$\\\\
	Respuesta.-\; Sean $(\tan x)'=\sec^2 x$ y $(\sec x)'=\tan x \sec x$, entonces usando las reglas de derivación para el producto,
	$$f'(x)=\sec^2 x \sec x + \tan x (\tan x \sec x)=\sec^3x + \tan^2 x \sec x.$$\\

    %--------------------27.
    \item $f(x)=x\tan x$.\\\\
	Respuesta.-\; Sea $(\tan x)'=\sec^2 x$, entonces usando las reglas de derivación para el producto se tiene,
	$$f'(x)=\tan x + x\sec^2 x.$$\\

    %--------------------28.
    \item $f(x)=\dfrac{1}{x}+\dfrac{2}{x^2}+\dfrac{3}{x^3}$.\\\\
	Respuesta.-\; Usando la regla de derivación para cocientes tenemos,
	$$f'(x)=-\dfrac{1}{x^2}-\dfrac{4}{x^3}-\dfrac{9}{x^4}.$$\\

    %--------------------29.
    \item $f(x)=\dfrac{2x}{1-x^2}$.\\\\
	Respuesta.-\; Usando la regla de derivación para cocientes tenemos,
	$$f'(x)=\dfrac{2(1-x^2)-2x(-2x)}{(1-x^2)^2}=\dfrac{2+2x^2}{(1-x^2)^2}.$$\\

    %--------------------30.
    \item $f(x)=\dfrac{1+x-x^2}{1-x+x^2}$.\\\\
	Respuesta.-\; Usando la regla de derivación para cocientes tenemos,
	$$f'(x)=\dfrac{(1-2x)(1-x+x^2)-(1+x-x^2)(-1+2x)}{(1-x+x^2)^2}=\dfrac{2(1-2x)}{(1-x+x^2)^2}.$$\\

    %--------------------31.
    \item $f(x)=\dfrac{\sen x}{x}$.\\\\
	Respuesta.-\; Usando la regla de derivación para cocientes tenemos,
	$$f'(x)=\dfrac{(\cos x)x-\sen x}{x^2}.$$\\

    %--------------------32.
    \item $f(x)=\dfrac{1}{x+\sen x}$.\\\\
	Respuesta.-\; Usando la regla de derivación para cocientes tenemos,
	$$f'(x)=\dfrac{-1-\cos x}{(x+\cos x^2)^2}.$$\\

    %--------------------33.
    \item $f(x)=\dfrac{ax+b}{cx+d}$.\\\\
	Respuesta.-\; Usando la regla de derivación para cocientes tenemos,
	$$f'(x)=\dfrac{a(cx+d)-(ax+b)c}{(cx+d)^2}=\dfrac{ad-bd}{(cx+d)^2}.$$\\

    %--------------------34.
    \item $f(x)=\dfrac{\cos x}{2x^2+3}$.\\\\
	Respuesta.-\; Usando la regla de derivación para cocientes tenemos,
	$$f'(x)=\dfrac{-\sen x(2x^2+3)-(\cos x)4x}{(2x^2+3)^2}=-\dfrac{\sen x(2x^2+3)-4x\cos x}{(2x^2+3)^2}.$$\\

    %--------------------35.
    \item $f(x)=\dfrac{ax^2+bx+c}{\sen x + \cos x}$.\\\\
	Respuesta.-\; Usando la regla de derivación para cocientes tenemos,
	$$\begin{array}{rcl}
	    f'(x)&=&\dfrac{(2ax+b)(\sen x + \cos x)-(ax^2+bx+c)(\cos x - \sen x)}{(\sen x + \cos x)^2}\\\\
		 &=&\dfrac{(2ax+b)(\sen x + \cos x)+(ax^2+bx+c)(\sen x - \cos x)}{1+\sen 2x}\\\\
	\end{array}$$

    %--------------------36.
    \item Si $f(x)=(ax+b)\sen x + (cx+d)\cos x$, determinar valores de las constantes $a,b,c,d$ tal que $f'(x)=x\cos x$.\\\\
	Respuesta.-\; Sea 
	$$f'(x)=a\sen x + (ax+b)\cos x + c\cos x + (cx+d)(-\sen x)=(a-d-cx)\sen x + (ax+b+c)\cos x.$$
	Luego por hipótesis tenemos,
	$$x\cos x = (a-d-cx)\sen x+(ax+b+c)\cos x$$
	Comparando los coeficientes de $cos x$ y $\sen x$, se tiene
	$$a-d-cx = 0\qquad \mbox{y}\qquad ax+b+c=x \; \Rightarrow \;(a-1)x+b+c=0.$$
	De donde,
	$$cx=0 \quad \Rightarrow \quad c=0 \qquad \mbox{y}\qquad (a-1)x=0\quad \Rightarrow \quad a=1$$
	Igualando nos queda,
	$$1-d = 0 \quad \Rightarrow \quad d=1.$$
	Por lo tanto 
	$$\begin{array}{rcl}
	    x\cos x &=& (1-0x-1)\sen x + (1x-0+0)\cos x \\
		    &=&0\sen x +x \cos x\\
		    &=&x\cos x\\
	\end{array}$$
	\vspace{.7cm}

    %--------------------37.
    \item Si $g(x)=(ax^2+bx+c)\sen x + (dx^2+ex+f)\cos x,$ determinar valores de las constantes $a,b,c,d,e,f$ tal es que $g'(x)=x^2\sen x$.\\\\
	Respuesta.-\; Sea,
	$$\begin{array}{rcl}
	    g'(x)&=&(2ax+b)\sen x + (ax^2+bx+c)\cos x + (2dx+e)\cos x - (dx^2+cx+f)\sen x\\
		 &=&\left[-dx^2+(2a-e)x+(b+f)\right]\sen x + \left[ax^2+(b+2d)+(c+e)\right]\cos x\\
	\end{array}$$
	Ya que $g'(x)=x^2\sen x$ tenemos
	$$-dx^2+(2a-c)x+b+f=x^2\qquad \mbox{y}\qquad ax^2+(b+2d)+c+e=0.$$
	Luego, de la ecuación de la izquierda, igualando potencias similares de $x$ tenemos,
	$$d=-1,\quad 2a-e=0,\quad b+f=0.$$
	Por otro lado de la ecuación de la derecha, igualando potencias similares de $x$ tenemos,
	$$a=0,\quad b+2d=0,\quad c+e=0.$$
	Por último igualando todas estos resultados, nos queda
	$$2\cdot 0 - 0 \; \Rightarrow \; e=0,\qquad b+2(-1)=0 \; \Rightarrow \; b=2, \qquad c+0=0 \; \Rightarrow \; c=0,\qquad 2+f=0\; \Rightarrow \; f=-2.$$
	Por lo tanto,
	$$a=0,\quad b=2,\quad d=-1,\quad c=0,\quad f=-2.$$\\

    %--------------------38.
    \item Dada la fórmula 
	$$1+x+x^2+\ldots + x_n = \dfrac{x^{n+1}-1}{x-1}$$

	válida si $x\neq 1$, determinar por derivación, fórmulas para las siguientes sumas:\\

	\begin{enumerate}[(a)]

	    %---------- (a)
	    \item $1+2x+3x^2+\ldots + nx^{n-1}$.\\\\
		Respuesta.- Veamos que 
		$$(1+x+x^2+\ldots+x^n)' = 1+2x+3x^2+\ldots + nx^{n-1}.$$
		Por lo que,
		$$\begin{array}{rcl}
		    1+2x+3x^2+\ldots+nx^{n-1} &=&\left(\dfrac{x^{n+1}}{x-1}\right)'\\\\
					      &=&\dfrac{(x-1)\left[(n+1)x^n\right]-(x^{n+1}-1)1}{(x-1)^2}\\\\
					      &=&\dfrac{nx^{n+1}-(n+1)x^n + 1}{(x-1)^2}\\\\
		\end{array}$$


	    %---------- (b)
	    \item $1^2x+2^2x^2+3^2x^3+\ldots + n^2x^n$.\\\\
		Respuesta.-\; Sea 
		    $$(1+2x+3x^2+\ldots+nx^{n-1})'=\left(\dfrac{nx^{n+1}-(n+1)x^n + 1}{(x-1)^2}\right)'$$

		$$\begin{array}{cl}
		    \Rightarrow&2+6x+\ldots + n(n-1)x^{n-2}\\\\
			       &=\dfrac{(x-1)^2\left[n(n+1)x^n - n(n+1)x^{n-1}\right]-(2x-2)\left[nx^{n+1}-(n+1)x^n+1\right]}{(x-1)^4}\\\\
		    \Rightarrow&\left[1+4x+9x^2+\ldots + (n-1)^2x^{n-2}\right]+\left[1+2x+3x^2+\ldots + (n-1)x^{n-2}\right]\\\\
			       &=\dfrac{(n-1)\left[n(n+1)x^n - n(n+1)x^{n-1}\right]-2\left[nx^{n+1}-(n+1)x^n + 1\right]}{(x-1)^3}\\
		    \Rightarrow & \left(\dfrac{1}{x}\right) \left[1^2x+2^2x^2+\ldots+(n-1)^2x^{n-1}\right]+\left[1+2x+\ldots + (n-1)x^{n-2}\right]\\\\
				&= \dfrac{(n^2-n)x^{n+1}-2(n^2-1)x^n + n(n+1)x^{n-1}-2}{(x-1)^3}\\\\
		\end{array}$$
		Entonces, por el segundo término en la suma de la izquierda tenemos,
		$$\begin{array}{rcl}
		    1+2x+\ldots + (n-1)x^{n-2} &=& (1+x+\ldots + x^n)' -nx^{n-1} \\\\
					       &=&\left(\dfrac{x^{n+1}-1}{x-1}\right)'-nx^{n-1}\\\\
					       &=&\dfrac{nx^{n+1}-(n+1)x+1}{(x-1)^2}-nx^{n-1}\\\\
					       &=&\dfrac{(n+1)x^n-nx^{n-1}+1}{(x-1)^2}.\\\\
		\end{array}$$

		Por lo tanto, reemplazando esto en la expresión anterior,
		$$\begin{array}{cl}
		    &\dfrac{1}{x}\left[1^2x+2^2x^2+\ldots + (n-1)^2x^{n-1}\right]\\\\
		    &=\dfrac{(n^2-n)x^{n+1}-2(n^2-1)x^n + n(n+1)x^{n-1}-2}{(x-1)^3}-\dfrac{(n-1)x^n-nx^{n-1}+1}{(x-1)^2}\\\\
		    \Rightarrow & 1^2x+2^2x^2+\ldots + (n-1)^2 x^{n-1} + n^2 x^n\\\\
				&=x\left\{\dfrac{(n^2-n)x^{n+1}-2(n^2-1)x^n + n(n+1)x^{n-1}-2-\left[(x-1)((x-1)x^n+nx^{n-1}+1)\right]}{(x-1)^3}\right\}\\\\
		    \Rightarrow & 1^2+x+2^2x^2+\ldots + (n-1)^2 x^{n-1} + n^2 x^n\\\\
				&=\dfrac{n^2x^{n+3}-(2n^2+2n-1)x^{n+2}+(n+1)^2x^{n+1}-x^2-x}{(x-1)^3}\\\\
		\end{array}$$


	\end{enumerate}

    %--------------------39.
    \item Sea $f(x)=x^n$, siendo $n$ entero positivo. Utilizar el teorema del binomio para desarrollar $(x+h)^n$ y deducir la fórmula
    $$\dfrac{f(x+h)-f(x)}{h}=nx^{n-1}+\dfrac{n(n-1)}{2}x^{n-2}h + \ldots + nxh^{n-2}+h^{n-1}$$
    Expresar el segundo miembro en forma de sumatorio. Hágase que $h\to 0$ y deducir que $f'(x)=nx^{n-1}$. Indicar los teoremas relativos a límites que se han empleado.\\\\
	Respuesta.-\; Sea 
	$$(n+h)^n = \sum_{k=0}^n {n\choose k}x^kh^{n-k}.$$
	el teorema del binomio. Así tenemos,
	$$\begin{array}{rcl}
	    \dfrac{f(x+h)-f(x)}{h}&=&\dfrac{1}{h}\left[(x+h)^n - x^n\right]\\\\
				  &=&\displaystyle\dfrac{1}{h}\left[\left(\sum_{k=0}^n {n\choose k}x^k h^{n-k}\right)-x^n\right]\\\\
				  &=&\displaystyle\dfrac{1}{h}\left[\sum_{k=0}^{n-1}{n\choose k}x^kh^{n-k}\right]\\\\
				  &=&\displaystyle\sum_{k=0}^{n-1}{n\choose k}x^k h^{n-k-1}\\\\
				  &=&nx^{n-1}+\dfrac{n(n-1)}{2}x^{n-2}+\ldots + h^{n-1}\\\\
	\end{array}$$
	Tomando el límite de $h\to 0$ a ambos lados de la ecuación concluimos que,
	$$f'(x)=\lim_{h\to 0}\dfrac{f(x+h)-f(x)}{h}=nx^{n-1}.$$\\

\end{enumerate}

\end{comment}


\setcounter{section}{8}
\section{Ejercicios}
\begin{enumerate}[\bfseries 1.]

    %--------------------1.
    \item Sea $f(x)=\frac{1}{3}x^3-2x^2+3x+1$ para todo $x$. Hallar los puntos de la gráfica de $f$ en los que la recta tangente es horizontal.\\\\
	Respuesta.-\; Sea  $f'(x)=x^2-4x+3$, entonces para que la recta tangente sea horizontal igualamos la derivada a cero de la siguiente manera,
	$$f'(x)=x^2-4x+3=0$$
	de donde se obtiene que,
	$$x_1=3\qquad \mbox{y}\qquad x_2=1.$$\\

    %--------------------2.
    \item Sea $f(x)=\dfrac{2}{3}x^3+\dfrac{1}{2}x^2-x-1$ para todo $x$. Hallar los puntos de la gráfica de $f$ en los que la pendiente es:\\

	\begin{enumerate}[a)]

	    %---------- a)
	    \item $0$.\\\\
		Respuesta.-\; Sea $f'(x)=2x^2+x-1$, entonces
		$$f'(x)=2x^2+x-1=0\quad \Rightarrow \quad (2x-1)(x+1)=0 \quad \Rightarrow\quad x_1=-1,\quad x_2=\dfrac{1}{2}.$$

	    %---------- b)
	    \item $-1$.\\\\
		Respuesta.-\; Sea $f'(x)=2x^2+x-1$, entonces
		$$f'(x)=2x^2+x-1=-1\quad \Rightarrow \quad x(2x+1) = 0 \quad \Rightarrow \quad x_1=0,\quad x_2=-\dfrac{1}{2}.$$

	    %---------- c)
	    \item $5$.\\\\
		Respuesta.-\; Sea $f'(x)=2x^2+x-1$, entonces
		$$f'(x)=2x^2+x-1=5\quad \Rightarrow \quad 2x^2+x-6=0 \quad \Rightarrow \quad x_1=-2,\quad x_2=\dfrac{3}{2}.$$\\

	    %---------- d)

	\end{enumerate}

    %--------------------3.
    \item Sea $f(x)=x+\sen x$ para todo $x$. Hallar todos los puntos $x$ para los que la gráfica de $f$ en $\left(x,f(x)\right)$ tiene pendiente cero.\\\\
	Respuesta.- Para tal efecto igualamos la derivada de $f(x)$ a $0$.
	$$1+\cos x = 0 \quad \Rightarrow \quad \cos x = -1 \quad \Rightarrow \quad x=(2n+1)\pi,\; n\in \mathbb{Z}.$$\\

    %--------------------4.
    \item Sea $f(x)=x^2+ax+b$ para todo $x$. Hallar los valores de $a$ y $b$ tales que la recta $y=2x$ sea tangente a la gráfica de $f$ en el punto $(2,3)$.\\\\
	Respuesta.-\; Primero, derivamos $f(x)$, como sigue
	$$g'(x)=2x+a.$$
	Así, si la linea $y=2x$ es tangente a $f$ en el punto $(2,4)$, tenemos
	$$f'(2)=2\quad \Rightarrow \quad 4+a=2 \quad \Rightarrow \quad a=-2.$$
	Luego, el punto $(2,4)$ debe estar en la gráfica de $f$, es decir, 
	$$f(2)=4\quad \Rightarrow \quad 4+(-2)2+b=4 \quad \Rightarrow \quad b=4.$$
	Por lo tanto, los valores son $a=-2$ y $b=4$.\\\\

    %--------------------5.
    \item Hallar valores de las constantes $a,b$ y $c$ para los cuales las gráficas de los dos polinomios $f(x)=x^2+ax+b$ y $g(x)=x^3-c$ se intersecten en el punto $(1,2)$ y tengan la misma tangente en dicho punto.\\\\
	Repuesta.-\; Dado que $f$ y $g$ se intersectan en $(1,2)$, podríamos tener $f(1)=g(1)=2$, de donde 
	$$f(1)=2\;\rightarrow \; 1+a+b=2\qquad \mbox{y}\qquad g(1)=2\; \Rightarrow \; 1-c=2 \; \Rightarrow \; c=-1.$$
	Luego, calculamos las derivadas para que podamos encontrar la pendiente de las rectas tangentes en este punto,
	$$f'(x)=2x+a,\qquad g'(x)=3x^2.$$
	Por el hecho de que estos deben ser los mismos en el punto $(1,2)$ tenemos
	$$f'(1)=1\quad \Rightarrow \quad 2+a=3\quad \Rightarrow \quad a=1.$$
	Por lo que $1+a+b=2\; \Rightarrow \; b=0.$ Por lo tanto las constantes son
	$$a=1,\qquad b=0,\qquad c=-1.$$\\

    %--------------------6.
    \item Considérese la gráfica de la función $f$ definida por la ecuación $f(x)=x^2+ax+b$, siendo $a$ y $b$ constantes.

	\begin{enumerate}[(a)]

	    %---------- a)
	    \item Hallar la pendiente de la cuerda que une los puntos de la gráfica para los que $x=x_1$ y $x=x_2$.\\\\
		Respuesta.-\; Los puntos en la gráfica de $f$ en $x_1$ y $x_2$ son $\left(x_1,f(x_1)\right)$ y $\left(x_2,f(x_2)\right)$. Entonces la cuerda que los une tiene una pendiente dada 

	    %---------- b)
	    \item Hallar, en función de $x_1$ y $x_2$, todos los valores de $x$ para los que la tangente en $(x,f(x))$ tiene la misma pendiente que la cuerda de la parte a).\\\\
		Repuesta.-\;

	\end{enumerate}

    %--------------------7.
    \item Demostrar que la recta $y=-x$ es tangente a la curva dada por la ecuación $x^3-6x^2+8x$. Hallar los puntos de tangencia. ¿Vuelve la tangente a cortar la curva?.\\\\
	Demostración.-\;
	    \begin{center}
		\begin{tikzpicture}
		\begin{axis}[scale=.6,draw opacity =.5,samples=100,smooth, 
		  axis x line=center, % no box around the plot, only x and y axis
		  axis y line=center, % the * suppresses the arrow tips
		  ylabel = {$f(x)$},
		  xlabel = {$x$},
		  xlabel style={below right},
		  ylabel style={above left},
		  xmin=-1,xmax=4,ymin=-5,ymax=4,
		  enlargelimits=upper] % extend the axes a bit to the right and top
		  \addplot[black,opacity=1]{x^3-6*x^2+8*x};
		  \addplot[black,opacity=1]{-x};
		\end{axis}
	    \end{tikzpicture}
	    \end{center}
	    \vspace{.5cm}
	    Primero calculemos la derivada de la recta y la curva, respectivamente
	    $$y'=-1,\qquad y_0'=3x^2-12x+8$$
	    Luego igualando estas ecuaciones obtenemos
	    $$y_1=3x^2-12x+8=-1\quad \Rightarrow \quad  x_1 = 3;\quad x_2=1.$$
	    Luego, para que la linea sea tangente a la curva, el punto debe estar en la curva $y_0=x^3-6x^2+8x$ como sigue,
	    \begin{itemize}
		\item Para $x_1=3$, se tiene $y(3)=-3=-x$ por lo que $y=-x$ es tangente a la curva en $(3,-3)$.\\
		\item Para $x_2=1$, se tiene $y(1)=1\neq -x$ por lo que $x_2$ no es tangente a la curva.\\
	    \end{itemize}
	    Esta linea tangente también corta la curva en $(0,0)$.\\\\

    %--------------------8.
    \item Dibujar la gráfica de la función cúbica $f(x)=x-x^3$ en el intervalo cerrado $-2\leq x\leq 2$. Hallar las constantes $m$ y $b$ de modo que la recta $y=mx+b$ sea tangente a la gráfica de $f$ en el punto $(-1,0)$. Una segunda recta que pasa por $(-1,0)$ es también tangente a la gráfica de $f$ en el punto $(a,c)$. Determinar las coordenadas $a$ y $c$.\\\\
	Respuesta.-\; 
	    \begin{center}
		\begin{tikzpicture}
		\begin{axis}[scale=.6,draw opacity =.5,samples=100,smooth, 
		  axis x line=center, % no box around the plot, only x and y axis
		  axis y line=center, % the * suppresses the arrow tips
		  ylabel = {$f(x)$},
		  xlabel = {$x$},
		  xlabel style={below right},
		  ylabel style={above left},
		  xmin=-2,xmax=2,ymin=-7,ymax=7,
		  enlargelimits=upper] % extend the axes a bit to the right and top
		  \addplot[black,opacity=1]{x-x^3};
		\end{axis}
	    \end{tikzpicture}
	    \end{center}
	    \vspace{.5cm}
	    Sea $f'(x)=1-3x^2$, entonce la tangente de la linea en el punto $(-1,0)$ será
	    $$f'(-1)=1-3(-1)^2 = -2 \quad \Rightarrow \quad m=-2.$$
	    de donde $b$ estará dado por,
	    $$y=mx+b\quad \Rightarrow \quad 0=-2(-1)+b \quad \Rightarrow \quad b=-2.$$
	    Por lo tanto 
	    $$y=-2x-2.$$
	    Después, supongamos otra linea tangente $y_1=m_1x+b_1$ a $f$ en el punto $(a,c)$ con pendiente 
	    $$f'(a)=1-3a^2=m_1$$
	    sabiendo que esta recta pasa por $(-1,0)$, entonces 
	    $$y_1(1)=0\quad \Rightarrow \quad (1-3a^2)(-1)+b_1=0\quad \Rightarrow \quad b_1=1-3a^2$$
	    Por lo que la recta $y_1$ es de la forma,
	    $$y_1=(1-3a^2)x+(1-3a^2).$$
	    Por último, dado que el punto $(a,c)$ está tanto en esta linea $y_1$ como en la curva $f$ tenemos
	    $$f(a)=c\quad \Rightarrow \quad a-a^3=c$$
	    $$\mbox{y}\qquad$$ 
	    $$y_1(a)=c\quad \Rightarrow \quad (1-3a^2)a+(1-3a^2)=c.$$
	    Igualando $c$ se tiene,
	    $$a-a^3=(1-3a^2)a+(1-3a^2) \quad \Rightarrow\quad 2a^3+3a^2-1=0 \quad \Rightarrow \quad 2a=\dfrac{1}{2}.$$
	    Ya que $a-a^3=c$ entonces $c=\dfrac{3}{8}.$ Así, el otro punto tangente está dado por $\left(\frac{1}{2},\frac{3}{8}\right).$\\\\

    %--------------------9.
    \item Una función $f$ está definida del modo siguiente:
    $$f(x)=\left\{\begin{array}{rcl}
	    \dfrac{1}{|x|}&\mbox{si}&|x|>c,\\\\
	    a+bx^2 &\mbox{si}&|x|\leq c.\\\\
    \end{array}\right.$$
    Hallar los valores de $a$ y $b$ (en función de $c$) tales que $f'(c)$ exista.\\\\
	Respuesta.-\; Sabemos que la derivada $f'(c)$ existe si y sólo si
	$$\lim_{h\to 0}\dfrac{f(c+h)-f(c)}{h}$$
	Existe. También sabemos que el límite existe si
	$$\lim_{h\to 0^+}\dfrac{f(c+h)-f(c)}{h} = \lim_{h\to 0^-}\dfrac{f(c+h)-f(c)}{h}.$$
	Por lo que si tomamos $c$ y nos acercamos a $ax+b$ desde la derecha, a $x^2$ desde la izquierda y tomando a $f(c)=c^2$, entonces 
	$$\begin{array}{rcl}
	    \lim\limits_{h\to 0^+} \dfrac{f(c+h)-f(c)}{h} &=& \lim\limits_{h\to 0^-} \dfrac{f(c+h)-f(c)}{h} \\\\
	    \lim\limits_{h\to 0^+} \dfrac{a(c+h)+b-c^2}{h} &=& \lim\limits_{h\to 0^-} \dfrac{(c+h)^2-c^2}{h} \\\\
	    \lim\limits_{h\to 0^+} \dfrac{ac+b-c^2}{h} + a &=& 2c.\\\\
	\end{array}$$
	Dado que $ac+b-c^2$ es una constante, entonces $\lim\limits_{h\to 0^+}\dfrac{ac+b-c^2}{h} = 0.$ Por lo que nos queda la ecuación 
	$$a=2c.$$
	Ahora dado que $\lim\limits_{h\to 0^+}\dfrac{ac+b-c^2}{h} = 0$, entonces
	$$ac+b-c^2=0 \quad \Rightarrow \quad b=-c^2.$$\\

    %--------------------10.
    \item Resolver el ejercicio 9 cuando $f$ es:
    $$f(x)=\left\{\begin{array}{lcl}
	    \dfrac{1}{|x|}&\mbox{si}&|x|>c,\\\\
	    a+bx^2 &\mbox{si}&|x|\leq c.\\
    \end{array}\right.$$\\
	Respuesta.-\; Supongamos que $c>0$ de lo contrario $f(x)=\dfrac{1}{|x|}$ para todo $x$. Entonces, existe la derivada para todo los valores de las constantes de $a$ y $b$. Luego sabemos que si una función tiene derivada en un punto $x$, entonces también es continua en $x$, por lo tanto la función $f$ es continua en $c$, así,\\
	$$\begin{array}{rcl}
	    \lim\limits_{x\to c^+}f(x)&=&\lim\limits_{x\to c^-}f(x)\\\\
	    \lim\limits_{x\to c^+}\dfrac{1}{|x|}&=&\lim\limits_{x\to c^-}a+bx^2\\\\
	    \dfrac{1}{c}&=&a+bc^2\\\\
			ac+bc^3&=&1\\\\
	\end{array}$$
	por otro lado,
	$$\begin{array}{rcl}
	    \lim\limits_{x\to 0^+}f(x)&=&\lim\limits_{x\to 0^-}f(x)\\\\
	    \lim\limits_{x\to 0^+}\dfrac{\frac{1}{|c+h|}-a+bc^2}{h}&=&\lim\limits_{x\to 0^-}\dfrac{a+b(c+h)^2-(a+bc^2)}{h}\\\\
	    \lim\limits_{x\to 0^+}\dfrac{\frac{1}{c+h}-a+bc^2}{h}\cdot \dfrac{c+h}{c+h}&=&\lim\limits_{x\to 0^-}\dfrac{2bch+bh^2}{h}\\\\
	    \lim\limits_{x\to 0^+}\dfrac{1-ac-bc^3-ah-bc^2h}{h(c+h)}&=&\lim\limits_{x\to 0^-}\dfrac{h(2bc+bh)}{h}\\\\
	    \lim\limits_{x\to 0^+}\dfrac{ac+bx^3-ac-bc^3-ah-bc^2h}{h}&=&\lim\limits_{x\to 0^-}2bc+bh\\\\
	    \lim\limits_{x\to 0^+}\dfrac{-h(a+bc^2)}{h(c+h)}&=&2bc+\lim\limits_{x\to 0^-}bh\\\\
	    \lim\limits_{x\to 0^+}-\dfrac{a+bc^2}{c+h}&=&2bc.\\\\
	    -\dfrac{a+bc^2}{c}&=&2bc.\\
	\end{array}$$
	Por último resolvemos las dos ecuaciones anteriores:
	$$\left\{\begin{array}{rcl}
		-\dfrac{a+bc^2}{c}&=&2bc\\\\
			   ac+bc^3&=&1\\
	\end{array}\right.
	\quad \Rightarrow \quad 
	\left\{\begin{array}{rcl}
		b &=& -\dfrac{1}{2c^3}\\\\
		a &=& -\dfrac{3}{2c}\\
	    \end{array}\right.$$\\\\

    %--------------------11.
    \item Resolver el ejercicio 9 cuando $f$ es:
    $$f(x)=\left\{\begin{array}{lcl}
	\sen x &\mbox{si}&x\leq c\\\\
	ax+b &\mbox{si}&x>c.\\
    \end{array}\right.$$\\\\
	Respuesta.-\; Sabemos que $f'(c)$ existe si y sólo si 
	$$\lim_{h\to 0}\dfrac{f(c+h)-f(c)}{h}$$
	Existe. También sabemos que el límite existe si y sólo si los dos límites unilaterales existen y son iguales, es decir
	$$\lim_{h\to 0^+}\dfrac{f(c+h)-f(c)}{h}=\lim_{h\to 0^-}\dfrac{f(c+h)-f(c)}{h}$$
	Por lo tanto usando la definición de $f$ tenemos,
	$$\begin{array}{rcl}
	    \lim\limits_{h\to 0^+} \dfrac{a(c+h)+b-\sen c}{h} &=& \lim\limits_{h\to 0^-} \dfrac{\sen (c+h)-\sen c}{h} \\\\
	    \lim\limits_{h\to 0^+} \left(\dfrac{ac + b-\sen c}{h}\right)+a &=& \lim\limits_{h\to 0^-} \dfrac{\sen c \cos h + \sen h \cos c - \sen c}{h}\\\\
	    \lim\limits_{h\to 0^+} \left(\dfrac{ac + b-\sen c}{h}\right)+a &=& \lim\limits_{h\to 0^-} \left[\dfrac{\sen c ( \cos h -1)}{h}\right] + \cos c \\\\
	\end{array}$$

	Al simplificar el lado derecho usamos el hecho de que $\lim\limits_{h\to 0}\dfrac{\sen h}{h}=1$. Además para que existe el límite por el lado izquierdo debemos tener $ac+b-\sen c = 0$ de lo contrario el límite divergirá como $h\to 0$. Ahora, para la expresión de la derecha, vemos que el límite tiende a $0$. Eso se puede ver ya que 
	$$\lim_{h\to 0^-}\dfrac{\sen c(\cos h - 1)}{h}=\sen c \lim_{h\to c^-}\dfrac{\cos h - 1}{h} = \sen c \lim_{h\to c^-}\dfrac{\cos(0+h)-\cos 0}{h}$$
	Pero este límite es la derivada de $\cos x$ en $x=0$. Luego ya que $(\cos x)'=-\sen x$ y $\sen 0 =0$ el termino tiende a $0$. Por lo tanto
	$$\lim\limits_{h\to 0^+} \left(\dfrac{ac + b-\sen c}{h}\right)+a = \lim\limits_{h\to 0^-} \left[\dfrac{\sen c ( \cos h -1)}{h}\right] + \cos c \quad \Rightarrow \quad a=\cos c.$$
	Así, dado que $ac+b-\sen c=0$ entonces 
	$$b=\sen c - c\cos c.$$\\

    %--------------------12.
    \item Si $f(x)=\dfrac{1-\sqrt{x}}{1+\sqrt{x}}$ para $x>0$, hallar fórmulas para $Df(x),\; D^2f(x)$ y $D^3f(x)$.\\\\
	Respuesta.- \; La fórmula para $Df(x)$ es:
	$$Df(x)=\dfrac{(1+\sqrt{x})\cdot \dfrac{-1}{2\sqrt{x}}- (1-\sqrt{x})\cdot \dfrac{1}{2\sqrt{x}}}{(1+\sqrt{x})^2} = \dfrac{\dfrac{1}{2}-\dfrac{1}{2}-\dfrac{1}{2\sqrt{x}}-\dfrac{1}{2\sqrt{x}}}{(1+\sqrt{x})^2}=\dfrac{-\dfrac{2}{2\sqrt{x}}}{(1+\sqrt{x})^2} =\dfrac{-1}{\sqrt{x}(1+\sqrt{x})^2}.$$

	Para $D^2f(x)$,

	$$D^2f(x)=\dfrac{\dfrac{1}{2\sqrt{x}}(1+\sqrt{x})^2+2\sqrt{x}(1+\sqrt{x})\dfrac{1}{2\sqrt{x}}}{x(1+\sqrt{x})^4} = \dfrac{1+3\sqrt{x}}{2(x+\sqrt{x})^3}.$$

	Por último para $D^3f(x)$,

	$$D^3f(x)=\dfrac{\dfrac{3}{2\sqrt{x}}\cdot 2(x+\sqrt{x})^3-6(1+3\sqrt{x})(x+\sqrt{x})^2\left(1+\dfrac{1}{2\sqrt{x}}\right)}{4(x+\sqrt{x})^6}=- \dfrac{3(1+4\sqrt{x}+5x)}{4\left[\sqrt{x}(x+\sqrt{x})^4\right]}.$$\\

    %--------------------13.
    \item 


\end{enumerate}


    %---------- Relación entre integración y derivación
	%\input{codigoFuente/tom_apostol/5-RelacionEntreIntegracionDerivacion.tex}



    %---------- álgebra vectorial
	%\chapter{Álgebra vectorial}


\setcounter{section}{1}
\section{El espacio vectorial de las n-plas de números reales}

%--------------------definición 12.1.
\begin{tcolorbox}
    \begin{def.} Dos vectores $A$ y $B$ de $V_n$ son iguales siempre que coinciden sus componentes. Esto es, si $A=(a_1,a_2,...,a_n)$ y $B=(b_1,b_2,...,b_3)$, la ecuación vectorial $A=B$ tiene exactamente el mismo significado que las $n$ ecuaciones escalares $$a_1=b_1, \qquad a_2=b_2, \qquad a_n=b_n$$
    La suma $A+B$ se define como el vector obtenido sumando los componentes correspondientes: $$A+B = (a_1+b_1,a_2+b_2,...,a_n+b_n)$$
    La $c$ es un escalar, definimos $cA$ o $Ac$ como el vector obtenido multiplicando cada componente de $A$ por $c$: $$cA=(ca_1,ca_2,...,ca_n)$$
    \end{def.}
\end{tcolorbox}

%--------------------teorema 12.1
\begin{teo}

\begin{enumerate}[\bfseries a.]
    \item La adición de vectores es conmutativa. $$A+B=B+A$$\\
	Demostración.-\; Sea $V_n$ el espacio vectorial n-plas y $A=(a_1,a_2,...,a_n)$ y $B=(b_1,b_2,...b_n)$, por lo tanto por definición de adición  y propiedad de números reales, tenemos $$A+B = (a_1+b_1,a_2+b_2,...,a_n+b_n) = (b_1+a_1,b_2+a_2,...,b_n+a_n) = B+A$$.\\

    \item y asociativa, $$A+(B+C) = (A+B)+C$$\\
	Demostración.-\; Sea $V_n$ el espacio vectorial n-plas y $A = (a_1,a_2,...,a_n)$, $B=(b_1,b_2,...,b_n)$ y $C=(c_1,c_2,...,c_n)$ entonces $$A+(B+C) = A + (b_1+c_1,b_2+c_2,...,b_n+c_n) = (a_1+(b_1+c_1),a_2+(b_2+c_2),...,a_1+(b_n+c_n)) = $$
	$$((a_1+b_1)+c_1,(a_2+b_2)+c_2,...,(a_n+b_n)+c_n) = (a_1+b_1,a_2+b_2,...,b_n+c_n)+C = (A+B)+C$$\\

    \item La multiplicación por escalares es asociativa $$c(dA)=(cd)A$$\\
	Demostración.-\; Sea $c,d \in \mathbb{R}$ y $A\in V_n$ entonces 
	\begin{center}
	    \begin{tabular}{rcl}
		$c(dA)$&$=$&$c(da_1,da_2,...,da_n)$\\
		&$=$&$((cd)a_1,(cd)a_2,...,(cd)a_3)$\\
		&$=$&$(cd)A$\\\\
	    \end{tabular}
	\end{center}

    \item y satisface las dos leyes distributivas $$c(A+B)=cA+cB,\quad y \quad (c+d)A=cA+dA$$\\
	Demostración.-\; Las demostraciones son fáciles de realizar siempre y cuando se tomen en cuenta Las definiciones de 12.1.\\\\

    \item El vector con todos los componentes $0$ se llama vector cero y se representa con $O$. Tiene la propiedad.\\\\
	Demostración.-\; Existencia. Sea $O = (o,o,...,o)$ de donde $A+O = (a_1,a_2,...,a_n)+(o,o,...,o) = (a_1+o,a_2+o,...,a_n+o) = (a_1,a_2,...,a_n) = A$.\\
	Unicidad. Supongamos que $O,O^{'} \in V_n; O\neq O$ tal que 
	$$\left\{ \begin{array}{ccc} A+O=A&tomando \; A=O^{'}:&O^{'}+O = O^{'}\\ A+O^{'} = A&tomando \; A=O:&O+O^{'}=O\\ \end{array} \right.$$
	Por lo tanto $O=O^{'}$.\\\\
    \item El vector $(-1)A$ que también se representa con $-A$ se llama el apuesto a $A$. También escribimos $A-B$ en lugar de $A+(-B)$ y lo llamamos diferencia de $A$ y $B$. La ecuación $(A+B)-B=A$. Demuestra que la sustracción es la operación inversa de la adición. Obsérvese que $0A=O$ y que $1A=A$.\\\\

\end{enumerate}
    
\end{teo}

\section{Interpretación geométrica para $n\leq 3$}

%--------------------definición 12.2
\begin{tcolorbox}

    \begin{def.} Dos vectores $A$ y $B$ de $V_n$ tienen la misma dirección si $B=cA$ para un cierto escalar positivo $c$, y la dirección opuesta si $B=cA$ para un cierto $c$ negativo. Se llaman paralelos si $B=cA$ para un cierto $c$ no nulo.
    \end{def.}
\end{tcolorbox}

\section{Ejercicios}
\begin{enumerate}[\bfseries 1.]

%--------------------1.
\item Sean $A=(1,3,6)$, $B=(4,-3,3)$ y $C=(2,1,5)$ tres vectores de $V_3$. Determinar los componentes de cada uno de los vectores: 
\begin{enumerate}[\bfseries a)]

    %----------a)
    \item $A+B = (1,3,6) + (4,-3,3) = (1+4,3+(-3),6+3) = (5,0,9)$\\\\

    %----------b)
    \item $A-B = (1,3,6) - (4,-3,3) = (1-4,3-(-3),6-3) = (-3,6,3)$\\\\

    %----------c)
    \item $A+B-C = (1,3,6)+(4,-3,3)-(2,1,5) = (1+4-2,3-3-1,6+3-5) = (3,-1,4)$\\\\

    %----------d)
    \item $7A-2B-3C = 7(1,3,6)-2(4,-3,3)-3(2,1,5) = (7,21,42)-(8,-6,6)-(6,3,15) =$\\ $=(7-8-6,21-8-(-6)-3,42-6-15) = (-7,24,21)$\\\\

    %----------e)
    \item $2A+B-3C = 2(1,3,6) + (4,-3,3) 3(2,1,5) = (2+4-6,6-3-3,12+3-15) = (0,0,0)$\\\\

\end{enumerate}

%--------------------2.
\item Dibujar los vectores geométricos que unen el origen a los puntos $A=(2,1)$ y $B=(1,3).$ En la misma figura trazar el vector geométrico que une el origen al punto $C=A+t(B)$ para cada uno de los siguientes valores de $t$: $t=\frac{1}{2}$; $t=\frac{3}{4}$; $t=1;$ $t=2$; $t=-1$; $t=-2$.\\\\
    Respuesta.-\: 
    \begin{center}
	\begin{tabular}{clcl}
	    Si&$t=\frac{1}{3}$&$\Longrightarrow$&$C=(\frac{7}{3},2)$\\\\
	    Si & $t=\frac{1}{2}$ & $\Longrightarrow$ & $C=(\frac{5}{2},\frac{5}{2})$ \\\\
	    Si & $t=1$ & $\Longrightarrow$ & $C=(\frac{11}{4},\frac{13}{4})$ \\\\
	    Si & $t=2$ & $\Longrightarrow$ & $C=(3,4)$ \\\\
	    Si & $t=-1$ & $\Longrightarrow$ & $C=(4,7)$ \\\\
	    Si & $t=-2$ & $\Longrightarrow$ & $C=(1,-2)$ \\\\
	    Si & $t=\frac{3}{4}$ & $\Longrightarrow$ & $C=(0,-5)$ \\\\
	\end{tabular}
    \end{center}
    \begin{center}
	\begin{tikzpicture}[scale=.7]
	    % abscisa y ordenada
	    \tkzInit[xmax= 3,xmin=0,ymax=6,ymin=0]
	    \tiny\tkzLabelXY[opacity=0.6,step=1, orig=false]
	    % label x, f(x)
	    \tkzDrawX[opacity=0.6,label=x,right=0.3]
	    \tkzDrawY[opacity=0.6,label=f(x),below = -0.6]
	    \draw[red, ->](0,0)--(2,1);
	    \draw[red, ->](0,0)--(1,3);
	    \draw[blue, ->](0,0)--(7/3,2);
	    \draw[blue, ->](0,0)--(5/2,5/2);
	    \draw[blue, ->](0,0)--(11/4,13/4);
	    \draw[blue, ->](0,0)--(3,4);
	    \draw[blue, ->](0,0)--(4,7);
	    \draw[blue, ->](0,0)--(1,-2);
	    \draw[blue, ->](0,0)--(0,-5);
	\end{tikzpicture}
    \end{center}

%--------------------3.
\item resolver el ejercicio 2 si $C=tA+B$.\\\\
    Respuesta.-\; 
    \begin{center}
	\begin{tabular}{clcl}
	    Si&$t=\frac{1}{3}$&$\Longrightarrow$&$C=(\frac{5}{3},\frac{10}{3})$\\\\
	    Si & $t=\frac{1}{2}$ & $\Longrightarrow$ & $C=(2,\frac{7}{2})$ \\\\
	    Si & $t=1$ & $\Longrightarrow$ & $C=(\frac{5}{2},\frac{15}{4})$ \\\\
	    Si & $t=2$ & $\Longrightarrow$ & $C=(3,4)$ \\\\
	    Si & $t=-1$ & $\Longrightarrow$ & $C=(5,5)$ \\\\
	    Si & $t=-2$ & $\Longrightarrow$ & $C=(-1,2)$ \\\\
	    Si & $t=\frac{3}{4}$ & $\Longrightarrow$ & $C=(-3,1)$ \\\\
	\end{tabular}
    \end{center}
    \begin{center}
	\begin{tikzpicture}[scale=.9]
	    % abscisa y ordenada
	    \tkzInit[xmax= 3,xmin=-3,ymax=3,ymin=0]
	    \tiny\tkzLabelXY[opacity=0.6,step=1, orig=false]
	    % label x, f(x)
	    \tkzDrawX[opacity=0.6,label=x,right=0.3]
	    \tkzDrawY[opacity=0.6,label=f(x),below = -0.6]
	    \draw[red, ->](0,0)--(2,1);
	    \draw[red, ->](0,0)--(1,3);
	    \draw[blue, ->](0,0)--(5/3,10/3);
	    \draw[blue, ->](0,0)--(2,7/2);
	    \draw[blue, ->](0,0)--(5/2,15/4);
	    \draw[blue, ->](0,0)--(3,4);
	    \draw[blue, ->](0,0)--(5,5);
	    \draw[blue, ->](0,0)--(-1,2);
	    \draw[blue, ->](0,0)--(-3,1);
	\end{tikzpicture}
    \end{center}

%--------------------4.
\item Sean $A=(2,1)$, $B=(1,3)$ y $C=xA+yB$, en donde $x$ e $y$ son escalares.
\begin{enumerate}[\bfseries a)]
    
    %----------a)
    \item Trazar el vector que une el origen a $C$ para cada uno de los siguientes pares de valores de $x$ e $y$: $x=y=\frac{1}{2}$; $x=\frac{1}{4},\;y = \frac{3}{4}$; $x=\frac{1}{3}, \; y=\frac{2}{3}$; $x=2,\; y=-1$; $x=3,\; y=-2$; $x=-\frac{1}{2},\; y=\frac{3}{2}$; $x=-1,\; y=2$.\\\\
	Respuesta.-\; 
	\begin{center}
	    \begin{tabular}{clcl}
		Si&$x=y=\frac{1}{2}$&$\Longrightarrow$&$C=\left(\frac{3}{2},2\right)$\\\\
		Si & $x=\frac{1}{4},\;y = \frac{3}{4}$ & $\Longrightarrow$ & $C=\left(\frac{5}{4},\frac{5}{2}\right)$ \\\\
		Si & $x=\frac{1}{3}, \; y=\frac{2}{3}$ & $\Longrightarrow$ & $C=\left(\frac{4}{3},\frac{7}{3}\right)$ \\\\
		Si & $x=2,\; y=-1$ & $\Longrightarrow$ & $C=(3,-1)$ \\\\
		Si & $x=3,\; y=-2$ & $\Longrightarrow$ & $C=(4,-3)$ \\\\
		Si & $x=-\frac{1}{2},\; y=\frac{3}{2}$ & $\Longrightarrow$ & $C=\left(\frac{1}{2},4\right)$ \\\\
		Si & $x=-1,\; y=2$ & $\Longrightarrow$ & $C=(0,5)$ \\\\
	    \end{tabular}
	\end{center}
	\begin{center}
	    \begin{tikzpicture}[scale=.9]
		% abscisa y ordenada
		\tkzInit[xmax= 4,xmin=-3,ymax=4,ymin=-3]
		\tiny\tkzLabelXY[opacity=0.6,step=1, orig=false]
		% label x, f(x)
		\tkzDrawX[opacity=0.6,label=x,right=0.3]
		\tkzDrawY[opacity=0.6,label=f(x),below = -0.6]
		\draw[red, ->](0,0)--(2,1);
		\draw[red, ->](0,0)--(1,3);
		\draw[blue, ->](0,0)--(3/2,2);
		\draw[blue, ->](0,0)--(5/4,5/2);
		\draw[blue, ->](0,0)--(4/3,7/3);
		\draw[blue, ->](0,0)--(3,-1);
		\draw[blue, ->](0,0)--(4,-3);
		\draw[blue, ->](0,0)--(1/2,4);
		\draw[blue, ->](0,0)--(0,5);
		\draw[black](-1,7)--(4,-3);
	    \end{tikzpicture}
	\end{center}
    
    %----------b)
    \item ¿Qué conjunto es el de los puntos $C$ obtenidos cuando $x$ e $y$ toman todos los valores reales tales que $x+y=1$? (Hacer una conjetura y mostrar el lugar geométrico en la figura. No hacer la demostración).\\\\
	Respuesta.-\; Sea $x=3,\; y=-2$  y $x=-2,\; y=3$ podemos graficar el conjunto de los puntos $C$ tales que $x+y=1$.\\\\ 
    
    %----------c)
    \item Dar una idea del conjunto de todos los puntos $C$ obtenidos al variar independientemente $x$ e $y$ en los intervalos $0\leq x\leq 1, \; 0\leq y \leq 1,$ y hacer una representación de ese conjunto.\\\\
	Respuesta.-\; 
	\begin{center}
	    \begin{tikzpicture}[scale=.9]
		% abscisa y ordenada
		\tkzInit[xmax= 4,xmin=-3,ymax=4,ymin=-3]
		\tiny\tkzLabelXY[opacity=0.6,step=1, orig=false]
		% label x, f(x)
		\tkzDrawX[opacity=0.6,label=x,right=0.3]
		\tkzDrawY[opacity=0.6,label=f(x),below = -0.6]
		\draw[->](0,0)--(2,1);
		\draw[->](2,1)--(3,4);
		\draw[->](0,0)--(3,4);
	    \end{tikzpicture}
	\end{center}
    
    %----------d)
    \item ¿Qué conjunto es el de todos los puntos $C$ obtenidos si $x$ varía en el intervalo $0\leq x \leq 1$ e $y$ recorre todos los números reales?.\\\\
	Respuesta.-\; La banda horizontal obtenida sumando $xA$ a la línea $y = \frac{1}{3}x$ para cada uno $0 \leq x \leq 1$.\\\\

    %----------e)
    \item ¿Qué conjunto resulta si $x$ e $y$ recorren ambos todos los números reales?.\\\\
	Respuesta.-\; Todo $R^2$.\\\\

\end{enumerate}

%--------------------5.
\item Sean $A=(2,1)$ y $B=(1,3)$. Demostrar que todo vector $C=(c_1,c_2)$ de $V_2$ puede expresarse en la forma $C = xA+yB$. Expresar $x$ e $y$ en función de $c_1$ y $c_2$.\\\\
    Demostración.-\; Ya que $$C=xA+yB = (2x+y,x+3y)  = (c_1,c_2)$$
    se tiene $c_1 = 2x+y\;$ y $\;c_2 = x+3y$ de donde $$y=\dfrac{1}{5}(2c_2-c_1)$$  $$\;x=\dfrac{1}{5}(3c_1-c_2)$$
    Esto demuestra que cualquier vector en $\mathbb{R}^2$ se puede obtener como $xA+yB$ dado $C=(c_1,c_2)$ que calculamos.\\\\

%--------------------6.
\item Sea $A=(1,1,1)$, $B=(0,1,1)$ y $C=(1,1,0)$ tres vectores de $V_3$ y $D=xA+yB+zC$, donde $x$, $y$ $z$ son escalares.
\begin{enumerate}[\bfseries a)]

    %----------a)
    \item Determinar los componentes de $D$.\\\\
	Respuesta.-\; Tenemos que $D=x(1,1,1)+y(0,1,1)+z(1,1,0)$ de donde $D=(x,x,x)+(0,y,y)+(z,z,0)$
	así, $$D=(x+z,x+y+z,x+y)$$\\

    %----------b)
    \item Si $D=0$ demostrar que $x=y=z=0$.\\\\
	Demostración.-\; Sea $D=0=(0,0,0)$ entonces 
	\begin{center}
	    \begin{tabular}{rcl}
		$x+z=0$&$\Longrightarrow$&$x=-z$\\	
		$x+y+z=0$&$\Longrightarrow$&$y=0$\\
		$x+y=0$&$\Longrightarrow$&$x=-y$\\
	    \end{tabular}
	\end{center}
	de donde concluimos que $x=y=z=0$.\\\\

    %----------c)
    \item Hallar $x, y, z$ tales que $D=(1,2,3)$.\\\\
	Respuesta.-\; 
	\begin{center}
	    \begin{tabular}{rcll}
		$x+z=1$&$\Longrightarrow$&$z=1-x$&$\Longrightarrow z=-1$\\
		$x+y+z=2$&$\Longrightarrow$&$x+y+1-x=2$&$\Longrightarrow y=1$\\
		$x+y=3$&$\Longrightarrow$&$x+1=3$&$\Longrightarrow x=2$\\\\
	    \end{tabular}
	\end{center}

\end{enumerate}

%--------------------7.
\item Sean $A=(1,1,1)$, $B=(0,1,1)$ y $C=(2,1,1)$ tres vectores de $V_3$ y $D=xA+yB+zC$, en donde $x,y,z$ son escalares.
\begin{enumerate}[\bfseries a)]

    %----------a)
    \item Determinar los componentes de $D$.\\\\
	Respuesta.-\; Sea $D=x(1,1,1)+y(0,1,1)+z(2,1,1)$ entonces $D=(x+2z,x+y+z,x+y+z)$.\\\\


    %----------b)
    \item Hallar $x,y,z$ no todos nulos, tales que $D=0$.\\\\
	Respuesta.-\; Sea $x=2$, $y=-1$ y $z=-1$, entonces $$D=(x+2z,x+y+z,x+y+z)=(2-2,2-1-1,2-1-1)=(0,0,0)=O$$ \\

    %----------c)
    \item Demostrar que ninguna elección de $x,y,z$ hace $D=(1,2,3)$.\\\\
	Demostración.-\; Sea 
	\begin{center}
	    \begin{tabular}{rcll}
		$x+2z=1$&$\Longrightarrow$&$x=1-2z$&\\
		$x+y+z=2$&$\Longrightarrow$&$1-2z+y+z=2$&$\Longrightarrow y=z+1$\\
		$x+y+z=3$&$\Longrightarrow$&$1-2z+z+1+z=3$&$\Longrightarrow 2=3$\\
	    \end{tabular}
	\end{center}
	de donde encontramos un absurdo al declarar que $2=3$, por lo tanto no existe ninguna elección que satisfaga a $D=(1,2,3)$.\\\\

\end{enumerate}

%--------------------8.
\item Sean $A=(1,1,1,0)$, $B=(0,1,1,1)$, $C=(1,1,0,0)$ tres vectores de $V_4$ y $D=xA+yB+zC$ siendo $x,y,z$ escalares.
\begin{enumerate}[\bfseries a)]

    %----------a)
    \item Determinar los componentes de $D$.\\\\
	Respuesta.-\; Se tiene $D=x(1,1,1,0)+y(0,1,1,1)+z(1,1,0,0)$ entonces $D=(x+z,x+y+z,x+y,y)$\\\\

    %----------b)
    \item Si $D=0$, Demostrar que $x=y=z=0$\\\\
	Respuesta.-\; 
	\begin{center}
	\begin{tabular}{rcl}
	    $x+z=0$&&\\
	    $x+y+z=0$&$\Longrightarrow$&$z=0$\\
	    $x+y=0$&$\Longrightarrow$&$x=0$\\
	    $y=0$&&\\
	\end{tabular}
	\end{center}
	Por lo tanto $x=y=z=0$.\\\\

    %----------c)
    \item Hallar $x,y,z$ tales que $D=(1,5,3,4)$.\\\\
	Respuesta.-\;
	\begin{center}
	\begin{tabular}{rcl}
	 $x+z=1$&&\\
	 $x+y+z=5$&$\Longrightarrow$&$z=2$\\
	 $x+y=3$&$\Longrightarrow$&$x=-1$\\
	 $y=4$&&\\
	\end{tabular}
	\end{center}
	Por lo tanto $x=-1$, $y=4$ y $z=2$.\\\\

    %----------d)
    \item Demostrar que ninguna elección de $x,y,z$ hace $D=(1,2,3,4)$.\\\\
	Demostración.-\; La demostración es similar al problema 7c.\\\\

\end{enumerate}

%--------------------9.
\item En $V_n$ demostrar que dos vectores paralelos a un mismo vector son paralelos entre sí.\\\\
    Demostración.-\; Por definición de vectores paralelos se tiene $c_1A=C$ y $c_2B=C$ de donde $c_1A=c_2B$, en vista de que $c_1\cdot c_2 \neq 0$ entonces $B=c_1c_2^{-1} A$, por lo tanto concluimos que $A$ y $B$ son paralelos entre sí.\\\\ 

%--------------------10.
\item Dados cuatro vectores no nulos $A,B,C,D$ de $V_n$ tales que $C=A+B$ y $A$ paralelo a $D$. Demostrar que $C$ es paralelo a $D$ si y sólo si $B$ es paralelo a $D$.\\\\
    Demostración.\;  Sea $B=A-C$ de donde $B=cD-c_1D$ así, $B=(c-c_1)D$, (sabemos que $c-c_1\neq 0$, ya que $B$ es distinto de $0$).\\
    Por el contrario supongamos que $B$ es paralelo a $D$. Dado que ambos $A$ y $B$ son paralelos entre sí, esto por el problema anterior. Entonces, tenemos $A=xB$ siendo $x$ un escalar distinto de $0$. Esto implica $$C=C=xB+B \Longrightarrow C = (1-x)B.$$ Por lo tanto si $C$ es paralelo a $B$ y $B$ paralelo a $D$ entonces $C$ es paralelo a $D$.\\\\ 

%--------------------11.
\item 
\begin{enumerate}[\bfseries a)]
    
    %----------a)
    \item Demostrar, para los vectores $V_n$ las propiedades de la adición y de la multiplicación por escalares dadas en el teorema 12.1\\\\
	Demostración.-\; Sea $A,B \in V_n$ donde $A=(a_1,a_2,...,a_n)$ y $B=(b_1,b_2,...,b_n)$ entonces para la conmutatividad de adición tenemos $$A+B=(a_1,a_2,...,a_n)+(b_1,b_2,...,b_n) = (a_1+b_1,a_2+b_2,...,a_n+b_n) = (b_1+a_1,b_2+a_1,...,b_n+a_n) = B+A$$
	Para la asociatividad de adición se tiene,
	\begin{center}
	    \begin{tabular}{rcl}
		$A+(B+C)$&$=$&$(a_1,a_2,...,a_n)+\left[(b_1,b_2,...,b_n)+(c_1,c_2,...,c_n)\right]$\\
		&$=$&$\left[a_1+(b_1+c_1),a_2+(b_2+b_3),...,a_n+(b_n+c_n)\right]$\\
		&$=$&$\left[(a_1+b_1)+c_1,(a_2+b_2)+c_3,...,(a_n+b_n)+c_n\right]$\\
		&$=$&$(A+B)+C$\\
	    \end{tabular}
	\end{center}
	Luego para la asociatividad de la multiplicación escalar tenemos,
	\begin{center}
	    \begin{tabular}{rcl}
		$c(dA)$&$=$&$c\left[d(a_1,a_2,...,a_n)\right]$\\
		&$=$&$c(da_1,da_2,...,da_n)$\\
		&$=$&$\left[(cd)a_1,(cd)a_2,...,(cd)a_n\right]$\\
		&$=$&$(cd)(a_1,a_2,...,a_n)$\\
		&$=$&$(cd)A$\\
	    \end{tabular}
	\end{center}
	Para la primera ley distributiva se tiene,
	\begin{center}
	    \begin{tabular}{rcl}
		$c(A+B)$&$=$&$c\left[(a_1,a_2,...,a_n)+(b_1,b_2,...,b_n)\right]$\\
		&$=$&$\left[c(a_1+b_1),c(a_2+b_2),...,c(a_n+b_n)\right]$\\
		&$=$&$(ca_1+c_b1,ca_2+cb_2,...,ca_n+cb_n)$\\
		&$=$&$(c_1,ca_2,...,ca_n)+(cb_1,cb_2,...,cb_n)$\\
		&$=$&$c(a_1,a_2,...,a_n)+c(b_1,b_2,...,b_n)$\\
		&$=$&$cA+cB$\\
	    \end{tabular}
	\end{center}
	Para la segunda ley distributiva obtenemos,
	\begin{center}
	    \begin{tabular}{rcl}
		$(c+d)A$&$=$&$(c+d)(a_1,a_2,...,a_n)$\\
		&$=$&$(ca_1+cb_1,ca_2+cb_2,...,ca_n+da_n)$\\
		&$=$&$(c_1,ca_2,...,ca_n)+(cb_1,cb_2,...,cb_n)$\\
		&$=$&$c(a_1,a_2,...,a_n)+d(a_1,a_2,...,a_n)$\\
		&$=$&$cA+dA$\\\\
	    \end{tabular}
	\end{center}

    %----------b)
    \item Mediante vectores geométricos en el plano, representar el significado geométrico de las dos leyes distributivas $(c+d)A=cA+dA$ y $c(A+B) = cA+cB$.\\\\
	Respuesta.-\; La ley distributiva $(c + d) A = cA + dA$ significa que el vector $(c + d)$ Ase obtiene sumando la flecha $dA$ al final de la flecha $cA$.\\
La ley distributiva $c (A + B) = cA + cB$ significa que el vector $c (A + B)$ es el vértice del paralelogramo formado por $cA$ y $cB$.\\\\

\end{enumerate}

%--------------------12.
\item Si un cuadrilátero $OABC$ de $V_2$ es un paralelogramo que tiene $A$ y $C$ como vértices opuestos, demostrar que $A+\frac{1}{2}(C-A) =\frac{1}{2}B$. ¿Qué teorema relativo a los paralelogramos puede deducirse de esta igualdad?.\\\\
    Demostración.-\; Dado que este es un paralelogramo, tenemos $B=A+C$. Y por tanto,
    $$B=A+C\; \Longrightarrow\; \dfrac{1}{2}B=\dfrac{1}{2}(A+C)\; \Longrightarrow\; \dfrac{1}{2}B = A + \dfrac{1}{2}C - \dfrac{1}{2} A \;\Longrightarrow \; \dfrac{1}{2}B = A + \dfrac{1}{2}(C-A)$$\\

\end{enumerate}


\section{Producto escalar}

%--------------------definición 12.3
\begin{tcolorbox}
    \begin{def.}[Producto escalar ó interior de dos vectores] Si $A=(a_1,a_2,...,a_n)$ y $B=(b_1,b_2,...,b_n)$ son dos vectores de $V_n$ su producto escalar se representa con $A\cdot B$ y se define con la igualdad $$A\cdot B = \sum_{k=1}^n a_k\cdot b_k$$
    \end{def.}
\end{tcolorbox}

%--------------------teorema 12.2
\begin{teo} Para todos los vectores $A,B,C$ de $V_n$ y todos los escalares $c$ se tienen las propiedades siguientes:
\begin{center}
\begin{tabular}{cll}
    
    %----------(a)
     \textbf{(a)}&$A\cdot B = B\cdot A$& (ley conmutativa).\\

    %----------(b)
    \textbf{(b)} & $A\cdot(B+C) = A\cdot B + A\cdot C$ & (ley distributiva) \\

    %----------(c)
    \textbf{(c)} & $c(A\cdot B) = (cA)\cdot B = A\cdot (cB)$ & (homogeneidad) \\

    %----------(d)
    \textbf{(d)} & $A\cdot A > 0$ si $A\neq O$ & (positivadad) \\

    %----------(e)
    \textbf{(e)} & $A\cdot A = 0$ si $A=O$ & \\\\

\end{tabular}    
\end{center}
    Demostración.-\; Comencemos demostrando el inciso $(a)$ que es un consecuencia de la definición. $$A\cdot B = \sum_{k=1}^n a_kb_k = (a_1\cdot b_1) + (a_2\cdot b_2) + ... + (a_n\cdot b_n) = (b_n+a_n)+(b_2+a_n)+...+(b_n+a_n) = \sum_{k=1}^n b_k a_k = B\cdot A$$ 
    Las demostraciones $(b)$ y $(c)$ se demuestran con la definición de producto escalar, la definición de vectores y el teorema 12.1.\\
    Para demostrar las dos últimas, usamos la relación $A\cdot A = \sum a_k^2$. Puesto que cada térmi es no negativo, la suma es es no negativa. Además, la suma es cero si y sólo si cada término de la suma es cero y esto tan sólo puede ocurrir si $A=O$.\\\\

\end{teo}

%--------------------teorema 12.3
\begin{teo}[Desigualdad de Cauchy-Schwarz] Si $A$ y $B$ son vectores de $V_n$, tenemos $$(A\cdot B)^2 \leq (A\cdot A)(B\cdot B) \qquad \mbox{12.2}$$.
Además, el signo de desigualdad es el válido si y sólo si uno de los vectores es el producto de otro por un escalar.\\\\
    Demostración.-\; Expresando cada uno de los miembros de (12.2) en función de los componentes, obtenemos $$\left(\sum_{k=1}^n a_kb_k\right)^2\leq \left(\sum_{k=1}^n a_k^2\right)\left(\sum_{k=1}^n b_k^2\right)$$
    que es la desigualdad ya demostrada en el teorema I.41.\\\\
    Presentaremos otra demostración de 12.2 que no utiliza los componentes.\\
    Tal demostración es interesante porque hace ver que la desigualdad de Cauchy-Schwarz es una consecuencia de las cinco propiedades del producto escalar que se citan en el teorema 12.2 y no depende de la definicón que se utilizó para deducir esas propiedades.\\
    Para llevar a acabo esta demostración, observemos primero que 12.2 es trivial si $A$ ó $B$ es el vector cero. Por tanto, podemos suponer que $A$ y $B$ son ambos no nulos. Sea $C$ el vector 
    $$C=xA-yB, \quad \mbox{donde}\; x=B\cdot B \quad \mbox{y} \quad y=A\cdot B$$
    Las propiedades d) y e) implican que $C\cdot C \geq 0$. Cuando expresamos esto en función de $x$ e $y$, resulta 12.2. Para expresar $C\cdot C$ en función de $x$ e $y$, utilizamos las propiedades a), b) y c) obteniendo $$C\cdot C = (xA-yB)\cdot (xA-yB) = x^2(A\cdot A) - 2xy(A\cdot B) + y^2(B\cdot B).$$
    Utilizando las definiciones de $x$ e $y$ como también la desigualdad $C\cdot C\leq 0$, se llega a $$(B\cdot B)^2\cdot(A\cdot A) -2(A\cdot B)^2(B\cdot B)+(A\cdot B)^2(B\cdot B)\geq 0$$
    La propiedad d) implica que $B\cdot B \geq 0$ puesto que $B\neq 0$, con lo que podemos dividir por $(B\cdot B)$ obteniendo $$(B\cdot B)(A\cdot A)-(A\cdot B)^2\geq 0$$
    que coincide con 12.2. Esto también demuestra que el signo igual es válido en 12.2 si y sólo si $C=0$. Pero $C=0$ si y sólo si $xA=yB$. A su vez, esta igualdad se verifica si y sólo si uno de los vectores es el producto del otro por un escalar.\\\\

\end{teo}


\section{Longitud o norma de un vector}

%--------------------definicion 12.4
\begin{tcolorbox}
    \begin{def.} si $A$ es un vector en $V_n$ su longitud o norma se designa con $\|A\|$ y se define mediante la igualdad $$\|A\| = (A\cdot A)^{1/2}$$
    \end{def.}
\end{tcolorbox}

\begin{teo} Si $A$ es un vector de $V_n$ y $c$ un escalar, tenemos las siguientes propiedades.
\begin{center}
    \begin{tabular}{rcl}
	\textbf{a)}&$\|A\|>0$ si $A\neq 0$&(Positividad)\\	
	\textbf{b)}&$\|A\|=0$ si $A=0$&\\	
	\textbf{c)}&$\|cA\|=\left|c\right|\|A\|$&(homogeneidad)\\
    \end{tabular}
\end{center}

    Demostración.-\; Las propiedades a) y b) son consecuencia inmediata de las propiedades d) y e) del teorema 12.2. Para demostrar c) utilizamos la propiedad de homogeneidad del producto escalar obteniendo $$\|cA\|=(cA\cdot cA)^{1/2} = (c^2A\cdot A)^{1/2} ) (c^2)^{1/2}(A\cdot A)^{1/2} = |c|\|A\|.$$
    La desigualdad de Cauchy-Schwarz también se puede expresar en función de la norma. Ella establece que $$(A\cdot B)^2 \leq \|A\|^2 \|B\|^2\qquad \mbox{12.3}$$ $$o$$ $$|A\cdot B| = \|A\|\|B\| \qquad \mbox{12.4}$$\\\\

\end{teo}

%--------------------teorema 12.5
\begin{teo}[Desigualdad triangular]. Si $A$ y $B$ son vectores de $V_n$, tenemos $$\|A+B\| \leq \|A\|+\|B\|.$$\\
    Demostración.-\; para evitar las raíces cuadradas, escribimos la desigualdad triangular en la forma equivalente $$\|A\cdot B\|^2 = \left(\|A\|+\|B\|\right)^2\qquad \mbox{12.5}$$
    El primer miembro de 12.5 es $$\|A+B\|^2 = (A+B)\cdot (A+B) = A\cdot A + 2A\cdot B + B\cdot B = \|A\|^2 + 2A\cdot B + B\cdot B = \|A\|^2 + 2A\cdot B + \|B\|^2$$
    mientras que el segundo miembro es $$\left(\|A\|+\|B\|\right)^2 = \|A\|^2 + 2\|A\|\|B\|+\|B\|^2$$
    Comparando esas dos fórmulas, vemos que 12.5 es válida si y sólo si se tiene $$A\cdot B \leq \|A\|\|B\|\qquad 12.6$$
    Pero $A\cdot B\leq A\cdot B$ con lo que 12.6 resulta de la desigualdad de Cauchy-Schwarz, en la forma 12.4. Esto prueba que la desigualdad triangular es consecuencia de la desigualdad de Cauchy-Schwarz.\\
    La proposición reciproca también es cierta. Esto es, si la desigualdad triangular es cierta también lo es 12.6 para $A$ y para $-A$, de lo que obtenemos 12.3. Así pues, la desigualdad triangular y la de Cauchy-Schwarz son lógicamente equivalentes. Además, el signo de igualdad vale en una si y sólo si vale en la otra. Con lo que se complementa la demostración del teorema 12.5.\\\\

\end{teo}

\section{Ortogonalidad de vectores}

%--------------------definicion 12.5
\begin{tcolorbox}
\begin{def.} Dos vectores $A$ y $B$ de $V_n$ son perpendiculares u ortogonales si $A\cdot B=0$.
\end{def.}
\end{tcolorbox}


\section{Ejercicios}

\begin{enumerate}[\bfseries 1.]

%--------------------1.
\item Sean $A=(1,2,3,4)$, $B=(-1,2,-3,0)$ y $C=(0,1,0,1)$ tres vectores de $V_4$. Calcular cada uno de los siguientes productos.
\begin{enumerate}[\bfseries (a)]

    %----------(a)
    \item $A\cdot B = \sum\limits_{k=1}^4 a_kb_k = 1\cdot (-1) + 2\cdot 2 + 3\cdot (-3) + 4\cdot 0 = -1 + 4 - 9 + 0 = -6$.\\\\ 

    %----------(b)
    \item $B\cdot C = (-1)\cdot 0 + 2\cdot 1 + (-3)\cdot 0 + 0\cdot 1 = 2$.\\\\

    %----------(c)
    \item $A\cdot C = 1\cdot 0 + 2\cdot 1 + 3\cdot 0 + 4\cdot 1 = 6$.\\\\

    %----------(d)
    \item $A\cdot(B+C) = A\cdot B + A\cdot C = (-6) + 6 = 0$.\\\\

    %----------(e)
    \item $(A-B)\cdot C = A\cdot C - B \cdot C = 6 - 2 = 4$.\\\\

\end{enumerate}

%--------------------2.
\item Dados tres vectores $A=(2,4,-7)$, $B=(2,6,3)$ y $C=(3,4,-5)$. En cada una de las expresiones siguientes se pueden introducir paréntesis de una sola manera para obtener una expresión que tenga sentido. Introducir paréntesis y efectuar las operaciones.
\begin{enumerate}[\bfseries (a)]
    
    %----------(a)
    \item $(A\cdot B)C = \left[2\cdot 2 + 4\cdot 6 + (-7)\cdot 3\right]C = 7(3,4,-5) = (21,28,-35)$.\\\\

    %----------(b)
    \item $A\cdot (B + C) = (2,4,-7)\cdot (5,10,-2) = 2\cdot 5 + 4\cdot 10 + (-7)\cdot(-2) = 64$.\\\\ 

    %----------(c)
    \item $(A+B)\cdot C = (4,10,-4)\cdot (3,4,-5) = 4\cdot 3 + 10 \cdot 4, (-4)\cdot (-5) = 72$.\\\\

    %----------(d)
    \item $A(B\cdot C) = A\left[2\cdot 3 + 6\cdot 4 + 3\cdot (-5)\right] = (2,4,-7)15 = (30,60,-105)$.\\\\

    %----------(e)
    \item $A/(B\cdot C) = A/\left[2\cdot 3 + 6\cdot 4 + 3\cdot (-5)\right] = (2,4,-7)/15 = \left(\dfrac{2}{15},\dfrac{4}{15},-\dfrac{7}{15}\right)$.\\\\

\end{enumerate}

%-------------------3.
\item Demostrar si es o no cierta la proposición siguiente referentes a vectores en $V_n$: Si $A\cdot B=A\cdot C$ y $A\neq 0$, es $B=C$.\\\\
    Demostración.-\; Esta proposición es falsa ya que si $A=(1,1,1)$, $B=(1,-1,0)$ y $C=(0,-1,1)$ entonces $$A\cdot B = 1\cdot 1 + 1\cdot(-1)+1\cdot 0 = 0, \qquad A\cdot C = 1\cdot 0 + 1\cdot(-1)+1\cdot 1 = 0$$
    de donde $B\neq C$.\\\\ 

%--------------------4.
\item Demostrar si es o no cierta la proposición siguiente que se refiere a vectores $V_n$: Si $A\cdot B = 0$ para cada $B \in V_n$, es $A=0$ \\\\ 
    Demostración.-\; Sea $A,B \in V_n$. Si $A\cdot B = 0$ para todo $B$,  debemos tener en particular $A\cdot A = 0$. Pero sabemos que $A\cdot A = 0$ si y sólo si $A=0$, por lo tanto $A=0$.\\\\

%--------------------5.
\item Si $A=(2,1,-5)$ y $B=(1,-1,2)$, hallar un vector no nulo $C$ de $V_3$ tal que $A\cdot C=B\cdot C=0$.\\\\
    Respuesta.-\; Sea $C=(c_1,c_2,c_3)$ entonces $$A\cdot C = 2c_1+c_2-c_3=0$$ $$B\cdot C = c_1-c_2+2c_3 = 0$$
	de donde $$c_1=\dfrac{c_3-c_2}{2}\quad \Longrightarrow \quad \dfrac{5c_3 - 3c}{2}=0 \quad \Longrightarrow \quad c_2 = \dfrac{5c_3}{3}$$
	Si $c_3=3$ entonces tenemos $c_2=5$ y $c_1=-1$ de donde $C=(-1,5,3)$.\\\\

%--------------------6.
\item Si $A = (1,-2,3)$, $B=(3,1,2)$, hallar los escalares $x$ e $y$ tales que $C = xA + yB$ es un vector no nulo y que $C\cdot B = 0$.\\\\
    Respuesta.-\; Se tiene $$C = x(1,-2,3) + y(3,1,2 = (x+3y,-2x+y,3x+2y),$$ por lo tanto $$C\cdot B = (x+3x,-2x+y,3x+2y)\cdot (3,1,2) = 3(x+3y)+(-2x+y)+2(3x+2y) = 0 $$ $$\big\Downarrow$$ $$\; x = -2y$$ 
    Sea $y=1$ entonces $x=-2$ es la solución.\\\\

%--------------------7.
\item Si $A=(2,-1,2)$ y $B=(1,2,-2)$, hallar dos vectores $C$ y $D$ de $V_3$ que satisfagan todas las condiciones siguientes: $A=C+D$, $B\cdot D = 0$, $C$ paralelo a $B$.\\\\
    Respuesta.-\;  Sea $C=(c_1,c_2,c_3)$ y $D=(d_1,d_2,d_3)$, entonces $$A=C+D \;  \Longrightarrow\; c_1+d_1=2, \quad c_2+d_2=-1, \quad c_3+d_3 = 2.$$
    Luego, $$B\cdot D = 0\; \Longrightarrow \; d_1+2d_2-2d_3=0$$
    Finalmente, sabemos que $C$ es paralelo a $B$ implica que existe $x$ distinto de $0$ tal que 
    $$C=xB \; \Longrightarrow \; c_1=x, \quad c_2=2x, \quad c_3 = -2x$$
    Poniendo todo junto tenemos 
    \begin{center}
	\begin{tabular}{rcl}
	    $d_1$&$=$&$2-x$\\
	    $d_2$&$=$&$-1-2x$\\
	    $d_3$&$=$&$2+2x$\\
	\end{tabular}
    \end{center}
     de donde podemos encontrar $x$ de la siguiente manera,
     $$(2-x)+2(-1-2x)-2(2+2x)=0 \; \Longrightarrow\; x=-\dfrac{4}{9}$$
     por lo tanto, $$C=\left(-\dfrac{4}{9},-\dfrac{8}{9},\dfrac{8}{9}\right) \; \Longrightarrow \;  C=\dfrac{4}{9}(-1,-2,2)$$ $$D=\left(\dfrac{22}{9},\dfrac{1}{9},\dfrac{10}{9} \right) \; \Longrightarrow \; D=\dfrac{1}{9}(22,-1,10)$$\\

%--------------------8.
 \item Si $A=(1,2,3,4,5)$, $B=(1.\frac{1}{2},\frac{1}{3},\frac{1}{4},\frac{1}{5})$, hallar dos vectores $C$ y $D$ de $V_5$ que satisfagan todas las condiciones siguientes: $B=C+2D,$ $D\cdot A=0$, $C$ paralelo a $A$.\\\\
     Respuesta.-\; Sea $C=(c_1,c_2,c_3,c_4,c_5)$ y $D=(d_1,d_2,d_3,d_4,d_5)$ de donde para $B=C+2D$ se tiene, $$c_1+2d_1=1, \quad c_2+2d_2=2, \quad c_3+2d_3=3, \quad c_4+2d_4=4, \quad c_5+2d_5=5.$$
     Luego por definición de paralelismo de vectores se tiene  $C=xA$ obtenemos,
     $$c_1=x,\quad c_2=2x, \quad c_3=3x, \quad c_4=4x,\quad c_5=5x$$
     se sigue, $$d_1=\dfrac{1}{2} (1-x), \quad d_2=\dfrac{1}{2}\left(\dfrac{1}{2}-2x\right) , \quad d_3 =\dfrac{1}{2}\left(\dfrac{1}{3}-3x\right),\quad d_4=\dfrac{1}{2}\left(\dfrac{1}{4}-4x\right),\quad d_5=\dfrac{1}{2}\left(\dfrac{1}{5}-5x\right).$$
     Así, por el hecho de que $C=xA$, entonces $d_1,2d_2+3d_3+4d_4+5d_5=0$  implica,
     \begin{center}
	 \begin{tabular}{rcl}
	     $\dfrac{1}{2}(1-x)+\left(\dfrac{1}{2}-2x\right)+\dfrac{3}{2}\left(\dfrac{1}{3}-3x\right)+2\left(\dfrac{1}{4}-4x\right)+\dfrac{5}{2}\left(\dfrac{1}{5}-5x\right)$&$=$&$0$\\\\
	     $55x$&$=$&$0$\\\\
	     $x$&$=$&$\dfrac{1}{11}$\\\\
	 \end{tabular}
     \end{center}

     Por lo tanto $$C=\dfrac{1}{11}(1,2,3,4,5), \qquad D=\left(\dfrac{5}{11},\dfrac{7}{44},\dfrac{1}{33},\dfrac{-5}{88},\dfrac{-7}{55}\right)$$\\

%--------------------9.
\item 
\end{enumerate}




%--------------------michael spivak--------------------

    %---------- caratula
	%\begin{tabular}{r l }
Universidad: & \textbf{Mayor de San Ándres.}\\
Asignatura: & \textbf{Álgebra Lineal I}\\
Ejercicio: & \textbf{Práctica 1.}\\ 
Alumno: & \textbf{PAREDES AGUILERA CHRISTIAN LIMBERT.}
\end{tabular}
\begin{flushleft}
\begin{tikzpicture}
\draw(0,1)--(16.5,1);
\end{tikzpicture}
\end{flushleft}



    %---------- propiedades básicas de los números.
	%\chapter{Propiedades básicas de los números}
\pagenumbering{arabic} 
\section{Propiedades, definiciones y Teoremas}
\begin{tcolorbox}
%propiedad 1
\begin{prop}[Ley asociativa para la suma] 
$$a+(b+c) = (a+b)+c$$
\end{prop}

%propiedad 2
\begin{prop}[Existencia de una identidad] 
$$a+0 = 0+a = a$$
\end{prop}

%propiedad 3
\begin{prop}[Existencia de inversos para la suma]
$$a+(-a) = (-a) + a = 0$$
\end{prop}
\end{tcolorbox}

\begin{tcolorbox}
%definición 1.1
\begin{def.}
Conviene considerar la resta como una operación derivada de la suma: consideremos $a-b$ como una abreviación de $a+(-b)$\\
\end{def.}
\end{tcolorbox}

%teorema 1.1
\begin{teo}
Si un número $x$ satisface $a+x=a$ para cierto número $a$, entonces es $x=0$ (y en consecuencia esta ecuación se satisface también para cualquier $a$)\\\\
Demostración.- \; Si $a+x=a$ entonces $(-a)+(a+x)=(-a)+a=0$ de donde $\left[ (-a) + a \right] + x = 0$, por lo tanto $x=0$\\\\ 
\end{teo}

\begin{tcolorbox}

%propiedad 4
\begin{prop}[Ley conmutativa para la suma]
$$a+b = b+a$$
\end{prop}

%propiedad 5
\begin{prop}[Ley asociativa para la multiplicación]
$$a\cdot (b \cdot c) = (a \cdot b) \cdot c$$
\end{prop}

%propiedad 6
\begin{prop}[Existencia de una identidad para la multiplicación]
$$a \cdot 1 = 1 \cdot a = a; \; 1 \neq 0$$
\end{prop}

%propiedad 7
\begin{prop}[Existencia de inversos para la multiplicación]
$$a\cdot a^{-1} = a^{-1} \cdot a = 1; \; para \; a\neq 0$$
\end{prop}
\end{tcolorbox}

%teorema 1.2
\begin{teo}
Si es $a\cdot b = a \cdot c$ y $a\neq 0$, entonces $b=c$\\\\
Demostración.- \; Si $a\cdot b = a\cdot c$ y $a\neq 0$ entonces $a^{-1} \cdot (a \cdot b) = a^{-1} \cdot (a \cdot c)$ de donde $(a^{-1} \cdot a) \cdot b = (a^{-1} \cdot a) \cdot c$, por lo tanto $b=c$\\\\
\end{teo}

%teorema 1.3
\begin{teo}
Si $a\cdot b =0$ entonces $a=0$ ó $b=0$\\\\
Demostración.- \; Si $a\neq 0$ entonces $a^{-1} \cdot (a\cdot b) = 0$ de donde  $(a^{-1} \cdot a) \cdot b =0$ por lo tanto $b=0$\\
(Puede ocurrir que sea a la vez $a=0$ y $b=0$;) esta posibilidad no se excluye cuando decimos $a=0$ y $b=0$. \\\\
\end{teo}

\begin{tcolorbox}
%definición 1.2
\begin{def.}
Se define a la división en función de la multiplicación: el símbolo $a/b$ significa $a\cdot b^{-1}$. Puesto que $0^{-1}$ no tiene sentido, tampoco lo tiene $a/0;$la división por $0$ es siempre indefinida. 
\end{def.}
\end{tcolorbox}

\begin{tcolorbox}
%propiedad 8
\begin{prop}[Ley conmutativa para la multiplicación] 
$$a\cdot b = b\cdot a$$
\end{prop}

%propiedad 9
\begin{prop}[Ley distributiva]
$$a\cdot (b+c) = a \cdot b + a \cdot c$$
\end{prop}
\end{tcolorbox}

%teorema 1.4
\begin{teo}
Determinar cuando es $a-b=b-a$\\\\
Demostración.- \; Si $a-b=b-a$ entonces $(a-b) + b = (b-a) + b = b+ (b-a)$ de donde $a=b+b-a$, luego $a+a = =(b+b-a)+a=b+b$, en consecuencia $a\cdot (1+1)=b\cdot (1+1)$ y por lo tanto $a=b$\\\\
\end{teo}

%teorema 1.5
\begin{teo}
Demostrar que $a\cdot 0 = 0$\\\\
Demostración.- \; Sea $0 + 0 = 0$ entones $a \cdot (0+0) =0 \cdot a$ de donde $a\cdot 0 = 0$\\\\ 
\end{teo}

%teorema 1.6
\begin{teo}
Demostrar que $(-a)\cdot b = -(a\cdot b)$\\\\
Demostración.- \; Notemos que $(-a)\cdot b = \left[ (-a)+a\right] \cdot b = 0\cdot b = 0$, por lo tanto  $(-a)\cdot b =0$. Se sigue inmediatamente [sumando $-(a\cdot b)$ a ambos miembros] que $(-a)\cdot b = -(a\cdot b)$\\\\
\end{teo}

%teorema 1.7
\begin{teo}
Demostrar que $(-a) \dot (-b)=a\cdot b$\\\\
Demostración.- \; Notemos que  $(-a)\cdot (-b) + \left[ - (a \cdot b)\right] = (-a) \cdot (-b) + (-a) \cdot b = (-a)\cdot \left[ (-b)+b\right] = (-a)\cdot 0 =0$. En consecuencia sumando $(a \cdot b)$ a ambos lados se obtiene $(-a) \cdot (-b)=a \cdot b$\\\\
\end{teo}

%ejercicio 1.1
\begin{ej}
Resolver $x^2-3x+2 = (x-1)(x-2)$\\
\begin{center}
\begin{tabular}{rcl}
$(x-1)\cdot (x-2)$&$=$&$x\cdot (x-2)+ (-1) \cdot (x-2)$\\
&$=$&$x\cdot x + x\cdot (-2) + (-1) \cdot x + (-1) \cdot (-2)$\\
&$=$&$x^2+ x \left[(-2) + (-1)\right] + 2$\\
&$=$&$x^2 - 2x + 2$\\\\
\end{tabular}
\end{center}
\end{ej}

\begin{tcolorbox}
%definición 1.3
\begin{def.} Para los números $a$ que satisfagan:
\begin{itemize}
\item $a>0$, se llaman \textbf{positivos}
\item $a<0$ se llaman \textbf{negativos}\\
\end{itemize}
\end{def.}

%definición 1.4
\begin{def.}
$a<b$ puede interpretarse como $b-a>0$ \\
\end{def.}
Conviene considerar el conjunto de todos los números positivos, representados por $P$\\
\end{tcolorbox}

\begin{tcolorbox}
%propiedad 10
\begin{prop}[Ley de tricotomía] Para todo número $a$ se cumple una y sólo una de las siguientes igualdades:
\begin{enumerate}[\bfseries i)]
\item $a=0$
\item $a$ pertenece al conjunto $P$
\item $-a$ pertenece al conjunto $P$
\end{enumerate}
\end{prop}

%propiedad 11
\begin{prop}[La suma cerrada] Si $a$ y $b$ pertenecen a $P$, entonces $a+b$ pertenecen a $P$.
\end{prop}

%propiedad 12
\begin{prop}[La multiplicación es cerrada] Si $a$ y $b$ pertenecen a $P$, entonces $a\cdot b$ pertenece a $P$
\end{prop}
\end{tcolorbox}

\begin{tcolorbox}
%definición 1.5
\begin{def.} Estas tres propiedades deben complementarse con las siguientes definiciones.
\begin{center}
\begin{tabular}{r c l}
$a>b$&si&$a-b$ pertenece a $P$\\
$a<b$&si&$b>a$\\
$a\geq b $&si&$a>b$ ó $a=b$\\
$a\leq b$&si&$a<b$ ó $a=b$\\
\end{tabular}
Nótese en particular que $a>0$ si y sólo si $a$ pertenece a $P$.\\
\end{center}
\end{def.}

%definición 1.6
\begin{def.}
Si $a$ y $b$ son dos números cualesquiera, entonces se cumple una y sólo una de las siguientes igualdades:
\begin{enumerate}[\bfseries i)]
\item $a-b=0,$
\item $a-b$ pertence al conjunto $p,$
\item $-(a-b) = b-a$ pertenece al conjunto $p,$
\end{enumerate}
De las definiciones dadas se cumple una y sólo una de las sigueintes igualdades:
\begin{enumerate}[\bfseries i)]
\item $a=b,$
\item  $a>b,$
\item $b>a.$
\end{enumerate}
\end{def.}
\end{tcolorbox}

%teorema 1.8
\begin{teo}
Si $a<b$ y $b<c$ si y sólo si  $a<c$\\\\
Demostración.- \; Si $a<b$ de modo que $b-a$ pertenece a $P$, entonces evidentemente $(b+c)+(a+c)$ pertenece a $P$; así si $a<b$ entonces $a+c<b+c$. Igualmente, supongamos $a<b$ y $b<c$. Entonces $b-a$ y $c-b$ están en $P$ así que $(c-b)+(b-a)=c-a$ está en $P$.\\\\
\end{teo}

%teorema 1.9
\begin{teo}
Si $a<0$ y $b<0$, entonces $ab>0$\\\\
Demostración.- \; Por definición $0>a$ lo cual significa que $0-a=-a$ esta en $P$. Del mismo modo, $-b$ pertenece a $P$ y, en consecuencia por P12, $(-a)(-b)=ab$ está en $P$. Así pues $ab>0$  \\\\
\end{teo}

%teorema 1.10
\begin{teo}
Si $a\neq0$ es $a^2>0$\\\\
Demostremos por casos..- \;
Si $a>0$, entonces  \; $a\cdot a >0$ \; y \; $a^2>0$. Por otro lado, si $a<0$, entonces $0-a>0$ de modo que $(-a)(-a)>0$ \; y por lo tanto \; $a^2>0$\\\\
\end{teo}

\begin{tcolorbox}
%definición 1.7
\begin{def.}[Valor absoluto] Se define como:
\begin{center}
$|a| = \left\lbrace
\begin{array}{rr}
a, & a\geq 0\\
-a, & a \leq 0
\end{array}
\right.$\\
\end{center}
\end{def.}
\end{tcolorbox}

%teorema 1.11
\begin{teo}
Para todos los números $a$ y $b$ se tiene $$|a+b|\leq |a| + |b|$$\\\\
Demostración.- \; Demostración.- \; Vamos a considerar cuatro casos:
\begin{center}
\begin{tabular}{c r r}
$(1)$&$a\geq 0$&$b\geq 0$\\
$(2)$&$a\geq $&$b\leq 0$\\
$(4)$&$a\leq $&$b \geq 0$\\
$(5)$&$a\leq 0$&$b \leq 0$\\
\end{tabular}
\end{center}
En el caso ($1)$ tenemos también $a+b\geq 0$, esto es evidente; en efecto por definición $$|a+b|=a+b=|a|+|b|$$ de modo que en este caso se cumple la igualdad.\\
En el caso $(4)$ se tiene $a+b\leq 0$ y de nuevo se cumple la igualdad: $$|a+b|=-(a+b)=(-a)+(-b)=|a|+|b|$$
En el caso $(2)$, cuando $a\geq 0$ y $b\leq 0$, debemos demostrar que $$|a+b|\leq a - b$$ Este caso puede dividirse en dos subcasos. Si $a+b\geq 0$, entonces tenemos que demostrar que $$a+b \leq a-b$$ es decir, $$b\leq -b,$$  lo cual se cumple ciertamente puesto que $b$ es negativo y $-b$ positivo. Por otra parte, si $a+b\leq 0$ debemos demostrar que $$-a-b\leq a-b$$ es decir $$-a\leq a,$$ lo cual es verdad puesto que $a$ es positivo y \; $-a$ negativo.\\
Nótese finalmente que el caso $(3)$ puede despacharse sin ningún trabajo adicional aplicando el caso $(2)$ con $a$ \; y \; $b$ intercambiados.\\\\ 
Se puede dar una demostración mas corta dado que $$|a|=\sqrt{a^2} \; ó \; |a|^2=a^2$$. Sea $(|a+b|)^2=(a+b)$ Entonces 
\begin{center}
\begin{tabular}{r c r c l}
$(|a+b|)^2$&$=$&$(a+b)^2$&$=$&$a^2+2ab+b^2$\\\\
&&&$\leq$&$a^2+2|a|\dot |b|+b^2$\\\\
&&&$=$&$|a|^22|a|\dot |b|+|b|^2$\\\\
&&&$=$&$(|a|+|b|)^2$\\\\
\end{tabular}
\end{center}
De esto podemos concluir que $|a+b|\leq |a|+|b|$ porque $x^2<y^2$ implica $x<y$\\\\
Hay una tercera forma de probar que es utilizando el teorema anterior.\\
Puesto que $x=|x|$ \; ó \; $x=-|x|$, se tiene $-|x|\leq x \leq |x|$. Análogamente $-|y| \leq y \leq |y|$. Sumando ambas desigualdades se tiene: $$-(|x|+|y|)\leq x+y \leq |x|+|y|$$ y por tanto en virtud del teorema 4.2 se concluye que: $|x+y|\leq |x|+|y|$\\\\
\end{teo}

%problemas capítulo 1
\section{Problemas}
\begin{enumerate}[\bfseries 1.]

%----------------------------------1-------------------------------------
\item Demostrar lo siguiente:
\begin{enumerate}[\bfseries i)]
%i)
\item Si $ax=a$ para algún número $a\neq 0$, entonces $x=1$\\\\
Demostración.- \; Sea $a\neq 0$ entonces $(a^{-1}\cdot a)x=a\cdot a^{-1}$ por lo tanto $\cdot x = 1$\\\\
 
%ii)
\item $x^2-y^2=(x-y)(x+y)$\\\\
Demostración.- \; Partamos de $(x-y)(x+y)$ donde por la propiedad distributiva tenemos $(x-y)x+(x-y)y$, luego $x^2-xy+xy-y^2$, por lo tanto por las propiedades de inverso y  neutro $x^2-y^2$\\\\

%iii)
\item Si $x^2=y^2$, entonces $x=y$ o $x=-y$\\\\
Demostración.- \; Dada la hipótesis entonces $x^2+\left[ - (y^2) \right]=y+\left[ - (y^2) \right]$ y por propiedades de neutro y definición $x^2-y^2=0$, luego $(x-y)(x+y)$ y en virtud del teorema $ab=0$ entonces $a=0$ o $b=0$ no queda $x-y=0$ ó $x+y=0$, por lo tanto $x=y$ ó $x=-y$ \\\\ 

%iv)
\item $(x^3-y^3)=(x-y)(x^2+xy+y^2)$\\\\
Demostración.- \; Dado $(x-y)(x^2+xy+y^2)$ entonces por la propiedad distributiva $(x-y)x^2+(x-y)xy+(x-y)y^2 = x^3 -x^2y +x^2y-xy^2+xy^2-y^3$ por lo tanto en virtud de las propiedades de inverso y neutro $x^3-y^3$.\\\\

%v)
\item $x^n-y^n=(x+y)(x^{n-1}+x^{n-2} y + ... + xy^{n-2}+y^{n-1})$\\\\
Demostración.- \; 
\begin{center}
\begin{tabular}{r c l}
&&$(x-y)(x^{n-1}+x^{n-2}y+...+xy^{n-2}+y^{n-1})$\\
&=&$x(x^{n-1}+x^{n-2}y+...+xy^{n-2}+y^{n-1})-y(x^{n-1}+x^{n-2}y+...+xy^{n-2}+y^{n-1}$\\
&=&$x^n+x^{n-1}y+...+x^2y^{n-2}+xy^{n-1}-(x^{n-1}y+x^{n-2}y^2+...+xy^{n-1}+y^n)$\\
&=&$x^n-y^n$\\\\	
\end{tabular}
\end{center}

%vi)
\item $x^3+y^3=(x+y)(x^2-xy+y^2)$\\\\
Demostración.- \; Sea $(x+y)(x^2-xy+y^2)$ entonces por la propiedad distributiva $x^3 -x^2y + xy^2 + x^2y - xy^2 + y^2$, por lo tanto $x^3+y^3$\\\\
\end{enumerate}

%---------------------------------2----------------------------------
\item ¿Donde está el fallo en la siguiente demostración? Sea $x=y$. Entonces 
\begin{center}
\begin{tabular}{rcl}
$x^2$&$=$&$xy$\\
$x^2 + y^2$&$=$&$xy-y^2$\\
$(x+y)(x-y)$&$=$&$y(x-y)$\\
$x+y$&$=$&$y$\\
$2$&$=$&$1$\\
\end{tabular}
\end{center}
El fallo esta en que no se puede dividir un número por $0$ sabiendo que $x=y$\\\\

%--------------------------------3-----------------------------------
\item Demostrar lo siguiente:
\begin{enumerate}[\bfseries i)]
%1
\item $\dfrac{a}{b} = \dfrac{ac}{bc},\; si \; b, c\neq 0$\\\\
Demostración.- \;
Por definición tenemos que $\displaystyle\frac{a}{b}=ab^{-1}$, como $b, \; c \neq 0$ entonces $(ab)(c\cdot c^{-1})$, por  las propiedades asociativa y conmutativa, $(ac)(b^{-1}c^{-1})$ por lo tanto $\displaystyle\frac{ac}{bc}$ \\\\ 

%2
\item $\dfrac{a}{b} + \dfrac{c}{d} = \dfrac{ad+bc}{bd}, \; si \; b,d \neq 0$\\\\
    Demostración.- \; $(ad+bc)/(bd)=(ad+bc)(bd)^{-1}=(ad + bc)(b^{-1}d^{-1})=ab^{-1} + cd^{-1}=a/d + c/d$\\\\

%3
\item $(ab)^{1} = a^{-1} b^{-1}$, si $a,b \neq 0$ (Para hacer esto hace falta tener presente cómo se ha definido $(ab)^{-1}$)\\\\
Demostración.- \; Demostremos que  $a^{-1} b^{-1} (ab) = 1$, Sea $a^{-1} b^{-1} (ab) = \left( a^{-1} a \right)\left( b^{-1} b \right) = 1$\\\\

%4
\item $\dfrac{a}{b} \cdot \dfrac{c}{d} = \dfrac{ac}{db},$ si $b,d \neq 0$\\\\
Demostración.- \; Sea por definición $ab^{-1} \cdot cd^{-1}$ entonces  por la propiedad conmutativa $ac \cdot b^{-1}d^{-1}$, por lo tanto $\dfrac{ac}{bd}$ si $b,d \neq 0$\\\\
%corolario 1
\begin{col.}
Si $c\neq 0$ y $d\neq 0$ entonces $(cd^{-1})^{-1}=c^{-1}d$\\\\
Demostración.- \;
Por definición de $a^{-1}$ tenemos que $(cd^{-1})^{-1}=\displaystyle\frac{1}{cd^{-1}}$, por el teorema de posibilidad de la división $1=(c^{-1}d)(cd^{-1})$ y en virtud de los axiomas de conmutatividad y asociatividad $1=(c^{1}c)(dd^{-1})$, luego $1=1$. quedando demostrado el corolario.\\\\
\end{col.}

%5
\item $\dfrac{a}{b} / \dfrac{c}{d} = \dfrac{ad}{bc}$ si $b,c,d \neq 0$\\\\
Demostración.- \; Si $ab^{-1} \cdot \left( cd^{-1}\right)^{-1}$ en virtud del anterior corolario se tiene $ab^{-1}\cdot c^{-1}d$ y por lo tanto $\dfrac{ad}{bc}$ \\\\

%6
\item Si $b,\; c \neq 0$, entonces $\displaystyle\frac{a}{b}=\frac{c}{d}$ si sólo si $ad=bc$, Determinar también cuando es $\displaystyle\frac{a}{b}=\frac{b}{a}$\\\\
Demostración.- \; Sea $b,\; c \neq 0$ si sólo si $ab^{-1}=cd^{-1}$ entonces $(ab^{-1})b=cd^{-1}b$, por propiedades asociativa y conmutativa $a(b\cdot b^{-1})=(bc)d^{-1}$, $a=(bc)d^{-1}$ luego $ad=bc(d\cdot d^{-1})$, por lo tanto $ad=bc$.\\
Por otro lado, si $ab^{-1} = b^{-1}$ entonces  $a^2=b^2$,por lo tanto determinamos que $a=b$ ó $a=-b$ \\\\
\end{enumerate}

%--------------------------------4-------------------------------------
\item Encontrar todos los números $x$ para los que
\begin{enumerate}[\bfseries i)]
%i)
\item $4-x<3-2x$
\begin{center}
\begin{tabular}{c r c l l}
$\Rightarrow$&$4-x+2x$&$<$&$3-2x+2x$&\\
$\Rightarrow$&$x+4$&$<$&$3$&Axiomas\\
$\Rightarrow$&$4+(-4)x$&$<$&$3+(-4)$&\\
$\Rightarrow$&$x$&$<$&$-1$&propiedades\\\\
\end{tabular}
\end{center}

%ii)
\item $5-x^2<8$
\begin{center}
\begin{tabular}{crcll}
$\Rightarrow$&$(-5)+5-x^2+(-8)$&$<$&$(-8)+8+(-5)$&\\
$\Rightarrow$&$-x^2-8$&$<$&$-5$&\\
$\Rightarrow$&$-x^2-3$&$<$&$0$&\\
$\Rightarrow$&$-(-x^2-3)$&$>$&$-0$&\\
$\Rightarrow$&$x^2+3$&$>$&$0$&\\\\
\end{tabular}
\end{center}
Sea $x \neq 0$ entonces por teorema \; $x^2>0$ y por propiedad se cumple que $x^2+3$ siempre es positivo, y como $3>0$ entonces el valor de $x$ son todos los números reales.\\\\

%iii)
\item $5-x^2<-2$
\begin{center}
\begin{tabular}{crcll}
$\Rightarrow$&$(-5)+5-x^2$&$<$&$-2+(-5)$&\\
$\Rightarrow$&$-x^2$&$<$&$-7$&\\
$\Rightarrow$&$x^2$&$>$&$7$&\\
$\Rightarrow$&$x>\sqrt{7}$&$ó$&$x<-\sqrt{7}$&\\\\
\end{tabular}
\end{center}

%iv)
\item $(x-1)(x-3)>0$
\begin{center}
\begin{tabular}{crcll} 
$\Rightarrow$&$x-1>0$&$y$&$x-3>0$&\\
&&$ó$&&\\
&$x-1<0$&$y$&$x-3<0$&\\\\
$\Rightarrow$&$x>1$&$y$&$x>3$&\\
&&$ó$&&\\
&$x<1$&$y$&$x<3$&\\\\
$\Rightarrow$&$x>3$&$ó$&$x<1$&\\\\
\end{tabular}
\end{center}

%v)
\item $x^2-2x+2>0$\\\\
Completando cuadrados obtenemos que $x^2-2x+1^2 -1^2 +2 > 0$, después $(x-1)^2+1^2>0$, luego $x^2>0$, y en virtud de teorema nos queda que la desigualdad dada satisface a todos los números reales.\\\\ 

%vi)
\item $x^2+x+1>2$\\\\
Aplicando el teorema se tiene $x=\dfrac{-1 \pm \sqrt{2^2-4(-1)}}{2}$. luego $$\left( x>\dfrac{-1-\sqrt{5}}{2} \; \; \; y \; \; \; x>\dfrac{-1+\sqrt{5}}{2} \right) ó \left( x<\dfrac{-1-\sqrt{5}}{2}\; \; \; y \; \; \; x < \dfrac{-1+\sqrt{5}}{2} \right),$$ por lo tanto, $$x<\dfrac{-1-\sqrt{5}}{2}\; \cup \; x>\dfrac{-1+\sqrt{5}}{2} $$ \\\\

%vii)
\item $x^2-x+10>16$
\begin{center}
\begin{tabular}{crcll}
$\Rightarrow$&$x^2-x-6$&$>$&$0$&\\
$\Rightarrow$&$(x-3)(x+2)$&$>$&$0$&\\\\
$\Rightarrow$&$x>3$&$y$&$x>-2$&\\
&$$&$ó$&$$&\\
&$x<3$&$y$&$x<-2$&\\\\
$\Rightarrow$&$x>3$&$ó$&$x<-2$&\\\\
\end{tabular}
\end{center}

%viii)
\item $x^2+x+1>0$\\\\
Sabemos que $x^2>0, \; para \; x\neq 0$, luego será verdad para $x^2+x+1>0$, entonces la inecuación se cumple para todo $x \in \mathbb{R}$\\\\

%ix)
\item $(x-\pi)(x+5)(x-3)>0$ por la propiedad asociativa $(x-\pi)\left[ (x+5)(x-3) \right]>0$
\begin{center}
\begin{tabular}{crcll}\\
$\Rightarrow$&$x>\pi$&$\land$&$\left[(x>-5\land x>3) \lor (x<-5 \land x<3) \right]$&\\
&&$\lor$&\\
&$x< \pi $&$\lor$&$\left[ (x<-5 \land x>3) \lor (x>-5 \land x<3) \right]$&\\\\
$\Rightarrow$&$x > \pi $&$\land$&$(x>3 \lor x<-5)$&\\
&&$\lor$&&\\
&$x<\pi$&$\land$&$-5<x-3$&\\\\
$\Rightarrow$&$x<\pi$&$\lor$&$-5<x-3$&\\\\
\end{tabular}
\end{center}

%x)
\item $(x-\sqrt[3]{2})(x-\sqrt{2})>0$
\begin{center}
\begin{tabular}{crcll}
$\Rightarrow$&$x>\sqrt[3]{2}$&$y$&$x>\sqrt{2}$&\\
&&$ó$&&\\
&$x<\sqrt[3]{2}$&$y$&$x<\sqrt{2}$&\\\\
$\Rightarrow$&$x>\sqrt{2}$&$ó$&$x<\sqrt[3]{2}$&\\\\
\end{tabular}
\end{center}

%xi)
\item $2^x<8$\\\\
Podemos reescribir como $2^x<2^3$ y por propiedad de logaritmos que se vera mas adelante se tiene que $x<3$\\\\

%xii)
\item $x+3^x <4$\\\\
Visualizando, está claro que si $x=1$ entonces $1+3^2=4$, luego cualquier número menor a $1$ debería ser menor a $4$, por lo tanto $x<1$\\\\

%xiii)
\item $\displaystyle\frac{1}{x} + \frac{1}{1-x}>0$
\begin{center}
\begin{tabular}{crcll}
$\Rightarrow$&$\displaystyle\frac{1}{x(1-x)}$&$>$&$0$&\\\\
$\Rightarrow$&$\displaystyle\frac{1\cdot \left[ x(1-x)\right]^2}{x(1-x)}$&$>$&$0\cdot \left[ x(1-x)\right] ^2$&\\\\
$\Rightarrow$&$x(1-x)$&$>$&$0$&\\\\
$\Rightarrow$&$x>0$&$y$&$x<1$&\\
&&$ó$&&\\
&$x<0$&$y$&$x>1$&\\\\
$\Rightarrow$&$0\; \; <$&$x$&$<\; \; 1$&\\\\
\end{tabular}
\end{center}

%xiv)
\item $\displaystyle\frac{x-1}{x+1}>0$
\begin{center}
\begin{tabular}{crcll}
$\Rightarrow$&$\displaystyle\frac{(x-1)(x+1)^2}{>}$&$>$&$0(x+1)^2$&\\\\
$\rightarrow$&$(x-1)(x+1)$&$>$&$0$&\\\\
$\Rightarrow$&$x>1$&$y$&$x>-1$&\\
&&$ó$&&\\
&$x<1$&$y$&$x<-1$&\\\\
$\Rightarrow$&$x>1$&$ó$&$x<-1$&\\\\
\end{tabular}
\end{center} 
\end{enumerate}

%--------------------------------------5-------------------------------------
\item Demostrar lo siguiente:
\begin{enumerate}[\bfseries i)]
%i)
\item Si $a<b,$ y $c<d$, entonces $a+c<b+d$\\\\
Demostración.- \; Por hipótesis y propiedad de los números reales se tiene $b-a>0$ y $d-c>0$, luego $(b-a)+(d-c)>0$, así $a+c<b+d$\\\\

%ii)
\item Si $a<b$, entonce $-b<-a$\\\\
Demostración.- \; Sea $-1<0$, por teorema  \; $-1(a)>-1(b)$, luego por existencia de elementos neutros $-a>-b$ por lo tanto $-b<-a$\\\\

%iii)
\item Si $a<b$ y $c>d$, entonces $a-c<b-d$\\\\
Demostración.- \;
Si $a<b = b-a>0$ \; y  \; $c>d=d<c=c-d>0$, por propiedad de números reales \; $(b-a)+(c-d)>0$, luego $(b-d)+(-a+c)>0$ y en virtud del teorema 1.19 y definición \; $(b-d)-(a-c)>0$, por lo tanto $a-c<b-d$\\\\

%iv)
\item Si $a<b$ y $c>0$, entonces $ac<bc$\\\\
Demostración.- \; Por propiedad de números reales $c(b-a)>0$, luego $bc-ac>0$, así $ac<bc$\\\\

%v)
\item Si $a<b$ y $c<0$, entonces $ac>bc$\\\\
Demostración.- \; Sea $b-a>0$ y $0-c>0$, entonces $-c(b-a)>0$, luego $ac - bc >0$, así $ac>bc$\\\\

%vi)
\item Si $a>1$ entonces $a^2>a$\\\\
Demostración.- \; Sea $1<a$ y $a-1>0$, por propiedad $a(a-1)>0=a^2-a>0$, luego $a<a^2$ y $a^2>a$\\\\

%vii)
\item Si $0<a<1$, entonces $a^2<a$\\\\
Demostración.- \;
La demostración es similar al teorema 2.14. Por definición $0<a$ y $a<1$ por lo tanto $1-a>0$ y $a(1-a)>0$,\; $a^2<a$.\\\\

%viii)
\item Si $a\leq a < b$ \; y \; $0 \leq c < d$, entonces $ac<bd$\\\\
Demostración.- \;Tenemos que $a\geq 0$, $c\geq 0$, $a<d$ y $a<b$, en virtud de  teorema $ac\leq bc$ y $ac \leq ad$ ( cabe recalcar que por hipótesis podría dar el caso de $0 \leq 0$  por ello el símbolo $" \leq "$) por lo tanto $bc-ac \geq 0$ \; y \; $ad-ac \geq 0$. luego $ac-ac \leq ad+bc$ y $-ad-bc\leq -2ac$. \\ 
Por otro lado sea $b-a>0$ \; y \; $d-c>0$ entonces $(b-a)(d-c)>0$ \; y \; $db-ad-bc+ac>0$.\\
Si $-ad-bc\leq -2ac$ entonces $db -2ac +ac>0$ \; así \; $ac < bd$.\\\\

%ix)
\item Si $0\leq a < b$, entonces $a^2<b^2.$\\\\
Demostración.- \;
Por el problema anterior si $0\leq a < b$ entonces $a\cdot a < b\cdot b$\; y \; $a^2<b^2$\\\\

%x)
\item Si $a, \; b \geq 0$ y $a^2<b^2$, entonces $a<b$\\\\
Demostración.- \;
Si $b^2-a^2>0$, por teorema  $(b-a)(b+a)>0$, luego $( b-a>0 \; \land \; b+a>0 ) \; \lor \; ( b-a<0 \; \land \; b+a<0 ) $. Sea $a,\; b\geq 0$ queda $(b-a>0 \; \land \; b+a>0)$ por lo tanto $a<b$.\\\\
  
\end{enumerate}

%-------------------------------------6-------------------------------------
\item 
\begin{enumerate}[\bfseries a)]
%a)
\item Demostrar que si $0 \leq x<y $ entonces $x^n<y^n$\\\\
Demostración.- \; Sea $ac<bd$, $x^2=a$, $y^2=b$ y $c=x$, $d=x$ entonces  $x\cdot x \cdot x < y \cdot y \cdot y$. Si aplicamos $n$ veces dicho teorema $x\cdot x \cdot x \cdot ... \cdot x < y \cdot y \cdot y \cdot ... \cdot y $ se tiene $x^n<y^n$\\\\

%b)
\item Demostrar que si $x<y$ y $n$ es impar, entonces $x^n<y^n$\\\\
Demostración.- \; Si consideramos  $x\geq 0$ ya quedo demostrado anteriormente. Ahora consideremos el caso donde $x<y\leq 0$, por lo tanto $0\leq -y<-x$, así por la parte a) $-y^n < -x^n$, que significa que $n$ es impar, y por lo tanto $x^n < y^n$. Finalmente si $x<0\geq y$ entonces $x^n < 0 \leq y^n,$ ya que $n$ es impar.Así queda demostrado la proposición dada.\\\\

%c)
\item Demostrar que si $x^n=  ^n$ y $n$ es impar, entonces $x=y$\\\\
Demostración.- \;
Sea $n=2k-1$ y $x^n=y^n$ entonces $x^{2k-1}-y^{2k-1}=0$ y por teorema \; $(x^{2k-1}-y^{2k-1})(x^{(2k-1)-1}+x^{(2k-1)-2}y^{2k-1}+...+x^{2k-1}y^{(2k-1)-2}+y^{(2k-1)-1})=0.$ Sea $x, \; y \neq 0$ entonces por la propiedad de existencia de reciproco o inverso \, $x-y=0$ por lo tanto $x=y$\\\\

%d)
\item Demostrar que si $x^n=y^n$ y $n$ es par, entonces $x=y$ ó $x=-y$\\\\
Demostración.- \; Si $n$ es par, entonces $x,y\geq 0$ y $x^n=y^n$,luego $x=y$. Además, si $x,y\leq 0$ y $x^n=y^n$, entonces $-x,-y\geq 0$ y $(-x)^n=(-y)^n$, por lo tanto $x=y$. La única posibilidad es que $x$ e $y$ sea positivo y el otro negativo. En este caso, $x$ e $-y$ son ambos positivos o negativos. Además $x^n=(-y)^n,$ dado que $n$ es par se sigue de los casos anteriores que $x=-y$.\\\\
\end{enumerate}

%-----------------------------------------7-------------------------------
\item Demostrar que si $0<a < b$, entonces
$$a<\sqrt{ab}<\dfrac{a+b}{2} < b$$\\
Demostración.- \;
\begin{enumerate}[1.]
\item $a<\sqrt{ab}$\\\\
Si \; $4a<b$ entonces $a^2<ab$ y por raíz cuadrada dado que $a,\;b>0$ entonces $a<\sqrt{ab}$\\
\item $\sqrt{ab}<\dfrac{a+b}{2}$\\\\
En vista de que $a, \; b > 0$ y $a<b$ entonces $a-b>0$, \; $(a-b)^2>0$ por lo tanto, $a^2-2ab+b^2>0 \Rightarrow 2ab< a^2+b^2 \Rightarrow 2ab-2ab+2ab<a^2+b^2 \Rightarrow 4ab < a^2+2ab +b^2 \Rightarrow 4ab < (a+b^)2 \Rightarrow ab < \displaystyle \left( \frac{a+b}{2} \right) ^2 \Rightarrow \sqrt{ab}<\frac{a+b}{2} $ \\
\item $\displaystyle\frac{a+b}{2}<b$\\\\
Si $a<b$ entonces $a+b<2b$ por lo tanto $\displaystyle\frac{a+b}{2}<b$\\
\end{enumerate}
Y por la propiedad transitiva queda demostrado.\\\\
 
%----------------------------8----------------------------------
\item Aunque las propiedades básicas de las desigualdades fueron enunciadas en términos del conjunto $P$ de los números positivos, y $<$ fue definido en términos de $P$ este proceso puede ser invertido. Supóngase que las propiedades 10 al 13 se sustituyen por:\\

\textbf{P-10} Cualquier que sean los números $a$ y $b$, se cumple una y sólo una de las relaciones siguientes
\begin{itemize}
\item $a=b$
\item $a<b$
\item $b<a$\\
\end{itemize}
\textbf{P-11} Cualquiera que sean $a$, $b$ y $c$, si $a<b$ y $b<c$, entonces $a<c$.\\\\
\textbf{P-12} Cualquiera que sean $a$, $b$ y $c$, si $a<b,$ entonces $a+c<b+c.$\\\\
\textbf{P-13} Cualquiera que sean $a$, $b$ y $c$, si $a<b$, y $0<c$, entonces $ac<bc.$\\\\
Demostrar que las propiedades 10 al 13 se pueden deducir entonces como teoremas.\\\\
Demostración.- \; Con respecto a \textbf{P-11} se tiene  $b-a>0$ y $c-b>0$ de modo que $c-a>0$, por lo tanto $a<c$. Luego para \textbf{P-12} se tiene $b-a>0$, por propiedad de neutro aditivo $b-a+c-c>0$, en consecuencia $a+c<b+c$. Después para \textbf{P-13} tenemos $c(b-a)>0$ por lo tanto $ac<bc$. Por último si $a<0$ entonces $-a>0$; ya que si $-a<0$ se cumpliese, se tendría $0=a+(-a)<0$ el cual es un absurdo. En consecuencia, cualquier número $a$ satisface una de las condiciones $a=0$, $a>0$ ó $-a>0.$ Con esto queda demostrado \textbf{P-10.}\\\\

%-----------------------------9---------------------------------
\item Dese una expresión equivalente de cada una de las siguientes utilizando como mínimo una vez menos el signo de valor absoluto.\\
\begin{center}
\begin{tabular}{r r c l}
%i
$(i)$&$|\sqrt{2}+\sqrt{3}-\sqrt{5}+\sqrt{7}|$&$\Rightarrow$&$\sqrt{2}+\sqrt{3}-\sqrt{5}+\sqrt{7}$\\\\

%ii
$(ii)$&$||a+b|-|a|-|b||$&$\Rightarrow$&$|a+b|-|a|-|b|$\\\\

%iii
$(iii)$&$|\left( |a+b|+|c|-|a+b+c| \right)|$&$\Rightarrow$&$|a+b|+|c|-|a+b+c|$\\\\

%iv
$(iv)$&$|x^2-2xy+y2|$&$\Rightarrow$&$x^2-2xy+y2$\\\\

%v
$(v)$&$|\left(  |\sqrt{2}+ \sqrt{3}|-|\sqrt{5}-\sqrt{7}|  \right)|$&$\Rightarrow$&$\sqrt{2}+ \sqrt{3}|-|\sqrt{5}-\sqrt{7}$\\\\
\end{tabular}
\end{center}

%--------------------------10----------------------------------
\item Expresar lo siguiente prescindiendo de signos de valor absoluto, tratando por separado distintos casos cuando sea necesario.
\begin{enumerate}[\bfseries (i)]
%(i)
\item $|a+b|-|b|$
\begin{center}
\begin{tabular}{rcrclcrcl}
$a$&si&$a$&$\geq$&$-b$&y&$b$&$\geq$&$0$\\
$-a$&si&$a$&$\leq$&$-b$&y&$b$&$\leq$&$0$\\
$a+2b$&si&$a$&$\geq$&$-b$&y&$b$&$\leq$&$0$\\
$-a-2b$&si&$a$&$\leq$&$-b$&y&$b$&$\geq$&$0$\\
\end{tabular}
\end{center}

%(ii)
\item $|x|-|x^2|$
\begin{center}
\begin{tabular}{r c r c l}
$x-x^2$&si&$x$&$\geq$&$0$\\
$-x-x^2$&si&$x$&$\leq$&$0$\\\\
\end{tabular}
\end{center}

%(iii)
\item $|x|-|x^2|$
\begin{center}
\begin{tabular}{rcl}
$x-x^2$&$si$&$x\leq 0$\\
$-x-x^2$&$si$&$x\geq 0$\\
\end{tabular}
\end{center}

%(iv)
\item $a-|(a-|a|)|$ 
\begin{center}
\begin{tabular}{r c l}
$a$&$si$&$a\leq 0$\\
$3a$&$si$&$a\geq 0$\\
\end{tabular}
\end{center} 
\end{enumerate}

%-------------------------------------11--------------------------------------
\item Encontrar todos los números $x$ para los que se cumple
\begin{enumerate}[\bfseries (i)]
%(i)
\item $|x-3|=8$
\begin{center}
\begin{tabular}{rcccll}
$-8$&$=$&$x-3$&$=$&$8$&teorema 4.1\\
$-5$&$=$&$x$&$=$&$11$&\\\\
\end{tabular}
\end{center}

%(ii)
\item $|x-3|<8$
\begin{center}
\begin{tabular}{rcccll}
$-8$&$<$&$x-3$&$<$&$8$&teorema \\
$-5$&$<$&$x$&$<$&$11$&\\\\
\end{tabular}
\end{center}

%(iii)
\item $|x+4|<2$
\begin{center}
\begin{tabular}{rcccll}
$-2$&$<$&$x+4$&$<$&$-2$&teorema \\
$-6$&$<$&$x$&$<$&$-2$&\\\\
\end{tabular}
\end{center}

%(iv)
\item $|x-1|+|x-2|>1$\\\\
Por definición:
\begin{equation}
|x-1| = \left\lbrace
\begin{array}{rcr}
  x-1& si & x\geq 1\\
 1-x& si & x \leq 1\\\\
\end{array}
\right.
\end{equation}
\begin{equation}
|x-2| = \left\lbrace
\begin{array}{rcr}
  x-2& si & x\geq 2\\
 2-x& si & x \leq 2\\
\end{array}
\right.
\end{equation}
Por lo tanto queda comprobar:\\
\begin{center}
\begin{tabular}{c c c r c l}
Si&$x\leq 1$&$\Rightarrow$&$(1-x)+(2-x)>1$&$\Rightarrow$&$x<1$\\\\
Si&$1\leq x\leq 2$&$\Rightarrow$&$(x-1)+(2-x)>1$&$\Rightarrow$&$1>1$\\\\
Si&$x\geq 2$&$\Rightarrow$&$(x-1)+(x-2)>1$&$\Rightarrow$&$x>2$\\\\
\end{tabular}
\end{center}
Así: $x<1 \; \lor \; x>2$\\\\

%(v)
\item $|x-1|+|x+1|<2$\\\\
Por definición:
\begin{equation}
|x-1| = \left\lbrace
\begin{array}{rcr}
  x-1& si & x\geq 1\\
 1-x& si & x \leq 1\\\\
\end{array}
\right.
\end{equation}
\begin{equation}
|x+1| = \left\lbrace
\begin{array}{rcr}
  x+1& si & x\geq -1\\
 -1-x& si & x \leq -1\\
\end{array}
\right.
\end{equation}
Por lo tanto queda comprobar:\\
\begin{center}
\begin{tabular}{c c c r c l}
Si&$x\leq -1$&$\Rightarrow$&$(1-x)+(1-x)<2$&$\Rightarrow$&$x>-1$\\\\
Si&$-1\leq x \leq 1$&$\Rightarrow$&$(1-x)+(x+1)<2$&$\Rightarrow$&$2<2$\\\\
Si&$x\geq 1$&$\Rightarrow$&$(x-1)+(x+1)<2$&$\Rightarrow$&$x<1$\\\\
\end{tabular}
\end{center}
Pero es falso que $x$ satisface a $-1\leq x \leq 1$, y contradice a que $x$ satisface a todos los reales, por lo tanto no existe solución\\\\ 

%(vi)
\item $|x-1|+|x+1|<1$ \\\\ De la misma manera que el anterior ejercicio no tiene solución para ningún $x$.\\\\

%(vii)
\item $|x-1|\cdot |x+1|=0$\\\\
Por definición:
\begin{equation}
|x-1| = \left\lbrace
\begin{array}{rcr}
  x-1& si & x\geq 1\\
 1-x& si & x \leq 1\\\\
\end{array}
\right.
\end{equation}
\begin{equation}
|x+1| = \left\lbrace
\begin{array}{rcr}
  x+1& si & x\geq -1\\
 -1-x& si & x \leq -1\\
\end{array}
\right.
\end{equation}
queda comprobar:\\
\begin{center}
\begin{tabular}{c c c r c l}
Si&$x\leq -1$&$\Rightarrow$&$(1-x)+(-1-x)=0$&$\Rightarrow$&$x\leq-1 \, \cup \, x = 1 \, \cup \, x=-1 $\\\\
Si&$-1\leq x \leq 1$&$\Rightarrow$&$(1-x)\dot (x+1)=0$&$\Rightarrow$&$-1\leq x \leq 1 \; \cup \; x=1 \; \cup \; x=-1$\\\\
Si&$x\geq 1$&$\Rightarrow$&$(x-1)\dot (x+1)=0$&$\Rightarrow $&$x \notin \mathbb{R}$\\\\
\end{tabular}
\end{center}
Por lo tanto $x=1$ \; ó \; $x=-1$\\\\

%(viii)
\item $|x-1|\cdot |x+2|= 3$\\\\
Si $x>1$ ó $x<-2$, entonces la condición se convierte en $(x-1)(x+2)=3$ ó $x^2+x-5=0$, cuyas soluciones son según la formula general son $\dfrac{-1+\sqrt{21}}{2}$ y $\dfrac{-1-\sqrt{21}}{2}$. Puesto que el primer valor es $x>1$ y el segundo es $x<-2$, ambos son soluciones de $|x-1||x+2|=3.$ Para $-2<x<1,$ la condición se convierte en $(1-x)(x+2)=3$ ó $x^2+x+1=0$, la cual carece de soluciones.\\\\
\end{enumerate}

%--------------------------------12-----------------------------------------
\item Demostrar lo siguiente:
\begin{enumerate}[\bfseries (i)]
%(i)
\item $|xy|=|x|\cdot |y|$\\\\
Demostración.- \; Si $|xy|$ Por teorema  \; $\sqrt{(xy)^2}$ luego por propiedad \; $\sqrt{x^2 \cdot y^2}$, así $\sqrt{x^2}\cdot \sqrt{y^2}$ \; y \;  $|x|\cdot |y|$\\\\

%(ii)
\item $\left| \dfrac{1}{x} \right|=\dfrac{1}{|x|}$\\\\
Demostración.- \; Si $\left| \dfrac{1}{x} \right|$ por definición $\sqrt{(x^{-1})^2}$, después $\left( \dfrac{1}{x}\right) ^{2/2}$, por propiedad  \; $\dfrac{\sqrt{1^2}}{\sqrt{x^2}}$, luego $\dfrac{1}{|x|}$  \\\\ 

%(iii)
\item $\dfrac{|x|}{|y|}=\left| \dfrac{x}{y} \right|$ si $y\neq 0$\\\\
Demostración.- $$ \dfrac{|x|}{|y|} = \dfrac{\sqrt{x^2}}{\sqrt{y^2}}=\dfrac{x^{2/2}}{y^{2/2}}=\left( \dfrac{x}{y} \right)^{2/2} = \sqrt{\left( \dfrac{x}{y} \right)^2}=\left| \dfrac{x}{y} \right| $$ \\\\

%(iv)
\item $|x-y|\leq |x|+|y|$\\\\
Demostración.- \; Sea $(|x-y|)^2\leq( |x|+|y| )^2$, entonces:
\begin{center}
\begin{tabular}{r c l l}
$(|x-y|)^2$&$=$&$(x-y)^2$&\\\\
&$=$&$x^2-2xy+y^2$&\\\\
&$\leq$&$|x|^2+|-2xy|+|y|^2$&Ya que $-2xy\leq |-2xy|$\\\\
&$=$&$|x|^2+|-2||x||y|+|y|^2$&Por teorema\\\\
&$=$&$|x|^2+2|x||y|+|y|^2$&\\\\
&$=$&$(|x|+|y|)^2$&\\\\
\end{tabular}
\end{center}
luego por teorema \; $|x-y|\leq |x|+|y|$\\\\

%(v)
\item $|x|-|y|\leq |x-y|$\\\\
Demostración.- \; Su demostración es parecida al anterior teorema, 
\begin{center}
\begin{tabular}{r c l l}
$(|x|-|y|)^2$&$=$&$|x|^2-2|x||y|+|y|^2$&\\\\
&$\leq$&$x^2-2xy+y^2$& por el contrareciproco de $|2xy|\geq 2xy$\\\\
&$=$&$(x-y)^2$&\\\\
&$=$&$|x-y|^2$&\\\\
\end{tabular}
\end{center}
Por lo tanto $|x|-|y|\leq |x-y|$\\\\

%(vi)
\item $\left| |x|-|y| \right| \leq |x-y|$ (¿ Por qué se sigue esto inmediatamente del anterior teorema ?)\\\\
Demostración.- \;  Sea $\sqrt{(|x|-|y|)^2}$ entonces, $$\sqrt{(|x|-|y|)^2}=\sqrt{(x^2-2|x||y|+y^2}\leq \sqrt{x^2-2xy+y^2}$$ y por definición se tiene $|x-y|$\\\\

%(vii)
\item $|x+y+z| \leq |x|+|y|+|z|$\\\\
Demostración.- \: Sea $\sqrt{(x+y+9)^2}$ entonces,  $$\sqrt{x^2+z^2++y^2+2xy+2xz+2yz} \leq \sqrt{|x|^2+|z|^2++|y|^2+2|x||y|+2|x||z|+2|y||z|}$$ por lo tanto $\sqrt{(|x|+|y|+|z|)^2}$. La igualdad se prueba si $\forall x,y,z \geq 0$ ó $\forall x,y,z \leq 0$\\\\
\end{enumerate}

%------------------------------13-------------------------------------
\item El máximo de dos números $x$ e $y$ se denota por $max(x,y)$. Así $max(-1,3)=max(3,3)$ y $max(-1,-4)=max(-4,-1=-1)$. El mínimo de $x$ e $y$ se denota por $min(x,y)$. Demostrar que:
\begin{enumerate}[\bfseries 1.]
\item $max(x,y)=\dfrac{x+y+|y-x|}{2}$
\item $min(x,y)=\dfrac{x+y-|y-x|}{2}$
\end{enumerate}
Derivar una fórmula para $max(x,y,z)$ y $min(x,y,z),$ utilizando. por ejemplo, $max(x,y,z)=max(x,max(x,y))$ \\\\
Demostración.- \; Por definición de valor absoluto se tiene:
\begin{equation}
|x-y| = \left\lbrace
\begin{array}{crr}
x-y& si, & x\geq y\\
y-x& si, & x \leq y
\end{array}
\right.
\end{equation}
Por lo tanto 
\begin{itemize}
\item $max(x,y) = \dfrac{x+y+|x-y|}{2}= \dfrac{x+y+x-y}{2} =\dfrac{2x}{2}=x$\\
\item $max(x,y) = \dfrac{x+y+|x-y|}{2}= \dfrac{x+y+y-x}{2} =\dfrac{2y}{2}=y$
\end{itemize} 
La demostración es parecido para para $min(x,y)$\\
Se deriva una formula para $mas(x,y,z)=max(x,max(y,z))$ de la siguiente manera\\
$$max(x,max(y,z))=\dfrac{x+\dfrac{y+z+|y-z|}{2} + \left| x - \dfrac{y+z+|y-z|}{2} \right|}{2}$$\\\\ 

%----------------------------------14---------------------------------
\item Demostrar:
\begin{enumerate}[\bfseries (a)]
%(a)
\item Demostrar que $|a|=|-a|$\\\\
Demostración.- \; Si $a\geq 0$, para $|a|^2$ entonces $a^2$, luego $(-a)^2=|-a|^2$, así se demuestra que $|a|=|-a|$. Luego es evidente para $a \leq 0.$\\\\ 

%(b)
\item Demostrar que $-b \leq a \leq b$ si y sólo si $|a|\leq b.$ En particular se sigue que $-|a|\leq a \leq |a|.$\\\\
Demostración.- \; Sea $-a\leq b \land a\leq b$ entonces por definición de valor absoluto $|a|=a\leq b$ si $a\geq 0.$ Y $|a|=-a\leq b$ si $a\leq 0.$\\
Por otro lado si $|a|\leq b,$ entonces es claro que $b\geq 0$. Pero $|a|\leq b$ significa que $a\leq b$ si $a\leq 0$ como también $a\leq b$ si $a\leq 0$. Análogamente $|a|\leq b$ significa que $-a\leq b,$ y en consecuencia $-b\leq a$, si $a\leq 0$ y $-b\leq a,$ si $a\geq 0$, por lo tanto $-b\leq a \leq b$\\\\

%(c)
\item Utilizar este hecho para dar una nueva demostración de $|a+b| \leq |a|+|b|$\\\\
Demostración.- \; Sea $-|a| \leq a \leq |a|$ y $-|b|\leq b \leq |b|$ entonces $-(|a|+|b|) \leq a+b \leq |a|+|b|$, de donde $|a+b| \leq |a|+|b|.$\\\\
\end{enumerate}

%-----------------------------15------------------------------
\item Demostrar que si $x$ e $y$ son $0$ los dos, entonces:\\
\begin{itemize}
\item $x^2+xy+y^2>0$\\\\
Demostración.- \;Sea $(x-y)^2>0$ entonces $x^2+y^2>xy$. Por otro lado  si $x,y \neq 0$ por teorema  \; $x^2+y^2>0$, dado que $x^2+y^2>xy$ entonces se cumple $x^2+y^2+xy>0$.\\\\
\item $x^4+x^3y+x^2y^2+xy^3+y^4$ \\\\
Demostración.- \; Sea $(x^5-y^5)^2>0$, por teorema $\left[ (x-y)(x^4+x^3y+x^2y^2+xy^3+y^4) \right]^2>0$, así $(x-y)^2(x^4+x^3y+x^2y^2+xy^3+y^4)^2>0$, \; $(x^4+x^3y+x^2y^2+xy^3+y^4)^2>0$, por lo tanto $(x^4+x^3y+x^2y^2+xy^3+y^4)>0$ \\\\
\end{itemize}

%---------------------------16-------------------------------
\item 
\begin{enumerate}[\bfseries (a)]
%(a)
\item $(x+y)^2=x^2+y^2$ solamente cuando $x=0$ ó $y=0$\\\\
Demostración.- \; Sea $x=0$ \; y \; $x^2+xy+y^2$ por teorema $0\cdot y = 0$ entonces $x^2+y^2$. Se demuestra de la misma manera para $y=0$\\\\
\item $(x+y)^3=x^3+y^3$ solamente cuando $x=0$ ó $y=0$ ó $x=-y$\\\\
Demostración.- \; Es evidente para $x=0$ é $y=0$. Solo faltaría demostrar para $x=-y$. \\ Si $(x+y)^3=x^3+3x^2y+3xy^2+y^3$ entonces $(x+y)^3=x^3 +3(-y)^2 y+3(-y)y^2 +y^3=x^3+3y^3+3(-y)^3+y^3$, por lo tanto $x^3+y^3$.\\\\ 

%(b)
\item Haciendo uso del hecho que $$x^2+2xy+y^2=(x+y)^2 \geq 0$$
demostrar que el supuesto $4x^2 +6xy+4y^2<0$ lleva una contradicción.\\\\
Demostración.- \; 
\begin{center}
\begin{tabular}{r c l}
$4^2+8xy+4y^2$&$<$&$2xy$\\
$4(x^2+2xy+y^2)$&$<$&$2xy$\\
$x^2+2xy+y^2$&$<$&$xy/2$\\
\end{tabular}
\end{center}
Dado que $2xy<xy/2$ es falso, concluimos que $4x^2 +6xy+4y^2<0$ también es falso y así llegamos a una contradicción.\\\\

%(c)
\item Utilizando la parte $(b)$ decir cuando es $(x+y)^4=x^4+y^4$\\\\
Demostración.- \; Se tiene $(x+y)^2(x+y)^2$, por lo tanto se cumple que $x^4+y^4$, si $x=0$ ó $y=0$\\\\

%(d)
\item Hallar cuando es $(x+y)^5=x^5+y^5$. Ayuda: Partiendo del supuesto $(x+y)^5?x^5+y^5$ tiene que ser posible deducir la ecuación $x^3+2x^2y+y^3=0$, si $xy\neq 0$. Esto implica que $(x+y)^3=x^2y+xy^2=xu(x+y)$.\\
El lector tendría que ser ahora capaz de intuir cuando $(x+y)^n=x^n+y^n$.\\\\
Demostración.- \; Si $x^5+y^5=(x+y)^5=x^5+5x^4y+10x^3y^2+10x^2y^3+5xy^4+y^5$, entonces $0=5x^4y+10x^3y^2+10x^2y^3+5xy^4$ $0=5xy(x^3+2x^2+y+2xy^2+y^3)$. Así $x^3+2x^2+y+2xy^2+y^3=0$.\\
restando esta ecuación de $(x+y)^3=x^3+2x^2y+2xy^2+y^3$ obtenemos, $(x+y)^3=x^2y+xy^2=xy(x+y)$. Así pues, ó bien $x+y=0$ ó $(x+y)^2=xy;$ la última condición implica que $x^2+xy+y^2=0$, con lo que $x=0$ ó $y=0$. por lo tanto $x=0$ ó $y=0$ ó $x=-y$.\\\\
\end{enumerate}

%----------------------------17--------------------------------
\item 
\begin{enumerate}[\bfseries (a)]
%(a)
\item El valor mínimo de $2x^2-2x+4$\\\\
Para poder hallar el valor mínimo debemos llevar la ecuación a su forma canónica es decir, 
\begin{center}
\begin{tabular}{r c l}
$2x^2-3x+4$&=&$2\left( x^2-\dfrac{3}{2}x \right) +4$\\\\
&=&$2\left[ \left( x-\dfrac{3}{4} \right)^2 -\left(\dfrac{3}{4} \right)^2 \right]+4$\\\\
&=&$2\left( x-\dfrac{3}{4} \right)^2-2\left( \dfrac{3}{4} \right)^2+4$\\
\end{tabular}
\end{center}
El mínimo valor posible es $\dfrac{23}{8}$, cuando $\left(x-\dfrac{3}{4}\right)^2=0$ ó $x=\dfrac{3}{4}$\\\\

%(b)
\item El valor mínimo de $x^2-3x+2y^2+4y+2$\\\\
$$x^2-3x+2y^2+4y+2=\left( x- \dfrac{3}{2} \right)^2+2(y+1)^2-\dfrac{9}{4}$$
así el valor mínimo es $-\dfrac{9}{4}$, cuando $x=\dfrac{3}{2}$ y $y=-1$\\\\

%(c)
\item Hallar el valor mínimo de $x^2+4xy+5y^2-4x-6y+7$\\
\begin{center}
\begin{tabular}{r c l}
$x^2+4xy+5y^2-4x-6y+7$&=&$x^2+4(y-1)x+5y^2-6y+7$\\
&=&$[x+2(y-1)]^2+5y^2-6y+7-4(y-1)^2$\\
&=&$[x+2(y-19]^2+(y+1)^2+2$\\
\end{tabular}
\end{center}
Así el valor mínimo es 2, cuando $y=-1$ y $x=-2(y-1)=4$\\\\
\end{enumerate}

%---------------------------------18---------------------------------
\item
\begin{enumerate}[\bfseries (a)]
%(a)
\item Supóngase que $b^2-4c\geq 0$. Demostrar que los números
$$\dfrac{-b+\sqrt{b^2-4c}}{2}, \; \; \; \dfrac{-b- \sqrt{b^2-4c}}{2}$$
satisfacen ambos la ecuación $x^2+bx+c	=0$\\\\
Demostración.- \; Para probar que satisfaga a la ecuación dada, podemos empezar a completar al cuadrado de la siguiente manera: $x^2+bx+\left(\dfrac{b}{2}\right)^2=-c+\left(\dfrac{b}{2}\right)^2$, así $\left( x^2+\dfrac{b}{2} \right)^2=-c+\dfrac{b^2}{4}$. Por existencia de raíz cuadrada de los números reales no negativos $x+\dfrac{b}{2} = \pm \sqrt{\dfrac{b^2-4c}{4}}$, luego $x=\dfrac{-b \pm \sqrt{b^2-4c}}{2}$.\\\\ 

%(b)
\item Supóngase que $b^2-4c<0$. Demostrar que no existe ningún número $x$ que satisfaga $x^2+bx+c=0$; de hecho es $x^2+bx+c>0$ para todo $x$.\\\\
Demostración.- \: Tenemos $$x^2+bx+c=\left( x+\dfrac{b}{2}\right)^2+c-\dfrac{b^2}{4}\geq c- \dfrac{b^2}{4}$$ pero por hipótesis $c-\dfrac{b^2}{4}>0$, así $x^2+bx+c>0$ para todo $x$. \\\\

%(c)
\item Utilizar este hecho para dar otra demostración de que si $x$ e $y$ no son ambos $0$, entonces $x^2+xy+y^2>0$\\\\
Demostración.- Aplicando la parte $b$ con $y$ para $b$ e $y^2$ para $c$, tenemos $b^2-4c=y^2-4y^2<0$ para $x \neq 0$, entonces $x^2+xy+y^2>0$ para todo $x$.\\\\ 
\item ¿Para qué número $\alpha$ se cumple que $x^2+\alpha xy + y^2>0$ siempre que $x$ e $y$ no sean ambos $0$?\\\\
Demostración.- \; $\alpha$ debe satisfacer $(\alpha y)^2-4y^2<0,$ o $\alpha^2<4$, o $|\alpha|<2$\\\\

%(d)
\item Hállese el valor mínimo posible de $x^2+bx+c$ y de $ax^2+bx+c,$ para $a>0$\\\\
Demostración.- \; Por ser $$x^2+bx+c=\left( x+\dfrac{b}{2} \right)^2+ c-\dfrac{b^2}{4}\geq c- \dfrac{b^2}{4},$$ y puesto que $x^2+bx+c$ tiene el valor $c- \dfrac{b^2}{4}$ cuando $x=-\dfrac{b}{2}$, el valor mínimo es $c-\dfrac{b^2}{4}$.\\ Después $$ax^2+bx+c= a \left( x^2+\dfrac{b}{a}x+\dfrac{c}{a} \right),$$ el mínimo es $$a\left( \dfrac{c}{a} - \dfrac{b^2}{4a^2} \right) = c - \dfrac{b^2}{4a}$$\\\\
\end{enumerate}

%----------------------------------19---------------------------------
\item  El hecho de que $a^2 \geq 0$ para todo número $a$, por elemental que pueda parecer, es sin embargo la idea fundamental en que se basan en último instancia la mayor parte de las desigualdades. La primerísima de todas las desigualdades es la desigualdad de Schwarz: $$x_1y_1 + x_2y_2 \leq \sqrt{x_1^2 +x_2^2}\sqrt{y_1^2+y_2^2}.$$ Las tres demostraciones de la desigualdad de Schwarz que se esbozan más abajo tienen solamente una cosa en común: el estar basadas en el hecho de ser $a^2\geq 0$ para todo $a$.
\begin{enumerate}[\bfseries (a)]

%(a)
\item Demostrar que si $x_1=\lambda y_1$ \; y \; y $x_2=\lambda y_2$ para algún número $\lambda$, entonces vale el signo igual en la desigualdad de Schwarz. Demuéstrese lo mismo en el supuesto $y_1=y_2=0$: supóngase ahora que $y_1$ e $y_2$ no son ambos $0$ y que no existe ningún número $\lambda$ tal que $x_1=\lambda y_1$ \; y \; $x_2=\lambda y_2.$ Entonces
\begin{center}
\begin{tabular}{r c l}
$0$&$<$&$(\lambda y_1-x_1)^2+(\lambda y_2 -x_2)^2$\\\\
&$=$&$\lambda^2(y_1^2+y_2^2)-2\lambda(x_1 y_1 + x_2 y_2)+(x_1^2 +x_2^2)$\\
\end{tabular}
\end{center}
Utilizando el teorema anterior, completar la demostración de la desigualdad de Schwarz.\\\\
Demostración.- \; Primero, Si $x_1=\lambda y_1$ \; y \; $x_2=\lambda y_2.$, entonces remplazando en la desigualdad de Schwarz, $\lambda \cdot (y_1)^2 + \lambda \cdot (y_2)^2=\sqrt{(\lambda y_1)^2+(\lambda y_2)^2}\sqrt{y_1^2+y_2^2}$ luego por propiedades de raíz se cumple  $$\lambda(y_1^2+y_2^2) = \sqrt{\left[\lambda(y_1^2+y_2^2)\right]^2}$$
Vemos que también se cumple la igualdad para $y_1=y_2=0$ .\\
Por último Si un tal $\lambda$ no existe, entonces la ecuación carece de solución en $\lambda$, de modo que por el teorema 1.25 tenemos, $$\left[ \dfrac{2(x_1 y_1 + x_2 y_2)}{y_1^2 + y_2^2}  \right]^2 - \dfrac{4(x_1^2 +y_1^2)}{y_1^2 + y_2^2}<0$$ lo cual proporciona la desigualdad de Schwartz.\\\\

%(b)
\item Demostrar la desigualdad de Schwarz haciendo uso de $2xy\leq x^2+y^2$(¿Cómo se deduce esto?) con $$x=\dfrac{x_i}{\sqrt{x_1^2 + x_2^2}}, \; \; \; y = \dfrac{y_i}{\sqrt{y_1^2+y_2^2}}$$ primero para $i=1$ y después para $i=2$.\\\\
Demostración.- \;  En vista de que $(x-y)^2\geq 0$, tenemos $2xy\leq x^2+y^2$. Realizando el respectivo remplazo tenemos:
\begin{enumerate}[\bfseries 1)]
\item $$2\dfrac{x_1}{\sqrt{x_1^2 + x_2^2}}\cdot \dfrac{y_1}{\sqrt{y_1^2+y_2^2}}\leq \dfrac{x_1^2}{\sqrt{x_1^2 + x_2^2}} + \dfrac{y_1^2}{\sqrt{y_1^2+y_2^2}}$$
\item $$2\dfrac{x_2}{\sqrt{x_1^2 + x_2^2}}\cdot \dfrac{y_2}{\sqrt{y_1^2+y_2^2}}\leq \dfrac{x_2^2}{\sqrt{x_2^2 + x_2^2}} + \dfrac{y_1^2}{\sqrt{y_1^2+y_2^2}}$$
\end{enumerate}
Luego sumando $1)$ y $2)$ $$2\dfrac{x_1}{\sqrt{x_1^2 + x_2^2}}\cdot \dfrac{y_1}{\sqrt{y_1^2+y_2^2}}+2\dfrac{x_2}{\sqrt{x_1^2 + x_2^2}}\cdot \dfrac{y_2}{\sqrt{y_1^2+y_2^2}}\leq \dfrac{x_1^2}{\sqrt{x_1^2 + x_2^2}} + \dfrac{y_1^2}{\sqrt{y_1^2+y_2^2}}+\dfrac{x_2^2}{\sqrt{x_2^2 + x_2^2}} + \dfrac{y_1^2}{\sqrt{y_1^2+y_2^2}}$$ nos queda $$\dfrac{2(x_1y_1+x_2y_2)}{\sqrt{x_1^2+x_2^2}\sqrt{x_2^2+y_2^2}}\leq 2$$\\\\

%(c)
\item Demostrar la desigualdad de Schwarz demostrando primero que $$(x_1^2 + x_2^2)(y_1^2 + y_2^2)=(x_1 y_1 + x_2 y_2)^2 + (x_1 y_2 - x_2 y_1)^2$$
Demostración.- \; Es fácil ver que la igualdad se cumple, $$(x_1^2 + x_2^2)(y_1^2 + y_2^2)=(x_1 y_1)^2 +2x_1 y_1 x_2 y_2 + (x_2 y_2 ) ^2 + (x_1 y_2)^2 -2x_1 y_1 x_2 y_2 + (x_2 y_1)^2=(x_1 y_1 + x_2 y_2)^2 + (x_1 y_2 - x_2 y_1)^2$$
Ya que $(x_1 y_2 - x_2 y_1)^2\geq 0$ entonces, $$(x_1^2 + x_2^2)(y_1^2 + y_2^2)\geq (x_1 y_1 + x_2 y_2)^2$$\\\\

%(d)
\item Deducir de cada una de estas tres demostraciones que la igualdad se cumple solamente cuando $y_1 = y_2 = 0$ ó cuando existe un número $\lambda$ tal que $x_1=\lambda y_1$\; y \; $x_2= \lambda y_2$\\\\
Demostración.- \; La parte $a)$ ya prueba el resultado deseado.\\
En la parte $b)$ la igualdad se mantiene sólo si se cumple en $(1)$ y $(2)$. Sea $2xy=x^2+y^2$ sólo cuando $(x-y)^2=0$ es decir $x=y$ esto significa $$\dfrac{x_i}{\sqrt{x_1^2 + x_2^2}}= \dfrac{y_i}{\sqrt{y_1^2+y_2^2}} \; ; \; para \; x=1,2$$ para que podamos elegir $\lambda = \sqrt{x_1^2 + x_2^2} / \sqrt{y_1^2 + y_2^2}$.\\
En la parte $(c)$, la igualdad se cumple solamente cuando $(x_1 y_2 - x_2 y_1)^2 \geq 0$. Una posibilidad es $y_1 = y_2 = 0$. Si $y\leq 0$, entonces $x_l = (x_1 y_1)y_1$ \; y también $x_2 = (x_1 / y_1)y_1$ análogamente, si $y_2\leq 0$, entonces $\lambda = x2/y2$.\\\\
\end{enumerate}

%-------------------------------------20----------------------------------------
\item Demostrar que si $$|x - x_0|<\dfrac{\epsilon}{2} \hspace{0.5cm} y \hspace{0.5cm} |y - y_0|<\dfrac{e}{2}$$ entonces $$|(x+y)-(x_o + y_0)|<\epsilon,$$ $$|(x-y)-(x_o - y_0)|<\epsilon.$$\\
Demostración.- \; primeramente si $|(x+y)-(x_o + y_0)|= |(x-x_0)+(y-y_0)|$, por desigualdad triangular e hipótesis, $$ |(x-x_0)+(y-y_0)|\leq |x - x_0| + |y - y_0| <\frac{\epsilon}{2} + \frac{\epsilon}{2} =\epsilon$$
Demostramos de similar manera y por teorema 1.7, $$|(x-y)-(x_o - y_0)| = |(x - x_0)-(y - y_0)| \leq |x - x_0| + |y - y_0|< \frac{\epsilon}{2} + \frac{\epsilon}{2}=\epsilon$$\\

%---------------------------------------21---------------------------------------
\item Demostrar que si $$|x-x_0|< min \left( 1,\dfrac{\epsilon}{2(|y_0|+1)} \right) \hspace{0.5cm} y \hspace{0.5cm} |y-y_0|< \dfrac{\epsilon}{2(x_0+1)}$$ entonces $|xy - x_0 y_0 |< \epsilon.$\\\\
La primera igualdad de la hipótesis significa precisamente que: $$|x-x_0|<1 \hspace{0.5cm} y \hspace{0.5cm} |x - x_0| \dfrac{\epsilon}{2(|y_0|+1)}$$\\\\
Demostración.- \; puesto que $|x-x_0| < 1$ se tiene $$|x|-|x_0| \leq |x-x_0| < 1$$ de modo que $$|x| < 1 + |x_0|$$ Así pues 
\begin{center}
\begin{tabular}{r c l l}
$|xy - x_0 y_0|$&=&$| xy -xy_0 + xy_0 -x_0 y_0 |$&\\\\
&=&$|x(y - y_0) + y_0(x - x_0)|$&\\\\
&$\leq$&$|x| \cdot |y - y_0| + |y_0| \cdot |x - x_0|$&\\\\
&$<$&$(1+|x_0|) \cdot \dfrac{\epsilon}{2(|x_0|+1|)} +  |y_0| \cdot \dfrac{\epsilon}{2(|y_0|+1|)}$&ya que $ |y_0|\dfrac{\epsilon}{2(|y_0|+ 1|)} < |y_0| \dfrac{\epsilon}{2|y_0|}$ para $|y_0|\neq 0$\\\\
&$\leq$&$\dfrac{\epsilon}{2} + \dfrac{\epsilon}{2}$& ó $|y_0|\dfrac{\epsilon}{2(|y_0|+ 1|)} = \dfrac{\epsilon}{2}$ si $|y_0|=0$ \\\\
&$=$&$\epsilon$&\\\\
\end{tabular}
\end{center}

%---------------------------------------22-------------------------------------
\item Demostrar que si $y_0 \neq 0$ y $$|y - y_0|< min\left( \dfrac{|y_0|}{2}, \dfrac{\epsilon |y_0|^2}{2} \right),$$ 
entonces $y \neq 0$ y $$\left| \dfrac{1}{y} - \dfrac{1}{|y_0|} \right|$$\\\\
Demostración.- \; Se tiene $$|y_0| - |y| < |y - y_0| < \dfrac{y_0}{2},$$ de modo que $|y|<\dfrac{|y_0|}{2}$. En particular, $y \neq 0,$ y $$\dfrac{1}{|y|} < \dfrac{2}{y_0}.$$ Así pues $$\left| \dfrac{1}{y} - \dfrac{1}{y_0} \right| = \dfrac{|y_0 - y|}{|y| \cdot |y_0|} < \dfrac{2}{y_0} \cdot \dfrac{1}{|y_0|} \cdot \dfrac{\epsilon |y_o|^2}{2} = \epsilon$$\\\\

%--------------------------------------23-----------------------------------------
\item Sustituir los interrogantes del siguiente enunciado por expresiones que encierren $\epsilon, \; x_0 \; e \; y_0$ de tal manera que la conclusión sea válida:\\
Si $y_0$ y $$|y - y_0|< ? \hspace{0.5cm} y \hspace{0.5cm} |x - x_0|< ?$$ entonces $y \neq 0$ y $$\left| \dfrac{x}{y} - \dfrac{x_0}{y_0} \right|< \epsilon$$ \\
Sea $|x \dfrac{1}{y} - x_0 \dfrac{1}{y_0}|< \epsilon$ entonce $|x \cdot y^{-1} - x_o \cdot y_0^{-1}| < \epsilon$ por teorema 4.14 $$|x- x_0| < min \left( 1, \dfrac{\epsilon}{2(|y_0^{-1}|+1)} \right) \hspace{0.5cm} y \hspace{0.5cm} |y^{-1} - y_0^{-1}| < \dfrac{\epsilon}{2(|x_0|+1)}  $$ luego por teorema 4.15 $$|y - y_0| < min \left( \dfrac{\dfrac{\epsilon}{2 (|x_o|)+1} \cdot |y_0|^2}{2} \right) = \min \left( \dfrac{\epsilon \cdot |y_0|^2}{4(|x_o|+1)} \right)$$ \\\\

%--------------------------------------24-------------------------------------------
\item Este problema hace ver que la colocación de los paréntesis en una suma es irrelevante. Las demostraciones utilizan la $"$La inducción matemática$"$; si no se está familiarizado con este tipo de demostraciones, pero a pesar de todo se quiere tratar este problema, se puede esperar hasta haber visto el capítulo 2, en el que se explican las demostraciones por inducción. Convengamos, para fijar ideas que $a_1 + ... + a_n$ denota $$a_1 + (a_2(a_3+...+(a_{n-2}+(a_{n-1}+ a_n)))...)$$
Así $a_1+a_2+a_3$ denota $a_1(a_2+a_3)$, y $a_1+a_2+a_3+a_4$ denota $a_1(a_2+(a_3+a_4)),$ etc.
\begin{enumerate}[\bfseries (a)]
%(a)
\item Demostrar que $$(a_1+...+a_k)+a_{k+1}=a_1+...+a_{k+1}$$\\
Demostración.- \; Sea $k=1$ entonces $a_1+a_2=a_1+a_2$. Si la ecuación se cumple para $k$ entonces 
\begin{center}
\begin{tabular}{r c l}
$(a_1+...+a_{k+1})+a_{k+2}$&$=$&$[(a_1+...+a_k)+a_{k+1}]+a_{k+2}$\\
&$=$&$(a_1+...+a_k)+(a_{k+1}+a_{k+2})$\\
&$=$&$a_1+...+a_k + (a_{k+1}+a_{k+2})$\\
&$=$&$a_1+...+a_{k+2}$\\\\
\end{tabular}
\end{center}

%(b)
\item Demostrar que si $n\geq k,$ entonces $$(a_1+...+a_k) + (a_{k+1} + ... + a_n) = a_1 + ... + a_n$$
Demostración.- \; Para $k=1$ la ecuación se reduce a la definición de $a_1+...+a_k$. Si la ecuación se cumple para $k<n$ entonces,
\begin{center}
\begin{tabular}{r c l l}
$(a_1+...+a_{k+1})+(a_{k+2}+...+a_n)$&$=$&$([a_1+...+a_k]+a_{k+1}) + (a_{k+2}+...+a_n)$&parte $(a)$\\
&$=$&$(a_1+...+a_k)+(a_{k+1}+(a_{k+2}+...+a_n))$&por propiedad \\
&$=$&$(a_1+...+a_k)+(a_{k+1}+...+a_n)$&por definición\\
&$=$&$a_1+...+a_n$&hipótesis\\\\
\end{tabular}
\end{center}

%(c)
\item Sea $s(a_1,...,a_k)$ una suma formada con $a_1,...,a_k$. Demostrar que $$s(a_1,...,a_k) = a_1+...+a_k$$\\
Demostración.-\; El aserto es claro para $k=1$. Supóngase que se cumple para todo $l<k$, entonces
\begin{center}
\begin{tabular}{rcll}
$s(a_1,...,a_k)$&$=$&$s^{'}(a_1,...,a_l)+ s^{''}(a_{l+1},...,a_k)$&\\
&$=$&$(a_1+...+a_l)+(a_{l+1})+(a_{l+1}+...+a_k)$&hipótesis\\
&$=$&$a_1+...+a_k$&parte $(b)$\\\\
\end{tabular}
\end{center}
\end{enumerate}

%------------------------------------------25-----------------------------------------
\item Supóngase que por número se entiende sólo el $0$ ó el $1$ y que $+$ y $\cdot$ son las operaciones definidas mediante las siguiente tablas.
\begin{multicols}{2}
\begin{center}
\begin{tabular}{c c c }
+&0&1\\
0&0&1\\
1&1&0\\
\end{tabular}
\end{center}

\begin{center}
\begin{tabular}{c c c }
&0&1\\
0&0&0\\
1&0&1\\
\end{tabular}
\end{center}  
\end{multicols}
Comprobar que se cumplen las propiedades P1-P2, aunque 1+1=0\\\\
P2, P3, P4, P5, P6, P7, P8 resultan evidentes sin más que observar las tablas. Se presentan ocho casos para $P1$ y este número puede incluso reducirse: al cumplirse P2, resulta claro que $a+(n/b+c)=(a+b)+c$ si $a,b$ ó $c$ es $0,$ de modo que bastará comprobar el caso $a=b=c=1$. Una observación análoga puede hacerse para $P5$. Finalmente, P9 se cumple para $a=0$, ya que $0\cdot b = 0$ para todo $b,$ y para $a=1,$ ya que $1\cdot b = b$ para todo $b$.\\\\

\end{enumerate}


    %---------- distintas clases de números
	%\chapter{Distintas clases de números}
  \section{Problemas}
    \begin{enumerate}[\bfseries 1.]
      %---------------------------------------1-------------------------------------
      \item Demostrar por inducción las siguientes fórmulas: 
        %------------------------(a)-----------------------------
        \begin{enumerate}[\bfseries (i)]
          \item $1^2+...+n^2=\dfrac{n(n+1)(2n+1)}{6}$\\\\
          Demostración.- \; Sea $n=k$: $$1^2+...+k^2=\dfrac{k(k+1)(2k+1)}{6},$$ Para $k=1$, $$1^2=\dfrac{1(1+2)(2+1)}{6}$$ por lo tanto se cumple para $k=1$, Luego para $k=k+1$, $$1^2+...+(k+1)^2=\dfrac{(k+1)(k+2)(2k+3)}{6},$$ así cabe demostrar que:
            \begin{center}
              \begin{tabular}{r c l}
              $\dfrac{k(k+1)(2k+1)}{6}+(k+1)^2$&=&$\dfrac{(k+1)(k+2)(2k+3)}{6}$\\\\
              $\dfrac{2k^3+k^2+2k^2+k+6k^2+12k+6}{6}$&=&$\dfrac{2k^3+3k^2+6k^2+9k+4k+6}{6}$\\\\
              $\dfrac{2k^3+9k^2+13k+6}{6}$&=&$\dfrac{2k^3+9k^2+13k+6}{6}$\\\\
              \end{tabular}
            \end{center}

          %-----------------------(b)---------------------------------
          \item $1^3 + ... + n^3 = (1+...+n)^2$\\\\
          Demostración.- \; Sea $n=1$ entonces la igualdad es verdadera ya que $1^3 = 1^2$.Supongamos que se cumple para algún número $k \in \mathbb{Z}^+$,
          $$1^3 + ... + k^3 = (1+...+k)^2,$$ Luego suponemos que se cumple para $k+1$, $$1^3 + ... + k^3 + (k+1)^3 = (1+...+(k+1))^2$$
          Así solo falta demostrar que 
            \begin{center}
              \begin{tabular}{rcl}
              $(1+...+(k+1))^2$&$=$&$(1+...+k)^2 + 2(1+...+k)(k+1) + (k+1)^2$\\\\
              &$=$&$(1^2 + ... + k)^2 + 2\dfrac{k(k+1)}{2} (k+1) + (k+1)^2$\\\\
              &$=$&$1^3 + ... + k^3 + (k^3 + 2k^2 + k) + (k^2 + 2k +1)$\\\\
              &$=$&$1^3 + ... + k^3 + (k+1)^3$\\\\
              \end{tabular}
            \end{center}
          Por lo tanto es válido para cualquier $n \in \mathbb{Z}^+$\\\\
        \end{enumerate}

      %-----------------------------------------2-----------------------------------------
     \item Encontrar una fórmula para 
	 \begin{enumerate}[\bfseries (i)]
	%-----------------------(i)---------------------------
	 \item  $\displaystyle\sum_{i=1}^{n} (2i-1) = 1 +3 +5 + ... + (2n-1)$\\
            \begin{center}
              \begin{tabular}{r c c c l}
              1&=&1&=&$1^2$\\
              1+3&=&4&=&$2^2$\\
              1+3+5&=&9&=&$3^2$\\
              1+3+5+7&=&16&=&$4^2$\\
              1+3+5+7+9&=&25&=&$5^2$\\
              \end{tabular}
            \end{center}
          Por lo tanto $ \displaystyle\sum_{i=1}^{n} (2i-1) = 1 +3 +5 + ... + (2n-1) = n^2$\\\\

          %------------------------(ii)---------------------------
          \item $\displaystyle\sum_{i=1}^{n} (2i-1)^2 = 1^2 + 3^2 + 5^2 + ... + (2n-1)^2$
            \begin{center}
              \begin{tabular}{r c l l}
              $1^2 + 3^2 + 5^2 + ... + (2n-1)^2$&$=$&$\left[ 1^2 +2^2 +...+(2n)^2 \right] - \left[ 2^2 + 4^2 +6^2 +...+ (2n)^2\right]$&\\\\
              &$=$&$\left[ 1^2 + 2^2 + ...+ (2n)^2 \right] - 4\left[ 1^2 + 2^2 +3^2 + ... + n^2 \right]$&\\\\
              &$=$&$\dfrac{2n(2n+1)(4n+1)}{6} -\dfrac{4n(n+1)(2n+1)}{6}$&\\\\
              &$=$&$\dfrac{2n(2n+1)\left[ 4n+1 -2 (n+1) \right]}{6}$&\\\\
              &$=$&$\dfrac{2n(2n+1)(2n-1)}{6}$&\\\\ 
              &$=$&$\dfrac{n(2n+1)(2n-1)}{3}$&\\\\
              \end{tabular}
            \end{center}
        \end{enumerate}

      %---------------------------------------------3------------------------------------------
        \begin{def.}[Coeficiente Binomial]
          Si $0 \leq k \leq n,$ se define el coeficiente binomial $ {n \choose k} $ por $${n \choose k} = \dfrac{n!}{k!(n-k)!}=\dfrac{n(n-1)...(n - k + 1)}{k!}, \; si \; k \neq 0, \; n$$ $${n \choose 0} = {n \choose n} = 1.$$ Esto se convierte en un caso particular de la primera fórmula si se define $0! = 1.$
        \end{def.}

      \item 
        \begin{enumerate}[\bfseries (a)]
          %------------------------(a)---------------------------
          \item Demostrar que $${n +1 \choose k} = {n \choose k - 1} + {n \choose k}$$ Esta relación de lugar a la siguiente configuración, conocida por triángulo de Pascal: Todo número que no esté sobre uno de los lados es la suma de los dos números que tiene encima: El coeficiente binomial ${n \choose k}$ es el número k-ésimo de la fila $(n+1)$.
            \begin{center}
              \begin{tabular}{ccccccccccc}
                &    &    &    &    &  1 &    &    &    &    &   \\
                &    &    &    &  1 &    &  1 &    &    &    &   \\
                &    &    &  1 &    &  2 &    &  1 &    &    &   \\
                &    &  1 &    &  3 &    &  3 &    &  1 &    &   \\
                &  1 &    &  4 &    &  6 &    &  4 &    &  1 &   \\
              1 &    &  5 &    & 10 &    & 10 &    &  5 &    & 1 \\\\
              \end{tabular}
            \end{center}
          Demostración.- \; \\
            \begin{center}
              \begin{tabular}{r c l}
                $ {n \choose k-1}  +  {n \choose k} $&$=$&$\dfrac{n!}{(k-1)(n-k+1)!}+ \dfrac{n!}{k!(n-k)!}$\\\\
                &$=$&$\dfrac{kn!}{k!(n+1-k)!} + \dfrac{(n+1-k)n!}{k!(n+1-k)!}$\\\\
                &$=$&$\dfrac{(n+1)n!}{k!(n+1-k)!}$\\\\
                &$=$&$ {n+1 \choose k} $\\\\
              \end{tabular}
            \end{center}

          %------------------------(b)---------------------------
          \item Obsérvese que todos los números del triángulo de Pascal son números naturales. Utilícese la parte $(a)$ para demostrar por inducción que $ {n \choose k}$ es siempre un número natural.\\\\
          Demostración.- \; Se ve claramente que ${1 \choose 1}$ es un número natural. Supóngase que ${n \choose p}$ es un número natural para todo $p \leq n$. Al ser: $${ n+1 \choose p } = {n \choose p-1} + {n \choose p} \; para \; p \leq n,$$ se sigue que ${n+1 \choose p}$ es un número natural para todo $p \leq n,$ mientras que ${n+1 \choose n+1}$ es también un número natural. Así pues, ${n+1 \choose p}$ es un número natural para todo $p \leq n+1$\\\\

          %------------------------(c)---------------------------
          \item Dése otra demostración de que ${n \choose k}$ es un número natural, demostrando que ${n \choose k}$ es el número de conjuntos de exactamente $k$ enteros elegidos cada uno entre $1,...,n$.\\\\
          Demostración.- \; Existen $n(n -1) \cdot ... \cdot (n - k + 1)$ k-tuplas de enteros distintos elegidos entre $1, ..., n$, ya que el primero puede ser elegido de n maneras, el segundo de $n - 1$ maneras, etc. Ahora bien, cada conjunto formado exactamente por $k$ enteros distintos, da lugar a $k!$ k-tuplas, de
          modo que el número de conjuntos será $n(n — 1) \cdot ... \cdot (n - k + l)/k! = {n\choose k}$\\\\ 

          %------------------------(d)---------------------------
          \item Demostrar el \textbf{TEOREMA DEL BINOMIO}: Si $a$ \; y \; $b$ son números cualesquiera, entonces
          $$(a+b)^n = a^n + {n \choose 1} a^{n-1} b + {n \choose 2} a^{n-2} b^2 + ... + {n \choose n-1} a b^{n-1} + b^n = \displaystyle \sum_{j=0}^n {n \choose j} a^{n-j} b^j.$$\\\\
          El teorema del binomio resulta claro para $n=1.$ Sea algún número $k \in \mathbb{Z}^+$  
          $$(a+b)^k =  \sum\limits_{j=0}^n {n \choose j} a^{n-j} b^j.$$
          de donde  suponemos que se cumple  para $k+1$ por lo tanto $$(a+b)^k =  \sum\limits_{j=0}^{n+1} {n+1 \choose j} a^{n+1-j} b^j.$$
          entonces,
            \begin{center}
              \begin{tabular}{r c l l}
                $(a+b)^{n+1}$&=&$(a+b)(a+b)^n$&\\\\
                &=&$(a+b) \displaystyle \sum_{j=0}^n {n \choose j} a^{n-j} b^j$&\\\\
                &=&$\displaystyle \sum_{j=0}^n {n \choose j} a^{n+1-j} b^j + \sum_{j=0}^{n} {n \choose j} a^{n-j} b^{j+1}$&\\\\
                &=&$\displaystyle \sum_{j=0}^n {n \choose j} a^{n+1-j} b^j + \sum_{j=0}^{n+1} {n \choose j-1} a^{n+1-j} b^{j}$&\\\\
                &=&$\displaystyle \sum_{j=0}^{n+1} {n+1 \choose j} a^{n+1-j} b^j$& por la parte $(a)$\\\\
              \end{tabular}
            \end{center}
          con lo que el teorema del binomio es válido para $n+1.$ y por lo tanto para $n \in  \mathbb{Z}^+$\\\\

          %------------------------(e)---------------------------
          \item Demostrar que 
          \begin{enumerate}[\bfseries (i)]
            %---------------(i)-----------------
            \item $\displaystyle\sum_{j=0}^{n} {n \choose j} = {n \choose 0} + ... + {n \choose n} = 2^n$\\\\
            Demostración.- \; Por el teorema del binomio $2^n=(1+1)^n = \sum\limits_{j=0}^n {n \choose j}(1^j)(1^{n-j})=\sum_{j=0}^n {n \choose j}$\\\\

            %---------------(ii)-----------------
            \item $\displaystyle\sum_{j=0}^n (-1)^j {n \choose j} = {n \choose 0}- {n \choose 1}+...\pm {n \choose n} =0$\\\\
            Demostración.- \;  De igual manera por el teorema del binomio $0=(1+(-1))^n = \sum\limits_{j=0}^n(-1)^j {n \choose j}$\\\\ 

            %---------------(iii)-----------------
            \item $\displaystyle\sum_{l \; impar} {n \choose l} = {n \choose 1} + {n \choose 3}+ ... = 2^{n-1}$\\\\
            Demostración.- \;  Restando $(ii)$ de $(i)$ se tiene que,
              \begin{center}
                \begin{tabular}{rcl}
                  $2 {n \choose 0} + 2 {n \choose 2} + ... + 2{n \choose n}$&$=$&$2^n - 0$\\\\
                  $2\sum\limits_{j \; impar}^n {n \choose j} $&$=$&$2^n$\\\\ 
                  $\sum\limits_{j \; impar}^n {n \choose j}$&$=$&$2^n \cdot 2^{-1}$\\\\ 
                  $\sum\limits_{j \; impar}^n {n \choose j}$&$=$&$2^{n-1}$\\\\
                \end{tabular}
              \end{center}

            %---------------(iv)-----------------
            \item $\displaystyle\sum_{l \; par} {n \choose l} = {n \choose 0} + {n \choose 2} + ...  = 2^{n-1}$\\\\
            Demostración.- \; La demostración es similar al problema anterior pero esta vez sumamos $(i)$ en $(ii)$ \\\\ 
            \end{enumerate}
          \end{enumerate}

      %---------------------------------------------4------------------------------------------
      \item 
        \begin{enumerate}[\bfseries (a)]
        %------------------------(a)---------------------------
        \item Demostrar que $$\displaystyle\sum_{k=0}^{l} {n \choose k} {m \choose l-k} = {n+m \choose l}$$\\\\
        Demostración.- \; La multiplicación de series formales de potencias se realiza recolectando los términos con las mismas potencias de $x$: 
        $$\left( \sum\limits_{a_K}^{\infty} a_k x^k \right) \left( \sum\limits_{a_K}^{\infty} b_k x^k \right) = \sum\limits_{k=0}^{\infty} \left(\sum\limits_{j=0}^{k} a_j x^j b_{k-j} x^{k-j}  \right) = \sum\limits_{k=0}^{\infty} \left(\sum\limits_{j=0}^{k} a_j  b_{k-j}   \right) x^k$$
        Tenga en cuenta que los subíndices en la suma interna suman $k$, la potencia de $x$ en la suma externa.
        Luego aplicando la multiplicación de series de potencias a $(1+x)^m = \sum\limits_{k=0}^{\infty} {m \choose k}x^k$ y $(1+x)^n = \sum\limits_{k=0}^{\infty} {n \choose k}x^k$ de donde $$(1+x)^{m+n} = \sum\limits_{k=0}^{\infty} {m+n \choose k }x^k$$
        (Los índices en las sumas a $\infty$ ya que para $k>n$, ${n  \choose k} =0$), Se tiene:
        $$(1+x)^m (1+x)^n = \sum\limits_{k=0}^{\infty} \left[ \sum\limits_{j=0}^{k} {m \choose j} {n \choose k-j} \right]x^k$$
        ya que $(1+x)^m (1+x)^n = (1+x)^{m+n}$ entonces $${m+n \choose k} = \sum\limits_{j=0}^{k} {m \choose j} {n \choose k-j}$$\\ 
        

        %------------------------(b)---------------------------
        \item demostrar que $$\displaystyle\sum_{k=0}^{n} {n \choose k}^2 = {2n \choose n}$$\\\\
        Demostración.- \; Sea $m=n$, $l=n$ en la parte $(a)$ y notar que ${n \choose k} = {n \choose n-k}.$ de donde se tiene que $$ \sum\limits_{k=0}^{n} {n \choose k} \cdot  {n \choose k} = {n+n \choose n}$$ y por lo tanto $$\sum\limits_{k=0}^{n} {n \choose k}^2 = {2n \choose n}$$\\\\  
        
        \end{enumerate}

      %---------------------------------------------5------------------------------------------
     \item \begin{enumerate}[\bfseries (a)] 

     ------------------------(a)---------------------------
      \item Demostrar por inducción sobre $n$ que $$1 + r +r^2 + ... + r^n = \dfrac{1 - r^{n+1}}{1-r}$$ si $r\neq 1$ (Si es $r=1$, el cálculo de la suma no presenta problema alguno).\\\\
      Demostración.- \; Sea $n=1$ entonces $$1+r = \dfrac{1- r^2}{1-r}$$ el cual vemos que se cumple.\\
      Luego
      \begin{center}
      \begin{tabular}{r c l}
      $1+r+r^2 + ... + r^n + r^{n+1}$&$=$&$\dfrac{1- r^{r+1}}{1-r} + r^{r+1}$\\\\
      &$=$&$\dfrac{1 - r^{n+1} + r^{n+1} (1-r)}{1-r}$\\\\
      &$=$&$\dfrac{1 - r^{n+1}}{1-r}$\\\\
      \end{tabular}
      \end{center} 

      %------------------------(b)----------------------------
      \item Deducir este resultado poniendo $S=1+r+...+r^n$, multiplicando esta ecuación por $r$ y despejando $S$ entre las dos ecuaciones.\\\\
      Tenemos $r\cdot S = r + .... + r^n + r^{n+1}$, luego sea $S - rS$ entonces $S(1-r) = 1-r^{n+1}$ por lo tanto $S = \dfrac{1-r^{n+1}}{1-r}$\\\\
      \end{enumerate}

      %---------------------------------------------6------------------------------------------
      \item La fórmula para $1^2 + 2^2 + ... + n^2$ se puede obtener como sigue: Empezamos con la fórmula $$(k+1)^3 - k^3 = 3k^2 + 3k +1$$
      particularmente esta fórmula para $k=1,...,n$ \; y sumando, obtenemos
	 \begin{center}
	    \begin{tabular}{r c l}
	       $2^3 - 1^3$&=&$3\cdot 1^2 + 3 \cdot 1 +1$\\
	       $3^3 - 2^3$&=&$2^2 + 3\cdot 2 +1 $\\
	       $.$&=&\\
	       $.$&=&\\
	       $.$&=&\\
	       $(n+1)^3 - n^3$&=&$n^2 + 3 \cdot n + 1$\\
	       \hline
	       $(n+1)^3 - 1$&=&$3 \left[ 1^2 + ... + n^2 \right] + 3 \left[ 1 + .... + n \right] + n$\\
	    \end{tabular}
	 \end{center}
      De este modo podemos obtener $\sum\limits_{k=1}^n k^2$ una vez conocido $\sum\limits_{k=1}^n k $ (lo cual puede obtenerse mediante un procedimiento análogo). Aplíquese este método para obtener.
	 \begin{enumerate}[\bfseries (i)]
	 %-------------------------(i)--------------------------
	 \item $1^3 + ... + n^3$\\\\
	    Sea $(k+1)^4 = k^4 + 4k^3 + 6k^2 + 4k + 1^4$ entonces $$(k+1)^4 - k^4 = 4k^3 + 6k^2 +4k + 1, \; \; para \; k=1,...,n $$ por hipótesis tenemos $(n+1)^4 - 1 = 4 \sum\limits_{k=1}^n k^2 + 6 \sum\limits_{k=1}^n k^2 + 4 \sum\limits_{k=1}^n k + n,$ de modo que $$\sum\limits_{k=1}^n k^3 = \dfrac{ (n+1)^4 -1 - 6 \dfrac{n(n+1)(2n+1)}{6} - 4 \dfrac{n(n+1)}{2} - n}{4} = \dfrac{n^4}{4} + \dfrac{n^3}{2} + \dfrac{n^2}{4}$$\\\\

	 %------------------------(ii)---------------------------
	 \item $1^4 + ... + n^4$\\\\
	 Similar al anterior ejercicio partimos de $(k+1)^5 - k^5 = 5k^4 + 10k^3 + 10k^2 + 5k + 1 \; \; \; k=1,...,n$ para obtener $(k+1)^5 - k^5 = 5 \left( \sum\limits_{k=1}^n k^4 \right) + 10 \left( \sum\limits_{k=1}^n k^3 \right) + 10 \left( \sum\limits_{k=1}^n k^2 \right) + 5 \left( \sum\limits_{k=1}^n k \right) + n,$ así $$\sum\limits_{k=1}^n k^4 = \dfrac{(n+1)^5 - 1 - 10\left( \dfrac{n^4}{4} + \dfrac{n^3}{2} + \dfrac{n^2}{4} - 10 \dfrac{n(n+1)(n+2)}{6} - 5 \dfrac{n(n+1)(n+2)}{2} -n \right)}{5} = $$ $$\dfrac{n^5}{5} + \dfrac{n^4}{2} + \dfrac{n^3}{3} - \dfrac{n}{30}$$\\\\

	 %------------------------(iii)---------------------------
	 \item $\dfrac{1}{1 \cdot 2} + \dfrac{1}{2 \cdot 3} + ... + \dfrac{1}{n(n+1)}$\\\\ 
	 A partir de $$\dfrac{1}{k} - \dfrac{1}{k+1} = \dfrac{1}{k(k+1)}, \; \; \; k=1,...,n$$ obtenemos $$1-\dfrac{1}{n+1} = \displaystyle\sum_{k=1}^n \dfrac{1}{k(k+1)}$$ \\\\

	 %------------------------(iv)---------------------------
	 \item $\dfrac{3}{1^2 \cdot 2^2} + \dfrac{5}{2^2 \cdot 3^2} + ... + \dfrac{2n +1}{n^2 (n+1)^2}$\\\\
	 De $$\dfrac{1}{k^2} - \dfrac{1}{(k+1)^2} = \dfrac{2k+1}{k^2(k+1)^2}, \; \; \; k=1,...,n$$ obtenemos $$1-\dfrac{1}{(n+1)^2} = \displaystyle\sum_{k=1}^n \dfrac{2k+1}{k^2(k+1)^2}$$  \\\\
	 \end{enumerate}

      %---------------------------------------------7-------------------------------------------
      \item Utilizar el método del problema 6 para demostrar que $\displaystyle\sum_{k=1}^n k^p$ puede escribirse siempre en la forma $$\dfrac{n^{p+1}}{p+1} + An^p + Bn^{p-1} + Cn^{p-2} + ...$$
      Las diez primeras de estas expresiones son
      \begin{center}
      \begin{tabular}{r c l}
      $\displaystyle\sum_{k=1}^n k$&=&$\dfrac{1}{2}n^2 + \dfrac{1}{2}n$\\\\
      $\displaystyle\sum_{k=1}^n k^2$&=&$\dfrac{1}{3} n^3 + \dfrac{1}{2}n^2 + \dfrac{1}{6}n$\\\\
      $\displaystyle\sum_{k=1}^n k^3$&=&$\dfrac{1}{4}n^4 + \dfrac{1}{2} n^3 + \dfrac{1}{4} n^2$\\\\
      $\displaystyle\sum_{k=1}^n k^4$&=&$\dfrac{1}{5} n^6 + \dfrac{1}{2} n^4 + \dfrac{1}{3} n^3 - \dfrac{1}{30}n$\\\\
      $\displaystyle\sum_{k=1}^n k^5$&=&$\dfrac{1}{6}n^6 + \dfrac{1}{2} n^5 + \dfrac{5}{12}n^4 - \dfrac{1}{12}n^2$\\\\
      $\displaystyle\sum_{k=1}^n k^6$&=&$\dfrac{1}{7}n^7 + \dfrac{1}{2}n^6 + \dfrac{1}{2}n^5 - \dfrac{1}{6}n^3 + \dfrac{1}{42}n$\\\\
      $\displaystyle\sum_{k=1}^n k^7$&=&$\dfrac{1}{8}n^8 + \dfrac{1}{2}n^7 + \dfrac{7}{12}n^6 - \dfrac{7}{24}n^4 + \dfrac{1}{12}n^2$\\\\
      $\displaystyle\sum_{k=1}^n k^8$&=&$\dfrac{1}{9}n^9 + \dfrac{1}{2}n^8 + \dfrac{2}{3}n^7 - \dfrac{7}{15}n^5 + \dfrac{2}{9}n^3 - \dfrac{1}{30}n$\\\\
      $\displaystyle\sum_{k=1}^n k^9$&=&$\dfrac{1}{10}n^{10} + \dfrac{1}{2} n^9 + \dfrac{3}{4}n^8 - \dfrac{7}{10}n^6 + \dfrac{1}{2}n^4 - \dfrac{3}{20} n^2$\\\\
      $\displaystyle\sum_{k=1}^n k^10$&=&$\dfrac{1}{11}n^{11} + \dfrac{1}{2}n^{10} + \dfrac{5}{6}n^9 - 1n^7 + 1n5 - \dfrac{1}{2}n^3 + \dfrac{5}{66}n$\\\\
      \end{tabular}
      \end{center}
      Obsérvese que los coeficientes de la segunda columna son siempre $\dfrac{1}{2}$ y que después de la tercera columna las potencias de $n$ de coeficiente no nulo van decreciendo de dos en dos hasta llegar a $n^2$ o a $n$. Los coeficientes de todas las columnas, salvo las dos primeras, parecen bastante fortuitos, pero en realidad obedecen a cierta regla; encontrarla puede considerarse como una prueba de superperspicacia. Para descifrar todo el asunto, véase el problema 26-17)\\\\
      Demostración.- Sea  $(k+1)^{p+1}$ entonces por el teorema del binomio:
      \begin{center}
	 \begin{tabular}{crl}
	    $(k+1)^{p+1}$&$=$&${p+1 \choose p+1} k^{p+1}+{p+1 \choose p} k^p + ... + {p+1 \choose 1} k^p + {p+1 \choose 0}k^0$\\\\
	    $(k+1)^{p+1}$&$=$&$1 \cdot k^{p+1} + {p+1 \choose p} k^p + ... + {p+1 \choose 1}k + {p+1 \choose 0} k^0$\\\\
	    $(k+1)^{p+1}-k^{p+1}$&$=$&$(p+1)k^p + ... + (p+1)k+1 \cdot k^0$\\\\
	 \end{tabular}
      \end{center}
      Luego sumando para cada $k=1,...,n$ se tiene:
      \begin{center}
	 \begin{tabular}{rcl}
	    $2^{p+1} - 1^{p+1}$&$=$&$(o+1)1^p + ... + (p+1)1 + 1 \cdot 1^0$ \\
	    $3^{p+1} -2{p+1}$&$=$&$(p+1)2^p + ... + (p+1)2+1\cdot 2^0$\\
	    &.&\\
	    &.&\\
	    &.&\\
	    $(n+1)^{p+1}$&$=$&$(p+1)n^{p}+...+(p+1)n + 1 \cdot ^0$\\
	 \end{tabular} 
      \end{center}
      Luego por el anterior problema 
      \begin{center}
	 \begin{tabular}{rcl}
	    $(n+1)^{p+1}$ & $=$ & $(p+1)\sum\limits_{k=1}^n k^p + ... + (p+1)\sum\limits_{k=1}^n k^1 + \sum\limits_{k=1}^n k^0 + k^0$\\\\
	    $\dfrac{(n+1)^{p+1}}{(p+1)}$&$=$&$\sum\limits_{k=1}^n k^p + \dfrac{{p+1 \choose p-1}}{(p+1)} \sum\limits_{k=1}^n k^{p-1} + ... + \dfrac{(p+1)}{(p+1)} \sum\limits_{k=1}^n k^1 + \dfrac{1}{(p+1)} \left( \sum\limits_{k=1}^n k^0+ k^0 \right)$\\\\
	 \end{tabular}
      \end{center}
      Luego asumimos que la proposición es verdad para $p-1$ donde podríamos escribir como,
      \begin{center}
	 \begin{tabular}{rcl}
	    $\dfrac{(n+1)^{p+1}}{(p+1)}$&$=$&$\sum\limits_{k=1}^{n} k^p +$ términos que involucran las potencias de $n \leq p$\\\\
	    $\sum\limits_{k=1}^n k^p$&$=$&$\dfrac{(n+1)^{p+1}}{(p+1)} + $ términos que involucran las potencias de $n \leq p$\\\\
	 \end{tabular}
      \end{center}

      %---------------------------------------------8-------------------------------------------
      \item Demostrar que todo número natural es o par o impar.\\\\
      Demostración.- \; Asumimos que $n$ es impar o par, entonces debemos probar que $n+1$ también, es o bien impar o bien par.\\ 
      Sea $n$ par entonces $n=2k$ para algún $k$. Así $n+1=2k+1$  y por definición vemos que es impar.\\
      Luego sea $n$ impar, entonces $n=2k+1$ para algún $k$.  Por lo tanto $n+1=2k+1+1 = 2k +2 = 2(k+1)$. Así en cualquiera de los dos casos, $n+1$ es o bien par o impar. \\\\   

      %---------------------------------------------9-------------------------------------------
      \item Demostrar que si un conjunto $A$ de números naturales contiene $n_0$ y contiene $k+1$ siempre que contenga $k$, entonces $A$ contiene todos los números naturales $\geq n_0$.\\\\
	 Demostración.- \; Sea $B$ el conjunto de todos los números naturales $l$ tales que $n_0 -1 +1$ está en $A$. Entonces $1$ está en $B$, y $l+1$ está en  $B$ si $l$ está en $B$, es decir $k=n_0 -1 + l $, por lo tanto $k+1=(n_0 -1 )+(l+1)$ está en $A$, lo que implica que $l+1$ está en $B$ de modo que $B$ contiene todos los números naturales, los cuales significa que $A$  contiene todos los números naturales $\geq n_0$\\\\

       %---------------------------------------------10-------------------------------------------
      \item Demostrar el principio de inducción completa a partir del principio de buena ordenación.\\\\
      Demostración.-\; Supongamos que $A$ contiene a $1$ y que $A$ contiene a $n+1$, si contiene a $n$. Si $A$ no contiene todos los números naturales, entonces el conjunto $B$ de números naturales que no están en $A$ es distinto de $\emptyset$. Por lo tanto, $B$ tiene un número natural $n_0$. Ahora $n_0 ,\neq 0$ ya que $A$ contiene a $1$ entonces podemos escribir $n_0 = (n_0 -1) +1$, donde $n_0-1$ es un número natural. Luego $n_0-1$ no está en $B$, entonces $n_0 -1$ está en $A$. Por hipótesis, $n_0$ debe estar en $A$, entonces $n_o$ no está en $B$, el cual es una contradicción.             	 
      
      %---------------------------------------------11-------------------------------------------
      \item Demostrar el principio de inducción completa a partir del principio de inducción ordinario.\\\\
      Demostración.- \; Se sabe que $1$ está en $B$. Luego si $k$ está en $B$, entonces $1,...,k$ están todos en $A$, de modo que $k+1$ está en $A$ y así $1,...,k+1$ están en $A$, con lo que $k+1$ está en $B$. Por inducción, $B=N$, así que también $A=N$.\\\\

      %---------------------------------------------12--------------------------------------------
      \item 
	 \begin{enumerate}[\bfseries (a)]

	 %------------------------(a)---------------------------
	 \item Si $a$ es racional y \; $b$ es irracional ¿es $a+b$ necesariamente irracional? ¿Y si $a$ \; y \; $b$ es irracional? \\\\
	 Respuesta.- \; Si, puesto que si $a+b$ fuese racional, entonces $b=(a+b) -a$ seria racional. Luego si $a$ y $b$ son irracionales, entonces $a+b$ podría ser racional, ya que $b$ podría ser $r-a$ para algún número racional $a$.\\\\          

	 %------------------------(b)---------------------------
	 \item Si $a$ es racional y \; $b$ es irracional, ¿es $ab$ necesariamente irracional?\\\\
	 Respuesta.- \; Si $a=0$, entonces $ab$ es racional. Pero si $a\neq 0$ entonces $ab$ no podría ser racional, ya que entonces $b=(ab) \cdot a^{-1}$ sería racional.\\\\     

	 %------------------------(c)---------------------------
	 \item ¿Existe algún número $a$ tal que $a^2$ es irracional pero $a^4$ racional?\\\\
	 Si existe por ejemplo $\sqrt[4]{2}$\\\\

	 %------------------------(d)---------------------------
	 \item ¿Existen dos números irracionales tales que sean racionales tanto su suma como su producto?\\\\
	 Si existen por ejemplo $\sqrt{2}$ y $- \sqrt{2}$\\\\
      \end{enumerate}

      %---------------------------------------------13--------------------------------------------
      \item 
      \begin{enumerate}[\bfseries a)]
      %------------------------(a)---------------------------
      \item Demostrar que $\sqrt{3}$, $\sqrt{5}$ y $\sqrt{6}$ son irracionales. Indicación: Para tratar $\sqrt{3},$ por ejemplo, aplíquese el hecho de que todo entero es de la forma $3n$ ó $3n+1$ ó $3n+2$ ¿Por qué no es aplicable esta demostración para $\sqrt{4}$?\\\\
      Demostración.- \; Puesto que:
      \begin{center}
      \begin{tabular}{r c l c l}
      $(3n+1)^2$&=&$9n^2 + 6n + 1$&=&$3(3n^2+2n) + 1$\\
      $(3n+2)^2$&=&$9n^2+12n + 4$&=&$3(3n^2 + 4n + 1) + 1$\\
      \end{tabular}
      \end{center}
      queda demostrado que un número no es múltiplo de $3$ si es de la forma $3n+1$ ó $3n+2$.\\
      se sigue que $k^2$ es divisible por $3$, entones $k$ debe ser también divisible por $3$. Supóngase ahora que $\sqrt{3}$ fuese racional, y sea $\sqrt{3} = p/q$, donde $p$ \; y \; $q$ no tienen factores comunes. Entonces $p^2=3q^2,$ de modo que $p^2$ es divisible por $3$, así que también lo debe ser $p$. De este modo, $p=3p^{'}$ para algún número natural $p^{'}$, y en consecuencia $(3p^{'})^2 = 3q^2$ ó $(3p^{'})^2 = q^2.$ Así pues, $q$ es también divisible por $3$, lo cual es una contradicción.\\
      Las mismas demostraciones valen para $\sqrt{5}$ y $\sqrt{6}$, ya que las ecuaciones,
      \begin{center}
      \begin{tabular}{rclcl}
      $(5n+1)^2$&=&$25n^2 + 10n + 1$&=&$5(5n^2 + 2n)+1$\\
      $(5n+2)^2$&=&$25n^2 + 20n + 4$&=&$5(5n^2 + 4n)+4$\\
      $(5n+3)^2$&=&$25n^2 + 30n + 9$&=&$5(5n^2 + 6n + 1)+4$\\
      $(5n+4)^2$&=&$25n^2 + 40n + 16$&=&$5(5n^2+8n+3)+1$\\
      \end{tabular}
      \end{center}
      la ecuación correspondiente para los números de la forma $6n+m$ demuestran que si $k^2$ es divisible por $5$ ó $6$, entones también lo debe ser $k$. La demostración falla para $\sqrt{4}$, porque $(4n+2)^2$ es divisible por $4$.\\\\

      %------------------------(b)---------------------------
      \item Demostrar que $\sqrt[3]{2}$ y $\sqrt[3]{3}$ son irracionales.\\\\
      Demostración.- \; Puesto que,
      $$(2n+1)^3 = 8n^3 + 12n^2 + 6n + 1 = 2(4n^3 + 6n^2 + 3n) + 1,$$
      se sigue que si $k^3$ es par, entonces $k$ es par. Si $\sqrt[3]{2} = p/q,$ donde $p$ \; y \; $q$ no tienen factores comunes, entonces $p^3 = 2q^3,$ de modo que $p^3$ es divisible por $2,$ por lo que también lo debe ser $p.$ Así pues, $p=2p^{'}$ para algún número natural $p^{'}$ y en consecuencia $(2p^{'})^3 = 2q^3,$ ó $4(p^{'})^3 = q^3.$ Por lo tanto, $q$ es también par, lo cual es una contradicción.\\
      La demostración para $\sqrt[3]{3}$ es análogo, utilizando las ecuaciones.
      $$(3n+1)^3 = 27n^3 + 27n^3 + 27n^2 + 9n +1 = 3(9n^3 + 9n^2 + 3n) + 1,$$
      $$(3n+2)^3 = 27n^3 + 54n^2 + 36n + 8 = 3(9n^2 + 18n^2 + 12n + 2) + 2.$$\\\\
      \end{enumerate}

      %---------------------------------------------14--------------------------------------------
      \item Demostrar que:
      \begin{enumerate}[\bfseries (a)]
      %------------------------(a)---------------------------
      \item $\sqrt{2} + \sqrt{3}$ es irracional.\\\\
      Demostración.- \; Sea $\sqrt{2} + \sqrt{3}$ racional, entonces $\left( \sqrt{2} + \sqrt{3} \right)^2$ sería racional, luego $$5 + 2 \sqrt{6}$$ y en consecuencia $\sqrt{6}$ sería racional lo cual es falso.\\\\

      %------------------------(b)---------------------------
      \item $\sqrt{6} - \sqrt{2} - \sqrt{3}$ es irracional.\\\\
      Demostración.- \;  Sea $\sqrt{6} - \sqrt{2} - \sqrt{3}$ racional, entonces 
      \begin{center}
      \begin{tabular}{rcl}
      $\left[ \sqrt{6} + \left( \sqrt{2} + \sqrt{3} \right) \right]^2$&=&$6 + \left( \sqrt{2} + \sqrt{3} \right)^2 - 2 \sqrt{6} \left( \sqrt{2} + \sqrt{3} \right)$\\
      &=&$11 + 2\sqrt{6} \left[ 2 - \left( \sqrt{2} + \sqrt{3} \right) \right]$\\
      \end{tabular}
      \end{center}
      Así, $\sqrt{6} \left[ 2 - \left( \sqrt{2} + \sqrt{3} \right) \right]$ sería racional, con lo que de igual manera sería,
      \begin{center}
      \begin{tabular}{r c l}
      $\lbrace \sqrt{6} \left[ 1 - \left( \sqrt{2} + \sqrt{3} \right) \right] \rbrace ^2$&=&$6 \left[ 1 - \left( \sqrt{2} + \sqrt{3} \right) \right]^2$\\
      &=&$11 + 2\sqrt{6} \left[1 - \left( \sqrt{2} + \sqrt{3} \right) \right]$\\
      \end{tabular}
      \end{center}
      De este modo $\sqrt{6} - (\sqrt{2} + \sqrt{3})$ y $\sqrt{6} - 2 \left( \sqrt{2} + \sqrt{3} \right)$ serían racionales, lo que implicaría que $\sqrt{2} + \sqrt{3}$ fuese racional, en contradicción de la parte $a)$.\\\\
      \end{enumerate}

      %---------------------------------------------15--------------------------------------------
      \item 
      \begin{enumerate}[\bfseries (a)]
      %------------------------(a)---------------------------
      \item Demostrar que si $x=p+ \sqrt{q}$, donde $p$ \; y \; $q$ son racionales, y \; $m$ es un número natural, entonces $x^m = a + b \sqrt{q}$ siendo $a$ \; y \; $b$ números racionales.\\\\
      Demostración.- \; Sea $m=1$ entonces $(p + \sqrt{q})^1 = a + b\sqrt{q}$. Supongamos que se cumple para $m$, entonces $$(p+\sqrt{q})^{m+1} = (a + b\sqrt{q})(p + \sqrt{q}) = (ap+bq)+(a+pb)\sqrt{q}$$ donde $ap+bq$ y $a+bp$ son racionales.\\\\

      %------------------------(b)---------------------------
      \item Demostrar también que $(p - \sqrt{q})^m = a - b\sqrt{q}$\\\\
      Demostración.- \; Similar a la parte $a)$, se cumple para $m=1$. Si es verdad para $m$, entonces $$(p-\sqrt{q})^{m+1} = (a - b\sqrt{q})(p - \sqrt{q}) = (ap+bq)-(a+pb)\sqrt{q}$$.\\\\
      \end{enumerate}

      %---------------------------------------------16--------------------------------------------
      \item 
      \begin{enumerate}[\bfseries (a)]
      %------------------------(a)---------------------------
      \item Demostrar que si $m$ \; y \; $n$ son números naturales y $m^2/n^2 < 2$, entonces $\left( m+2n \right)^2 / \left( m + 2 \right)^2 > 2;$ demostrar, además que $$\dfrac{\left( m + 2n \right)^2}{\left( m + n \right)^2} - 2 < 2 - \dfrac{m^2}{n^2}$$\\\\
      Demostración.- \; Si $m^2/n^2 < 2$ entonces $m^2 < 2 n^2$, sumando $m^2$, $4mn$ y $2n^2$ tenemos $2m^2 + 4mn + 2n^2 < 4n^2 + m^2 + 4mn$, luego $2(m+n)^2 < (m+2n)^2$, así nos queda $(m + 2n)^2 / (m+n)^2 > 2$\\
      Para la segunda parte podemos partir de $m^2 - 2n^2<0$, entonces:
      \begin{center}
      \begin{tabular}{r c l l}
      $m^2 - 2n^2$&$<$&$0$&\\\\
      $m^3 - 2mn^2$&$<$&$0$&multiplicando por $m$\\\\
      $mn^2 + m^3 + mn^2 - 4mn^2$&$<$&$0$&escribiendo $mn^2$ de otra manera\\\\
      $mn^2 + 2n^3 + m^3 + m^2 n + mn^2 -2m^2 n - 4mn^2 -2n^3$&$<$&$0$&sumando $2n^3$ y $2m^2 n$\\\\
      $n^2 (m+2n) + \left[ (m^2 + 2mn + n^2)(m-2n) \right]$&$<$&$0$&\\\\
      $ n^2(m+2n)^2  +  \left[ (m+n)^2 (m+2n)(m-2n) \right]$&$<$&$0$&multiplicando por $m+2n$\\\\
      $n^2(m+2n)^2  +  \left[ (m+n)^2 (m^2 - 4n^2) \right]$&$<$&$0$&\\\\
      $\dfrac{n^2(m+2n)^2 - 4n^2(m+n)^2 + m^2(m+n)^2}{n^2(m+n)^2}$&$<$&$0$&dividimos por $n^2(m+n)^2$\\\\
      $\dfrac{(m+2n)^2 - 2(m+2)^2 - 2n^2(m+2)^2}{n^2(m+n)^2}$&$<$&$- \dfrac{m^2}{n^2}$&\\\\
      $\dfrac{(m+2n)^2}{(m+n)^2} - 2$&$<$&$2 - \dfrac{m^2}{n^2}$&\\\\
      \end{tabular}
      \end{center}

      %------------------------(b)---------------------------
      \item Demostrar los mismos resultados con todos los signos de desigualdad invertidos. \\\\
      Demostración.- \; Quedará de la siguiente forma, Si $m^2/n^2>2$, entonces $\left( m+2n \right)^2 / \left( m + 2 \right)^2 < 2$, luego demostrar que $$\dfrac{\left( m + 2n \right)^2}{\left( m + n \right)^2} - 2 > 2 - \dfrac{m^2}{n^2}$$
      Similar a la parte $a)$ tendremos $m^2 > 2n^2$, luego $2m^2 + 4mn + 2n^2 > 4n^2 + m^2 + 4mn$, así $ (m+2n)^2 > 2(m + n)^2 $\\
      Después se puede demostrar la segunda parte con facilidad siguiendo el ejemplo $a)$ pero invirtiendo la desigualdad ya que $n$ \; y \; $m$ son números natural.\\\\

      %------------------------(c)---------------------------
      \item Demostrar que si $m/n < \sqrt{2}$, entonces existe otro número racional $m^2 / n^2$ con $m/n < m^{'} / n^{'} < \sqrt{2}$\\\\
      Demostración.- \; Sea $m_1=m+2n$ y $n_1=m+n$, luego elegimos y remplazamos en,  
      \begin{center}
      \begin{tabular}{rclcr}
      $m^{'}$ & $=$ & $m_1+2n_1$ & $=$ & $3m+4n$\\
      $n^{'}$ & $=$ & $m_1+n_1$  & $=$ & $2m+3n$\\
      \end{tabular}
      \end{center}
      \end{enumerate}
      De donde $\dfrac{m}{n}<h$ si y sólo si $0<m^{'} - mn^{'}=(3m+4n)n - (2m + 3n)m=2(2n^2 - m^2)$. Es claro para $\dfrac{m}{n} < \sqrt{2}$ que $2n^2 - m^2 >0$\\
      Por otro lado tenemos $\dfrac{m^{'}}{n^{'}}$ si y sólo si $0<2n^{'^{2}} - m^{'^{2}}=2(2m+3n)^2-(3m+4n)^2=2n^2 - m^2.$ Como antes, $2n^2-m^2>0$ se sigue de $\dfrac{m}{n}<\sqrt{2}$\\\\
 
      %---------------------------------------------17----------------------------------------------
      \item Parece normal que $\sqrt{n}$ tenga que ser irracional siempre que el número natural $n$ no sea el cuadrado de otro número natural. Aunque puede usarse en realidad el método del problema 13 del capitulo 2 de Michael Spivak para tratar cualquier caso particular, no está claro, sin más, que este método tenga que dar necesariamente resultados, y para una demostración del caso general se necesita más información. Un número natural $p$ se dice que es un número primo si es imposible escribir $p=ab$; por conveniencia se considera que $1$ no es un número primo. Los primeros números primos son $2,3,5,7,11,13,17,19.$ Si $n>1$ no es primo, entonces $n=ab,$ con $a$ \; y \; $b$ ambos $<n;$ si uno de los dos $a$ \; o \; $b$ no es primo, puede ser factorizado de manera parecido; continuando de esta manera se demuestra que se puede escribir $n$ como producto de números primos. Por ejemplo, $28=2\cdot 2\cdot 7.$\\
      \begin{enumerate}[\bfseries a)]
         %------------------------(a)---------------------------
         \item Conviértase este argumento en una demostración riguroso por inducción completa. (En realidad, cualquier matemático razonable aceptaría este argumento informal, pero ello se debería en parte a que para él estaría claro cómo formularla rigurosamente.)\\
         Un teorema fundamental acerca de enteros, que no demostraremos aquí, afirma que esta factorización es única, salvo en lo que respeta al orden de los factores. Así, por ejemplo, $28$ no puede escribirse nunca como producto de números primos uno de los cuales sea $3$, ni puede ser escrito de manera que $2$ aparezca una sola vez (ahora debería verse clara la razón de no admitir a $1$ como número primo.)\\\\
         demostración.- \; Supóngase que para todo número $<n$ puede ser escrito  como un producto de primos. Si $n>1$ no es primo, entonces $n=ab$, para $a,b<n$. Pero $a$ \; y \; $b$ son ambos producto de primos, así que $n=ab$ lo es también.\\\\

         %------------------------(b)---------------------------
         \item Utilizando este hecho, demostrar que $\sqrt{n}$ es irracional a no ser que $n=m^2$ para algún número natural $m.$\\\\
         Demostración.- \; Sea $\sqrt{n} = \dfrac{p}{q}$, entonces $nb^2 = a^2,$  luego si descomponemos en producto de factores primos, $nb^2$ y $a^2$ deberían coincidir. Ahora según lo explicado anteriormente, cada número primo debe aparecer un número par de veces en $a^2$ y $b^2$, y por lo tanto deberá ocurrir lo mismo con $n.$ Esto implica que $n$ es un cuadrado perfecto.\\\\

         %------------------------(c)---------------------------
         \item Demostrar que $\sqrt[k]{n}$ es irracional a no ser que $n=m^k$\\\\
         Demostración.- \; La Demostración es parecida a la parte $b$ pero haciendo uso del hecho de que cada número primo entra en $a^k$ y en $b^k$ un número de veces que es múltiplo de $k$\\\\   

         %------------------------(d)---------------------------
         \item Al tratar de números primos no se puede omitir la hermosa demostración de Euclides de que existe un número infinito de ellos. Demuestre que no puede haber sólo un número finito de números primos $p_1, p_2, p_3,...,p_n$ considerando $p_1\cdot p_2 \cdot ... \cdot p_k + 1$\\\\
         Demostración.- \; Si $p_1,...,p_n$ fuesen los únicos números primos, entonces $p_1 \cdot p_2 \cdot ... \cdot p_n + 1$ no podría ser primo, ya que es mayor que cada uno de ellos, de modo que tiene que ser divisible por un número primo. Pero es claro que este número primo no es ninguno de los $p_1,...,p_n$, lo cual constituye una contradicción. Para poder explicarlo mejor si $p_1,...,p_n$ son los $n$ primeros números primos, entonces el primo que ocupa el lugar $n+1$ es $\leq p_1 \cdot p_2 \cdot ... \cdot p_n+1$. Sin embargo, $p_1\cdot p_2 \cdot ...\cdot p_n + 1$ no tiene que ser necesariamente primo.\\\\       
         \end{enumerate} 

      %----------------------------------18.---------------------------------------
      \item
	 \begin{enumerate}[\bfseries (a)]
	    %--------------------(a)-------------------------
	    \item Demostrar que si $x$ satisface
	       $$x_n + a_{n-1}x^{n-1} + ... + a_0 = 0$$
	       para algunos enteros $a_{n-1},...,a_n$ entonces $x$ es irracional si no es entero. (¿Por qué es esto una generalización del problema 17?)\\\\
	       Demostración.- \; Supóngase que es $x=p/q$ donde $p$ y $q$ son números naturales primos entre si. Entonces $$\dfrac{p^n}{q^n} + a_{n-1} \dfrac{p^{n-1}}{q^{n-1}} + ... + a_0 = 0,$$
	       con lo que 
	       $$p^n + a_{n-1} p ^{n-1}q + ... + a_0 q^n = 0$$
	       Ahora bien, si $q \neq \pm 1,$ entonces $q$ tiene por lo menos un divisor primo. Este divisor primo divide a cada uno de los términos que siguen a $p^n$, con lo que también deberá dividir a $p^n$. Dividirá por lo tanto a $p$, lo cual es una contradicción. Así pues, $q= \pm 1,$ lo que significa que $x$ es entero.\\\\

	    %--------------------(b)--------------------------
	    \item Demostrar que $\sqrt{2} + \sqrt[3]{2}$ es irracional.\\\\ 
	       Demostración.- \; Sea $a=\sqrt{2} + 2^{1/3}$. Se demostrará por contradicción. Supongamos que $a$ es racional, entones,
	       \begin{center}
		  \begin{tabular}{crl}
		     $2$ & $=$ & $(2^{1/3})^3$ \\
			 & $=$ & $(a- \sqrt{2})^3$ \\
			 & $=$ & $a^3 - 3a^2 \sqrt{2} + 3a \cdot 2 - 2^{3/2}$ \\
			 & $=$ & $a^3 + 6a - \sqrt{2}(3a^2+2)$ \\
		  \end{tabular}
	       \end{center}
	       por lo tanto, $$\sqrt{2}=\dfrac{a^3 + 6a -2}{3a^2 + 2} \in \mathbb{Q}$$
	       es bien sabido que $\sqrt{2}$ es irracional. De ahí llegamos a una contradicción.\\\\
      \end{enumerate}

      %------------------------------------------19.----------------------------------------------
   \item Demostrar la desigualdad de Bernoulli: Si $h>-1$, entonces $$(1+h)^n \geq 1 + nh$$
      ¿Por qué es esto trivial si $h>0$?\\\\
      Demostración.-\; Si $n=1$, entonces $(1+h)^n=1+nh$. Supóngase que $(1+h)^n \geq 1+nh.$ Entonces $(1+h)^n+1=(1+h)(1+h)^n \geq (1+h)(1+nh)$, ya que $1+h>0$ luego $1+(n+1)h+nh^2\geq 1+(n+1)h$\\
      Para $h>0$, la igualdad se sigue directamente del teorema del binomio, ya que todos los demás términos que aparecer en el desarrollo de $(1+h)^n$ son positivos.\\\\
      
      %------------------------------------------20.--------------------------------------------------
   \item La sucesión de Fibonacci $a_1,a_2,a_3,...$ se define como sigue:
      \begin{center}
	 \begin{tabular}{rcll}
	    $a_1$ & $=$ & $1,$ & \\
	    $a_2$ & $=$ & $1,$ & \\
	    $a_n$ & $=$ & $a_{n-1} + a_{n-2}$ & para $n \geq 3$\\
	 \end{tabular}
      \end{center}
      Esta sucesión, cuyos primeros términos son $1,1,2,3,5,8,...,$ fue descubierta por Fibonacci (1175-1250, aprox.) en relación con un problema de conejos. Fibonacci supuso que una pareja de conejos criaba una nueva pareja cada mes y que después de dos meses cada nueva pareja se comportaba del mismo modo. El número $a_n$ de parejas nacidas en el n-ésimo mes es $n_{n-1} + a_{n-2}$, puesto que nace una pareja por cada pareja nacida en el mes anterior, y además, cad pareja nacida hace dos meses produce ahora una nueva pareja. Es verdaderamente asombroso el número de resultados interesantes relacionados con esta sucesión, hasta el punto de existir una Asociación de Fibonacci que publica una revista, the Fibonacci Quarterly.\\
      Demostrar que $$a_n = \dfrac{\left( \dfrac{1+\sqrt{5}}{2} \right)^n - \left( \dfrac{1-\sqrt{5}}{2} \right)^n}{\sqrt{5}}$$\\
      Demostración.- \; Al ser $$\dfrac{\left( \dfrac{1+\sqrt{5}}{2} \right)^1 - \left( \dfrac{1-\sqrt{5}}{2} \right)^1}{\sqrt{5}}=\dfrac{\sqrt{5}}{\sqrt{5}}=1$$
      La fórmula es válida para $n=1$ y también se cumple para $n=1$. Supóngase que es válida para todo $k<n,$ donde $n\geq 3.$ En tal caso es válida en particular para $n-1$ y para $n-2$, luego por hipótesis
      \begin{center}
	 \begin{tabular}{rcl}
	    $a_n$ & $=$ & $a_{n-1} + a_{n-2}$\\\\
		  & $=$ & $\dfrac{\left( \dfrac{1+\sqrt{5}}{2} \right)^{n-2} - \left( \dfrac{1-\sqrt{5}}{2} \right)^{n-2} + \left( \dfrac{1+\sqrt{5}}{2} \right)^{n-1} - \left( \dfrac{1-\sqrt{5}}{2} \right)^{n-1}}{\sqrt{5}}$\\\\
		  & $=$ & $\dfrac{\left( \dfrac{1+\sqrt{5}}{2}\right)^{n-2} \left( \dfrac{1+\sqrt{5}}{2} \right) - \left( 1 - \dfrac{1-\sqrt{5}}{2} \right)^{n-2} \left( 1 - \dfrac{1- \sqrt{5}}{2} \right)}{\sqrt{5}}$\\\\
		  & $=$ & $\dfrac{\left( \dfrac{1+\sqrt{5}}{2} \right)^{n-2} \left( \dfrac{1+\sqrt{5}}{2}\right)^2 - \left( 1 - \dfrac{1-\sqrt{5}}{2} \right)^{n-2} \left( 1 - \dfrac{1-\sqrt{5}}{2} \right)^2}{\sqrt{5}}$\\\\
	       & $=$ & $\dfrac{\left( \dfrac{1+\sqrt{5}}{2} \right)^n - \left( \dfrac{1-\sqrt{5}}{2} \right)^n}{\sqrt{5}}$\\\\
	 \end{tabular}
      \end{center}

      %-------------------------------------------------21.------------------------------------------------
   \item La desigualdad de Schwarz (problema 1-19) tiene en realidad una forma más general: $$\displaystyle\sum_{i=1}^n x_1 y_1 \leq \sqrt{\sum_{i=1}^n x_i^2} \sqrt{\sum_{i=1}^n y_i^2}$$
      Dar de esto tres demostraciones, análogas a las tres demostraciones del problema 1-19\\\\
      Demostración.-\; 
      \begin{enumerate}[\bfseries i)]
	 \item Como antes, la demostración es trivial si para todo $y_i=0$ o si hay algún numero $\lambda$ con $x_i=\lambda y_i$ para todo $i$. Es decir, 
	    \begin{center} 
	       \begin{tabular}{rcl}
		  $0$ & $<$ & $\sum\limits_{i=1}n (\lambda y_i - x_i)^2$\\\\
		   & $=$ & $\lambda \left( \sum\limits_{i=1}^n y_i^2 \right) -2 \lambda \left( \sum\limits_{i=1}^n x_i y_i\right) + \sum\limits_{i=1}^n x_i^2$\\\\
	       \end{tabular}
	       así por el problema 1-18 queda demostrado. 
	    \end{center}
	 \item Usando $2xy \leq x^2 + y^2$ con 
            $$x=\dfrac{x_i}{\sqrt{\sum\limits_{i=1}^n x_i^2}}, \,\,\,\, y=\dfrac{y_i}{\sqrt{\sum\limits_{i=1}^n y_i^2}}$$
            obtenemos, 
            $$\dfrac{2x_iy_1}{\sqrt{\sum\limits_{i=1}^n x_i^2} \sqrt{\sum\limits_{i=1}^n y_i^2}} \leq \dfrac{x_i^2}{\sum\limits_{i=1}^n x_i^2} + \dfrac{y_i^2}{\sum\limits_{i=1}^n y_i^2} \,\,\,\, (1)$$
            luego
            $$\dfrac{\sum\limits_{i=1}^n 2x_iy_i}{\sqrt{\sum\limits_{i=1}^n x_i^2} \sqrt{\sum\limits_{i=1}^n y_i^2}} \leq \dfrac{\sum\limits_{i=1}^n x_i^2}{\sum\limits_{i=1}^n x_i^2} + \dfrac{\sum\limits_{i=1}^n y_i^2}{\sum\limits_{i=1}^n y_i^2}=2$$
            Nuevamente, la igualdad se cumple solo si se cumple en $(1)$ para todo $i$, lo que significa que, 
            $$\dfrac{x_i}{\sqrt{\sum\limits_{i=1}^n x_i^2}} = \dfrac{y_i}{\sqrt{\sum\limits_{i=1}^n y_i^2}}$$
            para todo $i$. Si todo $y_i$ es distinto de $0$. Esto significa que $x_i=\lambda y_i$ para 
            $$\lambda = \dfrac{\sqrt{\sum\limits_{i=1}^n x_i^2}}{\sqrt{\sum\limits_{i=1}^n y_i^2}}$$
         \item La demostración depende de la siguiente igualdad:
            $$\sum\limits_{i=1}^n x_i^2 \cdot \sum\limits_{i=1}^n y_i^2 = \left( \sum\limits_{i=1}^n x_iy_i \right)^n + \sum\limits_{i<j} (x_i y_j -x_j y_i)^2$$
            al verificar esta igualdad notamos que,
            $$\sum\limits_{i=1}x_i^2 \cdot \sum\limits_{i=1}^n y_i^2 = \sum\limits_{i=1}^n x_i^2 y_i^2 + \sum\limits_{i\neq j} x_i2 y_j^2$$
             y por lo tanto,
             $$\left( \sum\limits_{i=1}^n x_i y_i \right)^2 = \sum\limits_{i=1}^n (x_i y_i)^2 + \sum\limits_{i\neq j} x_i y_i x_j y_j$$
             La diferencia es
             \begin{center} 
                \begin{tabular}{rlc}
                   $\sum\limits_{i \neq j} (x_i^2 y_j^2 - x_i y_i x_j y_j)$ & $=$ & $2 \sum\limits_{i<j} (2_i^2 y_j^2 + x_j^2 y_i^2 - x_i y_i x_j y_j)$\\\\
                      & $=$ & $2 \sum\limits_{i<j} (x_i y_j - x_j y_i)^2$\\\\
                \end{tabular}
             \end{center}
             Si la igualdad se cumple en la desigualdad de Schwarz, para todo $x_iy_j = x_j y_i$. Si algún $y_i \neq 0$ y $y_i \neq 0$, luego $x_i=\dfrac{x_1}{y_1}y_i$ para todo $i$, así tenemos que $\lambda = \dfrac{x_1}{y_1}$\\\\
      \end{enumerate}

      %--------------------------------------------------22.-----------------------------------
       \item El resultado del problema 1-7 tiene una generalización importante:\\
          Si $a_i,...,a_n \geq 0,$ entonces la \textbf{media aritmética} 
          $$A_n=\dfrac{a_1+...+a_n}{n}$$
          y la \textbf{media geométrica}
          $$G_n=\sqrt[n]{a_1\cdot \cdot \cdot a_n}$$
          satisfacen
          $$G_n \leq A_n$$
          \begin{enumerate}[\bfseries (a)]
             %--------------------(a)-----------------------------
             \item Supóngase que $a_1<A_n.$ Entonces algún $a_i$ tiene que satisfacer $a_i>A_n$, pongamos que sea $a_2>A_n$. Sea $\bar{a}_1 = A_n$ y sea $\bar{a}_2=a_1+a_2 - \bar{a}_1$. Demostrar que $$\bar{a}_1 \bar{a}_2 \geq a_1 a_2$$ 
                ¿Por qué la repetición de este proceso un suficiente número de veces demuestra que $G_n \leq A_n$? (He aquí otra ocasión en que resulta ser un buen ejercicio establecer una demostración formal por inducción, al tiempo que se da una explicación informal.)¿Cuándo se cumple la igualdad en la fórmula $G_n A_n$?\\\\
		  Demostración.-\; Tenemos que probar que $$A_n (a_1+a_2-A_n)\geq a_1 a_2$$ 
		  Vemos que es lo mismo demostrar que $A_n^2 - (a_1 + a_2)A_n + a_1 a_2 \leq 0$ de donde $(A_n - a_1)(A_n-a_2)\leq 0$, el cual sabemos que es verdad porque $(A_n -a_1)>0$ y $(A_n - a_2)<0$\\\\

               %------------------(b)------------------------
             \item Haciendo uso del hecho de ser $G_n \leq A_n$ cuando $n=2$, demostrar por inducción sobre $k$, que $G_n \leq A_n$ para $n=2_k$\\\\
		 Demostración.- \; Sabemos que $G_n \leq A_n$ cuando $n=2^1$. Supóngase que $G_n \leq A_n$ para $n=2^k$ para luego $m=2^{k+1}=2n$, entonces,
		  \begin{center}
		      \begin{tabular}{rcll}
			  $G_m$&$=$&$\sqrt[m]{a_1 \cdot \cdot \cdot a_m}$&\\\\
			  &$=$&$\sqrt{\sqrt[n]{a_1 \cdot \cdot \cdot a_n} \sqrt[n]{a_{n+1}\cdot \cdot \cdot a_m}}$&\\\\
			  &$\leq$&$\dfrac{\sqrt[n]{a_1 \cdot \cdot \cdot a_n} + \sqrt[n]{a_{n+1}\cdot \cdot \cdot a_m}}{2}$& ya que $G_2 \leq A_2$\\\\
			  &$\leq$&$\dfrac{\dfrac{a_1 + ... + a_n}{n} + \dfrac{a_{n+1} + ... + a_m}{n}}{2}$&\\\\
			  &$=$&$\dfrac{a_1 + ... + a_m}{2n}$&\\\\
			  &$=$&$A_m$&\\\\
		      \end{tabular}
		  \end{center}

                %------------------(c)---------------------------
             \item Para un $n$ general, sea $2_m>n.$ Aplíquese la parte $(b)$ a los $2_m$ números 
                $$a_1,\cdot \cdot \cdot , a_n,\,\,\,\, \underbrace{A_n,\cdot \cdot \cdot A_n}_{{2^{m}-n \;veces}}$$
                para demostrar que $G_n \leq A_n$\\\\
		  Demostrar.- \; Aplicando la parte $(b)$  a los $2^m$ números, para $k=2^m-n$
		  \begin{center}
		      \begin{tabular}{rcl}
			  $(a_1 \cdot \cdot \cdot a_n)(A_n)^k$&$\leq$&$\left[\dfrac{a_1 + ... + a_n + k A_n}{2^m}\right]^{2^m}$\\\\
			  &$=$&$\left[\dfrac{n A_n + k A_n}{2^m}\right]^{2^m}$\\\\
			  &$=$&$(A_n)^{2^m}$\\\\
		      \end{tabular}
		  \end{center}
		  así,
		  $$a_1\cdot \cdot \cdot a_n \leq (A_n)^{2^m - k} = (A_n)^n$$\\\\
          \end{enumerate}

      %----------------------------------------------23.----------------------------------------
       \item Lo que sigue es una definición recursiva de $a_n:$\\
          \begin{center}
             \begin{tabular}{rcl}
                $a_1$&$=$&$a$\\
                $a_{n+1}$&$=$&$a_n \cdot a$\\
             \end{tabular}
          \end{center}
          Demostrar por inducción que
          \begin{center}
             \begin{tabular}{rcl}
                $a_{n+m}$&$=$&$a_n \cdot a_m$\\
                $(a^n)^m$&$=$&$a^{nm}$\\
             \end{tabular}
          \end{center}
	  Demostración.-\; La primera ecuación es verdad para $m=1$ ya que $a^{n+1}=a^n \cdot a^1$. Supongamos que $a^{n+m}=a^n \cdot a^m$, entonces 
	  \begin{center}
	      \begin{tabular}{rcl}
		  $a^{n+(n+1)}$&$=$&$a^{(n+1)+1} \cdot a$\\
		  &$=$&$(a^n \cdot a^m)\cdot a$\\
		  &$=$&$a^n \cdot (a^m \cdot a)$\\
		  &$=$&$a^n \cdot a^{m+1}$\\
	      \end{tabular}
	  \end{center}
	  por lo tanto la primera ecuación es verdad para $m+1$\\\\
	  La segunda ecuación es verdad para $m=1$ ya que $(a^n)^1=a^{n\cdot 1}$. Supóngase que $(a^n)^m=a^{nm}$, entonces, 
	  \begin{center}
	      \begin{tabular}{rcl}
		  $$&$=$&$$\\
		  &$=$&$(a^n \cdot a^m)\cdot a$\\
		  &$=$&$a^n \cdot (a^m \cdot a)$\\
		  &$=$&$a^n \cdot a^{m+1}$\\\\
	      \end{tabular}
	  \end{center}

      %--------------------------------------------24.--------------------------------------
       \item Supóngase que conocemos las propiedades P1 y P4 de los números naturales, pero que no se ha hablado de multiplicación. Entonces se puede dar la siguiente definición recursiva de multiplicación: $$1\cdot b=b\,\,\,\,\,\,\,\,\, (a+1)\cdot b =a\cdot b +b$$
          Demostrar lo siguiente (¡en el orden indicado!)
          \begin{center}
             \begin{tabular}{rcll}
                $a\cdot (a+c)$&$=$&$a\cdot b + a\cdot c$&Utilizar inducción sobre $a$\\
                $a \cdot 1$&$=$&$a$&\\
                $a \cdot b$&$=$&$b\cdot a$&lo anterior era el caso $b=1$\\\\
             \end{tabular}
          \end{center}
          Demostración.-\; Al ser 
	  \begin{center}
	      \begin{tabular}{rcll}
		  $1\cdot (b+c)$&$=$&$b+c$&\\
		  &$=$&$1\cdot b + 1 \cdot c$&por definición,\\
	      \end{tabular}
	  \end{center}
	  el primer resultado es válido para $a=1$. Supóngase que $a\cdot(b+c)=a\cdot b + a\cdot c$ para todo $b$ y $c$. Entonces,
	  \begin{center}
	      \begin{tabular}{rcl}
		  $(a+1)\cdot (b+c)$&$=$&$a\cdot (b+c)+(b+c)$\\
		  &$=$&$(a\cdot b + a\cdot c)+(b+c)$\\
		  &$=$&$(a\cdot b + b) + (a\cdot c + c)$\\
		  &$=$&$(a+1)\cdot b + (a+1)\cdot c$\\
	      \end{tabular}
	  \end{center}
	  La ecuación $a\cdot 1 = a$ vale para $a=1$ por definición. Supóngase que $a\cdot 1 = a$. Entonces
	  \begin{center}
	      \begin{tabular}{rcl}
		  $(a+1)\cdot 1$&$=$&$a\cdot 1 + 1 \cdot 1$\\
		  &$=$&$a+1$\\
	      \end{tabular}
	  \end{center}
	  Para $b=1$, la ecuación $a\cdot b=b\cdot a$ es consecuencia de $a\cdot 1=a$, que acaba de ser demostrada, y de $1\cdot a = a$, que vale por definición. Supóngase que $a\cdot b=b\cdot a$, entonces,
	  \begin{center}
	      \begin{tabular}{rcl}
		  $a\cdot (b+1)$&$=$&$a\cdot b + a\cdot 1$\\
		  &$=$&$a\cdot b + a$\\
		  &$=$&$b\cdot a + a$\\
		  &$=$&$(b+1)\cdot a$\\\\
	      \end{tabular}
	  \end{center}

      %------------------------------------------25.------------------------------------------------
       \item En este capítulo hemos empezado con los números naturales y gradualmente hemos ido ampliando hasta los reales. Un estudio completamente riguroso de este proceso requiere de por sí un pequeño libro. Nadie ha encontrado la manera de llegar a los números reales como dados, entonces los números naturales pueden ser definidos como los números naturales de la forma $1,1+1,1+1+1$, etc. Todo el objeto de este problema consiste en hacer ver que existe una manera matemática riguroso de decir \textbf{etc.} 
          \begin{enumerate}[\bfseries (a)]
             %--------------------(a)---------------------
             \item Se dice que un conjunto $A$ de números reales es \textbf{inductivo} si
                \begin{enumerate}[\bfseries (i)]
		    %----------(i)----------
                   \item $\mathbb{R}$ es inductivo.\\\\
                      Demostración.-\;  Está claro según la definición.\\\\
		    %----------(ii)----------
                   \item El conjunto de los números reales positivos es inductivo.\\\\
                      Demostración.-\; Esto está claro, ya que $1$ es positivo, y si $k$ es positivo, entonces por definición $k+1$ es positivo.\\\\
		    %---------(iii)---------
                   \item El conjunto de los números reales positivos distintos de $\dfrac{1}{2}$ es inductivo.\\\\
                      Demostración.-\; Está claro que $1$ pertenece a este conjunto. Si para el mismo no se cumpliera la condición $2$, existiría entonces en el conjunto algún $k$ con $k+1=1/2$. Pero esto es falso, ya que $k=-1/2$ no es positivo.\\\\
		    %----------(iv)----------
                   \item El conjunto de los números reales positivos distintos de $5$ no es inductivo.\\\\
                      Demostración.-\; Este conjunto contiene $4$, pero no $4+1$.\\\\
		    %----------(v)----------
                   \item Si $A$ y $B$ son inductivos, entonces el conjunto $C$ de los números reales que están a la vez en $A$ y en $B$ es también inductivo.\\\\
                      Demostración.-\; Al estar $1$ en $A$ y en $B$, también está en $C$. Si $k$ está en $C$, entonces $k$ está a la vez en $A$ y en $B$, con lo que $k+1$ está en $A$ y en $B$, de modo que $k+1$ está en $C$.\\\\
                \end{enumerate}

            %--------------------(b)------------------------
             \item Un número real $n$ será llamado \textbf{número natural} si $n$ está en todo conjunto inductivo.
                \begin{enumerate}[\bfseries (i)]
		    %----------(i)----------
                   \item Demostrar que $1$ es un número natural.\\\\
                      Demostración.-\; $1$ es un número natural, puesto que $1$ está en todo conjunto inductivo, por la misma definición de conjunto inductivo.\\\\

	 	    %----------(ii)---------
                   \item Demostrar que $k+1$ es un  número natural si $k$ es un número natural.\\\\
                      Demostración.-\; Si $k$ es un número natural, entonces $k$ está en todo conjunto inductivo. Así pues, $k+1$ está en todo conjunto inductivo. Por lo tanto, $k+1$ es un número natural.\\\\
                \end{enumerate}
          \end{enumerate}

          %----------------------------------26.------------------------------------------
       \item Un rompecabezas consiste en disponer de tres vástagos cilíndricos, el primero de los cuales lleva engastados $n$ anillos concéntricos de diámetro decreciente. Se puede quitar el anillo superior de un vástago para engastarlo sobre otro vástago siempre que al hacer esto último el anillo desplazado no venga a caer sobre otro de diámetro inferior. Por ejemplo, si el anillo más pequeño se pasa al vástago 2 y el que le sigue pasar también al vástago 3 encima del que le sigue en tamaño. Demostrar que la pila completa se puede pasar al vástago 3 en $2_n+1$ pasos  y no en menos.\\\\
          Demostración.-\; Si hay solo $n=1$ anillos, claramente se puede mover al eje $3$ en $1=2^1 - 1$ movimientos. Suponiendo el resultado para $k$ anillos, luego dados $k+1$ anillos,
	  \begin{enumerate}[\bfseries (a)]
	  	\item Mueve los anillos elevados a la $k$ al eje 2 en $2k-1$ movimientos,
		\item mueva el anillo inferior al eje 3,
		\item mueva los $k$ anillos superiores de nuevo al eje $3$ en movimientos $2k-1$.
	  \end{enumerate}
	Esto toma $2(2k - 1) + 1 = 2k + 1 - 1$ se mueve. Si $2k - 1$ movimientos es el mínimo posible para $k$ anillos, luego $2k + 1- 1$ es el mínimo para $k + 1$ anillos, ya que la parte inferior El anillo no se puede mover en absoluto hasta que los primeros $k$ anillos se muevan a algún lugar, tomando al menos $2k - 1$ se mueve, el anillo inferior debe moverse al eje $3$, tomando al menos $1$ movimiento, y luego los otros anillos deben colocarse encima, tomando al menos otros movimientos $2^k  - 1$.\\\\

          %.-------------------------------27.------------------------------------------
       \item Hubo un tiempo en que la universidad $B$ se preciaba de tener $17$ profesores numerarios de matemáticas. La tradición obligaba a que el almuerzo comunitario semanal, al que concurrían fielmente los $17$, todo miembro que hubiese descubierto un error en una de sus publicaciones tenia que hacer público este hecho y a continuación dimitir. Una declaración de este tipo no se había producido nunca porque ninguno de los profesores era consciente de la existencia de un error en su propio trabajo. Lo cual, sin embargo, mo quiere decir que no existieran errores. De hecho, en el transcurso de los años, por lo menos un error había sido descubierto en el trabajo de cada uno de los miembros por otro de entro ellos.La existencia de este error había sido comunicada a todos los demás miembros del departamento salvo al responsable, con objeto de evitar dimisiones.\\
          Llegó un fatídico año en que el departamento aumentó el número de sus miembros con un visitante de otra universidad, un Profesor $X$ que venía con la esperanza de que se le ofreciera un puesto permanente al final del año académico. Una vez que vio frustrada su esperanza, el Profesor $X$ tomó su venganza en el último almuerzo comunitario del año diciendo: Me ha sido muy grata mi estancia entre ustedes, pero hay una cosa que creo que es mi deber comunicarles. Por lo menos uno de entre ustedes tiene publicado un resultado incorrecto, lo cual ha sido descubierto por otro del departamento. ¿Qué ocurrió al año siguiente?\\\\
          Respuesta.-\; Primero Suponga que solo hay $2$ profesores $A$ y $B$, cada uno consciente del error en el trabajo del otro, pero sin darse cuenta de cualquier error en el suyo. Entonces ninguno se sorprende por la declaración del profesor $X$, pero cada uno espera que el otro sea sorprendido, y dimitir en el primer almuerzo del próximo año. Cuando esto no suceda, cada uno se da cuenta que esto solo puede ser porque él también ha cometido un error. Entonces, el la próxima reunión, ambos renunciaran.\\
	  A continuación, considere el caso de 3 profesores, $A$, $B$ y $C$. El profesor $C$ sabe que el profesor $A$ es consciente de un error en el trabajo del profesor $B$, ya sea porque el profesor $A$ encontró el error e informó, o porque encontró el error e informó al profesor $A$. Del mismo modo, él sabe que el profesor $B$ sabe que hay un error en el trabajo del profesor $A$. Pero el profesor $C$ piensa que no a cometido errores, por lo que a el respecta, la situación frente a los profesores $A$ y $B$ es precisamente el analizado en el párrafo anterior. El profesor $C$ está asumiendo, de que nadie cree que exista un error cuando uno no lo hace. Entonces el profesor $C$ espera tanto al profesor $A$ como al profesor $B$ renunciar en la segunda reunión. Por supuesto de manera similar los profesores $A$ y $B$ esperan que los otros dos renuncien en la segunda reunión. Cuando nadie renuncia todos se dan cuenta de que ha cometido un error, por lo que todos renuncian en la tercera reunión. Podría ser demostrado por inducción.\\\\ 

          %-------------------------------------28--------------------------------------------
       \item Después de imaginarse, o de consultar, la solución del problema 27, considere lo siguiente: Cada uno de los miembros del departamento era ya sabedor de lo que el Profesor $X$ afirmaba. ¿Cómo pudo pues su afirmación cambiar las cosas?\\\\
          Respuesta.-\; Ganar es una buena idea comenzar con el caso en el que el departamento consta solo de Profesores $A$ y $B$. Ahora, por supuesto, ambos profesores saben que alguien ha publicado un resultado incorrecto, pero el Profesor $A$ piensa que el Profesor $B$ no lo sabe, y viceversa. Una vez que el Profesor $X$ hace su anuncio, el Profesor $A$ sabe que el Profesor $B$ lo sabe. Y por eso espera que el Profesor $B$ renuncie en la próxima reunión. En el caso de tres profesores, la situación es más complicada. Cada uno sabe que alguien ha cometido un error, y además cada uno sabe que los demás saben Por ejemplo, el Profesor $C$ sabe que el Profesor $A$ lo sabe, ya que él y el Profesor $A$ han discutido el error en el trabajo del Profesor $B$, y él sabe de manera similar que el Profesor $B$ lo sabe. Pero el Profesor $C$ no cree que el Profesor $A$ sepa que el Profesor $B$ lo sabe. Así que el anuncio del Profesor $X$ cambia las cosas: ahora el Profesor $C$ sabe que el Profesor $A$ sabe que el Profesor $B$ sabe. Bueno, puedes ver lo que pasa en general. Esto parece probar que las declaraciones como $A$ sabía que $B$ sabía que $C$ sabía que realmente tiene sentido.\\\\


    \end{enumerate}
    



    %---------- funciones
	%\chapter{Funciones}


    %definición 1.1
\begin{tcolorbox}
    \begin{def.}
	El conjunto de los números a los cuales se aplica una función recibe el nombre de \textbf{dominio} de la función.
    \end{def.}
\end{tcolorbox}

    %definición 1.2
\begin{tcolorbox}
    \begin{def.}
	Si $f$ \; y \; $g$ son dos funciones cualesquiera, podemos definir una nueva función $f+g$ denominada \textbf{suma} de $f+g$ mediante la ecuación:
	$$(f+g)(x)=f(x)+g(x)$$
	Para el conjunto de todos los $x$ que están a la vez en el dominio de $f$ y en el dominio de $g$, es decir: $$dominio \; (f+g)=dominio \; f \; \cap \; dominio \; g$$\\
    \end{def.}
\end{tcolorbox}

    %definición 1.3
\begin{tcolorbox}
    \begin{def.}
	El dominio de $f \cdot g$ es $dominio \; f \cap \; dominio \; g$ $$(f \cdot g)(x)=f(x)\cdot g(x)$$\\
    \end{def.}
\end{tcolorbox}

    %definición 1.4 
\begin{tcolorbox}
    \begin{def.}
	Se expresa por dominio $f$ $\cap$ dominio $g$ $\cap$ $\lbrace x:g(x)\neq 0 \rbrace$
	$$\left( \dfrac{f}{g}\right) (x)=\dfrac{f(x)}{g(x)}$$ \\
    \end{def.}
\end{tcolorbox}

%definición 1.5
\begin{tcolorbox}
    \begin{def.}[Función constante]
	$$(c \cdot g)(x)=c \cdot g(x)$$\\
    \end{def.}
\end{tcolorbox}

%teorema 1.1
\begin{teo}
    $(f+g)+h=f+(g+h)$\\\\
    Demostración.- \; \textbf{La demostración es característica de casi todas las demostraciones que prueban que dos funciones son iguales: se debe hacer ver que las dos funciones tienen el mismo dominio y el mismo valor para cualquier número del dominio.} Obsérvese que al interpretar la definición de cada lado se obtiene:
    \begin{center}
	\begin{tabular}{r c l}
	    $\left[ (f+g) + h \right](x)$&=&$(f+g)(x)+h(x)$\\
	    &=&$\left[ f(x) +g(x) \right] +h(x)$\\\\
	    &y&\\\\
	    $\left[ f+(g+h) \right](x)$&=&$f(x)+(g+h)(x)$\\
	    &=&$f(x)+\left[ g(x)+h(x) \right]$\\\\
	\end{tabular}
    \end{center}
    Es esta demostración no se ha mencionado la igualdad de los dos dominios porque esta igualdad parece obvia desde el momento en que empezamos a escribir estas ecuaciones: el dominio de $(f+g)+h$ y el de $f+(g+h)$ es evidentemente dominio $f$ $\cap$ dominio $g$ $\cap$ dominio $h$. Nosotros escribimos, naturalmente $f+g+h$ por $(f+g)+h=f+(g+h)$\\\\
\end{teo}

%teorema 1.2
\begin{teo}
    Es igual fácil demostrar que $(f\cdot g)\cdot g=f\cdot (g \cdot h)$ y ésta función se designa por $f \cdot g \cdot h$. Las ecuaciones $f+g=g+f$ \; y \; $f\cdot g=g \cdot f$ no deben presentar ninguna dificultad.\\\\
\end{teo}

%definición 1.6
\begin{tcolorbox}
    \begin{def.}[Composición de función]
	$$(f \circ g)(x)=f(g(x))$$
	El dominio de $f\circ g$ es $\lbrace $ $x$ : $x$ está en el dominio de $g$ \: y \; $g(x)$ está en el dominio de $f$ $\rbrace$
	$$D_{f \circ g}= \lbrace x \; / \; x \in D_g \; \land \; g(x)\in D_f \rbrace$$
    \end{def.}
\end{tcolorbox}

    %propiedad 1.1
\begin{tcolorbox}
    \begin{prop}
    $(f \circ g) \circ h = f \circ (g \circ h)$   La demostración es una trivalidad.
    \end{prop}
\end{tcolorbox}

%definición  1.7 
\begin{tcolorbox}
    \begin{def.} 
	Una \textbf{función} es una colección de pares de números con la siguiente propiedad: Si $(a,b)$ \; y \; $(a,c)$ pertenecen ambos a la colección, entonces $b=c$; en otras palabras, la colección no debe contener dos pares distintos con el mismo primer elemento.\\
    \end{def.}
\end{tcolorbox}

%definición 1.8
\begin{tcolorbox}
    \begin{def.} 
	Si $f$ es una función, el \textbf{dominio} de $f$ es el conjunto de todos los $a$ para los que existe algún $b$ tal que $(a,b)$ está en $f$. Si $a$ está en el dominio de $f$, se sigue de la definición de función que existe, en efecto, un número $b$ único tal que $(a,b)$ está en $f$. Este $b$ único se designa por $f(a)$.\\  
    \end{def.}
\end{tcolorbox}


\section{Problemas}

    \begin{enumerate}[\bfseries 1.]

	%--------------------1.
	\item Sea $f(x)=1/(1+x)$. Interpretar lo siguiente:
	    \begin{enumerate}[\bfseries (i)]

		%----------(i)
		\item $f(f(x))$ (¿Para que $x$ tiene sentido?)\\\\
		Respuesta.- \; Sea $f\left( \dfrac{1}{1+x} \right)$ entonces $\dfrac{1}{1 + \dfrac{1}{1+x}}$, por lo tanto $\dfrac{1-x}{x+2}$ de donde llegamos a la conclusión de que $x$ se cumple para todo número real de $1$ y $-2$\\\\

		%----------(ii)
		\item $f\left( \dfrac{1}{x} \right)$\\\\
		Respuesta.- \; $\dfrac{1}{1 + \dfrac{1}{x}}=\dfrac{1}{\dfrac{x+1}{x}}=\dfrac{x}{x+1}$ por lo tanto se cumple para todo $x\neq -1, 0$\\\\

		%----------(iii)
		\item $f(cx)$\\\\
		Respuesta.- \; $\dfrac{1}{1+cx}$ donde se cumple para todo $x\neq -1$ si $c\neq 0$\\\\

		%----------(iv)
		\item $f(x+y)$\\\\
		Respuesta.- \; $\dfrac{1}{1+x+y}$ donde se cumple para todo $x+y\neq-1$\\\\

		%----------(v)
		\item $f(x) + f(y)$\\\\
		Respuesta.- \; $\dfrac{1}{1+x} + \dfrac{1}{1+y}=\dfrac{x+y+2}{(1+x)(1+y)}$ siempre y cuando $x\neq-1$ y $y \neq -1$\\\\

		%----------(vi)
		\item ¿Para que números $c$ existe un número $x$ tal que $f(cx)=f(x)$?\\\\
		Respuesta.- \; Para todo $c$ ya que $f(c\cdot 0)=f(0)$\\\\

		%----------(vii)
		\item ¿Para que números $c$ se cumple que $f(cx)=f(x)$ para dos números distintos $x$?\\\\
		Respuesta.- \;  Solamente $c=1$ ya que $f(x)=f(cx)$ implica que $x=cx$, y esto debe cumplirse por lo menos para un $x \geq 1$\\\\

	    \end{enumerate}

	%--------------------2.  
	\item Sea $g(x)=x^2$ y sea 
	\begin{equation*}
	    h(x) = \left\lbrace
		\begin{array}{rl}
		    0, & x \; racional\\
		    1, & x \; irracional
		\end{array}
	    \right.
	\end{equation*}

	\begin{enumerate}[\bfseries (i)]

	    %----------(i)
	    \item ¿Para cuáles $y$ es $h(y) \leq y$?\\\\
	    Respuesta-. \; Se cumple para $y\geq 0$ si $y$ es racional, o para todo $y\geq 1$\\\\

	    %----------(ii)
	    \item ¿Para cuáles $y$ es $h(y) \leq g(y)$?\\\\
	    Respuesta-. \; Para $-1\leq y \leq 1$ siempre que $y$ sea racional y para todo $y$ tal que $|y|\leq1$\\\\

	    %----------(iii)
	    \item ¿Qué es $g(h(z)) - h(z)$?\\\\
	    Respuesta-. \; 
	    \begin{equation*}
		g(h(z)) = \left\lbrace
		    \begin{array}{rl}
			0, & z^2 \; racional\\
			1, & z^2 \; irracional
		    \end{array}
		\right.
	    \end{equation*}

	    Por lo tanto el resultado es $0$\\\\

	    %----------(iv)
	    \item ¿Para cuáles $w$ es $g(w)\leq w $?\\\\
	    Respuesta-. \; Para todo $w$ tal que $0\leq w\leq 1$\\\\

	    %----------(v)
	    \item ¿Para cuáles $\epsilon$ es $g(g(\epsilon)) = g(\epsilon)$?\\\\
	    Respuesta-. \; Para $-1,0,1$\\\\ 

	\end{enumerate}

	%--------------------3.
	\item Encontrar el dominio de las funciones definidas por las siguientes fórmulas:
	    \begin{enumerate}[\bfseries (i)]

	    %----------(i)
	    \item $f(x)=\sqrt{1-x^2}$\\\\
	    Respuesta.- \; Por la propiedad de raíz cuadrada, se tiene  $1-x^2 \geq 0$ entonces $x^2 \leq 1$ por lo tanto el dominio son todos los $x$ tal que $|x| \leq 1$\\\\

	    %-----------(ii)
	    \item $f(x)=\sqrt{1-\sqrt{1-x^2}}$\\\\
	    Respuesta.- \; Se observa claramente que el dominio es $-1\leq x \leq 1$\\\\

	    %-----------(iii)
	    \item $f(x)=\dfrac{1}{x-1} + \dfrac{1}{x-2}$\\\\
	    Respuesta.- \; Operando un poco tenemos $$f(x) = \dfrac{2x-3}{(x-1)(x-2)},$$ sabemos que el denominador no puede ser $0$ por lo tanto el $D_{f} = \lbrace x\; / \; x \neq 1, \; x\neq  2 \rbrace$\\\\ 
    
	    %-----------(iv)
	    \item $f(x)=\sqrt{1-x^2} + \sqrt{x^2-1}$\\\\
	    Respuesta.- \; Claramente notamos que el dominio de $f$ son $-1$ y $1$ ya que si se toma otros números daría un número imaginario.\\\\

	    %-----------(v)
	    \item $f(x)=\sqrt{1-x}+\sqrt{x-2}$\\\\
	    Respuesta.- \; Notamos que no se cumple para ningún $x$ ya que si $0\leq x \leq 1$ entonces no se cumple para $\sqrt{x-2} $ y si $x\geq 2$ no se cumple para $\sqrt{1-x}$\\\\ 

	    \end{enumerate}

	%--------------------4.
	\item Sean $S(x)=x^2,$ $P(x)=2^x$ y $s(x)=sen x$. Determinar los siguientes valores. En cada caso la solución debe ser un número.\\\\

	\begin{enumerate}[\bfseries (i)]

	    %----------(i)
	    \item $(S \circ P)(y)$\\\\
	    Respuesta.- \; Por definición se tiene que $(S \circ P)(y)=S(P(y))$ entonces $S(2^y)=2^{2y}$ siempre y cuando $D_{S \circ P}=\lbrace y / y \in D_P \land P(y) \in D_S\rbrace$\\\\
	    
	    %----------(ii)
	    \item $(S \circ s)(y)$\\\\
	    Respuesta.- \; Por definición tenemos que $(S \circ s)(y)=S(s(y))$ entonces $S(\sen y)=\sen^2 y$ siempre y cuando $D_{S \circ s}=\lbrace y / y \in D_s \land S(y) \in D_S \rbrace$\\\\

	    %----------(iii)
	    \item $(S \circ P \circ s)(t)+(s \circ P)(t)$\\\\
	    Respuesta.- \; $(S \circ P \circ s)(t)+(s \circ P)(t) = S((P \circ s)(t))+s(P(t)) = S(P(s(t))) + s(P(t))=S(P(\sen t)) + s(2^t)=S(2^{\sen t}) +  \sen 2^t = 2^{2 \sen t} +\sen 2^t$\\\\  

	    %----------(iv)
	    \item $s(t^3)$\\\\
	    Respuesta.- \; $s(t^3)=\sen t^3$\\\\

	\end{enumerate}

	    %--------------------5.
	\item Expresar cada una de las siguientes funciones en términos de $S,P,s$ usando solamente $+,\cdot , \circ$\\\\
	\begin{enumerate}[\bfseries (i)]
	    
	    %----------(i)
	    \item $f(x)=2^{\sen x}$\\\\
	    Respuesta.- \; Claramente vemos que $P \circ s$\\\\ 

	    %----------(ii)
	    \item $f(x) = \sen 2^x$\\\\
	    Respuesta.- \; $s \circ P$\\\\

	    %----------(iii)
	    \item $f(x) = \sen x^2$\\\\
	    Respuesta.- \; $s \circ S$\\\\

	    %----------(iv)
	    \item $f(x) = \sen x$\\\\
	    Respuesta.- \; $S \circ s$\\\\

	    %----------(v)
	    \item $f(t) = 2^{2t}$\\\\
	    Respuesta.- \; $P \circ P$\\\\

	    %----------(vi)
	    \item $f(u)=\sen (2^u + 2^{u^2})$\\\\
	    Respuesta.- \; $s \circ (P + P \circ S)$\\\\

	    %----------(vii)
	    \item $f(y) = \sen (\sen (\sen (2^{2^{2^{\sen y}}})))$\\\\
	    Respuesta.- \; $s \circ s \circ s \circ P \circ P \circ P \circ s$\\\\

	    %----------(viii)
	    \item $f(a)= 2^{\sen^2 a} +  \sen(a^2) + 2^{sen(a^2 + \sen a)}$\\\\
	    Respuesta.- \; $P \circ S \circ s  + s \circ S + P \circ s \circ (S + s)$\\\\

	\end{enumerate}

	%--------------------6.
	\item 
	\begin{enumerate}[\bfseries (a)]

	    %----------(a)
	    \item Si $x_1, ... , x_n$ son números distintos, encontrar una función polinómica $f_i$ de grado $n-1$ que tome el valor $1$ en $x_i$ y $0$ en $x_j$ para $j \neq i.$ Indicación: El producto de todos los $(x-x_j)$ para $j \neq i$ es $0$ en $x_j$ si $j \neq i.$ Este producto es designado generalmente por $$\prod\limits_{j=1_{j \neq i}}^n (x-x_j)$$ donde el símbolo $\prod$ (pi mayúscula) desempeña para productos el mismo papel que $\sum$ para sumas.\\\\

	Respuesta.- \; Una forma de pensar sobre esta pregunta es considerar una solución fija $n$ y elegir un conjunto de distintas $x_1, x_2, ..., x_n$. Por ejemplo supongamos que elegimos $n=3$ $x_1=1$, $x_2=2$, $x_3 = 3.$ Entonces supongamos que queremos encontrar un polinomio $f_i(x_1)=f_1(1)=1,$ pero $f_1(x_2)=f_1(2)=f_1(3)=0.$ Es decir, $F_1$ es un cuadrático que tiene ceros en $x=2$ \; y\; $x=3$, pero es igual a $1$ en $x=1.$ Naturalmente, esto sugiere mirar un polinomio de la forma $$a(x-2)(x-3),$$ para que la igualdad sea igual a $1$ por alguna constante $a.$ Pero, ¿Qué es esta constante? Bueno, si nos conectamos con $x=1$, debemos tener $$f_1(1)=1=a(x-2)(x-3)=2a,$$ por lo tanto $a=1/2$ y la solución deseada es $$f_1(x)=\dfrac{1}{2}(x-2)(x-3).$$ Del mismo modo, si tratamos de encontrar un polinomio $f_2(x)$ tal que $f_2(2)=1$ con raíces en $x=1,3$ tendríamos que resolver la ecuación $1=a(2-1)(2-3),$ lo que da $a=-1$ por lo tanto $f_2(x)=-(x-1)(x-3)$\\
	    Ahora veamos el caso general. El polinomio $f_i(x)$ satisface $f_i(x_i)$ \; y \; $f_i(x_j)=0$ para todo $j \neq i$, entonces debe tomar la forma 
	    \[ f_i(x)=a \prod_{j \neq i} (x-x_j)  \]
	    Para alguna constante $a$. Para encontrar esta constante, aplicamos $x=x_1$:
	    \[ f_i(x_i)=1=a\prod_{j \neq i} (x_i-x_j), \]
	    por lo tanto:
	    \[ a= \dfrac{1}{\displaystyle\prod_{j \neq i} (x_i-x_j)}  \]  
	    Así queda
	    \[ f_i(x)= \prod_{j \neq i} \dfrac{(x-x_j)}{(x_i-x_j)} \]\\\\

	    %----------(b)
	    \item Encontrar ahora una función polinómica de grado $n-1$ tal que $f(x_1)=a_1,$ donde $a_1,...,a_n$ son números dados. (Utilícense las Funciones $f_1$ de la parte $(a)$.) La fórmula que se obtenga es la llamada \textbf{Fórmula de interpolación de Lagrange}\\\\

	    Respuesta.- \;Sea \[ f(x) = \sum_{j=1} a_i f_i(x) \] entonces  \[ f(x) = \sum_{j=1} a_i \prod_{j \neq i} \dfrac{(x-x_j)}{(x_i-x_j)} \]\\\\ 

	\end{enumerate}

	%--------------------7.
	\item
	\begin{enumerate}[\bfseries (a)]

	    %----------(a)
	    \item Demostrar que para cualquier función polinómica $f$ y cualquier número $a$ existe función polinómica $g$ y un número $b$ tales que $f(x)=(x-a)g(x)+b$ para todo $x$. (La idea es esencialmente dividir $f(x)$ por $(x-a)$ mediante la división larga hasta encontrar un resto constante.)\\\\
	    Demostración.- \; Si el grado de $f$ es $1$, entonces $f$ es de la forma $$f(x)=cx+d=cx+d +ac-ac=c(x-a)+(d+ac)$$ de tal modo que $g(x)=c$ y $b=d+ac$. Por inducción supongamos que el resultado es válido para polinomios de grado $\leq k.$ Si $f$ tiene grado $k+1$, entonces $f$ tiene la forma $$f(x)=a_{k+1}x^{k+1} + ... + a_1x + a_0$$ luego para grados $\leq k$ se tiene $$f(x)-a_{k+1}x^{k+1} =(x-a)g(x)+b$$ así $$f(x) = (x-a)\left[g(x) + a_{k+1}(x-a)^k\right]+b$$\\\\

	    %----------(b)
	    \item Demostrar que si $f(a)=0$, entonces $f(x)=(x-a)g(x)$ para alguna función polinómica $g$. (La reciproca es evidente)\\\\
	    Demostración.- \; Por la parte $(a)$, podemos poner que $f(x)=(x-a)g(x)+b$, entonces $$0=f(a)=(a-a)g(a)+b=b$$ de modo que $f(x)=(x-a)g(x)$\\\\

	    %----------(c)
	    \item Demostrar que si $f$ es una función polinómica de grado $n$, entonces $f$ tiene a lo sumo $n$ raíces, es decir, existen a lo sumo $n$ números $a$ tales que $f(a)=0$\\\\
	    Demostración.- \; Supóngase que $f$ tiene $n$ raíces $a_1,...,a_n$. Entonces según la parte $(b)$ podemos poner $f(x)(x-a)g_1(x)$ donde el grado de $g_1(x)$ es $n-1$. Pero $$0=f(a_2)=(a_2-a_1)g_1(a_2)$$ de modo que $g_1(a_2)=0$, ya que $a_2 \neq a_1$. Podemos pues escribir $$f(x)(x-a_2)g_2(x),$$ donde el grado de $g_2$ es $n-2$. Prosiguiendo de esta manera, obtenemos que $$f(x)=(x-a_1)(x-a_2)\cdot ... \cdot (x-a_n)c$$ para algún número $c \neq 0.$ Está claro que $f(a)\neq 0$ si $a \neq a_1,...,a_n.$ Así pues, $f$ puede tener a lo sumo $n$ raíces.\\\\

	    %----------(d)
	    \item Demostrar que para todo $n$ existe una función polinómica de grado $n$ con raíces. Si $n$ es par, encontrar una función polinómica de grado $n$ sin raíces, y si $n$ es impar, encontrar una con una sola raíz\\\\
	    Demostración.- \; Si $f(x)=(x-1)(x-2)\cdot ... \cdot (x-n),$ entonces $f$ tiene $n$ raíces. Si $n$ es par, entonces $f(x)=x^n + 1$ no tiene raíces. Si $n$ es impar, entonces $f(x)=x^n$ tiene una raíz única, que es $0.$\\\\

	\end{enumerate}

	%-----------------8.
	\item ¿Para qué números $a,b,c$ y $d$ la función 
	$$f(x)=\dfrac{ax+d}{cx+b}$$ 
	satisface $f(f(x))=x$ para todo $x$?\\\\
	Respuesta.-\; Si $$x=f(f(x))=\dfrac{a\left(\dfrac{ax+b}{cx+d}\right)+b}{c\left(\dfrac{ax+b}{cx+d}\right)+d}$$ para todo $x$, entonces $$x=\dfrac{a^2x+ab+bcx+bd}{acx+bc+cdx+d^2}$$ y por lo tanto $$\left(ac+cd\right)x^2 + \left(d^2-a^2\right)x - ab - bd = 0$$ para todo $x$, de modo que 
	\begin{center}
	    \begin{tabular}{rcl}
		$ac+cd$&=&$0$\\
		$ab+bd$&=&$0$\\
		$d^2-a^2$&=&$0$\\
	    \end{tabular}
	\end{center}
	Se sigue que $a=d$ ó $a=-d.$ Una posibilidad es $a=d=0$, en cuyo caso $f(x)=\dfrac{b}{cx}$ que satisface $f(f(x))=x$ para todo $x\neq 0$. Si $a=d\neq 0$, entonces $b=c=0$ con lo que $f(x)=x$. La tercera posibilidad es $a+d=0$, de modo que $f(x)=\dfrac{ax+b}{cx-a}$, la cual satisface $f(f(x))=x$ para todo $x\neq \dfrac{a}{c}$ la cual satisface $f(f(x))=x$ para todo $x\neq \dfrac{a}{c}$. Estrictamente hablando, podemos añadir la condición $f(x)\neq \dfrac{a}{c}$ para $x\neq \dfrac{a}{c}$, lo que significa que $$\dfrac{ax+b}{cx-a}\neq \dfrac{a}{c}, \; ó \; a^2+bc\neq 0.$$\\\\

	%--------------------9.
	\item 
	\begin{enumerate}[\bfseries (a)]

	    %----------(a)
	    \item Si $A$ es un conjunto cualquiera de números reales, defínase una función $C_A$ como sigue:  
		   $$C_A(x) = \left\lbrace
			\begin{array}{c l}
			    $1$,&si \; $x$ \; está \; en \;$A$\\
			    $0$,&si \; $x$\; no\; está \; en \;  $A$
			\end{array}
			    \right.$$
		    Encuéntrese expresiones para $C_{A\cap B}, \; C_{A \cup B}$ y $C_{\mathbb{R}-A}$, en términos de $C_A$ y $C_B$. \\\\
	    Respuesta.-\; Según la definición de teoría de conjunto tenemos,
	    \begin{center}
		\begin{tabular}{rcl}
		    $C_{A\cap B}$&$=$&$C_A \cdot C_B$\\
		    $C_{A\cup B}$&$=$&$C_A + C_B - C_A \cdot C_B$\\
		    $C_{\mathbb{R} - A}$&$=$&$1 - C_A$\\\\
		\end{tabular}
	    \end{center}
	    
	    %----------(b)
	    \item Supóngase que $f$ es una función tal que $f(x)=0$ o $1$ para todo $x$. Demostrar que existe un conjunto $A$ tal que $f=C_A$\\\\
	    Demostración.-\; Sea $A=\lbrace x\in \mathbb{R}: f(x)=1\rbrace$ , entonces $f=C_A$.\\\\

	    %----------(c)
	    \item Demostrar que $f=f^2$ si y sólo si $f=C_A$ para algún conjunto $A$\\\\
	    Demostración.-\; Sea $f=f^2$, entonces para cada real $x$, $f(x)=f[f(x)]^2$, así $f(x)=0$ó $f(x)=1$, luego por la parte $b)$, $f=C_A$ para algún $A$.\\
	    Por otro lado sea $f=C_A$ para algún $A$. Entonces si $x\in A$, $f(x)=1=1^2=f(x)^2$, mientras si $x\notin A,$ $f(x)=0=0^2=f(x)^2$, así en cualquier caso $f(x)=[f(x)]^2$ y $f=f^2$\\\\

	\end{enumerate}

	%--------------------10.
	\item 
	\begin{enumerate}[\bfseries (a)]

	    %----------(a)
	    \item ¿Para qué funciones $f$ existe una función $g$ tal que $f=g^2$?\\\\
	    Respuesta.-\; Debido a que algún número elevado al cuadrado siempre será no negativo podemos afirmar que las funciones $f$ satisfacen a todo $x$ tal que $f(x)\geq 0$\\\\

	    %----------(b)
	    \item ¿Para qué función $f$ existe una función $g$ tal que $f=1/g$?\\\\
	    Respuesta.-\; Dado a que un número divido entre cero es indeterminado se ve claramente que satisfacen a todo $x$ tal que $f(x)\neq 0$\\\\

	    %----------(c)
	    \item ¿Para qué funciones $b$ y $c$ podemos encontrar una función $x$ tal que $$(x(t))^2 + b(t)x(t)+c(t)=0$$ para todos los números $t$?\\\\
	    Respuesta.-\; Por teorema se observa que para las funciones $b$ \; y \; $c$ que satisfacen $(b(t))^2 - 4c(t) \geq 0$ para todo $t$\\\\ 

	    %----------(d)
	    \item ¿Qué condiciones deben satisfacer las funciones $a$ y $b$ si ha de existir una función $x$ tal que $$a(t)x(t)+b(t)=0$$ para todos los números $t$? ¿Cuántas funciones $x$ de éstas existirán?\\\\
	    Respuesta.-\; Es facil notar que $b(t)$ tiene que ser igual a $0$ siempre que $a(t) = 0$. Si $a(t) \neq 0$ para todo $t$, entonces existe una función única con esta condición, que es $x(t) = a(t)/b(t)$. Si $a(t)=0$ para algún $t$, entonces puede elegirse arbitrariamente $x(t)$, de modo que existen infinitas funciones que satisfacen la condición.\\\\

	\end{enumerate}

	%--------------------11.
	\item 

	\begin{enumerate}[\bfseries (a)]

	    %----------(a)
	    \item Supóngase que $H$ es una función $e$ y un número tal que $H(H(y))=y$. ¿Cuál es el valor de $$H(H(H...(H(y))))?$$\\\\
	    Respuesta.-\; Si aplicamos la hipótesis, tendremos que aplicar $78$ veces la función, luego $76$ y así, hasta llegar a $2$, donde la función sera $H(H(y))$, y una vez más por hipótesis tenemos como resultado $y$.\\\\

	    %----------(b)
	    \item La misma pregunta sustituyendo $80$ por $81$\\\\
	    Respuesta.-\; Sea $H(H(y))$ la $78ava$ vez de la función, entonces la $81ava$ vez será $H(H(H(y))$, por lo tanto queda como resultado $H(y)$.\\\\ 

	    %----------(c)
	    \item La misma pregunta si $H(H(y))=H(y)$\\\\
	    Respuesta.-\; Análogamente a la parte $a)$ si la $80ava$ vez es $y$ entonces por hipótesis nos queda $H(y)$.\\\\ 

	    %----------(d)
	    \item Encuéntrese una función $H$ tal que $H(H(x))=H(x)$ para todos los números $x$ y tal que $H(1)=36$, $H(2)=\dfrac{\pi}{3}$, $H(13)=47$, $H(36)36$, $H(\pi / 3)\dfrac{\pi}{3}$, $H(47)=47$\\\\
	    Respuesta.-\; Dar a $H(l)$, $H(2)$, $H(13)$, $H(36)$, $H(\pi /3)$, y $H(47)$ los valores especificados y hágase $H(x) = 0$ para $x \neq 1, 2, 13, 36, \pi /3, 47.$ Al ser, en particular, $H(0) = 0$, la condición $H(H(x)) = H(x)$ se cumple para todo $x$.\\\\

	    %----------(e)
	    \item Encontrar una función $H$ tal que $H(H(x))=H(x)$ para todo $x$ y tal que $H(1)=7$, $H(17)=18$\\\\
	    Respuesta.-\; Hágase $H(1) = 7$, $H(7) = 7$, $H(17) = 18$, $H(18) = 18$, y $H(x) = 0$ para $x \neq l , 7, 17, 18$.\\\\

	\end{enumerate}

	%--------------------12.
	\item Una función $f$ es par si $f(x)=f(-x),$ e impar si $f(x)=-f(-x)$. Por ejemplo, $f$ es par si $f(x)=x^2$ ó $f(x)=|x|$ ó $f(x)=\cos x$, mientras que $f$ es impar si $f(x)=x$ ó $f(x)=\sen x$.

	\begin{enumerate}[\bfseries (a)]

	    %----------(a)
	    \item Determinar si $f+g$ es par, impar o no necesariamente ninguna de las dos cosas, en los cuatro casos obtenidos al tomar $f$ par o impar y $g$ par o impar. (Las soluciones pueden ser convenientemente dispuestas en una tabla $2$ x $2$)\\\\
	    Respuesta.-\; Sea $f(x)=x^2$ y $g(x)=|x|$ entonces $f(-x)+g(-x)=(-x)^2 + |-x| = x^2 + |x| = f(x) + g(x)$ por lo tanto par y par es par.\\
	    Sea $f(x) = x$ y $g(x)=x$ entonces $-f(-x) + (-g(-x)) = -(-x) + [-x(-x)] = x + x = f(x) + g(x)$, por lo tanto impar e impar es impar.\\
	    Los otros dos últimos se prueba fácilmente y se llega a la conclusión de que ni uno ni lo otro.
	    \begin{center}
		\begin{tabular}{c|cc}
		    &\textbf{Par}&\textbf{Impar}\\
		    \hline\\
		    \textbf{Par}&Par&Ninguno\\\\
		    \textbf{Impar}&Ninguno&Par\\
		\end{tabular}
	    \end{center}
	    \vspace{1cm}

	    %----------(b)
	    \item Hágase lo mismo para $f\cdot g$\\\\
	    Respuesta.-\; Sea $f(x)=x^2$ \; y \; $g(x)=|x|$, entonces $f(-x) \cdot g(-x) = x^2 \cdot |x| = f(x) \cdot g(x)$, por lo tanto se cumple para par y par.\\
	    Sea $f(x)=x$ \; y \; $g(x)=x$, entonces $-f(-x) \cdot -g(-x) = -(-x) \cdot -(-x) = x \cdot x = f(x) \cdot g(x)$, por lo tanto impar impar da impar\\
	    Sea $f(x)=x^2$ \; y \; $g(x)=x$, podemos crear otra función llamada $h$ que contiene a $x^2 \cdot x$ por lo tanto $h(x)=x^3 = -(-x)^2$ y así demostramos que par e impar es impar.\\
	    De igual forma al anterior se puede probar que impar y par es impar.
	    \begin{center}
		\begin{tabular}{c|cc}
		    &\textbf{Par}&\textbf{Impar}\\
		    \hline\\
		    \textbf{Par}&Par&Impar\\\\
		    \textbf{Impar}&Impar&Par\\\\
		\end{tabular}
	    \end{center}
	    \vspace{1cm}

	    %----------(c)
	    \item Hágase lo mismo para $f\circ g$\\\\
	    Respuesta.-\; Sea $f(x)=x$ \; y \; $g(x)=x$, luego $h(x)=(f \circ g)(x)$ entonces $h(x) = x$ luego $-f(-x) = x$, por lo tanto impar e impar da impar.\\
	    De similar manera se puede encontrar para los demás problemas y queda:
	    \begin{center}
		\begin{tabular}{c|cc}
		    &\textbf{Par}&\textbf{Impar}\\
		    \hline\\
		    \textbf{Par}&Par&Par\\\\
		    \textbf{Impar}&Par&Impar\\\\
		\end{tabular}
	    \end{center}
	    \vspace{1cm}

	    %----------(d)
	    \item Demostrar que para toda función par $f$ puede escribirse $f(x)=g(|x|)$, para una infinidad de funciones $g$.\\\\
	    Demostración.-\; Sea $g(x)=f(x)$ sabemos que $f$ es par si $f(x)=f(-x)$, de donde $g(x)=f(-x)$, luego  por definición de valor absoluto se tiene $g(|x|)=f(|-x|)$, y por lo tanto $f(x)=g(|x|)$\\\\

	\end{enumerate}

	%--------------------13.
	\item 
	\begin{enumerate}[\bfseries (a)]

	    %----------(a)
	    \item Demostrar que para toda función $f$ con dominio $\mathbf{R}$ puede ser puesta en la forma $f=E+O,$ con $E$ par y $O$ impar.\\\\
	Demostración.-\; Sea $E(x) = \dfrac{f(x) + f(-x)}{2}$ entonces $E(-x) = \dfrac{f(-x) + f(x)}{2} = E(x)$ donde vemos que $E(x)$ es una función par. Luego por hipótesis $O(x) = f(x) - E(x) = f(x) - \dfrac{f(x)+f(-x)}{2} = \dfrac{f(x)-f(-x)}{2}$, así $O(-x) = \dfrac{f(-x)-f(x)}{2} = -O(x)$ entonces $O(x)$ es impar. Por lo tanto queda demostrado la proposición.\\\\

	    %----------(b)
	    \item Demuéstrese que esta manera de expresar $f$ es única. (Si se intenta resolver primero la parte $(b)$ despejando $E$ y $O$, se encontrará probablemente la solución a la parte $(a)$)\\\\
	    Demostración.-\; Si $f=E+O$, siendo $E$ par y $O$ impar, entonces $$f(x)=E(x)+O(x)$$ $$f(-x)=E(x)-O(x)$$\\\\

	\end{enumerate}

	%--------------------14.
	\item Si $f$ es una función cualquiera, definir una nueva función $|f|$ mediante $|f|(x)=|f(x)|$. Si $f$ y $g$ son funciones, definir dos nuevas funciones, $max(f,g)$ y $min(f,g)$ mediante $$max(f,g)(x)=max(f(x),g(x)),$$ $$min(f,g)(x)=min(f(x),g(x))$$ Encontrar una expresión para $max(f,g)$ y $min(f,g)$ en términos de $| |$.\\\\
	Respuesta.-\; Por problema $1.13$ se tiene que $$max(f,g)=\dfrac{f+g+|f-g|}{2};$$ $$min(f,g)=\dfrac{f+g-|f-g|}{2}$$\\\\

	%--------------------15.
	\item 

	\begin{enumerate}[\bfseries (a)]
	    
	    %----------(a)
	    \item Demostrar que $f=max(f,0)+min(f,0)$. Esta manera particular de escribir $f$ es bastante usada; las funciones $max(f,0)$ y $min(f,0)$ se llaman respectivamente parte positiva y parte negativa de $f$\\\\
	    Demostración.-\; Esta proposición mostrará que se puede dividir una función en sus partes no negativas y no positivas. Es decir para todo los elementos $x$ de algún dominio, es cierto que el valor de la función $f$ en un punto $x$ es igual a la suma dada, que consiste en la parte no negativa de $max(f(x),0)$ y la parte no positiva de $f$, $min(f(x),0)$.\\
	    Para probarlo, lo dividiremos en dos casos. Sabemos que ó $f(x)\geq 0$ ó $f(x)\leq 0$. Si $f(x)\geq 0$ entonces $max(f(x),0)=f(x)$ y $min(f(x),0)=0$ por lo que nuestra ecuación se reduce a $f(x)=f(x)+0$. Por otro lado si $f(x)\leq 0,$ entonces $max(f(x),0)=0$ y $min(f(x),0)=f(x)$, por lo que nuestra ecuación se reduce a $f(x)=0+f(x)$.\\
	    En cualquier caso, nuestro lado derecho se reduce a $f(x)$ y sabemos que al menos uno de estos dos casos es verdadero; por lo tanto concluimos que $\forall x, f(x)= max(f(x),0)+min(f(x),0)$ ó $f=max (f,0)+min(f,0)$\\\\

	    %----------(b)
	    \item Una función $f$ se dice que es no negativa si $f(x)\geq 0$ para todo $x$. Demostrar que para cualquier función $f$ puede ponerse $f=g-h$ de infinitas maneras con $g$ y $h$ no negativas. (La manera corriente es $g=max(f,0)$ y $h=-min(f,0)$. Cualquier número puede ciertamente expresarse de infinitas maneras como diferencia de dos números no negativos.)\\\\
	    Demostración.-\; Comenzamos con la observación de que, para cualquier número real no negativo $r$, hay infinitos números reales no negativos $s, t$ tales que $$r=s-t$$ De hecho, para cada $n \in \mathbb{N}$, tomamos $s_n = 2r + n$ y $t_n = r + n$. Entonces, dado que $r\geq 0$, tanto $s_n$ como $t_n$ son no negativos. Además, $$s_n=t_n=2r+n-r-n=r$$ Ahora, para cada número real $x$, tenemos que $f(x)\geq  0$. Por lo tanto, a partir de la observación anterior, vemos que hay infinitos números reales no negativos $s_x$ y $t_x$ tales que $$f(x)=s_x-t_x$$ para cada $x \in \mathbb{R}$. Así que definimos funciones no negativas $g$ y $h$ como sigue $$g(x)=s_x\;\; y \;\; h(x)=t_x$$. Entonces hemos demostrado que hay infinitas opciones de tales funciones. Además, tenemos que $$f(x)=g(x)-h(x)$$. Por lo tanto, hemos demostrado que hay infinitas funciones no negativas $g$ y $h$ tales que $$f = g-h$$\\\\

	\end{enumerate}

	%--------------------16.
	\item Supongase que $f$ satisface $f(x+y)=f(x)+f(y)$ para todo $x$ e $y$. 

	\begin{enumerate}[\bfseries (a)]

	    %----------(a)
	    \item Demostrar que $f(x_1,+...+x_n)=f(x_1)+...+f(x_n)$\\\\
	    Demostración.-\; El resultado se cumple para $n=1$, $f(x_1)=f(x_1)$. Luego si $f(x_1 + .... + x_n)=f(x_1)+ ... + f(x_n)$ para todo $x_1,...,x_n$, entonces
	    \begin{center}
		\begin{tabular}{rcll}
		    $f(x_1 + ... + x_{n+1})$ & $=$ & $f(\left[x_1 + ... + x_n\right]+x_{n+1})$&\\
		       & $=$ & $f(x_1 + ... + x_n) + f(x_{n+1})$&por hipótesis\\
		       & $=$ & $f(x_1)+...+f(n)+f(x_{n+1})$&\\\\
		\end{tabular}
	    \end{center}

	    %----------(b)
	    \item Demostrar que existe algún número $c$ tal que $f(x)=cx$ para todos los números racionales $x$ (en este punto no intentamos decir nada acerca de $f(x)$ cuando $x$ es irracional). Indicación: Piénsese primero en cómo debe ser $c$. Demostrar luego que $f(x)=cx$, primero cuando $x$ es un entero, después cuando $x$ es el reciproco de un entero, y finalmente para todo racional $x$.\\\\
	    Demostración.-\; Sea $c=f(1)$. Luego para cualquier número natural $n$ y el inciso $(a)$,  $$f(n)=f(1+...+1)=f(1)+...+f(1)=n\cdot f(1)=cn \,\,\,\, (1)$$ 
	    Al ser $$f(x)+f(0)=f(x+0)=f(x),$$ entonces $f(0)=0$. Ahora, puesto que $$f(x)+f(-x)=f(x+(-x))=f(0)=0,$$ 
	    resulta que $f(-x)=-f(x)$. En particular, para cualquier número natural $n$ y por $(1)$, $$f(-n)=-f(n)=-cn=c\cdot (-n)$$
	    Además $$f\left(\dfrac{1}{n}\right) + ... + f\left(\dfrac{1}{n}\right)=f\left(\dfrac{1}{n} + ... + \dfrac{1}{n}\right)=f\left( \dfrac{n}{n}\right)=f(1)=c$$ de modo que, $$f\left(\dfrac{1}{n}\right)=c\cdot \dfrac{1}{n},$$
	    y en consecuencia $$f\left(\dfrac{1}{-n}\right)=f\left(- \dfrac{1}{n}\right)=-f\left(\dfrac{1}{n}\right)=-c \cdot \dfrac{1}{n} = c \left(\dfrac{1}{n}\right)$$
	    Por último, cualquier número racional puede escribirse en la forma $m/n$, siendo $m$ un número natural y $n$ un entero;
	    $$f\left(\dfrac{m}{n}\right)=f\left(\dfrac{1}{n} + ... + \dfrac{1}{n}\right)=f\left(\dfrac{1}{n}\right) + ... + f\left(\dfrac{1}{n}\right)=mc\cdot \dfrac{1}{n}=c\cdot \dfrac{m}{n}$$\\\\

	\end{enumerate}

	%--------------------17
	\item Si $f(x)=0$ para todo $x$, entonces $f$ satisface $f(x+y)=f(x) + f(y)$ para todo $x$ e $y$ también $f(x\cdot y)=f(x)\cdot f(y)$ para todo $x$ e $y$. Supóngase ahora que $f$ satisface estas dos propiedades, pero que $f(x)$ no es siempre $0$. Demostrar que 

	\begin{enumerate}[\bfseries (a)]

	    %----------(a)
	    \item Demostrar que $f(1)=1$\\\\
		Demostración.-\; Al ser $f(a)=f(a\cdot 1)=f(a)\cdot f(1)$ y $f(a)\neq 0$ para algún $a$, resulta ser $f(1)=1$\\\\

	    %----------(b)
	    \item Demostrar que $f(x)=x$ si $x$ es racional \\\\
		Demostración.-\; Por el problema 16, $f(x)=f(1)\cdot x = x$ para todo número racional $x$. \\\\ 

	    %----------(c)
	    \item Demostrar que $f(x)>0$ si $x>0$. (Esta parte es artificiosa, pero habiendo puesto atención a las observaciones filosóficas que van con los problemas de los dos últimos capítulos, se sabrá lo que hacer.)\\\\
		Demostración.-\; Si $c>0$ entonces $c=d^2$ para algún $d$, de modo que $f(c)=f(d^2)=(f(d))^2\geq 0$. Por otro lado, no podemos tener $f(c)=0,$ ya que esto implicaría que $$f(a)=f\left( c\cdot \dfrac{a}{c}\right) = f(c)\cdot f\left(\dfrac{a}{c}\right) = 0 \qquad para \; todo \; a$$\\\\

	    %----------(d)
	    \item Demostrar que $f(x)>f(y)$ si $x>y$\\\\
		Demostración.-\; Si $x>y$, entonces $x-y>0$, luego por la parte $(c)$ tenemos que $f(x)-f(y)>0$. \\\\ 

	    %----------(e)
	    \item Demostrar que $f(x)=x$ para todo $x$. Indicación: Hágase uso del hecho de que entre dos números calesquiera existe un número racional\\\\
		Demostración.-\; Sea $f(x)>x$ para algún $x$. Elíjase un número racional $r$ con $x<r<f(x)$. Entonces, según las partes $(b)$ y $(d)$, $$f(x)<f(r)=r<f(x),$$ 
		lo cual constituye una contradicción. Análogamente, es imposible que $f(x)<x$ ya que si $f(x)<r<x$ entonces $$f(x)<r=f(r)<f(x).$$\\\\

	\end{enumerate}

	%--------------------18.
	\item  ¿Qué condiciones precisas deben satisfacer $f,g,h$ y $k$ para que $f(x)g(y)=h(x)k(y)$ para todo $x \; e \; y$? \\\\
	    Respuesta.-\; Se satisface la ecuación si $f=0$ ó $g=0$ y $h=0$ ó $k=0$. De no ocurrir esto, existirá algún $x$ con $f(x)\neq 0$ y algún $y$ con $g(y)\neq 0$, entonces $0 \neq ff(x)g(y) = h(x)k(y)$, de modo que también se tendrá $h(x) \neq 0$ y $k(y)\neq 0$. Haciendo $\alpha = h(x)/f(x)$, tenemos también $h(x^{'} = \alpha f(x^{'}$ para todo $x^{'}$ para todo $x^{'}$. Tenemos pues. que $g=\alpha k$ y $h=\alpha f$ para cierto número $\alpha=0$. \\\\
	
	%--------------------19.
	\item 

	\begin{enumerate}[\bfseries (a)]

	    %----------(a)
	    \item Demostrar que no existen funciones $f$ y $g$ con alguna de las propiedades siguientes:\\\\

	    \begin{enumerate}[\bfseries (i)]

		%-----(i)
		\item $f(x)+g(y)=xy$ para todo $x$ e $y$.\\\\
		    Demostración.-\; Si $f(x)+g(y)=xy \; \forall x,y$ entonces para $y=0$ tenemos $f(x)+g(0)=0 \; \forall x.$, de donde $f(x)=-g(0)$, e implica que $f$ es una función constante. Luego $$xy=f(x)+g(y)=-g(0) +g(y)  \; \forall y$$ porque $f(x)$ es constante para cualquier $x$. Por otro lado sabemos que  $g(0)$ es una constante y $g(y)$  no depende de $x$, sin embargo su diferencia está dada por $g(y) - g(0) = xy$. Y finalmente sea $x=0$ entonces $g(y)=g(0) \; \forall y$,  por lo tanto se concluye que $$xy=f(x)+g(x) = -g(0) + g(0) = 0 \; \forall x,y$$  ya que si tomamos $x=y=1$ implica que $1=0$ donde llegamos a un absurdo.\\\\  

		%-----(ii)
		\item $f(x) \cdot g(y) = x+y$ para todo $x$ e $y$.\\\\
		    Demostración.-\; Sea $y=0$, obtenemos $f(x)=x/g(0)$. De la misma forma si $x=0$, entonces $g(y)=y/f(0)$. Por lo tanto $$  f(x) \cdot g(y) = x+y \Longrightarrow \dfrac{x}{g(0)}\cdot \dfrac{y}{f(0)}=x+y \qquad \forall x, \; e \; \forall y$$ Supongamos que $y=0$, entonces $\dfrac{x}{g(0)}\cdot \dfrac{0}{f(0)}=x \quad \forall x \Longrightarrow 0=x \quad \forall x$, lo cual es absurdo.\\\\

	    \end{enumerate}

	    %----------(b)
	    \item Hallar funciones $f$ y $g$ tales que $f(x+y) = g(xy)$ para todo $x$ e $y$. \\\\
		Respuesta.-\; Sean $f$ y $g$ la misma función constante. Argumentos similares a los utilizados en la parte $(a)$ muestran que estas son las únicas opciones posibles. \\\\ 

	\end{enumerate}

	%--------------------20.
	\item 
	\begin{enumerate}[\bfseries (a)]

	    %----------(a)
	    \item Hallar una función $f$ que no sea constante y tal que $|f(y) - f(x)| \leq |y-x|$.\\\\
		Respuesta.-\; Podemos ver que la función $f(x)=x$ satisface la condición $|f(y) - f(x)| \leq |y-x|$\\\\

	    %----------(b)
	    \item Supóngase que $f(y) - f(x) \leq (y-x)^2$ para todo $x$ e $y.$ (¿ Por qué esto implica $|f(y) - f(x)| \leq (y-x)^2$?) Demostrar que $f$ es una constante. Indicación: Divídase el intervalo $[x,y]$ en $n$ partes iguales.\\\\
		Demostración.-\; Supongamos, que puede probar que la siguiente desigualdad es cierta para todos $x , y \in \mathbb{R}$, y $n \in \mathbb{N}$: $$|f(y) - f(x)| \leq \dfrac{(y-x)^2}{n}$$ Ahora mantengamos los valores de $x$ e $y$ constantes. Podemos suponer $x\neq y$ (porque si $x=y$ entonces $f(x)=f(y)$ y así terminaríamos la demostración). Entonces, en el lado derecho, el numerador $(y-x)^2$ es distinto de $0$, y mayor a cero. Por lo tanto, podemos dividir por $(y-x)^2$, de donde: $$\dfrac{|f(y) - f(x)|}{(y-x)^2} \leq \dfrac{1}{n}$$ En el lado izquierdo tenemos un número no negativo que es constante (ya que $x$ e $y$ se mantienen constantes, el numerador no es negativo y el denominador es positivo). Este número es menor que cada fracción $\dfrac{1}{n}$ para todos los números naturales $n\geq 1$. Esto implica que el lado izquierdo es igual a cero:
		$$\dfrac{|f(y) - f(x)|}{(y-x)^2} = 0$$ una vez mas multiplicamos por $(y-x)^2$ entonces $$|f(y)-f(x)|=0,$$ de donde $$|f(y)-f(x)|=0 \Longrightarrow f(y)=f(x)$$
		Dado que esto es cierto para todos los valores $x$, $y$  terminamos la demostración.\\\\

	\end{enumerate}

	%--------------------21.
	\item  Demostrar o dar un contraejemplo de las siguientes proposiciones:

	    \begin{enumerate}[\bfseries (a)]

		%----------(a)
		\item $f \circ (g+h) = f \circ g + f \circ h.$\\\\
		    Demostración.-\; Esto es falso en general ya que si designamos a $g$ y $h$ la función identidad y  $f$ sea $x^2$ entonces $$\left[ f \circ (g+h) \right](x) = f\left(  g + h\right)(x)=f \left[ g(x) +  h(x) \right] = f(x+x)= f(2x) = 4x^2.$$ 
		    luego por la parte derecha de la ecuación se tendra:
		    $$\left[ \left( f\circ g \right) + \left( f \circ h \right)\right]\left(x \right) = \left(f \circ g \right) \left(x\right) + \left( f \circ h \right)\left(x\right) = f\left[g\left(x\right)\right] + f\left[h \left(x\right)\right] = f\left(x\right) + g\left(x\right) = x^2 + x$$
		    De donde $4x^2 \neq x^2 + x$\\\\

		%----------(b)
		\item $(g+h) \circ f = g\circ f + h \circ f.$\\\\
		    Demostración.-\; Por definición de composición de función tenemos 
		    \begin{center}
			\begin{tabular}{rcll}
			    $\left[ \left( g+ h\right) \circ f \right] \left(x\right)$ & $=$ & $\left(g+h\right)\left[f(x)\right]$ & \\ 
			    & $=$ & $g\left[ f(x)\right] + h\left[f(x)\right]$ & por definición\\
			    & $=$ & $\left(g\circ f\right)(x) + \left(h \circ f\right)(x)$ & \\
			    & $=$ & $\left[ \left(g\circ f\right)  + \left(h \circ f\right)\right](x)$ &\\\\
			\end{tabular} 
		    \end{center}
		    Así  $(g+h)\circ f = (g\circ f)+(h \circ f)$\\\\

		%----------(c)
		\item $\dfrac{1}{f\circ g} = \dfrac{1}{f} \circ g.$\\\\
		    Demostración.-\; Por definición se tiene,
		    \begin{center}
			\begin{tabular}{rcl}
			    $\left( \dfrac{1}{f\circ g}\right)(x)$ & $=$ & $\dfrac{1}{(f\circ g)(x)}$\\\\
			    & $=$ & $\dfrac{1}{f\left[ g(x)\right] }$\\\\
			    & $=$ & $\left( \dfrac{1}{f}\right) \left[ g(x)\right]$\\\\
			    & $=$ & $\left( \dfrac{1}{f} \circ g\right)(x)$\\\\
			\end{tabular}
		    \end{center}
		    Así, $1/(f\circ g)=(1/f)\circ g$\\\\

		%----------(d)
		\item $\dfrac{1}{f \circ g} = f \circ \left( \dfrac{1}{g}\right).$\\\\
		    Demostración.-\;  Esto es falso ya que si consideramos $f(x)=x+1$ y $g(x)=x^2$, entonces 
		    $$\left( \dfrac{1}{f \circ g}\right) (x) = \dfrac{1}{(f\circ g)(x)} = \dfrac{1}{f\left[ g(x) \right]} = \dfrac{1}{f(x^2)} = \dfrac{1}{x^2+1}$$ y por otro lado 
		    $$\left[f\circ \left( \dfrac{1}{g}\right)\right](x)  = f\left[ \left( \dfrac{1}{g}\right) (x)\right] = f\left( \dfrac{1}{g(x)}\right) = f \left( \dfrac{1}{x^2}\right)= \dfrac{1}{x^2} + 1$$ 
		    de  donde $\dfrac{1}{x^2 + 1} \neq \dfrac{1}{x^2} + 1$\\\\

	    \end{enumerate}

	%--------------------22.
	\item  
	\begin{enumerate}[\bfseries (a)]
	    
	    %----------(a)
	    \item Supóngase que $g=h \circ f$. Demostrar que si $f(x)=f(y)$, entonces $g(x)=g(y)$.\\\\
		Demostración.-\;  $g(x)=h\left(f(x)\right) =h\left(f(y)\right) = g(y) $ esto por definición  e hipótesis.\\\\ 
	    
	    %----------(b)
	    \item Recíprocamente, supóngase que $f$ y $g$ son dos funciones tales que $g(x)=g(y)$ siempre que $f(x)=f(y)$. Demostrar que $g = h\circ f$ para alguna función $h$. Indicación: Inténtese definir $h(z)$ cuando $z$ es de la forma $z=f(x)$ (Éstos son los únicos $z$ que importan) y aplicar la hipótesis para demostrar que la definicón es consistente.\\\\
		Demostración.-\; Si $z=f(x)$, defínase $h(z)=g(x)$. Esta definición tiene sentido, ya que si $z=f(x^{'},$ entonces $g(x)=g(x^{'})$ según la parte $(a)$. Tenemos entonces, para todo $x$ del dominio de $f$, $g(x)=h(f(x))$.\\\\

	\end{enumerate}

	%--------------------23.
	\item Supóngase que $f\circ g = I$ donde $I(x)=x$. demostrar que 

	\begin{enumerate}[\bfseries (a)]

	    %----------(a)
	    \item Si $x\neq y$, entonces $g(x)\neq g(y)$\\\\
		Demostración.-\; Supongamos que $x\neq y$ y $g(x)=g(y)$ esto implica que $x=I(x)=f(g(x))=f(g(y))=y$. Donde vemos una contradicción.\\\\

	    %-----------(b)
	    \item Todo número $b$ puede escribirse $b=f(a)$ para algún número $a$.\\\\
		Demostración.-\; Por hipótesis $b=f(g(b))$ donde basta con poner $a=g(b)$.\\\\

	\end{enumerate}

	%--------------------24.
	\item 
	
	\begin{enumerate}[\bfseries (a)]

	    %----------(a)
	    \item Supóngase que $g$ es una función con la propiedad de ser $g(x)\neq g(y)$ si $x\neq y$. Demuéstrese que existe una función $f$ tal que $f\circ g = I$\\\\
		Demostración.-\; Es equivalente enunciar que si $x=y$, entonces $g(x)=g(y)$. en consecuencia del problema $22b$.\\\\

	    %----------(b)
	    \item Supóngase que $f$ es una función tal que todo número $b$ puede escribirse en la forma $b=f(a)$ para algún número $a$. Demostrar que existe una función $g$ tal que $f\circ g=I$\\\\
		Demostración.-\; Para cada $x$, elíjase un número $a$ tal que $x=f(a)$. Llámese a este número $g(x)$. Entonces $f(g(x))=x=I(x)$ para todo $x$.\\\\ 

	\end{enumerate}

	%--------------------25.
	\item Hallar una función $f$ tal que $g\circ f=I$ para alguna función $g$, pero tal que no exista ninguna función $h$ con $f\circ h = I$\\\\
	    Respuesta.-\; Basta hallar una función $f$ tal que $f(x)\neq f(y)$ si $x\neq y$, pero tal que no todo número sea de la forma $f(x)$, pues entonces según el problema $24(a)$ existirá una función $g$ con $g\circ f=I$, y según el problema $23(b)$ no existiría ninguna función $g$ con $f\circ g=I$. Una función que reúne estas condiciones es:
	    $$f(x) = \left\{ \begin{array}{lc} 
		x,&x\leq 0\\
		\\ x+1,& x>0\\
	    \end{array}\right.$$
	    ningún número  de los comprendidos entre $0$ y $1$ es de la forma $f(x)$.\\\\

	%--------------------26.
	\item Supóngase $f\circ g = I$ y $h\circ f = I$. Demostrar que $g=h$. Indicación: Aplíquese el hecho de que la composición es asociativa.\\\\
	    Demostración.-\; Sea $h\circ f\circ g$ entonces $h\circ (f \circ g)=h\circ I = h,$ como también $h\circ f \circ g = (h \circ f) \circ g = I \circ g = g.$\\\\ 

	%--------------------27.
	\item 
	    \begin{enumerate}[\bfseries (a)]

		%----------(a)
		\item Supóngase $f(x)=x+1.$ ¿Existen funciones $g$ tales que $f\circ g = g\circ f$?\\\\
		    Respuesta.-\; La condición $f\circ g = f \circ g$ significa que $f(x) + 1 = g(x+1)$ para todo $x$. Existen muchas funciones $g$ que satisfacen esta condición. La función $g$ puede en efecto definirse arbitrariamente para $0\leq x < 1$ y para otros $x$ pueden determinarse sus valores mediante esta ecuación.\\\\

		%----------(b)
		\item Supóngase que $f$ es una función constante. ¿Para qué funciones $g$ se cumple $f\circ g = g\circ f$?\\\\
		    Respuesta.-\; Si $f(x)=c$ para todo $x$, entonces $f\circ g = g\circ f$ si y sólo si $c=f(g(x)) = g(f(x))=g(c)$, es decir, $c=g(c)$\\\\ 

		%----------(c)
		\item Supóngase que $f\circ g = g\circ f$ para todas las funciones $g$. Demostrar que $f$ es la función identidad $f(x)=x$\\\\
		    Respuesta.-\; Si $f\circ g = g  \circ f$ para todo $g$, entonces se cumple esto en particular para todas las funciones constantes $g(x)=c$. Se sigue de la parte $(b)$ que $f(c)=c$ para todo $c.$\\\\

	    \end{enumerate}

	%--------------------28.
	\item
	\begin{enumerate}[\bfseries (a)]

	    %----------(a)
	    \item Sea $F$ el conjunto de todas las funciones cuyo dominio es $\mathbb{R}$. Demuéstrese que con las definiciones de $+$ y $\cdot $ dadas en este capítulo, se cumplen todas las propiedades $P1-P9$, excepto $P7$, siempre que $0$ y $1$ se interpreten como funciones constantes.\\\\
	    Demostración.-\; Se comprueba fácilmente.\\\\

	    %----------(b)
	    \item Demostrar que $P7$ no se cumple.\\\\
		Demostración.-\; Sea $f$ una función con $f(x)=0$ para algún $x$, pero no para todo $x$. Entonces $f\neq 0$, pero claramente no existe ninguna función $g$ con $f(x)\cdot g(x) = 1$ para todo $x$.\\\\

	    %----------(c)
	    \item Demostrar que no pueden cumplirse $P10-P12.$ En otros términos, demostrar que no existe ninguna colección $P$ de funciones en $F$, tales que $P10-P12$ se cumplen para $P$. (Es suficiente, y esto simplificará las cosas, considere sólo funciones que sean $0$, excepto en dos puntos $x_0$ y $x_1$).\\\\
		Demostración.-\; Sean $f$ y $g$ dos funciones cuyos valores son todos $0$ excepto en $x_0$ y $x_1$, siendo $f(x_0)=1$, $f(x_1)=0$, $gx_0 = 0$, $g(x_1)=1.$ Ninguna de ellas es $0$, de modo que o bien $f$ o bien $-f$ tendría que estar en $P$ y lo mismo podría decirse de $g$ o $-g$. Pero $(\pm f)(\pm g) = 0$, en contradicción con $P12$.\\\\

	    %----------(d)
	    \item Supóngase que se ha definido $f<g$ en el sentido de que $f(x)<g(x)$ para todo $x$. ¿Cuáles de las propiedades $P^{'}-P^{'}13$ (del problema $1-8$)? se cumplen ahora?\\\\
		Respuesta.-\; $P^{'} 11, P^{'} 12$ y $P^{'}13$ se cumplen. $P^{'}10$ es falso; si bien se cumple a lo sumo una de las condiciones, o es necesariamente cierto que se tenga que cumplir por lo menos una de ellas. Por ejemplo, si $f(x)>0$ para algún $x$ y $y<0$ para otro $x$, entonces ninguna de las condiciones $f=0,$ $f<0$ ó $f>0$ para todo $x$.\\\\

	    %----------(e)
	    \item si $f<g$, ¿Se cumple $h\circ f < h \circ g$? ¿Es $f\circ h<g\circ h$?\\\\
		Respuesta.-\; No para el primer ejemplo; si $h(x)=-x,$ entonces $f<g$ implica, en realidad , que $h\circ f > h \circ g$. Si para el segundo, ya que $f(h(x)<g(h(x)$ para todo $x$.\\\\

	\end{enumerate}

    \end{enumerate}

\section{Pares ordenados}
    
    %--------------------definición 1.9
    \begin{tcolorbox}
	\begin{def.}
	    $(a,b) = \lbrace {a},{a,b} \rbrace$\\
	\end{def.}
    \end{tcolorbox}

    %--------------------teorema 
    \begin{teo}
	Si $(a,b) = (c,d)$ entonces $a=c$ y $b=d$\\\\
	    
	    Demostración.-\; La hipótesis significa que $$\lbrace {a},{a,b} \rbrace \lbrace {c},{a,d}\rbrace,$$
	    Ahora bien, $\lbrace {a},{a,b} \rbrace$ contiene justamente dos elementos ${a}$ y ${a,b}$ y $a$ es el único elemento común a estos dos elementos de $\lbrace{a},{a,b}\rbrace$. Por lo tanto, $a=c$. Así pues tenemos $$\lbrace {a},{a,b}\rbrace = \lbrace {a},{a,d} \rbrace,$$ y solamente queda por demostrar $b=d.$ Conviene distinguir dos casos.\\
	    \textbf{Caso 1 $b=a$.} En este caso, ${a,b}={a}$, de modo que el conjunto $\lbrace {a},{a,b}\rbrace$ tiene en realidad un solo elemento que es ${a}$. Lo mismo vale para $\lbrace {a},{a,b}\rbrace$, de modo que ${a,d}={a},$ lo cual implica $d=a=b.$\\
	    \textbf{Caso 2. $b\neq a$}. En este caso, $b$ pertenece a uno de los elementos de $\lbrace {a}, {a,b}\rbrace$, pero no al otro. Debe, por lo tanto cumplirse que $b$ pertenece a uno de los elementos de $\lbrace {a},{a,b}\rbrace$, pero no al otro. Esto solamente puede ocurrir si $b$ pertenece a ${a,d}$, pero no a ${a}$; así pues, $b=a$ o $b=d$, pero $b\neq a$, con lo que $b=d$.\\\\

    \end{teo}



    %---------- gráficas
	%\input{codigoFuente/michael_spivak/4-gráficas.tex}

    %---------- límites
	%\chapter{Limites}

%--------------------definición de límite
\begin{tcolorbox}[colframe=white]
    \begin{def.}
	La función $f$ tiende hacia el límite $l$ en $a$ $\left(\lim\limits_{x \to a} f(x) = l \right)$ significa: para todo $\epsilon > 0$ existe algún $\delta > 0$ tal que, para todo $x$, si $0<|x-a|<\delta$, entonces $|f(x)-l|<\epsilon$.\\\\
	Existe algún $\epsilon >0$ tal que para todo $\delta > 0$ existe algún $x$ para el cual es $0<|x-a|<\delta$, pero no $|f(x)-l|<\epsilon$.
    \end{def.}
\end{tcolorbox}
\vspace{.5cm}

%--------------------teorema 1.1
\begin{teo}
Una función no puede tender hacia dos límites diferentes en $a$. En otros términos si $f$ tiende hacia $l$ en $a$, y $f$ tiende hacia $m$ en $a$, entonces $l=m$.\\\\
    Demostración.-\; Puesto que $f$ tiende hacia $l$ en $a$, sabemos que para todo $\epsilon>0$ existe algún número $\delta_1 >0$ tal que, para todo $x$, si $0<|x-a|<\delta_1$, entonces $|f(x)-l|<\epsilon$.\\
    Sabemos también, puesto que $f$ tiende hacia $m$ en $a$, que existe algún $\delta_2 >0$ tal que, para todo $x$, si $0<|x-a|<\delta_2$, entonces $|f(x)-m|<\epsilon$.\\
    Hemos empleado dos números $delta_1$ y $\delta_2$, ya que no podemos asegurar que el $\delta$ que va bien en una definición irá bien en la otra. Sin embargo, de hecho, es ahora fácil concluir que para todo $\epsilon>0$ existe algún $\delta >0$ tal que, para todo $x$, $$si \; 0<|x-a|<\delta=\min(\delta_1,\delta_2), \; entonces \; |f(x)-l|<\epsilon \;\; y \;\; |f(x)-m|<\epsilon$$  
    Para completar la demostración solamente nos queda tomar un $\epsilon>0$ particular para el cual las dos condiciones $|f(x)-l|<\epsilon$ y $|f(x)-m|<\epsilon$ no puedan cumplirse a la vez si $l\neq m$\\
    Si $l\neq m$, de modo que $|m-l|>0$ podemos tomar como $\epsilon$ a $|l-m|/2$. Se sigue que existe un  $\delta > 0$ tal que, para todo $x$, $$si \; 0<|x-a|<\delta, \; entonces \; |f(x)-l|<\dfrac{|l-m|}{2} \; \; y \; \; |f(x)-m|<\dfrac{|l-m|}{2}$$
    Esto implica que para $0<|x-a|<\delta$ tenemos $$|l-m| = |l - f(x) + f(x) -m|\leq |l-f(x)| + |f(x)-m|<\dfrac{|l-m|}{2}+\dfrac{|l-m|}{2}=|l-m|$$ El cual es una contradicción.\\\\
\end{teo}
\vspace{.5cm}

    %-------------------lema1.1
    \begin{lema}
	Si $x$ está cerca de $x_0$ e $y$ está cerca de $y_0$, entonces $x+y$ estará cerca de $x_0 + y_0$, $xy$ estará cerca de $x_0y_0$ y $1/y$ estará cerca de $1/y_0$.\\\\
	\begin{enumerate}[\bfseries (1)]

	    %----------(1)
	    \item Si $|x-x_0|<\dfrac{\epsilon}{2}$ y $|y-y_0|<\dfrac{\epsilon}{2}$ \; entonces \; $|(x+y) - (x_0+y_0)|<\epsilon$.\\\\
		Demostración.-\;$$|(x+y)-(x_0+y_0)| = |(x-x_0)+(y-y_0)| \leq |x-x_0| + |y-y_0| < \dfrac{\epsilon}{2} + \dfrac{\epsilon}{2} = \epsilon$$\\\\

	    %----------(2)
	    \item Si $|x-x_0| < \min\left(1,\dfrac{\epsilon}{2(|y_0|+1)}\right)$ \; y\; $|y-y_0| < \dfrac{\epsilon}{2(|x_0| + 1)}$\; entonces \;$|xy-x_0y_0|<\epsilon$.\\\\
		Demostración.-\; Puesto que $|x-x_0|<1$ se tiene $$|x| - |x_0| \leq |x-x_0|<1,$$
		de modo que $$|x| < 1 + |x_0|$$ así pues 
		\begin{center}
		    \begin{tabular}{rcl}
			$|xy-x_0y_0|$&$=$&$|x(y-y_0) + y_0(x-x_0)|$\\\\
			&$\leq$&$|x|\cdot |y-y_0| + |y_0|\cdot |x-x_0|$\\\\
			&$<$&$(1+|x_0|)\cdot \dfrac{\epsilon}{2(|x_0|+1)} + |y_0|\cdot \dfrac{\epsilon}{2(|y_0|+1)}$\\\\
			&$<$&$\dfrac{\epsilon}{2} + \dfrac{\epsilon}{2} = \epsilon$\\\\
		    \end{tabular}
		\end{center}
		Notemos que $\dfrac{|y_0|}{|y_0|-1}<1$, por lo tanto $\dfrac{|y_0|}{|y_0|-1}\cdot \dfrac{\epsilon}{2}< \dfrac{\epsilon}{2}$.\\\\
		\vspace{.5cm}

	    %----------(3)
	    \item Si $y_0 \neq 0$ y $|y-y_0| < \min\left(\dfrac{|y_0|}{2},\dfrac{\epsilon|y_0|^2}{2}\right)$ \; entonces \; $y\neq 0$ y $\left| \dfrac{1}{y} - \dfrac{1}{y_0} \right|< \epsilon$.\\\\
		Demostración.-\; Se tiene $$|y_0|-|y|<|y-y_0|<\dfrac{|y_0|}{2},$$
		de modo que $-|y|< -\dfrac{|y_0|}{2} \Longrightarrow |y|>|y_0|/2$. En particular. $y\neq 0$, y $$\dfrac{1}{|y|}<\dfrac{2}{|y_0|}$$ Así pues 
		$$\left|\dfrac{1}{y}-\dfrac{1}{y_0}\right| = \dfrac{|y_0-y|}{|y|\cdot |y_0|} = \dfrac{1}{|y|}\cdot \dfrac{|y_0-y|}{|y_0|} < \dfrac{2}{|y_0|}\cdot \dfrac{1}{|y_0|} \cdot \dfrac{\epsilon |y_0|^2}{2} = \epsilon$$\\\\
	\end{enumerate}
    \end{lema}
\vspace{1cm}

%--------------------teorema 2
\begin{teo}
    Si $\lim\limits_{x \to a} f(x) = l$ y $\lim\limits_{x \to a} g(x) = m$, entonces 
    \begin{enumerate}[\bfseries (1)]
	\item  $\lim\limits_{x \to a} (f+g)(x) = l+m$
	\item  $\lim\limits_{x \to a} (f\cdot g)(x) = l\cdot m$\\\\
	Además, si $m\neq 0$, entonces
	\item  $\lim\limits_{x \to a} (\dfrac{1}{g})(x) = \dfrac{1}{m}$
    \end{enumerate}
	Demostración.-\; La hipótesis significa que para todo $\epsilon > 0$ existen $\delta_1, \delta_2>0$ tales que, para todo $x$, $$si \; 0<|x-a|<\delta_1, \; entonces \; |f(x)-l|<\epsilon$$ 
	$$y \quad si \; 0<|x-a|<\delta_2, \; entonces \; |g(x)-m < \epsilon|$$
	Esto significa ( ya que después de todo, $\epsilon/2$ es también un número positivo) que existen $\delta_1,\delta_2>0$ tales que, para todo $x$,
	$$si \; 0<|x-a|<\delta_1, \; entonces \; |f(x)-l|<\dfrac{\epsilon}{2}$$
	$$y \quad si \; 0<|x-a|<\delta_2, \; entonces \; |g(x)-m|<\dfrac{\epsilon}{2}$$
	Sea ahora $\delta = \min(\delta_1,\delta_2)$. Si $0<|x-a|<\delta$, entonces $0<|x-a|<\delta_1$ y $0<|x-a|<\delta_2$ se cumplen las dos, de modo que es a la vez $$|f(x)-l|<\dfrac{\epsilon}{2} \quad y \quad |g(x)-m|<\dfrac{\epsilon}{2}$$
	pero según la parte $(1)$ del lema anterior esto implica que $|(f+g)(x) - (l+m)|< \epsilon$.\\\\
	Para demostrar $(2)$ procedemos de la misma manera, después de consultar la parte $(2)$ del lema. Si $\epsilon>0$ existen $\delta_1,\delta_2>0$ tales que, para todo $x$ $$si\; 0<|x-a|<\delta_1, \; entonces \; |f(x)-l|<\min\left(1,\dfrac{\epsilon}{2(|m|+1)}\right),$$ $$y\quad si\; 0<|x-a|<\delta_2,\; entonces \; |g(x)-m|<\dfrac{\epsilon}{2(|l|)+1}$$ Pongamos de nuevo $\delta = \min(\delta_1,\delta_2)$. Si $0<|x-a|<\delta,$ entonces $$|f(x)-l|<\min\left(1,\dfrac{\epsilon}{2(|m|+1)}\right) \qquad y \qquad |g(x)-m|<\dfrac{\delta}{2(|l|+1)}$$
	Así pues, según el lema, $|(f\cdot g)(x)-l\cdot m|<\epsilon$, y esto demuestra $(2)$.\\\\
	Finalmente, si $\epsilon>0$ existe un $\delta>0$ tal que, para todo $x$, $$si\; 0<|x-a|<\delta, \; entonces \; |g(x)-m|< \min\left(\dfrac{|m|}{2},\dfrac{\epsilon|m|^2}{2}\right)$$
	Pero según la parte $(3)$ del lema, esto significa, en primer lugar que $g(x)\neq 0$, de modo que $(1/g)(x)$ tiene sentido, y en segundo lugar que $$\left|\left(\dfrac{1}{g}\right)(x)-\dfrac{1}{m}\right|<\epsilon$$
	Esto demuestra $(3)$.\\\\
\end{teo}

%--------------------definición 5.2
\begin{tcolorbox}[colframe=white]
    \begin{def.}
	$\lim\limits_{x \to a^+} f(x)=l$ significa que para todo $\epsilon>0$, existe un $\delta>0$ tal que, para todo $x$, $$si \; 0<x-a<\delta, \; entonces \; |f(x)-l|<\epsilon$$	
	La condición $0<x-a<\delta$ es equivalente a $0<|x-a|<\delta$ y $x>a$
    \end{def.}
\end{tcolorbox}

%--------------------definición 5.3
\begin{tcolorbox}[colframe=white]
    \begin{def.}
	$\lim\limits_{x \to a^-} f(x)=l$ significa que para todo $\epsilon>0$, existe un $\delta>0$ tal que, para todo $x$, $$si \; 0<a-x<\delta, \; entonces \; |f(x)-l|<\epsilon$$	
    \end{def.}
\end{tcolorbox}

%--------------------definición 5.4
\begin{tcolorbox}[colframe=white]
    \begin{def.}
	$\lim\limits_{x \to \infty} f(x)=l$ significa que para todo $\epsilon>0$, existe un número $N$ grande, que, para todo $x$, $$si \; x>N, \; entonces \; |f(x)-l|<\epsilon$$	
    \end{def.}
\end{tcolorbox}

\section{Problemas}
\begin{enumerate}[\Large \bfseries 1.]

%--------------------1.
\item Hallar los siguientes limites (Estos limites se obtienen todos, después de algunos cálculos, de las distintas partes del teorema 2; téngase cuidado en averiguar cuáles son las partes que se aplican, pero sin preocuparse de escribirlas.)
\begin{enumerate}[\bfseries (i)]

    %----------(i)
    \item $\lim\limits_{x \to 1}\dfrac{x^2-1}{x+1} = \dfrac{1^2 - 1}{1 + 1} = \dfrac{0}{2} = 0$\\\\

    %----------(ii)
    \item $\lim\limits_{x \to 2} \dfrac{x^3 - 8}{x - 2} = \dfrac{(x-2)(x^2+2x+4)}{x-2} = 2^2+4+4 = 12$\\\\

    %----------(iii)
    \item $\lim\limits_{x \to 3} \dfrac{x^3-8}{x-2} = \dfrac{3^3-8}{3-2} =19$\\\\

    %----------(iv)
    \item $\lim\limits_{x\to y}\dfrac{x^n - y^n}{x-y} = \dfrac{(x-y)(x^{n-1}+x^{n-2}y + ... + xy^{n-2}+y^{n-1})}{x-y} = x^{n-1}+x^{n-2}y + ... + xy^{n-2}+y^{n-1}=$ $$= ny^{n-1}$$

    %----------(v)
    \item $\lim\limits_{y\to x}\dfrac{x^n - y^n}{x-y} = x^{n-1}+x^{n-2}y + ... + xy^{n-2}+y^{n-1}= nx^{n-1}$\\\\

    %----------(vi)
    \item $\lim\limits_{h\to 0}  = \dfrac{\sqrt{a+h}-\sqrt{a}}{h} = \dfrac{\sqrt{a+h}+\sqrt{a}}{h}\cdot \dfrac{\sqrt{a+h}+\sqrt{a}}{\sqrt{a+h}+\sqrt{a}} = \dfrac{(\sqrt{a+h})^2 - (\sqrt{a})^2}{h(\sqrt{a+h} + \sqrt{a})}=\dfrac{1}{\sqrt{a+h}+\sqrt{a}} = $ $$=\dfrac{1}{2\sqrt{a}}$$\\\\
    \vspace{1cm}

\end{enumerate}

%--------------------2.
\item Hallar los límites siguientes:
\begin{enumerate}[\bfseries (i)]
    
    %----------(i)
    \item $\lim\limits_{x \to 1} \dfrac{1-\sqrt{x}}{1-x} =\lim\limits_{x \to 1} \dfrac{1-\sqrt{x}}{1-x}\cdot \dfrac{1+\sqrt{x}}{1+\sqrt{x}} =\lim\limits_{x \to 1} \dfrac{1^2 - (\sqrt{x})^2}{(1-x)(1+\sqrt{x})} =\lim\limits_{x \to 1} \dfrac{1}{1+\sqrt{x}} = \dfrac{1}{2}$\\\\
    
    %----------(iii)
    \item $\lim\limits_{x \to 0} \dfrac{1-\sqrt{1-x^2}}{x} =\lim\limits_{x \to 0} \dfrac{1-\sqrt{1-x^2}}{x} \cdot \dfrac{1+\sqrt{1-x^2}}{1+\sqrt{1-x^2}} =\lim\limits_{x \to 0} \dfrac{x}{1+\sqrt{1-x^2}} = 0$\\\\
    
    %----------(iii)
    \item $\lim\limits_{x \to 0} \dfrac{1-\sqrt{1-x^2}}{x^2} = \lim\limits_{x \to 0} \dfrac{1-\sqrt{1-x^2}}{x^2}\cdot \dfrac{1+\sqrt{1-x^2}}{1+\sqrt{1-x^2}} = \lim\limits_{x \to 0} \dfrac{1}{1+\sqrt{1+x^2}} = \dfrac{1}{2}$\\\\

\end{enumerate}

%--------------------3.
\item En cada uno de los siguientes casos, encontrar un $\delta$ tal que, $|f(x)-l|<\epsilon$ para todo $x$ que satisface $0<|x-a|<\delta$\\
\begin{enumerate}[\bfseries (i)]

    %----------(i).
    \item $f(x)=x^4; \; l=a^4$\\\\
	Respuesta.-\; Por la parte $(2)$ del lema anterior se tiene $$|x^2-a^2| < \min\left(1,\dfrac{\epsilon}{2(|a|^2+1)}\right).$$ Si aplicamos una vez mas la parte $(2)$ del lema obtenemos $$|x-a|<\min\left(1,\dfrac{\min\left(1,\dfrac{\epsilon}{2(|a|^2+1)}\right)}{2(|a|+1)}\right)=\min \left(1,\dfrac{\epsilon}{4(|a|^2 + 1)(|a|+1)}\right)=\delta$$\\

    %----------(ii).
    \item $f(x) =\dfrac{1}{x}; \; a=1,\; l=1$\\\\
	Respuesta.-\; Por la parte $(3)$ del lema se tiene $\left|\dfrac{1}{x} - 1\right|<\epsilon$ por lo tanto $|y-1|< \min\left(\dfrac{1}{2},\dfrac{\epsilon}{2}\right)$\\\\

    %----------(iii).
    \item $f(x)=x^4 + \dfrac{1}{x}; \; a=1, \; l=2$\\\\
	Respuesta.-\; Por la primera parte del lema se tiene $\left|\left(x^4+\dfrac{1}{x}\right)-(1+1)\right|<\epsilon$ de donde $$|x^4-1|<\dfrac{\epsilon}{2} \quad y \quad \left|\dfrac{1}{x}-1\right|<\dfrac{\epsilon}{2}$$
	Luego por el inciso $(i)$ y $(ii)$ $$\left|x-1\right|<\min\left(\dfrac{1}{2},\dfrac{\dfrac{\epsilon}{2}}{2}\right) \;\; y \;\; \left|x-1\right|<\min\left(1,\dfrac{1,\min\left(\dfrac{\dfrac{\epsilon}{2}}{2(1+1)}\right)}{2(1+1)}\right) \; \Longrightarrow \;|x-1|<\min\left(\dfrac{1}{2},\dfrac{\epsilon}{4},1,\dfrac{\epsilon}{32}\right)$$
	y por lo tanto $$|x-1|<\min\left(\dfrac{1}{2},\dfrac{\epsilon}{32}\right)=\delta$$\\


    %----------(iv).
    \item $f(x)=\dfrac{x}{1 + \sen^2 x}; \; a=0, \; l=0$\\\\
	Respuesta.-\; Sea $\left|\dfrac{x}{1+\sen^2 x}\right|<\epsilon \quad y \quad  |x|<\delta$ pero $\left|\dfrac{x}{1+\sen^2 x}\right|\leq |x|$ por lo tanto \\ $\left|\dfrac{x}{1+\sen^2 x}\right|\leq |x|<\delta=\epsilon$\\\\

    %----------(v).
    \item $f(x)=\sqrt{|x|};\; a=0,\; l=0$\\\\
	Respuesta.-\; Sea $\left|\sqrt{|x|}\right|<\epsilon$ entonces $\left|(|x|)^{1/2}\right| = \left(\sqrt{x^2}\right)^{1/2} = \left[(x^2)^{1/2}\right]^{1/2}=\sqrt{x}<\epsilon$, luego sabemos que la raíz cuadrada de $x$ debe ser siempre mayor o igual a $0$ por lo tanto $|x|<\epsilon^2$, de donde concluimos que $\delta=\epsilon^2$\\\\

    %----------(vi).
    \item $f(x)=\sqrt{x};\; a=1,\; l=1$\\\\
	Respuesta.-\; Si $\epsilon > 1$,  póngase $\delta = 1$. Entonces $|x-1|<\delta$ implica que $0<x<2$ con lo que $0<\sqrt{x}<2$ y $|\sqrt{x}-1|<1$. Si $\epsilon < 1$, entonces $(1-\epsilon)^2<x<(1+\epsilon)^2$ implica que $|\sqrt{x}-1|<\epsilon$, de modo que podemos elegir  un $\delta$ tal que $(1-\epsilon)^2 \leq 1 - \delta$ y $1+\delta\leq (1-\epsilon)^2$. Podemos elegir, pues $\delta = 2\epsilon - \epsilon^2$\\\\

\end{enumerate}

%--------------------4.
\item Para cada una de las funciones del problema $4-17$, decir para qué números $a$ existe el límite $\lim\limits_{x\to a}f(x)$

\begin{enumerate}[\bfseries (i)]
    
    %----------(i)
    \item Existe el límite si $a$ no es un entero, ya que en los puntos enteros la función tiene un salto.\\\\
    
    %----------(ii)
    \item Existe el límite si $a$ no es un entero.\\\\
    
    %----------(iii)
    \item De la misma forma que el inciso $(ii)$.\\\\
    
    %----------(iv)
    \item Existe para todo $a$.\\\\
    
    %----------(v)
    \item Existe para todo $a$ si sólo si sea $a=0$ y $a=\dfrac{1}{n}, \; n \in \mathbb{Z}, n\neq 0$.\\\\
    
    %----------(vi)
    \item El límite no existe para los puntos $|a| < 1 \; y \; a\neq \dfrac{1}{n}$\\\\

\end{enumerate}

%--------------------5.
\item 
\begin{enumerate}[\bfseries (a)]
    
    %----------(a)
    \item Hágase lo mismo para cada una de las funciones del problema $4-19$
    \begin{enumerate}[\bfseries (i)]
	
	%----------(i).
	\item Existe para cualquier número que tenga la forma $n+\dfrac{k}{10}, \; n,k\in \mathbb{Z}$\\\\
	
	%----------(ii).
	\item Existe para cualquier número que tenga la forma $n + \dfrac{k}{100},\; n,k \in \mathbb{Z}$\\\\
	
	%----------(iii).
	\item No es posible para ningún $a$.\\\\
	
	%----------(iv).
	\item De la misma forma que el anterior inciso.\\\\
	
	%----------(v).
	\item Existe para todo $a$ excepto para los que terminan en $7999...$\\\\
	
	%----------(vi).
	\item Existe para todo $a$ excepto para los que terminan en $1999...$.\\\\

    \end{enumerate}

    %----------(b)
    \item El mismo problema usando decimales infinitos que terminen en una fila de ceros en lugar de los que terminan en una fila de nueves.
    \begin{enumerate}[\bfseries (i)]
	
	%----------(i).
	\item De igual forma de la parte $(a)$ inciso $(i)$.\\\\
	
	%----------(ii).
	\item De igual forma de la parte $(a)$ inciso $(ii)$.\\\\
	
	%----------(iii).
	\item De igual forma de la parte $(a)$ inciso $(iii)$.\\\\
	
	%----------(iv).
	\item De igual forma de la parte $(a)$ inciso $(iii)$.\\\\
	
	%----------(v).
	\item Existe para todo $a$ excepto para los que terminan en $8000...$\\\\
	
	%----------(vi).
	\item Existe para todo $a$ excepto para los que terminan en $2000...$\\\\

    \end{enumerate}

\end{enumerate}

%--------------------6.
\item Supóngase que las funciones $f$ y $g$ tiene n la siguiente propiedad: Para todo $\epsilon>0$ y todo $x$, $$si \; 0<|x-2|<\sen^2\left(\dfrac{\epsilon^2}{9}\right) + \epsilon, \; entonces \; |f(x)-2|<\epsilon,$$ $$si \; 0<|x-2|<\epsilon^2, \; entonces \; |g(x)-4| < \epsilon.$$\\
Para cada $\epsilon>0$ hallar un $\delta>0$ tal que, para todo $x$,\\
\begin{enumerate}[\bfseries (i)]
    
    %----------(i)
    \item Si $0<|x-2|<\delta$, entonces $|f(x)+g(x)-6|<\epsilon$.\\\\
	Respuesta.-\; Por la primera parte del lema se tiene $|f(x)-2|<\dfrac{\epsilon}{2}$ y $|g(x)-4|<\dfrac{\epsilon}{2}$, luego remplazamos $\epsilon$ por $\epsilon/2$ de donde nos queda $$0<|x-2|<\sen^2 \left[\dfrac{\left(\dfrac{\epsilon}{2}\right)^2}{9}\right] \quad y \quad |x-2|<\left(\dfrac{\epsilon}{2}\right)^2$$
	Por último, solo hace verificar para todo $\epsilon>0$ existe  algún $\delta>0$. En este caso solo hace falta elegir $$0<|x-2|<\min\left[\sen^2 \left(\dfrac{\epsilon^2}{36}\right)+\epsilon,\dfrac{\epsilon^2}{4}\right] = \delta$$\\

    %----------(ii)
    \item Si $0<|x-2|<\delta$, entonces $|f(x)g(x) - 8|<\epsilon$\\\\\
	Respuesta.-\; Por la segunda parte del lema demostrado tenemos que $$|f(x)-2|<\min\left(1,\dfrac{\epsilon}{2(|4|+1)}\right) \quad y \quad |g(x)-4|<\dfrac{\epsilon}{2(|2|+1)}$$ ya que $|f(x)g(x)-2\cdot 4|<\epsilon$.\\ 
	Luego reemplazando en $\epsilon$ a cada parte obteniendo, 
	$$0<|x-2|<\min\left\{\sen^2 \left[\dfrac{\min\left(\dfrac{\epsilon}{10}\right)^2}{9}\right] + \min\left(1,\dfrac{\epsilon}{10}\right),\left[\min\left(1,\dfrac{\epsilon}{6}\right)\right]^2\right\}=\delta$$\\\\

    %----------(iii)
    \item Si $0<|x-2|<\delta$, entonces $\left|\dfrac{1}{g(x)}-\dfrac{1}{4}\right|<\epsilon$\\\\
	Respuesta.-\; Por la tercera parte del lema se tiene que $|g(x) - 4|<\min\left(\dfrac{|4|}{2},\dfrac{\epsilon|4|^2}{2}\right)$, luego remplazando en $\epsilon$ obtenemos $$|x-2|<\left[\min\left(2,8\epsilon \right)\right]^2 = \delta$$.\\

    %----------(iv)
    \item Si $0<|x-2|<\delta$, entonces $\left|\dfrac{f(x)}{g(x)}-\dfrac{1}{2}\right|<\delta$\\\\
	Respuesta.-\; Sea $\left|f(x)\dfrac{1}{g(x)}-2dfrac{1}{4}\right|$ entonces $$|f(x)-2|<\min\left(1,\dfrac{\epsilon}{2(|1/4|+1)}\right) \quad y \quad \dfrac{1}{g(x)} - \dfrac{1}{4}<\dfrac{\epsilon}{2(|2|+1)}$$, de donde $$0<|x-2|\min\left\{\sen^2\left[\dfrac{\left(\min(1,2\epsilon/5)\right)^2}{9}\right] + \min(1,2\epsilon/5),\left[\min\left(2,\dfrac{8\epsilon}{2(|2|+1)}\right)\right]^2\right\}=\delta$$\\\\ 

\end{enumerate}

%--------------------7.
\item Dese un ejemplo de una función $f$ para la cual la siguiente proposición sea falsa: Si $|f(x)-l|<\epsilon$ cuando $0<|x-a|<\delta$, entonces $|f(x)-l|<\epsilon/2$ cuando $0<|x-a|<\delta/2$.\\\\
    Respuesta.-\; Tomemos $a=0$ y $l=0$. Para $\epsilon>0$, se tiene $$|x-0|<\epsilon^2 \quad \Longrightarrow \quad |\sqrt{|x|}-0|<\epsilon$$. Aquí $\delta=\epsilon^2$. Pero si $$0<|x-0|<\dfrac{\epsilon^2}{2} \quad \Longrightarrow \quad |\sqrt{|x|}-0|< \dfrac{\epsilon^2}{4}=\delta.$$\\ El cual no se cumple la proposición buscada.\\\\

%--------------------8.
\item 
\begin{enumerate}[\bfseries (a)]

    %----------(a)
    \item Si no existen los límites $\lim\limits_{x \to a} f(x)$ y $\lim\limits_{x \to a} g(x)$, ¿pueden existir $\lim\limits_{x \to a} [f(x)+g(x)]$ o $\lim\limits_{x \to a} f(x)g(x)$?\\\\
	Respuesta.-\; Si. Por ejemplo considere $$f(x)=\dfrac{1}{x}, \quad g(x)=1-\dfrac{1}{x}$$ 
	Luego observe que $\lim\limits_{x \to 0} f(x)$ y $\lim\limits_{x\to 0} g(x)$ no existen, mientras que $f(x)+g(x)=1$ tiene un límite en $x=0$. De similar forma, si tomamos $f(x)=g(x)=\dfrac{|x|}{x}$ entonces $\lim\limits_{x\to 0} f(x)$ no existe, mientras que $f(x)\cdot g(x)=\dfrac{|x|^2}{x^2}$ es $1$ y, por lo tanto, existe el límite en $0$.\\\\

    %----------(b)
    \item Si existen los límites $\lim\limits_{x\to a} f(x)$ y  $\lim\limits_{x\to a} [f(x) + g(x)]$, ¿debe existir $\lim\limits_{x\to a} g(x)$?\\\\
	Respuesta.-\; Si, ya que $$g(x)=[f(x)+g(x)]-f(x)$$\\

    %----------(c)
    \item Si existe el límite $\lim\limits_{x\to a}f(x)$ y no existe el límite $\lim\limits_{x\to a} g(x)$, ¿puede existir $\lim\limits_{x\to a}[f(x)+g(x)]$?\\\\
	Respuesta.-\; No, ya que es sólo otro modo de enunciar la parte $(b)$.\\\\

    %----------(d)
    \item Si existe los límites $\lim\limits_{x\to a} f(x)=\lim\limits_{x\to a} f(x)g(x),$ ¿se sigue de ello que existe $\lim\limits_{x\to a} [f(x)+g(x)]$?\\\\
	Respuesta.-\; No, el razonamiento es análogo a la parte $(b)$, ya que si $g=(f\cdot g)/f$ no será aplicable si $\lim\limits_{x\to a}f(x)=0$.\\\\ 

\end{enumerate}

%--------------------9.
\item Demostrar que $\lim\limits_{x\to a} f(x) = \lim\limits_{h \to 0}f(a+h).$\\\\
    Demostración.-\; Sea $\lim\limits_{x\to a} f(x)$ y $g(h) f(a+h)$. Entonces para todo $\epsilon>0$ existe algún $\delta > 0$, tal que, para todo $x$, si $0<|x-a|<\delta$, entonces $|f(x)-l|<\epsilon$. Ahora bien, si $0<|h-0|<delta$, entonces $|(h+a)-a|<\delta$, de modo que $|f(h+a)-l|<\epsilon$. Esta desigualdad puede escribirse $|g(x)-l|<\epsilon$. Así pues, $\lim\limits_{h\to 0}g(h) = l,$ lo cual puede escribirse también $\lim\limits_{h\to 0} f(a+h) = l$. el mismo razonamiento demuestra que si $\lim_{h\to 0} f(a+h) = m$, entonces $\lim\limits_{x\to a} f(x) = m.$ Así pues, existe uno cualquiera de los dos límites si existe el otro, y en este caso son iguales.\\\\ 

%--------------------10.
\item 
\begin{enumerate}[\bfseries (a)]

    %----------(a)
    \item Demostrar que $\lim\limits_{x\to a} f(x) = l$ si y sólo si $\lim\limits_{x \to a} [f(x)-l] = 0$\\\\
	Demostración.-\; Por definición vemos que Para todo $\epsilon>0$ existe algún $\delta > 0$ tal que, para todo $x$, si $0<|x-a|<\delta$ entonces $|f(x) - l| < \epsilon$. Esta último desigualdad se puede escribir como $|[f(x)-l]-0| < \epsilon$ de modo que $\lim\limits_{x\to a}[f(x)-l] = 0$. El razonamiento en sentido inverso es igual de simple e intuitivo.\\\\

    %----------(b)
    \item Demostrar que $\lim\limits_{x\to 0}  =  \lim\limits_{x\to a}f(x-a)$\\\\
	Demostración.-\; Supóngase que $\lim\limits_{x\to 0} f(x) = m$Queremos mostrar que $\lim\limits_{x\to a} f(x-a) = m$. Para todo $\epsilon>0$ existen algún $\delta > 0$ tal que, para todo $x$ con $0<|x-0|<\delta$, entonces $|f(x)-m|<\epsilon \qquad (1)$. Si $0<|y-a|=|(y-a)-0|<\delta$, entonces por $(1)$ implica que $|f(y-a)-m| < \epsilon$, por lo tanto $\lim\limits_{y\to a} f(y-a) = m$.\\
	Por el contrario, supóngase $\lim\limits_{x\to a} f(x-a) = m$, donde queremos demostrar $\lim\limits_{x\to 0} f(x) = m$. Sea $\epsilon>0$, entonces existe $\delta > 0$ tal que, para todo $x$ con $0<|x-a|<\delta$, entonces $|f(x-a)-m|<\epsilon \qquad (2)$. Si $0<|y|=|(y+a)-a|<\delta$, luego por $(2)$ implica que $|f(y)-m|=|f[(y+a)-a]-m|<\epsilon$, por lo tanto $\lim\limits_{y\to 0} f(y) = m$.\\\\

    %----------(c)
    \item Demostrar que $\lim\limits_{x\to 0} f(x) = \lim\limits_{x\to 0} f(x^3)$.\\\\
	Demostración.-\; Sea $\lim_{x\to 0} f(x) = l.$ Para todo $\epsilon > 0$ existe algún $\delta > 0$, tal que, para todo $x$, si $0<|x|<\delta$ entonces $|f(x)-l|<\epsilon$. Tomemos $0<|x|<\min(1,\delta)$, entonces $0<|x^3|<\delta$, para comprender mejor tomemos un número en particular, por ejemplo $x = 0.9$ donde $0<|0.9|<\min(1,\delta)$ entonces se cumple que $0<|0.9^{3}|<\delta$. Así pues $\lim\limits_{x\to 0} f(x) = l$. Por otro lado, supongamos que $\lim\limits_{x\to 0} f(x^3)$ existe, pongamos $\lim\limits_{x\to 0} f(x^3) = m,$ entonces para todo $\epsilon>0$ existe algún $\delta > 0$ tal que, para todo $x$, si $0<|x|<\delta$, entonces $|f(x^3)-m|<\delta$. Si $0<|x|<\delta^3$, tenemos $0<|\sqrt[3]{x}|<\delta$, de modo que $|f(\sqrt[3]{x^3}^3)-m|<\epsilon$. Por lo tanto, $\lim\limits_{x\to 0} f(x) = m$.\\\\

    %----------(d)
    \item Dar un ejemplo en el que exista $\lim\limits_{x\to 0} f(x^2),$ pero no $\lim\limits_{x\to 0} f(x)$.\\\\
	Respuesta.-\; Sea $f(x)=1$ para $x\leq 0$ y $f(x) = -1$ para $x<0$. Entonces $\lim\limits_{x\to 0} f(x^2) = 1,$ pero $\lim\limits_{x \to 0} f(x)$ no existe. \\\\

\end{enumerate}

%--------------------11.
\item Supóngase que existe un $\delta > 0$ tal que $f(x) = g(x)$ cuando $0<|x-a|<\delta$. Demostrar que $\lim\limits_{x \to a} f(x) = \lim\limits_{x \to a} g(x)$.\\\\
    Demostración.-\; Asumamos que  $\lim\limits_{x \to a} f(x) = l$. Deseamos demostrar que $\lim\limits_{x \to a} g(x) = l$. Sea $\epsilon > 0$, de donde existe algún $\delta_1 > 0$ tal que si $0<|x-a|<\delta_1$, entonces $|f(x)-l|<\epsilon$. Luego pongamos $\delta^{'} = \min(\delta,\delta_1)$ que complace  $0<|x-a|<\delta^{'}$, en virtud de como se define $\delta^{'}$ sabemos que si $0<|x-a|<\delta_1$ y $0<|x-a|<\delta$ tal que $f(x)=g(x)$ entonces $|f(x)-l|<\epsilon$, de donde concluimos que $|g(x) - l|<\epsilon$.\\\\

%--------------------12.
\item 
\begin{enumerate}[\bfseries (a)]

    %----------(a)
    \item Supóngase que $f(x)\leq g(x)$ para todo $x$. Demostrar que $\lim\limits_{x \to a} = \lim\limits_{x\to a} g(x)$ siempre que estos existan.\\\\
	Demostración.-\; Demostremos por reducción al absurdo. Supóngase que $l = \lim\limits_{x\to a} f(x) > \lim\limits_{x\to a} g(x) = m$. Luego sea $l-m>0$, existe entonces un $\delta > 0$ tal que, si $0<|x-a|<\delta$, entonces $l-f(x)<\epsilon/2$ y $|m-g(x)|<\epsilon/2$. Así pues, para $0|x-a|<\delta$ tenemos $$g(x)<m+\epsilon/2 = l - \epsilon/2 < f(x),$$ contrario a la hipótesis.\\\\

    %----------(b)
    \item ¿De qué modo puede obtenerse una hipótesis más débil?.\\\\
	Respuesta.-\; Basta suponer que $f(x)\leq g(x)$ para todo $x$ que satisfaga $0|x-a|<\delta,$ para algún $\delta>0$.\\\\

    %----------(c)
    \item Si $f(x)<g(x)$ para todo $x$. ¿Se sigue de ello necesariamente que $\lim\limits_{x\to a}f(x)<\lim\limits_{x\to a}g(x)$?\\\\
	Respuesta.-\; No necesariamente ya que si $f(x)=0$ y $g(x) =  |x|$ para $x\neq 0$, y $g(0) = 1$ entonces $\lim\limits_{x\to a} f(x) = 0 = \lim\limits_{x\to a}g(x)$.\\\\

\end{enumerate}

%--------------------13.
\item Supóngase que $f(x)\leq g(x) \leq h(x)$ y que $\lim\limits_{x \to a} f(x) = \lim\limits_{x\to a} g(x)$. Demostrar que existe $\lim\limits_{x\to a} g(x)$ y que $\lim\limits_{x\to a} g(x) = \lim\limits_{x\to a} f(x) = \lim\limits_{x\to a} h(x)$.\\\\
	Demostración.-\; Intuitivamente vemos que $g(x)$ esta entre $f(x)$ y $h(x)$ donde se aproximan a un mismo número. Sea $\lim\limits_{x\to a} f(x) = l$. Para todo $\epsilon>0$, existe $\delta>0$ tal que si para todo $x$, si $0<|x-a|<\delta$, entonces $|h(x)-l|<\epsilon$, como también para $|f(x) -l|<\epsilon$, así pues, si $0<|x-a|<\delta$, entonces $$l-\epsilon<f(x)\leq g(x) \leq h(x) < l + \epsilon$$ de modo que $|g(x) - l|<\epsilon$.\\\\

%--------------------14.
\item 
\begin{enumerate}[\bfseries (a)]

    %----------a)
    \item Demostrar que si $\lim\limits_{x\to 0} f(x)/x = l$ y $b\neq 0$, entonces $\lim\limits_{x\to 0} f(bx)/x = bl.$\\\\
	Demostración.-\; Tengamos en cuenta que ${x\to 0}$ implica ${bx\to 0}$ siempre que $b$ sea distinto de $0$. Luego $g(x)=\dfrac{f(x)}{x}$ de donde $\lim\limits_{x\to a} g(x) = l$, así $\lim\limits_{bx\to 0} g(bx) = l$, aclaremos que cuando $g(bx)$ solo ponemos un valor diferente sin alterar la función en si, es decir, sea $bx=y$ y $\lim\limits_{bx\to 0} g(bx) = l$ entonces $\lim\limits_{y \to 0} g(y) = l$ que es igual a nuestra hipótesis $\left(\lim\limits_{x\to 0} g(x) = l\right)$.
	Por lo tanto tenemos,
	$$\lim_{x\to 0} \dfrac{f(bx)}{x} = \lim_{x\to 0} b\dfrac{f(bx)}{bx} = b\lim_{y\to 0} \dfrac{f(y)}{y} = bl$$.\\

    %----------b)
    \item ¿Qué ocurre si $b=0$?\\\\
	Respuesta.-\; Si $b=0$ entonces $\dfrac{f(bx)}{bx} = \dfrac{f(0)}{0} $ el cual no esta definido, por lo tanto el límite no existe, a menos que $f(0) = 0$.\\\\

    %----------c)
    \item La parte $(a)$ nos permite hallar $\lim\limits_{x\to 0}(\sen 2x)/x$ en función de $\lim\limits_{x\to 0} (\sen x)/x$. Hallar este límite por otro procedimiento.\\\\
	Respuesta.-\; $$\lim_{x\to 0} \dfrac{\sen 2x}{x} = \lim_{x\to 0}\dfrac{2(\sen x)(\cos x)}{x} = 2 \lim_{x\to 0} \cos x \lim_{x\to 0} \dfrac{\sen x}{x} = 2.$$\\

\end{enumerate}

%--------------------15.
\item Calcular los límites siguientes en función del número $\alpha = \lim\limits_{x\to 0} (\sen x)/x.$
\begin{enumerate}[\bfseries (i)]

    %-----------(i)
    \item $\lim\limits_{x\to 0} \dfrac{\sen 2x}{x}$\\\\
	Respuesta.-\; $$\lim_{x\to 0} \dfrac{\sen 2x}{x} = \lim_{x\to 0}\dfrac{2(\sen x)(\cos x)}{x} = 2 \lim_{x\to 0} \cos x \lim_{x\to 0} \dfrac{\sen x}{x} = 2.$$\\\\

    %----------(ii)
    \item $\lim\limits_{x \to 0} \dfrac{\sen ax}{\sen bx}$.\\\\
    Respuesta.-\; $$\lim_{x \to 0} \dfrac{\sen ax}{\sen bx}\cdot \dfrac{x}{x} = \lim_{x\to 0} \dfrac{\sen ax}{x} \lim_{x\to 0} \dfrac{x}{\sen bx}  = \lim_{x\to 0} \dfrac{\sen ax}{x} \dfrac{1}{b \cdot \lim_{x\to 0} \frac{\sen x}{x}} = a\alpha \cdot \dfrac{1}{b\alpha} = \dfrac{a}{b}$$\\

    %----------(iii)
    \item $\lim\limits_{x\to 0} \dfrac{\sen^2 2x}{x}$.\\\\
	Respuesta.-\; $$\lim_{x\to 0} \sen 2x \lim_{x\to 0} \dfrac{\sen x}{x} = 0\cdot 2\alpha = 0 $$\\\\ 

    %----------(iv)
    \item $\lim\limits_{x\to 0} \dfrac{\sen^2 2x}{x^2}$\\\\
	Respuesta.-\; $$\left(\lim_{x\to 0} \dfrac{\sen 2x}{x}\right) = 4\alpha ^2$$\\\\ 

\end{enumerate}

%--------------------16.
\item 

\end{enumerate}


    %---------- funciones continuas
	%\chapter{Funciones Continuas}

% ---------- definición 
\begin{tcolorbox}[colframe = white]
    \begin{def.} La función $f$ es continua  en $a$ si $$\lim\limits_{x\to a} f(x) = f(a)$$
	Para todo $\epsilon > 0$ existe un $\delta > 0$ tal que, para todo $x$, si $0<|x-a|<\delta$.\\
	Pero en este caso, en que el límite es $f(a)$, la frase $$0<|x-a|<\delta$$
	puede cambiarse por la condición más sencilla $$|x-a|<\delta$$
	puesto que si $x=a$ se cumple ciertamente que $|f(x)-f(a)|<\epsilon$.
    \end{def.}
\end{tcolorbox}

%---------- teorema 1
\begin{teo} Si $f$ y $g$ son continuas en $a$, entonces 
    \begin{center}
	\begin{tabular}{rl}
	    $(1)$ & $f+g$ es continua en $a$.\\
	    $(2)$ & $f\cdot g$ es continua en $a$.\\
    Además, si $g(a)\neq 0$, entonces 
	    $(3)$&$1/g$ es continua en $a$\\
	\end{tabular}
    \end{center}	
    Demostración.-\; Puesto que $f$ y $g$ son continuas en $a$,
    $$\lim_{x\to a}f(x) = f(a) \qquad y \qquad \lim_{x\to a} g(x) = g(a).$$
    Por el teorema 2(1) del capítulo 5 esto implica que 
    $$\lim_{x\to a} (f+g)(x) = f(a)+g(a) = (f+g)(a),$$
    lo cual es precisamente la afirmación de que $f+g$ es continua en $a$.\\
    Para $f\cdot g$ se tiene que $$\lim_{x\to a}(f\cdot g)(x) = f(a)\cdot fg(a) = (f\cdot g)(a)$$
    Por último para $1/g$ tenemos que $$\lim_{x\to a} 1/g = 1/g(a) , \qquad para \; g(a)\neq 0$$\\
\end{teo}

%---------- teorema 2
\begin{teo} Si $g$ es continua en $a$, y $f$ es continua en $g(a)$, entonces $f\circ g$ es continua en $a$.\\\\
    Demostración.-\; Sea $\epsilon > 0$. Queremos hallar un $\delta > 0$ tal que para todo $x$, 

    \begin{center}
	Si $|x-a|<\delta$ entonces $|(f\circ g)(x)-(f\circ g)(a)|<\epsilon$, es decir, $|f(g(x))-f(g(a))|<\epsilon$
    \end{center}

	Tendremos que aplicar primero la continuidad de $f$ para estimar cómo de cerca tiene que estar $g(x)$ de $g(a)$ para que se cumpla esta desigualdad. Puesto que $f$ es continua en $g(a)$, existe un $\delta^{'} > 0$ tal que para todo $y$,

    \begin{center}
	Si $|y-g(a)|<\delta^{'}$, entonces $|f(y)-f(g(a))|<\epsilon. \qquad (1)$ 
    \end{center}

    En particular, esto significa que

    \begin{center}
	Si $|g(x)-g(a)|<\delta^{'}$, entonces $|f(g(x))-f(g(a))|<\epsilon. \qquad (2)$
    \end{center}
    
    Aplicamos ahora la continuidad de $g$ para estimar cómo de cerca tiene que estar $x$ de $a$ para que se cumpla la desigualdad $|g(x)-g(a)|<\delta^{'}$. El número $\delta^{'}$ es un número positivo como cualquier otro número positivo; podemos, por lo tanto, tomar $\delta^{'}$ como el $epsilon$ de la definición de continuidad de $g$ en $a$. Deducimos que existe un $\delta > 0$ tal que, para todo $x$,

    \begin{center}
	Si $|x-a|<\delta,$ entonces $|g(x)-g(a)|<\delta^{'}, \qquad (3)$
     \end{center}

     combinando (2) y (3) vemos que para todo $x$,

    \begin{center}
	Si $|x-a|<\delta,$ entonces $|f(g(x))-f(g(a))|<\epsilon.$
     \end{center}
     \vspace{1cm}

\end{teo}

% definición
\begin{tcolorbox}[colframe = white]
    \begin{def.} Si $f$ es continua en $x$ para todo $x$ en $(a,b)$, entonces se dice que $f$ es continua en $(a,b)$ si 
	\begin{center}
	    $f$ es continua en $x$ para todo $x$ de $(a,b), \qquad (1)$\\
	    \vspace{.5cm}
	    $\lim\limits_{x\to a^+} f(x) = f(a)$ y $\lim\limits_{x\to b^-}f(x) = f(b). \qquad (2)$
	\end{center}
    \end{def.}
\end{tcolorbox}

%---------- teorema 3
\begin{teo} Supóngase que $f$ es continua en $a$, y $f(a)>0$. Entonces existe un número $\delta>0$ tal que $f(x)>0$ para todo $x$ que satisface $|x-a|<\delta$. Análogamente, si $f(a)>0$, entonces existe un número $\delta>0$ tal que $f(x)<0$ para todo $x$ que satisface $|x-a|<\delta.$\\\\
    Demostración.-\; Considérese el caso $f(a)>0$ puesto que $f$ que es continua en $a$, si $\epsilon>0$ existe un $\delta>0$ tal que, para todo $x$,
    \begin{center}
	Si $|x-a|<\delta$, entonces $|f(x)-f(a)|<\epsilon.$
    \end{center}
    Puesto que $f(a)>0$ podemos tomar a $f(a)$ como el $epsilon$. Así, pues, existe $\delta>0$ tal que para todo $x$,
    \begin{center}
	Si $|x-a|<\delta$, entonces $|f(x)-f(a)|<f(a)$
    \end{center}
    Y esta última igualdad implica $f(x)>0$.\\
    Puede darse una demostración análoga en el caso $f(a)<0$; tómese $\epsilon=-f(a)$. O también se puede aplicar el primer caso a la función $-f$.\\\\

\end{teo}

\section{Problemas}
\begin{enumerate}[\Large\bfseries 1.]

%--------------------1.
\item ¿para cuáles de las siguientes funciones $f$ existe una función $F$ de dominio $R$ tal que $F(x)=f(x)$ para todo $x$ del dominio de $f$?

    \begin{enumerate}[\Large\bfseries (i)]

	%---------- (i)
	\item $f(x) = \dfrac{x^2 - 4}{x - 2}$\\\\
	    Respuesta.-\; Sabiendo que el límite cuando $x$ tiende a $2$ existe, entonces existe una función $F$ de dominio $R$ tal que $F(x)=f(x)$ para todo $x$ del dominio de $f$.\\\\

	%---------- (ii)
	\item $f(x) = \dfrac{|x|}{x}$\\\\
	    Respuesta.-\; No existe $F$, ya que $\lim\limits_{x\to 0} \dfrac{|x|}{x}$ no existe.\\\\

	%---------- (iii)
	\item $f(x) = 0,\; x$ irracional.\\\\
	    Respuesta.-\; Existe $F$ de dominio $R$ tal que $F(x)=f(x)$ para todo $x$ del dominio de $f$.\\\\

	%---------- (iv)
	\item $f(x) = 1/q, \; x = p/q$ racional en fracción irreducible.
	    Respuesta.-\; No existe $F$, ya que $F(a)$ tendría que ser $0$ para los $a$ irracionales, y entonces $F$ no podría ser continua en $a$ si $a$ es racional.\\\\

    \end{enumerate}

%--------------------2.
\item ¿En qué puntos son continuas las funciones de los problemas 4-17 y 4-19?.\\\\
    Respuesta.-\; Problema 4-17.\\
    Para (i), (ii) y (iii) son continuas para todos los puntos menos para los enteros. Para (iv) es continua en todos los puntos.Para (v) es entera para todos los puntos excepto para $0$ y $1/n$ para $n$ en los enteros.\\\\
    Problema 4-19.\\
    (i) todos los puntos que no sean de la forma $n+k/10$ para todos los enteros $k$ y $n$. El (ii) para todo los puntos que no sea de la forma $n+k/100$ para todos los enteros $k$ y $n$. (iii) y (iv) para ningún punto. (v) para todos los puntos que el decimal no termine en $7999 \ldots$. Y (vi) para todos los puntos que el decimal contenga al menos un $1$.\\\\

%--------------------3.
\item 
    \begin{enumerate}[\Large\bfseries (a)]

	%---------- (a)
	\item Supóngase que $f$ es una función que satisface $|f(x)|\leq |x|$ para todo $x$. Demostrar que $f$ es continua en $0$.[Observe que $f(0)$ debe ser igual a $0$.]\\\\
	    Demostración.-\; Supongamos que $|f(x)|\leq |x|$. afirmamos que, $\lim\limits_{x\to 0} f(x)=0$. De hecho dado $\epsilon>0$, tomamos $\delta=\epsilon$. Si $|x|<\delta$ entonces $|f(x)|\leq |x|<\delta = \epsilon$. Esto prueba que $\lim\limits_{x\to 0} f(x) = 0$. Para concluir que $f$ es constante en $0$, tenga en cuenta que, como se señala en la pregunta, aplicar $|f(x)|\leq |x|$ para todo $x$, en $x=0$ se da $|f(0)|\leq 0$ y en consecuencia $f(0)=0$.  Por lo tanto $\lim\limits_{x \to 0} f(x) = 0$ implica que $\lim\limits_{x\to 0} f(x) = f(0)$, así $f$ es constante en $0$.\\\\

	%---------- (b)
	\item Dar un ejemplo de una función $f$ que no sea continua en ningún $a\neq 0$.\\\\
	    Respuesta.-\; Sea $f(x)=0$ para $x$ irracional, y $f(x)=x$ para $x$ racional.\\\\

	%---------- (c)
	\item Supóngase que $g$ es continua en $0$, $g(0)=0,$ y $|f(x)|\leq |g(x)|.$ Demostrar que $f$ es continua en $0$.\\\\
	    Demostración.-\; La condición $|f(x)|\leq |g(x)|$ para todo $x$ y $g(0)=0$ implica que $ff(0)=0$, así que sólo tenemos que demostrar que $\lim\limits_{x\to 0} f(x) = 0$.\\
	    Sea $\epsilon>0$, luego ya que $g$ es continua en $0$, existe un $\delta >0$ tal que $|x|<\delta$ entonces $|g(x)-g(0)|=|g(x)|<\epsilon$. Usando $|f(x)|\leq |g(x)|$ para todo $x$, vemos que $|x|<\delta$ implica $|f(x)|\leq |g(x)|<\epsilon$. Por lo tanto esto demuestra que $\lim\limits_{x\to 0} f(x) = 0$.\\\\

    \end{enumerate}

%--------------------4.
\item Dar un ejemplo de una función $f$ que no sea continua en ningún punto, pero tal que $|f|$ sea continua en todos lo puntos.\\\\
    Respuesta.-\; Sea $f(x)=1$ para $x$ racional, y $f(x)=-1$ para $x$ irracional.\\\\

%--------------------5.
\item Para todo número $a$, hallar la función que sea continua en $a$, pero no lo sea en ningún otro punto.\\\\
    Respuesta.-\; Sea $f(x)=a$ para $x$ irracional, y $f(x)=x$ para $x$ racional.\\\\

%--------------------6.
\item 
\begin{enumerate}[\bfseries (a)]

    %---------- (a)
    \item Hallar una función $f$ que sea descontinua en $1,\frac{1}{2},\frac{1}{3},\ldots,$ pero continua en todos los demás puntos.\\\\
	Respuesta.-\; Define $f$ como sigue,
	$$f(x) = \left\{\begin{array}{rl}
	0,&x\leq 0\\\\
	\dfrac{1}{\left[\dfrac{1}{x}\right]},&0<x\leq 1\\\\
	  2,&x>1\\\\
	\end{array}\right.$$

    %---------- (b)
    \item Hallar una función $f$ que sea descontinua en $1,\frac{1}{2},\frac{1}{3},\ldots,$ y en $0$, pero sea continua en ningún en todos los demás puntos.\\\\
	Respuesta.-\; Sea 
	$$f(x) = \left\{\begin{array}{rl}
	-1,&x\leq 0\\\\
	\dfrac{1}{\left[\dfrac{1}{x}\right]},&0<x\leq 1\\\\
	  2,&x>1\\\\
	\end{array}\right.$$

\end{enumerate}

%--------------------7.
\item Supóngase que  $f$ satisface $(x+y) = f(x)+f(y)$, y que $f$ es continua en $0$. Demostrar que $f$ es continua en $a$ para todo $a$.\\\\
    Demostración.-\; Sea $f(x+0) = f(x)+f(0)$, por lo tanto $f(0)=0$, entonces
    $$\begin{array}{rcl}
	\lim\limits_{h\to 0} f(a+h) - f(a)&=&\lim\limits_{h\to 0}f(a)+f(h)-f(a)\\\\
	&=&\lim\limits_{h\to 0}f(h)\\\\
	&=&\lim\limits_{h\to 0} f(h)-f(0) = 0\\\\
    \end{array}$$

%--------------------8.
\item Supóngase que $f$ es continua en $a$ y $f(a)=0$. Demostrar que si $\alpha \neq 0$, entonces $f+\alpha$ es distinta de $0$ en algún intervalo abierto que contiene $a$.\\\\
    Demostración.-\; Sabiendo que $(f+\alpha)(a)\neq 0$, entonces por el teorema 3, $f+\alpha$ es distinto de cero en algún intervalo que contiene a $a$.\\\\

%--------------------9.
\item 
\begin{enumerate}[\bfseries (a)]

    %---------- (a)
    \item Supóngase que $f$ no es continua en $a$. Demostrar que para algún $\epsilon>0$ existen números $x$ tan próximos como se quiere de $a$ con $|f(x)-f(a)|>\epsilon$.\\\\
	Demostración.-\; Lógicamente equivalente a la definición de continuidad se tiene 
	$$\exists\,\epsilon>0, \forall \delta>0, \exists\, x |x-a|<\delta \; y \; |f(x)-f(a)|\geq \epsilon$$
	Existe $\epsilon > 0$ tal que $|f(x)-f(a)|>\epsilon$. Luego sea $\epsilon^{'}=\dfrac{1}{2}\epsilon$, entonces tenemos $|f(x)-f(a)|\geq \epsilon > \epsilon^{'}.$\\\\

    %---------- (b)
    \item Dedúzcase que para algún $\epsilon>0$, o bien existen números $x$ tan próximos como se quiera de $a$ con $f(x)<f(a)-\epsilon$ o bien existen números $x$ tan próximos como se quiera de $a$ con $f(x)>f(a)+\epsilon$.\\\\
	Demostración.-\; La demostración es directa aplicando la reciproca de la definicion de continuidad. Como se vio en el inciso $a$.\\\\

\end{enumerate}

%--------------------10.
\item 
\begin{enumerate}[\bfseries (a)]

    %---------- (a)
    \item Demostrar que si $f$ es continua en $a$, entonces también lo es $|f|$.\\\\
	Demostración.-\; Ya que $\lim\limits_{x\to a} f(x) = l \; \Longrightarrow \; \lim\limits_{x\to a} |f|(x) = |l|$ como se vio en el problema 5-16, entonces 
	$$\lim_{x\to a} |f|(x) = \left|\lim_{x\to a} f(x)\right| = |f(a)| = |f|(a).$$\\

    %---------- (b)
    \item Demostrar que toda función continua $f$ puede escribirse en la forma $f=E+O$, donde $E$ es par y continua y $O$ es impar y continua.\\\\
	Demostración.-\; Por el problema 13 del capítulo 3 (funciones) mostramos que $E$ y $O$ son continuas si $f$ lo es.\\\\

    %---------- (c)
    \item Demostrar que si $f$ y $g$ son continuas, también lo son $\max(f,g)$ y $\min(f,g)$.\\\\
	Demostración.-\; Por la parte a) y sabiendo que 
	$$\begin{array}{rcl}
	    \max(f,g)&=&\dfrac{f+g+|f-g|}{2}\\\\
	    \min(f,g)&=&\dfrac{f+g-|f-g|}{2}\\\\
	    \end{array}$$

    %---------- (d)
    \item Demostrar que toda función continua $f$ puede escribirse en la forma $f=g-h$, donde $g$ y $h$ son no negativas y continuas.\\\\
	Demostración.-\; Por el problema 15 del capítulo 3 (funciones) podemos comprobar que $f=g-h$ siempre que $f$ sea continua.\\\\ 

\end{enumerate}

%--------------------11.
\item Demostrar el teorema 1(3) aplicando el teorema 2 y la continuidad de la función $f(x)=1/x$.\\\\
    Demostración.-\; Sea $f\circ g = \dfrac{1}{g}$ y $f$ es continua en $g(a)$ para $g(a)\neq 0$, entonces por el teorema 2, se tiene que $\dfrac{1}{g}$ es continua en $a$ para $g(a)\neq 0$.\\\\

%--------------------12.
\item 
    \begin{enumerate}[\bfseries (a)]

	%---------- (a)
	\item Demostrar que si $f$ es continua en $l$, y $\lim\limits_{x\to a} g(x) = l$ entonces $\lim\limits_{x\to a}f(g(x)) = f(l)$.\\\\
	    Demostración.-\; Sea 
	    $$G(x) = \left\{\begin{array}{rcl}
		    g(x)&si&x\neq a\\\\
			l&si&x=a\\\\
	    \end{array}\right.$$
	    entonces $G$ es continua en $a$, ya que $G(a)=l=\lim\limits_{x\to a} g(x) = \lim_{x\to a} G(x)$. Así $f\circ G$ es continua en $a$ esto por el teorema 2. Luego 
	    $$f(l)=f(G(a))=(f\circ G)(a) = \lim\limits_{x\to a} = \lim\limits_{x\to a}(f\circ G)(x) = \lim\limits_{x\to 0} f(g(x)).$$\\

	%---------- (b)
	\item Demostrar que si no se supone la continuidad de $f$ en $l$, entonces no se cumple, por lo general, que $\lim\limits_{x\to a} f(g(x)) = f\left(\lim\limits_{x\to a} g(x)\right)$.\\\\
	    Demostración.-\; Sea $g(x)=l+x-a$ y 
	    $$f(x) = \left\{ \begin{array}{rl}
		    0,&x\neq 1\\
		    1, & x=1\\
		\end{array}\right.$$

		Luego $\lim\limits_{x\to a} g(x) = l$, así $f\left(\lim\limits_{x\to a} g(x)\right) = f(l) = l$, pero $g(x)\neq l$ para $x\neq a$, por lo tanto $\lim\limits_{x\to a} f(g(x)) = \lim\limits_{x\to a} = 0$.\\\\

    \end{enumerate}

%--------------------13.
\item
\begin{enumerate}[\bfseries (a)]

    %---------- (a)
    \item Demostrar que si $f$ es continua en $[a,b]$, entonces existe una función $g$ el cual es contuna en $\mathbb{R}$ y que satisface a $g(x)=f(x)$ para todo $x$ en $[a,b]$.\\\\
    Demostración.-\; 

    %---------- (b)
    \item Hágase ver con un ejemplo que ésta afirmación es falsa si se sustituye $[a,b]$ por $(a,b)$.\\\\
	Respuesta.-\; Definimos $f(x)=1/(x^2-1)$ en el intervalo $(-1,1)$. Es continuo, pero no existe $\lim\limits_{x\to 1^+} g(x)$ ni $\lim\limits_{x\to 1^-} f(x)$. Ahora, para que $f$ se extienda a una función $g$ que sea constante en toda la línea real, es necesario que existan tanto $\lim\limits_{x \to -1} g (x)$ como $\lim\limits_{x \to 1} g (x)$, lo que requiere $\lim\limits_{x \to -1^+}  f (x)$ y $\lim\limits_{x \to 1^-} f (x)$ para existir. Entonces $f$ no se puede extender a una función que sea constante en toda la línea real.\\\\

\end{enumerate}

%--------------------14.
\item
\begin{enumerate}[\bfseries (a)]

    %---------- (a)
    \item Supóngase que $g$ y $h$ son continuos en $a$, y que $g(a)=h(a)$, Defínase $f(x)$  como $g(x)$ si $x\geq a$ y $h(x)$ si $x\leq a$. Demuestre que $f$ es continua en $a$.\\\\
	Demostración.-\; Sea 
	$$f(x) = \left\{\begin{array}{rcl}
	    g(x)&si&x\geq a\\
	    h(x)&si&x\leq a\\
	\end{array}\right.$$
	Luego se tiene $$\lim\limits_{x\to a} g(x) = g(a)\quad  y \quad \lim\limits_{x\to a} h(x) = h(a)$$ de donde $$g(a) = \lim\limits_{x\to a^+} g(x) = \lim\limits_{x\to a^+} f(x) = f(a) \quad y \quad h(a) = \lim\limits_{x\to a^-} h(x) = \lim\limits_{x\to a^-} f(x) = f(a)$$
	y por lo tanto $$\lim\limits_{x\to a} f(x) = f(a)$$\\

    %---------- (b)
    \item Supóngase que $g$ es continuo en $[a,b]$ y $h$ es continuo en $[b,c]$ y $g(b)=h(b)$. Sea $f(x)$ igual $g(x)$ para $x$ en $[a,b]$ y $h(x)$ para $x$ en $[b,c]$. Demuestre que $f$ es continuo en $[a,c]$. (Así pues, las funciones continuas pueden ser $"$pegados juntos$"$).\\\\
	Demostración.-\; 

\end{enumerate}



\end{enumerate}


    %---------- Tres teoremas fuertes
	%
\chapter{Tres teoremas fuertes}


% -------------------- teorema 1 --------------------
\begin{tcolorbox}[colback = white]
    \begin{teo}
	Si $f$ es continua en $[a,b]$ y $f(a)<0<f(b)$ entonces existe algún $x$ en $[a,b]$ tal que $f(x)=0$.\\\\
	Geométricamente, esto significa que la gráfica de una función continua que empieza por debajo del eje horizontal y termina por encima del mismo debe cruzar a este eje en algún punto.
    \end{teo}
\end{tcolorbox}

% -------------------- teorema 2 --------------------
\begin{tcolorbox}[colback = white]
    \begin{teo}
	Si $f$ es continua en $[a,b]$, entonces $f$ está acotada superiormente en $[a,b]$, es decir, existe algún número $N$ tal que $f(x)\leq N$ para todo $x$ en $[a,b]$.\\\\
	Geométricamente, este teorema significa que la gráfica $f$ queda por debajo de alguna línea paralela al eje horizontal. 
    \end{teo}
\end{tcolorbox}

% -------------------- teorema 3 --------------------
\begin{tcolorbox}[colback = white]
    \begin{teo}
	Si $f$ es continua en $[a,b]$ entonces existe algún número $y$ en $[a,b]$ tal que $f(y)\leq f(x)$ para todo $x$ en $[a,b]$.\\\\
	Se dice que una función continua en un intervalo cerrado alcanza su valor máximo en dicho intervalo.
    \end{teo}
\end{tcolorbox}

% -------------------- teorema 4 --------------------
\begin{teo}
    Si $f$ es continua en $[a,b]$ y $f(a)<c<f(b)$, entonces existe algún $x$ en $[a,b]$ tal que $f(x)=c$.\\\\
    Demostración.-\; Sea $g=f-c$. Entonces $g$ es continua, y $\; g(a)+c < c < g(b) + c \; \Longrightarrow \; g(a)<0<g(b)$. Por el teorema 1, existe algún $x$ en $[a,b]$ tal que $g(x)=0$. Pero esto significa que $f(x)=c.$\\\\
\end{teo}

% -------------------- teorema 5 --------------------
\begin{teo}
    Si $f$ es continua en $[a,b]$ y $f(a)>c>f(b)$, entonces existe algún $x$ en $[a,b]$ tal que $f(x)=c$.\\\\
    Demostración.-\; La función $-f$ es continua en $[a,b]$ y $-f(a)<-c<-f(b)$. Por el teorema 4 existe algún $x$ en $[a,b]$ tal que $-f(x)=-c$, lo que significa que $f(x)=c$.\\\\
\end{teo}

Si una función continua en un intervalo toma dos valores, entonces toma todos los valores comprendidos entre ellos; esta ligera generalización del teorema 1 recibe a menudo el nombre de \textbf{teorema de los valores intermedios}.\\\\ 

%---------------------- teorema 6 ----------------------
\begin{teo}
    Si $f$ es continua en $[a,b]$, entonces $f$ es acotada inferiormente en $[a,b]$, es decir, existe algún número $N$ tal que $f(x)\geq N$ para todo $x$ en $[a,b]$.\\\\
    Demostración.-\; La función $-f$ es continua en $[a,b]$, así por el teorema 2 existe un número $M$ tal que $-f(x)\leq M$ para todo $x$ en $[a,b]$. Pero esto significa que $f(x)\geq -M$ para todo $x$ en $[a,b]$, así podemos poner $N=-M$.\\\\
\end{teo}

Los teoremas 2 y 6 juntos muestran  que una función continua $f$ en $[a,b]$ son acotados en $[a,b]$, es decir, existe un número $N$ tal que $|f(x)|\leq N$ para todo $x$ en $[a,b]$. En efecto, puesto que el teorema 2 asegura la existencia de un número $N_1$ tal que $f(x)\leq N,$ para todo $x$ de $[a,b]$ y el teorema 6 asegura la existencia de un número $N_2$ tal que $f(x)\geq N$, para todo $x$ en $[a,b]$, podemos tomar $N=\max(|N_1|,|N_2|)$.\\\\

%---------------------- teorema 7 ----------------------
\begin{teo}
    Si $f$ es continua en $[a,b]$, entonces existe algún $y$ en $[a,b]$ tal que $f(y)\leq f(x)$ para todo $x$ en $[a,b]$.\\\\
    Demostración.-\; La función $-f$ es continua en $[a,b]$; por el teorema 3 existe algún $y$ en $[a,b]$ tal que $-f(y)\geq -f(x)$ para todo $x$ en $[a,b]$, lo que significa que $f(y)\leq f(x)$ para todo $x$ en $[a,b]$.\\\\
\end{teo}

%---------------------- teorema 8 ----------------------
\begin{teo}
    Todo número positivo posee una raíz cuadrada. En otras palabras, si $\alpha > 0$, entonces existe algún número $x$ tal que $x^2=\alpha$.\\\\
    Demostración.-\; Consideremos la función $f(x)=x^2$, el cual es ciertamente continuo. Notemos que la afirmación del teorema puede ser expresado en términos de $f$: $"$el número $\alpha$ posee una raíz cuadrada$"$ significa que $f$ toma el valor $alpha$. La demostración de este hecho acerca de $f$ será una consecuencia fácil del teorema 4.\\
    Existe, evidentemente, un número $b>0$ tal que $f(b)>\alpha$; en efecto, si $\alpha>1$ podemos tomar $b=\alpha$, mientras que si $\alpha<1$ podemos tomar $b=1$. Puesto que $f(0)<\alpha <f(b)$, el teorema 4 aplicado a $[0,b]$ implica que para algún $x$ de $[0,b]$, tenemos $f(x)=\alpha$, es decir, $x^2=\alpha$.\\\\
    Precisamente el mismo raciocinio puede aplicarse para demostrar que todo número positivo tiene una raíz n-ésima, cualquiera que sea el número $n$. Si $n$ es impar, se puede decir mas: todo número tiene una raíz n-ésima. Para demostrarlo basta observar que si el número positivo $\alpha$ tiene la raíz n-ésima $x$, es decir, si $x^n=\alpha$, entonces $(-x)^n=-\alpha$ (puesto que $n$ es impar), de modo que $\alpha$ tiene una raíz n-ésima $-\alpha$. Afirmar que, para un $n$ impar, cualquier número $\alpha$ tiene una raíz n-ésima equivale a afirmar que la ecuación
    $$x^n - \alpha = 0$$
    tiene una raíz si $n$ es impar. El resultado expresado de este modo es susceptible de gran generalización.\\\\
\end{teo}

%---------------------- teorema 9 ----------------------
\begin{teo}
    Si $n$ es impar, entonces cualquier ecuación 
    $$x^n + a_{n-1}x^{n-1}+\ldots + a_0 = 0$$
    posee raíz.\\\\
	Demostración.-\; Tendremos que considerar, evidentemente, la función 
	$$f(x) = x^n + a_{n-1}x^{n-1} + \ldots + a_0$$
	habría que demostrar que $f$ es unas veces positiva y otras veces negativa. La idea intuitiva es que para un $|x|$ grande, la función se parece mucho más a $g(x)=x^n$ y puesto que $n$ es impar, ésta función es positiva para $x$ grandes positivos y negativos para $x$ grandes negativos. Un poco de cálculo algebraico es todo lo que hace falta para dar formar a esta idea intuitiva.\\
	Para analizar debidamente la función $f$ conviene escribir 
	$$f(x) = x^n + a_{n-1}x^{n-1}+\ldots + a_0 = x^n\left(1+\dfrac{a_{n-1}}{x} + \ldots + \dfrac{a_0}{x^n}\right)$$
	obsérvese que
	$$\bigg| \dfrac{a_{n-1}}{x} + \dfrac{a_{n-2}}{x^2} + \ldots + \dfrac{a_0}{x^n} \bigg|\leq \dfrac{|a_{n-1}|}{|x|} + \ldots + \dfrac{|a_0|}{|x^n|}$$
	En consecuencia, si elegimos un $x$ que satisfaga
	$$|x|>1,2n|a_{n-1}|,\ldots,2n|a_0|\qquad (*)$$
	entonces $|x^k| > |x|$ y 
	$$\dfrac{|a_{n-k}}{x^k}<\dfrac{a_{n-k}}{|x|}<\dfrac{|a_{n-k}|}{2n|a_{n-k}}=\dfrac{1}{2n}$$
	de modo que 
	$$\bigg| \dfrac{a_{n-1}}{x} + \dfrac{a_{n-2}}{x^2} + \ldots + \dfrac{a_0}{x^n} \bigg|\leq \dfrac{1}{2n}+\ldots + \dfrac{1}{2n}=\dfrac{1}{2}$$
	expresado de otra forma,
	$$-\dfrac{1}{2}\leq \dfrac{a_{n-1}}{x}+\ldots + \dfrac{a_0}{x_n}\leq \dfrac{1}{2},$$
	lo cual implica que 
	$$\dfrac{1}{2}\leq 1 + \dfrac{a_{n-1}}{x}+\ldots + \dfrac{a_0}{x_n}..$$
	Por lo tanto, si elegimos un $x_1>0$ que satisfaga $(*)$, entonces 
	$$\dfrac{x_1^n}{2} \leq x_1^n \left(1+\dfrac{a_{n-1}}{x_1}+\ldots + \dfrac{a_0}{x^n}\right) = f(x_1)$$
	así que $f(x_1)>0$. Por otro lado, si $x_2<0$ satisface $(*)$, entonces $x_2^n < 0$ y 
	$$\dfrac{x_2^n}{2}\geq x_2^n \left(1+\dfrac{a_{n-1}}{x_2}+\ldots + \dfrac{a_0}{x_2^n}\right)=f(x_2),$$
	así $f(x_2)<0$. \\
	Ahora aplicando el teorema 1 para el intervalo $[x_2,x_1]$ llegamos a la conclusión de que existe un $x$ en $[x_2,x_1]$ tal que $f(x)=0$.\\\\

\end{teo}

%---------------------- teorema 10 ----------------------
\begin{teo}
    Si $n$ es par y $f(x)=x^n + a_{n-1} x^{n-1} + \ldots + a_0$, entonces existe un número $y$ tal que $f(y)\leq f(x)$ para todo $x$.\\\\
	Demostración.-\; Lo mismo que en el teorema 9, si 
	$$M = \max(1,2n|a_{n-1}|,\ldots , 2n|a_0|),$$
	entonces para todo $x$ con $|x|\geq M,$ tenemos
	$$\dfrac{1}{2}\leq 1 + \dfrac{a_{n-1}}{x}+\ldots + \dfrac{a_0}{x^n}$$
	Al ser $n$ par, $x^n \geq 0$ para todo $x$, de modo que 
	$$\dfrac{x^n}{2}\leq x^n \left(1+\dfrac{a_{n-1}}{x} + \ldots + \dfrac{a_0}{x^n}\right) = f(x),$$
	siempre que $|x|\geq M$. Consideremos ahora el número $f(0)$. Sea $b>0$ un número tal que $b^n \geq 2f(0)$ y también $b>M$. Entonces si $x\geq b,$ tenemos 
	$$f(x)\geq \dfrac{x^n}{2}\geq \dfrac{b^n}{2}\geq f(0).$$
	Análogamente, si $x\leq -b$, entonces
	$$f(x)\geq \dfrac{}{}\geq \dfrac{}{} = \dfrac{}{}\geq f(0).$$
	Resumiendo ahora el teorema 7 a la función $f$ en el intervalo $[-b,b]$. Se deduce que existe un número $y$ tal que
	$$(1)\qquad \mbox{si}\; -b\leq x\leq b, \; \mbox{entonce}\; f(y)\leq f(x).$$
	En particular, $f(y)\leq f(0).$ De este modo
	$$(2)\qquad \mbox{si}\; x\leq -b \; \mbox{o}\; x\geq b,\; \mbox{entonces}\; f(x)\geq f(0)\geq f(y).$$
	Cambiando (1) y (2) vemos que $f(y)\leq f(x)$ para todo $x$.\\\\
\end{teo}

%---------------------- teorema 11 ----------------------
\begin{teo}
    Consideremos la ecuación
    $$(*)\qquad x^n + a_{n-1}x^{n-1}+ \ldots + a_0 = c,$$
    y supongamos que $n$ es par. Entonces existe un número $m$ tal que $(*)$ posee una solución para $c\geq m$ y no posee ninguna para $c<m.$\\\\
    Demostración.-\; Sea $f(x) = x^n + a_{n-1}x^{n-1}+\ldots + a_0$
    Según el teorema 10, existe un número $y$ tal que $f(y)\leq f(x)$ para todo $x$.\\
    Sea $m=f(y)$. Si $c<m$ entonces la ecuación $(*)$ no tiene, evidentemente, ninguna solución, puesto que el primer miembro tiene un valor $\geq m$. Si $c=m$ entonces $(*)$ tiene $y$ como solución. Finalmente, supongamos $c>m$. Sea $b$ un número tal que $b>y$, $f(b)>c$. Entonces $f(y)=m<c<f(b)$. En consecuencia, según el teorema 4, existe algún número $x$ en $[y,b]$ tal que $f(x)=c$, con lo que $x$ es una solución de $(*).$\\\\
\end{teo}





\section{Problemas}

\begin{enumerate}[\Large\bfseries 1.]

%-------------------- 1.
\item Para cada una de las siguientes funciones, decidir cuáles está acotadas superiormente o inferiormente en el intervalo indicado, y cuáles de ellas alcanzan sus valores máximo y mínimo.

    \begin{enumerate}[\bfseries (i)]

	%----------(i)
	\item $f(x) = x^2$ en $(-1,1)$.\\\\
	    Respuesta.-\; Se encuentra acotada superior como inferiormente. El mínimo es $0$ e no tiene máximo.\\\\

	%----------(ii)
	\item $f(x) = x^3$ en  $(-1,1)$.\\\\
	    Respuesta.-\; Se encuentra acotada superior como inferiormente. No tiene máximo ni mínimo\\\\

	%----------(iii)
	\item $f(x) = x^2$ en $\mathbf{R}$.\\\\
	    Respuesta.-\; No está acotado superior pero si inferiormente. Su mínimo es $0$ y no tiene  máximo.\\\\ 

	%----------(iv)
	\item $f(x)=x^2$ en $[0,\infty)$.\\\\
	    Respuesta.-\; Está acotada inferiormente pero no así superiormente. Su mínimo es $0$ y no tiene máximo.\\\\ 

	%----------(v)
	\item $f(x) = \left\{\begin{array}{ll} x^2, & x\leq a \\ a+2, & x\geq a \end{array}\right.$ en $(-a-1,a+1)$\\\\
	    Respuesta.-\; Es acotado superior e inferiormente. Se entiende que $a>-1$ (de modo que $-a-1<a+1$). Si $-1<a\leq 1/2$, entonces $a<-a-1$, así $f(x)=a+2$ para todo $x$ en $(-a-1,a+1)$, por lo tanto $a+2$ es el máximo y mínimo valor. Si $-1/2<a\leq 0$, entonces $f$ tiene el mínimo valor en $a^2$, y si $a\geq 0$, entonces $f$ tiene un mínimo valor en $0$. Ya que $a+2>(a+1)^2$ solo para $[-1-\sqrt{5}]/2 < a < [1+\sqrt{5}]/2$, cuando $a\geq -1/2$ésta función $f$ tiene un máximo valor solo para $a\leq [1+\sqrt{5}]/2$ (el máximo valor será $a+2$).\\\\

	%----------(vi)
	\item $f(x) = \left\{\begin{array}{ll}x^2, & x<a \\ a+2, & x\geq a \end{array}\right.$ en $[-a-1,a+1].$\\\\
	    Respuesta.-\; Está acotado superior e inferiormente. Como en la parte (v), se asume que $a>-1$. Si $a\leq -1/2$ entonces $f$ tiene el valor mínimo y un máximo $3/2$. Si $a\geq 0$, entonces $f$ tiene un valor mínimo en $0$, y un valor máximo $max(a^2,a+2)$. Si $-1/2<a<0$ , entonces $f$ tiene un máximo valor $3/2$ y no así con un valor mínimo.\\\\

	%----------(vii)
	\item $f(x) = \left\{\begin{array}{ll} 0,& x\; \mbox{irracional}  \\  1/q,& x=p/q \; \mbox{fracción irreducible} \end{array}\right.$ en $[0,1].$\\\\
	    Respuesta.-\; Acotada superior e inferiormente. El mínimo es $0$ y el máximo es $1$.\\\\

	%----------(viii)
	\item $f(x) = \left\{\begin{array}{rcl}1, & x \; \mbox{irracional} \\ 1/q, & x = p/q \; \mbox{fracción irreducible}\end{array}\right.$ en $[0,1]$.\\\\
	    Respuesta.-\; Acotada superior e inferiormente.  El máximo es $1$ y no existe un mínimo.\\\\

	%----------(ix)
	\item $f(x) = \left\{\begin{array}{ll}1, & x \; \mbox{irracional} \\ 0, & x=p/q\; \mbox{fracción irreducible}\end{array}\right.$ en $[0,1]$.\\\\
	    Respuesta.-\; Acotada superior e inferiormente. El mínimo es $-1$ y el máximo es $1$.\\\\

	%----------(x)
	\item $f(x) = \left\{ \begin{array}{ll} x, & x\; \mbox{racional} \\ 0, & x\; \mbox{irracional}\end{array} \right.$ en $[0,a]$.\\\\
		Respuesta.-\; Acotada superior e inferiormente. El mínimo es $0$ y el máximo es $a$.\\\\

	%----------(xi)
	\item $f(x) = \sen^2(\cos x + \sqrt{1-a^2})$ en $[0,a^3]$.\\\\
	    Respuesta.-\; Ya que es continua $f$ tiene máximo como también mínimo.\\\\

	%----------(xii)
	\item $f(x)=[x]$ en $[0,a]$.\\\\
	    Respuesta.-\; Acotada superior e inferiormente. El mínimo es $0$ y el máximo es $a$.\\\\

    \end{enumerate}

%-------------------- 2.
\item Para cada una de las siguientes funcione polinómicas $f$, hallar un entero $n$ tal que $f(x)=0$ para algún $x$ entre $n$ y $n+1$.\\
    \begin{enumerate}[\bfseries (i)]

	%----------(i)
	\item $f(x)=x^3-x+3$.\\\\
	    Respuesta.-\; $n=-2$, ya que $f(-2) = (-2)^3+2+3 = -3 < 0 <  3 =  (-1)^3 - (-1) + 3$\\\\

	%----------(ii)
	\item $f(x) = x^5+5x^4 + 2x + 1$.\\\\
	    Respuesta.-\; $n=-5$ ya que $f(-5) = -11<0<f(-4)$.\\\\

	%----------(iii)
	\item $f(x) = x^5 + x + 1$.\\\\
	    Respuesta.-\; $n=-1$ ya que, $f(-1) = -1<0f(0)$.\\\\ 

	%----------(iv)
	\item $4x^2-4x+1$\\\\
	    Respuesta.-\; No existe un entero $n$ tal que $f(x)=0$.\\\\

    \end{enumerate}

%-------------------- 3.
\item Demostrar que existe algún número $x$ tal que

    \begin{enumerate}[\bfseries (i)]

	%----------(i)
	\item $x^{179} + \dfrac{163}{1+x^2+\sen^2 x} = 119$.\\\\
	    Respuesta.-\; Si $x^{179}$  y  $\dfrac{163}{1+x^2+\sen^2 x}$, son continuas en $\mathbb{R}$ entonces $f(x) = x^{179}+\dfrac{163}{1+x^2+\sen^2 x}$ es continua en $\mathbb{R}$ y $f(1)>0$, mientras que $f(-2)<0$, de modo que $f(x)=0$ para algún $x$ en $(-2,1)$.\\\\

	%----------(ii)
	\item $\sen x =x-1.$\\\\
	    Respuesta.-\; Sea $f(x) = \sen x - x + 1$ entonces $f$ es continua en $\mathbb{R}$ y $f(0)>0$, mientras que $f(2)<0$, así por el teoremas 4 se tiene que $f(x)=c$ para algún $x$ en $(0,2)$.\\\\

    \end{enumerate}

%-------------------- 4.
\item Este problema es una continuación del problema 3-7
    \begin{enumerate}[\bfseries (a)]

	%----------(a)
	\item Si $n-k$ es par, y $\geq 0$, hallar una función polinómica de grado $n$ que tenga exactamente $k$ raíces.\\\\
	    Respuesta.-\; Sea $l = (n-k)/2$ de donde 
	    $$f(x) = (x^{2(n-k)/2} + 1)(x-1)(x-2)\cdot \cdot \cdot (x-k).$$\\\\
	
	%----------(b)
	\item Una raíz $a$ de una función polinómica $f$ se dice que tiene multiplicidad $m$ si $f(x)=(x-a)^m g(x),$ donde $g$ es una función polinómica que no tiene la raíz $a$. Sea $f$ una función polinómica de grado $n$. Supóngase que $f$ tiene $k$ raíces, contando multiplicidades, es decir supóngase que $k$ es la suma de las multiplicidades de todas las raíces. Demostrar que $n-k$ es par.\\\\
	    Demostración.-\; Por la condición dada, $f$ es una función polinómica real de grado $n$ tal que $f$ tiene exactamente $k$ raíces en $\mathbb{R}$ contando multiplicidades. Probaremos que $n-k$ es par. Para ello consideraremos los siguientes casos.\\\\
	    \textbf{Caso 1}.- Si $n=k$ es trivial decir que $n-k=0$ de donde se sabe que es par.\\\\
	    \textbf{Caso 2}.- Si $n>k$, sea $x_1,x_2,\ldots , _m$ raíces reales de $f$ con multiplicidades $k_1,k_2,\ldots, k_m$ respectivamente y por lo tanto,
	    $$k_1 + k_2 + \ldots + k_m = k.$$
	    Entonces $f$ puede ser escrito como,
	    $$f(x) = (x-x_1)^{k_1}(x-x_2)^{k_2}\cdots (x-x_m)^{k_m}p_1(x)p_2(x)\cdots p_i(x)$$
	    donde $p_i(x)$ son polinomios irreducibles en $\mathbb{R}$ tal que el grado de $p_i$  suma $n-k$. Ahora recordemos que todo polinomio irreducible en $\mathbb{R}$ debe tener de grado un entero par. Esto se debe a que cada polinomio de orden impar tiene al menos una raíz real, esto por el teorema 9, por lo tanto $p_i(x)$ no puede ser irreducible en $\mathbb{R}$. Ahora observe que sin pérdida de generalidad hemos asumido que hay $l$ polinomios irreducibles tales que la suma de sus grados $n-k$. Dado que cada uno de los $l$ polinomios tienen grado par, entonces la suma de sus grados debe ser un entero par. Se sigue que $n-k$ es un entero par.\\\\ 

    \end{enumerate}

%-------------------- 5.
\item Supóngase que $f$ es continua en $[a,b]$ y que $f(x)$ es siempre racional. ¿Qué puede decirse acerca de $f$?.\\\\
    Respuesta.-\; $f$ es constante, ya que si $f$ tomara dos valores distintos, entonces $f$ tomaría todos los valores intermedios, incluyendo valores irracionales, es decir, si no fuera constante, entonces existe dos números racionales $r_1$ y $r_2$ tal que para algún $c,d$ se tiene $a\leq c<d\leq$, $f(c)=r_1$ y $f(d) = r_2$. Por el teorema 7.4 en el intervalo $[c,d]$, $f$ toma todos los valores entre $r_1$ y $r_2$, donde se concluye que existe algún número irracional, contradiciendo el hecho de que $f$ solo toma valores racionales.\\\\

%-------------------- 6.
\item Supóngase que $f$ es una función continua en $[-1,1]$ tal que $x^2+f^2(x) = 1$ para todo $x$. (Esto significa que $(x,f(x))$ siempre está sobre el circulo unidad.) Demostrar que o bien es $f(x)=\sqrt{1-x^2}$ para todo $x$, o bien $f(x)=-\sqrt{1-x^2}$ para todo $x$.\\\\
    Demostración.-\; De lo contrario, $f$ toma valores tanto positivos como negativos, por lo que $f$ tendría el valor $0$ en  $(-1, 1)$, lo cual es imposible, ya que $\sqrt{1-x^2} \neq 0$ para $x$ en $(-1,1)$.\\\\ 
 

%-------------------- 7.
\item ¿Cuántas funciones continuas $f$ existen satisfaciendo $f^2(x)=x^2$ para todo $x$?.\\\\
    Respuesta.-\; Existen 4 funciones continuas que satisfacen la condición dada, es decir,
    $$\begin{array}{rcl}
	f(x)&=&x\\
	f(x)&=&-x\\
	f(x)&=&|x|\\
	f(x)&=&-|x|\\\\
    \end{array}$$

%-------------------- 8.
\item Supóngase que $f$ y $g$ son continuas, que $f^2=g^2$, y que $f(x)\neq 0$ para todo $x$. Demostrar que o bien $f(x)=g(x)$ para todo $x$, o bien $f(x)=-g(x)$ para todo $x$.\\\\
    Demostración.-\; Si no fuera así, entonces $f(x)=g(x)$ para algún $x$ y $f(y)=-g(y)$ para algún $y$. Pero ya que $f(x)\neq 0 \; \forall \;x,$ entonces será o bien siempre positiva o bien siempre negativa. Así pues, $g(x)$ y $g(y)$ tendría distinto signo. Esto implicaría que $g(z)=0$ para algún $z$, lo cual es imposible, ya que $0\neq f(z) = \pm g(z).$\\\\

%-------------------- 9.
\item 
    \begin{enumerate}[\bfseries (a)]

	%----------(a)
	\item Supóngase que $f$ es continuo, que $f(x)=0$ solo para $x=a$, y que $f(x)>0$ tanto para algún $x>a$, así como para algún $x<a$. ¿Que puede decirse acerca de $f(x)$ para todo $x\neq a$?.\\\\ 
	    Respuesta.-\; Por hipótesis, existe algún $x_1\in (a,\infty)$ tal que $f(x_1)>0$. Ahora si existe algún $y_i\in (a,\infty)$ con $f(y_1)<0$, entonces debe existir $z_1\in (a,\infty)$ entre $x_1$ y $y_1$ tal que $f(z_1)=0$. Pero esto contradice  que $f$ es cero solo en $x=a$. Por lo tanto, no existe algún $y_1\in (a,\infty)$ con $f(y_1)<0$.\\
	    Esto es, $f(x)>0$ para todo  $x\in(a,\infty).$ Similarmente, $f(x)>0$ para todo  $x\in(-\infty,a).$ Por lo tanto podemos decir que $f(x)>0$ para todo $x\neq a$.\\\\

	%----------(b)
	\item Supongamos ahora que $f$ es continuo y que $f(x)=0$ solo para $x=a$, pero supongamos, en cambio, que $f(x)>0$ para algún $x>a$ y $f(x)<0$ para algún $x<a$. Ahora que puede decir de $f(x)$ para $x\neq a$?.\\\\
	    Respuesta.-\; Por hipótesis, existe algún $x_1\in (a,\infty)$ tal que $f(x_1)>0.$. Ahora si existe $y_1\in (a,\infty)$ con $f(y_1)<0$, entonces existe algún $z_1\in (a,\infty)$ entre $x_1$ y $y_1$ tal que $f(z_1)=0.$ Esto contradice que $f$ es cero sólo en $x=a$. Por lo tanto, no existe algún $y_1\in (a,\infty)$ con $f(y_1)<0.$\\
	    Esto es, $f(x)>0$ para todo $x\in (a,\infty).$ Luego por similar argumento, $f(x)<0$ para todo $x\in (-\infty,a)$. Así, $f(x)>0$ para todo $x>a$ y $f(x)<0$ para todo $x<a$.\\\\

	%----------(c)
	\item Discutir el signo de $x^3+x^2+xy^2+y^3$ cuando $x$ e $y$ no son ambos $0$.\\\\
	    Respuesta.-\; Para $y\neq 0$, sea $f(x)=x^3+x^2y+xy^2+y^3$. Luego
	    $$f(x)=\dfrac{x^4-y^4}{x-y}$$

    \end{enumerate}

%-------------------- 10.
\item Supóngase $f$ y $g$ son continuas en $[a,b]$ y que $f(a)<g(a)$, pero $f(b)>g(b)$. Demostrar que $f(x)=g(x)$ para algún $x$ en $[a,b]$.\\\\
    Demostración.-\; Sea $$h=f-g$$
    entonces por el teorema 1 se tiene $$h(x)=0$$
    por lo que $$f(x)=g(x)\; \mbox{para algún }x \in [a,b].$$\\

%-------------------- 11.
\item Supóngase que $f$ es una función continua en $[0,1]$ y que $f(x)$ es en $[0,1]$ para cada $x$. Demostrar que $f(x)=x$ para algún número $x$.\\\\
    Demostración.-\; Para $f(0)=0$ o $f(1)=1$ entonces se puede elegir $x=0$ o $x=1$. Ya que $x$ es continuo entonces $$g(x)=x-f(x)$$ también es continuo. Luego, por el teorema 1 se tiene,
    $$f(x) - x = 0 \; \Longrightarrow \; x=f(x) \; \mbox{para algún}\; x \in [0,1].$$\\

%-------------------- 12.
\item  
    \begin{enumerate}[\bfseries (a)]

	%----------(a)
	\item El problema 11 muestra que $f$ intersecta la diagonal del cuadrado. Demostrar que $f$ debe cortar a la otra diagonal.\\\\
	    Demostración.-\; Vemos que la linea representa una función $f$ en $[0,1]$, dado por,
	    $$f(x)=x.$$
	    es continuo sobre $[0,1]$. Ahora supongamos una nueva función $g$ en $[0,1]$ tal que
	    $$g(x)=x-f(x)$$
	    de donde,
	    $$\begin{array}{l}
		g(0)=0-f(0)=-f(0)\leq 0\\\\
		g(1)=1-f(1)\\
	    \end{array}$$

	    De las dos funciones anteriores se tiene,
	    $$\begin{array}{l}
		f(0)<g(0)\\\\
		f(1)>g(1)\\
	    \end{array}$$

	    Por último definamos una nueva función continua $h$ de forma que,
	    $$h=f-g$$
	    entonces,
	    $$\begin{array}{l}
		h(1)=f(1)-g(1)>0\\\\
		h(0)=f(0)-g(0)<0\\
	    \end{array}$$
	    
	    Luego, existe algún punto $c$ en $[0,1]$ por lo que ambas curvas es,
	    $$h(c)=0$$
	    y 
	    $$\begin{array}{rcl}
		f(c)-g(c)&=&0\\
		f(c)&=&g(c)\\
	    \end{array}$$

	    Por lo tanto, hay algún $c$ en $[0,1]$ donde $f$ intersecta a la otra linea diagonal.\\\\

	%----------(b)
	\item Demostrar el siguiente hecho más general: Si $g$ es continuo en $[0,1]$ y $g(0)=0$, $g(1)=1$ o $g(0)=1$, $g(1)=0$, entonces $f(x)=g(x)$ para algún $x$.\\\\
	    Demostración.-\;  Sea $f$ en $[0,1]$ entonces
	    $$\begin{array}{l}
		f(0)=0\\\\
		f(1)=1\\
	    \end{array}$$

	    La otra linea punteada representa una función $g$ en $[0,1]$ dada por,
	    $$\begin{array}{l}
		g(0)=0\\\\
		g(1)=1\\
	    \end{array}$$

	    De donde 

	    $$\begin{array}{l}
		f(0)<g(0)\\\\
		f(1)>g(1)\\
	    \end{array}$$

	    Ahora definimos una nueva función continua $h$ de forma que,
	    $$h=f-g$$
	    entonces,
	    $$\begin{array}{l}
		h(1)=f(1)-g(1)>0\\\\
		h(0)=f(0)-g(0)<0\\
	    \end{array}$$
	    
	    Luego, existe algún punto $c$ en $[0,1]$ por lo que ambas curvas es,
	    $$h(c)=0$$
	    y 
	    $$\begin{array}{rcl}
		f(c)-g(c)&=&0\\
		f(c)&=&g(c)\\
	    \end{array}$$

	    Por lo tanto, un continuo $f(g)$ y $g$, existe $f(x)=g(x)$ para algún $x$.\\\\

    \end{enumerate}

%-------------------- 13.
\item 
    \begin{enumerate}[\bfseries (a)]

	%----------(a)
	\item Sea $f(x)=\sen 1/X$ para $x\neq 0$ y sea $f(0)=0,$ ¿Es $f$ continuo en $[-1,1]$?. Demostrar que $f$ satisface la conclusión del teorema de valor intermedio en $[-1,1]$; en otras palabras, si $f$ toma dos valores comprendidos en $[-1,1]$, toma también todos los valores intermedios.\\\\
	    Demostración.-\; Sea la secuencia ${x_n}_n$ en $[-1,1]$ definida por,
	    $$x_ = \dfrac{2}{\pi(4n-3)},\quad n\geq 1$$
	    luego notamos que 
	    $$\lim_{x\to \infty} x_n = 0$$
	    Ahora aplicando la función $f$ se tiene,
	    $$\lim_{n\to \infty}f(x_n) = \lim_{n\to \infty}\sin\left(\dfrac{1}{\frac{2}{\pi(4n-3)}}\right) = \lim_{n\to \infty}\sin\left(\dfrac{\pi(4n-3)}{2}\right) = 1, \quad \forall n\geq 1.$$
	    por lo tanto la función $f$, como tal, no es continuo en $[-1,1]$.\\\\
	    Ahora demostraremos que $f$ satisface el teorema de valor intermedio en $[-1,1]$. Ya que $f(0)=0$ según la hipótesis, podemos decir que $f$ es continuo en $[0,1]$. Vamos a considerar los siguientes casos:\\

	    \begin{enumerate}[\bfseries C1]

		%----------C1
		\item Si $a<b$ son dos puntos de $[-1,1]$ con $a,b>0$ o $a,b<0$, entonces $f$ toma cada valor entre $f(a)$ y $f(b)$ en el intervalo $[a,b]$ ya que $f$ es continuo en $[a,b]$.\\

		%----------C2
		\item Si $a<0<b$, entonces $f$ toma todos los valores entre $-1$ y $1$ en $[a,b]$. \\ 

	    \end{enumerate}

	    Así $f$ tomas todos los valores entre $f(a)$ y $f(b)$. Lo mismo ocurre para $a=0$ o $b=0$.\\\\


	%----------(b)
	\item Supóngase que $f$ satisface la conclusión del teorema del valor intermedio y que $f$ toma cada valor solo una vez. Demostrar que $f$ es continua.\\\\
	Demostración.-\; Si $f$ no fuese continua en $a$, entonces por el problema 6-9(b) para algún $\epsilon>0$ existirían $x$ tan cerca como se quiera de $a$ con $f(x)>f(a)+\epsilon$
	$f(x)<f(a)-\epsilon$. Supongamos que ocurre lo primero. Podemos incluso suponer que existen $x$ tan cerca como se quiera de $a$ y $>a$, o bien tan cerca como se quiera de $a$ y $<a$. Supongamos también aquí lo primero. Tomemos un $x>a$ con $f(x)>f(a)+\epsilon$. Según el teorema de los valores intermedios, existe un $x^{'}$ entre $a$ y $x$ con $f(x^{'})<f(a)+\epsilon$. Pero existe también $y$ entre $a$ y $x^{'}$ con $f(y)<f(a)+\epsilon$. Pero existe también $y$ entre $a$ y $x^{'}$ con $f(y)>f(a)+\epsilon$. Según el teorema de los valores intermedios, $f$ forma el valor $f(a)+\epsilon$ entre $x$ y $x^{'}$ y también entre $x^{'}$ e $y$, contrariamente a la hipótesis.\\\\

	%----------(c)
	\item Generalizar para el caso donde $f$ toma cada valor solo un número finito de veces.\\\\
	    Respuesta.-\; Lo mismo que en $(b)$ elíjase $x_1>a$ con $f(x_1)>f(a)+\epsilon$. Después elijase $x_1^{'}$ entre $a$ y $x_1$ con $f(x_1^{'})<f(a)+\epsilon.$ Luego elíjase $x_2$ entre $a$ y $x_1^{'}$ con $f(x_2)>f(a)+\epsilon$ y $x_2^{'}$ entre $a$ y $x_2$ con $f(x_2^{'})<f(a)+\epsilon$, etc. Entonces $f$ toma el valor $f(a)+\epsilon$ en cada uno de los intervalos $[x_n^{'},x_n]$ en contradicción con la hipótesis.\\\\

    \end{enumerate}

% -------------------- 14.
\item Si $f$ es una función continua en $[0,1]$, sea $\|f\|$ el máximo valor de $|f|$ en $[0,1]$.

    \begin{enumerate}[\bfseries (a)]

	%----------(a)
	\item Demostrar que para algún número $c$ tenemos $\|cf\| = |c|\cdot \|f\|$.\\\\
	    Demostración.-\; Ya que $|cf|=|c|\cdot |f(x)|$ para todo $x$ de $[0,1].$ entonces podemos elegir un $x_0$ tal que $|f|(x_0)=\|f\|$, y por lo tanto $\|cf\| = |c|\cdot \|f\|$.\\\\

	%----------(b)
	\item Demostrar que $\|f+g\|\leq \|f\|+\|g\|.$ Dar un ejemplo donde $\|f+g\|\neq \|f\|+\|g\|.$\\\\
	    Demostración.-\; Para las todos funciones dadas, tenemos
	    $$\begin{array}{rcl}
		|f+g|(x)&=&|f(x)+g(x)|\\
		|f+g|(x)&\leq&|f(x)|+|g(x)|\\
		|f+g|(x)&\leq&|f|(x)+|g|(x)\\
	    \end{array}$$
	    También sabemos que si $f$ o $g$ tienen el máximo valor en $x_0$ entonces,
	    $$\begin{array}{rcl}
		|f|(x_0)&=&\|f\|\\
		|g|(x_0)&=&\|g\|\\
	    \end{array}$$

	    Luego si la función $|f+g|$ tiene el máximo valor en $x_0$ entonces,
	    $$\begin{array}{rcl}
		\|f+g\|&=&|f+g|(x_0)\\
		\|f+g\|&\leq&|f|(x_0) + |g|(x_0)\\
		       \|f+g\|&\leq&\|f\| + \|g\|\\
	    \end{array}$$
	    Por último, sea $f(x)=x+3$ y $g(x)=x-4$ entonces se cumple que $\|f+g\|\neq \|f\|+\|g\|$.\\\\

	%----------(c)
	\item Demostrar que $\|h-f\|\leq \|h-g\|+\|g-f\|$.\\\\
	    Demostración.-\; Para las todos funciones dadas, tenemos
	    $$\begin{array}{rcl}
		|h-f|(x)&=&|(h-g)-(g-f)|(x)\\
		|h-f|(x)&\leq&|(h-g)(x)|+|(g-f)(x)|\\
		|h-f|(x)&\leq&|h-g|(x)+|g-f|(x)\\
	    \end{array}$$
	    También sabemos que si $f$ o $g$ tienen el máximo valor en $x_0$ entonces,
	    $$\begin{array}{rcl}
		|f|(x_0)&=&\|f\|\\
		|g|(x_0)&=&\|g\|\\
		|h|(x_0)&=&\|h\|\\
	    \end{array}$$

	    Luego si la función $|f+g|$ tiene el máximo valor en $x_0$ entonces,
	    $$\begin{array}{rcl}
		\|h-f\|&=&|h-f|(x_0)\\
		\|h-f\|&\leq&|h-g|(x_0) + |g-f|(x_0)\\
		       \|h-f\|&\leq&\|h-g\| + \|g-f\|\\\\
	    \end{array}$$

    \end{enumerate}
   
% -------------------- 15.
\item Supóngase que $\phi$ es continua y $\lim\limits_{x\to\infty} \phi(x)/x^n = 0 = \lim\limits_{x\to -\infty} \phi (x)/x^n$.

    \begin{enumerate}[\bfseries (a)]

	%----------(a)
	\item Demostrar que si $n$ es impar, entonces existe un número $x$ tal que $x^n+\phi(x)=0$.\\\\
	    Demostración.-\; Sea $b>0$ tal que $$\bigg|\dfrac{\phi(b)}{b^n}\bigg|<\dfrac{1}{2}$$
	    entonces,
	    $$b^n+\phi(b)=b^n\left(1+\dfrac{\phi(b)}{b^n}\right)>\dfrac{1}{2}>0$$
	    De la misma manera, sea $a<0$ tal que
	    $$\bigg|\dfrac{\phi(a)}{a^n}\bigg|<\dfrac{1}{2}$$
	    entonces, ya que $n$ es impar,
	    $$a^n + \phi(a)=a^n\left(1+\dfrac{\phi(a)}{a^n}\right)<\dfrac{a^n}{2}<0$$
	    Por lo tanto, existe un $x$ tal que 
	    $$x^n + \phi(x)=0.$$\\

	%----------(b)
	\item Demostrar que si $n$ es par, entonces existe un número $y$ tal que $y^n+\phi(y)\leq x^n + \phi(x)$ para todo $x$.\\\\
	    Demostración.-\; Sea $b>0$ tal que 
	    $$b^n>2\phi(0)$$
	    Y $|x|>b$ 
	    $$\bigg|\dfrac{\phi(x)}{x^n}<\dfrac{1}{2}\bigg|$$
	    de donde tenemos,
	    $$\begin{array}{rcl}
		x^n+\phi(x)&>&x^n\left(1+\dfrac{\phi(x)}{x^n}\right)\\\\
		x^n+\phi(x)&>&\dfrac{x^n}{2}\\\\
		x^n+\phi(x)&>&\dfrac{b^n}{2}\\\\
		x^n+\phi(x)&>&\phi(0).\\\\
	    \end{array}$$

	    Así, el mínimo de $x^n+\phi(x)$ para $x$ in $[-b,b]$ es el mínimo del intevalo.
	    Y por lo tanto, existe un número $y$, para todo $x$, tal que,
	    $$y^n+\phi(y)\leq x^n +\phi(x).$$\\

    \end{enumerate}

% -------------------- 16.
\item 
    \begin{enumerate}[\bfseries (a)]

	%----------(a)
	\item Supóngase que $f$ es continua en $(a,b)$ y $\lim\limits_{x\to a^+} f(x)=\lim\limits_{x\to b^-} f(x)=\infty$. Demostrar que $f$ tiene un mínimo en todo el intervalo $(a,b)$.\\\\
	    Demostración.-\;

    \end{enumerate}

\end{enumerate}


    %---------- Cotas superiores mínimas
	%\chapter{Cotas superiores mínimas}

%------------------- Definición 8.1
\begin{tcolorbox}
    \begin{def.}
	Un conjunto $A$ de números reales está acotado superiormente si existe un número $x$ tal que 
	\begin{center}
	    $x\geq a$ para todo $a$ de $A$.
	\end{center}
	este número $x$ se denomina una cota superior de $A$.\\\\
    $A$ está acotado superiormente si y sólo si existe un número $x$ que es una cota superior de $A$.(y en este caso existirán muchas cotas superiores de $A$);\\\\
    \end{def.}
\end{tcolorbox}

%------------------- Definición 8.2
\begin{tcolorbox}
    \begin{def.}
	Un número $x$ es una cota superior mínima de $A$ si
	\begin{enumerate}[\bfseries (1)]
	    \item $x$ es una cota superior de $A$,
	    \item si $y$ es una cota superior de $A$, entonces $x\leq y$.
	\end{enumerate}
	El término \textbf{supremo} de $A$ es sinónimo al de cota superior mínima y tiene una ventaja: se puede abreviar mediante un símbolo muy adecuado
	$$\sup A$$
    \end{def.}
\end{tcolorbox}
\vspace{0.2cm}

    Si $x$ e $y$ son ambos cotas superiores mínimas de $A$, entonces $x=y$.\\\\
	Demostración.-\; En efecto, en este caso
	\begin{center}
	\begin{tabular}{ll}
	    $x\leq y,$ & ya que $y$ es una cota superior, y $x$ es una cota superior mínima,\\
	    $y\leq x$ & ya que $x$ es un cota superior, e $y$ es una cota superior mínima.
	\end{tabular}
	\end{center}
	por lo tanto, $x=y$. \\\\

%------------------- Definición 8.3
\begin{tcolorbox}
    \begin{def.}
	Un conjunto $A$ de números reales está acotado inferiormente si existe un número $x$ tal que
	\begin{center}
	    $x\leq a$ para todo $a$ de $A$.
	\end{center}
	Dicho número $x$ se denomina una cota inferior de $A$.
    \end{def.}
\end{tcolorbox}

\begin{tcolorbox}
    \begin{def.}
	Un número $x$ es la cota inferior máxima de $A$ si
	\begin{enumerate}[\bfseries (1)]
	    \item $x$ es una cota inferior de $A$, y
	    \item si $y$ es una cota inferior de $A$, entonces $x\geq y$.
	\end{enumerate}
	La cota inferior máxima de $A$ se denomina también el \textbf{ínfimo} de $A$, abreviadamente
	$$\inf A$$
    \end{def.}
\end{tcolorbox}

%------------------- P13
\begin{tcolorbox}
\begin{prop}[Propiedad de la cota superior mínima]
    Si $A$ es un conjunto de números reales, $A\neq \emptyset$, y $A$ está acotado superiormente, entonces $A$ posee una cota superior mínima. 
\end{prop}
\end{tcolorbox}

El enorme significado de P13 se hará patente sólo de manera gradual, aunque ya estamos en condiciones de comprobar su importancia dando las demostraciones que omitimos en el Capítulo 7.\\


\setcounter{chapter}{7}
\setcounter{teo}{0}
\begin{teo}
    Si $f$ es continua en $[a,b]$ y $f(a)<0<f(b)$, entonces existe algún número $x$ de $[a,b]$ tal que $f(x)=0.$\\\\
	Demostración.-\; La demostración es tan sólo una versión rigurosa del método esbozado al final del capítulo 7: localizaremos el menor número $x$ de $[a,b]$ tal que $f(x)=0$.\\
	Definamos el conjunto $A$, de la manera siguiente:

	$$A=\lbrace x:a\leq x \leq b, \; \mbox{y}\; f \; \mbox{negativa en en el intervalo}\; [a,x]\rbrace.$$

	Obviamente $A\neq \emptyset$ ya que $a$ pertenece a $A$; de hecho, existe un $\delta >0$  tal que $A$ contiene a todos los puntos $x$ que satisfacen $a\leq x < a+\delta$ ya que $f$ es continua en $[a,b]$ y $f(a)<0$. Análogamente, $b$ es una cota superior de $A$ y, de hecho, existe un $\delta>0$ tal que todos los puntos $x$ que satisfacen $b-\delta<x\leq b$ son cotas superiores de $A$; esto también se deduce del Problema 6-16, ya que $f(b)>0$.\\
	A partir de estas observaciones se deduce que $A$ posee cota superior mínima $\alpha$ y que $a<\alpha<b$. Ahora demostraremos que $f(\alpha)=0$, excluyendo las posibilidades $f(\alpha)<0$ y $f(\alpha)>0$.\\
	Supongamos primero que $f(\alpha)<0$. Según el teorema 6-3, existe un $\delta>0$ tal que $f(x)<0$ si $\alpha- \delta<x<\alpha+\delta$. Ha de existir un número $x_0$ de $A$ que satisface $\alpha-\delta<x_0<\alpha$ (ya que sino $\alpha$ no sería la mínima cota superior de $A$). Esto significa que $f$ es negativa en todo el intervalo $[a,x_0]$. Pero si $x_1$ es un número situado entre $\alpha$ y $\alpha+\delta$, entonces $f$ también es negativa en todo el intervalo a $A$. Pero esto contradice el hecho de que $\alpha$ sea una cota superior de $A$; concluimos pues que la suposición que hemos hecho anteriormente, de que $f(\alpha)<0$ sebe ser falsa.\\
	Supongamos ahora que $f(\alpha)>0$. Entonces existe un número $\delta>0$ tal que $f(x)>0$ si $\alpha-\delta<x<\alpha+\delta$. Una vez más, sabemos que existe un $x_0$ de $A$ que satisface $\alpha-\delta<x_0<\alpha$; pero esto significa que $f$ es negativa en $[a,x_0]$, lo cual es imposible ya que $f(x_0)>0$. Así pues, la suposición de que $f(\alpha)>0$ conduce a una contradicción, quedando sólo la posibilidad de que $f(\alpha)=0$.\\
\end{teo}

Las demostraciones de los teoremas 2 y 3 del capítulo 7 requieren un sencillo resultado preliminar, que va a desempeñar una función muy similar a la del teorema 6-3 en la demostración anterior.\\\\

\setcounter{chapter}{8}
\setcounter{teo}{0}

\begin{teo}
    Si $f$ es continua en $a$, existe un número $\delta>0$ tal que $f$ está acotada superiormente en el intervalo $(a-\delta,a+\delta)$.\\\\
    Demostración.-\; Como el $\lim\limits_{x\to a} f(x) =f(a)$, para cada $\epsilon>0$, existe un $\delta>0$ tal que, para todo $x$,
    \begin{center}
	si $|x-a|<\delta$, entonces $|f(x)-f(a)|<\epsilon,$
    \end{center}
    Tan sólo es necesario aplicar esta propiedad a algún $\epsilon$ en particular, por ejemplo $\epsilon=1$. Deducimos pues que existe un $\delta>0$ tal que, para todo $x$,

    \begin{center}
	si $|x-a|<\delta$, entonces $|f(x)-f(a)|<1,$
    \end{center}

    y, en particular, si $|x-a|<\delta$ entonces $f(x)-f(a)<1$. Esto completa la demostración: en el intervalo $(a-\delta,a+\delta)$ la función $f$ está acotada superiormente por $f(a)+1.$\\\\
\end{teo}

Por supuesto, ahora podríamos demostrar también que $f$ está acotada inferiormente en algún intervalo $(a-\delta,a+\delta)$, concluyendo, por tanto, que $f$ está acotada en algún intervalo abierto que contiene a $a$.\\
En este sentido, cabe destacar en particular la observación de que si $\lim\limits_{x\to a^{+}},$ entonces existe un $\delta>0$ tal que $f$ está acotada en el conjunto $\lbrace x:a\leq x < a+\delta\rbrace$, pudiendo hacerse una observación análoga si $\lim\limits_{x\to b^-}=f(b)$.\\\\

\setcounter{chapter}{7}
\setcounter{teo}{1}
\begin{teo}
    Si $f$ es continua en $[a,b]$, entonces $f$ está acotada superiormente en $[a,b]$.\\\\
	Demostración.-\; Sea,
	$$A=\lbrace x:a\leq x \leq b \;\mbox{y}\; f \; \mbox{está acotada superiormente en}\; [a,x]\rbrace$$
	Obviamente $A\neq \emptyset$ (ya que $a$ pertenece a $A$), y está acotada superiormente por $b$, de manera que $A$ posee una cota superior mínima $\alpha$. Observemos que estamos aplicando el término acotado superiormente tanto al conjunto $A$, localizado en el eje horizontal, como a la función $f$, es decir, a conjuntos del tipo $\lbrace f(y):a\leq y \leq x\rbrace$, localizados en el eje vertical.\\
	La primera etapa de la demostración consiste en probar que $\alpha=b$. Supongamos, por el contrario, que $a<b$. Según el teorema 1 existe un $\delta >0$ tal que $f$ está acotada en $(a-\delta,a+\delta)$. Como $\alpha$ es la cota superior mínima de $A$ existe algún $x_0$ de $A$ que satisface $\alpha-\delta<x_0<\alpha$. Esto significa que $f$ está acotada en $[a,x_0]$. Pero si $x_1$ es cualquier número tal que $\alpha<x_1<\alpha+\delta$, entonces $f$ también está acotada en $[x_0,x_1]$. Por lo tanto $f$ está acotada en $[a,x_1]$, de manera que $x_1$ pertenece a $A$, lo que contradice el hecho de que $\alpha$ sea una cota superior a $A$. Esta contradicción demuestra que $\alpha=b$. Debemos mencionar un detalle: en la demostración hemos puesto implícitamente que $a<\alpha$ de manera que $f$ está definida en algún intervalo $(\alpha-\delta,\alpha+\delta)$; la posibilidad $a=\alpha$ puede excluirse de manera similar, utilizando el hecho de que existe un $\delta>0$ tal que $f$ está acotada en $\lbrace x:a\leq x < a + \delta\rbrace$.\\
	La demostración todavía no es completa; únicamente sabemos que $f$ está acotada en $[a,x]$ para todo $x<b$, no necesariamente que $f$ está acotada en $[a,b]$. Sin embargo, sólo es necesario añadir una pequeña observación.\\
	Existe un $\delta>0$ tal que $f$ está acotada en $\lbrace x:b-\delta < x \leq b \rbrace$. Existe también un $x_0$ de $A$ tal que $b-\delta<x_0<b$. De manera que $f$ está acotada en $[a,x_0]$ y también en $[x_0,b],$ por tanto $f$ está acotada en $[a,b]$.\\\\
\end{teo}

%-------------------- teorema 7.3
\begin{teo}
    Si $f$ es continua en $[a,b]$, entonces existe un número $y$ de $[a,b]$ tal que $f(y)\geq f(x)$ para todo $x$ de $[a,b]$.\\\\
	Demostración.-\; Sabemos que $f$ está acotada en $[a,b]$, lo que significa que le conjunto
	$$\lbrace f(x):x \mbox{ pertenezca a }[a,b]\rbrace$$
	está acotado. Además, dicho conjunto es, obviamente, distinto del $\emptyset$, de manera que admite una cota superior mínima $\alpha$. Como $\alpha\geq f(x)$ para $x$ de $[a,b]$, basta demostrar que $\alpha=f(y)$ para algún $y$ de $[a,b]$.\\
	Supongamos, por el contrario, que $\alpha\neq f(y)$ para todo $y$ de $[a,b]$. Entonces la función $g$ definida mediante
	$$g(x)=\dfrac{1}{\alpha-f(x)},\quad x\in [a,b]$$
	es continua en $[a,b]$, ya que el denominador de la expresión del lado derecho de la igualdad nunca vale $0$. Por otra parte, $\alpha$ es la mínima cota superior de $\lbrace f(x):x \mbox{ pertenezca a }[a,b] \rbrace$, esto significa que 
	\begin{center}
	    para cada $\epsilon > 0$ existe un $x$ de $[a,b]$ con $\alpha - f(x)<\epsilon$.
	\end{center}
	Esto, a su vez, significa que 
	\begin{center}
	    para cada $\epsilon > 0$ existe un $x$ de $[a,b]$ con $g(x)>1/\epsilon$.
	\end{center}
	Pero esto quiere decir que $g$ no está acotada en $[a,b]$, lo que contradice el resultado del teorema anterior.\\\\
\end{teo}

\setcounter{chapter}{8}
\setcounter{teo}{1}
%-------------------- teorema 8.2
\begin{teo}
    $\mathbb{N}$ no está acotado superiormente.\\\\
	Demostración.-\; Supongamos que $\mathbb{N}$ estuviera acotado superiormente. Como $\mathbb{N}=\emptyset$, existiría una cota superior mínima $\alpha$ de $\mathbb{N}$. Entonces
	\begin{center}
	    $\alpha\geq n$ para todo $n$ de $\mathbb{N}.$
	\end{center}
	Por consiguiente,
	\begin{center}
	    $\alpha\geq n+1$ para todo $n$ de $\mathbb{N},$
	\end{center}
	ya que si $n$ pertenece a $\mathbb{N}$, $n+1$ también. Pero esto significa que 
	\begin{center}
	    $\alpha-1\geq n$ para todo $n$ de $\mathbb{N},$
	\end{center}
	lo cual quiere decir que $\alpha-1$ también es una cota superior de $\mathbb{N}$, lo que contradice el hecho de que $\alpha$ sea la cota superior mínima de $\mathbb{N}$.\\\\
\end{teo}

% ------------------- teorema 8.3
\begin{teo}
    Para cualquier $\epsilon>0$ existe un número natural $n$ con $1/n<\epsilon$.\\\\
    Demostración.-\; Supongamos que no fuese así; entonces $1/n\geq \epsilon$ para todo $n$ de $\mathbb{N}$. Por tanto, $n\leq 1/\epsilon$ para todo $n$ de $\mathbb{N}.$ Pero esto significa que $1/\epsilon$ es una cota superior de $\mathbb{N},$ lo cual contradice el resultado del teorema 8.2.\\\\
\end{teo}


\section{Problemas}

\begin{enumerate}[\bfseries 1.]

    %-------------------- 1.
    \item Hallar la cota superior mínima y la cota inferior máxima (si existe) de los siguientes conjuntos. Decida también cuáles de ellos poseen un elemento máximo y un elemento mínimo (es decir, en qué casos la cota superior mínima y cota inferior máxima pertenecen al conjunto).

	\begin{enumerate}[\bfseries (i)]

	    %---------- (i)
	    \item $\left\{ \dfrac{1}{n}: n\mbox{ en }\mathbb{N}\right\}$.\\\\
		Respuesta.-\; Sea $A$ el conjunto dado. Vemos que el $\sup A = \max A = 1$, $\inf A = 0$ y $\min A$ no existe.\\\\

	    %---------- (ii)
	    \item $\left\{ \dfrac{1}{n}: n \mbox{ en } \mathbb{Z} \mbox{ y } n\neq 0\right\}$.\\\\
		Respuesta.-\; Sea $A$ el conjunto dado. Podemos ver que $\sup A = \max A = 1$ y $\inf A  = \min A = -1$.\\\\

	    %---------- (iii)
	    \item $\left\{ x:x=0 \mbox{ o } x=1/n \mbox{ para algún } n \mbox{ en } \mathbb{N}\right\}$.\\\\
		Respuesta.-\;Llamemos $A$ al conjunto dado. Todos los números están en el intervalo $[0,1]$. De donde el $0$ es el mayor límite inferior y el $1$ es el menor límite superior, es decir,
		$\sup A = \max A = 1$ y $\inf A = \min A = 0$.\\\\

	    %---------- (iv)
	    \item $\left\{ x:0\leq x \leq \sqrt{2} \mbox{ y } x \mbox{ racional }\right\}$.\\\\
		Respuesta.-\; Sea $A$ el conjunto dado. En este caso $0$ es el mayor límite inferior y está contenido en el conjunto y $\sqrt{2}$ es el límite superior mínimo pero no está en el conjunto, por lo tanto, $\inf A = \min A = 0$ y $\sup A = \sqrt{2}.$\\\\

	    %---------- (v)
	    \item $\left\{ x:x^2+x+1\geq 0\right\}$.\\\\
		Respuesta.-\; No existe ni máximo ni mínimo, tampoco supremo o ínfimo.\\\\

	    %---------- (vi)
	    \item $\left\{ x:x^2+x-1<0\right\}$.\\\\
		Respuesta.-\; Sea $A=\left\{ x:x^2+x-1<0\right\}$, entonces se tiene $$x^2+x-1<0 \; \Leftrightarrow \; \dfrac{-1-\sqrt{5}}{2}<x<\dfrac{-1+\sqrt{5}}{2}$$. Por lo tanto $\sup A = \dfrac{-1-\sqrt{5}}{2}$ y $A=\dfrac{-1+\sqrt{5}}{2}$ los dos no contenidos en $A$.\\\\ 

	    %---------- (vii)
	    \item $\left\{ x:x<0\mbox{ y } x^2+x-1<0\right\}$.\\\\
		Respuesta.-\; Designemos al conjunto dado con $A$. Luego ya que 
		$$x<0 \mbox{ y } x^2+x-1<0\; \Leftrightarrow \; \dfrac{-1-\sqrt{5}}{2}<x<0$$
		Entonces $\sup A = 0 \notin A,\; \inf A = \dfrac{-1-\sqrt{5}}{2}\notin A.$\\\\

	    %---------- (viii)
	    \item $\left\{ \dfrac{1}{n}+(-1)^n:n \mbox{ en } \mathbb{N}\right\}$.\\\\
		Respuesta.-\; Sea $A$ el conjunto designado y sea $a_n = \dfrac{1}{n}+(-1)^n$. Entonces $a_1=0$, para indices pares $a_n=\dfrac{1}{n}+1.$ La sucesión es decreciente, converge en $1$ y el mayor elemento se obtiene en $n=2$, $a_2=\dfrac{3}{2}$. Para indices impares $a_n=\dfrac{1}{n}-1$, la secuencia es decreciente, converge en $-1$ pero este número no se consigue, por lo que 
		$$-1<a_n\leq \dfrac{3}{2}.$$
		De donde concluimos que $\inf A = -1$ pero no existe el mínimo y $\sup A \max A = \dfrac{3}{2}.$\\\\

	\end{enumerate}

    %------------------- 2.
    \item 

\end{enumerate}



    %---------- Derivadas
	%\chapter{Derivadas}

\begin{tcolorbox}
    \begin{def.}
    La función $f$ es \textbf{\boldmath diferenciable en $a$} si existe el  
    $$\lim_{h\to 0}\dfrac{f(a+h)-f(a)}{h}$$
    En este caso, dicho límite se presenta mediante $f'(a)$ y se denomina la derivada de $f$ en $a$. (Diremos también que $f$ es diferenciable si $f$ es diferenciable en $a$ para todo $a$ del dominio de $f.$)\\\\
    La derivada $f'$ son representadps a menudo mediante
    $$\dfrac{dg(x)}{dx}, \qquad \dfrac{df(x)}{dx}\bigg|_{x=a}$$
    \end{def.}
\end{tcolorbox}

Recordemos que la diferenciabilidad se supone que es una mejora respecto a la simple continuidad. Esto lo demuestran los numerosos ejemplos de funciones que son continuas, pero no diferenciables; sin embargo, hay que destacar un punto importante:\\

% -------------------- teorema 1
\begin{teo}
    Si $f$ es diferenciable en el punto $a$, entonces $f$ es continua en $a$.\\\\
	Demostración.-\;
	$$\lim_{h\to 0} \left[f(a+h)-f(a)\right] = \lim_{h\to 0} \left[\dfrac{f(a+h)-f(a)}{h}\cdot h\right] = \lim_{h\to 0} \dfrac{f(a+h)-f(a)}{h}\cdot \lim_{h\to 0} h = f'(a)\cdot 0 = 0.$$\\
	La ecuación $\lim\limits_{h\to 0}f(a+h)-f(a)=0$ es equivalente a $\lim\limits_{x\to a} f(x)=f(a);$ así, $f$ es continua en $a$.\\\\
\end{teo}

Es muy importante recordar el Teorema 1, e igualmente importante recordar que el recíproco no es cierto. Una función diferenciable es continua, pero una función continua no necesariamente es diferenciable.\\

Las distintas funciones $f^{((k)}$, para $k\leq 2$ se denominan generalmente derivadas de orden superior de $f$. Y en la notación se tiene $$\dfrac{d\left(\frac{df(x)}{dx}\right)}{dx}\dfrac{d^2f(x)}{dx^2}.$$

\section{Problemas}

\begin{enumerate}[\bfseries 1]

    %-------------------- 1.
    \item 
	\begin{enumerate}[(a)]

	    %---------- (a)
	    \item Demuestre, aplicando directamente la definición, que si $f(x)=1/x$, entonces $f'(a)=-1/a^2$ para $a\neq 0$.\\\\
		Demostración.-\; Por definición se tiene $$f'(a)=\lim_{h\to 0}\dfrac{\frac{1}{a+h}-\frac{1}{a}}{h} = \lim_{h\to 0} \dfrac{\frac{a-(a+h)}{a(a+h)}}{h} = \lim_{h\to a}\dfrac{-h}{a(a+h)h} = -\dfrac{1}{a^2}, \quad \mbox{siempre que}\; x\neq 0.$$\\

	    %---------- (b)
	    \item Demuestre que la recta tangente a la gráfica de $f$ en el punto $(a,1/a)$ sólo corta a la gráfica de $f$ en este punto.\\\\
		Demostración.-\; La pendiente de la tangente cuando $x=a$ es $-\dfrac{1}{a^2}$, de donde la ecuación de la recta tangente es,
		$$y-\dfrac{1}{a} = -\dfrac{1}{a^2}(x-a)\quad \Rightarrow \quad a^2y-a=-x+a \qquad \Rightarrow \quad a^2y+x=2a.$$
		Luego $y=\dfrac{1}{x}$ ya que necesitamos encontrar el punto de intersección, y en consecuencia,
		$$a^2\dfrac{1}{x}+x=2a\quad \Rightarrow \quad x^2-2ax+a^2=0 \quad \Rightarrow \quad (x-a)^2=0 \quad \Rightarrow \quad x=a$$
		Por lo tanto, el único punto de intersección será $\left(a,\dfrac{1}{a}\right)$.\\\\

	\end{enumerate}

    %-------------------- 2.
    \item
	\begin{enumerate}[(a)]

	    %---------- (a)
	    \item Demuestre que si $f(x)=1/x^2$, entonces $f'(a)=-2/a_3$ para $a\neq 0$.\\\\
		Demostración.-\; Por definición tenemos,
		$$f'(a)=\lim_{h\to 0}\dfrac{\frac{1}{(a+h)^2}-\frac{1}{a^2}}{h} = \lim_{h\to 0} \dfrac{\frac{a^2-(a+h)^2}{a^2(a+h)^2}}{h} = \lim_{h\to a}\dfrac{-h(2a+h)}{ha^2(a+h)^2} = -\dfrac{2}{a^3}, \quad \mbox{siempre que}\; x\neq 0.$$\\

	    %---------- (b)
	    \item Demuestre que la recta tangente a $f$ en el punto $(a,1/a^2)$ corta a $f$ en otro punto, que se encuentra en el lado opuesto del eje vertical.\\\\
		Demostración.-\; Ya que $f'(a)=-\dfrac{2}{a^3}$ que representa la pendiente de la tangente entonces la ecuación de la tangente estará dada para $(a,1/a^2)$ por,
		$$y-y_1=m(x-x_1)\quad \Rightarrow \quad y-\dfrac{1}{a^2}=-\dfrac{2}{a^3}(x-a)\quad \Rightarrow \quad y=-\dfrac{2x}{a^3}+\dfrac{3}{a^2}$$
		luego resolviendo para $x$ e $y$ sabiendo que $y=\dfrac{1}{x^2}$ tenemos,
		$$x_1 = a,\quad x_2 = -\dfrac{a}{2} \qquad \mbox{e}\qquad y_1 = \dfrac{1}{a^2}, \quad y_2=\dfrac{4}{a^2}.$$
		Por lo que la tangente intersecta a la $f$ en $(a,1/a^2)$ y en $(-a/2,4/a^2)$. Así estos puntos son apuestos al eje vertical.\\\\



	\end{enumerate}

    %-------------------- 3.
    \item Demuestre que si $f(x) = \sqrt{x}$, entonces $f'(a) = 1/\left(2\sqrt{a}\right)$, para $a > 0$.\\\\
	Demostración.-\; Por definición,
	$$f'(a) = \lim_{h\to 0}\dfrac{\sqrt{a+h}-\sqrt{a}}{h} \cdot \dfrac{\sqrt{a+h}-\sqrt{a}}{\sqrt{a+h}-\sqrt{a}} = \lim_{a\to 0}\dfrac{a+h-a}{h\sqrt{a+h}\sqrt{a}} = \dfrac{1}{2\sqrt{a}}$$
	para $a>0$.\\\\

    %-------------------- 4.
    \item Para cada número natural $n$, sea $S_n(x)=x^n$. Recordando que $S_1'(x)=1, S_2'(x)=2x$, y que $S_3'(x)=3x^2$, encuentre una fórmula para $X_n'(x)$. Demuestre que la fórmula es correcta.\\\\
	Demostración.-\; Usando el teorema del binomio se tiene,
	$$\begin{array}{rcl}
	    S_n'(x)&=& \displaystyle\lim_{h\to 0}\dfrac{(x+h)^n-x^n}{h}\\\\
		   &=&\displaystyle\lim_{h\to 0} \dfrac{\displaystyle\sum_{j=0}^n {n\choose j} \left(x^{n-j} h^{j}\right)-x^n}{h}\\\\
		   &=&\displaystyle \lim_{h\to 0}\dfrac{h\left[\displaystyle\sum_{h\to 0}{n\choose j} x^{n-j}h^{j-1}\right]}{h}\\\\
		   &=&\displaystyle \lim_{h\to 0}\left[{n\choose 1}x^{n-1}h^{1-1}+{n\choose 2}x^{n-2}h^{2-1}+\ldots + {n\choose n}x^{0}h^{n-1}\right]\\\\
		   &=&\displaystyle \lim_{h\to 0}\left[{n\choose 1}x^{n-1}\cdot 1+{n\choose 2}x^{n-2}h^{2-1}+\ldots + {n\choose n}x^{0}h^{n-1}\right]\\\\
		   &=&\displaystyle{n\choose 1}x^{n-1} = nx^{n-1}\\\\
       \end{array}$$

    %-------------------- 5.
   \item Halle $f'$ si $f(x)=[x]$.\\\\
       Respuesta.-\; Sea $x$ un número no entero, entonces
       $$f'(x)=\lim_{h\to 0} \dfrac{[x+h]-[x]}{h}=\lim_{h\to 0}\dfrac{0}{h}=0.$$
       Luego si $x$ es un número entero, entonces por el límite por la izquierda se tiene,
       $$\lim_{h\to 0^-}\dfrac{[x+h]-[h]}{h}=\lim_{h\to 0^-}\dfrac{-1}{h}=\infty$$
       por último si $x$ es un número entero, entonces por el límite por la derecha se tiene,
       $$\lim_{h\to 0^+}\dfrac{[x+h]-[x]}{h}=\lim_{h\to 0^+}\dfrac{0}{h}=0.$$
       Por lo tanto, $f'(x)=0, \; \forall x \notin \mathbb{Z}$ y $f'(x)=\mbox{ no existe, } \forall x\in \mathbb{Z}$.\\\\

    %-------------------- 6.
   \item Demuestre, aplicando la definición:\\

   \begin{enumerate}[(a)]

       %---------- (a)
       \item si $g(x)=f(x)+c$, entonces $g'(x)=f'(x)$.\\\\
	   Demostración.-\; Por definición se tiene,
	   $$g'(x) = \lim_{h\to 0} \dfrac{f(x+h)+c-[f(x)-c]}{h} = \lim_{h\to 0}\dfrac{f(x+h)-f(x)}{h}=f'(x).$$\\\\

       %---------- (b)
       \item si $g(x)=cf(x)$, entonces $g'(x)=cf'(x)$.\\\\
	   Demostración.-\; Por definición,
	   $$g'(x) = \lim_{h\to 0} \dfrac{cf(x+h)-cf(x)}{h} = \lim_{h\to 0}c\left(\dfrac{f(x+h)-f(x)}{h}\right)=cf'(x).$$\\\\

    \end{enumerate}

    %-------------------- 7.
    \item Suponga que $f(x)=x^3$.\\

	\begin{enumerate}[(a)]

	    %---------- (a)
	    \item ¿Cuál es el valor de $f'(9), f'(25), f'(36)$?.\\\\
		Respuesta.-\; Se sabe que $f'(x)=x^n = nx^{n-1}$ por lo que 
		$$f'(x)=3x^{2}$$
		Así,
		$$\begin{array}{rclcl}
		    f'(9) &=& 3\cdot 9^{2} &=& 243.\\
		    f'(25) &=& 3\cdot 25^{2} &=& 1875.\\
		    f'(36) &=& 3\cdot 36^{2} &=& 3888.\\
		\end{array}$$
		\vspace{.7cm}

	    %---------- (b)
	    \item ¿Y el valor de $f'\left(3^2\right),f'\left(5^2\right),f'\left(6^2\right)$?.\\\\
		Respuesta.-\; Sea $f'(x)=3x^{2}$, entonces
		$$\begin{array}{rclcl}
		    f'\left(3^2\right) &=& 3\cdot \left(3^2\right)^{2} &=& 243.\\
		    f'\left(5^2\right) &=& 3\cdot \left(5^2\right)^{2} &=& 1875.\\
		    f'\left(6^2\right) &=& 3\cdot \left(6^2\right)^{2} &=& 3888.\\
		\end{array}$$
		\vspace{.7cm}


	    %---------- (c)
	    \item Calcule $f'\left(a^2\right),f'\left(x^2\right)$. Si no encuentra este problema trivial es que no tiene en cuenta una cuestión muy importante: $f'(x^2)$ significa la derivada de $f$ en el punto que denominamos $x^2$; no es la derivada en el punto $x$ de la función $g(x)=f\left(x^2\right)$. \\\\
		Respuesta.- Ya que $f'(x)=3x^{2}$, entonces
		$$\begin{array}{rclcl}
		    f'\left(a^2\right)&=&3\left(a^2\right)^2&=&3a^4\\
		    f'\left(x^2\right)&=&3\left(x^2\right)^2&=&3x^4\\
		\end{array}$$

	    %---------- (d)
	    \item Para aclarar la cuestión anterior. Si $f(x)=x^3$, compare $f'\left(x^2\right)$ y $g'(x)$ donde $g(x)=f\left(x^2\right)$.\\\\
		Respuesta.-\; Es $f'\left(x^2\right) = 3\left(x^2\right)^2 =3x^4$, pero 
		$$\begin{array}{rcl}
		    g'(x)&=&\lim\limits_{h\to 0}\dfrac{f(x+h)^2-f\left(x^2\right)}{h}\\\\
			 &=&\lim\limits_{h\to 0}\dfrac{\left[(x-h)^2\right]^3-\left(x^2\right)^3}{h}\\\\
			 &=&\lim\limits_{h\to 0}\dfrac{h\left(6x^5+15x^5h+20x^3h^2+15x^2h^3+6xh^4+h^5\right)}{h}\\\\
			 &=&6x^5.\\\\
	    \end{array}$$

	\end{enumerate}

    %-------------------- 8.
    \item 
	\begin{enumerate}[(a)]

	    %---------- (a)
	    \item Suponga que $g(x)=f(x+c).$ Demuestre (partiendo de la definición) que $g'(x)=f'(x+c)$. Dibuje un esquema para ilustrarlo. Para resolver el problema debe escribir las definiciones de $g'(x)$ y $f'(x+c)$ correctamente. El objetivo del Problema 7 era convencerle de que aunque este problema es fácil no se trata de una trivialidad, y que hay algo que debe demostrarse: no se puede simplemente poner primas en la ecuación $g(x) = f ( x + c)$.\\\\
		Demostración.-\; Por el hecho de que $g(x)=f(x+c)$ y por definición de diferencia tenemos, 
		$$g'(x) = \lim_{h\to 0}\dfrac{g(x+h)-g(x)}{h}=\lim_{h\to 0}\dfrac{f\left[(x+c)+h\right]-f(x-c)}{h} = f'(x+c).$$\\

	    %---------- (b)
	    \item Con el objeto de enfatizar el anterior punto. Demuestre que si $g(x)=f(cx)$, entonces $g(x)=c\cdot f'(cx)$. Intente también visualizar gráficamente por qué esta igualdad es cierta.\\\\
		Demostración.-\; Por el hecho de que $g(x)=f(cx)$ y por definición de diferencia se tiene,
		$$\begin{array}{rcl}
		    g'(x)&=&\lim\limits_{h\to 0}\dfrac{g(x+h)-g(x)}{h}\\\\
			 &=&\lim\limits_{h\to 0}\dfrac{f(cx+ch)-f(cx)}{h}\\\\
			 &=&\lim\limits_{h\to 0}\dfrac{c\left[f(cx+ch)-f(cx)\right]}{ch}\\\\
		\end{array}$$
		Sea $ch=k$ de donde
		$$c\lim_{k\to 0}\dfrac{f(cx+k)-f(cx)}{k}\quad \Rightarrow \quad cf'(cx).$$\\

	    %---------- (c)
	    \item Suponga que $f$ es diferenciable y periódica, con periodo $a$ (por ejemplo, $f(x+a)=f(x)$) para todo $x$). Demuestre que $f'$ es también periódica.\\\\
		Demostración.-\; Por hipótesis tenemos,
		$$f'(x)=\lim_{h\to 0}\dfrac{f(x+h)-f(x)}{h} = \lim_{h\to 0}\dfrac{f\left[(x+a)+h\right]-f(x-a)}{h} = f'(x+a).$$\\

	\end{enumerate}

    %-------------------- 9.
    \item Halle $f'(x)$ y también $f'(x+3)$ en los siguientes casos. Hay que ser muy metódico para no cometer un error en algún paso.\\

	\begin{enumerate}[(a)]

	    %---------- (a)
	    \item $f(x)=(x+3)^5$.\\\\
	    Respuesta.-\; si $g(x)=x^5$, entonces $g'(x)=5x^4$. Ahora, $f(x)=g(x+3)$ por tanto $$f'(x)=g'(x+3)=5(x+3)^4\quad  \mbox{y} \quad f'(x+3) = 5[(x+3)+3]^4 =5(x+6)^4.$$\\

	    %---------- (b)
	    \item $f(x+3)=x^5$.\\\\
		Respuesta.-\; Sea $t=x+3\; \Rightarrow \; x=t-3$, entonces
		$$f(t)=(t-3)^5.$$
		De donde, $f(x)=(x-3)^5$.\\

		Si $g(x)=x^5$, entonces $g'(x)=5x^4$. Ahora,  $f(x)=g(x-3)$, por tanto  
		$$f'(x) = 5(x-3)^4 \qquad \mbox{y}\qquad f'(x+3) = 5[(x+3)-3]^4 = 5x^4.$$\\

	    %---------- (c)
	    \item $f(x+3)=(x+5)^7$.\\\\
		Respuesta.-\; Sea $t=x+3$, de donde $x=t-3$, entonces
		$$f(t) = \left[(t-3)+5\right]^7 = (t+2)^7.$$
		Podemos reescribir esta última función como, $f(x)=(x+2)^7$.\\

		Sea $g(x)=x^7$ que implica $g'(x)=7x^6$, por lo que,
		$$f'(x)=g'(x+2)=7(x+2)^6\qquad \mbox{y}\qquad f'(x+3)=g'(x+3+2)=7(x+5)^6.$$\\

	\end{enumerate}

    %-------------------- 10.
    \item Halle $f'(x)$ si $f(x)=g(t+x),$ y si $f(t)=g(t+x)$. Las respuestas no son idénticas.\\\\
	Respuesta.-\; Por definición e hipótesis,
	$$f'(x)=\lim_{h\to 0} \dfrac{f(x+h)-f(x)}{h}=\lim_{h\to 0}\dfrac{g(t+x+h)-g(t+x)}{h}=\lim_{h\to 0}\dfrac{g\left[(t+x)+h\right]-g(t+x)}{h}=g'(t+x).$$
	Por otro lado,
	$$f'(x)=\lim_{h\to 0}\dfrac{f(x+h)-f(x)}{h}=\lim_{h\to 0}\dfrac{g(x+h+x)-f(x+x)}{h}=\lim_{h\to 0}\dfrac{g(2x+h)-f(2x)}{h} = g'(2x).$$\\

    %-------------------- 11.
    \item 
	\begin{enumerate}[(a)]

	    %---------- (a)
	    \item Demuestre que Galileo se equivocó: si un cuerpo cae una distancia $s(t)$ en $t$ segundos, y $s'$ es proporcional a $s$, entonces $s$ no puede ser una función de la forma $s(t)=ct^2.$\\\\
		Demostración.-\; Si $x$ es una función de la forma $s(t)=ct^2$ entonces $s'(t)=2ct$. Luego sustituyendo el valor $t=\sqrt{\dfrac{s}{c}}$, se tiene
		$$s'(t)=2c \sqrt{\dfrac{s}{c}}=2\sqrt{cs}.$$
		Dado que $s'(t)$ es directamente proporcional a $\sqrt{s(t)}$ entonces contradice la afirmación obtenida.\\\\ 

	    %---------- (b)
	    \item Demuestre que las siguientes afirmaciones sobre $s$ son ciertas, si $s(t)=(a/2)t^2$ (la primera afirmación demostrará por qué hemos hecho el cambio de $c$ a $a/2$):

		\begin{enumerate}[(i)]

		    %----- (i)
		    \item $s''(t)=a$ (la aceleración es constante).\\\\
			Demostración.-\; Sea $s(t)=\dfrac{a}{2}t^2$, entonces
			$$s'(t)=\dfrac{a}{2}\lim_{g\to 0}\dfrac{(t+h)^2-t^2}{h}=\dfrac{a}{2}(2t)=at.$$
			Así,
			$$s''(t)=\lim_{h\to 0}\dfrac{s'(t+h)-s'(t)}{h}=a\lim_{h\to 0}\dfrac{(t+h)-t}{h}=a.$$
			Donde vemos que la aceleración es constante.\\\\

		    %----- (ii)
		    \item $[s'(t)]^2=2as(t)$.\\\\
			Demostración.-\; Sea $t^2=\dfrac{2s(t)}{a}$, entonces
			$$\left[s'(t)\right]^2 = a^2t^2 = a^2\left[\dfrac{2s(t)}{a}\right] = 2as(t).$$\\

		\end{enumerate}

	    %---------- (c)
	    \item Si $s$ mide en pies, el valor de $a$ es $32$. ¿Cuántos segundos tendrá que permanecer fuera de la trayectoria de una lámpara que cae del techo, desde una altura de $400$ pues?. Si no se aparta, ¿cuál será la velocidad de la lámpara cuando le golpee? ¿A qué altura encontraba la lámpara cuando se desplazaba a la mitad de dicha velocidad?.\\\\
		Respuesta.-\; Ya que $a=32$, entonces
		$$s(t)=\dfrac{a}{2}t^2 \;\Rightarrow \; 400=\dfrac{32}{2}t^2 \; \Rightarrow \; t=5\; \mbox{s}.$$
		Luego,
		$$s'(t)=(2\cdot 32 \cdot 400)^{1/2} = 160\; \mbox{m/s}.$$
		Por último para $s'(t)=80$, tenemos
		$$(80)^2 = 2\cdot 32 \cdot s(t)\; \Rightarrow \; s(t)=100\; \mbox{pies desde arriba}.$$\\

	\end{enumerate}

    %-------------------- 12.
    \item Suponga que en una carretera el límite de velocidad se especifica en cada punto. En otras palabras, existe una cierta función $L$ tal que la velocidad límite a $x$ millas desde el inicio de la carretera es $L(x)$. Dos automóviles, $A$ y $B$, se desplazan por dicha carretera; la posición del automóvil $A$ en el tiempo $t$ es $a(t)$, y la del automóvil $B$ es $b(t)$.

    \begin{enumerate}[(a)]

	%---------- (a)
	\item ¿Qué ecuación expresa el hecho de que el automóvil $A$ siempre se desplaza a la velocidad límite? (La respuesta no es $a'(t) = L(t).)$\\\\
	    Respuesta.-\; La ecuación estará dada por,
	    $$a'(t)=L(x)=L\left[a(t)\right].$$

	%---------- (b)
	\item Suponga que $A$ siempre se desplaza a la velocidad límite, y que la posición de $B$ en el tiempo $t$ es la posición de $A$ en el tiempo $t - 1$. Demuestre que $B$ también se desplaza en todo momento a la velocidad límite.\\\\
	    Demostración.-\; Sea $b(t)=a(t-1)$ y $L(x)=L\left[a(t)\right]$, entonces para $B$ se tiene
	    $$b'(t)=L\left[b(t)\right]\quad \Rightarrow \quad a'(t-1)=L\left[a(t-1)\right].$$
	    Así, si $t-1$ es reemplazado por $t$, tenemos $a'(t)=L(t)$ el cual es cierto. Por lo que $B$ se desplaza a la velocidad límite.\\\\

	%---------- (c)
	\item Suponga, por el contrario, que $B$ siempre se mantiene a una distancia constante por detrás de $A$. ¿En qué condiciones $B$ se desplazará todavía en todo momento a la velocidad límite?.\\\\
	    Respuesta.-\; Sea $b(t)=a(t)-d$ para alguna constante $d>0$, entonces dado $a'(t)=L\left[a(t)\right]$, se tiene
	    $$b'(t)=L\left[b(t)\right] \; \Rightarrow \; b'(t)=L\left[a(t)-d\right] = L\left[a(t)\right] = L\left[b(t)+d\right]$$
	    Esto es, $B$ se desplaza a la velocidad límite, si $L$ es una función periódica con periodo $d$.\\\\
		
    \end{enumerate}

    %-------------------- 13.
    \item Suponga que $f(a)=g(a)$ y que la derivada por la izquierda de $f$ en $a$ es igual a la derivada por la derecha de $g$ en $a$. Defina $h(x)=f(x)$ para $x\leq a,$ y $h(x)=g(x)$ para $x\geq a$. Demuestre que $h$ es diferenciable en $a$.\\\\
	Demostración.-\; Sea 
	$$\lim_{t\to 0+}\dfrac{h(a+t)-h(a)}{t}=\lim_{t\to 0+}\dfrac{g(a+t)-g(a)}{t}$$
	$$\mbox{y}$$
	$$\lim_{t\to 0-}\dfrac{h(a+t)-h(a)}{t}=\lim_{t\to 0-}\dfrac{f(a+t)-f(a)}{t}.$$

	Y por el hecho de la existencia del límite por la derecha y por la izquierda entonces existe el límite:
	$$\lim_{t\to 0}\dfrac{h(a+t)-h(a)}{t}.$$\\

    %-------------------- 14.
    \item Sea $f(x)=x^2$ si $x$ es racional, y $f(x)=0$ si $x$ es irracional. Demuestre que $f$ es diferencial en $0$.\\\\
	Demostración.-\; Por definición, 
	$$f'(0)=\lim_{h\to 0}\dfrac{f(0+h)-f(0)}{h}=\lim_{h\to 0}\dfrac{h^2-0}{h}=0.$$
	Luego,
	$$\lim_{h\to 0^+}\lim_{h\to 0}\dfrac{f(0+h)-f(0)}{h}=0= \lim_{h\to 0^-}\dfrac{f(0+h)-f(0)}{h}$$

	Entonces, $f$ es diferencial en $0$.\\\\

    %-------------------- 15.
    \item 
	\begin{enumerate}[(a)]

	    %---------- (a)
	    \item Sea $f$ una función tal que $|f(x)|\leq x^2$ para todo $x$. Demuestre que $f$ es diferenciable en el punto $0$.\\\\
		Demostración.-\; Tenemos que
		$$|f(x)|\leq x^2 \quad \Rightarrow \quad |f(0)|=0\quad \Rightarrow \quad f(0)=0.$$
		Ahora, vemos que
		$$\bigg|\dfrac{f(h)}{h}\bigg|\leq \dfrac{h^2}{|h|}\quad \Rightarrow \quad \bigg|\dfrac{f(h)}{h}\bigg|\leq |h|.$$
		Por lo tanto,
		$$\begin{array}{rcl}
		    \lim\limits_{h\to 0}\dfrac{f(h)}{h}&\leq&\lim\limits_{h\to 0}|h|\\\\
		    \lim\limits_{h\to 0} \dfrac{f(h)}{h}&=&0\\\\
		    \lim\limits_{h\to 0} \dfrac{f(h)-0}{h}&=&0\\\\
		    \lim\limits_{h\to 0} \dfrac{f(0+h)-f(0)}{0}&=&0\\\\
					      f'(0)&=&0\\\\
		\end{array}$$
		Así, ya que $f'(0)$ existe entonces $f$ es diferencial en $0$.\\\\

	    %---------- (b)
	    \item Este resultado se puede generalizar si $x^2$ se sustituye por $|g(x)|$, en el caso de que $g$ cumpla una determina propiedad. ¿Cuál?.\\\\
		Respuesta.-\; Reemplacemos $x^2=|g(x)|$ por lo que nos queda
		$$|f(x)|\leq |g(x)|.$$
		Al ser $f$ diferenciable en $0$ entonces $f'(0)$ debe existir, por lo tanto
		$$|g(x)|\geq |f(x)|\quad \Rightarrow \quad |g(0)|\geq |f(0)|\quad \Rightarrow \quad |g(0)|\geq 0.$$
		Luego para $g$ tan pequeño como se quiera,
		$$|g(0)|=0\quad \Rightarrow \quad g(0)=0.$$
		Además podemos observar,
		$$\begin{array}{rcl}
		    \bigg|\dfrac{f(h)}{h}\bigg|&\leq&\bigg|\dfrac{g(h)}{h}\bigg|\\\\
		    \lim\limits_{h\to 0}\bigg|\dfrac{f(h)}{h}\bigg|&\leq&\lim\limits_{h\to 0}\bigg|\dfrac{g(h)}{h}\bigg|\\\\
		    \lim\limits_{h\to 0} \bigg|\dfrac{f(h)-f(0)}{h}\bigg| & \leq &\lim\limits_{h\to 0} \bigg|\dfrac{g(h)-f(0)}{h}\bigg|\\\\
					       |f'(0)|&\leq&|g'(0)|\\\\
					       0&\leq&|g'(0)|\\\\
					       |g'(0)|&\geq&0\\
		\end{array}$$	
		Luego ya que $g'(0)$ podría ser lo más pequeño que se quiera, entonces
		$$|g'(0)|=0\quad \Rightarrow \quad g'(0)=0.$$
		Por lo tanto $f$ es diferenciable en $0$ si se tiene,
		$$g(0)=0\qquad \mbox{y} \qquad g'(0).$$\\

	\end{enumerate}

    %-------------------- 16.
    \item Sea $\alpha>1.$ Si $f$ satisface $|f(x)|\leq |x|^{\alpha}$, demuestre que $f$ es diferenciable en $0$.\\\\
	Demostración.-\; Sea $|f(x)|\leq |x|^\alpha\;\Rightarrow \; -x^\alpha\leq f(x)\leq x^\alpha$ y $f(0)=0$ entonces,
	$$-h^{\alpha-1}\leq \dfrac{f(h)-f(0)}{h}\leq h^{\alpha-1}\quad \Rightarrow \quad -h^\alpha \leq f(h)\leq h^\alpha.$$
	Por lo que concluimos que $f$ es diferencial en $0$ y $f'(0)=0.$\\\\


    %-------------------- 17.
    \item Sea $0<\beta<1.$ Demuestre que si $f$ satisface $|f(x)|\geq |x|^{\beta}$ y $f(0)=a,$ entonces $f$ no es diferenciable en $0$.\\\\
	Demostración.-\; Sea $|f(x)|\geq |x|^\beta\;\Rightarrow \; -x^\beta\leq f(x)\leq x^\beta$ y $f(0)=0$ entonces,

	$$-h^{\beta-1}\leq \dfrac{f(h)-f(0)}{h}\leq h^{\beta-1}.$$
	Ya que $0<\beta < 1,$, $h^\beta$ será $\dfrac{1}{h^{1-\beta}}$, el cual tiende a $\infty$. Por lo tanto $f$ no es diferenciable en $0$.\\\\

    %-------------------- 18.
    \item Sea $f(x)=0$ si $x$ es irracional, y $1/q$ si $x=p/q$, fracción irreducible. Demuestre que $f$ no es diferenciable en $a$ para cualquier $a$.\\\\
	Demostración.-\; Si $a$ es racional, al no ser $f$ continua en $a$ será $f$ derivable en $a$. Si $a=0 a_1a_2a_3\ldots$ es irracional y $h$ es racional, entonces $a+h$ es irracional, co lo que $f(a+h)-f(a)=0$. Pero si $h = -0.00\ldots 0 a_{n+1}a_{n+2}\ldots,$ entonces $a+h=0a_1a_2\ldots a_n 000 \ldots$, con lo que $f(a+h)\geq 10^{-n}$, mientras que $|h|<10^{-n}$, la cual hace que $|[f(a+h)-f(a)]/h|\geq 1$. Así pues, $[f(a+h)-f(a)/h]$ es $0$ para valores de $h$ tan  pequeños como se quiera y tiene valores absolutos mayores que $1$ también con $h$ tan pequeño como se quiera, lo cual dice que $\lim\limits_{h\to 0}\dfrac{f(a+h)-f(a)}{h}$ no existe.\\\\

    %-------------------- 19.
    \item 
	\begin{enumerate}[(a)]

	    %---------- (a)
	    \item Suponga que $f(a)=g(a)=h(a)$, que $f(x)\leq g(x)\leq h(x)$ para todo $x$, y que $f'(a)=h'(a)$. Demuestre que $g$ es diferenciable en $a$, y que $f'(a)=g'(a)=h'(a)$.\\\\
		Demostración.-\; Ya que $f(x)\leq g(x)\leq h(x)$ y $f(a)=g(a)=h(a)$, entonces

		$$\begin{array}{rcccl}
		    f(x)&\leq&g(x)&\leq&h(x)\\\\
		    f(a+t)&\leq&g(a+t)&\leq&h(a+t)\\\\
		    \dfrac{f(a+t)-f(a)}{t}&\leq&\dfrac{g(a+t)-g(a)}{t}&\leq&\dfrac{h(a+t)-h(a)}{t}\\\\
		    \lim\limits_{h\to 0}\dfrac{f(a+t)f(a)}{t}&\leq&\lim\limits_{h\to 0}\dfrac{g(a+t)-g(a)}{t}&\leq&\lim\limits_{h\to 0}\dfrac{h(a+t)-h(a)}{t}\\\\
		    f'(a)&\leq&g'(a)&\leq&h'(a)\\\\
		\end{array}$$
		Luego sabemos que $f'(a)=h'(a)$ lo que implica $g'(a)$ existe y $f'(a)=g'(a)=h'(a)$. Por lo tanto $g$ es diferenciable en $a$.\\\\


	    %---------- (b)
	    \item Demuestre que la conclusión no es cierta si se omite la hipótesis $f(a)=g(a)=h(a)$.\\\\
		Demostración.-\; Se dará un contraejemplo sin la condición $f(a)=g(a)=h(a)$.\\
		Sea $f(x)=-1$, $g(x)=\dfrac{1}{1+e^{-x}}$ y $h(x)=2.$ Entonces
		    $$f(a) = -1,\quad g(a)=\dfrac{1}{1+e^{-a}},\quad h(a)=2.$$
		    Por lo que la conclusión no es cierta.\\\\

	\end{enumerate}

    %-------------------- 20.
    \item Sea $f$ una función polinómica; veremos en el próximo capítulo que $f$ es diferenciable. La recta tangente a $f$ en $\left(a,f(a)\right)$ es la gráfica $g(x)=f'(a)(x-a)+f(a)$. Por tanto, $f(x)-g(x)$ es la función polinómica $d(x)=f(x)-f'(a)(x-a)-f(a)$. Ya hemos visto que si $f(x)=x^2$, entonces $d(x)=(x-a)^2,$ y si $f(x)=x^3,$ entonces $d(x)=(x-a)^2(x-2a)$.\\

	\begin{enumerate}[(a)]

	    %---------- (a)
	    \item Halle $d(x)$ cuando $f(x)=x^4$, y demuestre que es divisible por $(x-a)^2$.\\\\
		Demostración.-\; Se tiene,
		$$\begin{array}{rcl}
		    d(x)&=&f(x)-f(a)(x-a)-f(a)=x^4-4a^3(x-a)-a^4\\\\
			&=&x^4-4a^3x+3a^4\\\\
			&=&(x-a)(x^3+ax^2+a^2x-3a^3)\\
		\end{array}$$
		Por lo que $d(x)$ es divisible por $(x-a)^2$.\\\\

	    %---------- (b)
	    \item Parece, ciertamente que $d(x)$ siempre sea divisible por $(x-a)^2$. En general, las rectas paralelas a la tangente cortan la gráfica de la función en dos puntos; la recta tangente corta a la gráfica sólo una vez cerca del punto, de manera que la intersección debería ser una doble intersección. Para dar una demostración riguroso, observe en primer lugar que 
	    $$\dfrac{d(x)}{x-a}=\dfrac{f(x)-f(a)}{x-a}-f'(a).$$
	    Ahora responda las siguientes cuestiones. ¿Por qué $f(x)-f(a)$ es divisible por $(x-a)$? ¿Por qué existe una función polinómica $h$ tal que $h(x)=d(x)/(x-a)$ para $x\neq a$? ¿Por qué el $\lim\limits_{x\to a} h(x) = 0$? ¿Por qué $h(a)=0$? ¿Por qué esto resuelve el problema?.\\\\
		Repuesta.-\; Tenemos $d(x)=f(x)-f'(a)(x-a)-f(a)$, entonces
		$$d(x)=f(x)-f'(a)(x-a)-f(a) \quad \Rightarrow \quad \dfrac{d(x)}{x-a}=\dfrac{f(x)-f(a)}{x-a}-f'(a)$$
		También sabemos que
		$$d(x)=[f(x)-f(a)]-f'(a)(x-a)$$
		de donde $d(x)$ es divisible por $(x-a)^2$ lo que implica que $f(x)-f(a)$ es divisible por $(x-a)$. Así,
		$$\dfrac{f(x)-f(a)}{x-a}$$
		es una función polinómica. Y por lo tanto la función $h$ es un polinomio, como sigue
		$$h(x)=\dfrac{d(x)}{x-a}\quad \Rightarrow \quad h(x)=\dfrac{f(x)-f(a)}{x-a}-f'(a)$$
		para $x\neq a$. Ahora tenemos 
		$$\begin{array}{rcl}
		    \lim\limits_{x\to a} h(x) &=&\lim\limits_{x\to a} \left[\dfrac{f(x)-f(a)}{x-a}-f'(a)\right]\\\\
		    \lim\limits_{x\to a} h(x) &=&\lim\limits_{x\to a} \left[\dfrac{f(x)-f(a)}{x-a}\right]-f'(a)\\\\
		    \lim\limits_{x\to a} h(x) &=&f'(a)-f'(a)\\\\
		    \lim\limits_{x\to a} h(x) &=&0\\\\
					      h(a)&=&0\\\\
		\end{array}$$

		    Significa que $h$ tiene a $a$ como una raíz. Esto indica que $\dfrac{d(x)}{x-a}$ es divisible por $x-a$. Y así, $d(x)$ es divisible por $(x-a)^2$.\\\\

	\end{enumerate}

    %-------------------- 21.
    \item 
	\begin{enumerate}[(a)]

	    %---------- (a)
	    \item Demuestre que $f'(a)=\lim\limits_{x\to a} \dfrac{f(x)-f(a)}{x-a}$.\\\\
		Demostración.-\; Sean $h=x-a$,  el hecho de que $\lim\limits_{x\to a}f(x)=\lim\limits_{h\to 0}f(a+h)$ y por un cambio infinitesimal $a\to a+h$, entonces 
		$$\begin{array}{rcl}
		     \lim\limits_{x\to a}\dfrac{f(x)-f(a)}{x-a}&=& \lim\limits_{x\to a+h}\dfrac{f(x)-f(a)}{x-a}\\\\
							       &=& \lim\limits_{h\to 0} \dfrac{f(a+h)-f(a)}{(a+h)-a}\\\\
							       &=& \lim\limits_{h\to 0}\dfrac{f(a+h)-f(a)}{h}\\\\
					&=& f'(a)\\\\
		\end{array}$$
		\vspace{.5cm}

	    %---------- (b)
	    \item Demuestre que las derivadas son una propiedad local: si $f(x)=g(x)$ para todo $x$ en algún intervalo abierto que contiene $a$, entonces $f'(a)=g'(a)$. (Esto significa que al calcular $f'(a)$, puede ignorarse a $f(x)$ para un determinado $x\neq a$. Evidentemente, ¡no! se puede ignorar a $f(x)$ para todos estos $x$ simultáneamente.)\\\\
		Demostración.-\; Se da $f$ y $g$ son iguales en un intervalo abierto que contiene $a$, entonces en el intervalo tenemos una pequeña cantidad $h\to 0$ como,
		$$\begin{array}{rcl}
		    \lim\limits_{h\to 0} f(a+h) &=& \lim\limits_{h\to 0} g(a+h)\\\\
		    \lim\limits_{h\to 0} \left[f(a+h)-f(a)\right] &=& \lim\limits_{h\to 0} \left[g(a+h)-g(a)\right]\\\\
		    \lim\limits_{h\to 0} \left[\dfrac{f(a+h)-f(a)}{h}\right] &=& \lim\limits_{h\to 0} \left[\dfrac{g(a+h)-g(a)}{h}\right]\\\\
									     f'(a)&=&g'(a)\\\\
		\end{array}$$

	\end{enumerate}

    %-------------------- 22.
    \item
	
\end{enumerate}


    %---------- Diferenciación
	%\chapter{Diferenciación}

% -------------------- teorema 1
\begin{teo}
    Si $f$ es una función constante, $f(x)=c,$ entonces
    \begin{center}
	$f'(a)=0$ para todo número $a$.
    \end{center}
    \vspace{.5cm}
	Demostración.-\; $$f'(a)=\lim\limits_{h\to 0}\dfrac{f(a+h)-f(a)}{h}=\lim\limits_{h\to 0}\dfrac{(c+h)-c}{h}=0.$$\\\\
\end{teo}

% -------------------- teorema 2
\begin{teo}
    Si $f$ es la función identidad, $f(x)=x,$ entonces
    \begin{center}
	$f'(a)=1$ para todo número $a$.
    \end{center}
    \vspace{.5cm}
	Demostración.-\;
	$$f'(a)=\lim_{h\to 0}\dfrac{f(a+h)-f(a)}{h}=\lim\limits_{h\to 0}\dfrac{a+h-a}{h}=\lim_{h\to 0}\dfrac{h}{h}=1.$$\\
\end{teo}

% -------------------- teorema 3
\begin{teo}
    Si $f$ y $g$ son diferencibles en $a$, entonces $f+g$ es también diferenciable en $a$, y
    $$(f+g)'(a)=f'(a)+g'(a).$$\\
	Demostración.-\; 
	$$\begin{array}{rcl}
	    (f+g)'(a) &=& \lim\limits_{h\to 0} \dfrac{(f+g)(a+h)-(f+g)(a)}{h}\\\\
		      &=& \lim\limits_{h\to 0} \dfrac{f(a+h)+g(a+h)-\left[f(a)+g(a)\right]}{h}\\\\
		      &=& \lim\limits_{h\to 0} \left[\dfrac{f(a+h)-f(a)}{h}+\dfrac{g(a+h)-g(a)}{h}\right]\\\\
		      &=& \lim\limits_{h\to 0} \dfrac{f(a+h)-f(a)}{h} \lim\limits_{h\to 0}\dfrac{g(a+h)-g(a)}{h}\\\\
		      &=& f'(a)+g'(a).\\\\
	\end{array}$$
\end{teo}

% -------------------- teorema 4
\begin{teo}
    Si $f$ y $g$ son diferenciables en $a$, entonces $f\cdot g$ es también diferenciable en $a$, y
    $$(f\cdot g)'(a)=f'(a)\cdot g(a)+f(a)\cdot g'(a).$$\\
	Demostración.-\; 
	$$\begin{array}{rcl}
	    (f\cdot g)'(a) &=& \lim\limits_{h\to 0} \dfrac{(f\cdot g)(a+h)-(f\cdot g)(a)}{h}\\\\
			   &=& \lim\limits_{h\to 0} \dfrac{f(a+h)g(a+h)-f(a)g(a)}{h}\\\\
			   &=& \lim\limits_{h\to 0} \left[\dfrac{f(a+h)\left(g(a+h)-g(a)\right)}{h}+\dfrac{\left(f(a+h)-f(a)\right)g(a)}{h}\right]\\\\
			   &=& \lim\limits_{h\to 0} f(a+h)\cdot \lim\limits_{h\to 0} \dfrac{g(a+h)-g(a)}{h}+\lim\limits_{h\to 0} \dfrac{f(a+h)-f(a)}{h} \cdot \lim\limits_{h\to 0} g(a)\\\\
			   &=&f(a)\cdot g'(a)+f'(a)\cdot g(a).\\\\
	\end{array}$$
	Observemos que hemos utilizado el hecho de que $\lim\limits_{h\to 0}f(a+h)-f(a)=0\; \Rightarrow \; \lim\limits_{h\to 0}f(a+h)=f(a).$\\\\
\end{teo}

% -------------------- teorema 5
\begin{teo}
    Si $f(x)=cf(x)$ y $f$ es diferenciable en $a$, entonces $g$ es diferenciable en $a$, y
    $$g'(a)=c\cdot f'(a).$$\\
	Demostración.-\; Si $h(x)=c$, de manera que $g=h\cdot f$, entonces,
	$$\begin{array}{rcl}
	    g'(a)&=&(h\cdot f)'(a)\\
		 &=&h(a)\cdot f'(a)+h'(a)\cdot f(a)\\
		 &=&c\cdot f'(a)+0\cdot f(a)\\
		 &=&c\cdot f'(a).\\\\
	\end{array}$$
	En particular, $(-f)'(a)=-f'(a)$, por tanto $(f-g)'(a)=\left(f+[-g]\right)'(a)=f'(a)-g'(a).$\\\\
\end{teo}

% -------------------- teorema 6
\begin{teo}
    Si $f(x)=x^n$ para algún número natural $n$, entonces
    \begin{center}
	$f'(x)=nx^{n-1}$ para todo número $a$.
    \end{center}
    \vspace{.5cm}
    	Demostración.-\; La demostración la haremos por inducción sobre $n$. Para $n=1$ se aplica simplemente el teorema 2. Supongamos ahora que el teorema es cierto para $n$, de manera que si $f(x)=x^n$, entonces 
	$$f'(a)=na^{n-1}\; \mbox{para todo}\; a.$$
	Sea $g(x)=x^{n+1}$. Si $I(x)=x$, la ecuación $x^{n+1}=x^n\cdot x$ se puede escribir como
	\begin{center}
	    $g(x)=f(x)\cdot I(x)$ para todo $x$.
	\end{center}
	así, $g=f\cdot I$. A partir del teorema 4 deducimos que 
	$$\begin{array}{rcl}
	    g'(a)&=&\left(f\cdot I\right)'(a)\\
		 &=&f'(a)\cdot I(a)+f(a)\cdot I'(a)\\
		 &=&na^{n-1}\cdot a + a^n \cdot 1\\
		 &=&na^n+a^n\\
		 &=&(n+1)a^n,\; \mbox{para todo}\; a.\\\\
	\end{array}$$
	Este es precisamente el caso $n+1$ que queríamos demostrar.\\
	Si $f(x)=x^{-n}=\dfrac{1}{x^n}$ para algún número natural $n$, entonces
	$$f'(x)=\dfrac{-nx^{n-1}}{x^{2n}}=(-n)x^{-n-1};$$
	así es válido tanto para enteros positivos como negativos. Si interpretamos $f(x)=x^0$ como $f(x)=1$ y $0\cdot x^{-1}$ como $f'(x)=0$, entonces se verifica también para $n=0$.\\\\
\end{teo}

% -------------------- teorema 7
\begin{teo}
    Si $g$ es diferenciable en $a$ y $g(a)\neq 0,$ entonces $1/g$ es diferenciable en $a$, y 
    $$\left(\dfrac{1}{g}\right)'(a)=\dfrac{-g'(a)}{\left[g(a)\right]^2}.$$\\
	Demostración.-\; Incluso antes de escribir
	$$\dfrac{\left(\dfrac{1}{g}\right)(a+h)-\left(\dfrac{1}{g}\right)(a)}{h}$$
	debemos asegurarnos que esta expresión tiene sentido; es necesario comprobar que $(1/g)(a+h)$ está definido para valores suficientemente pequeños de $h$. Para ello son necesarias solamente dos observaciones. Como $g$ es, por hipótesis, diferenciable en $a$, se deduce del teorema 9-1 que $g$ es continua en $a$. Como $g(a)\neq 0$, deducimos también, a partir del teorema 6-3, que existe un $\delta>0$ tal que $g(a+h)\neq 0$ para $|h|<\delta$. Por tanto, $(1/g)=(a+h)$ tiene sentido para valores de $h$ suficientemente pequeños, y así podemos escribir
	$$\begin{array}{rcl}
	    \lim\limits_{h\to 0} \dfrac{\left(\dfrac{1}{g}\right)(a+h)-\left(\dfrac{1}{g}\right)(a)}{h}&=& \lim\limits_{h\to 0} \dfrac{\dfrac{1}{g(a+h)}-\dfrac{1}{g(a)}}{h}\\\\			
				 &=& \lim\limits_{h\to 0} \dfrac{g(a)-g(a+h)}{h\left[g(a)\cdot g(a+h)\right]}\\\\
				 &=& \lim\limits_{h\to 0} \dfrac{-\left[g(a+h)-g(a)\right]}{h}\cdot \dfrac{1}{g(a)g(a+h)}\\\\
				 &=& \lim\limits_{h\to 0} \dfrac{-\left[g(a+h)-g(a)\right]}{h}\cdot \lim\limits_{h\to 0}\dfrac{1}{g(a)\cdot g(a+h)}\\\\
				 &=&-g'(a)\cdot \dfrac{1}{\left[g(a)\right]^2}.\\\\
	\end{array}$$

\end{teo}

% -------------------- teorema 8
\begin{teo}
    Si $f$ y $g$ son diferenciables en $a$ y $g(a)\neq 0$, entonces $f/g$ es diferenciable en $a$, y
    $$\left(\dfrac{f}{g}\right)'(a)=\dfrac{g(a)\cdot f'(a)-f(a)\cdot g'(a)}{\left[g(a)\right]^2}$$
	Demostración.-\; Como $f/g=f\cdot (1/g)$ obtenemos
	$$\begin{array}{rcl}
	    \left(\dfrac{f}{g}\right)'(a) &=& \left(f\cdot \dfrac{1}{g}\right)'(a)\\\\
					  &=& f'(a)\cdot \left(\dfrac{1}{g}\right)(a)+f(a)\cdot \left(\dfrac{1}{g}\right)'(a)\\\\
					  &=& \dfrac{f'(a)}{g(a)}+\dfrac{f'(a)\cdot g(a)-f(a)\cdot g'(a)}{\left[g(a)\right]^2}\\\\
					  &=& \dfrac{f'(a)\cdot g(a)-f(a)\cdot g'(a)}{\left[g(a)\right]^2}.\\\\
	\end{array}$$
\end{teo}
\vspace{.7cm}

% -------------------- teorema 9
\begin{teo}[Regla de la cadena]
    Si $g$ es diferenciable en $a$ y $f$ es diferenciable en $g(a)$, entonces $f\circ g$ es diferenciable en $a$ y 
    $$\left(f\circ g\right)'(a)=f'\left[g(a)\right]\cdot g'(a).$$\\
	Demostración.-\; Definamos una función $\phi$ de la manera siguiente:
	$$\phi(h)=\left\{\begin{array}{ll}
		\dfrac{f\left[g(a+h)\right]-f\left[g(a)\right]}{g(a+h)-g(a)}, & \mbox{si}\; g(a+h)-g(a)\neq 0\\\\
		f'\left[g(a)\right], & \mbox{si}\; g(a+h)-g(a)=0.\\
	\end{array}\right.$$
	Se intuye fácilmente que $\phi$ es continua en $0$: cuando $h$ es pequeño, $g(a+h)-g(a)$ también es pequeño, de manera que si $g(a+h)-g(a)$ no es $0$, entonces $\phi(h)$ se aproximará a $f'\left[g(a)\right]$; y si es $0$ entonces $\phi(h)$ es igual a $f'\left[g(a)\right]$, lo que es mejor todavía. Ya que la continuidad de $\phi$ es el punto crucial de toda la demostración, vamos a desarrollar rigurosamente este argumento intuitivo.\\
	Sabemos que $f$ es diferenciable en $g(a)$. Esto significa que 
	$$\lim_{k\to 0}\dfrac{f(g(a)+k)-f(g(a))}{k}=f'(g(a)).$$
	Así, si $\epsilon>0$ existe algún número $\delta'>0$ tal que, para todo $k$,
	\begin{center}
	    si $0<|k|<\delta'$, entonces $\bigg|\dfrac{f(g(a)+k)-f(g(a))}{k}-f'(g(a))\bigg|<\epsilon.$
	\end{center}
	Pero $g$ es diferenciable en $a$ y  por lo tanto continua en $a$, de manera que existe un $\delta>0$ tal que para todo $h$,
	\begin{center}
	    si $|h|<\delta$, entonces $|g(a+h)-g(a)|<\delta'$
	\end{center}
	Consideremos ahora cualquier $h$ con $|h|<\delta$. Si $k=g(a+h)-g(a)\neq 0,$ entonces
	$$\phi(h)=\dfrac{f(g(a+h))-f(g(a))}{g(a+h)-g(a)}=\dfrac{f(g(a)+k)-f(g(a))}{k};$$
	se deduce que $|k|<\delta'$, y por tanto deducimos que 
	$$|\phi(h)-f'(h(a))|<\epsilon.$$
	Por otro lado, si $g(a+h)-g(a)=0$, entonces $\phi(g)=f'(g(a))$, de manera que se verifica ciertamente que 
	$$|\phi(h)-f'(g(a))|<\epsilon.$$
	Por tanto hemos demostrado que 
	$$\lim_{h\to 0}\phi(h)=f'(g(a)),$$
	o sea que $\phi$ es continua en $0$. El resto de la demostración es fácil. Si $h\neq 0$, entonces tenemos 
	$$\dfrac{f(g(a+h))-f(g(a))}{h}=\phi(h)\cdot \dfrac{g(a+h)-g(a)}{h}$$
	incluso aunque $g(a+h)-g(a)=0$ (ya que en este caso ambos miembros de la igualdad son iguales a $0$). Por tanto
	$$(f\circ g)'(a)=\lim_{h\to 0}\dfrac{f(g(a+h))-f(g(a))}{h}\lim_{h\to 0}\phi(h)\cdot \lim_{h\to 0}\dfrac{g(a+h)-g(a)}{h}=f'(g(a))\cdot g'(a).$$
\end{teo}

\section{Problemas}
\begin{enumerate}[\bfseries 1.]

    %-------------------- 1.
    \item Como ejercicio de precalentamiento, halle $f'(x)$ para cada una de las siguientes $f$. (No se preocupe por el dominio de $f$ o de $f'$; obtenga tan sólo una fórmula para $f'(x)$ que dé la respuesta correcta cuando tenga sentido.)\\

	\begin{enumerate}[(i)]

	    %-------------------- (i) 
	    \item $f(x)=\sen(x+x^2).$\\\\ 
		Respuesta.-\; $f'(x)=\cos(x+x^2)\cdot (1+2x).$\\\\

	    %-------------------- (ii)
	    \item $f(x)=\sen x + \sen x^2$.\\\\
		Respuesta.-\; $f'(x)=\cos x + \cos(x^2)\cdot 2x.$\\\\

	    %-------------------- (iii)
	    \item $f(x)=\sen(\cos x)$.\\\\
		Respuesta.-\; $f'(x)=\cos(\cos x)\cdot (-\sen x)=-\sen x \cos(\cos x).$\\\\

	    %-------------------- (iv)
	    \item $f(x)=\sen(\sen x)$.\\\\
		Respuesta.-\; $f'(x)=\cos(\sen x)\cdot \cos x = \cos x \cos(\sen x).$\\\\

	    %-------------------- (v)
	    \item $f(x)=\sen\left(\dfrac{\cos x}{x}\right)$.\\\\
		Respuesta.-\; $f'(x)=\cos\left(\dfrac{\cos x}{x}\right)\cdot \dfrac{-x\sen x - \cos x}{x^2}$.\\\\

	    %-------------------- (vi)
	    \item $f(x)=\dfrac{\sen(\cos x)}{x}$.\\\\
		Respuesta.-\; $f'(x)=\dfrac{\cos(\cos x)\cdot (-\sen x)}{x^2}=-\dfrac{\sen x \cos(\cos x)}{x^2}$.\\\\

	    %-------------------- (vii)
	    \item $f(x)=\sen(x+\sen x)$.\\\\
		Respuesta.-\; $f'(x)=\cos(x+\sen x)\cdot (1+\cos x)$.\\\\

	    %-------------------- (viii)
	    \item $f(x)=\sen\left[\cos(\sen x)\right]$.\\\\
		Respuesta.-\; $f'(x)=\cos\left[\cos(\sen x)\right]\left[-\sen(\sen x)\cos x\right]$.\\\\

	\end{enumerate}

    %-------------------- 2.
    \item Halle $f(x)$ para cada una de las siguientes funciones $f.$ (El autor tardó 20 minutos en calcular las derivadas para la sección de soluciones, y al lector no debería costarle mucho más tiempo calcularlas. Aunque la rapidez en los cálculos no es un objetivo de las matemáticas, si se desea tratar con aplomo las aplicaciones teóricas de la Regla de la Cadena, estas aplicaciones concretas deberían ser un juego de niños; a los matemáticos les gusta hacer ver que ni siquiera saben sumar, pero la mayoría pueden hacerlo cuando lo necesitan.)\\

	\begin{enumerate}[(i)]

	    %-------------------- (i)
	    \item $f(x)=\sen \left[(x+1)^2(x+2)\right]$.\\\\
		Respuesta.-\; 
		$$\begin{array}{rcl}
		f'(x)&=&\cos\left[(x+1)^2(x+2)\right]\cdot \left[2(x+1)(x+2)+(x+1)^2\right]\\
		     &=&(x+1)(3x+5)\cos\left[(x+1)^2(x+2)\right].\\
		\end{array}$$
		\vspace{0.7cm}

	    %-------------------- (ii)
	    \item $f(x)=\sen^3\left(x^2+\sen x\right)$.\\\\
		Respuesta.-\; $f'(x)=3\sen^2\left(x^2+\sen x\right)\cdot \cos\left(x^2+\sen x\right)\cdot (2x+\cos x)$.\\\\


	    %-------------------- (iii)
	    \item $f(x)=\sen^2\left[(x+\sen x)^2\right]$.\\\\
		Respuesta.-\; 
		$$\begin{array}{rcl}
		    f'(x)&=&2\sen \left[(x+\sen x)^2 \right] \cos \left[(x+\sen x)^2\right] \cdot 2(x+\sen x)(1+\cos x)\\
			 &=&4(1+\cos x)(x+\sen x)\sen\left[(x+\sen x)^2\right]\cos \left[(x+\sen x)^2\right]\\
		\end{array}$$
		\vspace{0.7cm}

	    %-------------------- (iv)
	    \item $f(x)=\sen\left(\dfrac{x^3}{\cos x^3}\right)$.\\\\
		Respuesta.-\; 
		$$\begin{array}{rcl}
		    f'(x)&=&\cos\left(\dfrac{x^3}{\cos x^3}\right)\cdot \dfrac{3x^2\cos\left(x^3\right)-x^3\left[-\sen\left(x^3\right)\right]3x^2}{\cos^2\left(x^3\right)}\\\\
			 &=&\cos\left(\dfrac{x^3}{\cos x^3}\right)\cdot \dfrac{3x^2\cos\left(x^3\right)+x^3\sen\left(x^3\right)3x^2}{\cos^2\left(x^3\right)}
		\end{array}$$
		\vspace{0.7cm}


	    %-------------------- (v)
	    \item $f(x)=\sen(x\sen x)+\sen\left(\sen x^2\right)$.\\\\
		Respuesta.-\; Aplicando las reglas de derivada se tiene,
		$$f'(x)=\cos(x\sen x)\sen x + x\cos x + \cos\left(\sen x^2\right)\cos\left(x^2\right)2x.$$\\

	    %-------------------- (vi)
	    \item $f(x)=f(x)=(\cos x)^{31^2}$.\\\\
		Respuesta.-\; Aplicando las reglas de derivada se tiene,
		$$f'(x)=-\left(31^2-1\right)(\cos x)^{31^2-1}\sen x.$$\\

	    %-------------------- (vii)
	    \item $f(x)=\sen^2 x \sen x^2\sen^2 x^2$.\\\\
		Respuesta.-\; Aplicando las reglas de derivada se tiene,
		$$\begin{array}{rcl}
		    f'(x)&=&2\sen x \cos x \sen\left(x^2\right)\sen^2\left(x^2\right)+\sen^2 x \left[\cos\left(x^2\right)2x\sen^2\left(x^2\right)\right.\\
			 &+&\left.\sen\left(x^2\right)2\sen\left(x^2\right)\cos\left(x^2\right)2x\right]\\
			 &=&2\sen x \cos x \sen\left(x^2\right)\sen^2\left(x^2\right)+2x\sen^2 x\cos\left(x^2\right)\sen^2\left(x^2\right)\\
			 &+&4x\sen^2 x \sen^2\left(x^2\right)\cos\left(x^2\right).\\
		\end{array}$$
		\vspace{0.7cm}

	    %-------------------- (viii)
	    \item $f(x)=\sen^3\left[\sen^2(\sen x)\right]$.\\\\
		Respuesta.-\; Aplicando las reglas de derivada se tiene,
		    $$f'(x)=3\sen^2\left[\sen^2(\sen x)\right]\cos\left[\sen^2(\sen x)\right]2\sen(\sen x)\cos(\sen x)\cos x.$$\\

	    %-------------------- (ix)
	    \item $f(x)=\left(x+\sen^5 x\right)^6$.\\\\
		Respuesta.-\; Aplicando las reglas de derivada se tiene,
		$$f'(x)=6\left(x+\sen^5 x\right)\left[1+5\sen^4 x \cos x\right].$$\\

	    %-------------------- (x)
	    \item $f(x)=\sen\left[\sen(\sen(\sen(\sen x)))\right]$.\\\\
		Respuesta.-\; Aplicando las reglas de derivada se tiene,
		$$f'(x)=\cos\left(\sen(\sen (\sen x))\right)\cdot \cos (\sen (\sen (\sen x)))\cdot \cos (\sen (\sen x))\cdot \cos (\sen x)\cdot \cos x.$$\\

	    %-------------------- (xi)
	    \item $f(x)=\sen\left[(\sen^7 x^7 + 1)^7\right]$.\\\\
		Respuesta.-\; Aplicando las reglas de derivada se tiene,
		$$f'(x)=\cos\left[\left(\sen^7 x^7 + 1\right)^7\right]\cdot 7\left(\sen^7 x^7 + 1\right)^6\cdot 7\sen^6 x^7 \cdot \cos x^7 \cdot 7x^6.$$\\

	    %-------------------- (xii)
	    \item $f(x)=\left\{\left[\left(x^2+x\right)^3+x\right]^4+x\right\}^5$.\\\\
		Respuesta.-\; Aplicando las reglas de derivada se tiene,
		$$f'(x)=5\left[\left(x^2+x\right)^3+x\right]^4\cdot \left\{1+4\left[\left(x^2+x\right)^3+x\right]^3\left[1+3\left(x^2+x\right)^2\left(1+2x\right)\right]\right\}.$$\\

	    %-------------------- (xiii)
	    \item $f(x)=\sen\left[x^2+\sen\left(x^2+\sen^2 x\right)\right]$.\\\\
		Respuesta.-\; Aplicando las reglas de derivada se tiene,
		$$f'(x)=\cos\left[x^2+\sen\left(x^2+\sen x^2\right)\right]\cdot\left[2x+\cos\left(x^2+\sen x^2\right)\cdot \left(2x+2x\cos x^2\right)\right].$$\\

	    %-------------------- (xiv)
	    \item $f(x)=\sen\left\{6\cos\left[6\sen\left(6\cos 6x\right)\right]\right\}$.\\\\
		Respuesta.-\; Aplicando las reglas de derivada se tiene,
		$$\begin{array}{rcl}
		    f'(x)&=&\cos\left\{6\cos\left[6\sen\left(6\cos 6x\right)\right]\right\}\cdot\left\{ -6\sen\left[6\sen(6\sen 6x)\right]\right\}\cdot6\cos(6\sen 6x)\cdot 6[-\sen(6x)]\cdot 6.\\
		    &=&6^4\cos\left\{6\cos\left[6\sen\left(6\cos 6x\right)\right]\right\}\cdot\sen\left[6\sen(6\sen 6x)\right]\cdot\cos(6\sen 6x)\cdot [-\sen(6x)]\cdot.\\
		\end{array}$$
		\vspace{.7cm}

	    %-------------------- (xv)
	    \item $f(x)=\dfrac{\sen x^2 \sen^2 x}{1+\sen x}$.\\\\
		Respuesta.-\; Aplicando las reglas de derivada se tiene,
		$$\begin{array}{rcl}
		    f'(x)&=&\dfrac{\left[\sen\left(x^2\right)2x\sen^2 x +\sen \left(x^2\right)2\sen x\cos x \right]\cdot (1+\sen x)-\sen \left(x^2\right)\sen^2 x\cos x}{(1+\sen x)^2}\\\\
		    &=&\dfrac{\left[2x\sen\left(x^2\right)\sen^2 x +2\sen \left(x^2\right)\sen x\cos x \right]\cdot (1+\sen x)-\sen \left(x^2\right)\sen^2 x\cos x}{(1+\sen x)^2}\\\\
		\end{array}$$
		\vspace{.7cm}

	    %-------------------- (xvi)
	    \item $f(x)=\dfrac{1}{x-\dfrac{2}{x+\sen x}}$.\\\\

		Respuesta.-\; Aplicando las reglas de derivada se tiene,

		$$f'(x)=\dfrac{-\left[1-\dfrac{-2(1+\cos x)}{(x+\sen x)^2}\right]}{\left(x-\dfrac{2}{x+\sen x}\right)^2}$$\\

	    %-------------------- (xvii)
	    \item $f(x)=\sen\left[\dfrac{x^3}{\sen\left(\dfrac{x^3}{\sen x}\right)}\right]$.\\\\

		Respuesta.-\; Aplicando las reglas de derivada se tiene,

		$$f'(x)=\cos\left[\dfrac{x^3}{\sen\left(\dfrac{x^3}{\sen x}\right)}\right]\cdot \left[  \dfrac{3x^2\cdot \sen\left(\dfrac{x^3}{\sen x}\right)-x^3\cdot \cos \left(\dfrac{x^3}{\sen x}\right) \cdot \dfrac{3x^2\cdot \sen x - x^3\cdot \cos x}{\sen^2 x}}{\sen^2\left( \dfrac{x^3}{\sen x} \right)} \right].$$\\

	    %-------------------- (xviii)
	    \item $f(x)=\sen\left[\dfrac{x}{x-\sen\left(\dfrac{x}{x-\sen x}\right)}\right]$.\\\\

		Respuesta.-\; Aplicando las reglas de derivada se tiene,

	    $$\begin{array}{rcl}
		f'(x)&=&\cos\left[\dfrac{x}{x-\sen\left(\dfrac{x}{x-\sen x}\right)}\right]\cdot \\\\
		     && \dfrac{\left[x-\sen\left(\dfrac{x}{x-\sen x}\right)\right]-x\left[1-\cos\left(\dfrac{x}{x-\sen x}\right)\cdot \dfrac{(x-\sen x)-x(1-\cos x)}{(x-\sen x)^2}\right]}{\left[x-\sen\left(\dfrac{x}{x-\sen x}\right)\right]^2}
	    \end{array}$$
	    \vspace{.7cm}

	\end{enumerate}

    %-------------------- 3.
    \item Halle las derivadas de las funciones $\tan$, $\cotan$, $\sec$, $\cosec$. (No es necesario memorizar estas fórmulas, aunque se necesitarán de vez en cuando; si se expresan las soluciones de manera correcta, resultan sencillas y algo simétricas.)\\\\
	Respuesta.-\; Sea $f(x)=\tan x = \dfrac{\sen x}{\cos x}$, entonces
	$$f'(x)=\dfrac{\cos x \cos x  - \sen (-\sen x)}{\cos^2 x}=\dfrac{\cos^2 x + \sen^2 x}{\cos^2 x}=\dfrac{1}{\cos^2x}=\sec^2 x.$$

	Sea $f(x)=\cotan x = \dfrac{1}{\tan x}=\dfrac{\cos x}{\sen x}$, entonces
	$$f'(x)=\dfrac{-\sen x\sen x - \cos x \cos x}{\sen^2 x}=-\dfrac{1}{\sen^2 x}=\cosec^2 x.$$

	Sea $f(x)=\sec x = \dfrac{1}{\cos x} = \cos^{-1}x$, entonces
	$$f'(x)=-\cos^{-2}x\cdot (-\sen x)=\dfrac{\sen x}{\cos^2 x}=\tan x \sec x.$$

	Sea $f(x)=\cosec = \dfrac{1}{\sen x} = \sen^{-1}x$, entonces
	$$f'(x)=-\sen^{-2}x\cdot \cos x = -\dfrac{\cos x}{\sen^2 x} = -\cosec x \cotan x.$$\\

    %-------------------- 4.
    \item Para cada una de las siguientes funciones $f$, halle $f'(f(x))$ no $(f\circ f)(x)$.\\

	\begin{enumerate}[(i)]

	    %---------- (i)
	    \item $f(x)=\dfrac{1}{1+x}.$\\\\
		Respuesta.-\;  Sea $f'(x)=-\dfrac{1}{(1+x)^2}$, entonces
		$$f'\left(\dfrac{1}{1+x}\right)=-\dfrac{1}{\left(1+\dfrac{1}{1+x}\right)^2}=-\left(\dfrac{1+x}{2+x}\right)^2.$$\\

	    %---------- (ii)
	    \item $f(x)=\sen x.$\\\\
		Respuesta.-\; Se tiene $f'(\sen x) = \cos(x) =\cos(\sen x).$\\\\

	    %---------- (iii)
	    \item $f(x)=x^2$.\\\\
		Respuesta.-\; Se tiene $f'\left(x^2\right)=2x=2x^2.$\\\\

	    %---------- (iv)
	    \item $f(x)=17.$\\\\
		Respuesta.-\; Se tiene $f'(17)=0.$\\\\

	\end{enumerate}

    %-------------------- 5.
    \item Para cada una de las siguientes funciones $f$, halle $f\left[f'(x)\right]$.\\

	\begin{enumerate}[(i)]

	    %---------- (i)
	    \item $f(x)=\dfrac{1}{x}$.\\\\
		Respuesta.-\; Sea $f'(x)=-\dfrac{1}{x^2}$, entonces
		$f\left[f'(x)\right]=f\left(-\dfrac{1}{x^2}\right)=\dfrac{1}{-\dfrac{1}{x^2}}=-x^2.$\\\\

	    %---------- (ii)
	    \item $f(x)=x^2$.\\\\
		Respuesta.-\; Sea $f'(x)=2x$, entonces
		$f'\left(2x\right)=(2x)^2=4x^2.$\\\\

	    %---------- (iii)
	    \item $f(x)=17$.\\\\
		Respuesta.-\; Sea $f'(17)=0$, entonces
		$f'\left(17\right)=17$\\\\

	    %---------- (iv)
	    \item $f(x)=17x$.\\\\
		Respuesta.-\; Sea $f'(17x)=17$, entonces $f'\left(17\right)=17\cdot 17=289$.\\\\

	\end{enumerate}

    %-------------------- 6.
    \item Halle $f'$ en función de $g'$ si

	\begin{enumerate}[(i)]

	    %---------- (i)
	    \item $f(x)=g(x+g(a))$.\\\\
		Respuesta.-\; Por las reglas de derivación tenemos,
		$$f'(x)=g'\left[x+g(a)\right]\cdot \left[x+g(a)\right]'=g'\left[x+g(a)\right].$$\\

	    %---------- (ii)
	    \item $f(x)=g(x\cdot g(a))$.\\\\
		Respuesta.-\; Por las reglas de derivación tenemos,
		$$f'(x)=g'\left[x\cdot g(a)\right]\cdot \left[x\cdot g(a)\right]'=g'\left[x\cdot g(a)\right]\cdot g(a).$$\\

	    %---------- (iii)
	    \item $f(x)=g(x+g(x))$.\\\\
		Respuesta.-\; Por las reglas de derivación tenemos,
		$$f'(x)=g'\left[x+g(x)\right]\left[x+g(x)\right]'=g'\left[x+g(x)\right]\left[1-g'(x)\right].$$\\

	    %---------- (iv)
	    \item $f(x)=g(x)(x-a)$.\\\\
		Respuesta.-\; Por las reglas de derivación tenemos,
		$$f'(x)=g'(x)(x-a)+g'(x).$$\\

	    %---------- (v)
	    \item $f(x)=g(a)(x-a)$.\\\\
		Respuesta.-\; Por las reglas de derivación tenemos,
		$$f'(x)=g'(a)(x-a)+g(a)(x-a)' = g(a).$$\\

	    %---------- (vi)
	    \item $f(x+3)=g\left(x^2\right)$.\\\\
		Respuesta.-\; Sea $z=x+3\; \Rightarrow \; x=z-3$, entonces
		$$f'(z)=g'\left[(z-3)^2\right]\cdot \left[(z-3)^2\right]'=g'\left[(z-3)^2\right](2(z-3)=2g'\left[(z-3)^2\right](z-3).$$\\

	\end{enumerate}

    %-------------------- 7.
    \item 
	\begin{enumerate}[(a)]

	    %---------- (a)
	    \item Un objeto circular va aumentando de tamaño de manera no especificada, pero se sabe que cuando el radio es $6$, la tasa de variación del mismo es $4$. Halle la tasa de variación del área cuando el radio es $6$. (Si $r(t)$ y $A(t)$ representan el radio y el área en el tiempo $t$, entonces las funciones $r$ y $A$ satisfacen $A = \pi r^2$; tan sólo es necesario aplicar directamente la Regla de la Cadena.)\\\\
		Respuesta.-\; Encontrando la primera derivada con respecto de $t$ para se tiene,
		$$A'(t)=2\pi r\cdot r'(t)$$
		Dado que $r'(t)=4$ cuando $r=6$, entonces
		$$A'(6)=2\pi\cdot 6 \cdot 4 = 48\pi.$$\\

	    %---------- (b)
	    \item Suponga que el objeto circular que hemos estado observando es la sección transversal de un objeto esférico. Halle la tasa de variación del volumen cuando el radio es $6$. (Es necesario conocer la fórmula del volumen de una esfera; en caso de que el lector la haya olvidado, el volumen es $\frac{4}{3}\pi$ veces el cubo del radio.)\\\\
		Respuesta.-\; Encontrando la primera derivada con respecto de $t$ para se tiene,
		$$V'(t)=\frac{4}{3}\pi\cdot r^3 \cdot r'(t)$$
		Dado que $r'(t)=4$ cuando $r=6$, entonces
		$$V'(6)=4\pi\cdot 6^2 \cdot 4 = 576\pi.$$\\

	    %---------- (c)
	    \item Suponga ahora que la tasa de variación del área de la sección transversal circular es $5$ cuando el radio es $3$. Halle la tasa de variación del volumen cuando el radio es $3$. Este problema se puede resolver de dos maneras: primero, utilizando las fórmulas del área y el volumen en función del radio; y después expresando el volumen en función del área (para utilizar este método se necesita el Problema 9-3).\\\\
		Respuesta.-\; Sabemos que 
		$$A'(t)=2\pi r r'(t)$$
		y $A'=5$ cuando $r=3$, entonces
		$$5=2\pi 3r'$$
		Luego dividimos ambos lados por $6\pi$, de donde
		$$r'=\dfrac{5}{6\pi}.$$
		Sustituyendo el valor de $r$ y $r'$ nos queda
		$$V'8t)=4\pi r^2 r'(t)\quad \Rightarrow \quad V'=4\pi\cdot 3^2 \cdot \dfrac{5}{6\pi}=30.$$\\

	\end{enumerate}

    %-------------------- 8.
    \item El área entre dos círculos concéntricos variables vale siempre $9\pi\; cm^2$. La tasa de cambio del área del círculo mayor es de $10\pi\; cm^2/seg$. ¿A qué velocidad varía la circunferencia del círculo pequeño cuando su área es de $16\pi cm^2$?.\\\\
	Respuesta.-\; Sea $r_1$ y $r_2$ que representa el radio de de los círculos más pequeños y más grandes respectivamente. El área entre los dos círculos está dada por:
	$$A=\pi\left[(r_2)^2-(r_1)^2\right]$$
	Reemplacemos $A$ con $9\pi$, de donde 
	$$9\pi = \pi\left[(r_2)^2-(r_1)^2\right] \quad \Rightarrow \quad (r_2)^2-(r_1)^2=9$$
	Derivando tenemos,
	$$2r_2r'^2 - 2r_1r'_1 = 0\quad \Rightarrow \quad r_2'=\dfrac{r_1}{r_2}r_1'\qquad (1)$$

	Por otro lado el área del circulo mayor es dado por,
	$$A_2=\pi(r_2)^2$$
	Derivando se tiene,
	$$A_2'=2\pi r_2 r_2'$$

	Dada que la tasa de cambio del área del circulo más grande es $10\pi$, entonces
	$$10\pi = 2\pi r_2 r_2' \quad \Rightarrow \quad r_2'=\dfrac{5}{r_2}\qquad (2)$$
	Igualando (1) y (2),
	$$\dfrac{5}{r_2}=\dfrac{r_1}{r_2}r_1'\quad \Rightarrow \quad r_1'=\dfrac{5}{r_1}.$$\\
	El área del circulo pequeño es dada por,
	$$A_1=\pi(r_1)^2$$
	reemplazando $A_1$ con $16\pi$,
	$$16\pi=\pi(r_1)^2 \quad \Rightarrow \quad r_1=4$$
	Por lo tanto 
	$$r_1'=\dfrac{5}{4}.$$
	Así la circunferencia del circulo pequeño es,
	$$C'=2\pi r_1' \quad \Rightarrow \quad C'=2\pi \dfrac{5}{4} = \dfrac{5}{2}\pi.$$\\

    %-------------------- 9.
    \item Una partícula $A$ se desplaza a lo largo del eje horizontal positivo, y una partícula $B$ a lo largo de la gráfica de $f(x)=-\sqrt{3}x,\; x\leq 0$. En un momento dado, $A$ se encuentra en el punto $(5,0)$ y se desplaza a una velocidad de $3$ unidades del origen y se desplaza a una velocidad de $4$ unidades/seg. ¿Cuál es la tasa de variación de la distancia entre $A$ y $B$?.\\\\
	Respuesta.-\; Sean $x_1$ la coordenada el eje $x$ de $A$ y $x_2$ representa el eje $x$ de $B$ en el momento $t$ y el eje $y$ de $B$ en el momento $t$ es $-\sqrt{3}x_2.$\\
	La distancia $d$ entre $A$ y $B$ puede ser calculado usando el teorema de Pitágoras como,
	$$d^2=(x_1-x_2)^2+(-\sqrt{3}x_2)^2$$
	Derivando con respecto $t$ es,
	$$2d\cdot d' = 2(x_1-x_2)(x_1'-x_2')+2(-\sqrt{3}x_2)x_2' \quad \Rightarrow \quad d\cdot d'=(x_1-x_2)(x_1'-x_2')+(-\sqrt{3}x_2)x_2'\qquad (1)$$
	La distancia entre $B$ y el origen es 
	$$s=\sqrt{x_2^2+(-\sqrt{3}x_2)^2}=2x_2$$
	Derivando con respecto $t$ es,
	$$s'=2x_2'\quad \Rightarrow \quad x_2'=\dfrac{1}{2}s'.$$
	En el momento dado cuando $B$ es $3$ unidades desde el origen se tiene $3=2x_2$. Ya que $x_2\leq 0,$ entonces
	$$x_2=-\dfrac{3}{2}.$$
	Reemplazando $x_1=5$, $x_2=-\dfrac{3}{2},$ $x_1'=3$, $x_2'=\dfrac{1}{2}$, $s'=-2$ y $d=\sqrt{(5+\frac{3}{2})^2+(-\sqrt{3}\frac{3}{2})^2}=7$ en (1) se tiene,
	$$7d'=\left(5+\dfrac{3}{2}\right)(3+2)+\left(-\sqrt{3}\dfrac{-3}{2}\right)(-2)=27.3 \quad \Rightarrow \quad d'=3.9.$$\\


    %-------------------- 10.
    \item Sea $f(x)=x^2\sen 1/x$ para $x\neq 0$ y $f(0)=0$. Supongamos también que $h$ y $k$ son dos funciones tales que 
	$$h'(x)=\sen^2\left[\sen(x+1)\right]\qquad k'(x)=f(x+1)$$
	$$h(0)=3\qquad k(0)=0$$
	Halle

	\begin{enumerate}[(i)]

	    %---------- (i)
	    \item $(f\circ h)'(0)$.\\\\
		Respuesta.-\; Se tiene,
		$$\begin{array}{rcl}
		    (f\circ h)'(0)&=&f'\left[h(0)\right]h'(0) = f'(3)h'(0)\\\\
				  &=& \left[2\cdot 3\sen \dfrac{1}{3}+3^2\cos \dfrac{1}{3}\left(-\dfrac{1}{3^2}\right)\right]\cdot \sen^2\left[\sen(0+1)\right]\\\\
				  &=&\left(6\sen \dfrac{1}{3}-\cos \dfrac{1}{3}\right)\cdot \sen^2\left[\sen(1)\right]\\\\
		\end{array}$$
		\vspace{.7cm}

	    %---------- (ii)
	    \item $(k\circ f)'(0)$.\\\\
		Respuesta.-\; Sea $f(0)=0$, entonces
		$$\begin{array}{rcl}
		    (k\circ f)'(0)&=&k'\left[f(0)\right]\cdot f'(0)\\\\
				  &=&f(0+1)\cdot 0\\\\
				  &=&0\\\\
		\end{array}$$
		\vspace{.7cm}


	    %---------- (iii)
	    \item $\alpha'\left(x^2\right)$, donde $\alpha(x)=h\left(x^2\right)$. Ir con mucho cuidado en la resolución de este apartado.\\\\
		Respuesta.-\; Sea 
		$$\alpha'(x)=\left[h\left(x^2\right)\right]' = h'\left(x^2\right)\cdot \left(x^2\right)' = \sen^2\left[\sen\left(x^2+1\right)\right]\cdot 2x$$
		Así, para $\alpha'\left(x^2\right)$ se tiene,
		$$\alpha'\left(x^2\right)=2x^2\sen^2\left\{\sen\left[\left(x^2\right)^2+1\right]\right\}=2x^2 \sen^2\left[\sen\left(x^4+1\right)\right].$$\\

	\end{enumerate}

    %-------------------- 11.
    \item Halle $f'(0)$ si 
	$$f(x)=\left\{\begin{array}{ll}
		g(x)\sen \dfrac{1}{x},&x\neq 0\\\\
		0,&x=0\\
	\end{array}\right.$$
	 y $$g(0)=g'(0)=0.$$\\
	 Respuesta.-\; Por definición se tiene,
	 $$f'(0)=\lim\limits_{h\to 0}\dfrac{f(h)-f(0)}{h}=\lim\limits_{h\to 0}\dfrac{g(h)\sen \frac{1}{h}-0}{h}=\lim\limits_{h\to 0}\dfrac{g(h)}{h}\sen \dfrac{1}{h}.$$
	 Ya que $\lim\limits_{x \to 0} g(x)=0 \; \Rightarrow \; \lim\limits_{x\to 0}g(x)\sen \dfrac{1}{x}=0$, entonces
	 $$\lim\limits_{h\to 0}\dfrac{g(h)}{h}\sen\dfrac{1}{h}=0=f'(0).$$\\

    %-------------------- 12.
     \item Utilizando la derivada de $f(x)=1/x$, tal como se ha hallado en el problema 9-1, calcule $(1/g)'(x)$ mediante la regla de la cadena.\\\\
	 Respuesta.-\; Sea $\dfrac{1}{g}=f\left[g\right]=f\circ g$,  por la regla de la cadena se tiene,
	 $$\left(f\circ g\right)'(x)=f'\left[g(x)\right]\cdot g'(x)=\dfrac{-1}{\left[g(x)\right]^2}\cdot g'(x)=\dfrac{-g'(x)}{\left[g(x)\right]^2}.$$\\

    %-------------------- 13.
     \item 
	 \begin{enumerate}[(a)]

	     %---------- (a)
	     \item Aplicando el problema 9-3, halle $f'(x)$ para $-1<x<1,$ si $f(x)=\sqrt{1-x^2}$.\\\\
		 Respuesta.-\; Sabiendo que $f(x)=\sqrt{x}\;\Rightarrow \; f'(a)=\dfrac{1}{2\sqrt{a}}$ y la regla de la cadena se tiene,
		 $$f'(x)=\dfrac{1}{2}(1-x^2)^{-1/2}(-2x)=\dfrac{-x}{\sqrt{1-x^2}}.$$\\

	     %---------- (b)
	     \item Demuestre que la tangente a la gráfica de $f$ en $\left(a,\sqrt{1-a^2}\right)$ corta a la gráfica solamente en este punto (y así demuestre que la definición geométrica de tangente coincide con la nuestra).\\\\
		 Demostración.-\; La pendiente de la tangente a $\left(a,\sqrt{1-a^2}\right)$ es,
		 $$f'(a)=\dfrac{-a}{\sqrt{1-a^2}}$$
		 Después, la ecuación de la tangente viene dado por,
		 $$y=mx+c=\dfrac{-a}{\sqrt{1-a^2}}x+c$$
		 Luego ya que la tangente pasa a través de los puntos $\left(a,\sqrt{1-a^2}\right)$, entonces
		 $$\sqrt{1-a^2}=\dfrac{-a}{\sqrt{1-a^2}}a+c\quad \Rightarrow \quad c=\sqrt{1-a^2}+\dfrac{a^2}{\sqrt{1-a^2}}=\dfrac{1}{\sqrt{1-a^2}},$$
		 de donde la tangente se convertirá en,
		 $$y=\dfrac{-a}{\sqrt{1-a^2}}x+\dfrac{1}{\sqrt{1-a^2}}$$
		 Por último sea $f(x)=\sqrt{1-a^2}=y$, entonces
		 $$\begin{array}{rcl}
		     \sqrt{1-x^2}=\dfrac{-a}{\sqrt{1-a^2}}x+\dfrac{1}{\sqrt{1-a^2}}& \Rightarrow & \sqrt{\left(1-a^2\right)\left(1-x^2\right)}=-ax+1\\\\
										   & \Rightarrow & 1-a^2-x^2+a^2x^2=a^2x^2-2ax+1\\\\
										   &=&x^2-2ax+a^2=0\\\\
										   &=&x=a.\\
		 \end{array}$$
		 Por lo tanto la curva y la tangente se cortan en un sólo punto $\left(a,\sqrt{1-a^2}\right)$\\\\

	 \end{enumerate}

    %-------------------- 14.
     \item Demuestre análogamente que las tangentes a una elipse o a una hipérbola cortan a las gráficas correspondientes solamente una vez.\\\\
	 Demostración.-\; La ecuación general de la elipse es,
	 $$\dfrac{x^2}{a^2}+\dfrac{y^2}{b^2}=1 \quad \Rightarrow \quad y=\pm b \sqrt{1-\dfrac{x^2}{a^2}} \quad \Rightarrow \quad y= b \sqrt{1-\dfrac{x^2}{a^2}}.\qquad (1)$$
	 Tomando la derivada de la elipse se tiene,
	 $$y'=\dfrac{-\dfrac{b2x}{a^2}}{2\sqrt{1-\dfrac{x^2}{a^2}}}=\dfrac{-bx}{a^2\sqrt{1-\dfrac{x^2}{a^2}}}$$
	 por lo que la pendiente de la tangente será con respecto de $k$ estará dada por,
	 $$\dfrac{-bk}{a^2\sqrt{1-\dfrac{k^2}{a^2}}}$$
	 Por otro lado, la ecuación de la tangente es,
	 $$y=\dfrac{-bk}{a^2\sqrt{1-\dfrac{k^2}{a^2}}}x+c$$
	 Reemplazando $x$ por $k$ e $y$ con $b\sqrt{1-\dfrac{k^2}{a^2}}$,
	 $$b\sqrt{1-\dfrac{k^2}{a^2}}=\dfrac{-bk}{a^2\sqrt{1-\dfrac{k^2}{a^2}}}k+c,$$
	 de donde 
	 $$c=b\sqrt{1-\dfrac{k^2}{a^2}}+\dfrac{bk^2}{a^2\sqrt{1-\dfrac{k^2}{a^2}}}=\dfrac{a^2 b\left(\sqrt{ 1-\dfrac{k^2}{a^2}}\right)^2 - bk^2}{a^2 \sqrt{ 1- \dfrac{k^2}{a^2}} }=\dfrac{b}{\sqrt{1-\dfrac{k^2}{a^2}}}$$
	 Por lo tanto la ecuación de la tangente viene dada por,
	 $$y=\dfrac{-bk}{a^2\sqrt{1-\dfrac{k^2}{a^2}}}x+\dfrac{b}{\sqrt{1-\dfrac{k^2}{a^2}}}\qquad (2)$$
	 De (1) y (2), se tiene,
	 $$b\sqrt{1-\dfrac{x^2}{a^2}}=\dfrac{-bk}{a^2\sqrt{1-\dfrac{k^2}{a^2}}} x + \dfrac{b}{\sqrt{1-\dfrac{k^2}{a^2}}}\quad \Rightarrow \quad x^2-2kx+k^2=0\; \Rightarrow \; (x-k)^2=0\; \Rightarrow \; x=k.$$\\

    %-------------------- 15.
     \item Si $f+g$ es diferenciable en $a$, ¿son $f$ y $g$ necesariamente diferenciables en $a$? Si $f\cdot g$ y $f$ son diferenciables en $a$, ¿qué condiciones debe cumplir $f$ para que $g$ sea diferenciable en $a$?.\\\\
	 Respuesta.-\; Supongamos que $f$ no es diferenciable en ninguna parte. Sea $g=-f$, entonces  
	 $$(f+g)(x)=f(x)-f(x)=0.$$
	 Esta función cero, es diferenciable en cualquier parte.\\\\
	 Por otro lado, supongamos que $f\cdot g$ y $f$ son diferenciables en $a$. Por el teorema 8 (Si $f$ y $g$ son diferenciables en $a$ y $f(a)\neq 0$, entonces $f/g$ es diferenciable en $a$), la condición que debe cumplir $f$ para que $g$ sea diferenciable en $a$ será, 
	 $$g=\dfrac{f\cdot g}{f}.$$\\

    %-------------------- 16.
     \item
	 \begin{enumerate}[(a)]

	     %---------- (a)
	     \item Demuestre que si $f$ es diferenciable en $a$, entonces $|f|$ también es diferenciable en $a$, si $f(a)\neq 0.$\\\\
		 Demostración.-\; Al ser $f$ derivable en $a$ es continua en $a$. Luego al ser $f(a)\neq 0$, se sigue que $f(x)\neq 0$ para todos los $x$ de un intervalo entorno de $a$. Así pues, $f=|f|$ o $-f=|f|$ en este intervalo, con lo que $|f|'(a)=f'(a)$ o $|f|'(a)=-f'(a)$. Se puede hacer uso también de la regla de la cadena y del hecho que si $f(x)=\sqrt{x}\; \Rightarrow \; f'(a)=\dfrac{1}{2\sqrt{a}}$. Sea $|f|=\sqrt{f^2}$ con lo que 
		 $$|f|'(x)=\dfrac{1}{2\sqrt{f(x)^2}} \cdot 2f(x)f'(x)=f'(x)\cdot \dfrac{f(x)}{|f(x)|}.$$\\

	     %---------- (b)
	     \item Dé un contraejemplo si $f(a)=0$.\\\\
		 Respuesta.-\; Sea $f(x)=x$, entonces $f(0)=0$ y $|f|(x)=|x|.$ De donde sabemos que $|f|$ no es diferenciable en $0$.\\\\ 

	     %---------- (c)
	     \item Demuestre que si $f$ y $g$ son diferenciables en $a$, entonces las funciones $\max(f,g)$ y $\min(f,g)$ son diferenciables en $a$, si $f(a)\neq g(a)$.\\\\
		 Demostración.-\; Sea $r>0$ para cualquier $x\in (a-r,a+r)$ y sea $f(x)>g(x)$, dados por $f(x)=\max{f,g}(x)=f(x)$ y $\min{f,g}(x)=g(x).$ entonces por definición de difernciabilidad existen $f'(a)$ y $g'(a)$ tal que
		 $$f'(a)=\lim_{h\to 0}\dfrac{f(a+h)-f(a)}{h}\quad \mbox{y}\quad g'(a)=\lim_{h\to 0}\dfrac{g(a+h)-g(a)}{h}.$$
		 Luego por definición de límites se tiene que para todo $\epsilon>0$ existe algún $\delta_1>0$ de donde
		 $$\bigg|\dfrac{f(a+h)-f(a)}{h}-f'(a)\bigg|<\epsilon\; \mbox{siempre que}|h|<\delta_1$$
		 y sea $\epsilon>0$ con $\delta_2>0$,
		 $$\bigg|\dfrac{g(a+h)-g(a)}{h}-g'(a)\bigg|<\epsilon\; \mbox{siempre que}|h|<\delta_2$$
		 Pongamos a $\delta'=\dfrac{\min{\delta_1,\delta_2,r}}{2}$, entonces para cualquier $x\in\left(a-\delta',a+\delta'\right)$, tenemos $\max{f,g}(x)=f(x)$ y $\min{f-g}(x)=g(x)$ de la siguiente forma,
		 $$\bigg|\dfrac{\max{f,g}(a+h)-\max{f,g}(a)}{h}-f'(a)\bigg|<\epsilon\; \mbox{siempre que }|h|<\delta'$$
		 y
		 $$\bigg|\dfrac{\min{f,g}(a+h)-\min{f,g}(a)}{h}-g'(a)\bigg|<\epsilon\; \mbox{siempre que }|h|<\delta'.$$
		 Pero ya que $\max{f,g}(a+h)=f(a+h)$ y $\min{f,g}(a+h)=g(a+h)$ para $|h|<\delta'$, entonces $\max{f,g}$ y $\min{f,g}$ son diferenciables en $a$.\\\\  

	     %---------- (d)
	     \item Dé un contraejemplo si $f(a)=g(a)$.\\\\
		 Respuesta.-\; Sean $f(x)=x$ y $g(x)=0$ de donde $f(0)=g(0)$, entonces
		 $$\max{f,g}(x)=\left\{\begin{array}{rcl}
			 x & \mbox{si} & x>0\\
			 0 & \mbox{si} & x\leq 0,\\
		 \end{array}\right. \quad \mbox{y}\quad 
		    \min{f,g}(x)=\left\{\begin{array}{rcl}
			 0 & \mbox{si} & x\geq 0\\
			 x & \mbox{si} & x< 0.\\
		    \end{array}\right.
		 $$
		 Derivando la parte derecha $\max{f,g}(x)=x$ para $0$ se tiene $1$, la derivada $\max{f,g}(x)=0$ para $0$ es $0$. Con respecto a la parte izquierda se tiene las derivada de $\min{f,g}(x)=0$ y $\min{f,g}(x)=x$ cuando tiende a $0$ como $0$ y $1$ respectivamente. Por lo que se demuestra que $\max{f,g}$ y $\min{f,g}$ no son diferenciales en $a=0$.\\\\

	 \end{enumerate}

    %------------------- 17
     \item De un ejemplo de funciones $f$ y $g$ tales que $g$ toma todos los valores, y $f\circ g$ y $g$ son diferenciables, pero $f$ no es diferenciable. ((El problema es trivial si no se exige que $g$ tome todos los valores; en este caso $g$ podría ser una función constante, o una función que sólo tomara valores de un intervalo $(a,b)$, en cuyo caso el comportamiento de $f$ fuera de $(a,b)$ sería irrelevante.).\\\\
	 Respuesta.-\; Sea $f(x)=\sqrt[3]{x^2}$ no diferenciable en $x=0$. Y sea $g(x)=x^3$ diferenciable en $x=0$, entonces
	 $$(f\circ g)(x)=f(g)(x)=f\left(x^3\right)= \sqrt[3]{\left(x^3\right)^2} = \sqrt[3]{x^6}=x^2.$$
	     donde $(f\circ g)(x)$ es diferenciable en $x=0.$\\\\

    %-------------------- 18
     \item 
	 \begin{enumerate}[(a)]

	     %---------- (a)
	     \item Si $g=f^2$ halle una fórmula para $g'$ (que incluya a $f'$).\\\\
		 Respuesta.- Aplicando la regla de la cadena, tenemos
		 $$g'=2f\cdot f'.$$\\

	     %---------- (b)
	     \item Si $g=\left(f'\right)^2$, halle una fórmula para $g'$ (que incluya a $f''$).\\\\
		 Respuesta.-  Aplicando la regla de la cadena, tenemos
		 $$g'=2f'\cdot f''.$$\\

	     %---------- (c)
	     \item Suponga que la función $f>0$ verifica que 
	     $$\left(f'\right)^2=f+\dfrac{1}{f^3}.$$
	     Halle una fórmula para $f''$ en función de $f$. (En este apartado, además de cálculos sencillos, es necesario tener cuidado.)\\\\
		 Respuesta.- Aplicando la regla de la cadena a los dos lados, tenemos
		 $$2f'\cdot f''=f'+\dfrac{-3}{f^4}\dot f' \quad \Rightarrow \quad f''=\dfrac{1}{2}-\dfrac{3}{2f^4}.$$\\
		 
	 \end{enumerate}

    %-------------------- 19
     \item Si $f$ es tres veces diferenciable y $f'(x)\neq 0$, la \textbf{derivada de Schwarz} de $f$ en $x$ se define mediante 
	 $$\mathscr{D}f(x)=\dfrac{f'''(x)}{f'(x)}-\dfrac{3}{2}\left(\dfrac{f''(x)}{f'(x)}\right)^2.$$

	 \begin{enumerate}[(a)]

	     %---------- (a)
	     \item Demuestre que $\mathscr{D}(f\circ g)=\left[\mathscr{D}f\circ g\right]\cdot g^{'^2}\mathscr{D}g.$\\\\
		 Demostración.-\; Primeramente calculemos $(f\cdot g)'$, $(f\cdot g)''$ y $(f\cdot g)'''$.
		 $$\begin{array}{rcl}
		     (f\circ g)'(x)&=&f'\left[g(x)\right]\cdot g'(x).\\\\
		     (f\circ g)''(x)&=&\left\{f'\left[g(x)\right]\cdot g'(x)\right\}'\\
				    &=&f''\left[g(x)\right]g'(x)^2+f'\left[g(x)\right]g''(x)\\\\
		     (f\circ g)'''(x)&=&\left\{f''\left[g(x)\right]g'(x)^2+f'\left[g(x)\right]g''(x)\right\}\\
				     &=&f'''\left[g(x)\right]g'(x)^3 + 2f''\left[g(x)\right]g''(x)g'(x)+f''\left[g(x)\right]g''(x)g'(x)+f'\left[g(x)\right]g'''(x)\\
				     &=&f'''\left[g(x)\right]g'(x)^3+3f''\left[g(x)\right]g''(x)g'(x)+f'\left[g(x)\right]g'''(x).\\
		 \end{array}$$

		 Por último calculamos la derivada de Schwarz para $f\circ g$,
		 $$\begin{array}{rcl}
		     \mathscr{D}(f\circ g)(x)&=&\dfrac{(f\circ g)'''(x)}{(f\circ g)'(x)}-\dfrac{3}{2}\left[\dfrac{(f\circ g)''(x)}{(f\circ g)'(x)}\right]^2\\\\
					     &=&\dfrac{f'''\left[g(x)\right]g'(x)^3}{f'\left[g(x)\right]\cdot g'(x)}+\dfrac{2f''\left[g(x)\right]g''(x)g'(x)}{f'\left[g(x)\right]\cdot g'(x)}+\dfrac{f'\left[g(x)\right]g'''(x)}{f'\left[g(x)\right]\cdot g'(x)}\\\\
					     &-&\dfrac{3}{2}\left\{\dfrac{f''\left[g(x)\right]g'(x)^2}{f'\left[g(x)\right]\cdot g'(x)}+\dfrac{f'\left[g(x)\right]g''(x)}{f'\left[g(x)\right]\cdot g'(x)}\right\}^2\\\\
					     &=&\dfrac{f'''\left[g(x)\right]g'(x)^2}{f'\left[g(x)\right]}+\dfrac{2f''\left[g(x)\right]g''(x)}{f'\left[g(x)\right]}+\dfrac{g'''(x)}{g'(x)}-\dfrac{3}{2}\left[\dfrac{f''\left[g(x)\right]g'(x)}{f'\left[g(x)\right]}+\dfrac{g''(x)}{g'(x)}\right]^2\\\\
					     &=&\dfrac{f'''\left[g(x)\right]g'(x)^2}{f'\left[g(x)\right]}+\dfrac{2f''\left[g(x)\right]g''(x)}{f'\left[g(x)\right]}+\dfrac{g'''(x)}{g'(x)}\\\\
					     &-&\dfrac{2}{3}\left\{\dfrac{f''\left[g(x)\right]g'(x)}{f'\left[g(x)\right]}\right\}^2-3\dfrac{f''\left[g(x)\right]g''(x)}{f'\left[g(x)\right]}-\dfrac{3}{2}\left[\dfrac{g''(x)}{g'(x)}\right]^2\\\\
					     &=&\left[\dfrac{f'''}{f'}\circ g(x)-\dfrac{3}{2}\dfrac{(f''\circ g)(x)}{(f'\circ g)(x)}\right]\cdot g'(x)^2+\dfrac{g'''(x)}{g'(x)}-\dfrac{3}{2}\dfrac{g''(x)}{g'(x)}\\\\
					     &=&\left[\mathscr{D}f\circ g(x)\right]\cdot g'(x)^2+\mathscr{D}g(x).\\\\
		 \end{array}$$
		 \vspace{.5cm}

	     %---------- (b)
	     \item Demuestre que si $f(x)=\dfrac{ax+b}{cx+d},$ con $ad-bc\neq 0$, entonces $\mathscr{D}f=0.$ Por consiguiente, $\mathscr{D}(f\circ g)=\mathscr{D}g.$\\\\
		 Demostración.-\; Usando la regla de la cadena de Leibniz se tiene,
		 $$\begin{array}{rcl}
		     f'(x)&=&\dfrac{a(cx+d)-(ax+b)c}{(cx+d)^2} = \dfrac{ad-bc}{(cx+d)^2}\\\\
		     f''(x)&=&-\dfrac{2c(ad-bc)}{(cx+d)^3}\\\\
		     f'''(x)&=&\dfrac{6c^2(ad-bc)}{(cx+d)^4}\\\\
		 \end{array}$$
		 Luego utilizamos la definición de la derivada de Schwarz, de la siguiente manera:
		 $$\begin{array}{rcl}
		     \mathscr{D}f(x)&=&\dfrac{f'''(x)}{f'(x)}-\dfrac{3}{2}\left[\dfrac{f''(x)}{f'(x)}\right]^2=\dfrac{\dfrac{6c^2(ad-bc)}{(cx+d)^4}}{\dfrac{ad-bc}{(cx+d)^2}}-\dfrac{3}{2}\left[-\dfrac{\dfrac{2c(ad-bc)}{(cx-d)^3}}{\dfrac{ad-bc}{(cx+d)^2}}\right]^2\\\\
				    &=&\dfrac{6c^2}{(cx+d)^2}-\dfrac{3}{2}\left(-\dfrac{2c}{cx+d}\right)^2=\dfrac{6c^2}{(cx+d)^2}-\dfrac{6c^2}{(cx+d)^2}=0.\\\\
		 \end{array}$$
		 Sea $\left[\mathscr{D}f\circ g(x)\right]\cdot g'(x)^2+\mathscr{D}g(x)$, entonces $\mathscr{D}(f\circ g)=\mathscr{D}g.$\\\\

	 \end{enumerate}

    %-------------------- 20.
     \item Suponga que existen $f^{(n)}(a)$ y $g^{(n)}(a)$. Demuestre la \textbf{fórmula de Leibniz}:
	 $$(f\cdot g)^{(n)}(a)=\sum_{k=0}^n {n\choose k}f^{(k)}(a)\cdot g^{(n-k)}(a).$$\\
	 Demostración.-\; Demostraremos por inducción matemática. Sea $n=1$, entonces 
	 $$(f\cdot g)'(a)=\sum_{k=0}^1 {1\choose k}f^{(k)}(a)\cdot g^{(1-k)}(a)= {1\choose 0} f(a)\cdot g'(a) + {1\choose 1}f'(a)\cdot g(a)=f'(a)g(a)+f(a)g'(a).$$
	 El cual se cumple para $n=1$. Luego la hipótesis de inducción estará dada por,
	 $$(f\cdot g)^{(n)}(a)=\sum_{k=0}^n {n\choose k}f^{(k)}(a)\cdot g^{(n-k)}(a),$$
	 que es cierta para $n$ y por no tanto $f^{(n+1)}(a)$ y $g^{(n+1)}(a)$ existen. Así,
	 $$\begin{array}{rcl}
	     (f\cdot g)^{(n+1)}(a)&=&(f\cdot g)^{(n)}(a)(f\cdot g)'(a)\\\\
				  &=&\displaystyle\left[\sum_{k=0}^n {n\choose k}f^{(k)}(a)\cdot g^{(n-k)}(a)\right]\left[f'(a)g(a)+f(a)g'(a)\right]\\\\
				  &=&\displaystyle\sum_{k=0}^n{n \choose k}\left[f^{(k+1)}(a)g^{(n-k)}(a)+f^{(k)}g^{(n+1-k)}(a)\right]\\\\
				  &=&\displaystyle\sum_{k=1}^{n+1}{n\choose k-1}f^{(k)}(a) g^{(n+1-k)}(a)+\sum_{k=0}^n {n\choose k}f^{(k)}(a)g^{(n+1-k)}(a)\\\\
	 \end{array}$$	
	 Ya que $\displaystyle{n+1\choose k}={n\choose k-1}+{n\choose k}$. Entonces,
	 $$(f\cdot g)^{(n+1)}(a)=\sum_{k=0}^{n+1}{n+1\choose k}f^{(k)}(a)g^{[(n+1)-k]}(a).$$\\

    %-------------------- 21.
     \item Demuestre que si $f^{(n)}\left[g(a)\right]$ y $g^{(n)}(a)$ existen ambas, entonces también existe $(f\circ g)^{(n)}(a)$. Con un poco de práctica el lector debería convencerse que no es sensato tratar de encontrar una fórmula para $(f \circ g)^{(n)}(a)$. Para demostrar que $(f \circ g)^{(n)}(a)$ existe, es necesario, por tanto, encontrar una proposición razonable acerca de $(f \circ g)^{(n)}(a)$ que pueda ser demostrada por inducción. Se puede intentar algo como: existe $(f \circ g)^{(n)}(a)$ y es una suma de términos, cada uno de los cuales es un producto de términos de la forma $\ldots$.\\\\ 
	 Demostración.-\; La fórmulas,
	 $$\begin{array}{rcl}
	     (f\circ g)'(x) &=& f'\left[g(x)\right]\cdot g'(x)\\
	     (f\circ g)''(x) &=& f''\left[g(x)\right] \cdot g'(x) + f'\left[g(x)\right]\cdot g''(x)\\
	     (f\circ g)'''(x) &=& f'''\left[g(x)\right]\cdot g'(x)^3 + 3f''\left[g(x)\right]\cdot g'(x)g''(x)+f'\left[g(x)\right]g'''(x),\\
	 \end{array}$$
	 Llevan a la siguiente conjetura: Si $f^{(n)}\left[g(a)\right]$ y $g^{(a)}(a)$ existen, entonces también existe $(f\circ g)^{(n)}(a)$ y es una suma de términos de la forma
	 $$c\cdot \left[g'(a)\right]^{m_1}\cdots \left[g^{(n)}(a)\right]^{m_n}\cdot f^{(k)}\left[g(a)\right],$$
	 para algún número $c$, enteros no negativos $m_1,\ldots , m_n$ y un número natural $k\leq n$. Para probar esta proposición utilizaremos el método de inducción, de donde notamos que es verdadero para $n=1$ con $a=m_1=k=1$. Ahora supóngase que para un cierto $n$, es cierto para todo número $a$ tal que $f^{(n)}\left[g(a)\right]$ y $g^{(n)}(a)$ existan. Supóngase también que $f^{(n+1)}\left[g(a)\right]$ y $g^{(n+1)}(a)$ existen. Entonces $g^{(k)}(x)$ podría existe para todo $k\leq n$ y todo $x$ en algún intervalo alrededor de $a$, y $f^{(k)}(y)$  debe existe para todo $k\leq n$ y todo $y$ en algún intervalo alrededor de $g(a)$. Ya que $g$ es continua en $a$, esto implica que $f^{(k)}\left[g(x)\right]$ exista para todo $x$ en algún intervalo alrededor de $a$. Así la proposición es verdadera para todo $x$, esto es, $(f\}circ g)^{(n)}$ es una suma de términos de la forma:
	 $$c\left[g'(x)\right]^{m_1}\cdots \left[g^{(n)}(x)\right]^{m_n}\cdot f^{(k)}\left[g(a)\right],\quad m_1,\ldots, m_n \geq 0,\quad 1\leq k \leq n.$$
	 Como consecuencia, $(f\circ g)^{(n+1)}(a)$ es una suma de términos de la forma
	 $$c\cdot m_\alpha \left[g'(x)\right]^{m_1}\cdot \ldots \cdot\left[g^{(n)}(x)\right]^{m_\alpha - 1}\cdot \ldots \cdot\left[g^{(n)}(a)\right]^{m_n}\cdot f^{(k)}\left[g(a)\right]\qquad m_\alpha>0$$
	 o de la forma
	 $$c\left[g'(x)\right]^{m_1+1} \cdot \ldots \cdot\left[g^{(n)}(x)\right]^{m_n}\cdot f^{(k+1)}\left[g(a)\right].$$
	 Para un número $c$.\\\\

    %-------------------- 22.
     \item 
	 \begin{enumerate}[(a)]

	     %---------- (a)
	     \item Si $f(x)=a_n x^n + a_{n-1} x^{n-1}+\ldots + a_0,$ halle una función $g$ tal que $g'=f$. Encuentre otra.\\\\
		 Respuesta.-\; Sea 
		 $$g(x)=\dfrac{a_n}{n+1}x^{n+1}+\dfrac{a_{n-1}}{n}x^n+\ldots + \dfrac{a_1}{2}x^2+a_0 x.$$
		 Derivando $g$ tenemos,
		 $$\begin{array}{rcl}
		     g'(x)&=&\dfrac{a_n}{n+1}(n+1)x^n + \dfrac{a_{n-1}}{n}\cdot n x^{n-1}+\ldots + \dfrac{a_1}{2}2x+a_0\\\\
			  &=& a_nx^n+a_{n-1}x^{n-1}+\ldots + a_0\\\\
			  &=&f(x).\\\\
		\end{array}$$
		Otro $g_1$ tal que $g_1'=f$ sería,
		$$g_1(x)=g(x)+c.$$\\


	     %---------- (b)
	     \item Si $$f(x)=\dfrac{b_2}{x^2}+\dfrac{b_3}{x^3}+\ldots + \dfrac{b_m}{x^m}$$
	     halle una función $g$ que verifique $g'=f.$\\\\
		 Respuesta.-\; Sea
		 $$g(x)=-\left[\dfrac{b_2}{x}+\dfrac{b_3}{2x^2}+\ldots + \dfrac{b_m}{(m-1)x^{m-1}}\right].$$
		 Derivando $g$ tenemos,
		 $$\begin{array}{rcl}
		     g'(x)&=&-\left[-\dfrac{b_2}{x^2}-\dfrac{2b_3 x}{2x^4}+\ldots + \dfrac{b_m(m-1)x^{m-2}}{(m-1)x^{2(m-1)}}\right]\\\\
			  &=&\dfrac{b_2}{x^2}+\dfrac{b_3}{x^3}+\ldots + \dfrac{b_m}{x^m}\\\\
			  &=&f(x).\\\\
		 \end{array}$$
		 \vspace{.5cm}


	     %---------- (c)
	     \item ¿Existe una función 
	     $$f(x)=a_nx^n + \ldots + a_0 + \dfrac{b_1}{x}+\ldots + \dfrac{b_m}{x^m}$$
	     tal que $f'(x)=1/x$?.\\\\
		 Respuesta.-\; No, ya que la derivada de $f$ es
		 $$f'(x)=na_n x^{n-1}+\ldots + a_1 - \dfrac{b_1}{x^2}-\dfrac{2b_2}{x_3}-\ldots - \dfrac{mb_m}{x^{m+1}}.$$\\

	 \end{enumerate}

    %-------------------- 23.
     \item Demuestre que existe una función polinómica $f$ de grado $n$ tal que\\
	 \begin{enumerate}[(a)]

	     %---------- (a)
	     \item $f'(x)=0$ para exactamente $n-1$ números $x$.\\\\
		 Demostración.-\; Ya que para todo $n$ existe una función polinómica de grado $n$. Es decir, existe $g$ una función polinómica de grado $n-1$ con $n-1$ raíces. Entonces para $f(x)=a_n x^n + a_{n-1} x^{n-1}+\ldots + a_0,$ de grado $n$ existe una función $g$ tal que $g=f'$. Esto por el problema 7(b) capítulo 3 Spivak y por el problema 22 (a) capítulo 10, Spivak.\\\\


	     %---------- (b)
	     \item $f'(x)=0$ para ningún $x$, si $n$ es impar.\\\\
		 Demostración.-\; Sea $n$ impar que implica $n-1$ es par. Si $g$ es una función polinómica de grado $n-1$ sin raíces (Capítulo 3, problema 7, Spivak). Entonces, existe un polinomio $f$ de grado $n$ tal que $f'=g$ (Capítulo 10, problema 22(a), Spivak). Por lo tanto $g$ no tiene raíces, así $f'(0)=0$ no tiene raíces.\\\\ 

	    %---------- (c)
	     \item $f'(x)=0$ para exactamente un $x$, si $n$ es par.\\\\
		 Demostración.-\; Sea $n$ par que implica $n-1$ par. Por el capítulo 3, problema 7, Spivak, existe un polinomio $g$ de grado $n-1$ con exactamente una raíz. Luego por el capítulo 10, problema 22(a), spivak. Existe una función polinómica $f$ de grado $n$ tal que $f'=g$. Por lo tanto $g$ tiene una sola raíz, así $f'(x)=0$ tiene exactamente una raíz.\\\\

	    %---------- (d)
	     \item $f'(x)=0$ para exactamente $k$ números $x$, si $n-k$ es impar.\\\\
		 Demostración.-\; Sea $n-k$ impar que implica $n-k-1$ es par. Por el capítulo 3, problema 7, Spivak, existe un polinomio $g$ de grado $n-k-1$ con exactamente $k$ raíces. Luego por el capítulo 10, problema 22(a), spivak, existe una función polinómica $f$ de grado $n$ tal que $f'=g$. Por lo tanto $g$ tiene $k$ raíces, así $f'(0)=0$ para exactamente $k$ números $x$.\\\\

	 \end{enumerate}

    %-------------------- 24.
     \item 
	 \begin{enumerate}[(a)]

	     %---------- (a)
	     \item El número $a$ se denomina una raíz doble de la función polinómica $f$ si $f(x)=(x-a)^2g(x)$ para alguna función polinómica $g$. Demuestre que $a$ es una raíz doble de $f$ si y sólo si $a$ es una raíz de $f$ y de $f'$.\\\\
		 Demostración.-\; Primero demostremos que $f=f'=0$. Para $f$ tenemos,
		 $$f(a)=(a-a)^2g(x)=0.$$
		 Luego por la regla de la cadena para $f'$ obtenemos,
		 $$f'(a)=2(a-a)g(a)+(a-a)^2g'(a)=0.$$
		 Por lo tanto $a$ es una raíz doble de $f$.\\

		 Demostramos ahora que si $f(a)=f'(a)=0$, entonces, $a$ es una raíz doble de $f$ si 
		 $$f(x)=(x-a)^2g(x)$$ 
		 para alguna función polinómica $g$. Ya que $f$ es un polinomio y $f(a)=0$, entonces por la división polinomial, existe una función polinómica $g_1$ tal que 
		 \begin{align} \tag{1}
		 f(x)=(x-a)g_1(x).
		 \end{align}
		 Luego calculamos la derivada de esta función,
		 $$f'(x)=g(x)+(x-a)g'(x),$$
		 Se sigue $f'(a)=0$ implica que $g_1(a)=0.$ Luego, por el hecho de que $g_1$ es un polinomio, podemos utilizar la división polinomial para construir otro polinomio, como sigue:
		 $$g_1(x)=(x-a)g_2(x).$$
		 Luego reemplazamos en (1), de donde
		 $$f(x)=(x-a)^2g_1(x)g_2(x).$$
		 Ya que $g_1$ y $g_2$ son polinomios, ponemos  $g=g_1\cdot g_2$, Por lo tanto
		 $$f(x)(x-a)^2g(x).$$\\

	    %---------- (b)
	     \item ¿Cuándo tiene $f(x)=ax^2+bx+c\; (a\neq 0)$ una raíz doble? ¿Cuál es la interpretación geométrica de esta condición?.\\\\
		 Respuesta.-\; Sea $f(x)=ax^2+bx+c$ una función polinómica con $a\neq 0$ e $y$ una raíz doble de $f$. Entonces por la parte (a) debemos encontrar $f(y)=f'(y)=0$ como sigue:
		 $$f'(y)=2ay+b=0 \quad \Rightarrow \quad y=-\dfrac{b}{2a},\quad a\neq 0.$$
		 Por otro lado, ya que $f(y)$ también es cero, la condición que se requerirá será:
		 $$a\cdot \dfrac{b^2}{4a^2}-\dfrac{b^2}{2a}+c=0 \quad \Rightarrow \quad 4ac-b^2=0$$
		 Geométrica, significa que la gráfica de $f$ toca al eje horizontal en el punto $-\dfrac{b}{2a}.$\\\\

	 \end{enumerate}

    %-------------------- 25.
     \item Si $f$ es diferenciable en $a$, sea $d(x)=f(x)-f'(a)(x-a)-f(a).$ Halle $d'(a).$\\\\
	 Respuesta.-\; Podemos ver que 
	 $$d'(a)=f'(a)-f'(a)(a-a)-f'(a)=f'(a)-f'(a)=0.$$\\

    %-------------------- 26.
     \item Este problema es parecido al problema 3-6 (spivak). Sean $a_1,\ldots , a_n$ y $b_1,\ldots, b_n$ números dados.\\
	 \begin{enumerate}[(a)]

	     %---------- (a)
	     \item Si $x_1,\ldots , x_n$ son números distintos, demuestre que existe una función polinómica $f$ de grado $2n-1$, tal que $f\left(x_j\right)=f'\left(x_j\right)=0$ para $j\neq i$, y $f(x_i)=a_i$ y $f'(x_i)=b_i$.\\\\
		 Demostración.-\; 

	     %---------- (b)
	     \item Demuestre que existe una función polinómica $f$ de grado $2n-1$ con $f(x_i)=a_i$ y $f'(x_i)=b_i$ para todo $i$.\\\\
		 Demostración.-\;
	    
	 \end{enumerate}

\end{enumerate}


    %---------- Significado de la derivada
	\chapter{Significado de la derivada}

    \begin{def.}
	Sea $f$ una función y $A$ un conjunto de números contenido en el dominio de $f$. Un punto $x$ de $A$ es un punto máximo de $f$ en $A$ si
	$$f(x)\geq f(y)\qquad \mbox{para todo} \; y \; \mbox{de}\; A.$$
	El número $f(x)$ se denomina el \textbf{valor máximo} de $f$ en $A$ (y también diremos que $f$ alcanza su valor máximo en el punto $x$ de $A$).\\\\
	$f$ tiene un \textbf{mínimo} en el punto $x$ de $A$ si $-f$ tiene un máximo en el punto $x$ de $A$.
    \end{def.}

En general, nos interesará el caso en que $A$ es un intervalo cerrado $[a,b];$ si $f$ es continua, entonces el Teorema 7-3 garantiza que $f$ alcanza realmente dicho valor máximo en $[a,b].$\\

Ahora ya estamos en condiciones para enunciar un teorema que ni siquiera depende de la existencia de cotas superiores mínimas.\\

\begin{teo}
    Sea $f$ cualquier función definida en $(a,b)$. Si $x$ es un punto máximo (o mínimo) de $f$ en $(a,b)$ y $f$ es diferenciable en $x$, entonces $f'(x)=0.$ (Observemos que no hemos supuesto la diferenciabilidad, ni siquiera la continuidad, de $f$ en otros puntos.)\\\\
	Demostración.-\; Consideremos el caso en que $f$ tiene un máximo en $x$. Si $h$ es cualquier número tal que $x+h$ pertenece a $(a,b)$, entonces
	$$f(x)\geq f(x+h),$$
	ya que $f$ tiene un máximo en el punto $x$ de $(a,b)$. Esto significa que 
	$$f(x+h)-f(x)\leq 0.$$
	De manera que, si $h>0$ tenemos
	$$\dfrac{f(x+h)-f(x)}{h}\leq 0,$$
	y por tanto
	$$\lim_{h\to 0^+}\dfrac{f(x+h)-f(x)}{h}\leq 0.$$
	Por otra parte, si $h<0$, tenemos
	$$\dfrac{f(x+h)-f(x)}{h}\geq 0,$$
	o sea
	$$\lim_{h\to 0^-}\dfrac{f(x+h)-f(x)}{h}\geq 0.$$
	Por hipótesis, $f$ es diferenciable en $x$, de manera que ambos límites deben ser iguales (de hecho son iguales a $f'(x)$). Esto significa que
	$$f'(x)\leq 0\quad \mbox{y}\quad f'(x)\geq 0,$$
	de lo cual se deduce que $f'(x)=0.$\\
\end{teo}

    \begin{def.}
	Sea $f$ una función, y $A$ un conjunto de números contenido en el dominio de $f$. Un punto $x$ de $A$ es un \textbf{punto máximo [mínimo] local} de $f$ en $A$ si existe algún $\delta>0$ tal que $x$ es un punto máximo [mínimo] de $f$ en $A\cap (x-\delta,x+\delta.)$
    \end{def.}

\begin{teo}
    Si $x$ es un máximo o mínimo local de $f$ en $(a,b)$ y $f$ es diferenciable en $x$, entonces $f'(x)=0$.\\\\
	Demostración.-\; Se trata de una aplicación del teorema 1 (capítulo 11, Spivak).
\end{teo}

El recíproco del teorema 2 no es cierto; la condición $f'(0)$ no implica que $x$ sea un punto máximo o mínimo local en $f$. Precisamente por esta razón, se ha adoptado una terminología especial para describir a aquellos números $x$ que satisfacen la condición $f'(0)$.

    \begin{def.}
	Un \textbf{punto critico} de una función $f$ es un número $x$ tal que 
	$$f'(x)=0.$$
	Al número $f(x)$ se le denomina \textbf{valor critico} de $f$.
    \end{def.}

Consideremos en primer lugar el problema de hallar el máximo o el mínimo de $f$ en un intervalo cerrado $[a,b]$. (En este caso, si $f$ es continua, sabemos que dicho valor máximo y mínimo debe existir.) Para localizarlos, deben considerarse tres clases de puntos:

\begin{enumerate}[(1)]
    \item Los puntos críticos de $f$ en $[a,b]$.
    \item Los puntos extremos $a$ y $b$.
    \item Aquellos puntos $x$ de $[a,b]$ tales que $f$ no es diferenciable en $x$.
\end{enumerate}

Si $x$ no pertenece al segundo no al tercer grupo entonces forzosamente debe pertenecer al primero.\\


\begin{obs}
    En el capítulo 7 ya resolvimos el problema de este tipo cuando demostramos que si $n$ es par, la función
    $$f(x)=x^n+a_{n-1}x^{n-1}+\ldots + a_0$$
    tiene un valor mínimo en toda la recta real. Dicho valor mínimo se puede encontrar resolviendo la ecuación, si es posible, y comparando los valores de $f(x)$ en dichos $x$.\\
\end{obs}

\begin{teo}[Teorema de Rolle]
    Si $f$ es continua en $[a,b]$ y diferenciable en $(a,b)$, y $f(a)=f(b)$, entonces existe un número $x$ en $(a,b)$ tal que $f'(x)=0$.\\\\
	Demostración.-\; A partir de la continuidad en $f$ en $[a,b]$ deducimos que $f$ tiene valor máximo y mínimo en $[a,b]$. Supongamos primero que el valor máximo se presenta en un punto $x$ de $(a,b)$. Entonces $f'(x)=0$ según el teorema 1, y la demostración queda completa. Supongamos ahora que el valor mínimo de $f$ se presenta en algún punto $x$ de $(a,b)$. Entonces, de nuevo $f'(x)=0$ según el teorema 1. Finalmente, supongamos que los valores máximo y mínimo se presentan ambos en los extremos del intervalo. Como $f(a)=f(b),$ dichos valores coinciden, de manera que $f$ es una función constante, y en este caso se puede elegir cualquier valor $x$ de $(a,b)$.
\end{teo}

\begin{teo}[Teorema del valor medio] \hypertarget{def1}
    Si $f$ es continua en $[a,b]$ y diferenciable en $(a,b)$, existe un número $x$ en $(a,b)$ tal que 
    $$f'(x)=\dfrac{f(b)-f(a)}{b-a}.$$\\
	Demostración.-\; Sea 
	$$h(x)=f(x)-\left[\dfrac{f(b)-f(a)}{b-a}\right](x-a).$$
	Evidentemente, $h$ es continua en $[a,b]$ y diferenciable en $(a,b)$, y 
	$$h(a)=f(a),\qquad h(b)=f(b)-\left[\dfrac{f(b)-f(a)}{b-a}\right](b-a)=f(a).$$
	Por tanto, se puede aplicar el teorema de Rolle a la función $h$ y deducir que existe algún $x$ en $(a,b)$ tal que
	$$0=h'(x)=f'(x)-\dfrac{f(b)-f(a)}{b-a},$$
	de modo que 
	$$f'(x)=\dfrac{f(b)-f(a)}{b-a}.$$\\
\end{teo}

\begin{cor}
    Si $f$ está definida en un intervalo y $f'(x)=0$ en todo $x$ del intervalo, entonces $f$ es constante en dicho intervalo.\\\\
	Demostración.-\; Sean $a$ y $b$ dos puntos del intervalo con $a\neq b$. Entonces existe algún $x$ de $(a,b)$ tal que 
	$$f'(x)=\dfrac{f(b)-f(a)}{b-a}.$$
	Pero $f'(x)=0$ para todo $x$ del intervalo, por tanto
	$$0=\dfrac{f(b)-f(a)}{b-a},$$
	y por consiguiente $f(a)=f(b)$. Así pues, el valor de $f$ en dos puntos cualesquiera del intervalo es el mismo, lo cual significa que $f$ es constante en el intervalo.\\\\
\end{cor}

\begin{cor}
    Si $f$ y $g$ están definidas en el mismo intervalo y $f'(x)=g'(x)$ para todo $x$ del intervalo, entonces existe algún número $c$ tal que $f=g+c.$\\\\
	Demostración.-\; Para todo $x$ del intervalo se verifica que $(f-g)'(x)=f'(x)-g'(x)=0,$ de manera que, según el corolario 1, existe un número $c$ tal que $f-g=c.$
\end{cor}

    \begin{def.}
	Una función es \textbf{creciente} en un intervalo si $f(a)<f(b)$ siendo $a$ y $b$ dos números del intervalo con $a<b$. La función $f$ es \textbf{decreciente} en un intervalo si $f(a)>f(b)$ para todo $a$ y $b$ del intervalo con $a<b$. (A menudo se dice simplemente que $f$ es creciente o decreciente, en cuyo caso se deduce que el intervalo es el dominio de $f$.) 
    \end{def.}

\begin{cor}
    Si $f'(x)>0$ para todo $x$ de un intervalo, entonces $f$ es creciente en dicho intervalo; si $f'(x)<0$ para todo $x$ del intervalo, entonces $f$ es decreciente en dicho intervalo.\\\\
	Demostración.-\; Consideremos el caso en que $f'(x)>0$. Sean $a$ y $b$ dos puntos del intervalo con $a<b$. Entonces existe algún punto $x$ en $(a,b)$ que verifica
	$$f'(x)=\dfrac{f(b)-f(a)}{b-a}.$$
	Pero $f'(x)>0$ para todo $x$ en $(a,b)$, por tanto 
	$$\dfrac{f(b)-f(a)}{b-a}>0.$$
	Como $b-a>0$ se deduce que $f(b)>f(a).$\\\\
	Consideremos ahora el caso en que $f'(x)<0$. Sean $a$ y $b$ dos puntos del intervalo con $a<b$. Entonces existe algún punto $x$ en $(a,b)$ que verifica
	$$f'(x)=\dfrac{f(b)-f(a)}{b-a}.$$
	Pero $f'(x)<0$ para todo $x$ en $(a,b)$, por tanto 
	$$\dfrac{f(b)-f(a)}{b-a}<0.$$
	De donde se deduce que $f(b)<f(a).$\\
\end{cor}

Podemos dar un esquema general para decidir si un punto crítico es un máximo local, un mínimo local o ninguna de las dos cosas:

\begin{enumerate}[(1)]
    \item Si $f'>0$ en algún intervalo a la izquierda de $x$ y $f'<0$ en algún intervalo a la derecha de $x$, entonces $x$ es un punto máximo local.
    \item  Si $f'<0$ en algún intervalo a la izquierda de $x$ y $f'>0$ en algún intervalo a la derecha de $x$, entonces $x$ es un punto mínimo local.
    \item Si $f'$ tiene el mismo signo en algún intervalo a la izquierda de $x$ que en algún intervalo a la derecha, entonces $x$ no es ningún punto máximo ni mínimo local.
\end{enumerate}

En varios problemas de este capítulo y de capítulos sucesivos se pide hacer una representación gráfica de funciones. En cada caso debe determinar

\begin{enumerate}[(1)]
    \item los puntos críticos de $f$,
    \item el valor de $f$ en los puntos críticos,
    \item el signo de $f'$ en las regiones entre los puntos críticos (si esto no está claro ya),
    \item los números $x$ tales que $f(x)=0$ (si es posible),
    \item el comportamiento de $f(x)$ cuando $x$ se hace grande o grande negativo (si es posible).
\end{enumerate}

Existe un criterio popular para hallar los máximos y mínimos locales, que depende del comportamiento de la función sólo en los puntos críticos.\\

%------------------- teorema 5 ----------------------------
\begin{teo}
    Supongamos que $f'(a)=0.$ Si $f''(a)>0,$ entonces $f$ tiene un mínimo local en $a$; si $f''(a)<0,$ entonces $f$ tiene un máximo local en $a.$\\\\
	Demostración.-\; Por definición,
	$$f''(a)=\lim_{h\to 0}\dfrac{f'(a+h)-f'(a)}{h}.$$
	Como $f'(a)=0$, esta igualdad puede escribirse como
	$$f''(a)=\lim_{h\to 0}\dfrac{f'(a+h)}{h}.$$
	Supongamos ahora que $f''(a)>0$. Entonces $\dfrac{f'(a+h)}{h}$ ha de ser positivo para valores suficientemente pequeños de $h$. Por tanto:
	\begin{center}
	    $f'(a+h)$ ha de ser positivo para valores de $h>0$ suficientemente pequeños. Y
	    $f'(a+h)$ ha de ser negativo para valores de $h<0$ suficientemente pequeños.
	\end{center}
	Esto significa por el corolario 3 (Spivak, capitulo 11) que $f$ es creciente en algún intervalo a la derecha de $a$ y $f$ es decreciente en algún intervalo a la izquierda de $a$. Por consiguiente, $f$ tiene un mínimo local en $a$. La demostración es análoga en el caso de que $f''(a)<0.$\\
\end{teo}

Aunque el Teorema 5 es muy útil en el caso de funciones polinómicas, para muchas otras funciones la segunda derivada es tan complicada que es más fácil considerar el signo de la primera derivada. Además, si a es un punto crítico de $f$ puede ocurrir que $f''(a)=0$. En este caso, el Teorema 5 no proporciona información: es posible que a sea un punto máximo local, un mínimo local o ninguna de las dos cosas.\\

%-------------------- teorema 6 -----------------------------
\begin{teo}
    Supongamos que $f''(a)$ existe. Si $f$ tiene mínimo local en $a$, entonces $f''(a)\geq 0$; si $f$ tiene un máximo local en $a$, entonces $f''(a)\leq 0.$\\\\
	Demostración.-\; Supongamos que $f$ tiene un mínimo local en $a$. Si $f''(a)<0$, entonces $f$ tendría también un máximo local en $a$, por el teorema $5$. Es decir, $f$ sería constante en algún intervalo que contiene a $a$, y por tanto $f''(a)=0$, lo cual es una contradicción. Por tanto, debe verificarse que $f''(a)\geq 0$. El caso de un máximo loca se trata de manera análoga.\\
\end{teo}

%-------------------- teorema 7. -----------------------------
\begin{teo}
    Supongamos que $f$ es continua en $a$ y que $f'(x)$ existe para todo $x$ de algún intervalo que contiene a $a$, excepto quizás en $x=a$. Supongamos, además, que $\lim\limits_{x\to 0}f'(x)$ existe. Entonces $f'(a)$ también existe y 
    $$f'(a)=\lim\limits_{x\to a}f'(x).$$\\
	Demostración.-\; Por definición,
	$$f'(a)=\lim\limits_{h\to 0}\dfrac{f(a+h)+f(a)}{h}.$$
	Para valores de $h>0$ suficientemente pequeños, la función $f$ es continua en $[a,a+h]$ y diferenciables en $(a,a+h)$ (lo mismo ocurre para valores de $h<0$ suficientemente pequeños). Según el teorema del valor medio, existe un número $\alpha_h$ en $(a,a+h)$ tal que 
	$$\dfrac{f(a+h)-f(a)}{h}=f'(\alpha_h).$$
	Además $\alpha_h$ tiende a $a$ cuando $h$ tiende a $0$, ya que $\alpha_h$ pertenece al intervalo $(a,a+h)$; como $\lim\limits_{x\to a}f'(x)$ existe, se deduce que 
	$$f'(a)=\lim_{h\to 0}\dfrac{f(a+h)-f(a)}{h}=\lim_{h\to 0}f'(\alpha_h)=\lim_{x\to a}f'(x).$$
	En otras palabras, sean $L=\lim\limits_{x\to a}f'(x)$, por definición de límite se tiene 
	\begin{center}
	    Para todo $\epsilon>0$, existe un $\delta>0$ tal que, si $0<|x-a|<\delta$, entonces $|f'(x)-L|<\epsilon.$
	\end{center}
	Ahora, si $0<|x-a|<\delta$ podríamos utilizar el teorema del valor medio para encontrar un punto $c$ entre $a$ y $x$ que satisfaga,
	$$\dfrac{f(x)-f(a)}{x-a}=f'(c).$$
	Notemos que $c$ satisface también a $0<|c-a|<\delta,$ tal que $|f'(c)-L|<\epsilon.$ Como consecuencia 
	$$\left|\dfrac{f(x)-f(a)}{x-a}-L\right|<\epsilon.$$ 
	Es decir,
	$$0<|x-a|<\delta \quad \Rightarrow \quad \left|\dfrac{f(x)-f(a)}{x-a}-L\right|<\epsilon.$$
	Por lo tanto, $$f'(a)=L.$$\\
\end{teo}

Incluso si $f$ es una función diferenciable en todo punto, es posible que $f'$ sea discontinua. Por ejemplo
$$f(x)=\left\{\begin{array}{ll}
    x^2\sen\dfrac{1}{x},&x\neq 0\\\\
    0,&x=0\\
\end{array}\right.$$

Ahora veremos una generalización del teorema del valor medio.\\\\

%-------------------- teorema 8. -----------------------------
\begin{teo}[Torema del valor medio de Cauchy]
    Si $f$ y $g$ son continuas en $[a,b]$ y diferenciables en $(a,b)$, entonces existe un número $x$ en $(a,b)$ tal que 
    $$[f(b)-f(a)]g'(x)=[g(b)-g(a)]f'(x).$$
    (Si $g(b)\neq g(a)$ y $g'(x)\neq 0$, esta ecuación puede escribirse como
    $$\dfrac{f(b)-f(a)}{g(b)-g(a)}=\dfrac{f'(x)}{g'(x)}.$$
    Observemos que si $g(x)=x$ para todo $x$, entonces $g'(x)=1$, y se obtiene el teorema del valor medio. Por otra parte aplicando el Teorema del valor medio a $f$ y a $g$ por separado, se deduce que existen un $x$ e $y$ en $(a,b)$ que verifican
    $$\dfrac{f(b)-f(a)}{g(b)-g(a)}=\dfrac{f'(x)}{g'(y)};$$
    pero no existe ninguna garantía de que los $x$ e $y$ hallados de esta manera sean iguales.
    Estas consideraciones pueden hacer pensar que el Teorema del Valor Medio de Cauchy es muy difícil de demostrar, pero en realidad basta aplicar uno de los artilugios más simples.)\\\\
	Demostración.-\; Sea 
	$$h(x)=f(x)\left[g(b)-g(a)\right]-g(x)\left[f(b)-f(a)\right].$$
	Entonces $h$ es continua en $[a,b]$, diferenciable en $(a,b)$ y 
	$$h(a)=f(a)g(b)-g(a)f(b)=h(b).$$
	Según el teorema de Rolle, $h'(x)=0$ para algún $x$ de $(a,b)$, lo que significa que 
	$$0=f'(x)\left[g(b)-g(a)\right]-f'(x)\left[f(b)-f(a)\right].$$
\end{teo}

El Teorema del Valor Medio de Cauchy es la herramienta básica necesaria para demostrar un teorema que facilita el cálculo de límites de la forma
$$\lim_{x\to a}\dfrac{f(x)}{g(x)}.$$
cuando
$$\lim_{x\to a}f(x)=0\quad \mbox{y}\quad \lim_{x\to a}g(x)=0.$$
En este caso no es aplicable el Teorema 5.2.\\\\

%-------------------- teorema 9. -----------------------------
\begin{teo}[Regla de L'Hopital]
    Supongamos que 
    $$\lim_{x\to a}f(x)=0 \quad \mbox{y}\quad \lim_{x\to a}g(x)=0,$$
    y supongamos también que $\lim\limits_{x\to a}f'(x)/g'(x)$ existe. Entonces $\lim\limits_{x\to a}f(x)/g(x)$ existe y 
    $$\lim_{x\to a}\dfrac{f(x)}{g(x)}=\lim_{x\to a}\dfrac{f'(x)}{g'(x)}.$$
    (Observe que el teorema 7 es un caso particular.)\\\\
	Demostración.-\; La hipótesis de que el $\lim\limits_{x\to a}f'(x)/g'(x)$ existe contiene dos suposiciones implícitas:
	\begin{enumerate}[(1)]
	    \item existe un intervalo $(a-\delta,a+\delta)$ tal que $f'(x)$ y $g'(x)$ existen para todo $x$ de $(a-\delta,a+\delta)$ excepto quizás para $x=a$,
	    \item en este intervalo $g'(x)\neq 0$ excepto quizás, de nuevo, en $x=a.$
	\end{enumerate}
\end{teo}
Por otra parte, no se supone ni siquiera que $f$ y $g$ estén definidas en el punto $a$. Si definimos $f(a)=g(a)=0$ (cambiando, si es necesario, los valores previos de $f(a)$ y $g(a)$), entonces $f$ y $g$ son continuas en el punto $a$. Si $a<x<a+\delta$, puede aplicarse a $f$ y a $g$ el Teorema del Valor Medio y el Teorema del Valor Medio de Cauchy en el intervalo $[a,x]$ (y lo mismo ocurre en el caso de que $a-\delta < x < a$). Aplicando en primer lugar el Teorema del Valor Medio a $g$, vemos que $g(x)\neq 0$, ya que si $f(x)=0$ entonces existiría algún $x_1$ en $(a,x)$ tal que $g'(x_1)=0$, lo que contradice (2). Aplicando ahora el Teorema del Valor Medio de Cauchy a $f$ y a $g$, vemos que existe un número $\alpha_x$ en $(a,x)$ tal que 
$$\left[f(x)-0\right]g'(\alpha_x)=\left[g(x)-0\right]f'(\alpha_x)$$
o
$$\dfrac{f(x)}{g(x)}=\dfrac{f'(\alpha_x)}{g'(\alpha_x)}.$$
Pero $\alpha_x$ se aproxima a $a$ cuando $x$ se aproxima a $a$, ya que $\alpha_x$ está en el intervalo $(a,x)$; como estamos suponiendo que $\lim\limits_{y\to a}f'(y)/g'(y)$ existe, se deduce que
$$\lim_{x\to a}\dfrac{f(x)}{g(x)}=\lim_{x\to a}\dfrac{f'(\alpha_x)}{g'(\alpha_x)}=\lim_{y\to a}\dfrac{f'(y)}{g'(y)}.$$


\section{Ejercicios}

\begin{enumerate}[\bfseries 1.]

    %-------------------- 1.
    \item Para cada una de las siguientes funciones, halle los valores máximos y mínimos en los intervalos indicados, determinando aquellos puntos del intervalo en los que la derivada es igual a $0$ y comparando los valores de la función en estos puntos con sus valores en los extremos del intervalo.
	\begin{enumerate}[(i)]

	    %---------- (i)
	    \item $f(x)=x^3-x^2-8x+1$ en $[-2,2]$.\\\\
		Respuesta.-\; Primeramente derivemos la función $f$.
		$$f'(x)=3x^2-2x-8.$$
		Luego igualemos a cero para hallar el grupo de candidatos para localizar el o los puntos máximos y mínimos.
		$$3x^2-2x-8=0 \quad \Rightarrow \quad x_1=2,\quad x_2=-\dfrac{4}{3}$$
		Ambos número $x_1=2$ y $x_2=-\dfrac{4}{3}$ pertenecen al intervalo $[-2,2]$, de manera que el primer grupo de candidatos para localizar el máximo y el mínimo es
		$$x_1=2,\qquad x_2=-\dfrac{4}{3}$$
		El segundo grupo incluye a los extremos del intervalo. Es decir,
		$$-2,2$$
		El tercer grupo es vacío, ya que $f$ es diferenciable en todas partes. Por último calculamos 
		$$\begin{array}{ccccl}
		    f\left(2\right) &=& 2^3-2^2-8\cdot 2+1&=&-11.\\\\
		    f\left(-\dfrac{4}{3}\right) &=& \left(-\dfrac{4}{3}\right)^3-\left(-\dfrac{4}{3}\right)^2-8\cdot \left(-\dfrac{4}{3}\right)+1&=&\dfrac{203}{27}.\\\\
		    f\left(-2\right) &=& -2^3-2^2-8\cdot (-2)+1&=&5.\\\\
		\end{array}$$
		Por lo tanto el mínimo viene dado por $-11$ y el máximo viene dado por $\dfrac{203}{27}.$\\\\

	    %---------- (ii)
	    \item $f(x)=x^5+x+1$ en $[-1,1]$.\\\\
		Respuesta.-\; 
		Respuesta.-\; Primeramente derivemos la función $f$.
		$$f'(x)=5x^4+1.$$
		Luego igualemos a cero para hallar el grupo de candidatos para localizar el o los puntos máximos y mínimos.
		$$5x^4+1=0 \quad \Rightarrow \quad x^4=-\dfrac{1}{5}.$$
		El cual no es posible para ningún $x$ real.\\\\
		El segundo grupo incluye a los extremos del intervalo. Es decir,
		$$-1,1$$
		El tercer grupo es vacío, ya que $f$ es diferenciable en todas partes. Por último calculamos 
		$$\begin{array}{ccccl}
		    f\left(1\right) &=& 1^5+1+1&=&3.\\\\
		    f\left(-1\right) &=& (-1)^5-1+1 &=&-1.\\\\
		\end{array}$$
		Por lo tanto el mínimo viene dado por $-1$ y el máximo viene dado por $3.$\\\\

	    %---------- (iii)
	    \item $f(x)=3x^4-8x^3+6x^2$ en $\left[-\frac{1}{2},\frac{1}{2}.\right]$\\\\
		Respuesta.-\; Primeramente derivemos la función $f$.
		$$f'(x)=4x^3-24x^2+12x.$$
		Luego igualemos a cero para hallar el grupo de candidatos para localizar el o los puntos máximos y mínimos.
		$$12x^3-24x^2+12x=0 \quad \Rightarrow \quad x(x^2-2x+1)=0\quad \Rightarrow \quad \left\{\begin{array}{rcl}x_1&=&0\\ x_2 &=& 1. \end{array}\right.$$
		Sólo el número $x_1=0$  pertenece al intervalo $[-\frac{1}{2},\frac{1}{2}]$, de manera que el primer grupo de candidatos para localizar el máximo y el mínimo es sólo el número:
		$$x_1=0.$$
		El segundo grupo incluye a los extremos del intervalo. Es decir,
		$$-\dfrac{1}{2},\;\dfrac{1}{2}.$$
		El tercer grupo es vacío, ya que $f$ es diferenciable en todas partes. Por último calculamos 
		$$\begin{array}{ccccl}
		    f\left(-\dfrac{1}{2}\right) &=& 3\left(-\dfrac{1}{2}\right)^4-8\left(-\dfrac{1}{2}\right)^3+ 6\left(-\dfrac{1}{2}\right)^2 &=&\dfrac{43}{16}.\\\\
		    f\left(0\right) &=& 3\cdot 0^4-8\cdot 0^3 + 6\cdot 0^2&=&0.\\\\
		    f\left(\dfrac{1}{2}\right) &=& 3\left(\dfrac{1}{2}\right)^4-8\left(\dfrac{1}{2}\right)^3+ 6\left(\dfrac{1}{2}\right)^2 &=&\dfrac{11}{16}.\\\\
		\end{array}$$
		Por lo tanto el mínimo viene dado por $0$ y el máximo viene dado por $\dfrac{43}{16}.$\\\\

	    %---------- (iv)
	    \item $f(x)=\dfrac{1}{x^5+x+1}$ en $\left[-\dfrac{1}{2},1\right]$.\\\\
		Respuesta.-\; Primeramente derivemos la función $f$.
		$$f'(x)=5x^4+1.$$
		Luego igualemos a cero para hallar el grupo de candidatos para localizar el o los puntos máximos y mínimos.
		$$35x^4+1=0 \quad \Rightarrow \quad x_4=-\dfrac{1}{5}.$$
		El cual no es posible para ningún $x$ real.\\\\
		El segundo grupo incluye a los extremos del intervalo. Es decir,
		$$-\dfrac{1}{2},\;1$$
		El tercer grupo es vacío, ya que $f$ es diferenciable en todas partes. Por último calculamos 
		$$\begin{array}{ccccl}
		    f\left(-\dfrac{1}{2}\right) &=& \dfrac{1}{\left(-\dfrac{1}{2}\right)^5+\left(-\dfrac{1}{2}\right)+1} &=&\dfrac{32}{15}.\\\\
		    f\left(1\right) &=& \dfrac{1}{1^5+1+1} &=&\dfrac{1}{3}.\\\\
		\end{array}$$
		Por lo tanto el mínimo viene dado por $\dfrac{32}{15}$ y el máximo viene dado por $\dfrac{1}{3}.$\\\\

	    %---------- (v)
	    \item $f(x)=\dfrac{x+1}{x^2+1}$ en $\left[-1,\dfrac{1}{2}\right]$.\\\\
		Respuesta.-\; Primeramente derivemos la función $f$.
		$$f'(x)=\dfrac{x^2+1-2(x+1)}{\left(x^2+1\right)^2}.$$
		Luego igualemos a cero para hallar el grupo de candidatos para localizar el o los puntos máximos y mínimos.
		$$\dfrac{x^2+1-2(x+1)}{\left(x^2+1\right)^2}=0 \quad \Rightarrow \quad x^2-2x-1=0\quad \Rightarrow \quad \left\{\begin{array}{rcl}x_1&=&-1+\sqrt{2}\\ x_2 &=& -1-\sqrt{2}. \end{array}\right.$$
		Sólo el número $x_1=-1+\sqrt{2}$ pertenece al intervalo $[-1,\frac{1}{2}]$, de manera que el primer grupo de candidatos para localizar el máximo y el mínimo es sólo el número:
		$$x_1=-1+\sqrt{2}.$$
		El segundo grupo incluye a los extremos del intervalo. Es decir,
		$$-1,\;\dfrac{1}{2}.$$
		El tercer grupo es vacío, ya que $f$ es diferenciable en todas partes. Por último calculamos 
		$$\begin{array}{ccccl}
		    f\left(-1\right) &=& \dfrac{-1+1}{\left(-1\right)^2+1}&=&0.\\\\
		    f\left(1+\sqrt{2}\right) &=& \dfrac{\left(-1+\sqrt{2}\right)+1}{\left(-1+\sqrt{2}\right)^2+1} &=&\dfrac{\sqrt{2}}{2\left(\sqrt{2}+2\right)}.\\\\
		    f\left(\dfrac{1}{2}\right) &=& \dfrac{\left(\dfrac{1}{2}\right)+1}{\left(\dfrac{1}{2}\right)^2+1} &=&\dfrac{6}{5}.\\\\
		\end{array}$$
		Por lo tanto el mínimo viene dado por $0$ y el máximo viene dado por $\dfrac{6}{5}.$\\\\

	    %---------- (vi)
	    \item $f(x)=\dfrac{x}{x^2-1}$ en $[0,5]$.\\\\
		Respuesta.-\; Primeramente derivemos la función $f$.
		$$f'(x)=-\dfrac{x^2+1}{\left(x^2-1\right)^2}$$
		Luego igualemos a cero para hallar el grupo de candidatos para localizar el o los puntos máximos y mínimos.
		$$-\dfrac{x^2+1}{\left(x^2-1\right)^2}=0 \quad \Rightarrow \quad x^2+1=0\quad  \Rightarrow \quad  x^2=-1.$$
		El cual no es posible para ningún $x$ real.\\\\
		El segundo grupo incluye a los extremos del intervalo. Es decir,
		$$0,\;5$$
		El tercer grupo es vacío, ya que $f$ es diferenciable en todas partes. Por último calculamos 
		$$\begin{array}{ccccl}
		    f\left(0\right) &=& \dfrac{0}{0^2-1} &=&0.\\\\
		    f\left(5\right) &=& \dfrac{5}{5^2-1} &=&\dfrac{5}{24}.\\\\
		\end{array}$$
		Por lo tanto el mínimo viene dado por $0$ y el máximo viene dado por $\dfrac{5}{24}.$\\\\

	\end{enumerate}

    %------------------ 2.
    \item Trace ahora la gráfica de cada una de las funciones del Problema 1 (Spivak, capítulo 11.) y halle todos los puntos máximos y mínimos locales.\\\\
	Respuesta.-\;

	\begin{enumerate}[(i)]

	    %---------- (i)
	    \item $f(x)=x^3-x^2-8x+1$ en $[-2,2]$.\\\\
		Respuesta.-\; 
		\begin{center}
		    \begin{tikzpicture}
			\begin{axis}[scale=.5,draw opacity =.5,samples=100,smooth, 
			  axis x line=center, 
			  axis y line=center,
			  ylabel = {$f(x)$},
			  xlabel = {$x$},
			  xlabel style={below right},
			  ylabel style={above left},
			  label style={font=\tiny},
			  tick label style={font=\tiny},
			  enlargelimits=upper] 
			  \addplot[black,opacity=1,domain=-2:2]{x^3-x^2-8*x+1};
			\end{axis}
		    \end{tikzpicture}
		\end{center}

		Para calcular los puntos máximos y mínimos locales, primero calculamos la derivada de $f$.
		$$f'(x)=3x^2-2x-8$$
		De donde los puntos críticos están dados por:
		$$3x^2-2x-8=0 \quad \Rightarrow \quad (3x+4)(x-2)=0 \quad \Rightarrow \quad x_1=-\dfrac{4}{3},\quad x_2=2.$$
		Ya que $f'$ existe, entonces podemos calcular $f''$.
		$$f''(x)=6x-2.$$
		Luego, calculamos $f''\left(-\dfrac{4}{3}\right)$ y $f''(2)$.
		$$f''\left(-\dfrac{4}{3}\right)=6\left(-\dfrac{4}{3}\right)-2=-10<0\quad ;\quad  f''(2)=6\cdot 2 - 2=10>0$$
		Por lo tanto, $-\dfrac{4}{3}$ es un punto máximo local y $2$ es un punto mínimo local.\\\\

	    %---------- (ii)
	    \item $f(x)=x^5+x+1$ en $[-1,1]$.\\\\
		Respuesta.-\; 
		\begin{center}
		    \begin{tikzpicture}
			\begin{axis}[scale=.5,draw opacity =.5,samples=100,smooth, 
			  axis x line=center, 
			  axis y line=center,
			  ylabel = {$f(x)$},
			  xlabel = {$x$},
			  xlabel style={below right},
			  ylabel style={above left},
			  label style={font=\tiny},
			  tick label style={font=\tiny},
			  enlargelimits=upper] 
			  \addplot[black,opacity=1,domain=-1:1]{x^5+x+1};
			\end{axis}
		    \end{tikzpicture}
		\end{center}

		Para calcular los puntos máximos y mínimos locales, primero calculamos la derivada de $f$.
		$$f'(x)=5x^4+1.$$
		En este caso podemos ver que no existen puntos críticos, ya que
		$$5x^4+1=0 \quad \Rightarrow \quad x^4=-\dfrac{1}{5}.$$
		no tiene soluciones reales. Y por lo tanto no tiene máximos ni mínimos locales.\\\\

	    %---------- (iii)
	    \item $f(x)=3x^4-8x^3+6x^2$ en $\left[-\frac{1}{2},\frac{1}{2}.\right]$\\\\
		Respuesta.-\; 
		\begin{center}
		    \begin{tikzpicture}
			\begin{axis}[scale=.5,draw opacity =.5,samples=100,smooth, 
			  axis x line=center, 
			  axis y line=center,
			  ylabel = {$f(x)$},
			  xlabel = {$x$},
			  xlabel style={below right},
			  ylabel style={above left},
			  label style={font=\tiny},
			  tick label style={font=\tiny},
			  enlargelimits=upper] 
			  \addplot[black,opacity=1,domain=-1/2:1/2]{3*x^4-8*x^3+6*x^2};
			\end{axis}
		    \end{tikzpicture}
		\end{center}

		Para calcular los puntos máximos y mínimos locales, primero calculamos la derivada de $f$.
		$$f'(x)=12x^3-24x^2+12x.$$
		De donde los puntos críticos están dados por:
		$$12x^3-24x^2+12x=0 \quad \Rightarrow \quad 12x(x^2-2x+1)=0 \quad \Rightarrow \quad x_1=0, \quad x_2=1.$$
		Ya que $f'$ existe, entonces podemos calcular $f''$.
		$$f''(x)=36x^2-48x+12.$$
		Luego, por el hecho de que $f''(1)$ no está contenido en el intervalo $\left[-\frac{1}{2},\frac{1}{2}.\right]$ solo calcularemos $f''\left(0\right)$.
		$$f''\left(0\right)=36\cdot 0^2 - 48\cdot 0 + 12 = 12.$$
		Por el teorema 11.6 de Spivak, vemos que $f''=12\geq 0$ en el intervalo $\left[-\frac{1}{2},\frac{1}{2}.\right]$, por lo tanto, $0$ es un punto máximo local.\\

	    %---------- (iv)
	    \item $f(x)=\dfrac{1}{x^5+x+1}$ en $\left[-\dfrac{1}{2},1\right]$.\\\\
		Respuesta.-\; 
		\begin{center}
		    \begin{tikzpicture}
			\begin{axis}[scale=.5,draw opacity =.5,samples=100,smooth, 
			  axis x line=center, 
			  axis y line=center,
			  ylabel = {$f(x)$},
			  xlabel = {$x$},
			  xlabel style={below right},
			  ylabel style={above left},
			  label style={font=\tiny},
			  tick label style={font=\tiny},
			  enlargelimits=upper] 
			  \addplot[black,opacity=1,domain=-1/2:1]{1/(x^5+x+1)};
			\end{axis}
		    \end{tikzpicture}
		\end{center}
		Para calcular los puntos máximos y mínimos locales, primero calculamos la derivada de $f$.
		$$f'(x)=-\dfrac{5x^4+1}{x^5+x+1}.$$
		En este caso podemos ver que no existen puntos críticos, ya que
		$$-\dfrac{5x^4+1}{x^5+x+1}=0 \quad \Rightarrow \quad 5x^4+1=0.$$
		no tiene soluciones en el intervalo $\left[-\dfrac{1}{2},1\right]$. Y por lo tanto no tiene máximos ni mínimos locales.\\\\

	    %---------- (v)
	    \item $f(x)=\dfrac{x+1}{x^2+1}$ en $\left[-1,\dfrac{1}{2}\right]$.\\\\
		Respuesta.-\; 
		\begin{center}
		    \begin{tikzpicture}
			\begin{axis}[scale=.5,draw opacity =.5,samples=100,smooth, 
			  axis x line=center, 
			  axis y line=center,
			  ylabel = {$f(x)$},
			  xlabel = {$x$},
			  xlabel style={below right},
			  ylabel style={above left},
			  label style={font=\tiny},
			  tick label style={font=\tiny},
			  enlargelimits=upper] 
			  \addplot[black,opacity=1,domain=-1:1/2]{(x+1)/(x^2+1)};
			\end{axis}
		    \end{tikzpicture}
		\end{center}
		Para calcular los puntos máximos y mínimos locales, primero calculamos la derivada de $f$.
		$$f'(x)=\dfrac{1-2x-x^2}{\left(1+x^2\right)^2}.$$
		De donde los puntos críticos están dados por:
		$$\dfrac{1-2x-x^2}{\left(1+x^2\right)^2}=0 \quad \Rightarrow \quad 1-2x-x^2=0 \quad \Rightarrow \quad x=-1\pm \sqrt{2}.$$
		Ya que $f'$ existe, entonces podemos calcular $f''$.
		$$f''(x)=\dfrac{-2\left[(x^2+1)(x+1)+2x(1-2x-x^2)\right]}{\left(1+x^2\right)^3}.$$
		Luego, por el hecho de que $f''(-1-\sqrt{2})$ no está contenido en el intervalo $\left[-1,\frac{1}{2}.\right]$ solo calcularemos $f''\left(-1+\sqrt{2}\right)$. Así, dado que 
		$$f''\left(-1+\sqrt{2}\right)<0$$
		Entonces $-1+\sqrt{2}$ es un máximo local.\\\\

	    %---------- (vi)
	    \item $f(x)=\dfrac{x}{x^2-1}$ en $[0,5]$.\\\\
		Respuesta.-\; 
		\begin{center}
		    \begin{tikzpicture}
			\begin{axis}[scale=.5,draw opacity =.5,samples=100,smooth, 
			  axis x line=center, 
			  axis y line=center,
			  ylabel = {$f(x)$},
			  xlabel = {$x$},
			  xlabel style={below right},
			  ylabel style={above left},
			  label style={font=\tiny},
			  tick label style={font=\tiny},
			  enlargelimits=upper] 
			  \addplot[black,opacity=1,domain=0:5]{(x)/(x^2-1)};
			\end{axis}
		    \end{tikzpicture}
		\end{center}
		Para calcular los puntos máximos y mínimos locales, primero calculamos la derivada de $f$.
		$$f'(x)=-\dfrac{x^2+1}{\left(x^2-1\right)^2}.$$
		En este caso podemos ver que no existen puntos críticos, ya que
		$$-\dfrac{x^2+1}{\left(x^2-1\right)^2}=0 \quad \Rightarrow \quad x^2-1=0.$$
		no tiene soluciones reales. Y por lo tanto no tiene máximos ni mínimos locales.\\\\

	\end{enumerate}

    %------------------ 3.
    \item Trace las gráficas de las siguientes funciones:

	\begin{enumerate}[(i)]

	    %---------- (i)
	    \item $f(x)=x+\dfrac{1}{x}.$\\\\
		Respuesta.-\; 
		\begin{center}
		    \begin{tikzpicture}
		    \begin{axis}[scale=.5,draw opacity =.5,samples=100,smooth, 
		      axis x line=center, 
		      axis y line=center,
		      ylabel = {$f(x)$},
		      xlabel = {$x$},
		      xlabel style={below right},
		      ylabel style={above left},
		      label style={font=\tiny},
		      tick label style={font=\tiny},
		      enlargelimits=upper] 
		      \addplot[black,opacity=1]{x+1/x};
		    \end{axis}
		\end{tikzpicture}
		\end{center}
		\vspace{.5cm}

	    %---------- (ii)
	    \item $f(x)=x+\dfrac{3}{x^2}$.\\\\
		Respuesta.-\;
		\begin{center}
		    \begin{tikzpicture}
		    \begin{axis}[scale=.5,draw opacity =.5,samples=100,smooth, 
		      axis x line=center, 
		      axis y line=center,
		      ylabel = {$f(x)$},
		      xlabel = {$x$},
		      xlabel style={below right},
		      ylabel style={above left},
		      xmin=-5,xmax=5,ymin=-7,ymax=20,
		      label style={font=\tiny},
		      tick label style={font=\tiny},
		      enlargelimits=upper] 
		      \addplot[black,opacity=1]{x+3/x^2};
		    \end{axis}
		\end{tikzpicture}
		\end{center}
		\vspace{.5cm}

	    %---------- (iii)
	    \item $f(x)=\dfrac{x^2}{x^2-1}$.\\\\
		Respuesta.-\;
		\begin{center}
		    \begin{tikzpicture}
		    \begin{axis}[scale=.5,draw opacity =.5,samples=100,smooth, 
		      axis x line=center, 
		      axis y line=center,
		      ylabel = {$f(x)$},
		      xlabel = {$x$},
		      xlabel style={below right},
		      ylabel style={above left},
		      label style={font=\tiny},
		      tick label style={font=\tiny},
		      enlargelimits=upper] 
		      \addplot[black,opacity=1]{x^2/(x^2-1)};
		    \end{axis}
		\end{tikzpicture}
		\end{center}
		\vspace{.5cm}

	    %---------- (iv)
	    \item $f(x)=\dfrac{1}{1+x^2}$.\\\\
		Respuesta.-\;
		\begin{center}
		    \begin{tikzpicture}
		    \begin{axis}[scale=.5,draw opacity =.5,samples=100,smooth, 
		      axis x line=center, 
		      axis y line=center,
		      ylabel = {$f(x)$},
		      xlabel = {$x$},
		      xlabel style={below right},
		      ylabel style={above left},
		      label style={font=\tiny},
		      tick label style={font=\tiny},
		      enlargelimits=upper] 
		      \addplot[black,opacity=1]{1/(1+x^2)};
		    \end{axis}
		\end{tikzpicture}
		\end{center}
		\vspace{.5cm}
	\end{enumerate}

    %------------------ 4.
    \item 
	\begin{enumerate}[(a)]

	    %---------- (a)
	    \item Si $a_1<\ldots < a_n$ halle el valor mínimo de $f(x)=\sum\limits_{i=1}^n (x-a_i)^2$.\\\\
		Respuesta.-\; Primero calculamos la derivada de $f$.
		$$f'(x)=2(x-a_1)\cdot 1 + 2(x-a_2)\cdot 1 + \ldots + 2(x-a_n)\cdot 1=2\sum\limits_{i=1}^n (x-a_i).$$
		Luego encontremos los puntos críticos igualando $f'(x)$ cero.
		$$\begin{array}{rcl}
		    2\displaystyle\sum_{i=1}^n (x-a_i)=0 &\Rightarrow& (x-a_1)+(x-a_2)+\ldots + (x-a_n) = 0\\\\
							 &\Rightarrow& nx=a_1+a_2+\ldots + a_n\\\\
							 &\Rightarrow& x=\dfrac{1}{n}\displaystyle\sum_{i=1}^n a_i.
		\end{array}$$
		Veamos ahora si este punto crítico es un mínimo o un máximo local. Para ello calculamos la segunda derivada de $f$.
		$$\begin{array}{rcl}
		    f''(x)&=&2'\left[\displaystyle\sum_{i=1}^n (x-a_i)\right]+2\left[\displaystyle\sum_{i=1}^n (x-a_i)\right]'\\\\
			  &=&2(1+1+\ldots +1)\\\\
			  &=&2n.
		\end{array}$$
		Ya que $2n>0$, entonces podemos decir que el punto crítico $x=\dfrac{1}{n}\displaystyle\sum_{i=1}^n a_i$ es un mínimo local de $f$. Por lo tanto,
		$$f\left(\dfrac{1}{n}\sum_{i=1}^n a_i\right)=\sum_{i=1}^n \left[\dfrac{1}{n}\sum_{i=1}^n \left(a_i-a_i\right)\right]^2.$$\\

	    %---------- (b)
	    \item Halle ahora el valor mínimo de $f(x)=\sum\limits_{i=1}^n |x-a_i|$. Este es un problema en el que el cálculo infinitesimal no nos puede ayudar. En los intervalos entre los $a_i$ la función $f$ es lineal, por tanto el valor mínimo se localiza en uno de los $a_i$, y estos son, precisamente, los puntos en los cuales la función $f$ no es diferenciable. Sin embargo, la solución es fácil de encontrar si se considera como varía $f(x)$ al pasar de un intervalo a otro.\\\\
		Respuesta.-\; Tomemos dos puntos $a$ y $b$ en $[a_{i-1},a_i]$ y $[a_i,a_{i+1}]$ respectivamente. Tal que 
		$$|a-a_i|=|b-a_i|.$$
		Notemos que
		$$\begin{array}{rclr}
		    |b-a_j| &=& |a-a_j|+|a-b| & \mbox{si } j\leq i-1\\
		    |b-a_j| &=& |a-a_j|-|a-b| & \mbox{si } j\geq i+1
		\end{array}$$
		Luego, vemos que
		$$\begin{array}{rcl}
		    f(b) &=& \displaystyle \sum_{j=1}^n |b-a_j|\\\\
			 &=& \displaystyle \sum_{j=1}^{i-1} |b-a_j|+|b-a_i|+\sum_{j=i+1}^n|b-a_j|\\\\
			 &=& \displaystyle \sum_{j=1}^{i-1} \left(|a-a_j|+|a-b|\right)+|b-a_i|+\sum_{j=i+1}^n\left(|a-a_j|-|a-b|\right)\\\\
			 &=& \displaystyle \sum_{j=1}^{i-1} |a-a_j|+(i-1)|a-b|+|b-a_i|+\sum_{j=i+1}^n|a-a_j|-(n-i)|a-b|\\\\
			 &=& \displaystyle \sum_{j=1}^{i-1} |a-a_j|+|b-a_i|+\sum_{j=i+1}^n|a-a_j|(i-1)-(n-i)|a-b|\\\\
			 &=& \displaystyle \sum_{j=1}^{i-1} |a-a_j|+|b-a_i|+\sum_{j=i+1}^n|a-a_j|+(2i-n-1)|a-b|\\\\
		\end{array}$$
		Se sigue que $f(b)\geq f(a)$ siempre que
		$$2i-n-1\geq 0 \quad \mbox{o}\quad i\geq \dfrac{n+1}{2}.$$
		Por otro lado, de manera similar, $f(b)\leq f(a)$ siempre que
		$$2i-n-1\leq 0 \quad \mbox{o}\quad i\leq \dfrac{n+1}{2}.$$
		Así, tenemos a $f$ decreciente si $i\leq \dfrac{n+1}{2}$ y $f$ creciente si $i\geq \dfrac{n+1}{2}.$ De esta manera $f$ alcanza su mínimo en $a_{\frac{n+1}{2}}$ si $n$ es impar y está en el intervalo $\left[a_{\frac{n}{2}},a_{\frac{n}{2}+1}\right]$. Por lo tanto, el valor mínimo de $f$ es $f\left(a_{\frac{n+1}{2}}\right)$ si $n$ es impar y $f\left(a_{\frac{n}{2}}\right)$ si $n$ es par.\\\\

	    %---------- (c)
	    \item Sea $a>0$. Demuestre que el valor máximo de 
	    $$f(x)=\dfrac{1}{1+|x|}+\dfrac{1}{1+|x-a|}$$
	    es $(2+a)/(1+a)$. (Puede hallarse por separado la derivada en cada uno de los intervalos $(-\infty,0),(0,a)$ y $(a,\infty)$).\\\\
		Respuesta.-\; La función dada se puede escribir como sigue:
		$$f(x)=\left\{\begin{array}{rclcl}
			\dfrac{1}{1-x}+\dfrac{1}{1-x+a} &\mbox{si}& x<0\\\\
			\dfrac{1}{1+x}+\dfrac{1}{1-x+a} &\mbox{si}& 0<x<a\\\\
			\dfrac{1}{1+x}+\dfrac{1}{1+x-a} &\mbox{si}& x>a
		\end{array}\right.$$
		Notemos que $f$ es diferenciable en cada intervalo $(-\infty,0),(0,a)$ y $(a,\infty)$. Por lo que podemos hallar sus respectivas derivadas,
		$$f'(x)=\left\{\begin{array}{rclcl}
			\dfrac{1}{(1-x)^2}+\dfrac{1}{(1-x+a)^2} &\mbox{si}& x<0\\\\
			\dfrac{1}{(1+x)^2}+\dfrac{1}{(1-x+a)^2} &\mbox{si}& 0<x<a\\\\
			\dfrac{1}{(1+x)^2}+\dfrac{1}{(1+x-a)^2} &\mbox{si}& x>a
		\end{array}\right.$$
		el cual demuestra que $f'(x)>0$ si $(-\infty,0)$ y $f'(x)<0$ si $(a,\infty)$, esto ya que $(1-x)^2$ y $(1\pm x+a)^2$ son positivos. Además, vemos que
		$$f(0)=1+\dfrac{1}{1+a}=f(a)>0$$
		ya que $a>0.$ Así, $f$ es creciente en $(-\infty,0]$ y decreciente en $[a,\infty]$, donde se concluye que $f$ en $\mathbb{R}$ alcanza su punto máximo en algún punto del intervalo $[0,a]$. Ahora, verificamos si el máximo de $f$ se alcanza en algún punto en $(0,a)$. Si $b$ es tal punto, entonces 
		$$-\dfrac{1}{(1+b)^2}+\dfrac{1}{(1-b+a)^2}=0,$$
		el cual implica,
		$$(1+x)^2-(1-x+a)^2=(2+a)(2b-a)=0.$$
		Así, $b=\dfrac{a}{2}.$ Es más,
		$$f(0)=f(a)=1+\dfrac{1}{1+a}=\dfrac{2+a}{1+a}.$$
		Por lo tanto, se tiene que
		$$f\left(\dfrac{a}{2}\right)=\dfrac{2}{1+\dfrac{a}{2}}=\dfrac{4}{2+a}<\dfrac{2+a}{1+a}$$
		lo que muestra que $\dfrac{2+a}{1+a}$ es el máximo valor de $f$ en $[0,a]$.\\\\

	\end{enumerate}

    %-------------------- 5.
    \item Para cada una de las siguientes funciones, halle todos los puntos máximos y mínimos locales.
	\begin{enumerate}[(i)]

	    %---------- (i)
	    \item \;
	    $$f(x)=\left\{\begin{array}{rl}
		    x, & x\neq 3,5,7,9\\
		    5, & x=3\\
		    -3, & x=5\\
		    9, & x=7\\
		    7, & x=9
	    \end{array}\right.$$
	    \vspace{0.4cm}

		Respuesta.-\; Notemos que todos los puntos locales máximo y mínimos deben estar en el conjunto $\left\{3,5,7,9\right\}$, ya que, aparte de estos puntos, $f$ cumple la función identidad. Ahora vemos por la definición de $f$ que 
		\begin{center}
		    $f(3)=5>3$ y $5>x$, para todo $x\in (3-\delta,3+\delta)$, para $0<\delta<1.$
		\end{center}
		Así, $3$ es un punto máximo local.\\

		Para $x=5$ tenemos por la definición de $f$ que
		\begin{center}
		    $f(5)=-3<x$ para todo $x\geq 0.$
		\end{center}
		Así, $5$ es un punto mínimo local.\\

		También vemos por la definición de $f$ que
		\begin{center}
		    $f(7)=9>7$ y $9>x$, para todo $x\in (7-\delta,7+\delta)$, para $0<\delta<1.$
		\end{center}
		Así, $7$ es un punto máximo local.\\

		Para $x=5$ tenemos por la definición de $f$ que
		\begin{center}
		    $f(9)=7<9$ y $7<x,$ para todo $x\in (9-\delta,9+\delta)$, para $0<\delta<1.$
		\end{center}
		Así, $9$ es un punto mínimo local.\\\\

	    %---------- (ii)
	    \item \;
	    $$f(x)=\left\{\begin{array}{rl}
		    0, & x \mbox{ irracional}\\
		    1/q, & x=p/q \mbox{ fracción irreducible}.
	    \end{array}\right.$$
	    \vspace{0.4cm}

		Respuesta.-\; Notemos que cada número irracional $x$ es un mínimo local de $f$ ya que , en el caso de que $f(x)=0$ y $f(y)=\dfrac{1}{q}>0,$ para cualquier racional $y=p/q$. Por otro lado, no existe un punto máximo local. De hecho, vemos por la definición de $f$ que el máximo local sólo puede ocurrir para los números racionales. Pero para cualquier racional $x=p/q$, $f(x)=1/q<p/q=x$ si $p>0$ y $f(x)=1/q>p/q=x$  si $p<0$ pero $f(x)>0=f(y)$, para cualquier número racional $y$.\\\\

	    %---------- (iii)
	    \item \;
	    $$f(x)=\left\{\begin{array}{rl}
		    x,& x \mbox{ racional}\\
		    0, x& \mbox{ irracional}.
	    \end{array}\right.$$
	    \vspace{0.4cm}

		Respuesta.-\; Observamos que todo número irracional positivo $x$ es un mínimo local de $f$. Ya que, en este caso, $f(x)=0$ y $f(y)=y>0$, para cualquier racional $y\geq0$. Por otro lado, para cualquier racional $y\leq 0$, tenemos $f(y)=y\leq 0$ y por lo tanto cualquier número irracional estrictamente negativo es un máximo local de $f$.\\\\

	    %---------- (iv)
	    \item \;
	    $$f(x)=\left\{\begin{array}{rl}
		    1, & x=1/n \mbox{ para algún } n\in \mathbb{N}\\
		    0, &x \mbox{ en los demás casos}.
	    \end{array}\right.$$
	    \vspace{0.4cm}

	    Respuesta.-\; Notemos que, para cada $n\in \mathbb{N}, 1/n$ es un punto local máximo a partir de la definición de $f$, $f(\frac{1}{n})=1,$ y $f(x)=0.$ Similarmente, vemos que para cada número real tal que $\left\{\frac{1}{n}:n\in \mathbb{N}\right\}$ es un punto mínimo local, ya que en este conjunto $f$ es idénticamente $0$. Además si $f$ es constante para cada número real tal que $\left\{\frac{1}{n}:n\in \mathbb{N}\right\}$, entonces es un punto máximo y mínimo local a la vez, excepto el punto $0$, ya que para cada $\epsilon>0$, $(-\delta, \delta)$ contiene infinitos $1/n$, esto por la propiedad Arquimediana de los números reales.\\\\

	    %---------- (v)
	    \item \;
	    $$f(x)=\left\{\begin{array}{rl}
		    1, & x \mbox{ si el desarrollo decimal de } x \mbox{contiene un } 5\\
		    0, &x \mbox{ en los demás casos}.
	    \end{array}\right.$$
	    \vspace{0.4cm}

		Respuesta.-\; Notemos que para cada
		$$x\in \left\{x\in \mathbb{R}:\mbox{el desarrollo decimal de } x \mbox{contiene a } 5\right\}$$
		es un punto máximo local a partir de la definición de $f$. Es este caso, $f(x)=1$ y $f$ es $0$. Similarmente, vemos que cada número real tal que, 
		$$\left\{x\in \mathbb{R}:\mbox{el desarrollo decimal de } x \mbox{contiene a } 5\right\}$$
		es un punto mínimo local, ya que en este conjunto $f$ es idénticamente $0$. Además, ya que $f$ es constante, tenemos que cada número en el conjunto 
		$$\left\{x\in \mathbb{R}:\mbox{el desarrollo decimal de } x \mbox{contiene a } 5\right\}$$
		es un punto máximo local y un punto mínimo local, excepto el punto $0$, ya que para cada $\delta>0,$ $(-\delta,\delta)$ contiene infinitos puntos con al menos un $5$ en su expansión decimal, esto por la propiedad Arquimediana de los números reales\\\\

	\end{enumerate}

    %-------------------- 6.
    \item Demuestre la siguiente propiedad (que se utiliza muchas veces de manera implícita): si $f$ es creciente en $(a, b)$ y continua en $a$ y $b$, entonces $f$ es creciente en $[a, b]$. En particular, si $f$ es continua en $[a, b]$ y $f > 0$ en $(a, b)$, entonces $f$ es creciente en $[a, b]$.\\\\
	Demostración.-\; Sea $x,y\in [a,b]$ tal que x<y. Ya que $f$ es creciente en $(a,b)$, entonces $f$ es también creciente en $(x,y)$. Por el teorema de valor medio, existe un $c\in (x,y)$ tal que 
	$$f(y)-f(x)=f'(c)(y-x)>0.$$
	Si $f$ es creciente en $(x,y)$,
	$$f'(c)\geq 0 \mbox{ para todo } c\in (x,y).$$
	Por el hecho de que $y-x>0$, entonces
	$$f(x)<f(y).$$
	De esta manera, $x$ e $y$ son número arbitrarios en $[a,b]$ y por lo tanto $f$ es creciente en $[a,b]$.\\\\

    %-------------------- 7.
    \item Se traza una recta desde el punto $(0, a)$ hasta el eje horizontal y desde allí otra a $(1, b)$, tal como se indica en la Figura 23 (Spivak). Demuestre que la longitud total es mínima cuando los ángulos $\alpha$ y $\beta$ son iguales. (Naturalmente, deberá entrar en juego una función: expresar la longitud en términos de $x$, donde $(x, 0)$ es el punto del eje horizontal. La línea a trazos de la Figura 23 sugiere una demostración geométrica; tanto en un caso como en otro puede resolverse el problema sin necesidad de hallar el punto $(x, 0)$.)\\\\
	Demostración.-\; De la figura dada, encontramos que la longitud total del camino está dada por: 
	$$f(x)=\sqrt{x^2+a^2}+\sqrt{(1-x)^2+b^2}.$$
	Para la longitud más corta, obtenemos la primera derivada y la igualamos con cero
	$$\dfrac{x}{\sqrt{x^2+a^2}}-\dfrac{1-x}{\sqrt{(1-x)^2+b^2}}=0.$$
	Reemplazando $\dfrac{x}{\sqrt{x^2+a^2}}$ con $\cos \alpha$ y $\dfrac{1-x}{\sqrt{(1-x)^2+b^2}}$ con $\cos \beta$, obtenemos
	$$\cos \alpha-\cos \beta=0 \quad \Rightarrow \quad \cos \alpha = \cos \beta \quad \Rightarrow \quad \alpha = \beta.$$\\

    %-------------------- 8.
    \item 
	\begin{enumerate}[(a)]

	    %---------- (a)
	    \item Sea $(x_0,y_0)$ un punto del plano, y sea $L$ la gráfica de la función $f(x)=mx+b.$ Halle el punto $\overline{x}$ tal que la distancia de $(x_0,y_0)$ a $(\overline{x},f(\overline{x}))$ sea mínima. [Observe que minimizar esta distancia equivale a minimizar su cuadrado. Esto puede simplificar, en cierta medida, los cálculos.]\\\\
		Respuesta.-\; La distancia entre $(x_0,y_0)$ y algún punto $x$ que satisfaga $f(x)=mx+b$ es:
		$$d^2=\left(x_0-x\right)^2+\left(y_0-mx-b\right)^2$$
		Derivando a ambos lados con respecto a $x$, 
		$$2dd'=-2\left(x_0-x\right)+2m\left(y_0+mx-b\right)$$
		Por la distancia más corta, entonces $d'=0$. Luego,
		$$-x_0+x-my_0+m^2x+bm=0 \quad \Rightarrow \quad x=\dfrac{x_0+my_0-bm}{m^2+1}.$$\\

	    %---------- (b)
	    \item Halle también $\overline{x}$ observando que la recta que va de $(x_0.y_0)$ a $\left(\overline{x},f(\overline{x})\right)$ es perpendicular a $L$.\\\\
		Respuesta.-\; La distancia más corta ocurre cuando la linea de $(x_0,y_0)$ es perpendicular a $L$,
		$$x=\dfrac{x_0+my_0-bm}{m^2+1}.$$\\

	    %---------- (c)
	    \item Halle la distancia de $(x_0,y_0)$ de $L$. Es decir, la distancia de $(x_0,y_0)$ a $\left(\overline{x},f(\overline{x})\right)$. [Facilitará los cálculos suponer primero que $b=0$; luego aplicar el resultado a la gráfica de $f(x)=mx+b$ y el punto $(x_0,y_0-b)$].\\\\
		Respuesta.-\; Sea la ecuación dada en el inciso (a),
		$$d=\sqrt{\left(x_0-x\right)^2+\left(y_0-mx-b\right)^2}.$$
		Supongamos $b=0$, entonces
		$$d=\sqrt{x_0^2-2x_0x+x^2+y_0^2-2my_0x+m^2x^2}\quad \Rightarrow \quad d=\sqrt{\left(m^2+1\right)x^2-2\left(x_0+my_0\right)x+x_0^2+y_0^2}.$$
		Reemplazando $x$ con $\dfrac{x_0+my_0}{m^2+1}$,
		$$\begin{array}{rcl}
		    d&=&\sqrt{\left(\dfrac{x_0+my_0}{m^2+1}\right)^2-2\dfrac{\left(x_0+my_0\right)^2}{m^2+1}+x_0^2+y_0^2}\\
		     &=&\sqrt{-\dfrac{\left(x_0+my_0\right)^2}{m^2+1}+x_0^2+y_0^2}\\\\
		     &=&\sqrt{\dfrac{-x_0^2-2mx_0y_0-m^2y_0^2+x_0^2m^2+y_0^2m^2+x_0^2+y_0^2}{m^2+1}}\\\\
		     &=&\sqrt{\dfrac{\left(y_0-mx_0\right)^2}{m^2+1}}\\\\
		\end{array}$$
		Reemplazando $y_0$ con $y_0-b$, concluimos que,
		$$\begin{array}{rcl}
		    d&=&\dfrac{\left(y_0-b-mx_0\right)^2}{m^2+1}\\\\
		     &=&\dfrac{y_0-b-mx_0}{\sqrt{m^2+1}}
		\end{array}$$
		\vspace{0.5cm}\\\\

	    %---------- (d)
	    \item Considere una recta descrita mediante la ecuación $Ax+By+C=0$. Demuestre que la distancia de $\left(x_0,y_0\right)$ a esta recta es $\left(Ax_0+By_0+C\right)/\sqrt{A^2+B^2}$.\\\\
		Demostración.-\; Sea la ecuación:
		$$Ax+By+C=0$$
		de donde, encontrar la pendiente $m=-\dfrac{A}{B}.$
		Substituyendo en la ecuación de la parte (c), se tiene
		$$d=\dfrac{y_0-b+\dfrac{A}{B}x_0}{\sqrt{\left(-\dfrac{A}{B}\right)^2}+1}$$
		Luego, multiplicando ambos lados por $\dfrac{B}{B}$,
		$$\begin{array}{rcll}
		    d&=&\dfrac{Ax_0+By_0-bB}{\sqrt{A^2+B^2}}&\\\\
		     &=&\dfrac{Ax_0+By_0+C}{\sqrt{A^2+B^2}}&\mbox{supongamos } bB=-C.
		\end{array}$$
		\vspace{0.5cm}\\\\

	\end{enumerate}

    %-------------------- 9.
    \item El problema anterior (8, Michael Spivak, capítulo 11) sugiere la siguiente cuestión: ¿cuál es la relación entre los puntos críticos de $f$ y los de $f^2$?.\\\\
	Respuesta.-\; Los puntos críticos de una función son los puntos en los que la función no es diferenciable o su primera derivada es cero. Sabemos que para definir los puntos críticos se iguala  $f'$ a cero y para los puntos críticos podemos igualar también $\left(f^2\right)'$ a cero, lo que implica
	$$\left(f^2\right)'=2ff'$$
	Por lo tanto, los puntos críticos de $f$ son subconjuntos de los puntos críticos de $f^2$.\\\\

    %-------------------- 10.
    \item Demuestre que entre todos los rectángulos de igual perímetro, el de mayor área es el cuadrado.\\\\
	Demostración.-\; Sean $l$ el largo, $a$ el ancho y $p$ el perímetro del rectángulo. Entonces,
	$$p=2(l+a)\quad \Rightarrow \quad a=\dfrac{p}{2}-l.$$
	Sabemos que el área, llamémosla $A$, está dada por:
	$$A=l \cdot a$$
	De donde,
	$$A=l\left(\dfrac{p}{2}-l\right)=\dfrac{p}{2}l-l^2.$$
	Derivando con respecto a $l$, se tiene
	$$A'=\dfrac{p}{2}-2l.$$
	Luego igualando a cero, se tiene
	$$\dfrac{p}{2}-2l=0\quad \Rightarrow \quad l=\dfrac{p}{4}.$$
	Por el corolario 11.3 y sabiendo que si $f'<0$ en $\left(-\infty,\frac{p}{4}\right)$  y $f'>0$ en $\left(\frac{p}{4},\infty\right)$. Entonces, $\frac{p}{4}$ es un punto mínimo local.\\

	Luego, reemplazando $l=\frac{p}{4}$ en $a=\frac{p}{2}-l$, se tiene
	$$a=\dfrac{p}{2}-\frac{p}{4}=l.$$
	Por lo tanto, el área del rectángulo es la más pequeña cuando su largo es igual a su ancho y se convierte en un cuadrado.\\\\

    %-------------------- 11.
    \item Entre todos los cilindros circulares rectos de volumen fijo $V$, halle el de menor superficie (incluyendo las superficies de las caras superior e inferior como en la Figura 24, Spivak, capítulo 11.).\\\\
	Respuesta.-\; El volumen $V$ de un cilindro con radio $r$ y altura $h$ es:
	$$V=\pi r^2h \quad \Rightarrow \quad h=\dfrac{V}{\pi r^2}.$$
	Luego, el área total $A$ de un cilindro está dada por:
	$$A=2\pi r^2+2\pi rh.$$
	Reemplazado $h$  en esta ecuación, se tiene
	$$A=2\pi r^2+2\pi r\dfrac{V}{\pi r^2}=2\pi r^2+\dfrac{2V}{r}.$$
	Derivando con respecto a $r$, se tiene
	$$A'=4\pi r - \dfrac{2V}{r^2}.$$
	Luego igualando a cero,
	$$4\pi r - \dfrac{2V}{r^2}=0\quad \Rightarrow \quad r=\dfrac{V}{\pi\left(\sqrt[3]{\dfrac{V}{2\pi}}\right)^2}.$$
	Por el corolario 11.3 y sabiendo que si $f'<0$ en $\left(-\infty,\dfrac{V}{\pi\left(\sqrt[3]{\dfrac{V}{2\pi}}\right)^2}\right)$  y $f'>0$ en $\left(\dfrac{V}{\pi\left(\sqrt[3]{\dfrac{V}{2\pi}}\right)^2},\infty\right)$. Entonces, $\dfrac{V}{\pi\left(\sqrt[3]{\dfrac{V}{2\pi}}\right)^2}$ es un punto mínimo local.\\
	Reemplazando $r$ en $h=\dfrac{V}{\pi r^2}$, obtenemos 
	$$h=\dfrac{V}{\pi\pi\left(\sqrt[3]{\dfrac{V}{2\pi}}\right)^2}.$$
	Por lo tanto, el área superficial del cilindro es la más pequeña cuando su radio es $\sqrt[3]{\dfrac{V}{2\pi}}$ y su altero es $\dfrac{V}{\pi\pi\left(\sqrt[3]{\dfrac{V}{2\pi}}\right)^2}$.\\\\

    %-------------------- 12.
    \item Un triángulo rectángulo cuya hipotenusa tiene una longitud $a$ gira alrededor de uno de sus lados, generando un cono circular recto. Halle el volumen máximo que puede tener este cono.\\\\
	Respuesta.-\; El volumen $V$ de un cono con radio $r$ y alguna altura $h$ es:
	$$V=\dfrac{1}{3}\pi r^2 h.$$
	Aplicando el teorema de Pitágoras,
	$$a^2=r^2+h^2 \quad \Rightarrow \quad h=\sqrt{a^2-r^2}.$$
	Reemplazando en $V=\dfrac{1}{3}\pi r^2 h$, se tiene
	$$V=\dfrac{1}{3}\pi r^2 \sqrt{a^2-r^2}\quad \Rightarrow\quad V=\dfrac{1}{3}\pi\sqrt{r^4a^2-r^6}.$$
	Derivando con respecto a $r$, 
	$$V'=\dfrac{1}{3}\pi\dfrac{4r^3a^2-6r^5}{\sqrt{r^4a^2-r^6}}.$$
	Luego igualando a cero,
	$$\dfrac{4r^3a^2-6r^5}{\sqrt{r^4a^2-r^6}}=0\quad \Rightarrow \quad 2a^2-3r^2=0 \quad \Rightarrow \quad r=\sqrt{\dfrac{2}{3}}a.$$
	Reemplazando $r$ en $V=\dfrac{1}{3}\pi r^2 h$, obtenemos 
	$$V=\dfrac{1}{3}\pi\sqrt{\dfrac{4}{9}a^6-\dfrac{8}{27}a^6}=\dfrac{1}{3}\pi\sqrt{\dfrac{4}{27}}a^6=\dfrac{2\sqrt{3}\pi}{27}a^3.$$
	El cual es el mayor volumen.\\\\

    %-------------------- 13.
    \item Demuestre que la suma de un número positivo y de su recíproco es al menos $2$.\\\\
	Demostración.-\; Sean $a$ un número positivo y $\dfrac{1}{a}$ para $a\neq 0$ es su reciproco. Definamos la suma $sum$ de estos dos números, como sigue
	$$sum=a+\dfrac{1}{a}.$$
	Derivando con respecto a $a$,
	$$sum'=1-\dfrac{1}{a^2}.$$
	Luego igualando a cero,
	$$1-\dfrac{1}{a^2}=0\quad \Rightarrow \quad a^2=1 \quad \Rightarrow \quad a=\pm 1.$$
	Ya que $a$ es un número positivo, solo se evaluará en $a=1$. Por el corolario 11.3, y sabiendo que si $sum'<0$ cuando $(0,1)$ y $sum'>0$ cuando $(1,\infty)$. Es decir $sum$ es decreciente cuando $(0,1)$ y creciente cuando $(1,\infty)$. Entonces $a$ es un punto mínimo. Luego reemplazando $a$ en $sum=a+\dfrac{1}{a}$ se tiene,
	$$sum=1+1=2.$$\\

    %-------------------- 14.
    \item Halle el trapezoide de mayor área que puede ser inscrito en un semicírculo de radio $a$, con una base situada a lo largo del diámetro.\\\\
	Respuesta.- \; Tracemos primero una gráfica.
	\begin{center}
	    \begin{tikzpicture}[baseline=(current bounding box.north)]
	      \def\Radius{1.8}
	      \path
		(-\Radius, 0) coordinate (A)
		-- coordinate (M)
		(\Radius, 0) coordinate (B)
		(M) +(60:\Radius) coordinate (C)
		+(120:\Radius) coordinate (D)
	      ;
	      % Draw semicircle
		  \draw
		    (B) arc(0:180:\Radius) -- cycle
		  ;
		  % Annotations
		  \path[inner sep=0pt];
		  \draw(-1.8,0)--(-1,1.5)--(1,1.5)--(1.8,0);
		  \draw(0,0)--(0,1.5)node[above]{\tiny$b_2$};
		  \draw(1,1.5)--(0,0)node[below]{\tiny$b_1=2a$}--(-1,1.5);
		  \draw(.6,.6)node[]{\tiny$a$};
		  \draw(0,.85)node[right]{\tiny$h$};
		  \draw(.25,1.3)node[right]{\tiny$\frac{b_2}{2}$};
	    \end{tikzpicture}
	\end{center}

	Por la esta figura, tenemos:
	$$b_2=2\sqrt{a^2-h^2}.$$
	Sabemos que el área del trapezoide está dado por:
	$$A=\dfrac{1}{2}(b_1+b_2)h.$$
	Reemplazando $b_1$ con $2a$ y $b_2$ con $2\sqrt{a^2-h^2}$, se tiene:
	$$\begin{array}{rcl}
	    A&=&\dfrac{1}{2}\left(2a+2\sqrt{a^2-h^2}\right)h\\\\
	     &=&ah+\sqrt{a^2h^2-h^4}.
	\end{array}$$
	Derivando con respecto a $x$,
	$$\begin{array}{rcl}
	    A'&=&a+\dfrac{2a^2h-4h^3}{2\sqrt{a^2h^2-h^4}}\\\\
	      &=&\dfrac{a\sqrt{a^2h^2-h^4}+a^2h-2h^3}{\sqrt{a^2h^2-h^4}}.
	\end{array}$$
	Luego igualando a cero,
	$$\begin{array}{rcl}
	    a\sqrt{a^2h^2-h^4}+a^2h-2h^3&=&0\\\\
	    a\sqrt{a^2-h^2}+a^2-2h^2&=&0\\\\
	    a^2\left(a^2-h^2\right) &=& 4h^4-4a^2h^2+a^4\\\\
	    h&=&\dfrac{\sqrt{3}}{2}a
	\end{array}$$

	Evaluando $h$ en $A=ah+\sqrt{a^2h^2-h^4}$, obtenemos:
	$$\begin{array}{rcl}
	    A&=&\dfrac{\sqrt{3}}{2}a^2+\sqrt{\dfrac{3}{4}a^4-\dfrac{9}{16}a^4}\\\\
	     &=&\dfrac{\sqrt{3}}{2}a^2+\dfrac{\sqrt{3}}{4}a^2\\\\
	     &=&\dfrac{3\sqrt{3}}{4}a^2
	\end{array}.$$
	\vspace{.5cm}

    %-------------------- 15.
    \item Dos pasillos de anchuras $a$ y $b$, forman un ángulo recto (Figura 25,Spivak, capítulo 11). ¿Cuál es la longitud máxima de una escalera que puede ser transportada horizontalmente alrededor de la esquina?.\\\\
	Respuesta.-\; Por la figura dada, tenemos:
	$$\sen \theta = \dfrac{b}{x}\quad \Rightarrow \quad x=\dfrac{b}{\sen \theta}.$$
	Y
	$$\cos \theta = \dfrac{a}{L-x}\quad \Rightarrow \quad L-x=\dfrac{a}{\cos \theta}.$$
	De donde, reemplazando la primera ecuación en la segunda, se tiene:
	$$L=\dfrac{a}{\cos \theta}+\dfrac{b}{\sen \theta}.$$
	Derivando con respecto a $\theta$,
	$$\begin{array}{rcl}
	    L'&=&\dfrac{a\sen \theta}{\cos^2\theta}-\dfrac{b\cos \theta}{\sen^2 \theta}\\\\
	      &=&\dfrac{a\sen^3\theta-b\cos^3\theta}{\cos^2\theta\sen^2\theta}.
	\end{array}$$
	Para la mayor longitud posible, igualamos $L'$ a cero,
	$$\begin{array}{rcl}
	    a\sen^3\theta-b\cos^3\theta&=&0\\\\
	    a\sen^3\theta&=&b\cos^3\theta\\\\
	    \dfrac{\sen^3\theta}{\cos^3\theta}&=&\dfrac{b}{a}\\\\
	    \tan^3\theta&=&\dfrac{b}{a}\\\\
	    \tan \theta &=& \sqrt[3]{\dfrac{b}{a}}\\\\
	    \sen\theta &=&\dfrac{\sqrt[3]{\dfrac{b}{a}}}{\sqrt{1+\left(\sqrt[3]{\dfrac{b}{a}}\right)^2}}\\\\
	    \cos\theta &=& \dfrac{1}{\sqrt{1+\left(\sqrt[3]{\dfrac{b}{a}}\right)^2}}\\\\
	\end{array}$$
	Reemplazando los valores de $\sen$ y $\cos$ en $L=\dfrac{a}{\cos \theta}+\dfrac{b}{\sen \theta}$, para obtener la mayor longitud posible, se tiene:
	$$L=a\sqrt{1+\left(\sqrt[3]{\dfrac{b}{a}}\right)^2}+b\dfrac{\sqrt{1+\left(\sqrt[3]{\dfrac{b}{a}}\right)^2}}{\sqrt[3]{\dfrac{b}{a}}}.$$\\

    %-------------------- 16.
    \item Se diseña un jardín en forma de un sector circular (Figura 26, Spivak, capítulo 11), de radio $R$ y ángulo central $\theta$ . El jardín debe tener un área fija $A$. ¿Para qué valor de $R$ y $\theta$ (en radianes) será mínima la longitud de la valla alrededor del perímetro del jardín?.\\\\
	Respuesta.-\; El área del jardín es:
	$$A=\dfrac{1}{2}r^2\theta\quad \Rightarrow \quad \theta=\dfrac{2A}{r^2}.$$
	Después, el perímetro del jardín está dado por:
	$$P=r\theta+2r.$$
	Luego, reemplazando por $\theta=\dfrac{2A}{r^2}$:
	$$P=\dfrac{2A}{r}+2r.$$
	Derivando con respecto a $r$,
	$$P'=\dfrac{-2A}{r^2}+2.$$
	Para hallar el valor de $r$ que minimiza $P$, igualamos $P'$ a cero,
	$$\begin{array}{rcl}
	    \dfrac{-2A}{r^2}+2&=&0\\\\
	    \dfrac{2A}{r^2}&=&-2\\\\
	    r^2&=&A\\\\
	    r&=&\sqrt{A}.
	\end{array}$$
	Reemplazando en $\theta=\dfrac{2A}{r^2}$, se tiene:
	$$\theta=\dfrac{2A}{(\sqrt{A})^2}=\dfrac{2A}{A}=2.$$\\

    %-------------------- 17.
    \item Un ángulo recto se desplaza a lo largo del diámetro de un círculo de radio a, tal como se muestra en la Figura 27 (Spivak, Capítulo 11). ¿Cuál es la mayor longitud posible $(A + B)$ interceptada por el círculo sobre el ángulo?.\\\\
	Respuesta.-\; Complementando el dibujo dado, podríamos construir:
	\begin{center}
	    \begin{tikzpicture}[scale=1]
		\draw[thick] (0,0) circle (2);
		\draw[thick] (-2,0) -- (2,0);
		\draw[thick] (-1.4,0) -- (-1.4,1.4)--(-.1,0);
		\draw(-1.4,0.7)node[left]{$A$};
		\draw(-.8,0)node[below]{$B-a$};
		\draw(-.8,1)node[below right]{$a$};
		\draw(.8,0)node[below]{$a$};
	    \end{tikzpicture}
	\end{center}
	Donde podemos definir lo siguiente:
	$$A=\sqrt{a^2-(B-a)^2}=\sqrt{2aB-B^2}$$
	$$S=A+B=\sqrt{2aB-B^2}+B.$$
	Derivando con respecto a $B$,
	$$S'=\dfrac{2a-2B}{\sqrt{2aB-B^2}}+1.$$
	Para la mayor longitud posible, igualamos $S'$ a cero,
	$$\begin{array}{rcl}
	    \dfrac{2a-2B}{2\sqrt{2aB-B^2}}&=&0\\\\
	    B-a&=&\sqrt{2aB-B^2}\\\\
	    B^2-2aB+a^2&=&2aB-B^2\\\\
	    2B^2-4aB+a^2&=&0\\\\
	    B&=&\dfrac{4a\pm \sqrt{16a^2-4\cdot 2a^2}}{2\cdot 2}\\\\
	    B&=&\dfrac{2a\pm \sqrt{2}a}{2}=\left(1+\pm \dfrac{\sqrt{2}}{2}\right)a.
	\end{array}$$
	Luego, sea $B=a\left(1+\dfrac{\sqrt{2}}{2}\right)$, entonces reemplazando en $A=\sqrt{2aB-B^2}$, se tiene
	$$\begin{array}{rcl}
	    A&=&\sqrt{2a^2\left(1+\dfrac{\sqrt{2}}{2}\right)-\left[a\left(1+\dfrac{\sqrt{2}}{2}\right)\right]^2}\\\\
	    A&=&\dfrac{\sqrt{2}a}{2}.
	\end{array}$$
	Sea ahora $B=a\left(1-\dfrac{\sqrt{2}}{2}\right)$, entonces reemplazando en $A=\sqrt{2aB-B^2}$, se tiene
	$$\begin{array}{rcl}
	    A&=&\sqrt{2a^2\left(1-\dfrac{\sqrt{2}}{2}\right)-\left[a\left(1-\dfrac{\sqrt{2}}{2}\right)\right]^2}\\\\
	    A&=&\dfrac{\sqrt{2}a}{2}.
	\end{array}$$
	\vspace{.5cm}

    %-------------------- 18.
    \item El ecólogo Ed debe cruzar un lago circular de $1$ milla de radio. Puede remar a través del lago a una velocidad de $2$ millas por hora, o caminar alrededor del lago a una velocidad de $4$ millas por hora; también puede remar un cierto trecho y completar el itinerario caminando (Figura 28, Spivak, capítulo 11). ¿Qué ruta debe tomar de manera que:

	\begin{enumerate}[(i)]

	    %---------- (i)
	    \item pueda ver la mayor cantidad de paisaje posible?.\\\\
		Respuesta.-\; Sea,
		\begin{center}
		    \begin{tikzpicture}[baseline=(current bounding box.north)]
		      \def\Radius{1.8}
		      \path
			(-\Radius, 0) coordinate (A)
			-- coordinate (M)
			(\Radius, 0) coordinate (B)
			(M) +(60:\Radius) coordinate (C)
			+(120:\Radius) coordinate (D)
		      ;
		      % Draw semicircle
			  \draw
			    (B) arc(0:180:\Radius) -- cycle
			  ;
			  % Annotations
			  \path[inner sep=0pt];
			  \draw(-1.8,0)--(1,1.5)--(0,0);
			  \draw(-1.5,-.1)node[above right]{$\theta$};
			  \draw(-.8,.7)node[above right]{$R$};
			  \draw(.1,-.1)node[above right]{$2\theta$};
			  \draw(.5,.5)node[above right]{$1m$};
			  \draw(1.5,.7)node[above right]{$w$};
		    \end{tikzpicture}
		\end{center}
		Sea $R$ que representa la distancia que el hombre recorrerá y $W$ la distancia que caminerá como se muestra en la figura anterior, de donde
		$$W=2r\theta = 2\theta$$
		Y
		$$R^2=1^2+1^2=-2\cdot 1\cdot 1 \cdot \cos(\pi-2\theta).$$
		Aplicando la identidad trigonométrica $\cos(\pi - \theta)=-\cos(\theta)$,
		$$R^2=2+2\cos(2\theta)\quad \Rightarrow \quad R=\sqrt{2+2\cos(2\theta)}.$$
		Luego, la longitud $D$ del camino total viene dada por:
		$$D=R+W=2\theta+\sqrt{2+2\cos(2\theta)}.$$
		Derivando con respecto a $\theta$,
		$$D'=2-\dfrac{4\sen(2\theta)}{2\sqrt{2+2\cos(2\theta)}}$$
		Para hallar la mayor longitud posible, igualamos $D'$ a cero,
		$$\begin{array}{rclr}
		    2-\dfrac{4\sen(2\theta)}{2\sqrt{2+2\cos(2\theta)}}&=&0&\\\\
		    \dfrac{\sen(2\theta)}{\sqrt{2+2\cos(2\theta)}}&=&1&\\\\
		    \sen(2\theta)&=&\sqrt{2+2\cos(2\theta)}&\\\\
		    \sen^2(2\theta)&=&2+2\cos(2\theta)&\\\\
			1-\cos^2(2\theta)-2\cos(2\theta)+1&=&0&\sen^2(\theta)=1-\cos^2(\theta)\\\\
							  \cos^2(2\theta)+2\cos(2\theta)+1&=&0&\\\\
							  \left[\cos(2\theta)+1\right]^2&=&0&\\\\
							  \cos(2\theta)&=&-1&\\\\
							  2\theta&=&\pi&2\theta=\arccos(-1)\\\\
							  \theta&=&\dfrac{\pi}{2}&
		\end{array}$$
		Reemplazando $\theta=\dfrac{\pi}{2}$ en $W=2\theta$ y $R=\sqrt{2+2\cos(2\theta)}$ se tiene
		$$W=\pi \qquad \mbox{y}\qquad R=\sqrt{2+2\cos(\pi)}=0.$$\\\\

	    %---------- (ii)
	    \item pueda cruzar tan rápida como sea posible?
		Respuesta.-\; Aplicando la fórmula,
		$$tiempo=\dfrac{distancia}{velocidad}$$
		Se tiene que el tiempo total es:
		$$T=\dfrac{W}{4}+\dfrac{R}{2}.$$
		Reeemplazando $W$ con $2\theta$ y $R$ con $\sqrt{2+2\cos(2\theta)}$,
		$$T=\dfrac{\theta}{2}+\dfrac{1}{2}\sqrt{2+2\cos(2\theta)}.$$
		Derivando con respecto a $\theta$,
		$$T'=\dfrac{1}{2}-\dfrac{\sen(2\theta)}{\sqrt{2+2\cos(2\theta)}}.$$
		Para el tiempo mínimo, igualamos $T'$ a cero,
		$$\begin{array}{rclr}
		    \dfrac{1}{2}-\dfrac{\sen(2\theta)}{\sqrt{2+2\cos(2\theta)}}&=&0&\\\\
		    2\sen(2\theta)&=&\sqrt{2+2\cos(2\theta)}&\\\\
		    4\sen^2(2\theta)&=&2+2\cos(2\theta)&\\\\
		    4-4\cos^2(2\theta)-2\cos(2\theta)-2&=&0&\sen^2(\theta)=1-\cos^2(\theta)\\\\
		    2\left[2-2\cos^2(2\theta)-\cos(2\theta)-1\right]&=&0\\\\
		    \left[2\cos(2\theta)-1\right]\left[\cos(2\theta)+1\right]&=&0&\\\\
		    \cos(2\theta)=\dfrac{1}{2}&\mbox{o}&\cos(2\theta)=-1&\\\\
		    2\theta=\arccos\left(\dfrac{1}{2}\right)&\mbox{o}&2\theta=\arccos(-1)&\\\\
		    \theta=\dfrac{\pi}{6}&\mbox{o}&\theta=\dfrac{\pi}{2}&
		\end{array}$$

		Para $\theta=\dfrac{\pi}{6},$
		$$W=\dfrac{\pi}{3}\qquad \mbox{y}\qquad R=\sqrt{2+2\cos\left(\dfrac{\pi}{3}\right)}=\sqrt{3}.$$
		Por lo tanto,
		$$T=\dfrac{\pi}{12}+\dfrac{\sqrt{3}}{2}.$$

		Para $\theta=\dfrac{\pi}{2},$
		$$W=\pi \qquad \mbox{y}\qquad R=\sqrt{2+2\cos(\pi)}=0.$$
		Por lo tanto,
		$$T=\dfrac{\pi}{4}.$$\\

	\end{enumerate}


    %-------------------- 19.
    \item 
	\begin{enumerate}[(a)]

	    %---------- (a)
	    \item Considere los puntos $A$ y $C$ de un círculo, con cetro $O$, formando un ángulo $\alpha=\angle AOC$ (figura 29, Spivak, capítulo 11). ¿Cómo debe elegirse el punto $B$ de manera que la suma de las áreas del triángulo $\triangle AOB$ y del triángulo $\triangle BOC$ sea máxima? Indicación: Exprese todo en función de $\theta = \angle AOB.$\\\\
		Respuesta.-\; La suma de las áreas de los dos triángulos es:
		$$\begin{array}{rcl}
		    S&=&\dfrac{1}{2}r^2\sen(\theta) + \dfrac{1}{2}r^2\sen(\alpha-\theta).\\\\
		     &=&\dfrac{1}{2}r^2\left[\sen(\theta)+\sen(\alpha-\theta)\right].\\\\
		\end{array}$$
		Diferenciando con respecto a $\theta$ se tiene,
		$$S'=\dfrac{1}{2}r^2\left[\cos(\theta)-\cos(\alpha-\theta)\right].$$
		Para que la suma de las áreas sea máxima, igualamos $S'$ a cero,
		$$\begin{array}{rcl}
		    \cos(\theta)-\cos(\alpha-\theta)&=&0\\\\
		    \cos(\theta)&=&\cos(\alpha-\theta)\\\\
		    \theta&=&\alpha-\theta\\\\
		    2\theta&=&\alpha
		\end{array}$$
		Por lo que el área máxima se tendra que poner $\theta=\dfrac{\alpha}{2}.$\\\\

	    %---------- (b)
	    \item Demuestre que para $n\geq 3$, de todos los n-ágonos inscritos en un círculo, el n-ágono regular es el de área máxima.\\\\
		Demostración.-\; Todo polígono inscrito en una circunferencia se puede dividir en triángulos isósceles como los del apartado (a) de este problema en el que probamos que el área máxima se da cuando los ángulos centrales son iguales.\\
		Generalizando este resultado para cualquier número de triángulos, encontramos que para el área máxima, todos los ángulos de los vértices deben ser iguales, por lo tanto, las bases son iguales y el polígono es regular.\\\\

	\end{enumerate}

    %-------------------- 20.
    \item Se dobla la esquina inferior derecha de una página de manera que coincida con el margen izquierdo del papel, como se muestra en la Figura 30 (Spivak, capítulo 11). Si la anchura del papel es $\alpha$ y la página es muy larga, demuestre que la longitud mínima del pliegue es $3\sqrt{3}\alpha/4$.\\\\
	Demostración.-\; Completando la figura 30, tenemos:
	\begin{center}
	    \begin{tikzpicture}[scale=1]
	    \end{tikzpicture}
	\end{center}
	Sea $\theta$ la medida del ángulo $BGF$. Como $\triangle BGF$ tiene un ángulo recto en $B$, entonces $m\angle BFG=90^{\circ}-\theta$. Luego, como $\triangle BGF$ es congruente con $\triangle EGF$, entonces $m\angle EFG=90^{\circ}-\theta.$ Se sigue,
	$$\begin{array}{rcl}
	    m\angle AFE&=&180^{\circ}-m\angle BFG-m\angle EFG\\
		       &=&180^{\circ}-(90^{\circ}-\theta)-(90^{\circ}-\theta)\\
		       &=&2\theta\\\\
	\end{array}$$
	Después podemos hallar el $\cos(2\theta)$, y hallar la longitud $L$.
	$$\begin{array}{rcl}
	    \cos(2\theta)&=&\dfrac{AF}{EF}=\dfrac{\alpha-L\sen \theta}{L\sen \theta}\\\\
			 L\sen(\theta)\cos(2\theta)&=&\alpha-L\sen(\theta)\\\\
			 L\sen(\theta)+L\sen(\theta)\cos(2\theta)&=&\alpha-L\sen(\theta)+L\sen(\theta)\\\\
			 L\sen(\theta)\left[\cos(2\theta)+1\right]&=&\alpha\\\\
			 \mbox{Aplicando el hecho de } && \cos(2\theta)=2\cos^2\theta-1 \Rightarrow 1+\cos(2\theta)=2\cos^2(\theta)\\\\ 
			 2L\sen(\theta)\cos^2(\theta)&=&\alpha\\\\
			 L&=&\dfrac{\alpha}{2\sen(\theta)\cos^2(\theta)}
	\end{array}$$

	Ahora diferenciando con respecto a $\theta$ se tiene,
	$$\begin{array}{rcl}
	    L'&=&-\dfrac{2\alpha\left[\cos(\theta)\cos^2(\theta)-2\sen^2(\theta)\cos(\theta)\right]}{\left[2L\sen(\theta)\cos^2(\theta)\right]^2}\\\\
	      &=&-\dfrac{2\alpha\left[\cos^3(\theta)-2\sen^2(\theta)\cos(\theta)\right]}{\left[2L\sen(\theta)\cos^2(\theta)\right]^2}\\\\
	\end{array}$$
	Para que la longitud sea mínima, igualamos $L'$ a cero,
	$$\begin{array}{rccl}
	    -\dfrac{2\alpha\left[\cos^3(\theta)-2\sen^2(\theta)\cos(\theta)\right]}{\left[2L\sen(\theta)\cos^2(\theta)\right]^2}&=&0&\\\\
	    \cos^3-2\sen^2(\theta)\cos(\theta)&=&0&\\\\
	    \cos^2(\theta)-2\sen^2(\theta)&=&0&\\\\\
							    1-3\sen^2\theta&=&0&\cos^2(\theta)=1+\sen^2(\theta)\\\\
							    \sen^2(\theta)&=&\dfrac{1}{3}&\\\\
							    \sen(\theta)&=&\dfrac{1}{\sqrt{3}}&\\\\
	\end{array}$$
	Luego, para hallar $\cos^2(\theta)$ podemos utilizar la identidad trigonométrica, $\cos^2(\theta)=1-\sen^2(\theta)$ y reemplazar $\sen^2(\theta)=\dfrac{1}{3}$, tenemos
	$$\cos^2(\theta)=1-\dfrac{1}{3}=\dfrac{2}{3}.$$
	Reemplazando en la expresión de $L$ se tiene,
	$$L=\dfrac{\alpha}{2\cdot\dfrac{1}{\sqrt{3}}\cdot\dfrac{2}{3}}=\dfrac{3\sqrt{3}\alpha}{4}.$$\\

    %-------------------- 21.
    \item La Figura 31 (Spivak, capítulo 11) muestra la gráfica de la derivada de $f$. Halle todos los puntos máximos y mínimos locales de $f$.\\\\
	Respuesta.-\; Por la gráfica notemos que:
	$$f'(1)=0.$$
	Por el corolario 11.3 y sus respectivas reglas de máximo y mínimo locales, podemos deducir que $f'$ es positivo cuando $x<1$ y negativo cuando $x>1$. Entonces $f(x)$ tiene un máximo local en $x=1$.\\
	Por otro lado vemos que:
	$$f'(3)=0.$$
	Ya que $f'$ es negativa cuando $x<3$ y positivo cuando $x>3$. Entonces por el corolario 11.3 y sus respectivas reglas de máximo y mínimo $f$ tiene un mínimo local en $x=3$.\\\\

    %-------------------- 22.
    \item Supongamos que $f$ es una función polinómica $f(x)=x^n+a_{n-1}x^{n-1}+\ldots + a_0$ con puntos críticos $-1,1,2,3,4$ y sus correspondientes valores críticos $6,1,2,4,3$. Trace la gráfica distinguiendo los casos $n$ par y $n$ impar.\\\\
	Respuesta.-\; Cuando $n$ es par, se tiene
	\begin{center}
	    \begin{tikzpicture}[scale=0.7]
		% abscisa y ordenada
		\tkzInit[xmax= 5,xmin=-3,ymax=8,ymin=-1]
		\tiny\tkzLabelXY[opacity=0.6,step=1, orig=false]
		% label x, f(x)
		\tkzDrawX[opacity= .6,label=x,right=0.3]
		\tkzDrawY[opacity= .6,label=f(x),below = -0.6]
		%puntos
		\filldraw[black](-1,6)node[below left]{$(-1,6)$} circle(2pt);
		\filldraw[black](1,1)node[below]{$(1,1)$} circle(2pt);
		\filldraw[black](2,2)node[above left]{$(2,2)$} circle(2pt);
		\filldraw[black](3,4)node[above]{$(3,4)$} circle(2pt);
		\filldraw[black](4,3)node[below]{$(3,4)$} circle(2pt);
		%lineas
		\draw[](-2.5,8)..controls(-1.3,6.1)..(-1,6)..controls(0,5.8)and(0,1)..(1,1);
		\draw[](1,1)..controls(1.3,1)and(1.5,2)..(2,2);
		\draw[](2,2)..controls(2.5,2)and(2.5,4)..(3,4);
		\draw[](3,4)..controls(3.3,4)and(3.5,3)..(4,3);
		\draw[](4,3)..controls(4.3,3)..(5,6);
	    \end{tikzpicture}
	\end{center}

	Para $n$ impar, se tiene
	
	\begin{center}
	    \begin{tikzpicture}[scale=0.7]
		% abscisa y ordenada
		\tkzInit[xmax= 5,xmin=-3,ymax=8,ymin=-1]
		\tiny\tkzLabelXY[opacity=0.6,step=1, orig=false]
		% label x, f(x)
		\tkzDrawX[opacity= .6,label=x,right=0.3]
		\tkzDrawY[opacity= .6,label=f(x),below = -0.6]
		%puntos
		\filldraw[black](-1,6)node[below left]{$(-1,6)$} circle(2pt);
		\filldraw[black](1,1)node[below]{$(1,1)$} circle(2pt);
		\filldraw[black](2,2)node[above left]{$(2,2)$} circle(2pt);
		\filldraw[black](3,4)node[above]{$(3,4)$} circle(2pt);
		\filldraw[black](4,3)node[below]{$(3,4)$} circle(2pt);
		%lineas
		\draw[](-2.5,3)..controls(-1.3,6)..(-1,6)..controls(0,5.8)and(0,1)..(1,1);
		\draw[](1,1)..controls(1.3,1)and(1.5,2)..(2,2);
		\draw[](2,2)..controls(2.5,2)and(2.5,4)..(3,4);
		\draw[](3,4)..controls(3.3,4)and(3.5,3)..(4,3);
		\draw[](4,3)..controls(4.3,3)..(5,6);
	    \end{tikzpicture}
	\end{center}
	\vspace{.5cm}

    %-------------------- 23.
    \item 
	\begin{enumerate}[(a)]

	    %---------- (a)
	    \item Suponga que los puntos críticos de la función polinómica $f(x)=x^n+a_{n-1}x^{n-1}+\ldots+a_0$ son $-1,1,2,3$ y que $f''(-1)=0$, $f''(2)<0$, $f''(3)=0$. Trace la gráfica de $f$ tan exacatamente como sea posible basándose en esta información.\\\\
		Respuesta.-\; Ya que, la segunda derivada de $x=1$ es positiva, entonces por el teorema 11.5 se tiene un mínimo local. Luego, como la segunda derivada en $x=2$ es positiva, entonces por el teorema 11.5 es un máximo local. Los puntos x=-1 y -3 son puntos de inflexión.
		Cuando $n$ es par, se tiene
		\begin{center}
		    \begin{tikzpicture}[scale=0.7]
			% abscisa y ordenada
			\tkzInit[xmax= 5,xmin=-3,ymax=8,ymin=-1]
			\tiny\tkzLabelXY[opacity=0.6,step=1, orig=false]
			% label x, f(x)
			\tkzDrawX[opacity= .6,label=x,right=0.3]
			\tkzDrawY[opacity= .6,label=f(x),below = -0.6]
			%puntos
			\filldraw[black](-1,6) circle(2pt);
			\filldraw[black](1,1) circle(2pt);
			\filldraw[black](2,2) circle(2pt);
			\filldraw[black](3,4) circle(2pt);
			\filldraw[black](4,3) circle(2pt);
			%lineas
			\draw[](-2.5,8)..controls(-1.3,6.1)..(-1,6)..controls(0,5.8)and(0,1)..(1,1);
			\draw[](1,1)..controls(1.3,1)and(1.5,2)..(2,2);
			\draw[](2,2)..controls(2.5,2)and(2.5,4)..(3,4);
			\draw[](3,4)..controls(3.3,4)and(3.5,3)..(4,3);
			\draw[](4,3)..controls(4.3,3)..(5,6);
		    \end{tikzpicture}
		\end{center}

		Para $n$ impar, se tiene
		
		\begin{center}
		    \begin{tikzpicture}[scale=0.7]
			% abscisa y ordenada
			\tkzInit[xmax= 5,xmin=-3,ymax=8,ymin=-1]
			\tiny\tkzLabelXY[opacity=0.6,step=1, orig=false]
			% label x, f(x)
			\tkzDrawX[opacity= .6,label=x,right=0.3]
			\tkzDrawY[opacity= .6,label=f(x),below = -0.6]
			%puntos
			\filldraw[black](-1,6) circle(2pt);
			\filldraw[black](1,1) circle(2pt);
			\filldraw[black](2,2) circle(2pt);
			\filldraw[black](3,4) circle(2pt);
			\filldraw[black](4,3) circle(2pt);
			%lineas
			\draw[](-2.5,3)..controls(-1.3,6)..(-1,6)..controls(0,5.8)and(0,1)..(1,1);
			\draw[](1,1)..controls(1.3,1)and(1.5,2)..(2,2);
			\draw[](2,2)..controls(2.5,2)and(2.5,4)..(3,4);
			\draw[](3,4)..controls(3.3,4)and(3.5,3)..(4,3);
			\draw[](4,3)..controls(4.3,3)..(5,6);
		    \end{tikzpicture}
		\end{center}
		\vspace{.5cm}

	    %---------- (b)
	    \item ¿Existe una función polinómica con las propiedades anteriores, excepto que $3$ no sea un punto crítico?.\\\\
		Respuesta.-\; Si $3$ no es uno de los puntos críticos, y $x=2$ es el máximo, entonces la función decrecerá en el intervalo $3$ hasta el infinito. Por lo tanto, la segunda derivada en $3$ no puede ser cero. Así, no es posible que exista una función polinómica con las propiedades anteriores.\\\\

	\end{enumerate}

    %-------------------- 24.
    \item Describa la gráfica de una función racional (en términos muy generales, análogamente a la descripción del texto de la gráfica de una función polinómica).\\\\
	Respuesta.-\; La gráfica de la función racional $f\left( x \right) = \dfrac{1}{x}$ tiene una asíntota vertical en $x=0$ y una asíntota horizontal en $y=0$. La gráfica de $f(x)=\dfrac{ax^n+\ldots}{bx^m + \ldots}$ tiene una asíntota vertical en $x=a$ si el denominador es cero en y el numerador no es cero.
	\begin{center}
	\begin{tabular}{rcl}
	    Si & $n<m$ & entonces la eje $x$ es la asíntota horizontal.\\
	    Si & $n=m$ & entonces la linea $y=\dfrac{a}{b}$ es la asíntota horizontal.\\
	    Si & $n>m$ & no habrá asíntotas horizontales.
	\end{tabular}
	\end{center}
	\vspace{.5cm}

    %-------------------- 25.
    \item 
	\begin{enumerate}[(a)]

	    %---------- (a)
	    \item Demuestre que dos funciones polinómicas de grados $m$ y $n$, respectivamente, se cortan a lo sumo en $\max(m,n)$ puntos.\\\\
		Demostración.-\; Sean dos funciones polinómicas $f$ de grado $n$ y $g$ de grado $m$ tal que $m\geq n$. Es decir,
		$$f(x)=a_nx^n+a_{n-1}x^{n-1}+\ldots + a_1x+a_0, \qquad g(x)=b_mx^m + b_{m-1}x^{m-1}+\ldots + b_1x+b_0,$$
		para $a_n\neq 0$ y $a_m\neq 0$. Luego vemos que,
		$$\begin{array}{rcl}
		    (f-g)(x) &=& f(x)-g(x)\\
			     &=& a_nx^n+a_{n-1}x^{n-1}+\ldots + a_1x+a_0 - \left(b_mx^m + b_{m-1}x^{m-1}+\ldots + b_1x+b_0\right)\\
			     &=& a_nx^n + \ldots +a_{m+1}x^{m+1} - b_mx^m + (a_m-b_m)x^m+\ldots + (a_1-b_1)x+(a_0+b_0)\\
		\end{array}$$
		Este  último ya que $n\geq m.$ Acá mostramos que $f-g$ es también un polinomio de grado $n=\max(m,n)$. Ahora, supongamos que $a$ es un punto de intersección de $f$ y $g$, si y sólo si, $f(a)=f(b)$. Podemos reescribir de la siguiente manera,
		$$(f-g)(a)=0.$$
		De esto, podemos deducir que los puntos de intersección de $f$ y $g$ son los ceros de $f-g$. Así, por el problema 7 parte (c) del capitulo 3 de Spivak tenemos que $f-g$ puede tener por lo mucho $n$ ceros. Como $n\geq m$, entonces $n=\max(m,n)$. Lo que demuestra que $f$ y $g$ se intersecan como máximo en $\max(m,n)$ puntos.\\\\
		

	    %---------- (b)
	    \item Para cada $m$ y $n$ muestre dos funciones polinómicas de grados $m$ y $n$ que se corten $\max(m,n)$ veces.\\\\
		Demostración.-\; Sea $p$ un polinomio de grado $n$ y $q(x)=x^m$ un polinomio de grado $m$ con $n\geq m$. Supongamos otro polinomio $f=p+q$, de la parte (a) tenemos que el grado de $f$ es $n$ ya que $n\geq m$. Luego, observemos que $p=f-q$ y el grado de $p$ es $n$. Así, $p=f-q$ pueden tener por lo mucho $n$ raíces, esto es, $f$ y $q$ se intersecan por lo mucho en $n=\max(m,n)$ veces.\\\\

	\end{enumerate}

    %-------------------- 26.
    \item Suponga que $f$ es una función polinómica de grado $n$ con $f\geq 0$ (por tanto $n$ debe ser par). Demuestre que $f+f'+f''+\ldots + f^{(n)}\geq 0.$\\\\
	Demostración.-\; Sea 
	$$h(x)=f(x)+f'(x)+\ldots + f^{(n)}(x).$$
	Diferenciando se tiene,
	$$h'(x)=f'(x)+f''(x)+\ldots + f^{(n)}(x)+f^{(n+1)}(x).$$
	Ya que, $f^{(n+1)}(x)=0$, entonces
	$$h'(x)=f'(x)+f''(x)+\ldots + f^{(n)}(x).$$
	Luego, supongamos
	$$g(x)=e^{-x}h(x).$$
	Diferenciando se tiene,
	$$\begin{array}{rcl}
	    g'(x)&=&e^{-x}\left[h'(x)-h(x)\right]\\\\
		 &=&-e^{-x}\left[h(x)-h'(x)\right]\\\\
		 &=&-e^{-x}f(x)\leq 0.
	\end{array}$$
	Entonces $g(x)$ es decreciente. Después, por el hecho de que $h(x)$ implica $\lim\limits_{x\to \infty} g(x)=0$, tenemos $g(x)\geq 0$. Por lo tanto $h(x)\geq 0$.\\\\

    %-------------------- 27.
    \item 
	\begin{enumerate}[(a)]

	    %---------- (a)
	    \item Suponga que la función polinómica $f(x)=x^n+a_{n-1}x^{n-1}+\ldots + a_0$ tiene exactamente $k$ puntos críticos y $f''(x)\neq 0$ para todos los puntos críticos $x$. Demuestre que $n-k$ es impar.\\\\
		Demostración.-\; Supongamos que nos dan la función $f$ con $k$ raíces, o cuya multiplicidad total de toda las raíces es $k$. Por el problema 7.4 Spivak, se sabe que $n-k$ es par. También sabemos que, una vez dados $n$ y $k$ tales que $n-k$ es par, existe alguna función polinómica de grado $n$ que tiene $k$ raíces, o cuya multiplicidad total de todas las raíces es $k$. Sea 
		$$f(x)=x^n+a_{n-1}x^{n-1}+\ldots + a_1x+a_0.$$
		Asumiendo que la función polinómica $f(x)$ tiene puntos críticos $k$  para los cuales $f''(x)\neq 0$. Lo que significa que $f'(x)$ tiene $k$ raíces únicas, (la unicidad se deduce del hecho de que $f''(x)\neq 0$). Ya que $f$ tiene grado $n$, $f'$ debería tener grado $n-1$. Luego por el hecho de que $f'(x)$ tiene $k$ raíces, se sigue que $n-1-k$ es par. Por lo tanto, no es difícil deducir que $n-k$ tendría que ser impar. \\\\

	    %---------- (b)
	    \item Para cada $n$, demuestre que existe una función polinómica $f$ de grado $n$ con $k$ puntos críticos, en cada uno de los cuales $f''$ es distinta de cero si $n-k$ es impart.\\\\
		Demostración.-\; Supongamos que para algunos números naturales $n$ y $k$, $n-k$ es impart. Esto significa que $n-k-1$ será par. De la parte discusión previa a la parte (a) se deduce que existe alguna función polinómica $g$ de grado $n-1$ con exactamente $k$ raíces. Sea $f$ la función tal que $f'=g$. De donde no es difícil deducir que esta función $f$ tiene grado $n$ y $k$ puntos críticos, que son en realidad raíces de la función $g$. Por lo tanto, la función $f$ es la función requerida.\\\\

	    %---------- (c)
	    \item Suponga que la función polinómica $f(x)=x^n+a_{n-1}x^{n-1}+\ldots + a_0$ tiene $k_1$ puntos máximos locales y $k_2$ puntos mínimos locales. Demuestre que $k_2=k_1+1$ si $n$ es par y $k_2=k;$ si $n$ es impar.\\\\
		Demostración.-\; Sea 
		$$f(x)=x^n+a_{n-1}x^{n-1}+\ldots + a_1x+a_0,$$
		un polinomio con $k_1$ puntos máximos locales y $k_2$ puntos mínimos locales. Notemos que $k=k_1+k_2$, que podemos ordenarlas en secuencias crecientes $a_1<a_2<\ldots < a_k$. Como el coeficiente principal de los polinomios es positivo, tenemos que $\lim_{x\to \infty} f(x)=\infty.$ Esto significa que la función es creciente a medida que tiende al infinito. Luego el punto crítico final $a_k$ debe ser el mínimo local, ya que la función es creciente a la derecha del mismo. De aquí se deduce que la función será decreciente a la izquierda de $a_k$. Decimos que el penúltimo punto crítico $a_{k-1}$ a  $k_1$ será máximo local ya que la función disminuirá a la derecha de él, y por tanto aumentará a la izquierda del mismo. Repitiendo esta deducción podemos deducir que los máximos y los mínimos cambiarán periódicamente. Es decir, los mínimos y máximos locales se distribuirán como en el gráfico siguiente.

		\begin{center}
		    \begin{tikzpicture}[scale=0.7]
			% abscisa y ordenada
			\tkzInit[xmax= 7,xmin=-1,ymax=4,ymin=-3]
			\tiny\tkzLabelXY[opacity=0.6,step=1, orig=false]
			% label x, f(x)
			\tkzDrawX[opacity= .6,label=x,right=0.3]
			\tkzDrawY[opacity= .6,label=f(x),below = -0.6]
			%puntos
			\filldraw[black](1.4,3) circle(2pt);
			\filldraw[black](2.63,-.5) circle(2pt);
			\filldraw[black](4.1,3) circle(2pt);
			\filldraw[black](5.52,-1.97) circle(2pt);
			%lineas
			%\draw[](-2.5,3)..controls(-1.3,6)..(-1,6)..controls(0,5.8)and(0,1)..(1,1);
			\draw[](0.7,-2)..controls(1.25,4.4)..(2.5,-.3);
			\draw[](2.5,-.3)..controls(2.65,-.8)..(3.5,2);
			\draw[](3.5,2)..controls(4.25,3.8)..(5.3,-1.5);
			\draw[](5.3,-1.5)..controls(5.6,-2.5)..(6.5,3);
		    \end{tikzpicture}
		\end{center}
		\vspace{.5cm}
		Ahora observaremos dos casos distintos, cuando $n$ es par y es impar.\\
		\begin{enumerate}[\textit{Caso} 1.]
		    \item Supongamos que $n$ es par. No es difícil deducir que $\lim_{x\to -\infty}f(x)=\infty.$ Si aplicamos la misma lógica que la anterior, podemos decir que $a_1$ será el mínimo local, ya que la función es creciente a la izquierda y decreciente a la derecha de $a_1$, así podremos deducir que $a_2$ será un máximo local. Por lo que es cierto que:
			$$a_n=\left\{\begin{array}{rr}
				\mbox{máximo local} & n=2m\\
				\mbox{mínimo local} & n=2m+1.
			\end{array}\right.$$
			Ya que $a_k$ es un mínimo local, basado en el anterior análisis, se sigue que $k$ es impar, es decir $k=2m+1$. Por lo tanto, los mínimos locales son $a_1,a_3,\ldots , a_k$ mientras que los máximos locales son $a_2,a_4,\ldots , a_{k_1}$.\\
		    \item Supongamos que $n$ es impar. No es difícil deducir que $\lim_{x\to -\infty}f(x)=-\infty.$ Si aplicamos la misma lógica que la anterior, podemos decir que $a_1$ será el máximo local, ya que la función es decreciente a la izquierda y creciente a la derecha de $a_1$, así podremos deducir que $a_2$ será un mínimo local. Por lo que es cierto que:
			$$a_n=\left\{\begin{array}{rr}
			    \mbox{máximo local} & n=2m+1\\
			    \mbox{mínimo local} & n=2m.
			\end{array}\right.$$
		Ya que $a_k$ es un mínimo local, basado en el anterior análisis, se sigue que $k$ es par, es decir $k=2m$. Por lo tanto, los mínimos locales son $a_1,a_3,\ldots , a_{k_1}$ mientras que los máximos locales son $a_2,a_4,\ldots , a_{k}$.\\\\
		\end{enumerate}
	    \item Queremos encontrar la función polinómica de grado $n$ con $k_1$ mínimos locales y $k_2$ máximos locales, donde $n,k_1,k_2$ son números dados. Sean $k=k_1+k_2$ y los números reales arbitrarios $a_1<a_2<\ldots < a_k$. Luego observemos la función
		$$g(x)=\prod_{i=i}^k (x-a_i)\left(1+x^2\right)^m$$
		donde $m=\frac{n-1-k}{2}$. No es difícil deducir que esta función tiene grado $n-1$. La razón por la que elegimos este tipo de funciones es porque $1+x^2>0$ para todo número real $x$ lo que implica que $(1+x^2)^m > 0$ y el signo de la función sólo depende del producto $\prod_{i=1}^k (x-a_i)$. Podemos notar que este producto es positivo siempre que $x\in(a_k,\infty)\cup (a_{k-1},a_k)\cup (a_{k-3},a_{k-2})\cup \ldots$. Mientras que será negativo si $x\in (a_{k-2},a_{k-1})\cup (a_{k-4},a_{k-3})\cup \ldots$. Sea la función $f$ tal que $f'(x)=g(x)$, de donde tenemos que $a_{k},a_{k-2},\ldots$ son mínimos locales mientras que $a_{k-1},a_{k-3},\ldots$ son máximos locales de la función $f$.\\\\
	\end{enumerate}

    %-------------------- 28.
    \item 
	\begin{enumerate}[(a)]

	    %---------- (a)
	    \item Demuestre que si $f'(x)>M$ para todo $x$ de $[a,b]$, entonces $f(b)\geq f(a)+M(b-a)$.\\\\
		Demostración.-\; Sea $h:[a,b]\to \mathbb{R}$ como sigue:
		$$h(x)=f(x)-\left[\dfrac{f(b)-f(a)}{b-a}\right](x-a).$$
		De donde notamos que $h$ es una función continua en $[a,b]$ ya que ambos, $f$ y $(x-a)$ son continuas. Además, $h$ es diferenciable en $(a,b)$, porque también $f$ y $(x-a)$ lo son. Luego observemos que,
		$$h(a)=f(a)-\left[\dfrac{f(b)-f(a)}{b-a}\right](a-a)=f(a)$$ 
		$$\mbox{y}\quad$$
		$$h(b)=f(b)-\left[\dfrac{f(b)-f(a)}{b-a}\right](b-a)=f(b)-f(b)+f(a)=f(a).$$
		Por lo que podemos utilizar el teorema de Rolle de la siguiente manera. Ya que $h$ es continua en $[a,b]$, diferenciable en $(a,b)$ y $h(a)=h(b)$, entonces existe un número $c$ en $(a,b)$ tal que $h'(c)=0$. Es decir,
		$$\begin{array}{rcl}
		    h(x)&=&f(x)-\dfrac{f(b)-f(a)}{b-a}x-a\dfrac{f(b)-f(a)}{b-a}\\\\
		    h'(x)&=&f'(x)-\dfrac{f(b)-f(a)}{b-a}\\\\
		    h'(c)&=&f'(c)-\dfrac{f(b)-f(a)}{b-a}=0.
		\end{array}$$
		Lo que implica por hipótesis que,
		$$\dfrac{f(b)-f(a)}{b-a}=f(c)\geq M \quad \Rightarrow \quad \dfrac{f(b)-f(a)}{b-a}\geq M.$$
		Así,
		$$f(b)\geq f(a)+M(b-a).$$\\

	    %---------- (b)
	    \item Demuestre que si $f'(x)\leq M$ para todo $x$ de $[a,b]$ entonces $f(b)\leq f(a)+M(b-a)$.\\\\
		Demostración.-\; Sea $h:[a,b]\to \mathbb{R}$ como sigue:
		$$h(x)=f(x)-\left[\dfrac{f(b)-f(a)}{b-a}\right](x-a).$$
		De donde notamos que $h$ es una función continua en $[a,b]$ ya que ambos, $f$ y $(x-a)$ son continuas. Además, $h$ es diferenciable en $(a,b)$, porque también $f$ y $(x-a)$ lo son. Luego observemos que,
		$$h(a)=f(a)-\left[\dfrac{f(b)-f(a)}{b-a}\right](a-a)=f(a)$$ 
		$$\mbox{y}\quad$$
		$$h(b)=f(b)-\left[\dfrac{f(b)-f(a)}{b-a}\right](b-a)=f(b)-f(b)+f(a)=f(a).$$
		Por lo que podemos utilizar el teorema de Rolle de la siguiente manera. Ya que $h$ es continua en $[a,b]$, diferenciable en $(a,b)$ y $h(a)=h(b)$, entonces existe un número $c$ en $(a,b)$ tal que $h'(c)=0$. Es decir,
		$$\begin{array}{rcl}
		    h(x)&=&f(x)-\dfrac{f(b)-f(a)}{b-a}x-a\dfrac{f(b)-f(a)}{b-a}\\\\
		    h'(x)&=&f'(x)-\dfrac{f(b)-f(a)}{b-a}\\\\
		    h'(c)&=&f'(c)-\dfrac{f(b)-f(a)}{b-a}=0.
		\end{array}$$
		Lo que implica por hipótesis que,
		$$\dfrac{f(b)-f(a)}{b-a}=f(c)\leq M \quad \Rightarrow \quad \dfrac{f(b)-f(a)}{b-a}\leq M.$$
		Así,
		$$f(b)\leq f(a)+M(b-a).$$\\

	    %---------- (c)
	    \item Formule un teorema análogo cuando $|f'(x)|\leq M$ para todo $x$ en $[a,b]$.\\\\
		Respuesta.-\; Para algún $M\in \mathbb{R}$, queremos hallar, 
		$$-M\leq f'(x)\leq M.$$
		Usando la parte (a) y (b) deducimos que,
		$$f(b)\geq f(a)-M(b-a)\quad \mbox{y}\quad f(b)\leq f(a)+M(b-a).$$
		Lo que implica,
		$$-M(b-a)\leq f(b)-f(a)\quad \mbox{y}\quad f(b)-f(a)\leq M(b-a).$$
		Así,
		$$-M(b-a)\leq f(b)-f(a)\leq M(b-a).$$
		Y por lo tanto,
		$$|f(b)-f(a)|\leq M|b-a|.$$\\

	\end{enumerate}

    %-------------------- 29.
    \item Suponga que $f'(x)\geq M>0$ para todo $x$ de $[0,1]$. Demuestre que existe un intervalo de longitud $\frac{1}{4}$ en el cual $|f|\geq M/4$.\\\\
	Demostración.-\; Ya que $f'>0$ en $[0,1]$, entonces sabemos que es continuo y estrictamente creciente, y  puede tomar el valor $0$ como máximo una vez. Se sigue que $f(x)\geq 0$ en $[1/2,1]$ o $f(x)\leq 0$ en $[0,1/2]$. Lo primero ocurre si toma el valor $0$ en algún punto menor que o igual a $1/2$; la segunda ocurre si toma el valor $0$ en algún punto mayor que o igual a $1/2$; y si no toma nunca el valor $0$, entonces o bien es siempre negativo en cuyo caso ocurre lo segundo, o siempre es positivo, en cuyo caso ocurre lo primero.\\
	Luego, suponga que $f(x)\geq 0$ en $[1/2,1]$, así por el teorema del valor medio, 
	$$\dfrac{f(3/4)-f(1/2)}{1/4}\geq M,$$ 
	así $f(3/4)-f(1/2)\geq M/4$, y ya que $f(1/2)\leq 0$ se sigue que $-f(1/4)\geq M/4$ o $f(1/4)\leq -M/4$. Pero sabiendo que $f$ es estrictamente creciente, tenemos $f(x)\leq-M/4$ para todo $x\in[0,1/4]$, por lo tanto $|f|>M/4$ en un intervalo de longitud $1/4$.\\\\

    %-------------------- 30.
    \item 
	\begin{enumerate}[(a)]

	    %---------- (a)
	    \item Suponga que $f'(x)>g'(x)$ para todo $x$, y que $f(a)=g(a)$. Demuestre que $f(x)>g(x)$ para $x>a$ y $f(x)<g(x)$ para $x<a$.\\\\
		Demostración.-\; Consideremos la función $h=f-g$, la cual es una función diferenciables. Ya que $f'(x)>g'(x)$ para todo $x$, entonces $h'(x)=f'(x)-g'(x)$, así  $h'(x)>0$ para todo $x$.\\
		Sea $x>a$. Por el teorema del valor medio, existe $c\in(a,x)$ tal que
		$$\dfrac{h(x)-h(a)}{x-a}=h'(c)>0,$$
		de donde $h(x)>h(a)$. Por el hecho de que $f(a)=g(a)$, se tiene $h(a)=0$, lo que implica $h(x)>0$. Por lo tanto concluimos que $f(x)>g(x)$.\\
		Por otro lado, sea $x<a$. Por el teorema del valor medio, existe $c\in(x,a)$ tal que
		$$\dfrac{h(a)-h(x)}{a-x}=h'(c)>0,$$
		así $h(x)<h(a)=0$ y $f(x)<g(x)$.\\\\


	    %---------- (b)
	    \item Demuestre mediante un ejemplo que estas conclusiones no son válidas sin la hipótesis $f(a)=g(a)$.\\\\
		Demostración.-\; Sea $f(x)=10x$, $g(x)=5x+100$ y tomemos $a=0$. Tenemos,
		$$f'(x)10>5=g'(x),\; \forall x.$$
		Pero no se cumple para todo $x$ tal que $f(x)>g(x)$, ya que $f$ comienza a ser mayor que $g$ cuando $x=20$.\\
		De manera similar, se cumplirá que $f(x)<g(x)$ cuando $x=-20$.\\
		No es posible encontrar un sólo ejemplo de un par de funciones $f$ y $g$ con un número $a$ tal que $f'(x)>g'(x)$ para todo $x$, ambos
		$$f(x)>g(x),\;\forall x>a\qquad \mbox{y}\qquad f(x)<g(x),\; \forall x<a$$
		siendo falsos. En su lugar, si $f(a)=g(a)$ entonces ambas afirmaciones son verdaderas. Si $f(a)>g(a)$, entonces repitiendo la primera parte de la prueba anterior, cuando se tenga $h(x)>h(a)$, tenemos $h(a)=f(a)-g(a)>0$, y así, todavía $h(x)>0$ o $f(x)>g(x)$ para $x>a$. Similarmente, si $f(a)<g(a)$ entonces por la segunda parte de la demostración anterior se tiene $f(x)<g(x)$ para $x<a.$\\\\

	    %---------- (c)
	    \item Suponga que $f(a)=g(a)$, que $f'(x)\geq g'(x)$ para todo $x$ y que $f'(x_0)>g'(x_0)$ para algún $x_0>a$. Demuestre que $f(x)>g(x)$ para todo $x\geq x_0$.\\\\
		Demostración.-\; Si $f(x_0)=g(x_0)$, entonces la prueba nos será la misma que la parte (a), inmediatamente nos permite concluir que $f(x)>g(x)$ para todo $x>x_0$.\\ 
		Si $f(x_0)>g(x_0)$, entonces por la parte final de $(b)$ nos permite concluir que $f(x)>g(x)$ para todo $x>x_0$. Así, que terminaremos la demostración si podemos demostrar que $f(x_0)\geq g(x_0)$. Aplicando el teorema del valor medio para $h$ en $[a,x_0]$ tenemos
		$$\dfrac{h(x_0)-h(a)}{x_0-a}=h'(c)\geq 0,$$
		así, $h(x_0)\geq h(a)=0$, en efecto $f(x_0)\geq g(x_0)$.\\\\

	\end{enumerate}

    %-------------------- 31.
    \item Halle las funciones $f$ tales que
	\begin{enumerate}[(a)]

	    %---------- (a)
	    \item $f'(x)=\sen x.$\\\\
		Respuesta.-\; Podemos considerar la función $f(x)=-\cos x + c$, para alguna $c$ constante. Luego,
		$$f'(x)=-(-\sen x)=\sen x.$$\\

	    %---------- (b)
	    \item $g''(x)=x^3$.\\\\
		Respuesta.-\; Consideremos la función $g(x)=\dfrac{x^5}{15}+x+c$ para alguna $c$ constante. Luego, tenemos la primera diferenciación como:
		$$g'(x)=\dfrac{5x^4}{15}+1=\dfrac{x^4}{3}+1$$
		La segunda diferenciación será:
		$$g''(x)=\dfrac{3x^3}{3}=x^3.$$\\

	    %---------- (c)
	    \item $f'''(x)=x+x^2$.\\\\
		Respuesta.-\; Podemos considerar la función $f(x)=\dfrac{x^4}{24}+\dfrac{x^5}{60}+x^2+x+c$ para alguna $c$ constante. Luego, tenemos la primera diferenciación como:
		$$f'(x)=\dfrac{x^3}{6}+\dfrac{x^4}{12}+x+1.$$
		La segunda diferenciación será:
		$$f''(x)=\dfrac{x^2}{2}+\dfrac{x^3}{3}+1.$$
		Y la tercera diferenciación será:
		$$f'''(x)=x+x^2.$$\\

	\end{enumerate}

    %-------------------- 32.
    \item Si bien es verdad que un peso que se suelta partiendo del reposo caerá $s(t) = 16r^2$ pies en $t$ segundos, este hecho experimental no menciona el comportamiento de los pesos que son lanzados hacia arriba o hacia abajo. Por otra parte, la ley $s''(t) = 32$ se cumple siempre y tiene la ambigüedad suficiente para explicar el comportamiento de un peso soltado desde cualquier altura y con cualquier velocidad inicial. Para mayor sencillez convengamos en medir las alturas hacia arriba desde el nivel del suelo; en este caso las velocidades son positivas para cuerpos que se elevan y negativas para cuerpos que caen, y todos los cuerpos caen según la ley $s''(t) = -32$.\\
	\begin{enumerate}[(a)]

	    %---------- (a)
	    \item Demuestre que $s$ es de la forma $s(t)=-16t^2+\alpha t + \beta.$\\\\
		Demostración.-\; Se tiene,
		$$\begin{array}{rcl}
		    s''(t)&=&-32\\
		    s'(t)&=&-32t + \alpha\\
		    s(t)&=&-16t^2+\alpha t + \beta.
		\end{array}$$
		\vspace{.5cm}

	    %---------- (b)
	    \item Haciendo $t=0$ en la fórmula para $s$, y después en la fórmula para $s'$, demuestre que $s(t)=-16t^2+v_0 t + s_0$, donde $s_0$ es la altura desde la cual el cuerpo es soltado en el tiempo $0$, y $v_0$ es la velocidad con el cual se suelta.\\\\
		Demostración.-\; Reemplazando $t=0$ en la ecuación para $s'(t)$ tenemos
		$$s'(0)=v_0=\alpha.$$
		Luego, reemplazando $t=0$ en la ecuación para $s(t)$ se tiene
		$$s(0)=s_0=\beta.$$
		Substituyendo los valores de $\alpha$ y $\beta$ en la ecuación para $s(t)$,
		$$s(t)=-16t^2+v_0t+s_0.$$\\

	    %---------- (c)
	    \item Se lanza un peso hacia arriba con una velocidad de $v$ pies por segundo desde el nivel del suelo. ¿A qué altura llegará? (A qué altura significa ¿cuál es la máxima altura para todos los tiempos?) ¿Cuál es su velocidad en el momento en que alcanza su altura máxima? ¿Cuál es la aceleración en dicho momento? ¿Cuándo llegará otra vez al suelo? ¿Cuál será su velocidad en el momento de alcanzar el suelo?.\\\\
		Respuesta.-\; Sea $s_0=0$ y $v_0=v$,
		$$\begin{array}{rcl}
		    s(t)&=&-16t^2+vt\\
		    s'(t)&=&-32t+v.
		\end{array}$$
		Para la altura máxima. Sea $s'(t)=0$, de donde
		$$\begin{array}{rcl}
		    -32t+v=0 &\Rightarrow & 32t=v\\
			     &\Rightarrow& t=\dfrac{v}{32}.
		\end{array}$$
		Reemplazando $t=\dfrac{v}{32}$ en la ecuación para $s(t)$ conseguimos la distancia máxima,
		$$\begin{array}{rcl}
		    s\left(\dfrac{v}{32}\right) &=& -16\left(\dfrac{v}{32}\right)^2 + v\left(\dfrac{v}{32}\right)\\\\
						&=& -\dfrac{v^2}{64} + \dfrac{v^2}{32}\\\\
						&=& \dfrac{v^2}{64}.
		\end{array}$$

		La velocidad en la altura máxima será:
		$$s'\left(\dfrac{v}{32}\right)=-\dfrac{32v}{32}+v=0.$$

		La aceleración en la altura máxima será:
		$$s''\left(\dfrac{v}{32}\right)=-32.$$

		Para encontrar el momento en el que el peso vuelve a tocar el suelo, sea $s(t)=0$, es decir
		$$-16t^2+vt=0.$$
		de donde,
		$$t(-16t+v)=0\quad \Rightarrow \quad t=0\quad \mbox{o} \quad t=\dfrac{v}{16}.$$
		Por lo tanto el peso tocará el suelo de nuevo en 
		$$t=\dfrac{v}{16}.$$\\

	\end{enumerate}

    %-------------------- 33.
    \item Una bala de cañón se lanza desde el suelo con velocidad v y según un ángulo a (Figura 32) de modo que su componente vertical de velocidad es $v \sen \alpha$ y la componente horizontal $v\cos \alpha$. Su distancia $s(t)$ sobre el nivel del suelo obedece a la ley $s(t) = -16t^2 + (v \sen \alpha)t$, mientras que su velocidad horizontal se mantiene con el valor constante $v \cos \alpha$.\\
	\begin{enumerate}

	    %---------- (a)
	    \item Demuestre que la trayectoria de la bala es una parábola (halle la posición para cada tiempo $t$, y demuestre que estos puntos están sobre una parábola).\\\\
		Demostración.-\; Sea $x$ el desplazamiento horizontal en el tiempo $t$, entonces
		$$x=(v\cos \alpha)t \quad \Rightarrow \quad t=\dfrac{x}{v\cos \alpha}.$$
		Dado que la distancia vertical sobre el suelo es,
		$$s(t9=-16t^2+(v\sen \alpha)t.$$
		Reemplazando $t$ con $\dfrac{x}{v\cos \alpha}$, se tiene:
		$$\begin{array}{rcl}
		    s(t) &=& -16\left(\dfrac{x}{v\cos \alpha}\right)^2 + (v\sen \alpha)\left(\dfrac{x}{v\cos \alpha}\right)\\\\
			 &=& -\dfrac{16}{v^2\cos^2 \alpha}x^2 + (\tan \alpha)x.
		\end{array}$$
		El cual tiene la forma 
		$$y=ax^2+bx+c$$
		Lo que $-\dfrac{16}{v^2\cos^2 \alpha}x^2 + (\tan \alpha)x$ representa una parábola.\\\\

	    %---------- (b)
	    \item Halle el ángulo $\alpha$ que hace máxima la distancia horizontal recorrida por la bala antes de alcanzar el suelo.\\\\
		Respuesta.-\; La distancia horizontal $x$ viene dado por:
		$$x=(v\cos \alpha)t.$$
		Diferenciando ambos lados con respecto a $\alpha$, obtenemos:
		$$x'=-(v\sen \alpha)t.$$
		Para poder minimizar la distancia horizontal, igualemos $x'=0$,
		$$\begin{array}{rcl}
		    -(v\sen \alpha)t &=& 0\\
			  \sen \alpha&=&0\\
			  \alpha&=&0.
		\end{array}$$
		\vspace{0.5cm}

	\end{enumerate}

    %-------------------- 34.
    \item 
	\begin{enumerate}

	    %---------- (a)
	    \item Dé un ejemplo de una función $f$ para la cual exista el $\lim\limits_{x\to \infty}f(x)$, pero no exista el $\lim\limits_{x\to \infty}f'(x)$.\\\\
		Respuesta.-\; Sea 
		$$f(x)=\dfrac{\sen\left(x^2\right)}{x}.$$
		De donde,
		$$\lim_{x\to infty}f(x)=0,$$
		Así,
		$$\begin{array}{rcl}
		    f'(x) &=& \dfrac{2x^2\sen\left(x^2\right)-\sen\left(x^2\right)}{x^2}\\\\
			  &=& 2\sen\left(x^2\right)-\dfrac{\sen\left(x^2\right)}{x^2}.
	    \end{array}$$
	    pero, $\lim\limits_{x\to \infty}f'(x)$ no existe.\\\\


	    %---------- (b)
	    \item Demuestre que si existen el $\lim\limits_{x\to \infty} f(x)$ y el $\lim\limits_{x\to \infty}f'(x)$, entonces $\lim\limits_{x\to \infty}f'(x)=0$.\\\\
		Demostración.-\; Sea $f$ una función diferenciable tales que $\lim\limits_{x\to \infty} f(x)$ y $\lim\limits_{x\to \infty}f'(x)$ existe. Entonces se tiene que mostrar que
		$$\lim_{x\to \infty}f'(x)=0.$$
		Sea $\lim\limits_{x\to \infty}f'(x)=a.$ Entonces, por la definición de límite, tenemos que existe un número natural $n\in \mathbb{N}$ tal que, para todo $n\geq N$
		$$|f'(x)-a|<\dfrac{|a|}{2},$$
		el cual es equivalente a
		$$-\dfrac{|a|}{2}<f'(x)-a<\dfrac{|a|}{2},\quad \mbox{para todo } n\geq N.$$
		Ahora, sea $y\in \mathbb{R}$ tal que $y>N$. Entonces usando el teorema del valor medio para $f$ en el intervalo $[N,y]$ obtenemos que existe un número real $x_0\in [N,y]$ tal que
		$$f(y)-f(N)=f'(x_0)(y-N),$$
		lo que implica que
		$$f(y)=f(N)+f'(x_0)(y-N)>f(N)+\left(a-\dfrac{|a|}{2}\right)(y-N).$$
		Ahora, la desigualdad anterior se cumple para todo $y>N$. Así, como $y\to \infty$, $f(y)$ se vuelve sin límites si $a\neq 0$. Por lo tanto, $\lim\limits_{y\to \infty}$ no existe si $a\neq 0$. A partir de la hipótesis tenemos que $\lim\limits_{y\to \infty}f(y)$ existe, de donde concluimos que $a=0$. Lo que complementa la demostración.\\\\

	    %---------- (c)
	    \item Demuestre que si existe el $\lim\limits_{x\to \infty}f(x)$ y existe el $\lim\limits_{x\to \infty} f''(x)$, entonces el $\lim\limits_{x\to \infty}f''(x)=0$. (Vea también el problema 20-22).\\\\
		Demostración.-\; Sea $f$ una función diferenciable tales que $\lim\limits_{x\to \infty}f(x)$ y $\lim\limits_{x\to \infty}f''(x)$ existe. Entonces, tenemos que demostrar que
		$$\lim_{x\to \infty}f''(x)=0.$$
		Luego, asumamos que
		$$\lim_{x\to \infty}f'(x)=0.$$
		Por la parte (b) tenemos que existe un número natural $N\in \mathbb{N}$ tal que 
		$$f'(x)>f'(N)+\left(a-\dfrac{|a|}{2}\right)(x-N), \mbox{ para todo }x>N.$$
		Esto muestra que
		$$\lim_{x\to \infty}f'(x)=\infty.$$
		Si $a\neq 0$. Por el teorema del valor medio para $f$ en el intervalo $[0,x]$ con $x\in \mathbb{R}$ y $x>N$ obtenemos
		$$f(x)-f(0)=f'(x_0)x,\mbox{ para algún }x_0\in[0,x].$$
		Lo que implica,
		$$f(x)=f(0)+f'(x_0)x,\mbox{ para algún }x_0\in[0,x],$$
		lo que a su vez se tiene
		$$\lim_{x\to \infty}f(x)=f(0)+f'(x_0)\lim_{x\to \infty}x=\infty.$$
		Pero esto contradice nuestra hipótesis en $f'$. Así tenemos que $a=0$, lo que completa la demostración.\\\\

	\end{enumerate}

    %-------------------- 35.
    \item Suponga que $f$ y $g$ son dos funciones diferenciables que satisfacen $fg'-f'g=0$. Demuestre que si $f(a)=0$ y $g(a)\neq 0$, entonces $f(x)=0$ para todo $x$ de un intervalo alrededor de $a$. Indicación: Demuestre que en cualquier intervalo en el que $f/g$ esté definida, es constante.\\\\
	Demostración.-\; Para cualquier intervalo en el que se defina $f/g$, se tiene
	$$\left(\dfrac{f}{g}\right)'=\dfrac{f'g-fg'}{g^2}.$$
	Por el hecho de que $fg'-f'g=0$, 
	$$\left(\dfrac{f}{g}\right)'=0.$$
	Por lo tanto, $f/g$ es constante.\\\\

    %-------------------- 36.
    \item Suponga que $|f(x)-f(y)|\leq |x-y|^n$ para $n>1$. Demuestre que $f$ es constante considerando $f'$. Compare con el problema 3.20.\\\\
	Demostración.-\; Ya que $|f(x)-f(y)|\leq |x-y|^n$ para $n>1$, entonces
	$$\dfrac{|f(x)-f(y)|}{|x-y|}\leq |x-y|^{n-1},\mbox{ para } n>1.$$
	Lo que equivale a
	$$0\leq \dfrac{|f(x)-f(y)|}{|x-y|}\leq |x-y|^m \mbox{ para }m>0.$$
	Luego notemos que
	$$\lim_{x\to x}\dfrac{|f(x)-f(y)|}{|x-y|}=0.$$
	Ya que, la función es continua, entonces
	$$\bigg|\lim_{y\to x}\dfrac{f(x)-f(y)}{x-y}\bigg|=0.$$
	Lo que implica que 
	$$\lim_{y\to x}\dfrac{f(x)-f(y)}{x-y}=0.$$
	Por lo tanto,
	$$f'(x)=0.$$\\

    %-------------------- 37.
    \item Una función $f$ es Lipschitz de orden $\alpha$ en $x$ si existe una constante $C$ tal que
    $$(*)\qquad |f(x)-f(y)|\leq C|x-y|^\alpha$$
    para todo $y$ de un intervalo alrededor de $x$. La función $f$ de Lipschitz de orden $\alpha$ en un intervalo si $(*)$ se verifica para todo $x$ e $y$ del intervalo.
	\begin{enumerate}[(a)]

	    %---------- (a)
	    \item Si $f$ es Lipzchitz de orden $\alpha>0$ en $x$, entonces $f$ es continua en $x$.\\\\
		Demostración.-\; Ya que $f$ es una función de Lipschitz de orden $\alpha>0$ en $x$. Por definición, existe un $\delta'>0$ tal que 
		$$|f(x)-f(y)|\leq C|x-y|^\alpha, \mbox{ para todo } y \in (x-\delta',x+\delta')\qquad (1)$$
		para alguna constante $C\geq 0$ (dado que $|f(x)-f(y)|,\;|x-y|\geq 0$). Sea $\epsilon>0$, entonces tomamos $0<\delta<\frac{1}{2}\min\left\{\left(\frac{\epsilon}{C}\right)^{\frac{1}{\alpha}},\delta'\right\}$. Por (1) ya que $\delta<\delta'$, tenemos en $y\in(x-\delta,x+\delta)$, que
		$$|f(x)-f(y)|\leq C|x-y|^{\alpha}. \qquad (2)$$
		Además, vemos que si $y\in (x-\delta,x+\delta)$, entonces
		$$|x-y|<2\delta<2\dfrac{1}{2}\left(\dfrac{\epsilon}{C}\right)^{\frac{1}{\alpha}}=\left(\dfrac{\epsilon}{C}\right)^{\frac{1}{\alpha}}.$$
		Por lo tanto, por (2) tenemos que
		$$|f(x)-f(y)|\leq C |x-y|^{\alpha}<C\left[\left(\dfrac{\epsilon}{C}\right)^{\frac{1}{\alpha}}\right]^\alpha = \epsilon.$$
		Esto muestra que
		\begin{center}
		    $|f(x)-f(y)|<\epsilon$ cuando $|x-y|<\delta$.
		\end{center}
		Así, $f$ es continua en $x$.\\\\


	    %---------- (b)
	    \item Si $f$ es Lipzchitz de orden $\alpha>0$ en un intervalo, entonces $f$ es uniformemente continua en este intervalo. (Vea el Capítulo 8, Apéndice.)\\\\
		Demostración.-\; Sea $\epsilon>0$. Entonces tomemos $0<\delta<\left(\dfrac{\epsilon}{C}\right)^{\frac{1}{\alpha}}$. Ahora ya que $f$ es Lipschitz de orden $\alpha>0$ en $[a,b]$ tenemos que 
		\begin{center}
		    $|f(x)-f(y)|\leq C|x-y|^\alpha$, para todo $x,y\in [a,b]. \qquad (3)$
		\end{center}
		Luego, sea $x,y$ cualesquiera puntos en $[a,b]$ tal que $|x-y|<\delta$. Entonces, vemos por (3), que
		$$|f(x)-f(y)|\leq C|x-y|^{\alpha}<C\delta^\alpha<C\left[\left(\dfrac{\epsilon}{C}\right)^{\frac{1}{\alpha}}\right]^\alpha=\epsilon,$$
		siempre que $|x-y|<\delta$. Así, $f$ es uniformemente continua en $[a,b]$.\\\\

	    %---------- (c)
	    \item Si $f$ es diferenciable en $x$, entonces $f$ es Lipzchitz de orden $1$ en $x$. ¿Es cierta la recíproca?.\\\\
		Demostración.-\; Como $f$ es una función diferenciable en $x$. Entonces, por definición de diferenciabilidad
		$$f'(x)=\lim_{y\to x}\dfrac{f(x)-f(y)}{x-y}=l, \mbox{ para algún } l\in \mathbb{R}.$$
		Después, por definición de límite,  para todo $\epsilon>0$, existe $\delta>0$ tal que
		$$\bigg|\dfrac{f(x)-f(y)}{x-y}-l\bigg|<\epsilon \mbox{ siempre que } |x-y|<\delta.$$
		Luego, por la desigualdad triangular
		$$\bigg|\dfrac{f(x)-f(y)}{x-y}-f'(x)\bigg| \geq \bigg|\bigg|\dfrac{f(x)-f(y)}{x-y}\bigg|-|f'(x)|\bigg|.$$
		Por tanto,
		$$\bigg|\bigg|\dfrac{f(x)-f(y)}{x-y}\bigg|-|f'(x)|\bigg|<\epsilon \mbox{ siempre que } |x-y|<\delta.$$
		De ello, se tiene que si $|x-y|<\delta,$ entonces
		$$\sqrt{\left(\bigg|\dfrac{f(x)-f(y)}{x-y}\bigg|-|f'(x)|\right)^2}<\epsilon \quad \Rightarrow \quad \bigg|\dfrac{f(x)-f(y)}{x-y}\bigg|-|f'(x)|<\sqrt{\epsilon^2}=|\epsilon|=\epsilon.$$
		Lo que implica 
		$$|f(x)-f(y)|<\left(|f'(x)|+\epsilon\right)|x-y|\mbox{ siempre que } |x-y|<\delta.$$
		Ahora, sea $C\geq\left(|f'(x)|+\epsilon\right)$, de donde
		$$|f(x)-f(y)|<C|x-y|\mbox{ siempre que } |x-y|<\delta,$$
		por lo tanto, $f$ es Lipschitz de orden $1$ en $x$.\\
		El recíproco no es cierto. Por ejemplo se puede tomar la función $f(x)=|x|$. Entonces $f$ es Lipschitz de orden $1$ en cualquier punto $x$, ya que
		$$|f(x)-f(y)|=||x|-|y||\leq |x-y|,$$
		donde la última desigualdad se cumple a partir de la desigualdad triangular. Pero $f$ no es diferenciable en $0$.\\\\

	
	    %---------- (d)
	    \item Si $f$ es diferenciable en $[a,b]$. ¿es $f$ Lipschiz de orden $1$ en $[a,b]$?.\\\\
		Respuesta.-\; Si $f$ es diferenciable en $[a,b]$, entonces $f$ puede no ser Lipschitz de orden $1$ en $[a,b]$. Por ejemplo, sea 
		$$f(x)=x^2\sen \dfrac{1}{x^2}.$$
		Podemos observar que $f$ es diferenciable en $[0,1]$, pero no es Lipschitz de orden $1$ en $[0,1]$.\\\\

	    %---------- (e)
	    \item Si $f$ es Lipschitz de orden $\alpha>1$ en $[a,b]$, entonces $f$ es constante en $[a,b]$.\\\\
		Demostración.-\; Sea $f$ una función Lipschitz de orden $\alpha>1$ en $[a,b]$. Entonces,tenemos que mostrar que $f$ es constante en $[a,b]$. Por definición de función Lipschitz se tiene,
		$$|f(x)-f(y)|\leq C|x-y|^\alpha, \; \mbox{para todo }x,y\in[a,b]$$
		con $C\geq 0$ constante. Como $|f(x)-f(y)|\leq |x-y|^{\alpha}$ para $\alpha>1$ tenemos
		$$\dfrac{|f(x)-f(y)|}{|x-y|}\leq |x-y|^{\alpha-1},\; \mbox{para } \alpha>1$$
		Que equivale a,
		$$0\leq \dfrac{|f(x)-f(y)|}{|x-y|}\leq |x-y|^\beta,\mbox{ para }\beta>0.$$
		Del problema 13 del capítulo 5 de Spivak, 
		$$\lim_{y\to x}\dfrac{|f(x)-f(y)|}{|x-y|}=0.$$
		De donde,
		$$\bigg|\lim_{y\to x}\dfrac{f(x)-f(y)}{x-y}\bigg|=0,$$
		lo que implica
		$$\lim_{y\to x}\dfrac{f(x)-f(y)}{x-y}=0.$$
		Por lo tanto, 
		$$f'(x)=0$$
		y $f$ es una función constante.\\\\

	\end{enumerate}

    %-------------------- 38.
    \item Demuestre que si
    $$\dfrac{a_0}{1}+\dfrac{a_1}{2}+\ldots+\dfrac{a_n}{n+1}=0,$$
    entonces
    $$a_0+a_1x+\ldots + a_n x^n = 0$$
    para algún $x$ de $(0,1)$.\\\\
	Demostración.-\; Consideremos la función $f:(0,1)\to (0,1)$ definida por:
	$$f(x)=a_0x+\dfrac{a_1x^2}{2}+\ldots+\dfrac{a_nx^{n+1}}{n+1}.$$
	Entonces, vemos que al ser una función polinomial $f$ es diferenciable en $(0,1)$. Es más,
	$$f(0)=0=f(1)=\dfrac{a_0}{1}+\dfrac{a_1}{2}+\ldots + \dfrac{a_n}{n+1}.$$
	Donde la última igualdad se cumple a partir de la hipótesis. Así, aplicando el teorema de Rolle a $f$ en $(0,1)$ se tiene
	$$f'(x)=0$$
	para algún $x\in(0,1)$. Ahora bien, calculamos que
	$$f'(x)=a_0+a_1x+\ldots + a_n x^n.$$
	Lo que completa la demostración.\\\\

    %-------------------- 39.
    \item Demuestre que, cualquiera que sea $m$, la función polinómica $f_m(x) = x 3 - 3x + m$ no tiene nunca dos raíces en $[0, 1]$.(Esto es una consecuencia fácil del Teorema de Rolle. Resulta instructivo, una vez efectuada la demostración analítica, trace las gráficas de $f_0$ y $f_1$, y considerar la posición de la gráfica de $f_m$ en relación con ellas.)\\\\
	Demostración.-\; Supongamos que $f_m$ tiene dos raíces in $[0,1]$, digamos $x_0$ y $x_1$ con $0\leq x_0<x_1\leq 1$. Entonces se tiene,
	$$f_m(x_0)=f_m(x_1)=0.$$
	Además, $f_m$ es diferenciable en $[0,1]$. Aplicando el teorema de Rolle, vemos que existe $x\in(x_0,x_1)$ tal que
	$$f_m'(x)=0.$$
	Ya que, $0\leq x_0<x_1\leq 1$ tenemos $x\in(0,1)$. Ahora, calculamos $f_m'(x)$ de donde,
	$$f_m'(x)=3x^2-2=3\left(x_2-1\right),$$
	lo que implica que $f_m'(x)=0$, si y sólo si $x=\pm 1$. Pero esto contradice la conclusión de que $x\in (0,1)$. Por lo tanto, la función polinómica $f_m(x)=x^3-3x+m$ no tiene dos raíces en $[0,1]$.\\\\

    %-------------------- 40.
    \item Suponga que $f$ es continua y diferenciable en $[0, 1]$, que $f(x)$ está en $[0, 1]$ para cada $x$, y que $f(x) \neq 1$ para todo $x$ de $[0, 1]$. Demuestre que existe exactamente un número $x$ en $[0,1]$ tal que $f(x) = x$. (La mitad de este problema ya ha sido resuelta en el Problema 7-11.)\\\\
	Demostración.-\; Suponga que existe dos números, $x_1$ y $x_2$, en $[0,1]$ tal que $f(x_1)=x_1$ y $f(x_2)=x_2$. Luego, consideremos la función $g:[0,1]\to [0,1]$ definido por la fórmula
	$$g(x)=f(x)-x.$$
	Entonces, vemos que $g$ es también diferenciable en $[0,1]$, ya que, tanto la función identidad como $f$ lo son. Además,
	$$g(x_i)=f(x_i)-x_i=x_i-x_i=0$$
	para $i=1,2$. Así, por el teorema de Rolle, existe un número $x\in(x_1,x_2)$ tal que
	$$g'(x)=0.$$
	Pero, esto implica que
	$$f'(x)=1$$
	como,
	$$g'(x)=f'(x)-1$$
	el cual contradice la hipótesis. Por lo tanto, mostramos que existe exactamente un número $x\in[0,1]$ tal que $f(x)=x.$\\\\

    %-------------------- 41.
    \item 
	\begin{enumerate}[(a)]

	    %---------- (a)
	    \item Demuestre que la función $f(x)=x^2-\cos x$ satisface $f(x)=0$ para exactamente dos números $x$.\\\\
		Demostración.-\; La función $f$ tiene al menos dos ceros en $[-1,1]$, ya que $f(0)<0$, mientras que $f(\pm 1)>0.$ Si $f$ tuviese más de dos ceros, entonces $f'$ podría tener al menos dos ceros. Pero
		$$f'(x)=2x+\sen x$$
		es una función creciente, ya que
		$$f''(x)=2+\cos x\geq 1$$
		para todo $x$.\\\\

	    %---------- (b)
	    \item Demuestre lo mismo para la función $f(x)=x^2-x\sen x - \cos x$.\\\\
		Demostración.-\; La función $f$ tiene al menos dos ceros ya que $f(0)<0$ mientras $f(\pm)>0.$ Si $f$ tiene más de dos ceros, entonces $f'$ podría tener al menos dos ceros. Pero
		$$f'(x)=2x-x\cos x = x(2-\cos x)$$
		es $0$ sólo para $x=0$.\\\\

	    %---------- (c)
	    \item Demuéstrelo también para la función $f(x)=2x^2-x\sen x - \cos^2 x$. (Será útil hacer algunas estimaciones preliminares para restringir la posible localización de los ceros de $f$.)\\\\
		Demostración.-\; Tenemos $f(0)<0,$ mientras $f(x)$ será $>0$ para $|x|$ grande, ya que $|x\sen x|$ es pequeño en comparación con $2x^2$ y $|\cos^2 x|\leq 1.$ De hecho, escribiendo 
		$$f(x)=2x(2x-\sen x)-\cos^2 x,$$
		y notando que $2x-\sen x >1$ para $x>1$, vemos que $f(x)>0$ para $x>1$, y también para $x<-1$, pues $f$ es par. Así, $f$ tiene por lo menos dos zeros en $[-1,1]$, y sin ceros fuera de $[-1,1]$. Si $f$ tiene más de dos zeros, entonces $f'$ podría tener dos ceros en $[-1,1]$. Pero
		$$\begin{array}{rcl}
		    f'(x)&=&4x-\sen x-x\cos x + 2\cos x \sen x\\
			 &=& 4x-\sen x - x\cos x+\sen 2x,
		\end{array}$$
		lo que, es creciente en $[-1,1]$, ya que
		$$f''(x)=4-2\cos x+x\sen x+2\cos 2x$$
		el cual, es $\geq 1$ en $[-1,1]$, donde $x\sen x>0$ en $[-1,1]$, mientras $|\cos x|,$ $|\cos 2x|\leq 1.$\\\\

	\end{enumerate}

    %-------------------- 42.
    \item 
	\begin{enumerate}[(a)]

	    %---------- (a)
	    \item Demuestre que si $f$ es una función dos veces diferenciable con $f(0)=0$, $f(1)=1$ y $f'(0)=f'(1)=0$, entonces $|f''(x)|\geq 4$ para algún $x$ de $(0,1)$. En términos más pintorescos: una partícula que recorre una distancia unidad en la unidad de tiempo,y empieza y termina con velocidad $0$, tiene algún momento una aceleración $\geq 4$. Indicación: Demuestre que o bien $f''(x)\geq 4$ para algún $x$ de $\left(0,\frac{1}{2}\right)$, o bien $f''(x)\leq -4$ para algún $x$ de $\left(\frac{1}{2},1\right)$.\\\\
		Demostración.-\; Suponga que $f''(x)<4$ para todo $x\in [0,\frac{1}{2}]$. Entonces, por el teorema de valor medio, para todo $x\in [0,\frac{1}{2}]$ tenemos
		$$\dfrac{f'(x)-f'(0)}{x-0}=f''\left(x'\right)$$
		para algún $x'\in [0,x]$. Así $f'(x)<4x$. Si establecemos $h(x)=2x^2,$ entonces $f(0)=h(0)$ y $f08x)<h'(x)$, y se sigue por el problema 30 que $f(x)<h(x)$ en $[0,\frac{1}{2}]$, por lo que $f(\frac{1}{2})=\dfrac{1}{2}$.\\
		El mismo tipo de análisis puede ser aplicado para $f$ en $[\frac{1}{2},1]$, El cual mostraremos que si $f''(x)>-4$ en $[\frac{1}{2},1]$ entonces $f(\frac{1}{2})=\dfrac{1}{2}$, (o podemos considerar la función $f(x)=1-f(1-x).$). Es obvio que no podemos tener ambas posibilidades, por lo tanto $|f''(x)|\geq 4$ para $[0,1]$.\\\\

	    %---------- (b)
	    \item Demuestre que de hecho debe verificarse que $|f''(x)|>4$ para algún $x$ de $(0,1)$.\\\\
		Demostración.-\; Note primero que no podemos tener $f''(x)=4$ para $0<x<\frac{1}{2}$ y también $f''(x)=-4$ para $\frac{1}{2}<x\leq 1,$ ya que esto podría implicar que $f'(x)=4x$ para $0\leq x \leq \frac{1}{2}$ y $f'(x)=-4x$ para $\frac{1}{2}\leq x\leq 1$, en cuyo caso $f''(\frac{1}{2})$ podría no existir. Por otro lado, si tenemos $f''(x)\leq 4$ para todo $x$ en $(0,\frac{1}{2})$ pero $f''(x)<4$ para al menos un $x$, entonces tenemos $f'(x_0)<3x_0$ para al menos un $x_0$, y en consecuencia $f(x)<2x^2$ para todo $x\geq x_0$, así que $f(\frac{1}{2})=\dfrac{1}{2}$; si tuviéramos también $f''(x)\geq -4$ para todo $x$ en $(\frac{1}{2},1)$, entonces $f(\frac{1}{2})\geq \dfrac{1}{2}$, es una contrdicción.\\\\

	\end{enumerate}

    %-------------------- 43.
    \item Suponga que $f$ es una función tal que $f'(x)=1/x$ para todo $x>0$ y $f(1)=0$. Demuestre que $f(xy)=f(x)-f(y)$ para todo $x,y>0$. Indicación: Halle $g'(x)$ cuando $g(x)=f(xy)$.\\\\
	Demostración.-\; Consideremos la función $f$ definida como sigue:
	$$g(x)=f(xy),\; y>0.$$
	Entonces, usando la regla de Leibniz vemos que
	$$g'(x)=yf'(xy)=y\dfrac{1}{xy}=\dfrac{1}{x}=f'(x).$$
	Así, existe una constante $c$ tal que
	$$g(x)=f(x)+c$$
	para todo $x>0.$ Luego, evaluamos en $x=1$,
	$$f(y)=g(1)=f(1)+c=c$$ 
	Por lo tanto,
	$$f(xy)=g(x)=f(x)+f(y)$$
	para todo $x,y>0$.\\\\

    %-------------------- 44.
    \item Suponga que $f$ satisface
    $$f''(x)+f'(x)g(x)-f(x)=0$$
    para alguna función $g$. Demuestre que si $f$ es $0$ en dos puntos, entonces $f$ es $0$ en el intervalo entre ellos. Indicación: Utilice el teorema 6.\\\\
    Demostración.-\; Por el teorema 6 (spivak, capítulo 11), si $x\in [x_0,y_1]$ es un máximo local, entonces $f''(x)=f(x)\leq 0$. Luego, si $x\in [x_0,x_1]$ es un mínimo local, entonces $f''8x)=f(x)\geq 0$, ya que $f'(x)=0.$ Ahora, suponga que existe un $x\in [x_0,x_1]$ tal que $f(x)\neq 0$. Entonces, $f(x)>0$ o $f(x)<0$.

    \begin{enumerate}[\text{Caso} 1.]

	\item $[f(x)>0]$ Observemos que si $x$ no puede ser un punto máximo local. Esto implica que existe $x'\in [x_0,x_1)$ tal que $x'$ es un máximo local y $f(x')>0$ o existe $x''\in(x,x_1]$ tal que $x''$ es un máximo local, y $f(x'')>0$ ya que $f(x_0)=f(x_1)=0.$ Pero no es posible, puesto que $f''(x')=f(x')$ y $f''(x'')=f(x'')$ por la ecuación dada. Por lo tanto, $f(x)$ no puede ser estrictamente positivo.

	\item $[f(x)<0]$ Similar al anterior caso, vemos que $f$ no es estricamente negativa en $[x_0,x_1]$. Por cierto, vemos que $x$ no puede ser un punto máximo local. Esto implica que que existe $x'\in[x_0,x)$ tal que $x'$ es un mínimo local y $f(x')<0$ o existe $x''\in (x,x_1]$ tal que $x''$ es un mínimo local y $f(x'')<0$, puesto que $f(x_0)=f(x_1)=0.$ Pero no es posible, ya que $f''(x')=f(x')$ y $f''(x'')=f(x'')$, por la ecuación dada. Por lo tanto, $f(x)$ po es estrictamente positivo.
 
    \end{enumerate}

    Así, $f(x)=0$ para todo $x\in [x_0,x_1]$.\\\\

    %-------------------- 45.
    \item  Suponga que $f$ es continua en $[a,b]$, n-veces diferenciable en $(a,b)$ y que $f(x)=0$ para $n+1$ puntos $x$ diferentes de $[a,b]$. Demuestre que $f^{(n)} (x)=0$ para algún $x$ de $(a,b)$.\\\\
	Demostración.-\; Sea $x_1,x_2,\ldots,x_n,x_{n+1}$ son los distintos $n+1$ raíces de $f$ en $[a,b]$. Entonces, notemos que 
	$$f(x_1)=0,\quad x_i=1,2,\ldots,n+1.$$
	Ahora, consideremos que el conjunto de intervalos $\left\{[x,x_j]|1\leq i < j \leq n+1\right\}$. Luego, $f$ es diferenciable en todos los elementos del conjunto $\left\{[x_i,x_j] | 1\leq i < j \leq n+1\right\}$. Entonces, por el teorema del valor medio en cada intervalo $[x_i,x_j]$ para $1\leq i < j \leq n+1$, existe un elemento $\alpha_i\in (x,x_j)$ para $1,2,\ldots,n$ tal que
	$$\dfrac{f(x_j)-f(x_i)}{x_j-x_i}=f'(\alpha_i).$$
	Se sigue que para cada intervalo $[x_i,x_j]$ existe $\alpha_i$ tal que 
	$$f'(\alpha_i)=0,\mbox{ ya que } f(x_i)=0 \mbox{ para cada }i.$$
	Luego,
	$$\alpha_i\in [a,b],\quad i=1,2,\ldots,n.$$
	Este resulta que $f'$ tiene $n$ ceros distintos en $[a,b]$. De manera similar a lo anterior, supongamos que el conjunto de intervalo $\left\{[\alpha_i,\alpha_j]|1\leq i <j\leq n\right\}$. Entonces, por el teorema de valor medio en cada intervalo $[\alpha_i,\alpha_j]$ para $1\leq i <j\leq n$, existe un elemento $\beta_i\in(\alpha_i,\alpha_j)$ para $i=1,2,\ldots,n-1$ tal que 
	$$\dfrac{f'(\alpha_j)-f'(\alpha_i)}{\alpha_j-\alpha_i}=f''(\beta_i).$$
	Se sigue que para cada intervalo $[\alpha_i,\alpha_j]$ existe $\beta_i$ tal que
	$$f''(\beta_i)=0,\mbox{ ya que } f'(\alpha_i)=0 \mbox{ para cada }i.$$
	Luego,
	$$\beta_i\in [a,b],\quad i=1,2,\ldots,n-1.$$
	Esto resulta que $f''$ tiene $n-1$ ceros distintos en $[a,b]$. Procediendo por dicho argumento hasta la derivada $n-ésima$, de $f$, se sigue del principio de inducción que $f^{(n)}$ tiene al menos una raíz en $[a,b]$. Por el teorema del valor medio, se demuestra que
	$$f^{(n)}(x)=0\mbox{ para algún } x\in[a,b].$$\\

    %-------------------- 46.
    \item Sean $x_1,\ldots, x_{n+1}$ puntos arbitrarios del intervalo $[a,b]$, y sea
    $$Q(x)=\prod_{i=1}^{n+1}(x-x_i).$$
    Suponga que $f$ es una función $(n+1)$-veces diferenciable y que $P$ es una función polinómica de grado $\leq n$ tal que $P(x_i)=f(x_i)$ para $i=1,\ldots,n+1$ (vea el problema 3-6). Demuestre que para cada $x$ de $[a,b]$ existe un número $c$ de $(a,b)$ tal que
    $$f(x)-P(x)=Q(x)\cdot \dfrac{f^{(n+1)}(c)}{(n+1)!}.$$
    Indicación: Considere la función 
    $$F(t)=Q(x)[f(t)-P(t)]-Q(t)[f(x)-P(x)].$$
    Demuestre que $F$ se anula en $n+2$ puntos diferentes de $[a,b]$ y utilices el problema 45.\\\\
	Demostración.-\; Si $x$ esta en $x_i$, entonces $f(x)-P(x)=Q(x)$, así podemos escoger cualquier $c$. Por otra parte, sea
	$$F(t)=Q(x)\left[f(t)-P(t)\right]-Q(t)\left[f(x)-P(x)\right].$$
	Entonces, para $i=1,\ldots , n+1$ se tiene
	\begin{center}
	    $F(x_i)=0$, ya que $f(x_i)-P_i=0$ y $Q(x_i)=0,$
	\end{center}
	como también
	$$F(x)=0.$$
	Por el problema 45, tenemos $F^{(n+1)}(c)=0$ para algún $c$ en $(a,b)$. Esto es,
	$$0=F^{(n+1)}(c)=Q(x)\left[f^{(n+1)}(c)-0\right]-(n+1)!\left[f(x)-P(x)\right].$$\\

    %-------------------- 47.
    \item Demuestre que 
    $$\dfrac{1}{9}<\sqrt{66}-8<\dfrac{1}{8}$$
    sin calcular con $\sqrt{66}$ con $2$ cifras decimales.\\\\
	Demostración.-\; Apliquemos el teorema del valor medio a $f(x)=\sqrt{x}$ en $[64,66]$:
	\begin{center}
	    $\dfrac{\sqrt{66}-\sqrt{64}}{66-64}=f'(x)=\dfrac{1}{2\sqrt{x}} = f'(x) = \dfrac{1}{2\sqrt{x}}$ para algún $x$ en $[64,66]$.
	\end{center}
	Puesto que $64<x<81$, tenemos $0<\sqrt{x}<9,$ por tanto
	$$\dfrac{1}{2\cdot 9}<\dfrac{\sqrt{66}-8}{2}<\dfrac{1}{2\cdot 8}.$$\\

    %-------------------- 48.
\item Demuestre la siguiente generalización del Teorema del Valor Medio: si $f$ es continua y diferenciable en $(a,b)$ y $\lim\limits_{y\to a^+}f(y)$ y $\lim\limits_{y\to b^-} f(y)$ existen, entonces existe algún $x$ de $(a,b)$ tal que
    $$f'(x)=\dfrac{\lim\limits_{y\to b^-} f(y)-\lim\limits_{y\to a^+}f(y)}{b-a}.$$
    (La demostración debe empezar: esto es una consecuencia trivial del teorema del valor medio porque...).\\\\
	Demostración.-\; Definamos la función $g:[a,b]\to \mathbb{R}$ como sigue:
	$$
	g(y) = 
	    \left\{
		\begin{array}{rcl}
		    \lim\limits_{y\to a^+} f(y)&\mbox{si}& y=a;\\\\
		    f(y)&\mbox{si}& a<y<b;\\\\
		    \lim\limits_{y\to b^-} f(y)&\mbox{si}& y=b.
		\end{array}
	    \right.
	$$
	Entonces, observamos que $g$ es continua en $[a,b]$. En efecto, a partir de la definición de $g$ tenemos que, para cualquier $\left\{y_n\right\}$ con $y_n\to a$, $g(y_n)\to g(a)$, y para cualquier secuencia $\left\{z_n\right\}$ con $z_n\to b,$ $g(z_n)\to g(b)$. Por otro lado, si $y\in (a,b)$ entonces $g(y)=f(y)$ así $g$ es continua en $[a,b]$. Además, como $g(y)=f(y)$ para todo $y\in (a,b)$, $g$ es diferenciable en $(a,b)$. Por lo tanto, por el teorema del valor medio existe un número $x\in (a,b)$ tal que
	$$g'(x)=\dfrac{g(b)-g(a)}{b-a}.$$
	Pero, como $g(y)=f(y)$ para todo $y\in (a,b)$ tenemos por definición de $g(a)$ y $g(b)$ tenemos que
	$$f'(x)=\dfrac{\lim\limits_{y\to b^-}f(y)-\lim\limits_{y\to a^+}f(y)}{b-a}.$$\\

    %-------------------- 49.
    \item Demuestre que la conclusión del teorema del Valor Medio de Cauchy puede escribirse en la forma
    $$\dfrac{f(b)-f(a)}{g(b)-g(a)}=\dfrac{f'(x)}{g'(x)},$$
    suponiendo, además, que $g(b)\neq g(a)$ y que $f'(x)$ y $g'(x)$ no se anulan simultáneamente en ningún punto de $(a,b)$.\\\\
	Demostración.-\; Tenemos por el Teorema Medio de Cauchy que si $f$ y $g$ son funciones continuas en $[a,b]$ y diferenciables en $(a,b)$, entonces existe un número $x$ tal que
	$$\left[f(b)-f(a)\right]g'(x)=\left[g(b)-g(a)\right]f'(x).$$
	Ahora, si $g(b)\neq g(a)$ y $f'(x)$  y $g'(x)$ no son simultáneamente cero en $(a,b)$, se sigue que ambos 
	$$\dfrac{1}{g(b)-g(a)}\; \quad \dfrac{f'(x)}{g'(x)}$$
	están definidos. Así, por el teorema del valor medio tenemos podemos escribir,
	$$\dfrac{f(b)-f(a)}{g(b)-g(a)}=\dfrac{f'(x)}{g'(x)}.$$
	Lo que completa la demostración.\\\\

    %-------------------- 50.
    \item Demuestre que si $f$ y $g$ son continuas en $[a,b]$ y diferenciables en $(a,b)$, y $g'(x)\neq 0$ para todo $x$ de $(a,b)$, entonces existe algún $x$ en $(a,b)$ con
    $$\dfrac{f'(x)}{g'(x)}=\dfrac{f(x)-f(a)}{g(b)-g(x)}.$$
    Indicación: Multiplique en cruz para ver lo que esto realmente significa.\\\\
	Respuesta.-\; Sea $h:[a,b]\to \mathbb{R}$ como sigue:
	$$h(x)=f(x)g(b)+f(x)g(a)-f(x)g(x).$$
	Entonces, por hipótesis, vemos que $g$ es continua en $[a,b]$ y diferenciable en $(a,b)$. Además
	$$h(a)=f(a)g(b)+f(a)g(a)-f(a)g(a)=f(a)g(b),$$
	y
	$$f(b)=f(b)g(b)+f(b)g(a)-f(b)g(b)=f(a)g(b).$$
	Así, $h(a)=h(b)=f(a)g(b)$. Luego, usando el teorema de Rolle para la función $h$, existe un número $x\in (a,b)$ tal que
	$$h'(x)=0.$$
	Calculemos $h'$:
	$$h'(x)=f'(x)g(b)+f(x)g'(x)-f'(x)g(x)-f(x)g'(x)=0$$
	que implica,
	$$f'(x)[g(b)-g(x)]=g'(x)[f(x)-f(a)].$$
	Después, ya que $g'(x)\neq 0$ para todo $x\in (a,b)$, se tiene que $g(b)\neq g(x)$ para $x\in (a,b)$. Por otro lado, si tenemos $x\in (a,b)$ tal que $g(x)=g(b)$ aplicando una vez más el teorema de Rolle, podría implicar que $g'(x)=0$ para algún $x$ in $(x,b)$. Lo que contradice el hecho de que $g'(x)\neq 0$. Por lo que podemos escribir la ecuación de arriba como sigue,
	$$\dfrac{f'(x)}{g'(x)}=\dfrac{f(x)-f(a)}{g(b)-g(x)}.$$\\


    %-------------------- 51.
    \item ¿Dónde se encuentra el error en la siguiente aplicación de la Regla de L'Hopital?:
    $$\lim_{x\to 1}\dfrac{x^3+x-2}{x^2-3x+2}=\lim_{x\to 1}\dfrac{3x^2+1}{2x-3}=\lim_{x\to 1}\dfrac{6x}{2}=3.$$
    (El límite es, en realidad, $-4$).\\\\
	Respuesta.-\; La regla de L'Hopital no es aplicable a la expresión
	$$\lim_{x\to 1}\dfrac{3x^2+1}{2x-3}=\lim_{x\to 1}\dfrac{6x}{2}$$
	puesto que $\lim\limits_{x\to 1}3x^2+1=3+1=4\neq 0$.\\\\

    %-------------------- 52.
    \item 
	\begin{enumerate}[(i)]

	    %---------- (i)
	    \item $\lim\limits_{x\to 0}\dfrac{x}{\tan x}$.\\\\
		Respuesta.-\; Dado que,
		$$\lim_{x\to 0}\dfrac{\sen x}{x}=1.$$
		Entonces,
		$$
		\begin{array}{rcl}
		    \lim\limits_{x\to 0} \dfrac{x}{\tan x} &=& \lim\limits_{x\to 0} \dfrac{x\cos x}{\sen x}\\\\
							   &=& \lim\limits_{x\to 0} \dfrac{x}{\sen x}\cdot \lim\limits_{x\to 0}\cos x\\\\
							   &=& 1.
		\end{array}
		$$
		Así, 
		$$\lim_{x\to 0}\dfrac{x}{\tan x}=1.$$\\

	    %---------- (ii)
	    \item $\lim\limits_{x\to 0}\dfrac{\cos^2 x - 1}{x^2}$.\\\\
		Respuesta.-\; 
		$$
		\begin{array}{rcl}
		    \lim\limits_{x\to 0}\dfrac{\cos^2 x - 1}{x^2}&=& \lim\limits_{x\to 0} \dfrac{-\sen^2 x}{x^2}\\\\ 
								 &=& -\lim\limits_{x\to 0}\left(\dfrac{\sen x}{x}\right)^2\\\\
								 &=& - (2)^2\\\\
								 &=& -1.
		\end{array}
		$$
		\vspace{.5cm}
		
	\end{enumerate}

    %-------------------- 53.
    \item Halle $f'(0)$ si
    $$
    f(x)=
    \left\{
	\begin{array}{rc}
	    \dfrac{g(x)}{x}, & x\neq 0\\\\
	    0, & x=0.
	\end{array}
    \right.
    $$
    y $g(0)=g'(0)=0$ y $g''(0)=17$.\\\\
	Respuesta.-\; Ya que $g(0)=0$ y $g$ es diferenciable, vemos que
	$$\lim_{x\to 0}g(x)=0\quad \mbox{y}\quad \lim_{x\to 0}h(x)=0,$$
	donde $h(x)=x.$ Después, $h'(x)=1$ y $\lim_{x\to 0}\dfrac{g'(x)}{h'(x)}$ existe. Luego, usando la regla de L'Hopital, 
	$$\lim_{x\to 0}f(x)=\lim_{x\to 0}\dfrac{g(x)}{x}=\lim_{x\to 0}g'(0)=0.$$
	Por lo que, $f$ es continua en $0$. Ahora, calculemos el límite
	$$\lim_{x\to 0}\dfrac{f(x)-f(0)}{x}=\lim_{x\to 0}\dfrac{f(x)}{x}\lim_{x\to 0}\dfrac{g(x)}{x^2}.$$
	Sea $u(x)=x^2$, entonces
	$$\lim_{x\to 0}g(x)=0 \quad \mbox{y}\quad \lim_{x\to 0}u(x)=0.$$
	De donde, $\lim\limits_{x\to 0}\frac{g'(x)}{u'(x)}$ existe ya que $u'(x)=2x$ y $\lim\limits_{x\to 0}\dfrac{g(x)}{x}=0$. Una vez más usando la regla de L'Hopital, se tiene
	$$\lim_{x\to 0}\dfrac{g(x)}{x^2}=\lim_{x\to 0}\dfrac{g'(x)}{2x}=\lim_{x\to 0}\dfrac{g''(x)}{2}.$$
	La última igualad se obtiene por la existencia de los $\lim\limits_{x\to 0}g'(x)=0$ y $\lim\limits_{x\to 0}2x=0$, así como $\lim\limits_{x\to 0}\frac{g''(x)}{2}$. Por último, sabemos que $g''(0)=17$, por lo tanto
	$$\lim_{x\to 0}\dfrac{g''(x)}{2}=\dfrac{17}{2}.$$
	Así,
	$$f'(0)=\dfrac{17}{2}.$$\\

    %-------------------- 54.
    \item Demuestre las siguientes formas de la Regla de L'Hopital (ninguna de ellas require un razonamiento esencialmente nuevo.)\\
	\begin{enumerate}[(a)]

	    %---------- (a)
	    \item Si $\lim\limits_{x\to a^+}f(x)=\lim\limits_{x\to a^+}=0$ y $\lim\limits_{x\to a^+}\dfrac{f'(x)}{g'(x)}=l$, entonces $\lim\limits_{x\to a^+}\dfrac{f(x)}{g(x)}=l$ (y análogamente para límites por la izquierda).\\\\
		Demostración.-\; Por la definición de límite notamos que
		$$\lim_{x\to a^+}f(x)=0\lim_{x\to a^+}g(x)$$
		implica que:
		\begin{enumerate}[i.]
		    \item existe un $\delta>0$ tal que $f'$ y $g'$ existen en el intervalo $(a,a+\delta)$,
		    \item además, $g'(x)\neq 0$ en $(a,a+\delta)$.
		\end{enumerate}
		Supongamos, sin perdida de generalidad, que tanto $f$ como $g$ son continuas por la derecha en $0$. De hecho, si $f$ y $g$ no fueran continuas en $0$, entonces habríamos definido $f(a)=g(a)=0$ cambiando los valores de $f(a)$ y $g(a)$ si fueran necesarios. Ahora aplicando los teoremas de valor medio y del valor medio de Cauchy para $f$ y $g$ en el intervalo $[a,x]$ para $x\in (a,a+\delta)$. Después notemos que $g(x)\neq 0$. Además, si $g(x)=0$, entonces habría un número $x_1\in(a,x)$ tal que $g'(x_1)=0$ contradiciendo la propiedad ii. anterior. Luego aplicando el valor medio de Cauchy a $f$ y $g$ observamos que existe un número $c_x\in (a,x)$ tal que
		$$\left[f(x)-f(a)\right]g'(c_x)=\left[g(x)-g(a)\right]f'(c_x),$$
		que equivale a la ecuación
		$$\dfrac{f(x)}{g(x)}=\dfrac{f'(c_x)}{g'(c_x)}.$$
		Por último, vemos que a medida que $x$ tiende a $a$ también $c_x$ tiende a $a$ ya que $c_x\in (a,x)$. De lo que se deduce que 
		$$\lim_{x\to a^+}\dfrac{f'(c_x)}{g'(c_x)}=\lim_{c_x\to a^+}\dfrac{f'(c_x)}{g'(c_x)}.$$
		Así, tenemos
		$$\lim_{x\to a^+}\dfrac{f(x)}{g(x)}=\lim_{x\to a^+}\dfrac{f'(c_x)}{g'(c_x)}=\lim_{c_x\to a^+}\dfrac{f'(c_x)}{g'(c_x)}=\lim_{y\to a^+}\dfrac{f'(y)}{g'(y)}.$$\\

	    %---------- (b)
	    \item Si $\lim\limits_{x\to a}f(x)=\lim\limits_{x\to a}g(x)=0$ y $\lim\limits_{x\to a}\dfrac{f'(x)}{g'(x)}=\infty$, entonces $\lim\limits_{x\to a}\dfrac{f(x)}{g(x)}=\infty$ (y análogamente para $-\infty$), o si se sustituye $x\to a$ por $x\to a^+$ o $x\to a^-$.\\\\
		Demostración.-\; Demostremos que $\lim\limits_{x\to a^+}\dfrac{f(x)}{g(x)}=\infty$. Por hipótesis se tiene,
		$$\lim_{x\to a}f(x)=0=\lim_{x\to a}g(x)$$
		implica que:
		\begin{enumerate}[i.]
		    \item Existe un $\delta>0$ tal que $f'$ y $g'$ existen en el intervalo $(a-\delta,a+\delta) | \left\{a\right\}$,
		    \item además, en el mismo intervalo, es decir $(a-\delta,a+\delta)|\left\{a\right\}$, $g'(x)\neq 0$.
		\end{enumerate}
		Supongamos sin pérdida de generalidad que tanto $f$ como $g$ son continuas a la derecha en $0$. De hecho si $f$ y $g$ no fueran continuas en $0$, entonces habríamos definido $f(a)=g(a)=0$ cambiando los valores de $f(a)$ y $g(a)$ si fuera necesario.\\
		Ahora, aplicando el teorema de valor medio y el teorema del valor medio de Cauchy para $f$ y $g$ en el intervalo $[a,x]$ en $x\in (a,a+\delta)$. Notamos que $g(x)\neq 0$. Es más, si $g(x)=0$, entonces podría existir un número $x_1\in(a,x)$ tal que $g'(x_1)=0$ contradiciendo la propiedad ii). Ahora aplicando el teorema de valor medio de Cauchy para $f$ y $g$ observamos que existe un número $c_x\in(a,x)$ tal que
		$$\left[f(x)-f(a)\right]g'(c_x)=\left[g(x)-g(a)\right]f'(c_x)$$
		lo que equivale a la ecuación
		$$\dfrac{f(x)}{g(x)}=\dfrac{f'(c_x)}{g'(c_x)}.$$
		Luego, vemos que como $x$ tiende a $a$, $c_x$ también tiende a $a$ ya que $c_x\in(a,x)$. Entonces, se sigue que
		$$\lim_{x\to a}\dfrac{f'(c_x)}{g'(c_x)}=\lim_{c_x\to a}\dfrac{f'(c_x)}{g'(c_x)}.$$
		Así, tenemos
		$$\lim_{x\to a}\dfrac{f(x)}{g(x)}=\lim_{x\to a}\dfrac{f'(c_x)}{g'(c_x)}=\lim_{c_x\to a}\dfrac{f'(c_x)}{g'(c_x)}=\lim_{y\to a}\dfrac{f'(y)}{g'(y)}=\infty.$$
		Similarmente, tenemos para $x\in (a-\delta,a)$, que
		$$\lim_{x\to a}\dfrac{f(x)}{g(x)}=\lim_{y\to a}\dfrac{f'(y)}{g'(y)}=\infty.$$
		Esto completa la demostración de la primera parte. Ahora siguiendo el argumento de la parte (a), en este caso con $l=\infty$  uno puede obtener los otros casos.\\\\

	    %---------- (c)
	    \item Si $\lim\limits_{x\to \infty} f(x)=\lim\limits_{x\to \infty}g(x)=0$ y $\lim\limits_{x\to \infty}\dfrac{f'(x)}{g'(x)}=l$, entonces $\lim\limits_{x\to \infty}\dfrac{f(x)}{g(x)}=l$ (y análogamente para $-\infty$), Indicación: Considere $\lim\limits_{x\to 0^+} f\left(\dfrac{1}{x}\right)g\left(\dfrac{1}{x}\right)$.\\\\
		Demostración.-\; Consideremos las funciones $\hat{f}(x)$ y $\hat{g}(x)$ definidas de la siguiente manera:
		$$\hat{f}(x)=f\left(\dfrac{1}{x}\right)\mbox{ y } \hat{g}\left(\dfrac{1}{x}\right).$$
		Notemos que como $x\to 0^+$, entonces $\dfrac{1}{x}\to \infty$ . Así, tenemos que
		$$\lim_{x\to 0^+}\hat{f}(x)=0=\lim_{x\to 0^+}\hat{g}(x)$$
		y $\lim\limits_{x\to 0^+}\dfrac{\hat{f}(x)}{\hat{g}(x)}$ existe y es igual a $l$. Por lo tanto, aplicando la parte (a) para $\hat{f}$ y $\hat{g}$, se tiene
		$$\lim_{x\to^+}\dfrac{\hat{f}(x)}{\hat{g}(x)}=l.$$
		Lo que implica que
		$$\lim_{x\to \infty}\dfrac{f(x)}{g(x)}=\lim_{x\to 0^+}\dfrac{\hat{f}(x)}{\hat{g}(x)}=l.$$\\
		El caso para $-\infty$ es análogo.\\\\

	    %---------- (d)
	    \item Si $\lim\limits_{x\to \infty} f(x)=\lim\limits_{x\to \infty}=0$ y $\lim\limits_{x\to \infty}f'(x)/g'(x)=l$, entonces $\lim\limits_{x\to \infty}f(x)/g(x)=\infty.$\\\\
		Demostración.-\; Consideremos las funciones $\hat{f}$ y $hat{g}$ definidas por:
		$$\hat{f}(x)=f\left(\dfrac{1}{x}\right)\quad \mbox{y}\quad \hat{g}(x)=g\left(\dfrac{1}{x}\right).$$
		Entonces, notemos que $x\to 0^+$, por lo que $\dfrac{1}{x}\to \infty.$ Así, se tiene
		$$\lim_{x\to 0^+}\hat{f}(x)=0=\lim_{x\to 0^+}\hat{g}(x)$$
		y  $\lim\limits_{x\to 0^+}\dfrac{\hat{f}'(x)}{\hat{g}'(x)}=\infty$. Por lo tanto, aplicando (b) a $\hat{f}$ y $\hat{g}$ se tiene
		$$\lim_{x\to 0^+}\dfrac{\hat{f}(x)}{\hat{g}(x)}=\infty.$$
		Lo que implica,
		$$\lim_{x\to \infty}\dfrac{f(x)}{g(x)}=\lim_{x\to 0^+}\dfrac{\hat{g}(x)}{\hat{f}(x)}=\infty.$$\\

	\end{enumerate}

    %-------------------- 55
    \item Existe otra forma de la Regla de L'Hopital que exige más manipulaciones algebraicas: Si $\lim\limits_{x\to \infty}f(x)=\lim\limits_{x\to \infty}g(x)=\infty$, y $\lim\limits_{x\to \infty}f'(x)(g'(x)=l,$ entonces $\lim\limits_{x\to \infty}f(x)/g(x)=l$. Demuéstrelo de la manera siguiente:

	\begin{enumerate}[(a)]

	    %---------- (a)
	    \item Para todo $\epsilon>0$ existe un número $a$ tal que
	    $$\bigg|\dfrac{f'(x)}{g'(x)}-l\bigg|<\epsilon \quad \mbox{para }x>a.$$
	    Aplique el Teorema del Valor Medio de Cauchy a $f$ y $g$ en $[a,x]$ para demostrar que
	    $$\bigg|\dfrac{f'(x)}{g'(x)}-l\bigg|z\epsilon \quad \mbox{para }x>a.$$\\
		Demostración.-\; Sea $\epsilon$ tal que $0<\epsilon<\frac{1}{2}$. Ya que $\lim\limits_{x\to \infty}\dfrac{f'(x)}{g'(x)}=l,$ existe un número positivo $a$ de modo que
		$$\Bigg|\dfrac{f(x)}{g(x)}-l\Bigg|<\epsilon \quad \mbox{para }x>a.$$
		Ahora, sea $c\in (a,\infty)$. Entonces, tenemos $a<c<\infty$. Luego, notemos que $g'(x)\neq 0$ para todo $x>a$. Por el teorema de valor medio se sigue que $g(x)\neq g(a)$ para todo $x>a$. \\
		Consideremos la función $h:[a,c]\to \mathbb{R}$ definida de la siguiente manera:
		$$h(x)=\dfrac{1-\dfrac{f(a)}{f(x)}}{1-\dfrac{g(a)}{g(x}},\quad \mbox{para }x\in[a,c].$$
		Note que,
		$$
		\begin{array}{rcl}
		    \lim\limits_{x\to \infty} h(x) &=& \lim\limits_{x\to \infty} \dfrac{1-\dfrac{f(a)}{f(x)}}{1-\dfrac{g(a)}{g(x)}}\\\\
						   &=& 1.
		\end{array}
		$$
		Esto ya que $\lim\limits_{x\to\infty}f(x)=\lim\limits_{x\to \infty}g(x)=\infty.$ Por lo tanto, existe $d\in [a,c]$ tal que
		$$1-\epsilon < h(x)<1-\epsilon,\quad \forall x \in [a,c].$$
		Se sigue que
		$$\dfrac{1}{2}<h(x)<\dfrac{3}{2},\quad \forall x\in [ac].$$
		Después, vemos que
		$$\dfrac{f(x)}{g(x)}=\dfrac{f(x)-f(a)}{g(x)-g(a)}\cdot \dfrac{1}{h(x)},\quad \forall x\in [a,c].$$
		Aplicando el teorema de valor medio de Cauchy a $f$ y $g$ en $[a,c]$ se tiene
		$$\dfrac{f(x)-f(a)}{g(x)-g(a)}=\dfrac{f'(X_x)}{g'(X_x)},\quad \mbox{para algún }X_x\in(a,c)$$
		Sabemos que $X_x$ depende de $x$. Así,
		$$X_x\to \infty \mbox{ cuando }x\to \infty.$$
		Dado que,
		$$\lim_{x\to \infty}\dfrac{f'(x)}{g'(x)}=l,$$
		entonces
		$$\lim_{x\to \infty}\dfrac{f(x)-f(a)}{g(x)-g(a)}=\lim_{x\to \infty}\dfrac{f'(X_x)}{g'(X_x)}=l.$$
		Por último, como $c$ es arbitraria mayor a $a$, entonces por definición de limite tenemos que
		$$\Bigg|\dfrac{f(x)-f(a)}{g(x)-g(a)}-l\Bigg|<\epsilon,\quad\mbox{para }x>a.$$\\

	    %---------- (b)
	    \item Escriba ahora
	    $$\dfrac{f(x)}{g(x)}=\dfrac{f(x)-f(a)}{g(x)-g(a)}\cdot \dfrac{f(x)}{f(x)-f(a)}\cdot \dfrac{g(x)-g(a)}{g(x)}$$
	    (¿por qué se puede suponer que $f(x)-f(a)\neq 0$? para $x$ grandes?) y deduzca que
	    $$\bigg|\dfrac{f(x)}{g(x)}-l\bigg|<2\epsilon\quad \mbox{para }x \mbox{ suficientemente grandes}.$$\\
		Demostración.-\; Notemos que
		$$\lim_{x\to \infty}f(x)=\infty.$$
		Entonces, existe un número positivo $M$ tal que
		$$f(x)>M,\quad \forall x.$$
		Recordemos que
		$$\dfrac{f(x)}{g(x)}=\dfrac{f'(X_x)}{g'(X_x)}\cdot \dfrac{1}{h(x)},\quad \forall x\in [a,c].$$
		Dado que,
		$$
		\begin{array}{rcl}
		    \lim\limits_{x\to \infty}\dfrac{f(x)}{g(x)} &=& \lim\limits_{x\to \infty} \dfrac{f'(X_x)}{g'(X_x)}\cdot \dfrac{1}{h(x)}\\\\
								&=&\lim\limits_{x\to \infty} \dfrac{f'(X_x)}{g'(X_x)}\lim\limits_{x\to \infty} \dfrac{1}{h(x)}\\\\
								&=& l,
		\end{array}
		$$
		ya que, $\lim\limits_{x\to \infty}h(x)=1,\; \lim\limits_{x\to \infty}\dfrac{f'(X_x)}{g'(X_x)}=l.$
		Entonces, por la definición de límite, para cada $\epsilon>0$ tenemos
		$$\Bigg|\dfrac{f(x)}{g(x)}-l\Bigg|<2\epsilon,\quad \mbox{para cada }x \mbox{ grande}.$$\\

	\end{enumerate}

    %-------------------- 56.
    \item Para completar la orgía de variantes de la regla de L’Hópital, aplicar el Problema 55 para demostrar unos cuantos casos más de la siguiente proposición general (existen tantas posibilidades que el lector debe seleccionar aquellas, si las hay, que sean de su interés):
    Si $\lim\limits_{x\to []}f(x)=\lim\limits_{x\to []}$ y $\lim\limits_{x\to []} \dfrac{f'(x)}{g'(x)}=(),$ entonces $\lim\limits_{x\to []} \dfrac{f(x)}{g(x)}=().$ Aquí $\left[\right]$ puede ser $a$ ó $a^+$ ó $a^-$ ó $-\infty$, $\left\{\right\}$ puede ser $0$ ó $\infty$ ó $-\infty$ y $()$ puede ser $l$ ó $\infty$ ó $-\infty$.\\\\
	Respuesta.-\; Si $\lim\limits_{x\to [\infty]}f(x)=\lim\limits_{x\to [\infty]}$ y $\lim\limits_{x\to [\infty]} \dfrac{f'(x)}{g'(x)}=(a),$ entonces $\lim\limits_{x\to [\infty]} \dfrac{f(x)}{g(x)}=(a).$\\\\

    %-------------------- 57.
    \item Si $f$ y $g$ son diferenciables y existe el $\lim\limits_{x\to a}\dfrac{f(x)}{g(x)}$, ¿puede concluirse que también existe el $\lim\limits_{x\to a}\dfrac{f'(x)}{g'(x)}$? (se trata del recíproco de la Regla de L'Hopital).\\\\
	Respuesta.-\; No, por ejemplo, sea $a=0$, $g(x)=x$, de donde
	$$
	f(x)=
	\left\{
	    \begin{array}{rr}
		x^2\sen \dfrac{1}{x}, & x\neq 0\\\\
		0, & x=0.
	    \end{array}
	\right.
	$$

    %-------------------- 58.
    \item Demuestre que si $f'$ es creciente, entonces toda tangente de $f$ corta a la gráfica de $f$ una sola vez. (En particular, esto se cumple en el caso de la función $f(x)=x^n$ si $n$ es par.)\\\\
	Demostración.-\; La linea tangente  a través de $\left(a,f(a)\right)$ es la gráfica de
	$$
	\begin{array}{rcl}
	    g(x) &=& f'(a)(x-a)+f(a)\\\\
		 &=& f'(a)x+f(a)+af'(a).
	\end{array}
	$$
	Si $g(x_o)=f(x_o)$ para algún $x_0\neq a$, entonces
	$$0=g'(x)-f'(x)=f'(a)-f'(x)$$
	para algún $x$ en $(a,x_o)$ o $(x_o,a).$ Lo que es imposible, ya que $f'$ es creciente.\\\\

    %-------------------- 59.
    \item Resuelva otra vez el problema 10-18 (c) cuando
    $$(f')^2=f-\dfrac{1}{f^2}.$$
    (¿Por qué está este problema en este capítulo?).\\\\
	Respuesta.-\; Diferenciando la expresión se tiene
	$$
	\begin{array}{rcl}
	    (f')^2 &=& f-\dfrac{1}{f^2}\\\\
	    2f'\cdot f'' &=& f'-\dfrac{2}{f^3}\cdot f'\\\\
	    f'' &=& \dfrac{f'-\dfrac{2}{f^3}\cdot f'}{2f'}\\\\
	    f''&=&\dfrac{1}{2}-\dfrac{1}{f^3}.
	\end{array}
	$$
	para todo $x$ donde $f'(x)\neq 0$. Luego, ya que $(f')^2=\dfrac{f^3-1}{f^2}$, tenemos $f'(x)=0$ solo para $f(x)=1$. Pero el teorema 7 (aplicando $f'$) implica que las formula vale en este caso, con $f''(x)=\dfrac{1}{2}-1=-\dfrac{1}{2}$.\\\\


    %-------------------- 60.
    \item 
	\begin{enumerate}[(a)]

	    %--------- (a)
	    \item Suponga que $f$ es diferenciable en $[a,b]$. Demuestre que si el mínimo de $f$ en $[a,b]$ se encuentra en el punto $a$, entonces $f'(a)\geq 0$, y si se encuentra en el punto $b$, entonces $f'(b)\leq 0.$ (Deberá examinarse la mitad de la demostración del Teorema 1.)\\\\
		Demostración.-\; Ya que $f$ tiene un mínimo en $a$, tenemos para todo $h>0$, que
		$$f(a+h)-f(a)\geq 0$$
		lo que equivale a 
		$$\dfrac{f(a+h)-f(a)}{h}\geq 0.$$
		De donde, tomamos el límite cuando $h\to 0^-$ y obtenemos
		$$f'(a)=\lim_{x\to 0^{-}}\dfrac{f(a+h)-f(a)}{h}\geq 0.$$
		Ya que $f$ tiene un mínimo en $b$, tenemos para todo $h>0$, que
		$$f(b-h)-f(b)\leq 0$$
		lo que equivale a 
		$$\dfrac{f(b-h)-f(b)}{h}\leq 0.$$
		Por lo que tomando el límite cuando $h\to 0^+$ obtenemos
		$$f'(b)=\lim_{x\to 0^+}\dfrac{f(b-h)-f(b)}{h}\leq 0.$$\\

	    %--------- (b)
	    \item Suponga que $f'(a)<0$ y que $f'(b)>0.$ Demuestre que $f'(x)=0$ para algún $x$ de $(a,b)$. Indicación: Considere el mínimo de $f$ en $[a,b]$; ¿por qué debe estar en algún punto de $(a,b)$?.\\\\
		Demostración.-\; Ya que $f$ es diferenciable en $[a,b]$, entonces $f$ es diferenciable ahí. Por lo tanto, $f$ tiene que estar acotado, lo que implica que hay un número $x\in [a,b]$ tal que $f(x)$ es el mínimo en $[a,b]$. Pero por la parte (a) vemos que $f$ no puede tener un mínimo en $a$ o $b$, de lo contrario contradice la hipótesis. Entonces, el mínimo de $f$ debe estar en $(a,b)$; es decir, existe un número $x\in(a,b)$ tal que $f(x)$ es el mínimo. Por lo que tenemos 
		$$f'(x)=0.$$\\

	    %--------- (c)
	    \item Demuestre que si $f'(a)<c<f'(b)$, entonces $f'(x)=c$ para algún $x$ de $(a,b)$. (Este resultado se conoce como el Teorema de Darboux. Advierta que no estamos suponiendo que $f'$ sea continua). Indicación: Construya una función adecuada a la cual se pueda aplicar la parte (b).\\\\ 
		Demostración.-\; Sea la función 
		$$g(t)=f(t)-ct,$$
		Entonces, ya que $f$ es diferenciable en $[a,b]$, 
		$$g'(t)=f'(t)-c.$$
		Por lo tanto, se tiene
		$$g'(a)=f(a)-c<0\quad\mbox{y}\quad g'(b)f'(b)-c>0.$$
		Ahora, aplicando la parte (b) para la función $g$, vemos que existe un número $x\in(a,b)$ tal que 
		$$g'(x)=0\quad \Rightarrow \quad f'(x)-c=0.$$
		Así, tenemos que $f'(x)=c$ para algún $x\in(a,b)$.\\\\

	\end{enumerate}

    %-------------------- 61.
    \item Suponga que $f$ es diferenciable en algún intervalo que contiene $a$ pero que $f'$ es discontinua en $a$. Demuestre lo siguiente:

	\begin{enumerate}[(a)]

	    %--------- (a)|
	    \item Ambos límites laterales $\lim\limits_{x\to a^+}f'(x)$ y $\lim\limits_{x\to a^-} f'(x)$ no pueden existir. (Se trata de una pequeña variación del Teorema 7.)\\\\
		Demostración.-\; Sea $(x_1,x_2)$ alrededor de $a$ donde $f$ es diferenciable. Demostraremos que $\lim\limits_{x\to a^+}$ y $\lim\limits_{x\to a^-}$ no existen. Sean
		$$\lim_{x\to a^-}f'(x)=l_1\quad \mbox{y}\quad \lim_{x\to a^+}f'(x)=l_2.$$
		Primero probaremos que $f'(a)=l_1$. Si es posible supongamos $l_1\neq f'(a)$

		\begin{enumerate}[\textbf{Caso} 1.]
		    \item Tomemos $l_1<f'(a)$. Sea $\epsilon>0$ tal que
		    $$l_1-\epsilon<f'(a).$$
		    Ya que, $\lim\limits_{x\to a^-}f'(x)=l_1$ existe un $\delta>0$ tal que
		    $$l_1-\epsilon<f'(x)<l_1+\epsilon,\quad \forall x\in (a-\delta,a)\cap(x_1,x_2)$$
		    Luego, sea $\alpha$ un elemento de $x\in (a-\delta,a)\cap(x_1,x_2)$. Entonces, tenemos
		    $$l_1-\epsilon<f'(\alpha)<l_1+\epsilon<f'(a).$$
		    Por el teorema de Darboux (Problema 60). Existe un punto $\mathcal{X}\in (\alpha,a)$ de modo que
		    $$f'\left(\mathcal{X}\right)=l_1+\epsilon.$$
		    Notemos que
		    $$\mathcal{X}\in \left(\alpha,a\right)\quad\Rightarrow \quad \mathcal{X}\in (a-\delta,a)\cap (x_1,x_2).$$
		    Se sigue que
		    $$f'\left(\mathcal{X}\right)<l_1+\epsilon.$$
		    Lo que es imposible. Por lo tanto, $l_1=f'(a)$.\\

		    \item Tomemos $l_1>f'(a)$. Sea $\epsilon>0$ tal que
		    $$l_1-\epsilon>f'(a).$$
		    Ya que $\lim\limits_{x\to a^-}f'(x)=l_1$, entonces existe un $\delta>0$ tal que
		    $$l_1-\epsilon<f'(x)<l_1+\epsilon,\quad \forall x\in (a-\delta,a)\cap(x_1,x_2).$$
		    Luego, sea $\beta$ un elemento de $x\in (a-\delta,a)\cap(x_1,x_2)$. Entonces, tenemos
		    $$f'(a)<l_1-\epsilon<f'(\beta).$$
		    Por el teorema de Darboux (Problema 60). Existe un punto $\mathcal{X}\in (\alpha,a)$ de modo que
		    $$f'\left(\mathcal{X}\right)=l_1-\epsilon.$$
		    Notemos que
		    $$\mathcal{X}\in \left(\beta,a\right)\quad\Rightarrow \quad \mathcal{X}\in (a-\delta,a)\cap (x_1,x_2).$$
		    Se sigue que
		    $$f'\left(\mathcal{X}\right)>l_1-\epsilon.$$
		    Lo que es imposible. Así, de los dos casos se sigue que
		    $$\lim_{x\to a^-}f'(x)=f'(a)=l_1.$$\\

		    Ahora demostraremos que $f'(a)=l_2$. Supongamos $l_2\neq f'(a).$\\

		    \begin{enumerate}[\textbf{Caso} 1.]
			\item Tomemos $l_2<f'(a)$. Sea $\epsilon>0$ tal que
			$$l_2+\epsilon<f'(a).$$
			Ya que $\lim\limits_{x\to a^+}f'(x)=l_2$, entonces existe un $\delta>0$ tal que
			$$l_2-\epsilon<f'(x)<l_2+\epsilon,\quad \forall x\in (a,a+\delta)\cap(x_1,x_2).$$
			Luego, sea $\alpha$ un elemento de $x\in (a,a+\delta)\cap(x_1,x_2)$. Entonces, tenemos
			$$l_2-\epsilon<f'(\alpha)<l_2+\epsilon<f'(a).$$
			Por el teorema de Darboux (Problema 60). Existe un punto $\mathcal{X}\in (a,\alpha)$ de modo que
			$$f'\left(\mathcal{X}\right)=l_2+\epsilon.$$
			Pero, notemos que
			$$\mathcal{X}\in \left(a,\alpha\right)\quad\Rightarrow \quad \mathcal{X}\in (a,a+\delta)\cap (x_1,x_2).$$
			De donde, se sigue que
			$$f'\left(\mathcal{X}\right)<l_2+\epsilon.$$
			Lo que es imposible. Por lo tanto, $l_2<f'(a)$ es erróneo.\\

			\item Tomemos $l_2>f'(a)$. Sea $\epsilon>0$ tal que
			$$l_2-\epsilon>f'(a).$$
			Ya que $\lim\limits_{x\to a^+}f'(x)=l_2$, entonces existe un $\delta>0$ tal que
			$$l_2-\epsilon<f'(x)<l_2+\epsilon,\quad \forall x\in (a,a+\delta)\cap(x_1,x_2).$$
			Luego, sea $\beta$ un elemento de $x\in (a,a+\delta)\cap(x_1,x_2)$. Entonces, tenemos
			$$f'(a)<l_2-\epsilon<f'(\beta).$$
			Por el teorema de Darboux (Problema 60). Existe un punto $\mathcal{X}\in (a,\beta)$ de modo que
			$$f'\left(\mathcal{X}\right)=l_2-\epsilon.$$
			Pero, notemos que
			$$\mathcal{X}\in \left(a,\beta\right)\quad\Rightarrow \quad \mathcal{X}\in (a,a+\delta)\cap (x_1,x_2).$$
			De donde, se sigue que
			$$f'\left(\mathcal{X}\right)>l_2-\epsilon.$$
			Lo que es imposible. Así, de los dos casos se sigue que
			$$\lim_{x\to a^+}f'(x)=f'(a)=l_2.$$

			En consecuencia tenemos 
			$$\lim_{x\to a^+}f'(x)=\lim_{x\to a^-}f'(x)=f'(a).$$

			Se sigue que $f'$ es continua en el punto $a$, lo que contradice la hipótesis. Así, nuestra suposición de que el límite de ambos lados existe es incorrecta . Que implica que $\lim\limits_{x\to a^+}f'(x)$ y $\lim\limits_{x\to a^-}f'(x)$ no existen.\\\\

		    \end{enumerate}

		\end{enumerate}


	    %--------- (b)
	    \item Ambos límites laterales no pueden existir incluso si se acepta que puedan tener valores iguales $a+\infty$ o $-\infty$. Indicación: Utilice el teorema de Darboux (Problema 60).\\\\
		Demostración.-\; Supongamos que el límite de $f'$ a ambos lados existe en el punto $a$si los valores se toman como infinitos; es decir, sin pérdida de generalidad. Supongamos
		$$\lim_{x\to a^+}f'(x)=\infty\quad \mbox{y} \lim_{x\to a^-}f'(x)=-\infty.$$
		y $f'(a)\neq 0$. Luego, sea $\epsilon>0$. Entonces, existe un $\delta>0$ y un número positivo $M$ tal que
		$$
		\begin{array}{rcr}
		    f'(x)&>&M \quad \forall x\in (a,a+\delta)\\
		    f'(x)&<&-M \quad \forall x\in (a-\delta,a)
		\end{array}
		$$
		Luego, consideremos dos elementos $\alpha$ y $\beta$ de modo que
		$$a\in(a,a+\delta)\quad \mbox{y}\quad \beta\in (a-\delta,a).$$
		Entonces, se tiene
		$$f'(a)>M\quad \mbox{y}\quad f'(\beta)<-M.$$
		Después, sea $[\beta,\alpha]$ donde
		$$f'(\beta)<0<f'(\alpha).$$
		Por el teorema de Darboux (Problema 60). Existe un elemento $z \in [\beta,\alpha]$ de modo que
		$$f'(z)=0.$$
		Pero, vemos que
		$$\left|f'(x)\right|>0\; \forall x\in [\beta,\alpha] \mbox{ donde } f'(a)\neq 0.$$
		Así, llegamos a una contradicción. Por lo tanto, nuestra suposición de que tanto el límite lateral de $f'$ en el punto $a$ existe, está erróneo. \\\\

	\end{enumerate}

    %----------------- 62
    \item Es fácil encontrar una función $f$ tal que $|f|$ sea diferenciable sin serlo $f$. Por ejemplo, podemos elegir $f(x)=1$ para $x$ racional y $f(x)=-1$ para $x$ irracional. En este ejemplo $f$ no es ni siquiera continua y esto no es una merca coincidencia. Demuestre que si $|f|$ es diferenciable en $a$ y $f$ es continua en $a$, entonces $f$ es también diferenciable en $a$. Indicación: Basta considerar solamente $a$ con $f(a)=0$. ¿Por qué? En este caso, ¿cómo debe ser $|f|'(a)$?.\\\\
	Demostración.-\; Sea $a$ cualquier punto en el dominio de la definición de $f$. Entonces, también $f(a)=0$ o $f(a)\neq 0$. Supongamos que $f(a)\neq 0$. Entonces, por continuidad de $f$ tenemos que existe un $\delta>0$ tal que en $(a-\delta,a+\delta)$, $f$ es $|f|$ o $-|f|$. Ya que $|f|$ es diferenciable en ambos casos, $f$ tiene que ser diferenciable. Así, tenemos que $f(a)=0$.\\
	Ahora, si $f(a)=0$, entonces $|f|(a)=0$; lo que implica que $a$ es el punto mínimo para $|f|$. Ya que $|f|$ es diferenciable tenemos
	$$|f|'(a)=0.$$
	Lo que implica por definición de diferenciablidad que
	$$
	\begin{array}{rcll}
	    0 &=& \displaystyle\lim_{h\to 0} \dfrac{|f(a+h)|-|f(a)|}{h}&\\\\
	      &=& \displaystyle\lim_{h\to 0} \dfrac{|f(a+h)|}{h}&\mbox{ya que }f(a)=0\\\\
	      &=& \displaystyle\lim_{h\to 0} \dfrac{f(a+h)}{h}.
	\end{array}
	$$
	Luego, $f'(a)$ existe y es igual a cero.\\\\

    %----------------- 63
    \item 
	\begin{enumerate}[(a)]

	    %---------- (a)
	    \item Sea $y\neq 0$ y sea $n$ par. Demuestre que $x^n+y^n=(x+y)^n$ solamente cuando $x=0$. Indicación: Si $x_0^n+y^n= (x_0+y)^n$, aplicar el Teorema de Rolle a $f(x)=x^n+y^n-(x+y)^n$ en $[0,x_0]$.\\\\
		Demostración.-\; Notemos que
		$$f(0)=y^n+y^n=0, \mbox{ y } f(x_0)=x_0^n+y^n-(x+y)^n=0$$
		donde se cumple  la segunda ecuación de nuestra hipótesis. Además, $f$ es diferenciable en $[0,x_0]$ o en $[x_0.0]$. Aplicando el Teorema de Rolle para la función $f$, tenemos que existe un número $x_1\in (0,x_0)$ o en $(x_0,0)$ tal que
		$$f'(x_1)=0.$$
		Calculando se tiene
		$$f'(x)=n x^{n-1}-n(x+y)^{n-1}.$$
		Por lo tanto, para $x_1$ se tiene
		$$f'(x_1)=n\left[x_1^{n-1}-(x_1+y)^{n-1}\right]=0$$
		lo que equivale a
		$$x_1^{n-1}=(x_1+y)^{n-1}.$$
		Pero como $n$ es par, $n-1$ es impar, por lo que
		$$x_1=x_1+y$$
		lo que implica que $y=0$; esto contradice la hipótesis. Así, $x$ tiene que ser $0$ para que $X^n+y^n=(x+y)^n$.\\\\

	    %---------- (b)
	    \item Demuestre que si $y\neq 0$ y $n$ es impar, entonces $x^n+y^n = (x+y)^n$ sólo si $x=0$ ó $x=-y$.\\\\
		Demostración.-\; Supongamos que $x\neq 0$ y 
		$$x^n+y^n=(x+y)^n.$$
		Entonces, tenemos que demostrar que $x=-y$. Consideremos la función $f$ definida en $[0,x_0]$, como sigue:
		$$f(x)=x^n+y^n-(x+y)^n.$$
		Por la parte (a), sabemos que $f(0)=0$, $f(x_0)=0$ y que
		$$f(-y)=-y^n+y^n(y-y)^n=0,$$
		donde se cumple la segunda igualdad, ya que $n$ es impar. Luego, aplicando el Teorema de Rolle para la función $f$ tenemos que existen $x_1\in (-y,0)$ y $x_2\in (0,x_0)$ tal que
		$$f'(x_1)=f'(x_2)=0.$$
		Calculando se tiene
		$$f'(x)=n x^{n-1}-n(x+y)^{n-1}.$$
		Por lo tanto, para $x_1$ y $x_2$ se tiene
		$$f'(x_1)=n\left[x_1^{n-1}-(x_1+y)^{n-1}\right]=0$$
		y
		$$f'(x_2)=n\left[x_2^{n-1}-(x_2+y)^{n-1}\right]=0.$$
		Pero como $n$ es impar, por el problema 6(d) del capítulo $1$ tenemos que $x_j=x_j+y$ o $x_j=-(x_j+y)$ con $j=1,2$. Después, por el hecho de que $y\neq 0$, $x_j\neq x_j+y$; debemos tener $x_j=-(x_j+y)$ para $j=1,2$. Así, $x_1=x_2$, lo que implica que $x_0=-y$.\\\\

	\end{enumerate}

    %----------------- 64
    \item Suponga que $f(0)=0$ y que $f'$ es creciente. Demuestre que la función $g(x)=\dfrac{f(x)}{x}$ es creciente en $(0,\infty)$. Indicación: Evidentemente hay que considerar a $g'(x)$. Demuestre que es positiva aplicando el Teorema del Valor Medio a $f$ en el intervalo adecuado (será útil recordar que la hipótesis $f(0)=0$ es esencial, como la demuestra la función $f(x)=1+x^2$).\\\\
	Demostración.-\; calculemos $g'(x)$:
	$$g'(x)=\dfrac{xf'(x)-f(x)}{x^2}.$$
	Ahora, para cualquier $x\in (0,\infty)$, aplicando el Teorema del Valor Medio a $f$ en el intervalo $[0,x]$ tenemos que existe un número $c_x\in (0,x)$ tal que
	$$f(x)-f(0)=xf'(c_x).$$
	Pero, por hipótesis tenemos que $f(0)=0$ de donde
	$$f(x)=xf'(c_x)<xf'(x)$$
	esto ya que $f'$ es creciente, $c_x<x$ y $x>0$. Esto demuestra que 
	$$f(x)=xf'(c_x)<xf'(x)$$
	lo que implica que que $g'(x)>0$. Así, demostramos que $g'(x)>0$ para todo $x\in (0,\infty)$ y $g$ es creciente es dicho intervalo.\\\\ 

    %----------------- 65
    \item Utilice derivadas para demostrar que si $n\geq 1$, entonces 
    $$(1+x)^n>1+nx\quad \mbox{para }-1<x<0 \mbox{ y } 0<x.$$
    (Observe que la igualdad se cumple para $x=0$).\\\\
	Demostración.-\; Consideremos la función $g(x)=(1+x)^n-(1+nx)$. Entonces, notemos que $g(0)=0$. Además, calculando la derivada vemos que
	$$g'(x)=n(1+x)^{n-1}-n=n\left[(1+x)^{n-1}-1\right]$$
	Ahora, ya que $n-1\geq 0$ vemos que para $x>0$,
	$$(1+x)^{n-1}-1>0,$$
	esto es, $g'(x)>0$ para $x>0$. Por otro lado, si $-1<x<0$, entonces $(1+x)<1$ y por lo tanto
	$$(1+x)^{n-1}<1$$
	lo que muestra que $g'(x)<0$ en $-1<x<0$. Así,  tenemos que $g$ es creciente para $x>0$ y decreciente en $-1<x<0$. Pero sabemos que $g(0)=0$. Por lo que tenemos $g(x)>0$ en $(0,\infty)$ y $f(x)<0$ en $-1<x<0$. Es decir,
	$$(1+x)^n>1+nx\quad \mbox{para }-1<x<0 \mbox{ y } 0<x.$$\\

    %----------------- 66
    \item Sea $f(x)=x^4\sen^2\dfrac{1}{2}$ para $x\neq 0$ y sea $f(0)=0$. 
	\begin{enumerate}

	    %---------- (a)
	    \item Demuestre que $0$ es un punto mínimo local de $f$.\\\\
		Demostración.-\; Sea $f$ un función no constante. Esto es, $f(a')=f(b')$ para algún $a'<b'$ en $[a,b]$. Para ser especifico, $f(a')<f(b')$. Por el problema 8-4(b)m existe $a'\leq c<d\leq b'$ con $f(c)=f(a')<f(b')=f(d)$ y $f(c)<f(x)<f(d)$ para todo $x\in(c,d)$. De donde  $a'$  no es un mínimo local para $f$.\\\\

	    %---------- (b)
	    \item Demuestre que $f'(x)=f''(0)=0.$\\\\
		Demostración.-\; Calculemos $f'(x)$:
		$$
		\begin{array}{rcl}
		    f'(0) &=& \displaystyle\lim_{x\to 0} \dfrac{x^4\sen^2\dfrac{1}{x}}{x}\\\\
			  &=& \displaystyle\lim_{x\to 0}x^3\sen^2\dfrac{1}{x}\\\\
			  &=& \displaystyle\lim_{x\to 0}x^3\lim_{x\to 0}\sen^2\dfrac{1}{x}\\\\
			  &=& \displaystyle\lim_{x\to 0} x \lim_{x\to 0}x^2\lim_{x\to 0}\sen^2\dfrac{1}{x}\\\\
			  &=& 0.
		\end{array}
		$$

		Ahora, para demostrar que $f''(0)=0$, con $x\neq 0$, que
		$$
		\begin{array}{rcl}
		    f'(x) &=& 4x^3\sen^2\dfrac{1}{x}-x^4\left(-\dfrac{1}{^2}\right)2\sen \dfrac{1}{x}\cos\dfrac{1}{x}\\\\
			  &=& 4x^3\sen^2\dfrac{1}{x}+x^2\sen\dfrac{2}{x}.
		\end{array}
		$$
		Por lo tanto, tenemos que

		$$f'(0)=\lim_{x\to 0}\dfrac{4x^3\sen^2\dfrac{1}{x}+x^2\sen\dfrac{2}{x}}{x}=0$$

		Así, $f''(0)=0.$\\

	\end{enumerate}

	Esta función constituye otro ejemplo demostrativo de que el Teorema 6 no puede ser mejorado.
También ilustra una sutileza acerca de los máximos y mínimos que frecuentemente pasa desaperci­
bida: una función puede no ser creciente en ningún intervalo a la derecha de un punto mínimo local
ni tampoco decreciente en ningún intervalo a la izquierda.\\\\

    %-------------------- 67
    \item 
	\begin{enumerate}[(a)]

	    %---------- (a)
	    \item Demuestre que si $f'(a)>0$ y $f'$ es continua en $a$, entonces $f$ es creciente en algún intervalo que contiene a $a$.\\\\
		El comportamiento de $f$ para $\alpha\geq 1$, que es mucho más difícil de analizar, se discute en el problema siguiente.\\\\
		Demostración.-\;

	    Las dos partes siguientes de este problema demuestran que la continuidad de $f'$ es esencial.\\

	    %---------- (b)
	    \item Si $g(x)=x^2\sen 1/x$, demuestre que existen números $x$ tan próximos como se quiere a $0$ con $g'(x)=1$ y también con $g'(x)=-1$.\\\\
		Demostración.-\;

	    %---------- (c)
	    \item Suponga que $0<\alpha<1$. Sea $f(x)=\alpha x+x^2\sen 1/x$ para $x\neq 0$ y sea $f(0)=0$. Demuestre que $f$ no es creciente en ningún intervalo abierto que contenga a $0$, probando que cualquiera de estos intervalos existen puntos $x$ con $f'(x)>0$ y también puntos $x$ con $f'(x)<0$.\\\\
		Demostración.-\;

		El comportamiento de $f$ para $\alpha\geq 1$, que es mucho más difícil de analizar, se discute en el problema siguiente.\\\\
	\end{enumerate}

	
\end{enumerate}



	


%--------------------george thomas--------------------

    %---------- caratula
	%\begin{tabular}{r l }
Universidad: & \textbf{Mayor de San Ándres.}\\
Asignatura: & \textbf{Álgebra Lineal I}\\
Ejercicio: & \textbf{Práctica 1.}\\ 
Alumno: & \textbf{PAREDES AGUILERA CHRISTIAN LIMBERT.}
\end{tabular}
\begin{flushleft}
\begin{tikzpicture}
\draw(0,1)--(16.5,1);
\end{tikzpicture}
\end{flushleft}



    %---------- funciones
	%\chapter{Funciones}

\section{Las funciones y sus gráficas}

    %--------------------definición 1.1.
    \begin{tcolorbox}[colframe=white]
	\begin{def.}
	    Una función $f$ de un conjunto $D$ a un conjunto $Y$ es una regla que asigna a cada elemento $x \in D$ un solo o único elemento $f(x) \in Y$\\
	\end{def.}
    \end{tcolorbox}

    %--------------------definición 1.2.
    \begin{tcolorbox}[colframe=white]
	\begin{def.}
	    Cuando definimos una función $y = f(x)$ mediante una fórmula, y el dominio no se establece de forma explícita o se restringe por el contexto, se supondrá que el dominio será el mayor conjunto de números reales $x$ para los cuales la fórmula proporciona valores reales para $y$ , el llamado \textbf{dominio natural}.\\\\
	    Cuando el rango de una función es un subconjunto de números reales, se dice que la función tiene \textbf{valores reales (o que es real valuada)}\\
	\end{def.}
    \end{tcolorbox}

    %--------------------definición 1.3.
    \begin{tcolorbox}[colframe=white]
	\begin{def.}[Valor absoluto]
	    $f(x)=\left\{\begin{array}{rcl}
		x&si&x\geq 0\\
		\\ -x&si&x<0\\
	    \end{array} \right.$
	\end{def.}
    \end{tcolorbox}

    %--------------------definición 1.4.
    \begin{tcolorbox}[colframe=white]
	\begin{def.}
	    Sea una funcion definida en un intervalo $I$ y sean $x_1$ y $x_2$ cualesquiera dos puntos en $I$ 
	    \begin{enumerate}[\bfseries 1.]
		\item Si $f(x_2) > f(x_1)$, siempre que $x_1<x_2$ entonces se dice que $f$ es \textbf{creciente} en $I$.
		\item Si $f(x_2) < f(x_1)$, siempre que $x_1<x_2$ entonces se dice que $f$ es $\textbf{decreciente}$ en $I$.\\
	    \end{enumerate}
	\end{def.}
    \end{tcolorbox}

    %--------------------definición 1.5.
    \begin{tcolorbox}[colframe=white]
	\begin{def.}
	Una función $y=f(x)$ es una 
	    \begin{enumerate}[\bfseries 1.]
		\item Función par de $x$ si $f(-x)=f(x)$.
		\item Función impar de $x$ si $f(-x)=-f(x)$.
	    \end{enumerate}
	Para toda $x$ en el dominio de la función.\\\\
	(Los nombres par e impar provienen de las potencias de $x$).\\
	\end{def.}
    \end{tcolorbox}

    %--------------------definición 1.6.
    \begin{tcolorbox}[colframe=white]
	\begin{def.}
	    Dos variables $x$ e $y$ son \textbf{proporcionales} (una con respecto a la otra) si una siempre es un múltiplo constante de la otra; esto es, si $y=kx$ para alguna constante $k$ distinta de $0$.\\\\
	    Si la variable $y$ es proporcional al recíproco $1/x$, entonces algunas veces se dice que $y$ es \textbf{inversamente proporcional} a $x$ (puesto que $1/x$ es el inverso multiplicativo de $x$).\\

	\end{def.}
    \end{tcolorbox}


\setcounter{section}{0}
\section{Ejercicios}

\begin{enumerate}[\Large \bfseries 1.]

    %--------------------1.
    \item $f(x)=1+x^2$ \\\\
	Respuesta.-\; Al evaluar $1+x^2$ vemos que $x$ se cumple para todos los reales, por lo tanto $f_D=\lbrace x / \; \forall \; x \in \mathbb{R} \rbrace$. Luego el rango viene dado por $f_R=\lbrace y=f(x) / y \geq 1 \rbrace$\\\\

    %--------------------2.
    \item $f(x)=1-\sqrt{x}$\\\\
       Respuesta.-\; El dominio viene dado por $f_D=\lbrace x / x \geq 0 \rbrace$. Y el rango viene dado por $f_R = \lbrace y = f(x) / y \leq 1 \rbrace$.\\\\

    %--------------------3.
    \item $F(x)=\sqrt{5x + 10}$\\\\
	Respuesta.-\; Sea $5x + 10 \geq 0$ ya que una raíz par no puede ser no negativo, entonces $x \geq 2$, por lo tanto el dominio viene dado por $f_D=\lbrace x / x \geq -2 \rbrace$. Luego el rango viene dado por $f_R = \lbrace y=f(x) / y \geq 0 \rbrace$.\\\\

    %--------------------4.
    \item $g(x)=\sqrt{x^2 - 3x}$\\\\
	Respuesta.-\; De igual forma al anterior ejercicio, evaluaremos $x^2 - 3x \geq 0$, de donde $x(x-3)\geq 0$, por lo tanto el dominio es $f_D=\lbrace x/\leq x \leq 0 \cup x \geq 3 \rbrace$. Luego el rango viene definido por $f_R=\lbrace y=f(x) / y \geq 0 \rbrace$.\\\\

    %--------------------5.
    \item $f(t)=\dfrac{4}{3-t}$ \\\\
	Respuesta.-\; Sabemos que no se puede dividir un número por $0$. Por lo tanto para hallar el dominio de la función debemos evaluar $3-t=0$, de donde $t=3$, así $f_D=\lbrace t / t\neq 3\rbrace$. Luego el rango viene dado por $f_R=\lbrace y=f(x) / y\neq 0\rbrace$.\\\\

    %--------------------6.
    \item $G(t)=\dfrac{2}{t^2 - 16}$ \\\\
	Respuesta.-\; De igual forma al anterior ejercicio evaluamos $t^2 - 16 = 0$, de donde $(t - 4)(t + 4)=0$, por lo tanto el dominio de la función viene dado por $f_D=\lbrace t / t \neq 4 \land t \neq -4 \rbrace$. Luego el rango viene dado por $f_R=\lbrace y=f(x) / 0 < y \leq - \dfrac{1}{8} \rbrace$ ya que al despejar $x$ nos queda  $x=\sqrt{\dfrac{2}{y} + 16}$ de donde se debe evaluar por un lado $\dfrac{2}{y}$ y por otro $\dfrac{2}{y} - 16 \geq 0$.\\\\

    En los ejercicios $7$ y $8$ ¿Cuál de las gráficas representa la gráfica de una función de $x$? ¿Cuáles no representan a funciones de $x$? Dé razones que apoyen sus respuestas.\\\\

    %--------------------7.
    \item El inciso $a.$ no es una función ya que no cumple con la prueba de la recta vertical ya una función sólo puede tener un valor $f(x)$ para cada $x$ en su dominio. Y el inciso $b.$ no representa la gráfica de una función.\\\\

    %--------------------8.
    \item Los incisos $a.$ y $b.$ no representan a funciones de $x$. El único que no representa una gráfica de una función es el inciso $b.$\\\\

    Determinación de fórmulas para funciones.\\\\

    %--------------------9.
    \item Exprese el área y el perímetro de un triángulo equilátero como una función del lado $x$ del triángulo.\\\\
	Respuesta.-\; El área se representa por $f(x)=\dfrac{\sqrt{3}a^2}{4}$ y el perímetro por $f(x)=3x$\\\\

    %--------------------10.
    \item Exprese la longitud del lado de un cuadrado como una función de la longitud $d$ de la diagonal del cuadrado. Exprese el área como una función de la longitud de la diagonal.\\\\
	Respuesta.-\; La longitud del lado de un cuadrado como función de longitud esta dado por $d=\sqrt{2a^2}$. El área es expresado por $A=\dfrac{d^2}{2}$\\\\

    %--------------------11.
    \item Exprese la longitud del lado de um cubo como una función de la longitud de la diagonal $d$ del cubo. Exprese el área de la superficie y el volumen del cubo como una función de la longitud de la diagonal.\\\\
	Respuesta-.\; La expresión de la longitud del lado del cubo como función de la longitud de la diagonal $d$ del cubo es  $$ L(d) = (\sqrt{2}/2)\cdot d $$ 
	Las expresiones del área de la superficie y el volumen del cubo como función de la longitud de la diagonal $d$ del cubo son:
	$$A(d) = 3\cdot d^2 \quad    y \quad  V(d) = (\sqrt{2}/4)\cdot d^3$$

    %--------------------12.
    \item Un punto $P$ en el primer cuadrante pertenece a la gráfica de la función $f(x)=\sqrt{x}$. Exprese las coordenadas de $P$ como funciones de la pendiente de la recta que une a $P$ con el origen.\\\\
	Respuesta.-\; Sea el punto en el origen $(0,0)$ y el punto $P$ tenga las coordenadas $(z,z^{'})$. Sabemos que una recta viene definido por $f(x)=ax+b$ entonces formando un sistema de ecuaciones tenemos:
	$$0=0x + b \quad y \quad z^{'} = az + b$$
	Luego $z^{'}=az$ de donde $a=\dfrac{z^{'}}{z}$, y así nos queda la función  
	$$f(x)=\dfrac{z^{'}}{z} x$$\\\\

    %--------------------13.
    \item Considere el punto $(x,y)$ que está en la gráfica de la recta $2x + 4y = 5$. Sea $L$ la distancia del punto $(x, y)$ al origen $(0, 0)$. Escriba $L$ como función de $x$.\\\\
	Respuesta.-\; Dado $(x,y) \in 2x+4y=5 ; (0,0)$ entonces $$x=\dfrac{5-4y}{2} \qquad \dfrac{5-2x}{4}$$
	Luego $L=\sqrt{(y-0)^2+(x-0)^2} = \sqrt{y^2 + \left( \dfrac{5-4y}{2}\right)}=\sqrt{y^2 + \dfrac{25+40y+16y^2}{4}}=\sqrt{\dfrac{4y^2}{4} + \dfrac{25 - 40y + 16y^2}{4}} = \dfrac{1}{2} \sqrt{20y^2 + 40y + 25}$\\\\
 
    %--------------------14.
    \item Considere el punto $(x, y)$ que está en la gráfica de $y = \sqrt{x - 3}$. Sea $L$ la distancia entre los puntos $(x,y)$ y $(4,0)$. Escriba $L$ como función de $y$.\\\\
	Respuesta.-\; $y=\sqrt{x-3}, (x,y)\in y=\sqrt{x-3}$ entonces calculamos la distancia entre $y=\sqrt{x-3}$ y $(4,0)$.\\
	$$y^2=x-3 \Longrightarrow x=y^2 + 3 \quad y \quad y=\sqrt{x-3}$$ 
	Así $L=\sqrt{(y-o)^2 + (x-y)^2} = \sqrt{y^2 + (y^2 + 3)^2} = \sqrt{y^2 + y^4 + 6y^2 + 9} = \sqrt{y^4+7y^2+9}$\\\\

    Las funciones y sus gráficas.\\\\
    En los ejercicios $15$ al $20$, determine el dominio y grafique las funciones\\\\

    %--------------------15.
    \item $f(x)=5-2x$\\\\
	Respuesta.-\; El dominio esta dado para todos los reales $x$.
	\begin{center}
	    \begin{tikzpicture}[scale=1,draw opacity = 0.6]
		% abscisa y ordenada
		\tkzInit[xmax= 3,xmin=-2,ymax=5,ymin=-1]
		\tiny\tkzLabelXY[opacity=0.6,step=1, orig=false]
		% etiqueta x, f(x)
		\tkzDrawX[opacity=0.6,label=x,right=0.3]
		\tkzDrawY[opacity=0.6,label=f(x),below = -0.6]
		%dominio y función
		\draw [domain=-1:3,thick,gray] plot(\x,{5-2*\x});
		\tkzText[opacity=0.6,above](2,3){\tiny $f(x)=5-2x$}
	    \end{tikzpicture}
	\end{center}
	\vspace{.5cm}

    %--------------------16.
    \item $f(x)=1-2x-x^2$\\\\ 
	Respuesta.-\; El dominio viene dado para todo real $x$ positivo.
	\begin{center}
	    \begin{tikzpicture}[scale=1,draw opacity = 0.6]
		% abscisa y ordenada
		\tkzInit[xmax= 2,xmin=-3,ymax=3,ymin=-3]
		\tiny\tkzLabelXY[opacity=0.6,step=1, orig=false]
		% etiqueta x, f(x)
		\tkzDrawX[opacity=0.6,label=x,right=0.3]
		\tkzDrawY[opacity=0.6,label=f(x),below = -0.6]
		%dominio y función
		\draw [domain=-3:1,thick,gray] plot(\x,{1-2*\x - \x*\x});
		\tkzText[opacity=0.6,above](1.3,1){\tiny $f(x)=1-2x-x^2$}
	    \end{tikzpicture}
	\end{center}
	\vspace{.5cm}

    %--------------------17.
    \item $g(x)=\sqrt{|x|}$\\\\ 
	Respuesta.-\; El dominio de la función es para $x \in \mathbb{R}$
	\begin{center}
	    \begin{tikzpicture}[scale=1,draw opacity = 0.6]
		% abscisa y ordenada
		\tkzInit[xmax= 3,xmin=-3,ymax=3,ymin=0]
		\tiny\tkzLabelXY[opacity=0.6,step=1, orig=false]
		% etiqueta x, f(x)
		\tkzDrawX[opacity=0.6,label=x,right=0.3]
		\tkzDrawY[opacity=0.6,label=f(x),below = -0.6]
		%dominio y función
		\draw [domain=-3:3,thick,gray] plot(\x,{abs(\x)^(1/2)});
		\tkzText[opacity=0.6,above](1,1.7){\tiny $g(x)=\sqrt{|x|}$}
	    \end{tikzpicture}
	\end{center}
	\vspace{.5cm}

    %--------------------18.
    \item $g(x) = \sqrt{-x}$\\\\
	Respuesta.-\; El dominio de la función se cumple para los números reales negativos.   
	\begin{center}
	    \begin{tikzpicture}[scale=1,draw opacity = 0.6]
		% abscisa y ordenada
		\tkzInit[xmax= 4,xmin=-1,ymax=1,ymin=-2]
		\tiny\tkzLabelXY[opacity=0.6,step=1, orig=false]
		% etiqueta x, f(x)
		\tkzDrawX[opacity=0.6,label=x,right=0.3]
		\tkzDrawY[opacity=0.6,label=f(x),below = -0.6]
		%dominio y función
		\draw [domain=0:4,thick,gray] plot(\x,{-\x^(1/2)});
		\tkzText[opacity=0.6,above](2,-1){\tiny $f(x)=\sqrt{-x}$}
	    \end{tikzpicture}
	\end{center}
	\vspace{.5cm}

    %--------------------19.
    \item $F(t)=t/|t|$\\\\
	Respuesta.-\; El dominio viene dado para todo número real menos el $0$. 
	\begin{center}
	    \begin{tikzpicture}[scale=1,draw opacity = 0.6]
		% abscisa y ordenada
		\tkzInit[xmax= 3,xmin=-3,ymax=2,ymin=-2]
		\tiny\tkzLabelXY[opacity=0.6,step=1, orig=false]
		% etiqueta x, f(x)
		\tkzDrawX[opacity=0.6,label=x,right=0.3]
		\tkzDrawY[opacity=0.6,label=f(x),below = -0.6]
		%dominio y función
		\draw [domain=.1:3,thick,gray] plot(\x,{\x/abs(\x)});
		\draw [domain=-3:-.1,thick,gray] plot(\x,{\x/abs(\x)});
		\tkzText[opacity=0.6,above](1,1){\tiny $F(t)=t/|t|$}
	    \end{tikzpicture}
	\end{center}
	\vspace{.5cm}

    %--------------------20.
    \item $G(t)=1/|t|$\\\\
	Respuesta.-\; El dominio se cumple para todo número real menos el $0$.
	\begin{center}
	    \begin{tikzpicture}[scale=1,draw opacity = 0.6]
		% abscisa y ordenada
		\tkzInit[xmax= 3,xmin=-3,ymax=5,ymin=0]
		\tiny\tkzLabelXY[opacity=0.6,step=1, orig=false]
		% etiqueta x, f(x)
		\tkzDrawX[opacity=0.6,label=x,right=0.3]
		\tkzDrawY[opacity=0.6,label=f(x),below = -0.6]
		%dominio y función
		\draw [domain=.2:3,thick,gray] plot(\x,{1/abs(\x)});
		\draw [domain=-3:-.2,thick,gray] plot(\x,{1/abs(\x)});
		\tkzText[opacity=0.6,above](2,2){\tiny $G(t)=1/|t|$}
	    \end{tikzpicture}
	\end{center}
	\vspace{.5cm}

    %--------------------21
    \item Determine el dominio de $y=\dfrac{x+3}{4-\sqrt{x^2-9}}$\\\\
	Respuesta.-\; Si $y=f(x)$ entonces el dominio esta dado por $D_f=\lbrace x / x\geq 3 \land x \neq 4 \rbrace$ \\\\

    %--------------------22.
    \item Determine el rango de $y=2+\dfrac{x^2}{x^2+4}$.\\\\
	Respuesta.-\; Si $y=f(x)$ entonces el rango viene dado para todo $y=f(x)$ tal que $y\geq 2$\\\\

    %--------------------23.
    \item Grafique las siguientes ecuaciones y explique por qué no son gráficas de funciones de $x$.\\\\
    
    \begin{enumerate}[\bfseries a.]
	
	%----------a.
	\item $|y|=x$\\\\
	    Respuesta.-\; No es una función de $x$ ya que $ \sqrt{y^2}=x \Longrightarrow y^2=x^2 \Longrightarrow \pm y = \pm x$\\\\

	%----------b.
	\item $y^2 = x^2$\\\\
	    Respuesta.-\; Por el anterior problema $23a.$\\\\
	
    \end{enumerate}

    %--------------------24.
    \item Grafique las siguientes ecuaciones y explique por qué no son gráficas de funciones de $x$\\\\

    \begin{enumerate}[\bfseries a.]

	%----------a.
	\item $|x|+|y|=1$\\\\
	    Respuesta.-\; Ya que $|y|=1 - |x| \Longrightarrow \sqrt{y^2} = 1 - |x| \Longrightarrow y^2 = (1-|x|)^2 \Longrightarrow \pm y= |1-|x||$\\\\

	%----------b.
	\item $|x+y|=1$\\\\
	Respuesta.-\;  Ya que $\sqrt{(x+y)^2}=1 \Longrightarrow (x+y)^2 = 1 \Longrightarrow x^2 + 2xy + y^2 = 1 \Longrightarrow y^2 = 1 - 2xy - x^2 \Longrightarrow \pm y = \sqrt{1-2xy-x^2}$\\\\
    \end{enumerate}

    Funciones definidas por partes\\\\
    En los ejercicios 25 a 28, grafique las funciones:\\\\

    %--------------------25.
    \item $f(x) = \left\{ \begin{array}{cc}
		    x,&0\leq x \leq 1\\
		    \\ 2-x,&1<x\leq 2 \\
		    \end{array} \right.$
	\begin{center}
	    \begin{tikzpicture}[scale=1,draw opacity = 0.6]
		% abscisa y ordenada
		\tkzInit[xmax= 3,xmin=-1,ymax=2,ymin=0]
		\tiny\tkzLabelXY[opacity=0.6,step=1, orig=false]
		% etiqueta x, f(x)
		\tkzDrawX[opacity=0.6,label=x,right=0.3]
		\tkzDrawY[opacity=0.6,label=f(x),below = -0.6]
		%dominio y función
		\draw [domain=0:1,thick,gray] plot(\x,{\x});
		\draw [domain=1:2,thick,gray] plot(\x,{2-\x});
	    \end{tikzpicture}
	\end{center}
	\vspace{.5cm}
    
    %--------------------26.
    \item $g(x) = \left\{ \begin{array}{cc}
		    1-x,&0\leq x \leq 1\\
		    \\ 2-x,&1<x\leq 2 \\
		    \end{array} \right.$
	\begin{center}
	    \begin{tikzpicture}[scale=1,draw opacity = 0.6]
		% abscisa y ordenada
		\tkzInit[xmax= 3,xmin=-1,ymax=2,ymin=0]
		\tiny\tkzLabelXY[opacity=0.6,step=1, orig=false]
		% etiqueta x, f(x)
		\tkzDrawX[opacity=0.6,label=x,right=0.3]
		\tkzDrawY[opacity=0.6,label=f(x),below = -0.6]
		%dominio y función
		\draw [domain=0:1,thick,gray] plot(\x,{1-\x});
		\draw [domain=1:2,thick,gray] plot(\x,{2-\x});
	    \end{tikzpicture}
	\end{center}
	\vspace{.5cm}
    
    %--------------------27.
    \item $F(x) = \left\{ \begin{array}{cc}
		    4-x^2,&\leq 1 \\
		    \\ x^2 + 2x,& x>1 \\
		    \end{array} \right.$
	\begin{center}
	    \begin{tikzpicture}[scale=1,draw opacity = 0.6]
		% abscisa y ordenada
		\tkzInit[xmax= 3,xmin=-3,ymax=8,ymin=0]
		\tiny\tkzLabelXY[opacity=0.6,step=1, orig=false]
		% etiqueta x, f(x)
		\tkzDrawX[opacity=0.6,label=x,right=0.3]
		\tkzDrawY[opacity=0.6,label=f(x),below = -0.6]
		%dominio y función
		\draw [domain=-2:1,thick,gray] plot(\x,{4-\x*\x});
		\draw [domain=1:2,thick,gray] plot(\x,{\x*\x + 2*\x});
	    \end{tikzpicture}
	\end{center}
	\vspace{.5cm}

    %--------------------28.
    \item $G(x) = \left\{ \begin{array}{cc}
		    1/x,&x<0\\
		    \\ x, & 0\leq x \\
		    \end{array} \right.$
	\begin{center}
	    \begin{tikzpicture}[scale=1,draw opacity = 0.6]
		% abscisa y ordenada
		\tkzInit[xmax= 3,xmin=-4,ymax=2,ymin=-8]
		\tiny\tkzLabelXY[opacity=0.6,step=1, orig=false]
		% etiqueta x, f(x)
		\tkzDrawX[opacity=0.6,label=x,right=0.3]
		\tkzDrawY[opacity=0.6,label=f(x),below = -0.6]
		%dominio y función
		\draw [domain=-3:0,thick,gray] plot(\x,{1*\x^(-1)});
		\draw [domain=0:2,thick,gray] plot(\x,{\x});
	    \end{tikzpicture}
	\end{center}
	\vspace{.5cm}

    Determine una fórmula para cada función graficada en los ejercicios $29$ a $32$\\\\

    %--------------------29
    \item 
    \begin{enumerate}[\bfseries a.]
	
	%----------a.
	\item  Sea $f(x)=ax+b$ entonces $0=b$ y $1=a+b$ luego $a=1$ por lo tanto $f(x)=x$. Por otro lado $1=a+b$ y $0=2a+2 \Longrightarrow a=-1$ de donde se tiene $f(x^{'})=-x+2$ así nos queda la función:
	$$f(x) = \left\{\begin{array}{r c l}
		x&si&0\leq x \leq 1\\
		\\ x+2&si&1\leq x \leq 2 \\
	    \end{array}\right.$$\\\\

	%----------b.
	\item Está definida por $$f(x)=\left\{\begin{array}{rcl} 
				    2&si&0\leq x < 1 \; y \; 2 \leq x < 3\\
				    \\  0&si& 1\leq x < 2 \; y \; 3\leq x \leq 4\\
				    \end{array}\right.$$\\\\

    \end{enumerate}

    %--------------------30.
    \item 
    \begin{enumerate}[\bfseries a.]

	%----------a.
	\item Similar al ejercicio anterior se tiene que la formula 
	$$f(x)= \left\{ \begin{array}{rcl}
		x+2&si& 0 \leq x \leq 2 \\
		\\ 1/2x+5/7&si& 0\leq x \leq 2 \\ \end{array}\right.$$\\\\

	%----------b.
	\item Se tiene 
	$$f(x)=\left\{\begin{array}{rcl} 
		-3x - 3&si&-1\leq x \leq 0\\
		\\-2x + 3&si&\\ \end{array} \right.$$\\\\

    \end{enumerate}

    %--------------------31.
    \item 
    \begin{enumerate}[\bfseries a.]

	%----------a.
	\item  
	$$f(x)= \left\{ \begin{array}{rcl}
		x+2&si& -1 \leq x < 0 \\
		\\ 1&si& 0 < x \leq 1 \\ 
		\\ -2x + 2&si&1 \leq x \leq 3\\
		\end{array}\right.$$\\\\
	
	%----------b.
	\item Sea $(a,b)$ y $(c,d)$ por lo tanto por capitulo $4$ de spivak $f(x)=\dfrac{d-b}{c-a}(x-a)+b$ entonces $(-2,-1)$ y $(0,0)$ así $f(x)=\dfrac{0-1}{0+2}(x+2) + 0 \quad \Longrightarrow \quad f(x) = -\dfrac{1}{2}x - 2$ \\\\
	Luego $f(x)=-2x+2$ y finalmente $f(x)=-1$ de donde,
	$$f(x)= \left\{ \begin{array}{rcl}
		- \dfrac{1}{2}x -2&si& -2 \leq x \leq 0 \\
		\\ 2x+2&si& 0 < x \leq 1 \\ 
		\\ -1&si&1 < x \leq 3\\
		\end{array}\right.$$\\\\
    \end{enumerate}

    %--------------------32.
    \item 
    \begin{enumerate}[\bfseries a.]

	%----------a.
	\item 
	$$f(x)= \left\{ \begin{array}{rcl}
		0&si& 0 \leq x \leq \dfrac{T}{2} \\\\
		\\ \dfrac{2x}{T} - 1&si& \dfrac{T}{2}< x \leq T \\\\ 
		\end{array}\right.$$\\\\

	%----------b.
	\item  
	$$f(x)= \left\{ \begin{array}{rcl}
		A&si& \dfrac{T}{2} \leq x < T \quad y \quad T \leq x < \dfrac{3T}{2}\\\\
		\\ -A&si&\dfrac{T}{2} \leq x < T \quad y \quad \dfrac{3T}{2} \leq x \leq 2T \\\\
		\end{array}\right.$$\\\\

    \end{enumerate}

Las funciones mayor entero y menor entero.\\\\

    %--------------------33.
    \item Para qué valores de $x$ es 
    \begin{enumerate}[\bfseries a.]
	\item $[x]=0$\\\\
	    respuesta.-\; Para $0\leq x < 1$\\\\
	\item $[x] =0$\\\\
	    Respuesta.-\; Para $-1 < x \leq 0$\\\\
    \end{enumerate}

    %--------------------34.
    \item ¿Cuáles valores $x$ de números reales satisfacen la ecuación $[x] = [x]$?\\\\
	Respuesta.-\; Sólo el $0$.\\\\

    %--------------------35.
    \item ¿Es cierto que $[-x] = -[x]$ para todo número real $x$? Justifique su respuesta.\\\\
	Respuesta.-\; Es cierto siempre y cuando sea $x$ un entero. Ya que si $x\in \mathbb{Z}$ entonces $x=n$ para algunos $n\in \mathbb{Z}$, por lo tanto $[x]=n$ y $-x=-n \Longrightarrow [-x] = n \Longrightarrow [-x] = -[x]$. Por otro lado sea $x \notin \mathbb{Z}$ y $[x]=n$ entonces  $n\leq x < n+1 \Longrightarrow -n-1<-x<-n \Longrightarrow [-x] = -n-1 = -[x] - 1$\\\\

    %--------------------36.
	\item Grafique la función 
		$$f(x) = \left\{\begin{array}{cl}
			[x],&x\leq 0\\
			 x,&x<0\\
			\end{array}\right.$$ 
	    ¿Por qué $f(x)$ se donomina parte entera de $x$?\\\\
	    Respuesta.-\; Se denomia porque hace corresponder el número inmediato anterior.\\\\

    Funciones crecientes y funciones decrecientes.\\\\
    Grafique las funciones en los ejercicios 37 a 46. Si tiene simetrías, ¿Qué tipo de simetría tienen? Especifique los intervalos en os que la función es creciente y los intervalos donde la función es decreciente.\\\\ 

    %--------------------37
    \item $y=-x^3$
	\begin{center}
	    \begin{tikzpicture}[scale=1,draw opacity = 0.6]
		% abscisa y ordenada
		\tkzInit[xmax= 2,xmin=-2,ymax=2,ymin=-2.2]
		\tiny\tkzLabelXY[opacity=0.6,step=1, orig=false]
		% etiqueta x, f(x)
		\tkzDrawX[opacity=0.6,label=x,right=0.3]
		\tkzDrawY[opacity=0.6,label=f(x),below = -0.6]
		%dominio y función
		\draw [domain=-1.3:1.3,thick,gray] plot(\x,{-\x^3});
	    \end{tikzpicture}
	\end{center}
	\vspace{.5cm}
	Respuesta.-\; Tiene simetría impar y el intervalo donde decrece esta dado por $(-\infty,\infty)$\\\\

    %--------------------38.
    \item $y=-\dfrac{1}{x^2}$\\\\
	Respuesta.-\; Tiene simetría par y esta dado por lo el intervalo decreciente de $-\infty<x<0$ y por el intervalo creciente $0<x<\infty$

    %--------------------39.
    \item $y=-\dfrac{1}{x}$\\\\
	Respuesta.-\; Tiene simetría impar y viene dado por los intervalos crecientes $-\infty < x < 0$  y $0<x<\infty$\\\\

    %--------------------40.
    \item $y=\dfrac{1}{|x|}$\\\\
	Respuesta.-\; Tiene simetría par y viene dado por el intervalo creciente $-\infty<x<0$ y el intervalo decreciente $0<x<\infty$\\\\

    %--------------------41.
    \item $y=\sqrt{|x|}$\\\\
	\begin{center}
	    \begin{tikzpicture}[scale=1,draw opacity = 0.6]
		% abscisa y ordenada
		\tkzInit[xmax= 2,xmin=-2,ymax=2,ymin=0]
		\tiny\tkzLabelXY[opacity=0.6,step=1, orig=false]
		% etiqueta x, f(x)
		\tkzDrawX[opacity=0.6,label=x,right=0.3]
		\tkzDrawY[opacity=0.6,label=f(x),below = -0.6]
		%dominio y función
		\draw [domain=-2:2,thick,gray] plot(\x,{sqrt(abs(\x))});
	    \end{tikzpicture}
	\end{center}
	\vspace{.5cm}
	Respuesta.-\; Tiene simetría par y esta dado por el intervalo  decreciente  $-\infty < x \leq 0$ y el intervalo creciente $0\leq x < \infty$\\\\

    %--------------------42.
    \item $y=\sqrt{-x}$\\\\
	\begin{center}
	    \begin{tikzpicture}[scale=1,draw opacity = 0.6]
		% abscisa y ordenada
		\tkzInit[xmax= 2,xmin=-2,ymax=2,ymin=0]
		\tiny\tkzLabelXY[opacity=0.6,step=1, orig=false]
		% etiqueta x, f(x)
		\tkzDrawX[opacity=0.6,label=x,right=0.3]
		\tkzDrawY[opacity=0.6,label=f(x),below = -0.6]
		%dominio y función
		\draw [domain=-2:0,thick,gray] plot(\x,{sqrt(-\x)});
	    \end{tikzpicture}
	\end{center}
	\vspace{.5cm}
	Respuesta.- No es ni par ni impar y viene dado por el intervalo decreciente $-\infty < x \leq 0$\\\\

    %-------------------43.
    \item $y=x^3/8$\\\\
	\begin{center}
	    \begin{tikzpicture}[scale=1,draw opacity = 0.6]
		% abscisa y ordenada
		\tkzInit[xmax= 3,xmin=-3,ymax=2,ymin=-2]
		\tiny\tkzLabelXY[opacity=0.6,step=1, orig=false]
		% etiqueta x, f(x)
		\tkzDrawX[opacity=0.6,label=x,right=0.3]
		\tkzDrawY[opacity=0.6,label=f(x),below = -0.6]
		%dominio y función
		\draw [domain=-2.5:2.5,thick,gray] plot(\x,{\x^3/8});
	    \end{tikzpicture}
	\end{center}
	\vspace{.5cm}
	Respuesta.-\; Tiene simetría impar y viene dado por el intervalo creciente $-\infty < x < \infty$\\\\

    %---------------------44.
    \item $y=-4\sqrt{x}$\\\\
	\begin{center}
	    \begin{tikzpicture}[scale=1,draw opacity = 0.6]
		% abscisa y ordenada
		\tkzInit[xmax= 2,xmin=-1,ymax=1,ymin=-4]
		\tiny\tkzLabelXY[opacity=0.6,step=1, orig=false]
		% etiqueta x, f(x)
		\tkzDrawX[opacity=0.6,label=x,right=0.3]
		\tkzDrawY[opacity=0.6,label=f(x),below = -0.6]
		%dominio y función
		\draw [domain=0:1,thick,gray] plot(\x,{-4*sqrt(\x)});
	    \end{tikzpicture}
	\end{center}
	\vspace{.5cm}
	Respuesta.-\; No es ni par ni impar y viene dado por el intervalo $0\leq x < \infty$\\\\

    %--------------------45.
    \item $y=-x^{3/2}$\\\\
	\begin{center}
	    \begin{tikzpicture}[scale=1,draw opacity = 0.6]
		% abscisa y ordenada
		\tkzInit[xmax= 2,xmin=-1,ymax=1,ymin=-1]
		\tiny\tkzLabelXY[opacity=0.6,step=1, orig=false]
		% etiqueta x, f(x)
		\tkzDrawX[opacity=0.6,label=x,right=0.3]
		\tkzDrawY[opacity=0.6,label=f(x),below = -0.6]
		%dominio y función
		\draw [domain=-1:1,thick,gray] plot(\x,{-(\x^(3/2))});
	    \end{tikzpicture}
	\end{center}
	\vspace{.5cm}
	Respuesta.-\; Tiene simetría impar y viene dado por el intervalo decreciente $-\infty < x < \infty$\\\\

    %---------------------46.
    \item $y=(-x)^{2/3}$\\\\
	\begin{center}
	    \begin{tikzpicture}[scale=1,draw opacity = 0.6]
		% abscisa y ordenada
		\tkzInit[xmax= 2,xmin=-1,ymax=1,ymin=-1]
		\tiny\tkzLabelXY[opacity=0.6,step=1, orig=false]
		% etiqueta x, f(x)
		\tkzDrawX[opacity=0.6,label=x,right=0.3]
		\tkzDrawY[opacity=0.6,label=f(x),below = -0.6]
		%dominio y función
		\draw [domain=-1:1,thick,gray] plot(\x,{\x^(2/3)});
	    \end{tikzpicture}
	\end{center}
	\vspace{.5cm}
	Respuesta.-\; La simetría es impar y viene dado por el intervalo creciente $-\infty < x < \infty$\\\\

    Funciones pares y funciones impares \\\\
    En los ejercicios $47$ a $58$, indique si la función es par, impar o de ninguno de estos tipos. Justifique su respuesta.\\\\

    %--------------------47.
    \item $f(x)=3$\\\\
	Respuesta.-\; Sea $f(-x)=3=f(x)$ entonces decimos que la función es par.\\\\

    %--------------------48.
    \item $f(x)=x^{-5}$\\\\
	Respuesta.-\; Sea $f(-x)=(-x)^{-5} = -\left(x^{-5}\right)=-f(x)$, por lo tanto la función es impar.\\\\

    %--------------------49.
    \item $f(x)=x^2 + 1$\\\\
	Respuesta.-\; Sea $f(-x)=(-x)^2 + 1 = x^2 + 1 = f(x)$, de donde se tiene que la función es par.\\\\

    %--------------------50.
    \item $f(x)=x^2 + x$\\\\
	Respuesta.-\; Sea $f(-x)=(-x)^2 + (-x) = x^2 - x$ de donde la función no es par ni impar.\\\\

    %--------------------51.
    \item $g(x)=x^3  + x$\\\\
	Respuesta.-\; Sea $f(-x) = (-x)^3 + (-x) = -(x^3 + x) = -f(x)$ por lo tanto la función es impar.\\\\

    %--------------------52.
    \item $g(x)=x^4 + 3x^2 - 1$\\\\
	Respuesta.-\; Sea $g(-x) = (-x)^4 + 3(-x)^2 - 1 = g(x)$ por lo tanto la función es par.\\\\

    %--------------------53.
    \item $g(x)=\dfrac{1}{x^2-1}$\\\\
	Respuesta.-\; Sea $g(-x)=\dfrac{1}{(-x)^2 - 1} = g(x)$ por lo tanto la función es par.\\\\

    %--------------------54.
    \item $g(x)=\dfrac{x}{x^2 - 1}$\\\\
	Respuesta.-\; Sea $g(-x)=\dfrac{-x}{(-x)^2 - 1} = - \dfrac{x}{x^2 - 1} = -g(x)$ de donde la función es impar.\\\\

    %--------------------55.
    \item $h(t)=\dfrac{1}{t-1}$\\\\
	Respuesta.-\; Sea $h(-t)=\dfrac{1}{-t-1}$ entonces la función no es par ni impar.\\\\

    %--------------------56.
    \item $h(t)=|t^3|$\\\\
	Respuesta.-\; Sea $h(-t)=|(-t)^3| = |t^3| = h(t)$ por lo tanto la función es par.\\\\

    %--------------------57.
    \item $h(t)=2t+1$\\\\
	Respuesta.-\; Sea $h(-t) = 2(-t) + 1$ entonces la función no es ni par ni impar.\\\\

    %--------------------58.
    \item $h(t)=2|t|+1$\\\\
	Respuesta.-\; Sea $h(-t)=2|-t| + 1 = 2t + 1 = h(t)$ entonces la función es par.\\\\

    Teoría y ejemplos\\\\

    %--------------------59.
    \item La variable $s$ es proporcional a $t$, y $s=25$ cuando $t=75$. Determine $t$ cuando $s=60.$\\\\
	Respuesta.-\; Sea $\dfrac{s}{r}$ entonces $\dfrac{25}{75}=\dfrac{60}{x} \quad \Rightarrow \quad \dfrac{1}{3} = \dfrac{60}{x} \quad \Rightarrow \quad x=180$\\\\ 

    %--------------------60.
    \item Energía cinética. La energía cinética $K$ de una masa es proprocional al cuadrado de su velocidad $v$. Si $K=12,960$ joules, cuando $v=18$ m/s, ¿Cuál es el valor de $K$ cuando $v=10$ m/s?.\\\\
	Respuesta.-\; Similar al anterior ejercicio se tiene $\dfrac{K}{v^2} = \dfrac{12,960}{18^2} = \dfrac{K}{10^2}$ entonces $K=4000$.\\\\

    %--------------------61.
    \item Las variables $r$ y $s$ son inversamente proporcional, mientras que $r=6$ cuando $s=4$. Determine $s$ cuando $r=10$.\\\\
	Respuesta.-\; Tenemos que $6\cdot 4 = s \cdot 10$ entonces queda que $s = 2.4$.\\\\

    %--------------------62.
    \item Ley de Boyle. La ley de Boyle establece que el volumen $V$ de un gas, a temperatura constante, aumenta cuando la presión $P$ disminuye, de manera que $V$ y $P$ son inversamente proporcionales. Si $P = 14.7 lb/in^2$ cuando $V = 1000 in^3$, entonces ¿cuál es el valor de $V$ cuando $P = 23.4 lbs/in^2$?.\\\\
	Respuesta.-\; Sea $V\cdot P = V^{'} \cdot P^{'}$ entonces $14.7 \cdot 1000 = V \cdot 23.4$ y por lo tanto $V=628.2 in^3$.\\\\

    %--------------------63.
    \item Una caja sin tapa se construye a partir de una pieza rectangular de cartón, cuyas dimensiones son $14$ por $22$ pulgadas (in). A la pieza de cartón se le cortan cuadrados de lado $x$ en cada esquina y luego se doblan hacia arriba los lados, como en la figura. Exprese el volumen $V$ de la caja como una función de $x$.\\\\
	Respuesta.-\; El volumen es dado por $V=L\cdot a\cdot h$ luego $h=x,\qquad a=14-2x, \qquad L=22-2x$ por lo tanto $$V(x)=(22-2x)(14-2x)\cdot x \quad \Longrightarrow \quad V(x)=4x^3 - 72x^2 + 308x$$\\\\

    %--------------------64.
    \item La siguiente figura muestra un rectángulo inscrito en un triángulo rectángulo isósceles, cuya hipotenusa tiene una longitud de dos unidades.
    \begin{enumerate}[\bfseries a.]
	
	%----------a.
	\item Exprese la coordenada de $P$ en términos de $x$. (Podría iniciar escribiendo una ecuación para la recta $AB$).\\\\
	    Respuesta.-\; Sea $y=mx+b$ luego en el punto $B$, se tiene la intersección de ambas rectas que forman un ángulo de $90^{\circ}$, así que, el ángulo que tiene que tener el punto $A$ es de $45^{\circ}$ o bien $m=-1$ por lo tanto $y=-x+b$ o bien m=-1 ya que la recta va hacia abajo.\\\\

	%----------b.
	\item Exprese el área del rectángulo en términos de $x$.\\\\
	    Respuesta.-\; El área de un rectángulo es $b\cdot a$ de donde $area = 2x \cdot y = 2x(b-x)$.\\\\

    \end{enumerate}

    En los ejercicios 65 y 66 relacione cada ecuación con su gráfica. No utilice un dispositivo para graficar y dé razones que justifiquen su respuesta.\\\\

    %--------------------65.
    \item  
    \begin{enumerate}[\bfseries a.]
	
	%----------a.
	\item $y=x^4 \quad \Rightarrow \quad h$\\\\

	%----------b.
	\item $y=x^7 \quad  \Rightarrow \quad f$\\\\

	%----------c.
	\item $y=x^10 \quad \Rightarrow \quad g$\\\\

    \end{enumerate}

    %--------------------66.
    \item
    \begin{enumerate}[\bfseries a.]
	
	%----------a.
	\item $y=5x \quad \Rightarrow \quad f$.\\\\

	%----------b.
	\item $y=5x \quad \Rightarrow \quad f$.\\\\

	%----------c.
	\item $y=x^5 \quad \Rightarrow \quad h$.\\\\

    \end{enumerate}

    %--------------------67.
    \item 
    \begin{enumerate}[\bfseries a.]

	%----------a.
	\item Grafique juntas las funciones $f(x)=x/2$ y $g(x)=1+(4/x)$ para identificar los valores de $x$ que satisfacen $$\dfrac{x}{2} > 1 + \dfrac{4}{x}$$\\
	\begin{center}
	    \begin{tikzpicture}[scale=1,draw opacity = 0.6]
		% abscisa y ordenada
		\tkzInit[xmax= 6,xmin=-4.5,ymax=3,ymin=-2]
		\tiny\tkzLabelXY[opacity=0.6,step=1, orig=false]
		% etiqueta x, f(x)
		\tkzDrawX[opacity=0.6,label=x,right=0.3]
		\tkzDrawY[opacity=0.6,label=f(x),below = -0.6]
		%dominio y función
		\draw [domain=-3:6,thick,gray] plot(\x,{\x/2});
		\draw [domain=2:6,thick,gray] plot(\x,{1+(4/\x)});
		\draw [domain=-4:-1.5,thick,gray] plot(\x,{1+(4/\x)});
	    \end{tikzpicture}
	\end{center}
	\vspace{.5cm}

	%----------b.
	\item Confirme algebraicamente los hallazgos del inciso $a)$\\\\
	    Respuesta.-\; Resolviendo la ecuación nos queda $x^2-2x-8>0$ donde se cumple para $x>4$ ó $x<-2$\\\\

    \end{enumerate}

    %--------------------68.
    \item 
    \begin{enumerate}[\bfseries a.]
	
	%----------a.
	\item Grafique juntas las funciones $f(x) = 3/(x-1)$ y $g(x)=2/(x+1)$ para identificar los valores de $x$ que satisfacen $$\dfrac{3}{x-1}<\dfrac{2}{x+1}$$\\
	\begin{center}
	    \begin{tikzpicture}[scale=1,draw opacity = 0.6]
		% abscisa y ordenada
		\tkzInit[xmax= 1,xmin=-12,ymax=1,ymin=-2]
		\tiny\tkzLabelXY[opacity=0.6,step=1, orig=false]
		% etiqueta x, f(x)
		\tkzDrawX[opacity=0.6,label=x,right=0.3]
		\tkzDrawY[opacity=0.6,label=f(x),below = -0.6]
		%dominio y función
		\draw [domain=-11:-3,thick,gray] plot(\x,{3/(\x-1)});
		\draw [domain=-11:-3,thick,gray] plot(\x,{2/(\x+1)});
	    \end{tikzpicture}
	\end{center}
	\vspace{.5cm}

	%----------b.
	\item Confirme algebraicamente los hallazgos del inicio $a)$.\\\\
	    Respuesta.-\; Sea $\dfrac{3}{x-1}<\dfrac{2}{x+1} \quad \Rightarrow \quad x<-5$\\\\

    \end{enumerate}

    %-------------------69.
    \item Para que una curva sea simétrica con respecto al eje $x$, el punto $(x, y)$ debe estar en la curva si y sólo si el punto $(x, -y)$ está en la curva. Explique por qué una curva que es simétrica con respecto al eje $x$ no es la gráfica de una función a menos que la función sea $y = 0$.\\\\
	Respuesta.-\; Esto se debe a que contradice a la definición de función. Es decir, a cada elemento $x$ se asigna un solo o único elemento $f(x)$. Si $y=0$ entonces $(x,y)=(x,-y)$ y por lo tanto se cumple la definición de función\\\\

    %--------------------70.
    \item Trescientos libros se venden en $\$ 40$ cada uno, lo que da por resultado un ingreso de $300\cdot \$ 40 $ = $\$12,000$. Por cada aumento de $\$ 5$ en el precio, se venden $25$ libros menos. Exprese el ingreso $R$ como una función del número $x$ de incrementos de $\$5$.\\\\
	Respuesta.-\; Veamos algunos ejemplos particulares:
	\begin{center}
	    \begin{tabular}{rcl}
		$300\cdot 40$&$=$&$12000$\\
		$(300 - 25)(40 + 5)$&$=$&$12375$\\
		$(300 - 50)(40 + 10)$&$=$&$12500$\\
		$(300 - 75)(40 + 15)$&$=$&$12375$\\
		$(300 - 100)(40 + 20)$&$=$&$12000$\\
	    \end{tabular}
	\end{center}
	Por lo tanto $R(x)=(300-5x)(40+x)=-125x^2 + 500x + 12000$\\\\

    %--------------------71.
    \item Se va a construir un corral con la forma de un triángulo rectángulo isósceles con catetos de longitud de $x$ pies (ft) e hipotenusa de longitud $h$ ft. Si los costos de la cerca son de $\$ 5 / ft$ para los catetos y $\$ lO/ft$ para la hipotenusa, escriba el costo total $C$ de la construcción como una función de $h$.\\\\
	Respuesta.-\; Sea $c^2+c^2=h^2 \quad \Rightarrow \quad h=c \sqrt{2} \quad \Rightarrow \quad c=h \sqrt{2}$, luego $C=2\cdot c \cdot 5 + h \cdot 10$ por lo tanto $C=10\cdot h \dfrac{1}{\sqrt{2}} + 1$\\\\

    %--------------------72.
    \item Costos industriales: Una central eléctrica se encuentra cerca de un río, donde éste tiene un ancho de 800 ft. Tender un cable de la planta a un lugar en la ciudad, 2 millas (mi) río abajo en el lado opuesto, tiene un costo de $180 por ft que cruce el río y $100 por ft en tierra a lo largo de la orilla del río.
    \begin{enumerate}[\bfseries a.]

	%----------a.
	\item Suponga que el cable va de la planta al punto $Q$, en el lado opuesto, lugar que se encuentra a $x$ ft del punto $P$, directamente opuesto a la planta. Escriba una función $C(x)$ que indique el costo de tender el cable en términos de la distancia $x$.\\\\
	    Respuesta.-\; Por el teorema de Pitágoras podemos establecer la función $C(x)$ como sigue: $$C(x)=\sqrt{x^2 + 800^2} \cdot 180 + (10560-x)\cdot 100$$

	%----------b.
	\item Genere una tabla de valores para determinar si la ubicación más barata para el punto $Q$ es menor a $2000$ ft o mayor a $2000$ ft del punto $P$.\\\\
	    Respuesta.-\;
	    \begin{center}
		\begin{tabular}{rclcl}
		    $C(x)$&$=$&$\sqrt{1900^2+800^2} + (10560-1900)\cdot 100$&$=$&$1270599.7$\\
		    $C(x)$&$=$&$\sqrt{2100^2+800^2} + (10560-2100)\cdot 100$&$=$&$1217079.5$\\\\
		\end{tabular}
	    \end{center}
	    Por lo tanto es mas barato ubicar el punto $Q$ a una distancia mayor a $2000$ ft.\\\\

    \end{enumerate}

\end{enumerate}

\section{Ejercicios}

En los ejercicios 1 y 2, determine dominios y rangos de $f,g,f+g \; y \; f\cdot g$\\\\
\begin{enumerate}[\Large \bfseries 1.]

%--------------------1.
\item $f(x)=x, \; g(x)=\sqrt{x-1}$\\\\
    Respuesta.-\; 
    \begin{center}
	\begin{tabular}{r l l}
	    Función&Dominio&Rango\\ 
	    \hline 
	    $f$ & $\forall x \in \mathbb{R}$ & $\forall f(x) \in \mathbb{R}$\\
	    $g$ & $x\geq 1$ & $f(x) \geq 0$\\
	    $f+g$ & $x\geq 1$ & $f(x)\geq 1$\\
	    $f\cdot g$ & $x\geq 1$ & $f(x) \geq 0$\\\\
	\end{tabular}
    \end{center}

%--------------------2.
\item $f(x)=\sqrt{x+1}, \; g(x)=\sqrt{x-1}$\\\\
    Respuesta.-\;
    \begin{center}
	\begin{tabular}{r l l}
	    Función&Dominio&Rango\\ 
	    \hline
	    $f$ & $x\geq -1$ & $f(x) \geq 0$\\
	    $g$ & $x\geq 1$ & $f(x) \geq 0$\\
	    $f+g$ & $x \geq 1$ & $f(x) \geq \sqrt{2}$\\
	    $f\cdot g$ & $x\geq 1$ & $f(x)\geq 0$\\\\
	\end{tabular}
    \end{center}

En los ejercicios 3 y 4, determine dominios y rangos de $f,g,f/g,g/f$.\\\\

%--------------------3.
\item $f(x)=2, \quad g(x)=x^2+1$\\\\
    Respuesta.-\; 
    \begin{center}
	\begin{tabular}{r l l}
	    Función&Dominio&Rango\\ 
	    \hline
	    $f$&$\forall\; x \in \mathbb{R}$&$f(x)=2$\\
	    $g$&$\forall \; x \in \mathbb{R}$&$f(x)\geq 1$\\
	    $f/g$&$\forall \; x \in \mathbb{R}$&$0<f(x)\leq 2$\\
	    $g/f$&$\forall \; x \in \mathbb{R}$&$f(x) \geq 0.5$\\\\
	\end{tabular}
    \end{center}

%--------------------4.
\item $f(x)=1, \quad g(x)=1 + \sqrt{x}$.\\\\
    Respuesta.-\;
    \begin{center}
	\begin{tabular}{r l l}
	    Función&Dominio&Rango\\ 
	    \hline
	    $f$&$\forall \; x \in \mathbb{R}$&$f(x)=1$\\
	    $g$&$x\geq 0$&$f(x)\geq 1$\\
	    $f/g$&$x\geq 0$&$0<f(x)\leq 1$\\
	    $g/f$&$x \geq 0$&$f(x)\geq 1$\\\\
	\end{tabular}
    \end{center}

Composición de funciones.\\\\

%--------------------5.
\item Si $f(x)=x+5$ y $g(x)=x^2-3,$ determine lo siguiente:\\\\
\begin{enumerate}[\bfseries a.]
    
    %----------a.
    \item $f(g(0)) = f(-3) = -3 + 5 = 2$\\\\

    %----------b.
    \item $g(f(0)) = g(5) = 5^2 - 3 = 22$\\\\

    %----------c.
    \item $f(g(x)) = f(x^2 - 3) = x^2 - 3 + 5 = x^2 + 2 \quad$ para $\quad D_{f\circ g} = \lbrace x \in D_g \; / \; g(x) \in D_f  \rbrace$\\\\

    %----------d.
    \item $g(f(x)) = g(x+5) = (x+5)^2 - 3 = x^2 + 10x + 25 - 3 = x^2 + 10x + 22 \quad$ para $\quad D_{g\circ f} = \lbrace x\in D_f / f(x) \in D_g \rbrace$\\\\

    %----------e.
    \item $f(f(-5)) = f(0) = 5$\\\\

    %----------f.
    \item $g(g(2)) = g(1) = 1 - 3 = -2$\\\\

    %----------g.
    \item $f(f(x)) = f(x+5) = x+ 5 + 5 = x + 10$\\\\

    %----------h.
    \item $g(g(x)) = g(x^2-3) = (x^2 - 3)^2 - 3 = x^4 - 6x^2 + 9 - 3 = x^4 - 6x^2 + 6$\\\\

\end{enumerate}

%--------------------6.
\item Si $f(x)=x-1$ y $g(x)=1/(x+1)$, determine lo siguiente.
\begin{enumerate}[\bfseries a.]
    
    %----------a.
    \item $f(g(1/2))= f(2/3) = \dfrac{2}{3} - 1 = -\dfrac{1}{3}$\\\\
    
    %----------b.
    \item $g(f(0)) = g(-1) = \dfrac{1}{-1+1} = indeterminado$\\\\
    
    %----------c.
    \item $f(g(x)) = f\left(\dfrac{1}{x+1}\right) = \dfrac{1}{x+1} - 1 = \dfrac{-x+2}{x+1} \quad $ para $\quad D_{f\circ g} = \lbrace \forall x \in D_g / g(x) \in D_f\rbrace$ \\\\
    
    %----------d.
    \item $g(f(x)) = g(x-1) = \dfrac{1}{x -1 + 1} = \dfrac{1}{x} \quad $ para $\quad D_{g\circ f} = \lbrace \forall\; x \in D_f / f(x) \in D_g \rbrace$\\\\
    
    %----------e.
    \item $f(f(-5)) = f(-6) = -6 - 1 = -7$\\\\
    
    %----------f.
    \item $g(g(2)) = g\left(\dfrac{1}{3}\right) = \dfrac{1}{\dfrac{1}{3} + 1} = \dfrac{3}{4}$\\\\
    
    %----------g.
    \item $f(f(x)) = f(x-1) = x - 2$\\\\
    
    %----------h.
    \item $g(g(x)) = g\left(\dfrac{1}{x+1}\right) = \dfrac{1}{\dfrac{1}{x+1} + 1} = \dfrac{x+1}{x+2} \quad x\neq -1, -2$\\\\

\end{enumerate}

En los ejercicios $7$ a $10$, escriba una fórmula para $f\circ g \circ h$\\\\

%--------------------7.
\item $f(x)=x+1, \quad g(x)=3x, \quad h(x)=4-x$\\\\
    Respuesta.-\; Se tiene $f(g(h(x))) = f(g(4-x)) = f(12-3x) = 12-3x + 1 = 13 - 3x \quad $ para $\quad D_{f\circ g\circ h} = \lbrace \forall \; x \in D_h / h(x) \in D_g \land g(h(x)) \in D_f\rbrace$\\\\

%--------------------8.
\item $f(x)=3x+4, \quad g(x)=2x-1, \quad h(x)=x^2$\\\\
    Respuesta.-\; Se tiene $f(g(h(x))) = f(g(x^2)) = f(2x^2-1) = 3(2x^2-1) + 4 = 6x^2 +1 $ para $D_{f\circ g\circ h}= \lbrace \forall x \in D_h / h(x) \in D_g \land g(h(x)) \in D_f \rbrace$\\\\

%--------------------9.
\item $f(x)=\sqrt{x+1},\quad \dfrac{1}{x+4}, \quad h(x)=\dfrac{1}{x}$\\\\ 
    Respuesta.-\; $f(g(h(x))) = f\left(g \left(\dfrac{1}{x}\right) \right) = f\left(\dfrac{1}{\dfrac{1}{x} + 4}\right) = \sqrt{\dfrac{1}{\dfrac{1}{x} + 4} + 1} = \sqrt{\dfrac{5x+1}{4x+1}}$ para $ x \neq 0, -\dfrac{1}{4}$.\\\\

%--------------------10.
\item $f(x)=\dfrac{x+2}{3-x}, \quad g(x)=\dfrac{x^2}{x^2+1}, \quad h(x)=\sqrt{2-x}$\\\\
    Respuesta.-\; $f\left( g \left( \sqrt{2-x}\right)\right) = f\left( \dfrac{|2-x|}{|2-x|+1}\right) = \dfrac{\dfrac{|2-x|}{|2-x|+1} + 2}{3 - \left(\dfrac{|2-x|}{|2-x|+1}\right)}$\\\\

Sean $f(x)=x-3, \quad g(x)=\sqrt{x}, \quad h(x)=x^3$ y $j(x)=2x$. Exprese cada una de las funciones de los ejercicios $11$ y $12$ como una composición de funciones que incluyan a una o más de $f,g,h$ y $j$.\\\\

%--------------------11.
\item
\begin{enumerate}[\bfseries a.]
    
    %----------a.
    \item $y=\sqrt{x}-3 \Rightarrow f(\sqrt{x}) \Rightarrow f(g(x))$\\\\

    %----------b.
    \item $y=2\sqrt{x} \Rightarrow j(\sqrt{x}) \Rightarrow j(g(x))$\\\\

    %----------c.
    \item $y=x^{1/4} \Rightarrow g(\sqrt{x}) \Rightarrow g(g(x))$\\\\

    %----------d.
    \item $y=4x \Rightarrow j(j(x))$\\\\

    %----------e.
    \item $y=\sqrt{(x-3)^3} \Rightarrow g((x-3)^3) \Rightarrow g(h(x-3)) \Rightarrow g(h(f(x)))$\\\\

    %----------f.
    \item $y=(2x-6)^3 \Rightarrow h(2x-6) = h(j(x-3)) = h(j(f(x))$.\\\\ 

\end{enumerate}

%--------------------12.
\item 
\begin{enumerate}[\bfseries a.]
    
    %----------a.
    \item $y=2x-3 \Rightarrow f(2x) \Rightarrow f(j(x))$\\\\
    
    %----------b.
    \item $y=x^{3/2} \Rightarrow g(x^3) \Rightarrow g(h(x))$\\\\
    
    %----------c.
    \item $y=x^9 \Rightarrow h(x^3) \Rightarrow h(h(x))$\\\\
    
    %----------d.
    \item $y=x-6 \Rightarrow f(f(x))$\\\\
    
    %----------e.
    \item $y=2\sqrt{x-3} \Rightarrow j(\sqrt{x-3}) \Rightarrow j(g(x-3)) \Rightarrow j(g(f(x)))$\\\\
    
    %----------f.
    \item $y=\sqrt{x^3-3} \Rightarrow g(x^3-3) \Rightarrow g(f(x^3)) \Rightarrow g(f(h(x)))$\\\\
    
\end{enumerate}

%--------------------13.
\item Copie complete la siguiente tabla:\\
\begin{center}
    \begin{tabular}{c c c c}
	     & $g(x)$ & $f(x)$ & $(f\circ g)(x)$\\\\
	\hline \\
	$a.$ & $x-7$ & $\sqrt{x}$ & $\sqrt{x-7}$\\\\
	$b.$ & $x+2$ & $2x$ & $3x+6$\\\\
	$c.$ & $x^2$ & $\sqrt{x-5}$ & $\sqrt{x^2-5}$\\\\
	$d.$ & $\dfrac{x}{x-1}$ & $\sqrt{x}{x-1}$ & $\dfrac{x}{(x-1)^2}$\\\\
	$e.$ & $x^3$ & $1+\dfrac{1}{x}$ & $1+\dfrac{1}{x^3}$\\\\
	$f.$ & $\dfrac{1}{x}$ & $\sqrt{x}$ & $\sqrt{\dfrac{1}{x}}$\\\\
    \end{tabular}
\end{center}

%--------------------14.
\item  Copie y complete la siguiente tabla.
\begin{center}
    \begin{tabular}{cccc}
	&$g(x)$&$f(x)$&$(f\circ g)(x)$\\\\
	\hline\\
	$a.$&$\dfrac{1}{x-1}$&$|x|$&$\left| \dfrac{1}{x-1}\right|$\\\\
	$b.$&$x+2$&$\dfrac{x-1}{x}$&$\dfrac{x}{x+1}$\\\\
	$c.$&$x^2$&$\sqrt{x}$&$|x|$\\\\
	$d.$&$\sqrt{x}$&$x^2$&$|x|$\\\\
    \end{tabular}
\end{center}

%--------------------15.
\item Evalúe cada expresión utilizando la siguiente tabla de valores.\\\\
\begin{enumerate}[\bfseries a.]

    %----------a.
    \item $f(g(-1)) = f(0) = -2$\\\\

    %----------b.
    \item $g(f(0)) = g(-2) = 2$\\\\ 

    %----------c.
    \item $f(f(-1)) = f(0) = -2$\\\\

    %----------d.
    \item $g(g(2)) = g(0) = 0$\\\\

    %----------e.
    \item $g(f(-2)) = g(1) = -1$\\\\

    %----------f.
    \item $f(g(1)) = f(-1) = 0$\\\\

\end{enumerate}

%--------------------16.
\item Evalúe cada expresión con el uso de las funciones.
$$f(x) = 2-x, \qquad g(x) = \left\{\begin{array}{lr} -x,& -2\leq x < 0\\ \\x-1, & 0\leq x \leq 2\\ \end{array}\right.$$\\
\begin{enumerate}[\bfseries a.]
    
    %----------a.
    \item $f(g(0)) = f(0-1) = 2 + 1 = 3$\\\\
    
    %----------b.
    \item $g(f(0)) = g(2) = 1$\\\\
    
    %----------c.
    \item $g(g(-1)) = g(1) = 0$\\\\
    
    %----------d.
    \item $f(f(2)) = f(0) = 2$\\\\
    
    %----------e.
    \item $g(f(0)) = g(2) = 1$\\\\
    
    %----------f.
    \item $f(g(1/2)) = f(-1/2) = 5/2$\\\\
\end{enumerate}

En los ejercicios 17 y 18, (a) escriba fórmulas para $f\circ g$ y $g\circ f$, luego determine (b) el dominio y (c) el rango de cada una.\\\\

%--------------------17.
\item $f(x) =\sqrt{x+1}, \quad g(x)=\dfrac{1}{x}$\\\\
    Respuesta.-\; 
    Para $f\circ g = f(g(x)) = f\left(\dfrac{1}{x}\right) = \sqrt{\dfrac{1}{x}+1}$ de donde el dominio viene dado por $x>0 \; \land \; x\leq -1$, y el rango viene dado por $f(x)\geq 0$. Luego para $g\circ f = g(f(x)) = g(\sqrt{x+1}) = \dfrac{1}{\sqrt{x+1}}$ de donde el dominio es $x>-1$ y el rango $\forall \; \mathbb{R}, x\neq 0$.\\\\ 

%--------------------18.
\item $f(x) = x^2, \quad g(x) = 1 - \sqrt{x}$\\\\
    Respuesta.-\; Para $f(1-\sqrt{x}) = (1-\sqrt{x})^2 = 1 - 2 \sqrt{x} + |x|$ el dominio viene dado por $x\geq 0$ y el rango por $f(g(x)) \geq 0$.\\
    Luego para $g(x^2) = 1 - \sqrt{x^2} = 1 - |x|$ el dominio viene dado por $x\geq 0$ y el rango $g(f(x)) \geq 1$.\\\\

%-------------------19.
\item Sea $f(x) = \dfrac{x}{x-2}$. Determine una función $y=g(x)$ de modo que $(f\circ g)(x) = x$.\\\\
    Respuesta.-\; Sea $f(g(x))=x$ entonces $\dfrac{g(x)}{g(x)-2} = x$ ó $\dfrac{y}{y-2}=x$ por lo tanto $y=g(x)=\dfrac{-2x}{1-x}$.\\\\ 

%--------------------20.
\item Sea $f(x)=2x^3 - 4$. Determine una función $y=g(x)$ de modo que $(f\circ g)(x)=x+2$.\\\\
    Respuesta.-\; Similar al anterior ejercicio tenemos que $2y^3 - 4 = x+2$ de donde nos queda $$y=g(x)=\sqrt[3]{\dfrac{x+6}{2}}$$\\

Traslación de gráficas.\\\\

%--------------------21.
\item La siguiente figura muestra la gráfica de $y=-x^2$ desplazada a dos posiciones nuevas. Escriba las ecuaciones de las gráficas nuevas.\\\\
    Respuesta.-\; 
    \begin{enumerate}[a)]
	\item $y=-(x+7)^2$\\
	\item $y=-(x-4)^2$\\\\
    \end{enumerate}

%-------------------22.
\item La siguiente figura muestra la gráfica de $y=x^2$ desplazada a dos posiciones nuevas. Escriba las ecuaciones de las gráficas nuevas.\\\\
    Respuesta.-\; 
    \begin{enumerate}[a)]
	\item $y=x^2 + 3$\\
	\item $x^2 - 5$\\\\
    \end{enumerate}

%--------------------23.
\item Relacione las ecuaciones listadas en los incisos $a)$ a $d)$ con las gráficas de la figura.\\\\
    Respuesta.-\;
    \begin{enumerate}[\bfseries a)]

	%----------a)
	\item $y=(x-1)^2 - 4 = $ Posición 4.\\\\

	%----------b)
	\item $y=(x-2)^2 + 2 = $ Posición 1.\\\\

	%----------c)
	\item $y=(x+2)^2 + 2 = $ Posición 2.\\\\

	%----------d)
	\item $y=(x+3)^2 - 2 = $ Posición 3.\\\\
    \end{enumerate}

%--------------------24.
\item La siguiente figura muestra la gráfica de $y=-x^2$ desplazada a cuatro posiciones nuevas. Escriba una ecuación para cada nueva gráfica.\\\\
    Respuesta.-\; 
    \begin{enumerate}[a)]
	\item $y=-(x-1)^2 + 4$\\
	\item $y=-(x+2)^2 + 3$\\
	\item $y=-(x+4)^2 - 1$\\
	\item $y=-(x-2)^2$\\\\
    \end{enumerate}

En los ejercicios $25$ a $34$ se establece cuántas unidades y en qué direcciones se trasladarán las gráficas de las ecuaciones dadas. Proporcione una ecuación para la gráfica desplazada: después, en el mismo plano cartesiano, trace la gráfica de la función original y la gráfica de la ecuación desplazada, anotando junto a cada gráfica la ecuación que le corresponda.\\\\

%--------------------25.
\item $x^2+y^2 = 49$ abajo $3$, izquierda $2$.\\\\ 
    Respuesta.-\; $$y=\sqrt{49-(x+2)^2} - 3 \quad \Rightarrow \quad (x+2)^2 + (y+3)^2 = 49$$
	\begin{center}
	    \begin{tikzpicture}[scale=.6, draw opacity = 0.6]
		% abscisa y ordenada
		\tkzInit[xmax= 7,xmin=-7,ymax=7,ymin=-1]
		\tiny\tkzLabelXY[opacity=0.6,step=1, orig=false]
		% etiqueta x, f(x)
		\tkzDrawX[opacity=0.6,label=x,right=0.3]
		\tkzDrawY[opacity=0.6,label=f(x),below = -0.6]
		%dominio y función
		\draw [domain=-8:4,thick,gray] plot(\x,{(49-(\x+2)^2)^(1/2) - 3});
		\draw [domain=-6:6,thick,gray] plot(\x,{(49-\x*\x)^(1/2)});
	    \end{tikzpicture}
	\end{center}
	\vspace{.5cm}

%--------------------26.
\item $x^2 + y^2 = 25$ Arriba $3$, izquierda $4$\\\\
    Respuesta.-\; $$(x+4)^2 + (y-3)^2 = 25$$
	\begin{center}
	    \begin{tikzpicture}[scale=.6, draw opacity = 0.6]
		% abscisa y ordenada
		\tkzInit[xmax= 5,xmin=-8,ymax=8,ymin=-1]
		\tiny\tkzLabelXY[opacity=0.6,step=1, orig=false]
		% etiqueta x, f(x)
		\tkzDrawX[opacity=0.6,label=x,right=0.3]
		\tkzDrawY[opacity=0.6,label=f(x),below = -0.6]
		%dominio y función
		\draw [domain=-8:0,thick,gray] plot(\x,{(25-(\x+4)^2)^(1/2) + 3});
		\draw [domain=-4:4,thick,gray] plot(\x,{(25-\x*\x)^(1/2)});
	    \end{tikzpicture}
	\end{center}
	\vspace{.5cm}

%--------------------27.
\item $y=x^3$ Izquierda $1$, abajo $1$.\\\\
    Respuesta.-\; $$y=(x+1)^3 - 1$$
	\begin{center}
	    \begin{tikzpicture}[scale=.6, draw opacity = 0.6]
		% abscisa y ordenada
		\tkzInit[xmax= 3,xmin=-3,ymax=8,ymin=-8]
		\tiny\tkzLabelXY[opacity=0.6,step=1, orig=false]
		% etiqueta x, f(x)
		\tkzDrawX[opacity=0.6,label=x,right=0.3]
		\tkzDrawY[opacity=0.6,label=f(x),below = -0.6]
		%dominio y función
		\draw [domain=-2:2,thick,gray] plot(\x,{\x^3});
		\draw [domain=-2.9:1,thick,gray] plot(\x,{(\x+1)^3 - 1});
	    \end{tikzpicture}
	\end{center}
	\vspace{.5cm}

%--------------------28.
\item $y = x^{2/3}$ Derecha 1, abajo 1\\\\
    Respuesta.-\; $$y=(x-1)^{2/3}  - 1$$
	\begin{center}
	    \begin{tikzpicture}[scale=.6, draw opacity = 0.6]
		% abscisa y ordenada
		\tkzInit[xmax= 5,xmin=-3,ymax=2,ymin=-2]
		\tiny\tkzLabelXY[opacity=0.6,step=1, orig=false]
		% etiqueta x, f(x)
		\tkzDrawX[opacity=0.6,label=x,right=0.3]
		\tkzDrawY[opacity=0.6,label=f(x),below = -0.6]
		%dominio y función
		\draw [domain=-2:2,thick,gray] plot(\x,{\x^(2/3)});
		\draw [domain=1:4,thick,gray] plot(\x,{(\x-1)^(2/3) - 1});
	    \end{tikzpicture}
	\end{center}
	\vspace{.5cm}

%--------------------29
\item $y = \sqrt{x}$ Izquierda 0.81\\\\
    Respuesta.-\; $$\sqrt{x+0.81}$$
	\begin{center}
	    \begin{tikzpicture}[scale=.6, draw opacity = 0.6]
		% abscisa y ordenada
		\tkzInit[xmax= 6,xmin=-2,ymax=2,ymin=-1]
		\tiny\tkzLabelXY[opacity=0.6,step=1, orig=false]
		% etiqueta x, f(x)
		\tkzDrawX[opacity=0.6,label=x,right=0.3]
		\tkzDrawY[opacity=0.6,label=f(x),below = -0.6]
		%dominio y función
		\draw [domain=0:6, thick,gray] plot(\x,{\x^(1/2)});
		\draw [domain=-.8:5,thick,gray] plot(\x,{(\x+0.81)^(1/2)});
	    \end{tikzpicture}
	\end{center}
	\vspace{.5cm}

%--------------------30.
\item $y = - \sqrt{x}$ Derecha 3 \\\\ 
    Respuesta.-\; $$y=-\sqrt{x-3}$$
	\begin{center}
	    \begin{tikzpicture}[scale=.6, draw opacity = 0.6]
		% abscisa y ordenada
		\tkzInit[xmax= 7,xmin=-1,ymax=1,ymin=-3]
		\tiny\tkzLabelXY[opacity=0.6,step=1, orig=false]
		% etiqueta x, f(x)
		\tkzDrawX[opacity=0.6,label=x,right=0.3]
		\tkzDrawY[opacity=0.6,label=f(x),below = -0.6]
		%dominio y función
		\draw [domain=0:4,thick,gray] plot(\x,{-1*(\x)^(1/2)});
		\draw [domain=3:6,thick,gray] plot(\x,{-1*(\x-3)^(1/2)});
	    \end{tikzpicture}
	\end{center}
	\vspace{.5cm}

%--------------------31.
\item  $y = 2x - 7$ Arriba 7 \\\\ 
    Respuesta.-\; $$y=2x$$
	\begin{center}
	    \begin{tikzpicture}[scale=.6, draw opacity = 0.6]
		% abscisa y ordenada
		\tkzInit[xmax= 4,xmin=-3,ymax=6,ymin=-9]
		\tiny\tkzLabelXY[opacity=0.6,step=1, orig=false]
		% etiqueta x, f(x)
		\tkzDrawX[opacity=0.6,label=x,right=0.3]
		\tkzDrawY[opacity=0.6,label=f(x),below = -0.6]
		%dominio y función
		\draw [domain=-1:3,thick,gray] plot(\x,{2*\x - 7});
		\draw [domain=-1:3,thick,gray] plot(\x,{2*\x});
	    \end{tikzpicture}
	\end{center}
	\vspace{.5cm}

%--------------------32.
\item $y = \dfrac{1}{2} (x + 1) + 5$ Abajo 5, derecha 1 \\\\ 
    Respuesta.-\; $$y=\dfrac{1}{2}x$$
	\begin{center}
	    \begin{tikzpicture}[scale=.6, draw opacity = 0.6]
		% abscisa y ordenada
		\tkzInit[xmax= 5,xmin=-6,ymax=7,ymin=-2]
		\tiny\tkzLabelXY[opacity=0.6,step=1, orig=false]
		% etiqueta x, f(x)
		\tkzDrawX[opacity=0.6,label=x,right=0.3]
		\tkzDrawY[opacity=0.6,label=f(x),below = -0.6]
		%dominio y función
		\draw [domain=-5:3,thick,gray] plot(\x,{(1/2)*(\x + 1) + 5});
		\draw [domain=-4:4,thick,gray] plot(\x,{(1/2)*\x});
	    \end{tikzpicture}
	\end{center}
	\vspace{.5cm}

%--------------------33.
\item $y=\dfrac{1}{x}$ Arriba 1, derecha 1\\\\ 
    Respuesta.-\; $$y=\dfrac{1}{x-1}  + 1$$
	\begin{center}
	    \begin{tikzpicture}[scale=.6, draw opacity = 0.6]
		% abscisa y ordenada
		\tkzInit[xmax= 7,xmin=-1,ymax=4,ymin=-1]
		\tiny\tkzLabelXY[opacity=0.6,step=1, orig=false]
		% etiqueta x, f(x)
		\tkzDrawX[opacity=0.6,label=x,right=0.3]
		\tkzDrawY[opacity=0.6,label=f(x),below = -0.6]
		%dominio y función
		\draw [domain=.4:6,thick,gray] plot(\x,{1/\x});
		\draw [domain=1.4:7,thick,gray] plot(\x,{1/(\x-1) + 1});
	    \end{tikzpicture}
	\end{center}
	\vspace{.5cm}

%--------------------34.
\item $y=\dfrac{1}{x^2}$ Izquierda 2, abajo 1 \\\\ 
    Respuesta.-\; $$y=\dfrac{1}{(x+2)^2} - 1$$
	\begin{center}
	    \begin{tikzpicture}[scale=.6, draw opacity = 0.6]
		% abscisa y ordenada
		\tkzInit[xmax= 8,xmin=-2,ymax=2,ymin=-1]
		\tiny\tkzLabelXY[opacity=0.6,step=1, orig=false]
		% etiqueta x, f(x)
		\tkzDrawX[opacity=0.6,label=x,right=0.3]
		\tkzDrawY[opacity=0.6,label=f(x),below = -0.6]
		%dominio y función
		\draw [domain=.8:6,thick,gray] plot(\x,{1/\x^2});
		\draw [domain=-1.3:5,thick,gray] plot(\x,{(1/(\x+2)^2) - 1});
	    \end{tikzpicture}
	\end{center}
	\vspace{.5cm}

Grafique las funciones de los ejercicios $35$ a $54$\\\\

%--------------------35.
\item $y=\sqrt{x+4}$\\\\
    Respuesta.-\;
	\begin{center}
	    \begin{tikzpicture}[scale=.6, draw opacity = 0.6]
		% abscisa y ordenada
		\tkzInit[xmax= 5,xmin=-4,ymax=3,ymin=-1]
		\tiny\tkzLabelXY[opacity=0.6,step=1, orig=false]
		% etiqueta x, f(x)
		\tkzDrawX[opacity=0.6,label=x,right=0.3]
		\tkzDrawY[opacity=0.6,label=f(x),below = -0.6]
		%dominio y función
		\draw [domain=-4:4,thick,gray] plot(\x,{(\x+4)^(1/2)});
	    \end{tikzpicture}
	\end{center}
	\vspace{.5cm}

%--------------------36.
\item $y=\sqrt{9-x}$\\\\
    Respuesta.-\;
	\begin{center}
	    \begin{tikzpicture}[scale=.6, draw opacity = 0.6]
		% abscisa y ordenada
		\tkzInit[xmax= 9,xmin=-8,ymax=4,ymin=-1]
		\tiny\tkzLabelXY[opacity=0.6,step=1, orig=false]
		% etiqueta x, f(x)
		\tkzDrawX[opacity=0.6,label=x,right=0.3]
		\tkzDrawY[opacity=0.6,label=f(x),below = -0.6]
		%dominio y función
		\draw [domain=-8:9,thick,gray] plot(\x,{(9-\x)^(1/2)});
	    \end{tikzpicture}
	\end{center}
	\vspace{.5cm}

%--------------------37.
\item $y=|x-2|$\\\\
    Respuesta.-\;
	\begin{center}
	    \begin{tikzpicture}[scale=.6, draw opacity = 0.6]
		% abscisa y ordenada
		\tkzInit[xmax= 6,xmin=-3,ymax=5,ymin=-1]
		\tiny\tkzLabelXY[opacity=0.6,step=1, orig=false]
		% etiqueta x, f(x)
		\tkzDrawX[opacity=0.6,label=x,right=0.3]
		\tkzDrawY[opacity=0.6,label=f(x),below = -0.6]
		%dominio y función
		\draw [domain=-2:6,thick,gray] plot(\x,{abs(\x-2)});
	    \end{tikzpicture}
	\end{center}
	\vspace{.5cm}
    
%--------------------38.
\item $y=|1-x|-1$\\\\
    Respuesta.-\;
	\begin{center}
	    \begin{tikzpicture}[scale=.6, draw opacity = 0.6]
		% abscisa y ordenada
		\tkzInit[xmax= 7,xmin=-4,ymax=5,ymin=-1]
		\tiny\tkzLabelXY[opacity=0.6,step=1, orig=false]
		% etiqueta x, f(x)
		\tkzDrawX[opacity=0.6,label=x,right=0.3]
		\tkzDrawY[opacity=0.6,label=f(x),below = -0.6]
		%dominio y función
		\draw [domain=-4:6,thick,gray] plot(\x,{abs(1-\x)-1});
	    \end{tikzpicture}
	\end{center}
	\vspace{.5cm}

%--------------------39.
\item $y=1+\sqrt{x-1}$\\\\
    Respuesta.-\;
	\begin{center}
	    \begin{tikzpicture}[scale=.6, draw opacity = 0.6]
		% abscisa y ordenada
		\tkzInit[xmax= 7,xmin=-1,ymax=3,ymin=-1]
		\tiny\tkzLabelXY[opacity=0.6,step=1, orig=false]
		% etiqueta x, f(x)
		\tkzDrawX[opacity=0.6,label=x,right=0.3]
		\tkzDrawY[opacity=0.6,label=f(x),below = -0.6]
		%dominio y función
		\draw [domain=1:6,thick,gray] plot(\x,{1 + (\x-1)^(1/2)});
	    \end{tikzpicture}
	\end{center}
	\vspace{.5cm}

%--------------------40.
    \item $y=1 - \sqrt{x}$\\\\
    Respuesta.-\;
	\begin{center}
	    \begin{tikzpicture}[scale=.6, draw opacity = 0.6]
		% abscisa y ordenada
		\tkzInit[xmax= 7,xmin=-1,ymax=1,ymin=-2]
		\tiny\tkzLabelXY[opacity=0.6,step=1, orig=false]
		% etiqueta x, f(x)
		\tkzDrawX[opacity=0.6,label=x,right=0.3]
		\tkzDrawY[opacity=0.6,label=f(x),below = -0.6]
		%dominio y función
		\draw [domain=0:6,thick,gray] plot(\x,{1-\x^(1/2)});
	    \end{tikzpicture}
	\end{center}
	\vspace{.5cm}

%--------------------41.
    \item $y=(x+1)^{2/3}$\\\\
    Respuesta.-\;
	\begin{center}
	    \begin{tikzpicture}[scale=.6, draw opacity = 0.6]
		% abscisa y ordenada
		\tkzInit[xmax=6,xmin=-2,ymax=3,ymin=-1]
		\tiny\tkzLabelXY[opacity=0.6,step=1, orig=false]
		% etiqueta x, f(x)
		\tkzDrawX[opacity=0.6,label=x,right=0.3]
		\tkzDrawY[opacity=0.6,label=f(x),below = -0.6]
		%dominio y función
		\draw [domain=-1:5,thick,gray] plot(\x,{(\x+1)^(2/3)});
	    \end{tikzpicture}
	\end{center}
	\vspace{.5cm}

%--------------------42.
    \item $y=(x-8)^{2/3}$\\\\
    Respuesta.-\;
	\begin{center}
	    \begin{tikzpicture}[scale=.6, draw opacity = 0.6]
		% abscisa y ordenada
		\tkzInit[xmax= 12,xmin=0,ymax=2,ymin=-1]
		\tiny\tkzLabelXY[opacity=0.6,step=1, orig=false]
		% etiqueta x, f(x)
		\tkzDrawX[opacity=0.6,label=x,right=0.3]
		\tkzDrawY[opacity=0.6,label=f(x),below = -0.6]
		%dominio y función
		\draw [domain=8:12,thick,gray] plot(\x,{(\x-8)^(2/3)});
	    \end{tikzpicture}
	\end{center}
	\vspace{.5cm}

%--------------------43.
\item $y=1-x^{2/3}$\\\\
    Respuesta.-\;
	\begin{center}
	    \begin{tikzpicture}[scale=.6, draw opacity = 0.6]
		% abscisa y ordenada
		\tkzInit[xmax= 7,xmin=-1,ymax=2,ymin=-1]
		\tiny\tkzLabelXY[opacity=0.6,step=1, orig=false]
		% etiqueta x, f(x)
		\tkzDrawX[opacity=0.6,label=x,right=0.3]
		\tkzDrawY[opacity=0.6,label=f(x),below = -0.6]
		%dominio y función
		\draw [domain=0:6,thick,gray] plot(\x,{1 - \x^(1/3)});
	    \end{tikzpicture}
	\end{center}
	\vspace{.5cm}

%--------------------44.
\item $y+4=x^{2/3}$\\\\
    Respuesta.-\;
	\begin{center}
	    \begin{tikzpicture}[scale=.6, draw opacity = 0.6]
		% abscisa y ordenada
		\tkzInit[xmax= 8,xmin=-2,ymax=1,ymin=-4]
		\tiny\tkzLabelXY[opacity=0.6,step=1, orig=false]
		% etiqueta x, f(x)
		\tkzDrawX[opacity=0.6,label=x,right=0.3]
		\tkzDrawY[opacity=0.6,label=f(x),below = -0.6]
		%dominio y función
		\draw [domain=0:6,thick,gray] plot(\x,{\x^(2/3) - 4});
	    \end{tikzpicture}
	\end{center}
	\vspace{.5cm}

%--------------------45.
\item $y=\sqrt[3]{x-1} - 1$\\\\
    Respuesta.-\;
	\begin{center}
	    \begin{tikzpicture}[scale=.6, draw opacity = 0.6]
		% abscisa y ordenada
		\tkzInit[xmax= 8,xmin=-2,ymax=2,ymin=-1]
		\tiny\tkzLabelXY[opacity=0.6,step=1, orig=false]
		% etiqueta x, f(x)
		\tkzDrawX[opacity=0.6,label=x,right=0.3]
		\tkzDrawY[opacity=0.6,label=f(x),below = -0.6]
		%dominio y función
		\draw [domain=1:6,thick,gray] plot(\x,{(\x-1)^(1/3)-1});
	    \end{tikzpicture}
	\end{center}
	\vspace{.5cm}

%--------------------46.
\item $y=(x+2)^{3/2} + 1$\\\\
    Respuesta.-\;
	\begin{center}
	    \begin{tikzpicture}[scale=.6, draw opacity = 0.6]
		% abscisa y ordenada
		\tkzInit[xmax= 4,xmin=-3,ymax=8,ymin=-1]
		\tiny\tkzLabelXY[opacity=0.6,step=1, orig=false]
		% etiqueta x, f(x)
		\tkzDrawX[opacity=0.6,label=x,right=0.3]
		\tkzDrawY[opacity=0.6,label=f(x),below = -0.6]
		%dominio y función
		\draw [domain=-2:2,thick,gray] plot(\x,{(\x+2)^(3/2) + 1});
	    \end{tikzpicture}
	\end{center}
	\vspace{.5cm}

%--------------------47.
\item $y=\dfrac{1}{x-2}$\\\\
    Respuesta.-\;
	\begin{center}
	    \begin{tikzpicture}[scale=.6, draw opacity = 0.6]
		% abscisa y ordenada
		\tkzInit[xmax= 8,xmin=-2,ymax=13,ymin=-6]
		\tiny\tkzLabelXY[opacity=0.6,step=1, orig=false]
		% etiqueta x, f(x)
		\tkzDrawX[opacity=0.6,label=x,right=0.3]
		\tkzDrawY[opacity=0.6,label=f(x),below = -0.6]
		%dominio y función
		\draw [domain=0.1:6,thick,gray] plot(\x,{1/(\x-2)});
	    \end{tikzpicture}
	\end{center}
	\vspace{.5cm}

%--------------------48.
\item $y=\dfrac{1}{x}-2$\\\\
    Respuesta.-\;
	\begin{center}
	    \begin{tikzpicture}[scale=.6, draw opacity = 0.6]
		% abscisa y ordenada
		\tkzInit[xmax= 8,xmin=-2,ymax=7,ymin=-2]
		\tiny\tkzLabelXY[opacity=0.6,step=1, orig=false]
		% etiqueta x, f(x)
		\tkzDrawX[opacity=0.6,label=x,right=0.3]
		\tkzDrawY[opacity=0.6,label=f(x),below = -0.6]
		%dominio y función
		\draw [domain=.1:6,thick,gray] plot(\x,{1/\x - 2});
	    \end{tikzpicture}
	\end{center}
	\vspace{.5cm}

%--------------------49.
\item $y=\dfrac{1}{x}+2$\\\\
    Respuesta.-\;
	\begin{center}
	    \begin{tikzpicture}[scale=.6, draw opacity = 0.6]
		% abscisa y ordenada
		\tkzInit[xmax= 8,xmin=-2,ymax=6,ymin=-1]
		\tiny\tkzLabelXY[opacity=0.6,step=1, orig=false]
		% etiqueta x, f(x)
		\tkzDrawX[opacity=0.6,label=x,right=0.3]
		\tkzDrawY[opacity=0.6,label=f(x),below = -0.6]
		%dominio y función
		\draw [domain=.2:6,thick,gray] plot(\x,{1/\x + 2});
	    \end{tikzpicture}
	\end{center}
	\vspace{.5cm}

%--------------------50.
\item $y=\dfrac{1}{x+2}$\\\\
    Respuesta.-\;
	\begin{center}
	    \begin{tikzpicture}[scale=.6, draw opacity = 0.6]
		% abscisa y ordenada
		\tkzInit[xmax= 7,xmin=-2,ymax=2,ymin=-1]
		\tiny\tkzLabelXY[opacity=0.6,step=1, orig=false]
		% etiqueta x, f(x)
		\tkzDrawX[opacity=0.6,label=x,right=0.3]
		\tkzDrawY[opacity=0.6,label=f(x),below = -0.6]
		%dominio y función
		\draw [domain=-1.6:5,thick,gray] plot(\x,{1/(\x+2)});
	    \end{tikzpicture}
	\end{center}
	\vspace{.5cm}

%--------------------51.
\item $y=\dfrac{1}{(x-1)^2}$\\\\
    Respuesta.-\;
	\begin{center}
	    \begin{tikzpicture}[scale=.6, draw opacity = 0.6]
		% abscisa y ordenada
		\tkzInit[xmax= 8,xmin=-2,ymax=3,ymin=-1]
		\tiny\tkzLabelXY[opacity=0.6,step=1, orig=false]
		% etiqueta x, f(x)
		\tkzDrawX[opacity=0.6,label=x,right=0.3]
		\tkzDrawY[opacity=0.6,label=f(x),below = -0.6]
		%dominio y función
		\draw [domain=1.5:8,thick,gray] plot(\x,{1/(\x-1)^2});
	    \end{tikzpicture}
	\end{center}
	\vspace{.5cm}

%--------------------52.
\item $y=\dfrac{1}{x^2} - 1$\\\\
    Respuesta.-\;
	\begin{center}
	    \begin{tikzpicture}[scale=.6, draw opacity = 0.6]
		% abscisa y ordenada
		\tkzInit[xmax= 8,xmin=-2,ymax=2,ymin=-1]
		\tiny\tkzLabelXY[opacity=0.6,step=1, orig=false]
		% etiqueta x, f(x)
		\tkzDrawX[opacity=0.6,label=x,right=0.3]
		\tkzDrawY[opacity=0.6,label=f(x),below = -0.6]
		%dominio y función
		\draw [domain=.8:6,thick,gray] plot(\x,{1/\x^2 - 1});
	    \end{tikzpicture}
	\end{center}
	\vspace{.5cm}

%--------------------53.
\item $y=\dfrac{1}{x^2}+1$\\\\
    Respuesta.-\;
	\begin{center}
	    \begin{tikzpicture}[scale=.6, draw opacity = 0.6]
		% abscisa y ordenada
		\tkzInit[xmax= 8,xmin=-2,ymax=2,ymin=-1]
		\tiny\tkzLabelXY[opacity=0.6,step=1, orig=false]
		% etiqueta x, f(x)
		\tkzDrawX[opacity=0.6,label=x,right=0.3]
		\tkzDrawY[opacity=0.6,label=f(x),below = -0.6]
		%dominio y función
		\draw [domain=.8:6,thick,gray] plot(\x,{1/\x^2 + 1});
	    \end{tikzpicture}
	\end{center}
	\vspace{.5cm}

%--------------------54.
\item $y=\dfrac{1}{(x+1)^2}$\\\\
    Respuesta.-\;
	\begin{center}
	    \begin{tikzpicture}[scale=.6, draw opacity = 0.6]
		% abscisa y ordenada
		\tkzInit[xmax= 8,xmin=-2,ymax=2,ymin=-1]
		\tiny\tkzLabelXY[opacity=0.6,step=1, orig=false]
		% etiqueta x, f(x)
		\tkzDrawX[opacity=0.6,label=x,right=0.3]
		\tkzDrawY[opacity=0.6,label=f(x),below = -0.6]
		%dominio y función
		\draw [domain=.1:7,thick,gray] plot(\x,{1/(\x+1)^2});
	    \end{tikzpicture}
	\end{center}
	\vspace{.5cm}

%--------------------55.
\item La siguiente figura muestra la gráfica de una función $f(x)$ con dominio $[0, 2]$ y rango $[0, 1]$. Determine los dominios y los rangos de las siguientes funciones, y trace sus gráficas.
    \begin{enumerate}[\bfseries a)]

	%----------a)
	\item $f(x)+2$\\\\
	    Respuesta.-\; El dominio viene dado por $[0,2]$ y el rango por $[2,3]$\\\\

	%----------b)
	\item $f(x)-1$\\\\
	    Respuesta.-\; El dominio es $[0,2]$ y el rango $[-1,0]$.\\\\ 

	%----------c)
	\item $2f(x)$\\\\
	    Respuesta.-\; El dominio es $[0,2]$ y el rango $[0,2]$.\\\\ 

	%----------d)
	\item $-f(x)$\\\\
	    Respuesta.-\; El dominio es $[0,2]$ y el rango $[-1,0]$.\\\\ 

	%----------e)
	\item $f(x+2)$\\\\
	    Respuesta.-\; El dominio es $[-2,0]$ y el rango $[0,1]$.\\\\ 

	%----------f)
	\item $f(x-1)$\\\\
	    Respuesta.-\; El dominio es $[1,3]$ y el rango $[0,1]$.\\\\ 

	%----------g)
	\item $f(-x)$\\\\
	    Respuesta.-\; El dominio es $[-2,0]$ y el rango $[0,1]$.\\\\ 

	%----------h)
	\item $-f(x+1)+1$\\\\
	    Respuesta.-\; El dominio es $[-1,1]$ y el rango $[0,1]$.\\\\ 

    \end{enumerate}

%--------------------56.
\item La siguiente figura muestra la gráfica de una función $g(t)$ con dominio $[-4, 0]$ y rango $[-3, 0]$. Determine los dominios y los rangos de las siguientes funciones, y trace sus gráficas.
    \begin{enumerate}[\bfseries a)]

	%----------a)
	\item $g(-t)$\\\\
	    Respuesta.-\; $D:[0,4]$ ; $R:[-3,0]$\\\\

	%----------b)
	\item $-g(t)$\\\\
	    Respuesta.-\; $D:[-4,0]$ ; $R:[0,3]$\\\\

	%----------c)
	\item $g(t)+3$\\\\
	    Respuesta.-\; $D:[-4,0]$ ; $R:[0,3]$\\\\

	%----------d)
	\item $1-g(t)$\\\\
	    Respuesta.-\; $D:[-4,0]$ ; $R:[1,4]$\\\\

	%----------e)
	\item $g(-t+2)$\\\\
	    Respuesta.-\; $D:[-2,2]$ ; $R:[-3,0]$\\\\

	%----------f)
	\item $g(t-2)$\\\\
	    Respuesta.-\; $D:[-2,2]$ ; $R:[-3,0]$\\\\

	%----------g)
	\item $g(1-t)$\\\\
	    Respuesta.-\; $D:[-1,3]$ ; $R:[-3,0]$\\\\

	%----------h)
	\item $-g(t-4)$\\\\
	    Respuesta.-\; $D:[0,4]$ ; $R:[0,3]$\\\\

    \end{enumerate}

Cambio de escala vertical y horizontal\\\\
En los ejercicios $57$ a $66$ se indica por qué factor y en qué dirección se estirarán o comprimirán las gráficas de las funciones dadas. Proporcione una ecuación para cada gráfica estirada o comprimida.\\\\

%--------------------57.
\item $y=x^2-1$ estirada verticalmente por un factor de $3$\\\\
    Respuesta.-\; $y=3(x^2-1)$\\\\

%--------------------58.
\item $y=x^2-1$, comprimida horizontalmente por un factor de $2$ \\\\
    Respuesta.-\; $y=\dfrac{x^2-1}{2}$ \\\\

%--------------------59.
\item $y=1 + \dfrac{1}{x^2}$, comprimida verticalmente por un factor de $2$ \\\\
    Respuesta.-\; $y=\dfrac{1}{2}+\dfrac{1}{2x^2}$ \\\\

%--------------------60.
\item $y=1+\dfrac{1}{x^2}$, estirada horizontalmente por un factor de $3$ \\\\
    Respuesta.-\; $y=1+\dfrac{3}{x^2}$ \\\\

%--------------------61.
\item $y=\sqrt{x+1}$, comprimida horizontalmente por un factor de $4$ \\\\
    Respuesta.-\; $y=\sqrt{\dfrac{x}{4} + 1}$ \\\\

%--------------------62.
\item $y=\sqrt{x+1}$, estirada verticalmente por un factor de $3$ \\\\
    Respuesta.-\; $y=3\sqrt{x+1}$ \\\\

%--------------------63.
\item $y=\sqrt{4-x^2}$, estirada horizontalmente por un factor de $2$ \\\\
    Respuesta.-\; $y=\sqrt{4-\dfrac{x^2}{2}}$ \\\\

%--------------------64.
\item $y=\sqrt{4-x^2}$, comprimida verticalmente por un factor de $3$ \\\\
    Respuesta.-\; $y=\dfrac{1}{3}\sqrt{4-x^2}$ \\\\

%--------------------65.
\item $y=1-x^3$, comprimida horizontalmente por un factor de $3$\\\\
    Respuesta.-\; $y=1-2x^3$ \\\\

%--------------------66.
\item $y=1-x^3$, estirada horizontalmente por un factor de $2$ \\\\
    Respuesta.-\; $y=2-2x^3$ \\\\

Graficación\\\\
En los ejercicios $67$ a $74$, trace la gráfica de cada función, pero sin localizar puntos; esto es, utilice la gráfica de una de las funciones estándar presentadas en las figuras $1.14$ a $1.17$ y aplique la transformación adecuada.\\\\

%--------------------67.
\item $y=-\sqrt{2x+1}$\\\\
    Respuesta.-\;
	\begin{center}
	    \begin{tikzpicture}[scale=.6, draw opacity = 0.6]
		% abscisa y ordenada
		\tkzInit[xmax= 8,xmin=-2,ymax=1,ymin=-4]
		\tiny\tkzLabelXY[opacity=0.6,step=1, orig=false]
		% etiqueta x, f(x)
		\tkzDrawX[opacity=0.6,label=x,right=0.3]
		\tkzDrawY[opacity=0.6,label=f(x),below = -0.6]
		%dominio y función
		\draw [domain=0.5:7,thick,gray] plot(\x,{-(2*\x+1)^(1/2)});
	    \end{tikzpicture}
	\end{center}
	\vspace{.5cm}

%--------------------68.
    \item $y=\sqrt{1 - \dfrac{x}{2}}$\\\\
    Respuesta.-\;
	\begin{center}
	    \begin{tikzpicture}[scale=.6, draw opacity = 0.6]
		% abscisa y ordenada
		\tkzInit[xmax= 3,xmin=-2,ymax=2,ymin=-1]
		\tiny\tkzLabelXY[opacity=0.6,step=1, orig=false]
		% etiqueta x, f(x)
		\tkzDrawX[opacity=0.6,label=x,right=0.3]
		\tkzDrawY[opacity=0.6,label=f(x),below = -0.6]
		%dominio y función
		\draw [domain=-1:2,thick,gray] plot(\x,{(1 - (\x/2))^(1/2)});
	    \end{tikzpicture}
	\end{center}
	\vspace{.5cm}
    
%--------------------69.
    \item $y=(x-1)^3 + 2$\\\\
    Respuesta.-\;
	\begin{center}
	    \begin{tikzpicture}[scale=.6, draw opacity = 0.6]
		% abscisa y ordenada
		\tkzInit[xmax= 4,xmin=-4,ymax=7,ymin=-5]
		\tiny\tkzLabelXY[opacity=0.6,step=1, orig=false]
		% etiqueta x, f(x)
		\tkzDrawX[opacity=0.6,label=x,right=0.3]
		\tkzDrawY[opacity=0.6,label=f(x),below = -0.6]
		%dominio y función
		\draw [domain=-.8:2.8,thick,gray] plot(\x,{(\x-1)^3 + 2});
	    \end{tikzpicture}
	\end{center}
	\vspace{.5cm}
    
%--------------------70.
    \item $y=(1-x)^3 + 2$\\\\
    Respuesta.-\;
	\begin{center}
	    \begin{tikzpicture}[scale=.6, draw opacity = 0.6]
		% abscisa y ordenada
		\tkzInit[xmax= 4,xmin=-2,ymax=6,ymin=-4]
		\tiny\tkzLabelXY[opacity=0.6,step=1, orig=false]
		% etiqueta x, f(x)
		\tkzDrawX[opacity=0.6,label=x,right=0.3]
		\tkzDrawY[opacity=0.6,label=f(x),below = -0.6]
		%dominio y función
		\draw [domain=-.7:2.8,thick,gray] plot(\x,{(1-\x)^3 + 2});
	    \end{tikzpicture}
	\end{center}
	\vspace{.5cm}
    
%--------------------71.
    \item $y=\dfrac{1}{2x} - 1$\\\\
    Respuesta.-\;
	\begin{center}
	    \begin{tikzpicture}[scale=.6, draw opacity = 0.6]
		% abscisa y ordenada
		\tkzInit[xmax= 8,xmin=-2,ymax=2,ymin=-1]
		\tiny\tkzLabelXY[opacity=0.6,step=1, orig=false]
		% etiqueta x, f(x)
		\tkzDrawX[opacity=0.6,label=x,right=0.3]
		\tkzDrawY[opacity=0.6,label=f(x),below = -0.6]
		%dominio y función
		\draw [domain=.1:7,thick,gray] plot(\x,{1/2*\x - 1});
	    \end{tikzpicture}
	\end{center}
	\vspace{.5cm}
    
%--------------------72.
    \item $y=\dfrac{2}{x^2} + 1$\\\\
    Respuesta.-\;
	\begin{center}
	    \begin{tikzpicture}[scale=.6, draw opacity = 0.6]
		% abscisa y ordenada
		\tkzInit[xmax= 8,xmin=-2,ymax=8,ymin=-1]
		\tiny\tkzLabelXY[opacity=0.6,step=1, orig=false]
		% etiqueta x, f(x)
		\tkzDrawX[opacity=0.6,label=x,right=0.3]
		\tkzDrawY[opacity=0.6,label=f(x),below = -0.6]
		%dominio y función
		\draw [domain=.5:7,thick,gray] plot(\x,{(2/\x^2) + 1});
	    \end{tikzpicture}
	\end{center}
	\vspace{.5cm}
    
%--------------------73.
\item $y=-\sqrt[3]{x}$\\\\
    Respuesta.-\;
	\begin{center}
	    \begin{tikzpicture}[scale=.6, draw opacity = 0.6]
		% abscisa y ordenada
		\tkzInit[xmax= 8,xmin=-2,ymax=1,ymin=-2]
		\tiny\tkzLabelXY[opacity=0.6,step=1, orig=false]
		% etiqueta x, f(x)
		\tkzDrawX[opacity=0.6,label=x,right=0.3]
		\tkzDrawY[opacity=0.6,label=f(x),below = -0.6]
		%dominio y función
		\draw [domain=.1:7,thick,gray] plot(\x,{-(\x)^(1/3)});
	    \end{tikzpicture}
	\end{center}
	\vspace{.5cm}
    
%--------------------74.
    \item $y=(-2x)^{2/3}$\\\\
    Respuesta.-\;
	\begin{center}
	    \begin{tikzpicture}[scale=.6, draw opacity = 0.6]
		% abscisa y ordenada
		\tkzInit[xmax= 1,xmin=-4,ymax=4,ymin=-1]
		\tiny\tkzLabelXY[opacity=0.6,step=1, orig=false]
		% etiqueta x, f(x)
		\tkzDrawX[opacity=0.6,label=x,right=0.3]
		\tkzDrawY[opacity=0.6,label=f(x),below = -0.6]
		%dominio y función
		\draw [domain=-4:0,thick,gray] plot(\x,{(-2*\x)^(2/3)});
	    \end{tikzpicture}
	\end{center}
	\vspace{.5cm}
    
%--------------------75.
\item $y=|x^2-1|$\\\\
    Respuesta.-\;
	\begin{center}
	    \begin{tikzpicture}[scale=.6, draw opacity = 0.6]
		% abscisa y ordenada
		\tkzInit[xmax= 2,xmin=-2,ymax=3,ymin=-1]
		\tiny\tkzLabelXY[opacity=0.6,step=1, orig=false]
		% etiqueta x, f(x)
		\tkzDrawX[opacity=0.6,label=x,right=0.3]
		\tkzDrawY[opacity=0.6,label=f(x),below = -0.6]
		%dominio y función
		\draw [domain=-2:2.5,thick,gray] plot(\x,{abs(\x^2-1)});
	    \end{tikzpicture}
	\end{center}
	\vspace{.5cm}
    
%--------------------76.
    \item $y=\sqrt{|x|}$\\\\
    Respuesta.-\;
	\begin{center}
	    \begin{tikzpicture}[scale=.6, draw opacity = 0.6]
		% abscisa y ordenada
		\tkzInit[xmax= 7,xmin=-6,ymax=3,ymin=-1]
		\tiny\tkzLabelXY[opacity=0.6,step=1, orig=false]
		% etiqueta x, f(x)
		\tkzDrawX[opacity=0.6,label=x,right=0.3]
		\tkzDrawY[opacity=0.6,label=f(x),below = -0.6]
		%dominio y función
		\draw [domain=-5:5,thick,gray] plot(\x,{(abs(\x))^(1/2)});
	    \end{tikzpicture}
	\end{center}
	\vspace{.5cm}

Combinación de funciones\\\\

%--------------------77.
\item Suponga que $f$ es una función par, $g$ es una función impar y ambas, $f$ y $g$, están definidas en toda la recta real $(-\infty, \infty)$. ¿Cuáles de las siguientes funciones (donde están definidas) son pares? ¿Cuáles son impares?.
    \begin{enumerate}[\bfseries a)]

	%----------a)
	\item $fg$, es impar.\\\\

	%----------b)
	\item $f/g$, es impar.\\\\

	%----------c)
	\item $g/f$, es impar.\\\\

	%----------d)
	\item $f^2 = ff$, es par.\\\\

	%----------e)
	\item $g^2 = gg$, es par.\\\\

	%----------f)
	\item $f\circ g$, es par.\\\\

	%----------g)
	\item $g\circ f$, es par.\\\\

	%----------h)
	\item $f\circ f$, es par.\\\\

	%----------i)
	\item $g\circ g$, es impar.\\\\

    \end{enumerate}
    Esto por Michael Spivak, Calculus I.\\\\

%--------------------78.
\item ¿Una función puede ser par e impar al mismo tiempo? Justifique su respuesta.\\\\
    Respuesta.-\;  Se define como función par a $f(x)=f(-x)$ y función impar como $f(-x)=-f(x)$ de donde $f(x) \neq -f(x)$ el cual nos indica que no puede ser par e impar al mismo tiempo ya que son contrarias y no compatibles.\\\\

%--------------------79.
\item (Continuación del ejemplo 1). Trace las gráficas de las funciones $f(x) = \sqrt{x}$ y $g(x) = \sqrt{1-x}$ junto con a) su suma, b) su producto, c) sus dos restas, d) sus dos cocientes.\\\\
    Respuesta.-\; 
    \begin{itemize}
    \item $\sqrt{x} + \sqrt{1-x}$
	\begin{center}
	    \begin{tikzpicture}[scale=.6, draw opacity = 0.6]
		% abscisa y ordenada
		\tkzInit[xmax= 2,xmin=-1,ymax=2,ymin=-1]
		\tiny\tkzLabelXY[opacity=0.6,step=1, orig=false]
		% etiqueta x, f(x)
		\tkzDrawX[opacity=0.6,label=x,right=0.3]
		\tkzDrawY[opacity=0.6,label=f(x),below = -0.6]
		%dominio y función
		\draw [domain=0:1,thick,gray] plot(\x,{(\x)^(1/2) + (1-\x)^(1/2)});
	    \end{tikzpicture}
	\end{center}
	\vspace{.5cm}

    \item $\sqrt{x}\cdot \sqrt{1-x}$
	\begin{center}
	    \begin{tikzpicture}[scale=.6, draw opacity = 0.6]
		% abscisa y ordenada
		\tkzInit[xmax= 2,xmin=-1,ymax=2,ymin=-1]
		\tiny\tkzLabelXY[opacity=0.6,step=1, orig=false]
		% etiqueta x, f(x)
		\tkzDrawX[opacity=0.6,label=x,right=0.3]
		\tkzDrawY[opacity=0.6,label=f(x),below = -0.6]
		%dominio y función
		\draw [domain=0:1,thick,gray] plot(\x,{(\x)^(1/2) * (1-\x)^(1/2)});
	    \end{tikzpicture}
	\end{center}
	\vspace{.5cm}

    \item $\sqrt{x} - \sqrt{1-x}$
	\begin{center}
	    \begin{tikzpicture}[scale=.6, draw opacity = 0.6]
		% abscisa y ordenada
		\tkzInit[xmax= 2,xmin=-1,ymax=2,ymin=-1]
		\tiny\tkzLabelXY[opacity=0.6,step=1, orig=false]
		% etiqueta x, f(x)
		\tkzDrawX[opacity=0.6,label=x,right=0.3]
		\tkzDrawY[opacity=0.6,label=f(x),below = -0.6]
		%dominio y función
		\draw [domain=0:1,thick,gray] plot(\x,{(\x)^(1/2) - (1-\x)^(1/2)});
	    \end{tikzpicture}
	\end{center}
	\vspace{.5cm}

    \item $\sqrt{1-x} - \sqrt{x}$ 
	\begin{center}
	    \begin{tikzpicture}[scale=.6, draw opacity = 0.6]
		% abscisa y ordenada
		\tkzInit[xmax= 2,xmin=-1,ymax=2,ymin=-1]
		\tiny\tkzLabelXY[opacity=0.6,step=1, orig=false]
		% etiqueta x, f(x)
		\tkzDrawX[opacity=0.6,label=x,right=0.3]
		\tkzDrawY[opacity=0.6,label=f(x),below = -0.6]
		%dominio y función
		\draw [domain=0:1,thick,gray] plot(\x,{(1-\x)^(1/2)-(\x)^(1/2)});
	    \end{tikzpicture}
	\end{center}
	\vspace{.5cm}

    \item $\sqrt{x} / \sqrt{1-x}$
	\begin{center}
	    \begin{tikzpicture}[scale=.6, draw opacity = 0.6]
		% abscisa y ordenada
		\tkzInit[xmax= 2,xmin=-1,ymax=2,ymin=-1]
		\tiny\tkzLabelXY[opacity=0.6,step=1, orig=false]
		% etiqueta x, f(x)
		\tkzDrawX[opacity=0.6,label=x,right=0.3]
		\tkzDrawY[opacity=0.6,label=f(x),below = -0.6]
		%dominio y función
		\draw [domain=0:.9,thick,gray] plot(\x,{(\x)^(1/2) / (1-\x)^(1/2)});
	    \end{tikzpicture}
	\end{center}
	\vspace{.5cm}

    \item $\sqrt{1-x} / \sqrt{x}$ 
	\begin{center}
	    \begin{tikzpicture}[scale=.6, draw opacity = 0.6]
		% abscisa y ordenada
		\tkzInit[xmax= 2,xmin=-1,ymax=2,ymin=-1]
		\tiny\tkzLabelXY[opacity=0.6,step=1, orig=false]
		% etiqueta x, f(x)
		\tkzDrawX[opacity=0.6,label=x,right=0.3]
		\tkzDrawY[opacity=0.6,label=f(x),below = -0.6]
		%dominio y función
		\draw [domain=0.1:.9,thick,gray] plot(\x,{(1-\x)^(1/2)/(\x)^(1/2)});
	    \end{tikzpicture}
	\end{center}
	\vspace{.5cm}

    \end{itemize}

%--------------------80.
\item Sean $f(x)=x-7$ y $g(x)=x^2$. Trace las gráficas de $f$ y $g$ junto con $f\circ g$ y $g\circ f$\\\\
    Respuesta.-\; 
	\begin{center}
	    \begin{tikzpicture}[scale=.6, draw opacity = 0.6]
		% abscisa y ordenada
		\tkzInit[xmax= 9,xmin=-5,ymax=8,ymin=-10]
		\tiny\tkzLabelXY[opacity=0.6,step=1, orig=false]
		% etiqueta x, f(x)
		\tkzDrawX[opacity=0.6,label=x,right=0.3]
		\tkzDrawY[opacity=0.6,label=f(x),below = -0.6]
		%dominio y función
		\draw [domain=-2:9,thick,gray] plot(\x,{\x-7});
		\draw [domain=-2.5:2.5,thick,gray] plot(\x,{\x^2});
		\draw [domain=4:10,thick,gray] plot(\x,{(\x-7)^2});
		\draw [domain=-2:3,thick,gray] plot(\x,{\x^2-7});
	    \end{tikzpicture}
	\end{center}
	\vspace{.5cm}

\end{enumerate}

\section{Funciones trigonométricas}

%--------------------definición 1.7
\begin{tcolorbox}[colframe=white]
    \begin{def.}
	$$s=r\cdot \theta$$
	$s$ es el arco subtendido por el ángulo central.\\
	$r$ es el radio de la circunferencia.\\
	$\theta$ es el ángulo en radianes.
    \end{def.}
\end{tcolorbox}

%--------------------definición 1.8
\begin{tcolorbox}[colframe=white]
    \begin{def.}
	Una función $f(x)$ es periódica si existe un número positivo $p$ tal que $f(x+p)=f(x)$ para todo valor de $x$. El menor de los valores posibles de $p$ es el periodo de $f$.
    \end{def.}
\end{tcolorbox}

%--------------------propiedad 1
\begin{tcolorbox}[colframe=white]
    \begin{prop}
	$$\cos^2 \theta \sin^2 \theta = 1$$
    \end{prop}
\end{tcolorbox}

%--------------------propiedad 2 
\begin{tcolorbox}[colframe=white]
    \begin{prop}[Suma de ángulos]
	$$\cos(A+B)=\cos A \cos B - \sin A \sin B$$
	$$\sin(A+B) = \sin A \cos B + \cos A \sin B$$
    \end{prop}
\end{tcolorbox}

%--------------------propiedad 3 
\begin{tcolorbox}[colframe=white]
    \begin{prop}[Doble de un ángulo]
	$$\cos 2\theta = \cos^2 \theta - \sin^2 \theta$$
	$$\sin 2\theta = 2\sin \theta \cos \theta$$
    \end{prop}
\end{tcolorbox}

%--------------------propiedad 4 
\begin{tcolorbox}[colframe=white]
    \begin{prop}[Mitad de un ángulo]
	$$\cos^2 \theta = \dfrac{1 + \cos 2\theta}{2}$$
	$$\sin^2 \theta = \dfrac{1 - \cos 2\theta}{2}$$
    \end{prop}
\end{tcolorbox}

%--------------------propiedad 5 
\begin{tcolorbox}[colframe=white]
    \begin{prop}[Ley de cosenos]
	$$c^2 = a^2 + b^2 - 2ab \cos \theta$$
    \end{prop}
\end{tcolorbox}

%--------------------5.
\begin{tcolorbox}[colframe=white]
    \begin{prop}[Desigualdades especiales]
	$$-|\theta| \leq \sin \theta \leq |\theta| \quad y \quad -|\theta| \leq 1 - \cos \theta \leq |\theta|$$
    \end{prop}
\end{tcolorbox}


\setcounter{section}{2}
\section{Ejercicios}

Radianes y grados\\
\begin{enumerate}[\Large \bfseries 1.]

%--------------------1.
\item En una circunferencia con radio de $10\; m$, ¿Cuál es la longitud de un arco que subtiende un  ángulo central de $a) \; 4\pi/5$ radianes y $b)\; 110^\circ$?\\\\
    Respuesta.-\; Para $a)$ se tiene $s=10 \cdot 4\pi/5 = 8\pi$, luego para $b)$ se tiene $s=10\cdot 110^\circ \cdot \dfrac{\pi}{180^\circ} = \dfrac{55}{9}\pi$\\\\

%--------------------2.
\item Un ángulo central en una circunferencia de radio $8$ está subtendido por un arco cuya longitud es de $10 \pi$. Determine la medida del ángulo en radianes y en grados.\\\\
    Respuesta.-\; Sea $s=r\cdot \theta$ entonces $\theta = \dfrac{s}{r}$ de donde $\theta = \dfrac{10\pi}{8} = \dfrac{5\pi}{4}$ radianes, luego $\dfrac{5\pi}{4}\cdot \dfrac{180^\circ}{\pi} = 225^\circ$\\\\

%--------------------3.
\item Se desea construir un ángulo de $80^\circ$ formando un arco en el perímetro de un disco de $12$ pulgadas de diámetro, y dibujando rectas de los extremos del arco al centro del disco. ¿De qué longitud debe ser el arco, redondeando a décimos de pulgada?.\\\\
    Respuesta.-\; Sea $r=6 \; pulg$ y $\theta = 80^\circ \cdot = \dfrac{\pi}{180^\circ}=\dfrac{4}{9}\pi$ entonces $$s=6\cdot \dfrac{4}{9}\pi = \dfrac{8}{3}\pi = 8.4\; pulg$$\\

%--------------------4.
\item Si una rueda de $1\;m$ de diámetro se hace rodar hacia adelante $30\; cm$ sobre el suelo, ¿qué ángulo girará? Dé su respuesta en radianes (redondeando al décimo más cercano) y en grados (redondeando al grado más cercano).\\\\
    Respuesta.-\; Sea $r=50cm$ y $s=30cm$ entonces $$\theta = \dfrac{30\; cm}{50 \; cm} = \dfrac{3}{5}\; rad.$$ Luego convertimos a grado de la siguiente manera $$\theta = \dfrac{3}{5}\cdot \dfrac{180^\circ}{\pi} = 34^{\circ}37^{'}$$ 

Evaluación de funciones trigonométricas\\\\

%--------------------5.
\item . Complete la siguiente tabla con los valores de la función. Si la función no está definida en el ángulo dado, indíquelo con la leyenda “IND”. No use calculadora ni tablas.
    \begin{center}
	\begin{tabular}{cccccc}
	    \hline\\
	    $\theta$&$-\pi$&$-\dfrac{2}{3}\pi$&$0$&$\dfrac{\pi}{2}$&$\dfrac{3}{4}\pi$\\\\
	    \hline\\
	    $\sen \theta$&$0$&$-\dfrac{\sqrt{3}}{2}$&$0$&$1$&$\dfrac{\sqrt{2}}{2}$\\\\
	    $\cos \theta$&$-1$&$-\dfrac{1}{2}$&$1$&$0$&$-\dfrac{1}{\sqrt{2}}$\\\\
	    $\tan \theta$&$0$&$\sqrt{3}$&$0$&$IND$&$-1$\\\\
	    $\cot \theta$&$IND$&$\dfrac{1}{\sqrt{3}}$&$IND$&$0$&$-1$\\\\
	    $sec\; \theta$&$-1$&$-2$&$1$&$IND$&$-\sqrt{2}$\\\\
	    $csc\; \theta$&$IND$&$-\dfrac{2}{\sqrt{3}}$&$IND$&$1$&$\sqrt{2}$\\\\
	    \hline
	\end{tabular}
    \end{center}

%--------------------6.
\item . Complete la siguiente tabla con los valores de la función. Si la función no está definida en el ángulo dado, indíquelo con la leyenda “IND”. No use calculadora ni tablas.
    \begin{center}
	\begin{tabular}{cccccc}
	    \hline\\
	    $\theta$&$-3\pi/2$&$-\pi/3$&$-\pi/6$&$\pi/4$&$4\pi/6$\\\\
	    \hline\\
	    $\sen \theta$&$-\dfrac{\sqrt{2}}{2}$&$-\dfrac{\sqrt{3}}{2}$&$0$&$1$&$\dfrac{\sqrt{2}}{2}$\\\\
	    $\cos \theta$&$-1$&$-\dfrac{1}{2}$&$1$&$0$&$-\dfrac{1}{\sqrt{2}}$\\\\
	    $\tan \theta$&$0$&$\sqrt{3}$&$0$&$IND$&$-1$\\\\
	    $\cot \theta$&$IND$&$\dfrac{1}{\sqrt{3}}$&$IND$&$0$&$-1$\\\\
	    $sec\; \theta$&$-1$&$-2$&$1$&$IND$&$-\sqrt{2}$\\\\
	    $csc\; \theta$&$IND$&$-\dfrac{2}{\sqrt{3}}$&$IND$&$1$&$\sqrt{2}$\\\\
	    \hline
	\end{tabular}
    \end{center}

En los ejercicios $7$ a $12$, uno de los valores $\sen x$, $\cos x$ y $\tan x$ está dado. Determine los otros dos si $x$ se encuentra en el intervalo indicado.\\\\

%--------------------7.
\item $\sen x = \dfrac{3}{5}, \; x \in \left[\dfrac{\pi}{2},\pi\right]$\\\\
    Respuesta.-\; Por le teorema de Pitágoras  $r^2=x^2+y^2 \Longrightarrow x^2 = 5^2 - 3^2 = 25 - 9 = \sqrt{16}=4$ luego $\cos x = - \dfrac{4}{5}, \;\tan x = -\dfrac{3}{4}$\\\\

%--------------------8.
\item $\tan x = 2, \; x\in \left[0,\dfrac{\pi}{2}\right]$\\\\
    Respuesta.-\; $\sen x = 2 \; ; \; \cos x = 1$\\\\

%--------------------9.
\item $\cos x = \dfrac{1}{3}, \; x \in \left[-\dfrac{\pi}{2},0\right]$\\\\
    Respuesta.-\; $\sen x= -\dfrac{\sqrt{8}}{3} \; ; \; \tan x = -\sqrt{8}$\\\\

%--------------------10.
\item $\cos x = -\dfrac{5}{13},\; x \in \left[\dfrac{\pi}{2},\pi\right]$\\\\
    Respuesta.-\; $ y^2 = 13^2 - 5^2 = 169 - 25 = 144 \Longrightarrow y=12$ por lo tanto $\sen x = \dfrac{12}{5},\; \tan x = \dfrac{12}{13}$\\\\

%--------------------11.
\item $\tan x = \dfrac{1}{2},\; x \in \left[\pi, \dfrac{3\pi}{2}\right]$\\\\
    Respuesta.-\; $\sen x = -\dfrac{1}{\sqrt{5}},\; \cos x = -\dfrac{2}{\sqrt{5}}$\\\\

%-------------------12.
\item $\sen x = -\dfrac{1}{2},\; x\in \left[\pi, \dfrac{3\pi}{2}\right]$\\\\
    Respuesta.-\; $\cos x = -\dfrac{\sqrt{3}}{2}, \; \tan \dfrac{1}{\sqrt{3}}$\\\\

Gráfica de funciones trigonométricas.\\\\
Grafíque las funciones de los ejercicios $13$ a $22$. ¿Cuál es el periodo de cada función?.\\\\

%--------------------13.
\item $\sen 2x$\\\\
    Respuesta.-\; La gráfica se tiene en el apartado python. Su periodo viene dado por $\pi$\\\\

%--------------------14.
\item $\sen(x/2)$\\\\
    Respuesta.-\; La gráfica se tiene en el apartado python. Su periodo es $4\pi$\\\\

%--------------------15.
\item $\cos \pi x$\\\\
    Respuesta.-\; La gráfica se tiene en el apartado python. Su periodo es $2$\\\\

%--------------------16.
\item $\cos \dfrac{\pi x}{2}$\\\\
    Respuesta.-\; La gráfica se tiene en el apartado python. Su periodo es $4$\\\\

%--------------------17.
\item $-\sen \dfrac{\pi x}{3}$\\\\
    Respuesta.-\; La gráfica se tiene en el apartado python. Su periodo es $6$\\\\

%--------------------18.
\item $- \cos 2\pi x$\\\\
    Respuesta.-\; La gráfica se tiene en el apartado python. Su periodo es $1$\\\\

%--------------------19.
\item $\cos \left(x - \dfrac{\pi}{2}\right)$\\\\
    Respuesta.-\; La gráfica se tiene en el apartado python. Su periodo es $2\pi$\\\\

%--------------------20.
\item $\sen \left(x+\dfrac{\pi}{6}\right)$\\\\
    Respuesta.-\; La gráfica se tiene en el apartado python. Su periodo es $\pi$\\\\

%--------------------21.
\item $\sen \left(x-\dfrac{\pi}{4}\right)+1$\\\\
    Respuesta.-\; La gráfica se tiene en el apartado python. Su periodo es $2\pi$\\\\

%--------------------22.
\item $\cos \left(x+\dfrac{2\pi}{3}\right)-2$\\\\
    Respuesta.-\; La gráfica se tiene en el apartado python. Su periodo es $2\pi$\\\\

Grafique las funciones de los ejercicios 23 a 26 en el plano $ts$ ($t$ es el eje horizontal, y $s$, el eje vertical). ¿Cuál es el periodo de cada función? ¿Qué tipo de simetría tienen las gráficas?\\\\

%--------------------23.
\item $s=cot\; 2t$\\\\
    Respuesta.-\; Periodo de $\pi/2$ y la simetría respecto al origen.\\\\

%--------------------24.
\item $s=-\tan \pi t$\\\\
    Respuesta.-\; Periodo de $1$ y la simetría respecto al origen.\\\\

%--------------------25.
\item $s = sec\left(\dfrac{\pi t}{2}\right)$\\\\
    Respuesta.-\; Periodo es $4$ y la simetría respecto al eje $y$.\\\\

%--------------------26.
\item $s = csc\left(\dfrac{t}{2}\right)$\\\\
    Respuesta.-\; Periodo de $4\pi$ y la simetría respecto al eje $y$.\\\\

%--------------------27.
\item 
    \begin{enumerate}[\bfseries a)]

	%--------------------a)
	\item Grafique $y=\cos x$ e $y=sec \; x$ en el mismo plano cartesiano para $-3\pi/2 \leq x \leq 3\pi/2$. Comente el comportamiento de $sec\, x$ en relación con los signos y valores de $\cos x$.\\\\

	%--------------------b)
	\item Grafique $y = \sen x$ e $y = csc \; x$ en el mismo plano cartesiano para $\pi \leq x \leq 2\pi$. Comente el comportamiento de $csc\; x$ en relación con los signos y valores de $\sen x$.\\\\
	    
    \end{enumerate}

%--------------------28.
\item Grafique $y = \tan x$ e $y = cot \; x$ en el mismo plano cartesiano para $-7\leq x \leq 7$. Comente el comportamiento de $cot \; x$ en relación con los signos y valores de $\tan x$.\\\\

%--------------------29.
\item Grafique $y=\sen x$ e $y=[\sen x]$ en el mismo plano cartesiano. ¿Cuál es su dominio y rango de $[\sen x]$?\\\\

%--------------------30.
\item Grafique $y=\sen x$ e $y = [\sen x]$ en el mismo plano cartesiano. ¿Cuáles son el dominio y rango de $[\sen x]$?\\\\

Uso de fórmulas de suma\\\\

Use las fórmulas de ángulos para deducir las identidades de los ejercicios $31$ a $36$\\\\

%--------------------31.
\item $\cos \left(x-\dfrac{\pi}{2}\right) = \sen x$\\\\
    Respuesta.-\; $\cos\left(x-\dfrac{\pi}{x}\right) = \cos x \cdot \cos\left(-\dfrac{\pi}{2}\right)-\sen x \cdot \sen \left(-\dfrac{\pi}{2}\right)=0\cdot \cos\left(-\dfrac{\pi}{2}\right) - \sen x \cdot (-1) = \sen x$\\\\

%--------------------32.
\item $\cos\left(x + \dfrac{\pi}{2}\right)=-\sen x$\\\\
    Respuesta.-\; Análogamente al anterior ejercicio se tiene $\cos\left(x + \dfrac{\pi}{2}\right)=-\sen x$\\\\

%--------------------33.
\item $\sen\left(x + \dfrac{\pi}{2}\right)=\cos x$\\\\
    Respuesta.-\; $\sen\left(x + \dfrac{\pi}{2}\right)= \sen x \cdot \cos \left(\dfrac{\pi}{2}\right) + \cos x \cdot \sen \left(\dfrac{\pi}{2}\right) = \sen x \cdot 0 + \cos x \cdot 1 = \cos x$\\\\

%--------------------34.
\item $\sen \left(x-\dfrac{\pi}{2}\right) = - \cos x$\\\\
    Respuesta.-\; Se puede resolver análogamente al ejercicio anterior poniendo a $\sen \left[x + \left(-\dfrac{\pi}{2}\right)\right]$\\\\

%--------------------35.
\item $\cos\left(A-B\right) = \cos A \cos B + \sen A \sen B$\\\\
    Respuesta.-\; Deducimos que $\cos(-B)=\cos B$ y $\sen(-B)=-\sen B$, por lo tanto: $$\cos(A-B)=\cos A \cos (-B) - \sen A \sen (-B) = \cos A \cos B + \sen A \sen B$$\\ 

%--------------------36.
\item $\sen (A-B) = \sen A \cos B - \cos A \sen B$\\\\
    Respuesta.-\; Análogamente al anterior ejercicio se tiene el resultado deseado.\\\\ 

%--------------------37.
\item ¿Qué pasa si tomamos $B=A$ en la identidad $\cos(A-B)=\cos A \cos B + \sen A \sen B$? ¿El resultado concuerda con algo que ya conoce?\\\\
    Respuesta.-\; Obtenemos $\cos 2A = \cos^2 A - \sen^2 A$ el cual es la formula para el doble de un ángulo.\\\\

%--------------------38.
\item ¿Qué pasa si tomamos $B=2\pi$ en las fórmulas de suma? ¿El resultado concuerda con algo que ya conocemos?\\\\
    Respuesta.-\; Si tomamos $\cos\left(A+\dfrac{\pi}{2}\right) = \cos A \cos\left(\dfrac{\pi}{2}\right) - \sen A \sen\left(\dfrac{\pi}{2}\right) = - \sen A$.\\ Ahora tomamos $\sen\left(A + \dfrac{\pi}{2}\right) = \sen A \cos\left(\dfrac{\pi}{2}\right) + \cos A \sen\left(\dfrac{\pi}{2}\right) = \cos A$, No encuentro algo que concuerde con el resultado, esto por por los apuntes del libro base.\\\\

En los ejercicios $39$ a $42$, exprese la cantidad dad en términos de $\sen x$ y $\cos x.$\\\\

%--------------------39.
\item $\cos(\pi + x) = \cos \pi \cos x - \sen \pi \sen B = -\cos x$\\\\

%--------------------40.
\item $\sen\left(\dfrac{3\pi}{2} - x\right) = \cos\left(\dfrac{3\pi}{2}\right) \cos x + \sen\left(\dfrac{3\pi}{2}\right)\sen x = -\sen x $\\\\

%--------------------41.
\item $\sen\left(\dfrac{3\pi}{2} - x\right) = \sen\left(\dfrac{3\pi}{2}\right) \cos x - \cos\left(\dfrac{3\pi}{2}\right) \sen x = -\cos x$\\\\

%--------------------42.
\item $\cos \left(\dfrac{3\pi}{2}+x\right) = \cos\left(\dfrac{3\pi}{2}\right) \cos x - \sen \left(\dfrac{3\pi}{2}\right) \sen x= \sen x$\\\\

%--------------------43.
\item Evalúe $\sen \dfrac{7\pi}{12}$ como $\sen\left(\dfrac{\pi}{4}+\dfrac{\pi}{3}\right)$\\\\
    Respuesta.-\; $\sen \left(\dfrac{\pi}{4} + \dfrac{\pi}{3}\right) = \sen\left(\dfrac{\pi}{4} \right) \cos\left(\dfrac{\pi}{3}\right) + \cos\left(\dfrac{\pi}{4} \right) \sen\left(\dfrac{\pi}{3}\right) = \dfrac{\sqrt{2}}{4}  +  \dfrac{\sqrt{2}\sqrt{3}}{4} = \dfrac{\sqrt{2}+\sqrt{6}}{4}$\\\\ 

%--------------------44.
\item Evalúe $\cos \dfrac{11\pi}{12}$ como $\cos\left(\dfrac{\pi}{4} + \dfrac{2\pi}{3}\right)$\\\\
    Respuesta.-\; Tenemos que $\cos \left(\dfrac{\pi}{4}\right) \cos \left(\dfrac{2\pi}{3} \right) - \sen \left( \dfrac{\pi}{4} \right) \sen \left( \dfrac{2\pi}{3} \right) = \dfrac{\sqrt{2}}{2} \dfrac{-1}{2} - \dfrac{\sqrt{2}}{2} \dfrac{\sqrt{3}}{2} = \dfrac{-\sqrt{2} - \sqrt{6}}{4}$\\\\

%--------------------45.
\item Evalúe $\cos \dfrac{\pi}{12}$\\\\
    Respuesta.-\; Sea $\cos \left(\dfrac{\pi}{3}-\dfrac{\pi}{4}\right)$ entonces se tiene \; $\cos\left(\dfrac{\pi}{3}\right) \cos\left(\dfrac{\pi}{4}\right) + \sen\left(\dfrac{\pi}{3}\right) \sen\left(\dfrac{\pi}{4}\right)$ por lo tanto $$\dfrac{\sqrt{2}+\sqrt{6}}{4}$$\\

%--------------------46.
\item Evalúe $\sen \dfrac{5\pi}{12}$\\\\
    Respuesta.-\; Sea $\sen \left( \dfrac{2\pi}{3} - \dfrac{\pi}{4}\right)$ entonces se tiene \; $\sen\left( \dfrac{2\pi}{3}\right) \cos\left(\dfrac{\pi}{4}\right) - \cos\left( \dfrac{2\pi}{3} \right) \sen\left(\dfrac{\pi}{4}\right)$ de donde $$\dfrac{\sqrt{3}}{2} \dfrac{\sqrt{2}}{2} - \dfrac{-1}{2}\dfrac{\sqrt{2}}{2} = \dfrac{\sqrt{6}+\sqrt{2}}{4}$$\\

Uso de las fórmulas para medio ángulo\\\\

Determine los valores de la función en los ejercicios $47$ a $50$\\\\

%--------------------47.
\item $\cos^2 \dfrac{\pi}{8}$\\\\
    Respuesta.-\; Aplicando las fórmulas para la mitad de un ángulo se tiene $\dfrac{1+\cos 2 \cdot \dfrac{\pi}{8}}{2} = \dfrac{2+\sqrt{2}}{4}$\\\\

%--------------------48.
\item $\cos^2 \dfrac{5\pi}{12} = \dfrac{1 + \cos \dfrac{5\pi}{6}}{2} = \dfrac{2-\sqrt{3}}{4}$\\\\

%--------------------49.
\item $\sen^2 \dfrac{\pi}{12} = \dfrac{1 - \cos \dfrac{\pi}{6}}{2} = \dfrac{2-\sqrt{3}}{4}$\\\\

%--------------------50.
\item $\sen^2 \dfrac{3\pi}{8} = \dfrac{1-\cos \dfrac{3\pi}{4}}{2} = \dfrac{2 + \sqrt{2}}{4}$\\\\

Solución de ecuaciones trigonométricas\\\\

Resuelva los ejercicios $51$ a $54$ para el ángulo $\theta$, donde $0\leq \theta\leq 2\pi$.\\\\

%--------------------51.
\item $\sen^2 \theta = \dfrac{3}{4}$\\\\
    Respuesta.-\; Sea $\sen^2 \theta = \dfrac{3}{4} \quad \Longrightarrow \quad \sen \theta = \pm \dfrac{\sqrt{3}}{2}$ entonces, se cumple para $\dfrac{\pi}{3},\dfrac{2\pi}{3},\dfrac{4\pi}{3},\dfrac{5\pi}{3}$\\\\

%--------------------52.
\item $\sen^2 \theta = \cos^2 \theta$\\\\
    Respuesta.-\; Viene dado por $\dfrac{\pi}{4}, \dfrac{3\pi}{4},\dfrac{5\pi}{4},\dfrac{7\pi}{4}$\\\\

%--------------------53.
\item $\sen 2\theta - \cos \theta = 0$\\\\
    Respuesta.-\; $\theta = \dfrac{\pi}{2},\dfrac{3\pi}{2},\dfrac{\pi}{6},\dfrac{5\pi}{6}$\\\\

%--------------------54.
\item $\cos 2 \theta + \cos \theta = 0$\\\\
    Respuesta.-\; $\theta = \dfrac{\pi}{3},\pi,\dfrac{5\pi}{3}$\\\\

Teoría y ejemplos\\\\

%--------------------55.
\item Fórmula de la tangente de una suma. La fórmula estándar para la tangente de la suma de dos ángulos es $$\tan(A+B)=\dfrac{\tan A + \tan B}{1 - \tan A \tan B}$$
Deduzca la fórmula.\\\\
    Respuesta.-\; Sabemos que $\tan(A+B) = \dfrac{\sen (A+B)}{\cos (A+B)}$, entonces por la fórmula de suma de ángulos se tiene $\dfrac{\sen A \cos B + \cos A \sen B}{\cos A \cos B - \sen A \sen B}$, luego multiplicamos por $\dfrac{\cos A  \cos B}{\cos A \cos B}$ y nos queda $\dfrac{\dfrac{\sen A \cos B}{\cos A \cos B} + \dfrac{\cos A \sin B}{\cos A \cos B}}{\dfrac{\cos A \cos B}{\cos A \cos B} - \dfrac{\sen A \sen B}{\cos A \cos B}}$ de donde nos queda $$\dfrac{\tan A + \tan B}{1 - \tan A \tan B}$$\\

%--------------------56.
\item Deduzca la fórmula para $\tan (A-B)$\\\\
    Respuesta.-\; Análogamente al ejercicio anterior tenemos $$\tan(A-B) = \dfrac{\tan A - \tan B}{1 + \tan A \tan B}$$\\\\

%--------------------57.
\item Aplique la ley de los cosenos en el triángulo de la siguiente figura para deducir la fórmula de $\cos (A - B)$.\\\\
    Respuesta.-\; Sea $c^2=a^2 + b^2 - 2ab\cos \theta$ entonces $c^2 = 2 - 2\cos(A-B)$. Por el teorema de Pitágoras tenemos que $c^2 = (\cos A - \cos B)^2 + (\sen A - \sen B)^2$ de donde nos que $c^2 = 2 - 2(\cos A \cos B + \sen A \sen B)$. Por lo tanto $$\cos (A-B) = \cos A \cos B + \sen A \sen B$$\\

%--------------------58.
\item 
    \begin{enumerate}[\bfseries a)]
	
	%----------a)
	\item Aplique la fórmula de $\cos (A-B)$ a la identidad $\sen \theta = \cos \left(\dfrac{\pi}{2} - \theta\right)$, para obtener la fórmula de suma de $\sen (A+B)$\\\\
	    Respuesta.-\; Aplicando tenemos $\sen \theta = \cos \dfrac{\pi}{2}\cos \theta + \sen\dfrac{\pi}{2} \sen\theta=\sen \theta.$\\\\

	%----------b)
	\item Deduzca la fórmula de $\cos (A+B)$ sustituyendo $B$ por $-B$ en la fórmula de $\cos (A-B)$ del ejercicio $35$.\\\\
	    Respuesta.-\; Sabemos que el $\cos (-B) = \cos (B)$ por lo tanto $\cos[A-(-B)] = \cos A \cos B - \sen A (\sen B)$\\\\ 

    \end{enumerate}

%--------------------59.
\item Un triángulo tiene lados $a = 2$ y $b = 3$, y el ángulo $C = 60^\circ$. Determine la longitud del lado $c$.\\\\
    Respuesta.-\; Por el teorema de cosenos tenemos $c^2 = 4 + 9 - 12\cos\dfrac{\pi}{3} = \sqrt{13 - 12 \cos\dfrac{\pi}{3}} =  \sqrt{7}$\\\\

%--------------------60.
\item  Un triángulo tiene lados $a = 2$ y $b = 3$, y el ángulo $C = 40^\circ$. Determine la longitud del lado $c$.\\\\
    Respuesta.-\; $c = \sqrt{13 - 12\cos\dfrac{2\pi}{9}} = 1.95$\\\\

%--------------------61.
\item Ley de los senos. La ley de los senos afirma que si $a,b$ y $c$ son los lados opuestos a los ángulos $A,B$ y $C$ en un triángulo, entonces, $$\dfrac{\sen A}{a} = \dfrac{\sen B}{b} = \dfrac{\sen C}{c}$$
Use las siguientes figuras y si lo requiere, la identidad $\sen(\pi - \theta) = \sen \theta$, para deducir la ley.\\\\
    Respuesta.-\; De la figura en el texto vemos que $\sen B=\dfrac{h}{c}$. Si $C$ es un ángulo agudo, entonces $\sen C = \dfrac{h}{b}$. Por otro lado, si $C$ es obtuso, entonces $C = \sen(\pi - C) = \dfrac{h}{b}$, por lo tanto en cualquier caso, $h=b \sen C = c \sen B \; \longrightarrow \; ah = ab \sen C = ac \sen B$.\\
    Luego por la ley de cosenos, $C = \dfrac{a^2 + b^2 - c^2}{2ab}$ y $B=\dfrac{a^2 + c^2 - b^2}{2ac}$, Además, dado que la suma de los ángulos interiores de un triangulo es $\pi$, tenemos $$\sen A = \sen [\pi - (B+C)] = \sen(B+C) = \sen B \cos C + \cos B \cos C = \dfrac{h}{bc}\cdot \dfrac{a^2+b^2-c^2}{2ab} + \dfrac{a^2+c^2 - b^2}{2ac}\cdot \dfrac{h}{b} = \dfrac{ah}{bc}$$ entonces $$ah = bc \sen A$$
    Por último combinando los resultados  $ah=ab \sen C,$ $ah=ac \sen B$ y $ah=bc\sen A$ y dividiendo por $abc$ tenemos $$\dfrac{h}{bc} = \dfrac{\sen A}{a} = \dfrac{\sen B}{b} = \dfrac{\sen C}{c}$$\\

%--------------------62.
\item Un triangulo tiene lados $a=2$ y $b=3$ y el ángulo $C=60^\circ$. Obtenga el seno del ángulo $B$ utilizando la ley de los senos.\\\\
    Respuesta.-\; Sea $\dfrac{\sen B}{3} = \dfrac{\sqrt{3}/2}{c}$ pero sabemos que $c=\sqrt{7}$ entonces $\sen B = \dfrac{3\sqrt{3}}{2\sqrt{7}}$\\\\

%--------------------63.
\item Un triángulo tiene un lado $c = 2$ y ángulos $A = \pi/4$ y $B = \pi / 3$. Determine la longitud $a$ del lado opuesto a $A$.\\\\
    Respuesta.-\; Sea $b^2 = a^2 + 4 - 4a \cos\dfrac{\pi}{3} = a^2 - 2a + 4$ luego por la ley de senos $\dfrac{\sqrt{2}/2}{a}=\dfrac{\sqrt{3}/2}{b} \; \Longrightarrow \; b = \sqrt{\dfrac{3}{2}} a$, así $a^2 + 4a - 8 = 0$. Por la fórmula cuadrática y el hecho que $a>0$, tenemos $$a=\dfrac{-4+\sqrt{4^2-4(1)(-8)}}{2}=\dfrac{4\sqrt{3}-4}{2}$$\\

%--------------------64.
\item La aproximación $\sen x \approx x$ Siempre es útil saber que cuando x se mide en radianes, $\sen x \approx x$ para valores numéricamente pequeños de $x$. En la sección $3.11$ veremos por qué es válida esta aproximación. El error de aproximación es menor que $1$ en $5000$ si $|x|<0.1$.
    \begin{enumerate}[\bfseries a)]
	
	%----------a)
	\item Con su calculadora graficadora en modo de radianes, trace las gráficas de $y = \sen x$ e $y = x$, juntas en una ventana, alrededor del origen. ¿Qué observa conforme $x$ se aproxima al origen?\\\\
	    Respuesta.-\; Vemos que las dos funciones coinciden.\\\\ 

	%----------b)
	\item Con su calculadora graficadora en modo de grados, trace las gráficas de $y = \sen x$ e $y = x$, juntas en una ventana, alrededor del origen. ¿Qué tan diferente es la figura obtenida en modo de radianes?\\\\
	    Respuesta.-\; Las curvas parecen líneas rectas que se cruzan cerca del origen cuando la calculadora está en modo de grados.\\\\

    \end{enumerate}

Curvas senoidales generales.\\\\
Para $$f(x)=A\sen\left(\dfrac{2\pi}{B}(x-C)\right)+D$$
Identifique $A,B,C$ y $D$ para las funciones seno de los ejercicios $65$ a $68$ y trace sus gráficas.\\\\

%---------------------65.
\item $y=2\sen(x + \pi) - 1$\\\\
    Respuesta.-\; $A=2, \; B=1 \; C = -\pi, \; D = -1$\\\\

%--------------------66.
\item $y=\dfrac{1}{2}\sen(\pi x - \pi) + \dfrac{1}{2}$\\\\
    Respuesta.-\; $A=1/2,\; B=2,\; C=1,\; D=1/2$\\\\

%--------------------67.
\item $y=-\dfrac{2}{\pi} \sen\left(\dfrac{\pi}{2}t\right)+\dfrac{1}{\pi}$\\\\
    Respuesta.-\; $A=-2/\pi,\; B=4,\; C=0,\; D=1/\pi$\\\\

%--------------------68.
\item $y=\dfrac{L}{2\pi}\sen \dfrac{2\pi t}{L}, \; L>0$\\\\
    Respuesta.-\; $A=L/2\pi,\; B=L,\; C=0,\; D=0$\\\\

Exploración con Computadora\\\\

En los ejercicios $69$ a $72$, investigará qué ocurre gráficamente con la función general seno 
$$f(x)=A\sen \left(\dfrac{2\pi}{B}(x - C)\right) + D$$
a medida que se modifican los valores de las constantes $A, B, C$ y $D$. Use un software matemático para ejecutar los pasos de los siguientes ejercicios\\\\

%--------------------69.
\item El periodo $B$. Considere las constantes $A=3,$ $C=D=0$.
    \begin{enumerate}[\bfseries a)]
	
	%----------a)
	\item Grafique $f(x)$ para los valores $B=1,3,2\pi,5\pi$ en el intervalo $-4\pi \leq x \leq 4\pi$. Describa qué le sucede a la gráfica de la función general seno conforme aumenta el periodo.\\\\
	    Respuesta.-\; La gráfica se alarga conforme aumenta el periodo.\\\\

	%----------b)
	\item ¿Qué le ocurre a la gráfica para valores negativos de $B$? Inténtelo con $B=-3$ y $B=-2\pi$\\\\
	    Respuesta.-\; De igual forma al inciso $a)$ la gráfica se expande.\\\\ 

    \end{enumerate}

%--------------------70
\item El desplazamiento horizontal $C$ Considere las constantes $A = 3, B = 6$ y $D = 0$.
    \begin{enumerate}[\bfseries a)]

	%----------a)
	\item Grafique $f(x)$ para los valores $C = 0$, $1$ y $2$ sobre el intervalo $-4\pi\leq x\leq 4\pi$. Describa qué le sucede a la gráfica de la función general seno, conforme $C$ aumenta otorgándole valores positivos.\\\\
	    Respuesta.-\; La gráfica se desplaza $C$ unidades hacia la derecha.\\\\

	%----------b)
	\item ¿Qué le sucede a la gráfica para valores negativos de $C$?\\\\
	    Respuesta.-\; El gráfico se desplaza hacia la izquierda $C$ unidades.\\\\

	%----------c)
	\item ¿Cuál es el menor valor positivo que debemos asignar a $C$, de manera que la gráfica no se desplace horizontalmente? Confirme su respuesta trazando una gráfica.\\\\
	    Respuesta.-\; $|C|=6$\\\\

	\end{enumerate}

%--------------------71.
\item El desplazamiento vertical D Considere las constantes $A = 3, B = 6, C = 0$.

    \begin{enumerate}[\bfseries a)]

	%----------a)
	\item Grafique $f(x)$ para los valores $D = 0, 1$ y $3$ sobre el intervalo $-4\pi \leq x \leq 4\pi$. Describa qué le sucede a la gráfica de la función general seno, conforme $D$ aumenta otorgándole valores positivos.\\\\
	    Respuesta.-\; La gráfica se desplaza horizontalmente hacia arriba.\\\\

	%----------b)
	\item ¿Qué le ocurre a la gráfica para valores negativos de $D$?\\\\
	    Respuesta.-\; Lo contrario al inciso $a)$.\\\\

    \end{enumerate}

%--------------------72.
\item La amplitud $A$ Considere las constantes $B = 6, C = D = 0$.

    \begin{enumerate}[\bfseries a)]

	%----------a)
	\item Describa qué le sucede a la gráfica de la función general seno, conforme $A$ aumenta otorgándole valores positivos. Confirme su respuesta graficando $f(x)$ para los valores $A = 1, 5$ y $9$.\\\\
	    Respuesta.-\; A medida que $A$ crece la amplitud también lo hace.\\\\

	%----------b)
	\item ¿Qué le ocurre a la gráfica para valores negativos de $A$?\\\\
	    Respuesta.-\; A medida que $A$ disminuye entonces la amplitud crece negativamente.\\\\

    \end{enumerate}

\end{enumerate}


\section{Ejercicios}

Se aplicará el software python, que se encuentra en el apartado Python.\\\\

\section{Ejercicios de práctica}

\begin{enumerate}[\bfseries 1.]

%--------------------1.
\item Exprese el área y el perímetro de un círculo como funciones de su radio. Luego, exprese el área del círculo como una función de su perímetro.\\\\
    Respuesta.-\; $P=2\pi r$ y $A = \pi r^2$ luego $A = \pi \left(\dfrac{P}{2\pi}\right)^2 = \dfrac{P^2}{4\pi}$\\\\

%--------------------2.
\item Exprese el radio de una esfera como una función de su área superficial. Luego, exprese el área superficial de la esfera como una función de su volumen.\\\\
    Respuesta.-\; $A_s = 4\pi r^2$ de donde $r=\sqrt{\dfrac{A_s}{4\pi}}$. Luego sea $V = \dfrac{4}{3} \pi r^3$ de donde $A_s = 4 \pi \left(\sqrt[3]{\dfrac{3V}{4 \pi}}\right)^2$\\\\

%--------------------3.
\item . Un punto $P$ en el primer cuadrante está sobre la parábola $y = x^2$. Exprese las coordenadas de $P$ como funciones del ángulo de inclinación de la recta que une $P$ con el origen.\\\\
    Respuesta.-\; Dado que el punto $P$ se encuentra en la parábola $y=x^2$ entonces $P=(a,a^2)$. Imaginemos el triangulo $ABC$ con $\angle ABC = \theta$, por lo tanto $$\tan \theta = \dfrac{|AC|}{|BC|} = \dfrac{a^2}{a} = a$$ de donde $$b=(\tan \theta)^2 = \tan^2 \theta$$\\

%--------------------4.
\item Un globo de aire caliente se eleva en línea recta desde el nivel del suelo, y es rastreado desde una estación que está localizada a $500\;ft$ del lugar de lanzamiento. Exprese la altura del globo como función del ángulo que forma la recta que va desde la estación hasta el globo en relación con el suelo.\\\\
    Respuesta.-\; Se tiene que la altura es igual a $500\tan \theta$.\\\\ 

En los ejercicios $5$ a $8$, determine si la gráfica de la función es simétrica con respecto al eje $y$, al origen o a ninguno de los dos.\\\\

%--------------------5.
\item $y=x^{1/5}$\\\\
    Respuesta.-\; La función es simétrica con respecto al origen.\\\\

%--------------------6.
\item $y=x^{2/5}$\\\\
    Respuesta.-\; L función es simétrica con respecto al eje $y$.\\\\

%--------------------7.
\item $y=x^2-2x-1$\\\\
    Respuesta.-\; La función es simétrica con respecto al eje $y$.\\\\

%--------------------8.
\item $y=e^{-x^2}$\\\\
    Respuesta.-\; La función es simétrica con respecto al eje $y$.\\\\

En los ejercicios $9$ a $16$, determine si la función es par, impar o ninguna de las dos.\\\\

%--------------------9.
\item $y=x^2 + 1$\\\\
    Respuesta.-\; La función es simétrica con respecto al eje $y$.\\\\

%--------------------10.
\item $y=x^5 - x^3 - x$\\\\
    Respuesta.-\; La función es simétrica con respecto al origen.\\\\

%--------------------11.
\item $y=1- \cos x$\\\\
    Respuesta.-\; La función es simétrica con respecto al eje $y$.\\\\

%--------------------12.
\item $y=sec \; x \tan x$\\\\
    Respuesta.-\; La función no es simétrica a ningún eje.\\\\

%--------------------13.
\item $y\dfrac{x^4 + 1}{x^3 - 2x}$\\\\
    Respuesta.-\; Es simétrica respecto al origen.\\\\ 

%--------------------14.
\item $y =  x - \sen x$\\\\
    Respuesta.-\; Es simétrica respecto al origen.\\\\

%--------------------15.
\item $y = x + \cos x$\\\\
    Respuesta.-\; Es simétrica respecto al origen.\\\\

%--------------------16.
\item $y=x \cos x$\\\\
    Respuesta.-\; No es simétrica a ningún eje.\\\\

%--------------------17.
\item Suponga que $f$ y $g$ son funciones impares definidas para toda la recta de los números reales. ¿Cuáles de las siguientes (donde estén definidas) son impares? ¿Y pares?
\begin{enumerate}[\bfseries a)]

    %----------a)
    \item $fg$\\\\
	Respuesta.-\; Se tiene una función par.\\\\

    %----------b)
    \item $f^3$\\\\
	Respuesta.-\; La función es impar.\\\\

    %----------c)
    \item $f(\sen x)$\\\\
	Respuesta.-\; La función es impar.\\\\

    %----------d)
    \item $g(sec \; x)$\\\\
	Respuesta.-\; La función es par.\\\\

    %----------e)
    \item $|g|$\\\\
	Respuesta.-\; La función es par.\\\\

\end{enumerate}

%--------------------18.
\item Si $f(a-x) = f(a+x)$, demuestre que $g(x)=f(x+a)$ es una función par.\\\\
    Demostración.-\; Por definición de función par se tiene que 
    $$g(-x) = f(-x+a) = f(x+a) = g(x)$$\\\\

En los ejercicios 19 a 20, determine $a)$ el dominio, $b)$ el rango.\\\\

%--------------------19.
\item $y=|x|-2$\\\\
    Respuesta.-\; El dominio viene dado para todo los números reales y el rango es para $y\geq -2$.\\\\

%--------------------20.
\item $y=-2 + \sqrt{1-x}$\\\\
    Respuesta.-\; El dominio viene dado para $x\geq 1$, y el rango para $y\geq -2$.\\\\

%--------------------21.
\item $y=\sqrt{16-x^2}$\\\\
    Respuesta.-\; El dominio es $-4\leq x \leq 4$ y el rango es $0\leq x \leq 4$\\\\

%--------------------22.
\item $y=3^{2-x}+1$\\\\
    Respuesta.-\; El dominio viene dado para todo número real, y el rango es para $y\leq 1$.\\\\

%--------------------23.
\item $y=2e^{-x} - 3$\\\\
    Respuesta.-\; El dominio es para todo número real, y el rango es $y\geq -3$.\\\\

%--------------------24.
\item $y=\tan(2x-\pi)$\\\\
    Respuesta.-\; El dominio esta dado para $x\leq \dfrac{3\pi}{4} + \dfrac{k\pi}{2}, k\in \mathbb(Z)$.\\\\

%--------------------25.
\item $y=2\sen(3x+\pi)-1$\\\\
    Respuesta.-\; El dominio viene dado para todo número real, y el rango viene dado para $-3\leq y \leq 1$

%--------------------26.
\item $y=x^{2/5}$\\\\
    Respuesta.-\; El dominio viene dado para todo número real, y el rango es $y\leq 0$.\\\\

%--------------------27.
\item $y=\ln(x-3)+1$\\\\
    Respuesta.-\; El dominio viene dado para $x>3$, y el rango está dado para todo número real.\\\\

%--------------------28.
\item $y=-1+\sqrt{2-x}$\\\\
    Respuesta.-\; El dominio y el rango viene dado para todo número real.\\\\

%--------------------29.
\item 

\end{enumerate}



    %---------- límites y continuidad
	%\chapter{Límites y continuidad}

%--------------------definición 2.1

\begin{tcolorbox}[colframe = white]
    \begin{def.}[La razón promedio de cambio] de $y=f(x)$ con respecto a $x$ en el intervalo $[x_1,x_2]$ sabiendo que $\triangle x = x_2 - x_1 = h$ es $$\dfrac{\triangle y}{\triangle x} = \dfrac{f(x_2) - f(x_1)}{x_2-x_1} = \dfrac{f(x_1+h) - f(x_1)}{h}, \quad h\neq 0$$
    \end{def.}
\end{tcolorbox}

%-------------------- Ejercicios 2.1
\section{Ejercicios}

\textbf{Razones promedio de cambio}\\\\\
En los ejercicios 1 a 6, determine la razón promedio de cambio de la función en el intervalo o intervalos dados.\\\\

\begin{enumerate}[\large\bfseries 1.]

%--------------------1.
\item $f(x) = x^3 + 1$ 
\begin{enumerate}[\bfseries a)]
    
    %----------a)
    \item $[2,3]$\\\\
	Respuesta.-\; $\dfrac{\triangle y}{\triangle x} = \dfrac{(3^3 + 1) - (2^3 + 1)}{3 - 2} = 19$\\\\

    %----------b)
    \item $[-1,1]$\\\\
	Respuesta.-\; $\dfrac{\triangle y}{\triangle x} = \dfrac{(1^3 + 1)-((-1)^3 + 1)}{1-(-1)} = 1$\\\\

\end{enumerate}

%--------------------2.
\item $g(x) = x^2 - 2x$
\begin{enumerate}[\bfseries a)]
    
    %----------a)
    \item $[1,3]$\\\\
	Respuesta.-\; $\dfrac{\triangle y}{\triangle x} = \dfrac{(3^2 - 2\cdot 3) - (1^2 - 2\cdot 1)}{3-1} = 2 $\\\\ 
    
    %----------b)
    \item $[-2,4]$\\\\
	Respuesta.-\; $\dfrac{\triangle y}{\triangle x} = \dfrac{(4^2 - 2\cdot 4) - ((-2)^2 - 2\cdot (-2))}{4-(-2)} = 0$\\\\

\end{enumerate}

%--------------------3.
\item $h(t) = cot \; t$
\begin{enumerate}[\bfseries (a)]

    %----------(a)
    \item $[\pi/4,3\pi/4]$\\\\
	Respuesta.-\; $\dfrac{\triangle y}{\triangle x} = \dfrac{cot(\pi/4)-cot(3\pi/4)}{\pi/4-3\pi/4} = \dfrac{1+1}{\dfrac{\pi - 3\pi}{4}} = \dfrac{8}{-2\pi}$\\\\

    %----------(b)
    \item $[\pi/6,\pi/2]$\\\\
	Respuesta.-\; $\dfrac{\triangle y}{\triangle x} = \dfrac{cot(\pi/6)-cot(\pi/2)}{\pi/6-\pi/2} = \dfrac{-3\sqrt{3}}{\pi}$\\\\

\end{enumerate}

%--------------------4.
\item $g(t) = 2 + \cos t$ 
\begin{enumerate}[\bfseries (a)]

    %----------(a)
    \item $[0,\pi]$\\\\
	Respuesta.-\; $\dfrac{2+\cos \pi - (2+\cos 0)}{\pi - 0} = -\dfrac{2}{\pi}$\\\\

    %----------(b)
    \item $[-\pi,\pi]$\\\\
	Respuesta.-\; $\dfrac{2 + \cos \pi - (2 - \cos \pi)}{\pi+\pi} = \dfrac{3-3}{2\pi} = 0$\\\\

\end{enumerate}

%--------------------5.
\item $R(\theta) = \sqrt{4\theta + 1}; \; [0,2]$\\\\ 
    Respuesta.-\; $\dfrac{\sqrt{4*2+1}+1 - (\sqrt{4*0+1}+1)}{2-0} = \dfrac{2}{2} = 1$\\\\

%--------------------6.
\item $P(\theta) = \theta^3 - 4\theta^2 + 5\theta; \; [1,2]$\\\\
    Respuesta.-\; $\dfrac{2^3 - 4\cdot 2^2 + 5\cdot 2 - (1^3 - 4^2 + 5)}{2-1} = 0$\\\\

\textbf{Pendiente de una curva en un punto}\\\\
En los ejercicios 7 a 14, utilice el método del ejemplo 3 para determinar $a)$ la pendiente de la curva en el punto $P$ dado, y $b)$ la ecuación de la recta tangente en $P$\\\\

%--------------------7.
\item $y = x^2 - 5, \quad P(2,-1)$

\begin{enumerate}[\bfseries a)]
    
    %----------a)
    \item Iniciamos con una recta secante que pasa por el punto $(2,-1)$ y el punto cercano $(2+h,(2+h)^2 - 5)$, luego hallamos la pendiente de la secante, $$\dfrac{\triangle y}{\triangle x} = \dfrac{(2+h)^2 - 5 - (2^2 - 5)}{2+h - 2} = \dfrac{4h+h^2}{h} = 4+h$$ 
    Luego aproximamos  $h$ a $0$ siendo la pendiente $m=4$.\\\\

    %----------b)
    \item La ecuación de la recta tangente viene dado por $y=mx + c$  de donde $y=4x+c$, luego reemplazamos $(2,-1)$, quedándonos $-1=4\cdot 2 + c \; \Longrightarrow\; c = -9$. Por lo tanto $$y=4x-9$$\\

\end{enumerate}

%--------------------8.
\item $y=7-x^2, \quad P(2,3)$ 

\begin{enumerate}[\bfseries a)]
    
    %----------a)
    \item Sea la recta secante que pasa por el punto $P(2,3)$ y el punto cercano $Q\left[2+h,7-(2+h)^2\right]$, hallamos la pendiente de la secante, $$\dfrac{\triangle y}{\triangle x} = \dfrac{7-(2+h)^2 - (7-2^2)}{2+h-2} = \dfrac{7-(2+h)^2 - 3}{h} = \dfrac{h(-h-4)}{h} = -h-4$$
    Con forme $Q$ se aproxima a $P$ a lo largo de la curva, $h$ se aproxima a cero, y la pendiente de la secante $-h-4$ se aproxima a $-4$. Tomamos $-4$ como la pendiente de la parábola en $P$.\\\\

    %----------b)
    \item La ecuación de la recta tangente viene dado por $y = mx + c$ de donde $y=-4x+c$, luego reemplazamos $(2,3)$, así $3 = -4(2) + c \Longrightarrow c = 11$. Por lo tanto $$y = -4x + 11$$.\\

\end{enumerate}

%--------------------9.
\item $y=x^2-2x-3, \quad P(2,-3)$

\begin{enumerate}[\bfseries a)]
    
    %----------a)
    \item Sea la recta secante que pasa por le punto $P(2,-3)$  y el punto cercano $Q\left[2+h, (2+h)^2 - 2(2+h) - 3\right]$, hallamos la pendiente de la secante, $$\dfrac{\triangle y}{\triangle x} = \dfrac{(2+h)^2 - 2(2+h) - 3 - (-3)}{h} = \dfrac{h^2 + 2h}{h} = h+2$$
    Con forme $Q$ se aproxima a $P$ a lo largo de la curva, $h$ se aproxima a cero, y la pendiente de la secante $h+2$ se aproxima a $2$. Tomamos $2$ como la pendiente de la parábola en $P$.\\\\

    %----------b)
    \item La ecuación de la recta tangente viene dado por $y = mx + c$ de donde $y=2x+c$, luego reemplazamos $(2,-3)$, así $-3 = 2(2) + c \Longrightarrow c = -7$. Por lo tanto $$y = -2x - 7$$.\\

\end{enumerate}

%--------------------10.
\item $y=x^2-4x, \quad P(1,-3)$ 

\begin{enumerate}[\bfseries a)]
    
    %----------a)
    \item Sea la recta secante que pasa por le punto $P(1,-3)$  y el punto cercano $Q\left[1+h, (1+h)^2 - 4(1+h) \right]$, hallamos la pendiente de la secante, $$\dfrac{\triangle y}{\triangle x} = \dfrac{(1+h)^2 - 4(1+h) - (-3)}{1+h - 1} = \dfrac{h^2 - 2h}{h} = h-2$$
    Con forme $Q$ se aproxima a $P$ a lo largo de la curva, $h$ se aproxima a cero, y la pendiente de la secante $h-2$ se aproxima a $-2$. Tomamos $-2$ como la pendiente de la parábola en $P$.\\\\

    %----------b)
    \item Tenemos $$\dfrac{\triangle y}{\triangle x} = m = \dfrac{y - y_1}{x - x_1} \; \Longrightarrow \; y-y_1 = m(x-x_1) \; \Longrightarrow \; y = -2(x-1) + (-3) \; \Longrightarrow \; y = -2x+2 - 3$$ por lo tanto $$y = -2x-1$$\\
     

\end{enumerate}

%--------------------11.
\item $y=x^3, \quad P(2,8)$

\begin{enumerate}[\bfseries a)]
    
    %----------a)
    \item Sea la recta secante que pasa por le punto $P(2,8)$  y el punto cercano $Q\left[2+h, (2+h)^3 \right]$, hallamos la pendiente de la secante, $$\dfrac{\triangle y}{\triangle x} = \dfrac{(2+h)^3 - 8}{2+h - 2} = \dfrac{h^3 + 6h^2 + 12h}{h} = h^2+6h+12$$
    Con forme $Q$ se aproxima a $P$ a lo largo de la curva, $h$ se aproxima a cero, y la pendiente de la secante $h^2 + 6h + 12$ se aproxima a $12$. Tomamos $12$ como la pendiente de la parábola en $P$.\\\\

    %----------b)
    \item Tenemos $$\dfrac{\triangle y}{\triangle x} = m = \dfrac{y - y_1}{x - x_1} \; \Longrightarrow \; y = m(x-x_1) + y_1\; \Longrightarrow \; y = 12(x-2) + 8 \; \Longrightarrow \; y = 12x - 24 + 8$$ por lo tanto $$y = 12x-16$$\\

\end{enumerate}

%--------------------12.
\item $y=2-x^3, \quad P(1,1)$

\begin{enumerate}[\bfseries a)]
    
    %----------a)
    \item Sea la recta secante que pasa por le punto $P(1,1)$  y el punto cercano $Q\left[1+h, 2-(1+h)^3 \right]$, hallamos la pendiente de la secante, $$\dfrac{\triangle y}{\triangle x} = \dfrac{2-(1+h)^3 - 1}{1+h - 1} = \dfrac{h^3 + 3h^2 + 3h}{h} = h^2+3h+3$$
    Con forme $Q$ se aproxima a $P$ a lo largo de la curva, $h$ se aproxima a cero, y la pendiente de la secante $h^2 + 3h + 3$ se aproxima a $3$. Tomamos $3$ como la pendiente de la parábola en $P$.\\\\

    %----------b)
    \item Tenemos $$\dfrac{\triangle y}{\triangle x} = m = \dfrac{y - y_1}{x - x_1} \; \Longrightarrow \; y = m(x-x_1) + y_1\; \Longrightarrow \; y = 3(x-1) + 1 \; \Longrightarrow \; y = 3x - 3 + 1$$ por lo tanto $$y = 3x-2$$\\

\end{enumerate}

%--------------------13.
\item $y=x^3 - 12x, \quad P(1,-11)$

\begin{enumerate}[\bfseries a)]
    
    %----------a)
    \item Sea la recta secante que pasa por le punto $P(1,-11)$  y el punto cercano $Q\left[1+h, (1+h)^3 -12(1+h)\right]$, hallamos la pendiente de la secante, $$\dfrac{\triangle y}{\triangle x} = \dfrac{(1+h)^3 - 12(1+h) + 11}{1+h - 1} = \dfrac{h^3 + 3h^2 - 9h}{h} = h^2+3h-12$$
    Con forme $Q$ se aproxima a $P$ a lo largo de la curva, $h$ se aproxima a cero, y la pendiente de la secante $h^2 + 3h - 9$ se aproxima a $-9$. Tomamos $-9$ como la pendiente de la parábola en $P$.\\\\

    %----------b)
    \item Tenemos $$\dfrac{\triangle y}{\triangle x} = m = \dfrac{y - y_1}{x - x_1} \; \Longrightarrow \; y = m(x-x_1) + y_1\; \Longrightarrow \; y = -9(x-1) - 11 \; \Longrightarrow \; y = -9x + 9 - 11$$ por lo tanto $$y = -9x-2$$\\

\end{enumerate}

%--------------------14.
\item $y=x^3-3x^2+4,\quad P(2,0)$

\begin{enumerate}[\bfseries a)]
    
    %----------a)
    \item Sea la recta secante que pasa por le punto $P(2,0)$  y el punto cercano $Q\left[2+h, (2+h)^3 -3(2+h)^2 + 4\right]$, hallamos la pendiente de la secante, $$\dfrac{\triangle y}{\triangle x} = \dfrac{(2+h)^3 - 3(2+h)^2 + 4 - 0}{2+h - 2} = \dfrac{h^3 + 3h^2}{h} = h^2+3h$$
    Con forme $Q$ se aproxima a $P$ a lo largo de la curva, $h$ se aproxima a cero, y la pendiente de la secante $h^2 + 3h$ se aproxima a $0$. Tomamos $0$ como la pendiente de la parábola en $P$.\\\\

    %----------b)
    \item Tenemos $$\dfrac{\triangle y}{\triangle x} = m = \dfrac{y - y_1}{x - x_1} \; \Longrightarrow \; y = m(x-x_1) + y_1\; \Longrightarrow \; y = 0(x-2) - 0$$ por lo tanto $$y = 0$$\\

\end{enumerate}

\textbf{Razones instantáneas de cambio}\\\\

%--------------------15.
\item Rapidez de un automóvil. La siguiente figura muestra la gráfica tiempo-distancia de un automóvil deportivo que acelera desde el reposo.
\begin{enumerate}[\bfseries a)]

    %----------a)
    \item Determine las pendientes de la secante $PQ_1, PQ_2, PQ_3$ y $PQ_4$, ordenándolas en una tabla como la de la figura 2.6.\\\\
	Respuesta.-\; 
	\begin{center}
	    \begin{tabular}{c|c}
		Q & PQ \\\\	
		\hline\\
		$Q_1(10,220)$ & $\dfrac{650-220}{20-10} = 43$\\\\
		$Q_2(14,380)$ & $\dfrac{650-380}{20-14} = 45$\\\\
		$Q_3(17,480)$ & $\dfrac{650-480}{20-16} = 43$\\\\
		$Q_4(17,550)$ & $\dfrac{650-550}{20-18} = 50$\\\\
	    \end{tabular}
	\end{center}
	Los resultados anteriores son redondeados.\\\\

    %----------b)
    \item Después estime la rapidez del automóvil para el tiempo $t=20s$.\\\\
	Respuesta.-\; Tomamos el punto mas cercano a $P$, en este caso $Q_4$ de donde la velocidad vendrá dado aproximadamente por $50m/s$.\\\\

\end{enumerate}

%--------------------16.
\item La siguiente figura muestra la gráfica de la distancia de caída libre contra el tiempo para un objeto que cae desde un módulo espacial que se encuentra a una distancia de 80 m de la superficie de la Luna.
\begin{enumerate}[\bfseries a)]
    
    %----------a)
    \item Estime las pendientes de las secantes $PQ_1$, $PQ_2$, $PQ_3$ y $PQ_4$, ordenándolas en una tabla como la de la figura 2.6.\\\\
	Respuesta.-\; 
	\begin{center}
	    \begin{tabular}{c|c}
		Q & PQ \\\\	
		\hline\\
		$Q_1(5,20)$ & $\dfrac{80-20}{10-5} = 12$\\\\
		$Q_2(7,38)$ & $\dfrac{80-38}{10-7} = 14$\\\\
		$Q_3(8.5,56)$ & $\dfrac{80-56}{10-8.5} = 16$\\\\
		$Q_4(9.5,71)$ & $\dfrac{80-71}{10-9.5} = 18$\\\\
	    \end{tabular}
	\end{center}

    %----------b)
    \item ¿Cuál será la rapidez aproximada del objeto cuando choca con la superficie de la Luna?\\\\
	Respuesta.-\; Será de  $18m/s$.\\\\ 

\end{enumerate}

%--------------------17.
\item En la siguiente tabla se registran las utilidades de una pequeña empresa en cada uno de sus primeros cinco años de operación:
\begin{enumerate}[\bfseries a)]

    %----------a)
    \item Trace los puntos que representan las utilidades como una función del año, y únalos mediante una curva suave.\\\\

    %----------b)
    \item ¿Cuál es la razón promedio de incremento de las utilidades entre $2012$ y $2014$?\\\\
	Respuesta.-\; $\dfrac{\triangle y}{\triangle x} = \dfrac{175-63}{2014-20112} = 56$\\\\ 

    %----------c)
    \item Use su gráfica para estimar la razón a la que cambiaron las utilidades en $2012$.\\\\
	Respuesta.-\; Sea $\dfrac{63-26}{2012-2011} = 37$ y $\dfrac{112-63}{2013-2012} = 49$ entonces $\dfrac{37+49}{2} = 43$.\\\\

\end{enumerate}

%--------------------18.
\item 

\end{enumerate}



\end{document}


